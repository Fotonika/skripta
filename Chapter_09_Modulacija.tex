\chapterimage{AOModulator.jpg} % Chapter heading image

\chapter{Modulacija svetlobe}

V optičnih napravah pogosto želimo spreminjati lastnosti svetlobnega
valovanja. Tak primer smo že spoznali pri obravnavi laserja, kjer za 
preklop dobrote potrebujemo element, ki hitro spreminja prepustnost. 
Še pomembnejša je modulacija valovanja pri optičnem prenosu informacij.

Svetlobno valovanje lahko moduliramo na več načinov. Z ustreznim moduliranjem
lomnega količnika lahko valovanju spreminjamo amplitudo\index{Elektro-optična modulacija!amplitudna} 
ali frekvenco oziroma fazo\index{Elektro-optična modulacija!frekvenčna}
\index{Elektro-optična modulacija!fazna}. 
\begin{figure}[h]
\centering
\def\svgwidth{140truemm} 
\input{slike/09_AMFM.pdf_tex}
\caption{Amplitudno (levo) in fazno oziroma frekvenčno moduliran signal (desno)
}
\label{fig:amfm}
\end{figure}

Delovanje optičnih modulatorjev temelji na različnih pojavih. V tem poglavju bomo 
podrobneje spoznali dva načina, to sta elektro-optični in elasto- oziroma akusto-optični pojav. 
Pri prvem pride do spremembe lomnega količnika snovi pod vplivom zunanjega električnega polja, 
pri drugem pa zaradi mehanske deformacije. Kadar mehansko deformacijo povzroči zvočno valovanje, 
takim modulatorjem pravimo akusto-optični. Na koncu bomo spoznali poseben zelo pomemben 
primer elektro-optičnih modulatorjev na osnovi tekočih kristalov.

\section{Elektro-optični pojav}
Elektro-optični pojav\index{Elektro-optični pojav} opisuje spremembe optičnih lastnosti 
snovi (dielektričnosti in lomnega količnika) pod vplivom zunanjega električnega polja. 
Omejimo se na statično zunanje polje oziroma
na polje, katerega frekvenca je bistveno manjša od optične frekvence. Omejitev na nizko 
frekvenco je potrebna zato, da optično polje še lahko obravnavamo linearno. 
Kako je v nasprotnem primeru, ko je frekvenca polja primerljiva z optično frekvenco, 
smo na široko obravnavali v poglavju o nelinearni optiki (poglavje~\ref{chap:NLO}).

Namesto dielektričnega tenzorja\index{Dielektričnost!inverzna} 
navadno vpeljemo inverzni dielektrični tenzor
\beq
\underline{b}=\underline{\epsilon}^{-1}.
\eeq
Izračunajmo zvezo med spremembo inverznega tenzorja $\delta b_{ij}$ 
in spremembo dielektričnega tenzorja $\delta \varepsilon_{ij}$. 
Če so spremembe majhne, velja 
\begin{equation}
\underline{\varepsilon} = \underline{\tilde{\varepsilon}} + \delta \underline{\varepsilon}=
(\underline{b}+\delta \underline{b})^{-1}=\left(\underline{b}(1+\underline{b}^{-1}
\delta \underline{b})\right)^{-1}=(1+\underline{b}^{-1}\delta \underline{b})^{-1}\underline{b}^{-1}
\approx \underline{b}^{-1}-\underline{b}^{-1}\delta \underline{b}\, \underline{b}^{-1}.
\label{7.2}
\end{equation}
Sprememba dielektričnega tenzorja je tako
\beq
 \delta \underline{\varepsilon}= -\underline{b}^{-1}\delta \underline{b}\, \underline{b}^{-1}
 = -\underline{\tilde{\varepsilon}}\, \delta \underline{b}\, \underline{\tilde{\varepsilon}}.
\eeq
Če je nemoten dielektrični tenzor $\underline{\tilde{\varepsilon}}$ diagonalen, velja
\begin{equation}
\delta\epsilon_{ij}=-\tilde{\epsilon}_{ik}\delta b_{kl}\tilde{\epsilon}_{lj}
=-\tilde{\epsilon}_{ii}\tilde{\epsilon}_{jj}\delta b_{ij}.
\label{7.3}
\end{equation}

Pri elektro-optičnem pojavu so spremembe tenzorja dielektričnosti zaradi vpliva zunanjega polja razmeroma 
majhne. Spremembo komponente $\delta b_{ij}$ lahko zato zapišemo kot potenčno vrsto zunanjega polja $E$, 
pri čemer upoštevajmo zgolj prva dva člena v razvoju
\boxeq{7.1}{
\delta b_{ij}=r_{ijk}E_{k}+q_{ijkl}E_{k}E_{l}.
}
Prvi člen, linearno sorazmeren zunanjem polju, opisuje linearni elektro-optični
ali Pockelsov\index{Pockelsov pojav} pojav\footnote{Nemški fizik Friedrich Carl Alwin Pockels, 1865--1913.}. 
Tenzor tretjega ranga $r_{ijk}$, ki je lastnost snovi, imenujemo elektro-optični 
tenzor\index{Elektro-optični tenzor}
ali tudi Pockelsov tenzor\index{Pockelsov tenzor|see {Elektro-optični tenzor}}. 
Pockelsov tenzor je različen od nič v snoveh brez centra inverzije, značilne vrednosti Pockelsovega
tenzorja pa so okoli $r \sim 10^{-12} - 10^{-10}$~m/V,

Kvadratnemu elektro-optičnemu pojavu pravimo Kerrov\index{Kerrov pojav}
pojav\footnote{Škotski fizik John Kerr, 1824--1907.}, tenzorju $q_{ijkl}$ pa Kerrov tenzor\index{Kerrov tenzor}. 
Kerrov pojav je praviloma precej šibkejši od Pockelsovega, vendar je različen od nič v vseh snoveh, ne glede na
njihove simetrijske lastnosti, torej tudi v tekočinah. 
Značilna vrednost Kerrovega tenzorja je $q \sim 10^{-24}$~m$^2$/V$^2$. Navadno ločimo dva primera Kerrovega
pojava: Kerrov elektro-optični pojav pri zunanjih poljih z nizko frekvenco, in optični Kerrov pojav, ki smo 
ga podrobneje spoznali pri obravnavi nelinearnih optičnih pojavov (poglavje~\ref{OKP}).

Za uporabo trdnih kristalov je pomemben
predvsem linearni člen, zato se bomo osredotočili le nanj in zapisali
\boxeq{eq:Pockels}{
\delta b_{ij}=r_{ijk}E_{k}. 
}

\subsection*{Elektro-optični ali Pockelsov tenzor}
Simetrija snovi pomembno vpliva na obliko tenzorjev, ki opisujejo njene lastnosti.
Pockelsov tenzor $r$ je tenzor tretjega ranga, zato je lahko različen
od nič le v kristalih brez centra inverzije. 
Simetrija kristala tudi v primeru, ko ni centra inverzije, močno
zmanjša število neodvisnih komponent $r_{ijk}$. 

Ker je inverzni dielektrični tenzor $b$ simetričen, je v prvih dveh indeksih simetričen
tudi Pockelsov tenzor
\beq
r_{ijk} = r_{jik}.
\eeq
V najmanj simetričnem primeru triklinskega kristala ima tako namesto 27 zgolj 
18 neodvisnih komponent, v kristalih z višjo simetrijo pa še manj. 

Podobno kot pri nelinearni susceptibilnosti (poglavje~\ref{Chap:Chi}) 
tudi elektro-optični tenzor pogosto zapišemo le z dvema komponentama. 
Prva dva indeksa, v katerih je $r_{ijk}$ simetričen, združimo
v enega z vrednostmi od 1 do 6 po dogovoru $xx=1$, $yy=2$, $zz=3$,
$yz=4$, $zx=5$ in $xy=6$. Tako postane $r_{ijk}$ matrika velikosti
$6\times3$, simetrični tenzor drugega ranga $b_{ij}$ pa šestdimenzionalen
vektor.

\begin{definition}
Naj bo $Q$ transformacijska matrika za dano simetrijsko operacijo. Potem za tenzorje
tretjega ranga velja
\beq
r_{ijk} = Q_{ip}Q_{jq}Q_{kr}r_{pqr}.
\eeq
Zapiši transformacijsko matriko $Q$ za vrtenje okoli osi $z$ za $\pi/2$ in pokaži, da so
v primeru štirištevne simetrije od nič različne le komponente $r_{xxz}=r_{yyz}, r_{zzz}, 
r_{yzx}=-r_{xzy}$ in $r_{xzx}=r_{yzy}$. Razmisli in izračunaj, kakšen bi bil tenzor $r$, če bi 
štirištevni simetriji dodali še zrcaljenje čez ravnino $xy$.
\end{definition}

Nekaj primerov Pockelsovih tenzorjev pri različnih kristalnih simetrijah
je podanih v tabeli~(\ref{table:Pockels}).

\begin{table}[h!]
 \centering
\begin{tabular}{|c|c|c|c|} \hline  
      Kristal & Grupa & Neničelne komponente tenzorja $r$ & Vrednost ($10^{-12}$~m/V)\\ \hline
      BaTiO$_3$\index{BaTiO$_3$} & 4mm & $r_{xzx} = r_{yzy} = r_{zxx} = r_{zyy} = 
      r_{51} = r_{42}$  &
	    (pri 1,55~$\mu$m) $r_{51} = 800$ \\
	      & & $r_{xxz} = r_{yyz} = r_{13} = r_{23}$ &  $r_{13} = 8$ \\
	      & & $r_{zzz} = r_{33}$ & $r_{33} = 28$ \\ \hline
      KDP\index{KDP} & 
      $\overline{4}$2m & $r_{yzx} = r_{zyx} = r_{xzy} = r_{zxy} = r_{41} = r_{52}$  &
	    $r_{41} = 8,77$ \\
	    & & $r_{xyz} = r_{yxz} = r_{63}$ &  $r_{63} = -10,3$ \\ \hline
      GaAs\index{GaAs}\index{ZnTe} &  $\overline{4}$3m&
	  $r_{yzx} = r_{zyx} = r_{xzy} = r_{zxy} = r_{xyz} = r_{yxz}$  & (pri 10,6~$\mu$m) $r_{41} = 1,5$ \\
	ZnTe  & &   $= r_{41} = r_{52}=r_{63}$  &(pri 3,4~$\mu$m) $r_{41} = 4,2$ 
	    \\ \hline
      LiNbO$_3$\index{LiNbO$_3$} & 3m & $r_{xzx} = r_{zxx} = r_{yzy} = r_{zyy} = r_{51} = r_{42}$  &
	    $r_{51} = 32,6$ \\
	     & & $r_{xxz} = r_{yyz} = r_{13} = r_{23}$ &  $r_{13} = 9,6$ \\
	      & & $r_{zzz} = r_{33}$ & $r_{33} = 30,9$ \\
	    & &  $r_{yyy} = - r_{xxy} = -r_{xyx} = -r_{yxx}  = $ & \\
	    & &  $=r_{22} =  -r_{12} =-r_{61} $  &
	    $r_{22}  = 6,8$ \\
\hline 
\end{tabular}
  \caption{Koeficienti Pockelsovega tenzorja za nekaj izbranih snovi. Če ni navedeno drugače, veljajo
  vrednosti pri valovni dolžini okoli 600~nm.}
\label{table:Pockels}
\end{table}

\begin{remark}
Komponente elektro-optičnega tenzorja zaradi nazornosti pogosto ponazarjamo grafično. V matriki $6\times 3$
s piko označimo komponente, ki so enake nič, s polnim krožcem neničelne komponente, povezava med 
komponentami pomeni njihovo enakost, prazen krožec in črtkana črta pa označujeta 
neničelno komponento nasprotnega predznaka. Kot primer sta podana prikaza tenzorjev za 
GaAs (levo) in LiNbO$_3$ (desno).
\begin{figure}[h!]
\centering
\def\svgwidth{20truemm} 
\input{slike/09_tensor.pdf_tex}\qquad \qquad
\def\svgwidth{20truemm} 
\input{slike/09_tensor2.pdf_tex}
\end{figure}
\end{remark}

\section{Longitudinalna modulacija}
Poglejmo podrobneje, kako električno polje spremeni optične lastnosti 
elektro-optičnega kristala in kako to vpliva na svetlobo, ki potuje skozi tak kristal.
Navadno se uporabljajo kristali, ki so dvolomni že brez zunanjega polja. 
Kot primer vzemimo kristal KH$_{2}$PO$_{4}$ (KDP)\index{KDP}, ki ima tetragonalno 
simetrijo ($\bar{4}2m$). Kot razberemo iz tabele~(\ref{table:Pockels}) ima 
elektro-optični tenzor dve neodvisni komponenti: $r_{41} = r_{52}=8,77 \times 10^{-12}$~m/V
in $r_{63}= -10,3 \times 10^{-12}$~m/V.

Kristal naj bo odrezan po kristalografskih oseh, svetloba naj skozi kristal potuje 
v smeri optične osi, to je smeri $z$, v isti smeri pa na kristal priključimo
polje $E_z$. Ker je smer električnega polja vzporedna s smerjo širjenja svetlobe, taki 
postavitvi pravimo longitudinalna in pojavu longitudinalna 
modulacija.\index{Elektro-optična modulacija!longitudinalna} 
\begin{figure}[h]
\centering
\def\svgwidth{80truemm} 
\input{slike/09_AMshema.pdf_tex}
\caption{Shema longitudinalne modulacije signala. Ker je polje priključeno v smeri
potovanja svetlobe, morata biti elektrodi transparentni. Z uporabo polarizatorja in 
analizatorja sestavimo amplitudni modulator (glej poglavje~\ref{chap:ampmod}).}
\label{fig:amshema}
\end{figure}

Inverzni tenzor dielektričnosti v odsotnosti zunanjega polja zapišemo kot
\beq
\underline{\tilde{b}} = 
\left[\begin{array}{ccc}
1/n_o^2 & 0& 0\\
0 & 1/n_o^2& 0\\
0 & 0&  1/n_e^2
\end{array}\right],
\label{7.8}
\eeq
pri čemer sta $n_o$ in $n_e$ redni in izredni lomni količnik. Ko priključimo 
polje, se tenzor dielektričnosti spremeni zaradi Pockelsovega pojava. Sprememba
inverznega tenzorja dielektričnosti je po enačbi~(\ref{eq:Pockels})
\begin{align}
\delta b_{xx} & =r_{xxx}E_x + r_{xxy}E_y + r_{xxz}E_z = 0,\nonumber \\
\delta b_{xy} & = \delta b_{yx} = r_{xyx}E_x + r_{xyy}E_y + r_{xyz}E_z = r_{63}E_z,\nonumber\\
\delta b_{xz} & = \delta b_{zx} =r_{xzz}E_z = 0,\nonumber\\
\delta b_{yy} & =r_{yyz}E_z = 0,\nonumber\\
\delta b_{yz} & = \delta b_{zy} =r_{yzz}E_z = 0,\nonumber\\
\delta b_{zz} & =r_{zzz}E_z = 0.
\end{align}
Vidimo, da je večina členov enaka nič, se pa zaradi zunanjega električnega
polja v smeri $z$ pojavi izvendiagonalna komponenta 
\beq
\underline{b} = 
\left[\begin{array}{ccc}
1/n_o^2 & 0& 0\\
0 & 1/n_o^2 & 0\\
0 & 0& 1/n_e^2
\end{array}\right] + \left[\begin{array}{ccc}
 0& r_{63}E_z& 0\\
r_{63}E_z & 0 & 0\\
0 & 0&  0
\end{array}\right] = \left[\begin{array}{ccc}
1/n_o^2 & r_{63}E_z& 0\\
r_{63}E_z& 1/n_o^2 & 0\\
0 & 0&  1/n_e^2
\end{array}\right].
\label{7.8a}
\eeq
Če želimo izračunati, kako se po kristalu pod napetostjo širi vpadni svetlobni
snop, moramo gornji dielektrični tenzor diagonalizirati. Lastne vrednosti novega tenzorja
in pripadajoče nove lastne osi so
\begin{align}
\lambda_1 &= \frac{1}{n_o^2}+ r_{63}E_z \quad \mathrm{in} \quad \mathbf{e}_1 = \frac{1}{\sqrt{2}}(1,1,0)\\
\lambda_2 &= \frac{1}{n_o^2}- r_{63}E_z \quad \mathrm{in} \quad \mathbf{e}_2 = \frac{1}{\sqrt{2}}(-1,1,0)\\
\lambda_3 &= \frac{1}{n_e^2} \quad \mathrm{in} \quad \mathbf{e}_3 = (0,0,1).
\end{align}
Vidimo, da so nove lastne osi zasukane za kot $45~^\circ$ glede na prvotne osi sistema.
V novem koordinatnem sistemu je inverzni dielektrični tenzor diagonalen in enak
\beq
\underline{b} = 
\left[\begin{array}{ccc}
1/n_o^2 + r_{63}E_z& 0& 0\\
0 & 1/n_o^2 - r_{63}E_z& 0\\
0 & 0& 1/n_e^2
\end{array}\right].
\eeq
\begin{figure}[h]
\centering
\def\svgwidth{60truemm} 
\input{slike/09_AMindikatrisa.pdf_tex}
\caption{Optično enoosni kristal postane pod napetostjo dvoosen. Indikatrisa, ki je pravokotno
na optično os brez polja krožnica, se pod vplivom napetosti spremeni v elipso. }
\label{fig:amn}
\end{figure}

Spomnimo se, da potuje svetloba skozi kristal vzdolž osi $z$. Brez zunanjega električnega
polja je kristal enoosen z optično osjo v smeri $z$. Lomni količnik je torej neodvisen od
polarizacije vpadnega valovanja in je enak $n_o$. Ko priključimo polje, postane kristal
optično dvoosen, saj so vse tri lastne vrednosti tenzorja dielektričnosti različne. Za žarek, 
ki potuje vzdolž osi $z$, torej obstajata dve lastni smeri $x'$ in $y'$ z ustreznima
novima lastnima količnikoma, ki ju izrazimo kot
\beq
\frac{1}{n_{x'}^2} = \frac{1}{n_o^2}+ r_{63}E_z \quad \mathrm{in} \quad 
\frac{1}{n_{y'}^2} = \frac{1}{n_o^2}- r_{63}E_z. 
\eeq
Kadar polarizacija vpadnega valovanja ne sovpada z novimi lastnimi osmi $x'$ ali $y'$, je 
svetloba po preletu kristala v splošnem eliptično polarizirana.

Za vsa eksperimentalno dosegljiva polja velja, da je $rE\ll1/n^2$, 
zato lahko gornja izraza razvijemo za majhne popravke
\beq
n_{x'} = \sqrt{\frac{n_o^2}{1+ n_o^2 r_{63}E_z}} \approx n_o \sqrt{1- n_o^2 r_{63}E_z}.
\eeq
Sledi
\boxeq{EOnx}{
n_{x'}\approx n_o - \frac{1}{2}n_o^3 r_{63}E_z.
}
Podobno izpeljemo še za drugo lastno vrednost
\boxeq{EOny}{
n_{y'}\approx n_o + \frac{1}{2}n_o^3 r_{63}E_z.
}
Različni lastni polarizaciji potujeta vzdolž osi $z$ z različnima hitrostma. Ko 
prepotujeta dolžino kristala $L$, pride med njima do fazne razlike
\beq
\Delta \phi = k_0 n_{y'} L - k_0 n_{x'} L = \frac{\omega}{c_0}L 
n_o^3 r_{63}E_z.
\label{phiAM}
\eeq
Prelet kristala torej doda vpadnemu valovanju fazni zamik, ki je odvisen od električne poljske
jakosti $E_z$. 

Vpeljemo še karakteristično napetost $U_\pi$, pri kateri je dodatna \index{$\pi$-napetost}
fazna razlika enaka $\pi$ in kristal deluje kot ploščica $\lambda/2$\index{Ploščica $\lambda/2$}
\boxeq{UpiL}{
U_\pi = \frac{\pi c_o}{\omega n_o^3 r_{63}} = \frac{\lambda}{2 n_o^3 r_{63}}.
}
Za kristal KDP je $\pi$-vrednost napetosti pri valovni dolžini $633$~nm okoli  $9000$~V. 
Izračunana napetost je precej velika. Velike delovne napetosti
so značilne za kristalne elektro-optične modulatorje in so njihova
glavna pomanjkljivost. 

\section{Transverzalna modulacija}
\index{Elektro-optična modulacija!transverzalna}
Iz praktičnih razlogov je navadno preprosteje priključiti električno polje v smeri, ki 
je pravokotna na smer širjenja svetlobe. Taki postavitvi pravimo transverzalna in pojavu
transverzalna modulacija\index{Elektro-optična modulacija!transverzalna}.

Tudi to postavitev obravnavajmo na primeru. Za zgled vzemimo kristal LiNbO$_3$, ki 
ima trigonalno simetrijo (3m) in po tabeli~(\ref{table:Pockels}) štiri 
neodvisne komponente: $r_{51}=r_{42}, r_{13}=r_{23}, r_{33}$ in $r_{22}=-r_{12}=-r_{61}$.

\begin{figure}[h]
\centering
\def\svgwidth{80truemm} 
\input{slike/09_TMshema.pdf_tex}
\caption{Shema transverzalne modulacije signala}
\label{fig:tmshema}
\end{figure}
\pagebreak
Naj se svetloba širi vzdolž osi $z$, ki je hkrati tudi optična os, 
električno polje pa priključimo v smeri $y$ (slika~\ref{fig:tmshema}). 
Krajši račun pokaže, da je inverzni dielektrični tenzor pod vplivom polja enak
\beq
\underline{b} = 
 \left[\begin{array}{ccc}
1/n_o^2  -r_{22}E_y& 0& 0\\
0& 1/n_o^2+r_{22}E_y& r_{51}E_y\\
0 & r_{51}E_y&  1/n_e^2
\end{array}\right].
\label{7.8b}
\eeq
Tudi v tem primeru tenzor diagonaliziramo in poiščemo nove lastne vrednosti.
Ob privzetku, da je sprememba zaradi električnega polja majhna ($rE\ll1$),
so nove lastne vrednosti enake
\begin{align}
\lambda_1 &\approx \frac{1}{n_o^2}-r_{22}E_y \\
\lambda_2 &\approx \frac{1}{n_o^2}+ r_{22}E_y \\
\lambda_3 &\approx \frac{1}{n_e^2},
\end{align}
kar ustreza lomnim količnikom 
\begin{align}
n_{x'} &\approx n_o(1+\frac{1}{2}n_o^2r_{22}E_y)\\
n_{y'} &\approx n_o(1-\frac{1}{2}n_o^2r_{22}E_y)\\
n_z' &\approx n_e.
\end{align}
Kako pa je z novimi lastnimi osmi? Hitro ugotovimo, da se tudi pri 
priključenem polju os $x$ ohranja. Pojavi se torej zasuk okoli osi $x$,
ki ga označimo s kotom $\vartheta$. Račun pokaže, da je za smiselne
vrednosti električnega polja ta kot zelo majhen ($\vartheta~
\approx~r_{51}E_y/(1/n_o^2-1/n_e^2) \sim~1$~mrad),
tako da lahko v približku rečemo, da se lastne osi ohranjajo. 

Če potuje svetloba vzdolž osi $z$, sta torej lomna količnika za 
polarizaciji v smeri $x$ in $y$ približno enaka $n_{x'}$ in $n_{y'}$, fazna razlika med 
polarizacijama po preletu kristala z dolžino $L$ pa je 
\beq
\Delta \phi = k_0 n_{y'} L - k_0 n_{x'} L = \frac{\omega}{c_0}L 
n_o^3 r_{22}E_y.
\label{fazaTM}
\eeq
Karakteristična $\pi$-napetost \index{$\pi$-napetost}je tako
\boxeq{UpiT}{
U_\pi = \frac{\lambda d}{2 Ln_o^3 r_{22}},
}
pri čemer moramo ločiti med $L$, ki je dolžina kristala v smeri $z$, in $d$, ki je  
širina v prečni smeri v kateri priključimo napetost. 
Za izbran kristal ($d=5$~mm, $L=1$~cm) je $\pi$-vrednost 
napetosti pri valovni dolžini $633$~nm okoli $2000$~V. 

\begin{remark}
Transverzalno modulacijo lahko dosežemo tudi tako, da se žarek širi vzdolž 
osi $y$, električno polje pa priključimo vzdolž optične osi $z$.
V tem primeru se lastne osi ohranijo in kristal ostane optično enoosen. Vendar 
pa ima tudi ta rešitev določene slabosti. Ker je kristal že sam po sebi dvolomen\index{Dvolomnost}, 
povzroči zunanje polje le majhno dodatno fazno razliko, zato je najbolje, če je dolžina 
kristala taka, da velja $k_{0}L(n_{o}-n_{e})=2N\pi$. Pri tem pa nastopi težava. 
Pogoj je lahko zaradi temperaturnega raztezanja in odvisnosti lomnih količnikov od temperature
izpolnjen le pri eni temperaturi, poleg tega se mora svetloba širiti natančno v smeri $y$.
Zato dvolomnost nemotenega kristala navadno kompenziramo, tako da postavimo 
dva enako dolga kristala zapored, pri čemer sta optični
osi med seboj pravokotni, modulacijska napetost na drugem kristalu pa ima
nasproten predznak. Tedaj se fazna razlika med obema polarizacijama zaradi 
naravne dvolomnosti odšteje, zaradi modulacijske napetosti pa sešteje.
\end{remark}

\section{Amplitudna modulacija}
\label{chap:ampmod}
\index{Elektro-optična modulacija!amplitudna}
Poglejmo, kako lahko elektro-optični pojav izkoristimo za modulacijo
amplitude svetlobnega snopa. Pod vplivom polja pride v kristalu do
faznega zamika med polarizacijama, ki je sorazmeren napetosti 
(enačbi~\ref{phiAM} in \ref{fazaTM}).
Če za tak kristal postavimo analizator, lahko z napetostjo spreminjamo 
prepuščeno moč svetlobe -- amplitudno moduliramo signal.

Vrnimo se k longitudinalni\index{Elektro-optična modulacija!longitudinalna}
modulaciji (slika~\ref{fig:amshema}).
Naj bo vpadna električna poljska jakost $E_0$ polarizirana v smeri $y$. 
Ko priključimo napetost, os $y$ ni več lastna os, ampak sta lastni osi zasukani 
za kot $45~^\circ$ glede na prvotni lastni osi (slika~\ref{fig:amn}). Vpadno 
valovanje razstavimo na komponenti $x'$ in $y'$
\beq
\mathbf{E}_0 = E_0\, \mathbf{e}_y = \frac{E_0}{\sqrt{2}}\left(\mathbf{e}_{x'} + \mathbf{e}_{y'}\right).
\eeq
Po prehodu skozi kristal pride med njima do fazne razlike $\Delta \phi$ 
(enačba~\ref{phiAM}), zato je polje $\mathbf{E}_1$ ob izstopu iz kristala
\beq
\mathbf{E}_1 = \frac{E_0}{\sqrt{2}}\left(e^{ik_0 n_{x'}L}\mathbf{e}_{x'} + 
e^{ik_0 n_{y'}L}\mathbf{e}_{y'}\right)
= \frac{E_0}{\sqrt{2}}e^{ik_0 n_{x'}L}\left(\mathbf{e}_{x'} + 
e^{i\Delta\phi}\mathbf{e}_{y'}\right).
\eeq
Analizator na izhodni strani je obrnjen v smeri $x$, to je pravokotno
na smer vpadne polarizacije, in prepusti le projekcijo obeh lastnih polarizacij
na os $x$
\begin{equation}
\mathbf{E}_2= \mathbf{E}_1 \cdot \mathbf{e}_x = 
\frac{E_0}{\sqrt{2}}e^{ik_0 n_{x'}L}
\left(\frac{1}{\sqrt{2}} -\frac{1}{\sqrt{2}} e^{i\Delta\phi}\right)\mathbf{e}_x .
\label{7.16}
\end{equation}
Gostota prepuščenega svetlobnega toka ob vpadnem toku $j_0$ je tako 
\begin{equation}
j=\frac{1}{4}j_{0}\left|1-e^{i\Delta\phi}\right|^{2}=\frac{1}{2}j_{0}(1-\cos\Delta\phi).
\label{7.17}
\end{equation}
Preoblikujemo izraz in zapišemo prepustnost takega modulatorja ob upoštevanju 
enačbe~(\ref{phiAM})
\boxeq{AMfinal}{
T = \frac{j}{j_0} = \sin\left(\frac{\Delta\phi}{2}\right)^2 = 
\sin\left(\frac{\pi n_o^3 r_{63}U}{\lambda}\right)^2 .
}
\begin{figure}[h]
\centering
\def\svgwidth{90truemm} 
\input{slike/09_AMT.pdf_tex}
\caption{Prepustnost amplitudnega modulatorja v odvisnosti od faznega zamika $\Delta \phi$, 
ki je sorazmeren priključeni napetosti $U$. Če pred vzorec dodamo ploščico $\lambda/4$, 
se pojavi stalni fazni zamik $\pi/2$ in odvisnost prepustnosti od priključene napetosti
je približno  linearna (modra črta).}
\label{fig:amt}
\end{figure}

Ko je napetost na kristalu enaka nič, je $\Delta \phi=0$ in tudi intenziteta 
prepuščene svetlobe $j=0$. To je pričakovano, saj sta analizator in polarizator prekrižana, 
vpadni žarek pa se širi vzdolž lastne osi kristala.
Prepustnost doseže največjo vrednost, ko je $\Delta \phi=\pi$, kar je ravno pri 
$\pi$-napetosti\index{$\pi$-napetost}. Ko torej napetost povečamo z 0 na $U_\pi$, se
prepustnost modulatorja spremeni z 0 na 1 (slika~\ref{fig:amt}).

Pogosto želimo, da je zveza med modulacijsko napetostjo in izhodno
gostoto toka linearna. To lahko dosežemo, če modulator deluje v okolici $\Delta\phi=\pi/2$
(slika~\ref{fig:amt}).
Ena rešitev bi bila dodati stalno visoko napetost, signal pa modulirati okoli
te vrednosti. Precej bolj praktična rešitev je, da med polarizator
in kristal dodamo ploščico $\lambda/4$\index{Ploščica $\lambda/4$}, ki da zahtevan stalni
fazni premik med rednim in izrednim valom. Potem lahko z razmeroma majhno napetostjo
linearno amplitudno moduliramo svetlobo\index{Elektro-optična modulacija!linearna}.

\section{Fazna in frekvenčna modulacija}
\index{Elektro-optična modulacija!fazna}
\index{Elektro-optična modulacija!frekvenčna}
Svetlobo smo amplitudno modulirali, tako da smo z zunanjim
poljem spremenili fazi lastnih valov, zaradi česar je postalo linearno
polarizirano vpadno valovanje po prehodu kristala eliptično polarizirano.
Spremembo polarizacije smo z analizatorjem prevedli v spremembo amplitude.

Včasih pa želimo modulirati fazo vpadne svetlobe. Vrnimo se k primeru longitudinalne
 modulacije. Fazno oziroma frekvenčno modulacijo dosežemo tako,
da vhodno polarizacijo izberemo vzporedno eni od novih lastnih osi kristala, 
na primeri osi $x'$, izhodni polarizator pa odstranimo (slika~\ref{fig:fmshema}). 
\begin{figure}[h]
\centering
\def\svgwidth{80truemm} 
\input{slike/09_FMshema.pdf_tex}
\caption{Shema fazne modulacije signala. Vpadna polarizacija je vzporedna eni od 
novih lastnih osi kristala, ki se pojavijo pod vplivom zunanjega polja.}
\label{fig:fmshema}
\end{figure}

Celoten fazni zamik po preletu skozi kristal zapišemo kot 
\beq
\phi =  k_0 n_{x'} L -\omega_0 t= \frac{\omega_0}{c_0}L \left(n_o -
\frac{1}{2}n_o^3 r_{63}\frac{U}{L}\right)-\omega_0 t,
\label{fmphi}
\eeq
pri čemer smo za lomni količnik zapisali skladno z enačbo~(\ref{EOnx}). Opazimo,
da je fazni zamik odvisen od priključene zunanje napetosti.

Obravnavajmo dva primera spreminjajoče se napetosti. V prvem primeru naj bo 
napetost linearna funkcija časa 
\beq
U= U_0 + \frac{\Delta U}{\Delta t}t.
\eeq
Celotna faza prepuščenega valovanja je potem
\beq
\phi = \frac{\omega_0}{c_0}L n_o - \frac{\omega_0 n_o^3 r_{63}}{2c_0}\left( U_0 + 
\frac{\Delta U}{\Delta t}t\right) - \omega_0 t.
\eeq
Trenutno frekvenco valovanja izračunamo kot negativni odvod faze po času
\beq
\omega = -\frac{d\phi}{dt} = \omega_0 + \frac{\omega_0 n_o^3 r_{63}}{2c_0}\frac{\Delta U}{\Delta t} 
\eeq
oziroma
\boxeq{FMc}{
\omega  = \omega_0 + \Delta \omega.
}
Linearno naraščajoča modulacijska napetost da torej konstanten frekvenčni premik, kar v optiki 
pogosto potrebujemo. Dosegljive spremembe frekvence so seveda dokaj majhne,
do nekaj sto~MHz, saj so omejene s hitrostjo spreminjanja napetosti.
Napetost seveda tudi ne more neomejeno naraščati. Kadar se napetost
vrača na nič, pride do frekvenčnega premika v nasprotni smeri, ki pa ga
lahko zanemarimo, če je čas vračanja bistveno krajši od časa naraščanja.

Poglejmo še drug primer, pri katerem se priključena napetost periodično spreminja. 
Zapišemo jo kot \begin{equation}
U=U_{0}\sin(\omega_{m}t).
\label{7.21}
\end{equation}
Vstavimo gornji izraz v enačbo~(\ref{fmphi}) in zapišemo fazo izhodnega valovanja 
\beq
\phi = \frac{\omega_0}{c_0}L n_o - \frac{\omega_0 n_o^3 r_{63}}{2c_0} U_0\sin(\omega_{m}t)
- \omega_0 t.
\eeq
V primeru linearno spreminjajoče napetosti smo na tem mestu fazo odvajali in dobili hitrost, ki 
je bila konstantna. V tem primeru pa z odvajanjem dobimo kotno hitrost, ki se spreminja s časom. Zato
se računa lotimo drugače. Konstantni člen v gornjem izrazu lahko izpustimo in zapišemo električno
poljsko jakost prepuščenega valovanja  
\beq
E = E_0 \cos\left( \omega_0 t + \frac{\omega_0 n_o^3 r_{63}}{2c_0} U_0\sin(\omega_{m}t)\right)
= E_0 \cos\left( \omega_0 t + \delta \sin(\omega_{m}t)\right),
\eeq
pri čemer je
\beq
\delta = \frac{\omega_0 n_o^3 r_{63}}{2c_0} U_0.
\eeq
Z uporabo Jacobi-Angerjevih\footnote{Nemška matematika Carl Gustav Jacob Jacobi, 1804--1851, in Carl
Theodor Anger, 1803--1858.} identitet 
\begin{align}
\cos\left(\delta\sin x\right)  &=J_0(\delta)+2J_2(\delta)\cos2x+
2J_4(\delta)\cos4x + \ldots\nonumber \quad \mathrm{in}\\
\sin\left(\delta\sin x\right) &=2J_1(\delta)\sin x+2J_3\sin3x+
2J_5\sin5x+\ldots
\label{JA}
\end{align}
je izhodno polje mogoče zapisati v obliki 
\boxeq{fmJA}{
\begin{split}
\frac{E}{E_0} &= J_0(\delta)\cos(\omega_0 t)+ \\
&+J_1(\delta) \cos(\omega_0+\omega_m )t-
J_1 (\delta) \cos(\omega_0 -\omega_m)t + \\ 
&+J_2(\delta)\cos(\omega_0 +2\omega_m)t + 
J_2(\delta)\cos(\omega_0 -2\omega_m)t+ \\
&+ J_3(\delta)\cos(\omega_0 +3\omega_m)t - J_3(\delta)\cos(\omega_0 -3\omega_m)t+\ldots
\end{split}
}
\begin{definition}
Ob upoštevanju Jacobi-Angerjevih identitet (enačbi~\ref{JA}) pokaži, da električno polje
izhodne svetlobe ob priključeni izmenični napetosti s frekvenco $\omega_m$ ustreza
polju v enačbi~(\ref{fmJA}).
\end{definition}
Zaradi periodične fazne modulacije se torej v spektru pojavijo stranski pasovi, ki so 
od osnovne frekvence $\omega_0$ odmaknjeni za večkratnike modulacijske frekvence $\omega_m$. 
Njihova velikost je podana s kvadratom Besselovih funkcij parametra $\delta$.
Ker je ta navadno majhen, se pogosto zadovoljimo le s prvim členom.

\begin{remark}
Elektro-optični pojav izkoriščamo tudi za uklanjanje žarkov. 
\index{Elektro-optični deflektor}
Najpreprostejši primer deflektorja je trikotna prizma z elektrodama na osnovnih 
ploskvah. Svetloba se ob prehodu skozi prizmo lomi v odvisnosti od njenega 
lomnega količnika, tega pa lahko spreminjamo 
z napetostjo na elektrodah. Praktično je bolj uporabna dvojna prizma. Sestavljena je iz dveh enakih 
prizem, ki skupaj tvorita kvader, pri tem pa optični osi zgornje in spodnje prizme
kažeta v nasprotnih smereh. S spreminjanjem napetosti, ki jo priključimo prečno na smer
razširjanja svetlobe, lahko zelo hitro in zelo natančno spreminjamo smer izhodnega žarka. 
Vendar ta pristop ni splošno uveljavljen, predvsem zaradi velike napetosti, ki je 
potrebna za znatno uklanjanje. Veliko bolj razširjen je akusto-optični pojav, ki ga 
bomo spoznali v naslednjem razdelku. 
\begin{figure}[h]
\centering
\def\svgwidth{95truemm} 
\input{slike/09_EOdefl.pdf_tex}
\caption{Shema elektro-optičnega deflektorja}
\label{fig:deflshema}
\end{figure}
\end{remark}

% \section{*Modulacija pri visokih frekvencah}
% Do zdaj smo obravnavali elektro-optično modulacijo pri statičnih poljih ali 
% poljih, ki so se le počasi spreminjali s časom. Poglejmo še,
% kaj se zgodi pri visokih modulacijskih frekvencah.
% 
% Elektro-optični pojav pri nizkih frekvencah ima dva prispevka: direktnega,
% kjer zunanje polje vpliva neposredno na elektronsko polarizabilnost,
% in posrednega preko piezoelektričnega pojava. Snovi, ki nimajo centra
% inverzije, so tudi piezoelektrične in se v zunanjem električnem polju
% deformirajo. Deformacija pa povzroči spremembo lomnega količnika,
% o čemer bomo podrobneje govorili v enem od naslednjih oddelkov. Celotno
% spremembo tenzorja $b_{ij}$ lahko zapišemo 
% \begin{eqnarray}
% \delta b_{ij} & = & r_{ijk}^{\ast}E_{k}+p_{ijlm}S_{lm}\nonumber \\
%  & = & r_{ijk}^{\ast}E_{k}+p_{ijlm}\pi_{lmk}E_{k}
% \end{eqnarray}
%  Tu je $S_{lm}=\pi_{lmk}E_{k}$ piezoelektrično povzročena deformacija.
% Pri nizkih frekvencah sta oba prispevka primerljivo velika in je efektivni
% elektro-optični tenzor $r_{ijk}=r_{ijk}^{\ast}+p_{ijlm}\pi_{lmk}$.
% Pri dovolj velikih frekvencah deformacija kristala ne more več slediti
% modulacijski napetosti in ostane le direktni prispevek $r_{ijk}^{\ast}$.
% To se zgodi nad akustičnimi resonancami kristala. Pri akustičnih resonancah,
% to je, kadar modulacija v kristalu vzbudi stoječe zvočno valovanje,
% pa se piezoelektrični prispevek resonančno poveča.
% 
% Pogoj za akustično resonanco je, da je dimenzija kristala mnogokratnik
% polovice valovne dolžine akustičnega vala v kristalu. Uporabne dimenzije
% kristalov so reda velikosti centimeter, hitrost zvočnih valov je okoli
% 5000~m/s, tako da so resonance v področju od nekaj sto kHz do
% nekaj deset MHz. Mogoče jih je tudi izkoristiti za povečanje elektro-optičnega
% efekta pri izbrani frekvenci.
% 
% Pri visokih frekvencah postane pomembna tudi električna vezava modulatorja.
% Kristal predstavlja neko kapacitivno breme. Njegova impedanca pada
% z rastočo frekvenco, zato je vedno večji del padca napetosti na notranjem
% uporu izvora napetosti. Pomagamo si lahko tako, da vzporedno s kristalom
% vežemo še tuljavo, tako da je resonančna frekvenca $1/(L_{t}C)$ nastalega
% nihajnega kroga enaka željeni modulacijski frekvenci $\omega_{m}$.
% Tedaj je večina padca napetosti na kristalu in tuljavi. Da resonanca
% ni preostra in da je na voljo dovolj širok pas modulacijskih frekvenc,
% vežemo vzporedno s kristalom še upor z upornostjo $R$. Širina modulacijskega
% pasu je 
% \begin{equation}
% \Delta\omega_{m}=\frac{1}{RC}\;.\label{7.24}
% \end{equation}
%  Na uporu se troši moč 
% \begin{equation}
% P=\frac{1}{2}\frac{U^{2}}{R}=\frac{1}{2}U^{2}C\Delta\omega_{m}=
% \frac{\epsilon\epsilon_{0}La}{2d}U^{2}C\Delta\omega_{m}\;,\label{7.25}
% \end{equation}
%  kjer je $a$ velikost kristala v prečni smeri. Naj bo $U$ ravno
% napetost, ki da fazno razliko $\pi$. Potem sledi z uporabo enačbe
% \ref{7.18} 
% \begin{equation}
% P=\frac{A}{L}\,\frac{\epsilon\epsilon_{0}\Delta\omega_{m}\lambda^{2}}{3n_{0}^{6}r^{2}}\;,\label{7.26}
% \end{equation}
%  kjer je $A$ prečni presek kristala. Potrebna moč je odvisna od lastnosti
% modulatorja in širine modulacijskega pasu. Pri širini modulacijskega
% pasu 1~MHz in preseku kristala 1~cm$^{2}$ je potrebna moč nekaj
% deset W, kar je za visokonapetosten in hiter izvor že znatna moč.

\section{Elasto-optični in akusto-optični pojav}
\index{Elasto-optični pojav}
\index{Akusto-optični pojav}
Pri elasto-optičnem pojavu dielektrične lastnosti snovi in njen lomni količnik
spreminjamo z mehansko deformacijo. Podobno kot pri elektro-optičnem pojavu opišemo pojav
s spremembo inverznega dielektričnega tenzorja\index{Dielektričnost!inverzna}
\begin{equation}
\underline{b} = \underline{\tilde{b}}+ \Delta\underline{b},
\end{equation}
pri čemer je $\underline{\tilde{b}}$ tenzor v odsotnosti mehanske deformacije, 
$\Delta\underline{b}$ pa sprememba tenzorja zaradi deformacije snovi. Zapišemo jo kot
\boxeq{7.27}{
 \Delta b_{ij}=p_{ijkl}S_{kl}.
}
Sorazmerna je s tenzorjem deformacije snovi oziroma 
Greenovim tenzorjem\footnote{Angleški matematični fizik George Green, 1793--1841.} v
linearnem približku
\begin{equation}
S_{kl}=\frac{1}{2}\left({\frac{\partial u_{k}}{\partial x_{l}}}+{\frac{\partial u_{l}}{\partial x_{k}}}\right),
\label{7.28}
\end{equation}
pri čemer je $\mathbf{u}$ vektor deformacije. 

Vpeljali smo še sorazmernostni faktor
$p_{ijkl}$, ki ga imenujemo elasto-optični tenzor\index{Elasto-optični tenzor}. 
Tenzor $p$ je različen od nič v vsaki snovi, ker povezuje dva simetrična tenzorja 
drugega ranga. Posledično je simetričen v prvem in drugem paru indeksov
\beq
p_{ijkl} = p_{jikl} = p_{ijlk} =p_{jilk}.
\eeq
V najbolj splošnem primeru triklinske kristalne simetrije 
ima tako 36 neodvisnih komponent, v bolj simetričnih snoveh pa se število 
neodvisnih komponent še zmanjša. Če vpeljemo skrajšan zapis indeksov ($xx = 1,
yy=2, zz = 3, yz = 4, zx = 5, xy = 6$), zapišemo tenzor za primer izotropne snovi kot
\beq
\underline{p}_{\textrm{izo}} = 
\left[\begin{array}{cccccc}
p_{11} & p_{12}& p_{12}&0&0&0\\
p_{12} & p_{11}& p_{12}&0&0&0\\
p_{12} & p_{12}& p_{11}&0&0&0\\
0 & 0& 0&p_{44}&0&0\\
0 & 0& 0&0&p_{44}&0\\
0 & 0& 0&0&0&p_{44}
\end{array}\right],
\label{tenzorp}
\eeq
pri čemer je $p_{44}= \frac{1}{2}(p_{11}-p_{12})$. Koeficienti tenzorja so 
brezdimenzijski, njihova tipična vrednost pa je $p\sim0,1$. Za vodo, na primer, 
velja $p_{11} \simeq p_{12} = 0,31$, za LiNbO$_3$\index{LiNbO$_3$} pa $p_{11} = -0,02, 
p_{12} = 0,08, p_{13} = 0,13,  p_{14} = -0,08, p_{31} = 0,17, p_{33} = 0,07,
p_{41} = -0,15, p_{44} = 0,12$.

Podobno kot pri elektro-optičnem pojavu lahko iz enačbe~(\ref{7.27}) izrazimo
spremembo dielektričnega tenzorja 
\begin{equation}
\Delta\epsilon_{ij}=-\tilde{\epsilon}_{ii}\tilde{\epsilon}_{jj}p_{ijkl}S_{kl},
\label{7.29}
\end{equation}
kjer smo že predpostavili, da je nemoteni $\epsilon$ diagonalen. 

Dvojni lom\index{Dvolomnost}, ki se pojavi v deformirani snovi, izkoriščamo za študij
mehanskih napetosti v modelih, ki so izdelani iz prozorne plastične
snovi. Nas bo v nadaljevanju zanimal uklon svetlobe na periodični
modulaciji lomnega količnika, ki nastane zaradi zvočnega valovanja v snovi. Takemu pojavu
pravimo tudi akusto-optični pojav.

\begin{definition}
Po izotropni snovi se širi longitudinalno valovanje vzdolž smeri $z$, tako da 
deformacijo v snovi zapišemo kot
\beq
\mathbf{u} = A \cos(q z - \Omega t)\mathbf{e}_z.
\eeq
Pokaži, da je taka snov dvolomna z optično osjo vzdolž osi z, lastni 
lomni količniki pa so 
\begin{align}
n_{x'} &\approx n(1+\frac{1}{2}n^2p_{12}q A \sin (q z - \Omega t))\\
n_{y'} &\approx n(1+\frac{1}{2}n^2p_{12}q A \sin (q z - \Omega t))\\
n_z' &\approx n(1+\frac{1}{2}n^2p_{11}q A \sin (q z - \Omega t)),
\end{align}
kjer je $n$ lomni količnik v odsotnosti motnje.
\end{definition}

\section{Uklon svetlobe na zvočnem valovanju}
\index{Akusto-optični pojav}
Vzbudimo v plasti prozorne izotropne snovi zvočno valovanje z valovno dolžino $\Lambda$, 
ki potuje v smeri $x$. To naredimo tako, da na eno stran snovi priključimo piezoelektrik, 
ki se pod izmenično napetostjo periodično krči in razteza s krožno frekvenco $\Omega$.
Na drugo stran kristala damo akustični absorber ali pa reflektor, tako da lahko 
v snovi vzbudimo tudi stoječe valovanje. 
Zaradi zvočnega valovanja se v snovi periodično spreminja gostota in 
z njo lomni količnik
\beq
n = \tilde{n} + \Delta n \sin \left(\frac{2\pi}{\Lambda} x- \Omega t\right).
\eeq
V zgoščini je lomni količnik nekoliko večji kot v razredčini, zato je optična pot na takem mestu
skozi plast daljša. Ravno svetlobno valovanje, ki vpada na plast pravokotno glede
na smer širjenja zvoka, po izstopu zato nima povsod enake faze, 
valovno čelo pa je periodično modulirano s periodo 
valovne dolžine zvočnega valovanja. Zvočno valovanje v snovi torej deluje kot 
optična fazna mrežica. Tipična frekvenca, s katero vzbujamo elastično
deformacijo, je okoli $\Omega=50$~MHz, ustrezna valovna dolžina pa okoli $\Lambda = 100~\mu$m. 
Frekvence, ki so v uporabi, navadno sežejo od nekaj MHz prek 10 GHz. Vsa ta valovanja imenujemo
zvočno valovanje. 

\begin{figure}[h]
\centering
\def\svgwidth{60truemm} 
\input{slike/09_AOshema.pdf_tex}
\caption{Vpadna svetloba se na stoječem zvočnem valovanju v snovi uklanja.}
\label{fig:ao}
\end{figure}

Oglejmo si dva limitna primera. V prvem primeru je debelina plasti $d$, 
v kateri vzbujamo zvočno valovanje, zelo majhna (slika~\ref{fig:ao_bragg} levo). 
Takrat modulator deluje kot tanka uklonska mrežica in pojavi se veliko 
uklonskih vrhov, intenziteta posameznega žarka pa je razmeroma majhna. 
Kote, pod katerimi se pojavijo ojačitve, izračunamo po preprosti enačbi
\boxeq{R-N}{
\Lambda (\sin \vartheta - \sin \beta ) = N \lambda,
}
pri čemer je $\lambda$ valovna dolžina svetlobe v snovi, $N$ pa celo število. Takemu pojavu 
pravimo Raman-Nathov uklon\footnote{Indijski fizik in nobelovec Sir Chandrasekhara 
Venkata Raman, 1888--1970, 
in indijski fizik N. S. Nagendra Nath.}\index{Raman-Nathov uklon}. 
Opazimo ga pri razmeroma nizkih zvočnih frekvencah 
(pod $\sim10$~MHz) in majhnih debelinah (pod $d\sim 1$~cm) pri poljubnem vpadnem 
kotu $\vartheta$.
\begin{figure}[h]
\centering
\def\svgwidth{50truemm} 
\input{slike/09_AO_1.pdf_tex}\qquad
\def\svgwidth{65truemm} 
\input{slike/09_AO_2.pdf_tex}
\caption{Ob vpadu svetlobe na tanko plast zvočnega valovanja se pojavi veliko uklonskih vrhov, 
na debeli plasti zvočnega valovanja pa je opazen zgolj en uklonjen vrh, 
pa še ta le ob izpolnjenem Braggovem pogoju (enačba~\ref{7.29a}).}
\label{fig:ao_bragg}
\end{figure}

V nasprotnem limitnem primeru se svetloba uklanja na ravnih zvočnih valovih in modulator deluje 
kot debela uklonska mrežica. V splošnem je delež uklonjene svetlobe na taki mrežici 
neuporabno majhen. 
Znaten postane le tedaj, kadar je izpolnjen Braggov\index{Braggov uklon}
pogoj\footnote{Angleška znanstvenika in nobelovca Sir William Henry Bragg, 1862--1942,
in Sir William Lawrence Bragg, 1890--1971.}
\boxeq{7.29a}{
2 \Lambda\sin\vartheta=\pm N\lambda.
}

Poglejmo natančneje, kako pridemo do gornjega pogoja. Zapišimo pogoj
za ohranitev gibalne količine fotona pri sipanju na zvočnem valu
\begin{equation}
\mathbf{k}_{0}\pm\mathbf{q}=\mathbf{k}_{1},
\label{7.30}
\end{equation}
kjer je $\mathbf{k}_{0}$ valovni vektor vpadne svetlobe, $\mathbf{k}_{1}$
valovni vektor uklonjenega svetlobnega snopa, $\mathbf{q}$ pa valovni
vektor zvočnega vala. Znak plus velja, kadar potuje zvok proti projekciji
$\mathbf{k}_{0}$ na $\mathbf{q}$, negativen predznak pa ob potovanju zvoka v nasprotno smer. 
Ker je frekvenca zvočnega vala dosti nižja od frekvence svetlobe, se frekvenca svetlobe 
pri sipanju le malo spremeni in $\mathbf{k}_{0}$ in $\mathbf{k}_{1}$ sta po velikosti skoraj enaka.
Tedaj je $q=2k_{0}\sin\vartheta$ (glej sliko~\ref{fig:ao_bragg3}), od koder sledi Braggov pogoj 
(enačba~\ref{7.29a}). Obenem je vpadni kot na zvočni val enak izhodnemu, kar pomeni, da se na
zvočnem valu Braggovo sipana svetloba zrcalno odbije. Razmere so torej
povsem analogne Braggovemu sipanju rentgenske svetlobe na kristalnih
ravninah. Kot bomo pokazali v nadaljevanju, je ob izpolnjenem Braggovem pogoju mogoče doseči, 
da se vsa vpadna svetloba uklanja.
\begin{figure}[h]
\centering
\def\svgwidth{40truemm} 
\input{slike/09_AO_3.pdf_tex}
\caption{K izpeljavi Braggovega pogoja}
\label{fig:ao_bragg3}
\end{figure}
\begin{remark}
 Poskusimo še malo bolj natančno oceniti, kdaj je v veljavi Raman-Nathov in kdaj Braggov režim. 
Izhajajmo iz pogoja, da je razširitev žarka na debelini plasti zvočnega valovanja dovolj
 majhna, da se snop ne širi iz območja zgoščine v območje razredčine, da se torej ne razširi za 
 več kot za $\Lambda/2$. Tako zapišemo
 divergenco kot (enačba~\ref{eq:divergenca-snopa}) $\theta \sim \lambda/w_0 \sim 2\lambda/\Lambda$,
  ki ne sme presegati razširitve $\theta \sim \Lambda/2d$.
  Sledi kriterij za debelino $d$, pri kateri preidemo iz enega v drug režim
\beq
d \sim \frac{\Lambda^2}{4 \lambda}.
\eeq
Če svetloba z $\lambda = 1~\mu$m vpade na kristal, v katerem je vzbujeno zvočno
valovanje z valovno dolžino $\Lambda = 0,2$~mm in frekvenco $\Omega = 150$~MHz, je mejna 
debelina $d \sim 1$~cm.
\end{remark}

Če je zvočno valovanje potujoče, kar smo v gornjem razmišljanju že privzeli
s tem, ko smo mu pripisali natanko določen valovni vektor $\mathbf{q}$,
se spremeni tudi frekvenca sipanega vala zaradi Dopplerjevega premika
pri odboju na zvočnem valovanju, ki potuje s hitrostjo $v_{z}$. Upoštevati
moramo le projekcijo na smer vpadne in odbite svetlobe, zato je 
\begin{equation}
\frac{\Delta\omega}{\omega}=\pm\frac{2v_{z}\sin\vartheta}{c}=
\pm\frac{2\Omega\Lambda\sin\vartheta}{2 \pi c}=\pm\frac{\Omega}{\omega},
\label{7.32}
\end{equation}
pri čemer smo uporabili Braggov pogoj (enačba~\ref{7.29a}). Sprememba frekvence
sipane svetlobe je torej kar enaka frekvenci zvočnega valovanja. To je seveda v skladu 
z gornjo zahtevo, da se pri uklonu na zvočnem valovanju ohranja energija
vpadnega fotona in kvanta zvočnega valovanja (fonona), ki se pri sipanju 
absorbira ali pri njem nastane.

Malenkost drugačno je obnašanje, ko v snovi vzbudimo stoječe zvočno valovanje. 
Takrat lahko sipanje obravnavamo kot vsoto sipanja na dveh valovanjih z valovnima 
vektorjema $\mathbf{q}$ in $-\mathbf{q}$. Smer Braggovo sipanega vala je obakrat enaka, 
frekvenca pa se enkrat poveča, drugič zmanjša za $\Omega$. Zato se pojavi utripanje
sipanega vala s frekvenco $2\Omega$.

\subsection*{Uporaba akusto-optičnih modulatorjev}
Spoznali smo, da lahko z zvočnim valovanjem spreminjamo smer vpadne svetlobe.
Bistvena razlika od navadnih uklonskih mrežic je dinamičnost akusto-optičnih modulatorjev, 
saj lahko uklonski kot svetlobe hitro spreminjamo, pri čemer pa smo omejeni s 
tem, da mora biti vsaj približno izpolnjen Braggov pogoj. S kombinacijo dveh med seboj 
pravokotnih akusto-optičnih modulatorjev lahko žarek 
premikamo po ravnini, kar s pridom uporabljamo v različnih optičnih napravah, 
na primer v optičnih pincetah, optičnih čitalcih ali 
optičnih litografskih zapisovalnikih.

Z vklapljanjem in izklapljanjem zvočnega valovanja, ki ga vzbujamo s piezoelektričnim elementom,
na katerega pritisnemo izmenično napetost, lahko moduliramo intenziteto
direktnega svetlobnega snopa. To potrebujemo na primer pri preklapljanju
kvalitete laserskega resonatorja.

Tretji primer uporabe je spreminjanje frekvence svetlobe. Možne so spremembe
do nekaj 100~MHz, kar je ravno primerno za uporabo v laserskih merilnikih
hitrosti, kjer merimo frekvenco utripanja med referenčno svetlobo in svetlobo, odbito od
merjenega predmeta. Če ima referenčna svetloba
isto frekvenco kot merilni snop, ni mogoče določiti predznaka hitrosti
predmeta, če pa referenčni svetlobi nekoliko spremenimo frekvenco,
se pojavi utripanje tudi tedaj, ko predmet miruje. Frekvenca utripanja
se poveča ali zmanjša glede na predznak hitrosti predmeta.

Naslednja pomembna uporaba je za uklepanje faz
v laserskem resonatorju. Če je v Braggovem elementu prisotno stoječe zvočno
valovanje, je amplituda direktnega snopa modulirana s frekvenco zvoka.
Kadar je frekvenca zvoka ravno enaka razmiku frekvenc laserskih nihanj,
lahko nastanejo uklenjene faze vzbujenih nihanj in s tem kratki, periodični
sunki svetlobe.

Zanimiva je tudi uporaba Braggovega elementa za izdelavo
hitrega frekvenčnega analizatorja električnih signalov.  
Piezoelektrični element vzbujamo z električnim signalom,
ki ima neznan spekter. Enak spekter imajo tudi vzbujeni zvočni valovi, 
pri čemer vsakemu valu določene frekvence ustreza določen kot odklona svetlobnega
snopa. Za Braggovim elementom postavimo lečo. Vsak delni uklonjeni
snop da v goriščni ravnini svetlo točko, katere položaj je odvisen
od kota odklona in torej od frekvence zvočnega vala. Spekter zaznamo
z vrstičnim detektorjem. Akusto-optični element oziroma Braggova celica 
torej frekvenčni spekter zvočnih valov prevede v prostorski
spekter prepuščene svetlobe. Prostorski spekter svetlobe pa lahko
analiziramo z lečo, ki nam v goriščni ravnini da prostorsko
Fourierevo transformacijo svetlobnega snopa pred lečo.

\section{*Račun akusto-optičnega pojava}
\index{Akusto-optični pojav}
Izračunajmo intenziteto svetlobe, ki se uklanja na zvočnem valovanju. Izhajamo 
iz valovne enačbe v nehomogenem sredstvu, kar je dokaj težaven problem
in se moramo zateči k približkom. Uporabili bomo metodo sklopljenih valov. 

Naj vzporeden snop zvočnega valovanja s širino $d$ in valovnim vektorjem $\mathbf{q}$ 
potuje v smeri $x$.
Nanj pod kotom $\vartheta$ glede na os $z$ vpada ravno svetlobno valovanje z valovnim vektorjem 
$\mathbf{k}=(k_{x},0,k_{z})= k(\sin\vartheta,0,\cos\vartheta)$.
Vse valovanje, vpadno na levi od zvočnega snopa in izhodno na njegovi desni,
obravnavajmo znotraj snovi, da nam ni treba upoštevati še loma, ki
le zaplete izraze. 

Privzemimo, da se v snovi zaradi zvočnih valov spremeni le velikost
dielektrične konstante. Ob upoštevanju zveze med spremembo dielektričnosti in deformacijo
v zvočnem valu~(enačba~\ref{7.29}) lahko spremembo dielektričnosti
zapišemo kot  
\begin{equation}
\epsilon=\tilde{\epsilon}+\Delta\epsilon = 
\tilde{\varepsilon} -\tilde{\epsilon}^{2}pS_{0}\sin(qx-\Omega t).
\label{7.33}
\end{equation}
Zaradi spremembe dielektričnosti pride do pojava
dodatne polarizacije $\Delta P$
\beq
\Delta P = \varepsilon_0 \Delta \varepsilon E = - \varepsilon_0 
\tilde{\varepsilon}^2 p S_0 \sin(qx-\Omega t)E.
\eeq
Dodatna polarizacija v valovno enačbo doprinese dodaten nehomogen člen, podobno
kot pri nelinearni optiki (enačba~\ref{8.3}). Zapišemo
\begin{equation}
\nabla^{2}E-\frac{\tilde{\epsilon}}{c^{2}}{\frac{\partial E^{2}}
{\partial t^{2}}}=\mu_{0}{\frac{\partial^2 \Delta P}{\partial t^{2}}},
\label{7.33a}
\end{equation}
pri čemer smo privzeli, da je $\nabla\cdot\mathbf{E}\approx 0$, čeprav je
$\epsilon$ funkcija kraja. 

Enačbo~(\ref{7.33a}) brez dodane polarizacije $\Delta P$ rešijo ravni valovi 
z valovnim vektorjem $\mathbf{k}$ in frekvenco $\omega$. Tej rešitvi se 
primešajo valovi z valovnim vektorjem $\mathbf{k}\pm n\mathbf{q}$
in frekvenco $\omega\pm n\Omega$. Zato iščemo rešitve v obliki vsote
ravnih valov, torej Fouriereve vrste
\begin{equation}
E=\sum_{n}A_{n}(z)e^{in(qx-\Omega t)}e^{i(k_{x}x+k_{z}z-\omega t)}.
\label{7.34}
\end{equation}
Zaradi sklopitve preko $\Delta P$ smo dopustili, da so amplitude
$A_{n}$ funkcije $z$. Če je $\Delta\epsilon$ dovolj majhen, se $A_{n}(z)$
le počasi spreminjajo.

Izračunajmo 
\begin{equation}
\nabla^{2}E=\sum_{n}\left( -[k_{z}^{2}+(k_{x}+nq)^{2}]A_{n}(z)+2ik_{z}A_{n}'(z)\right) \, e^{i[(k_x+nq)x+k_{z}z-(\omega+n\Omega)t]}.
\label{7.35}
\end{equation}
Člene z $A_{n}''$ lahko izpustimo, če je le $k_{z}A_{n}'\gg A_{n}''$ oziroma 
kadar se $A_{n}$ spreminjajo počasi v primerjavi z exp$(ik_{z}z)$. Drugi odvod 
polarizacije po času da
\beq
\begin{split}
\frac{\partial^2 \Delta P}{\partial t^2} =&-\frac{\varepsilon_0 \tilde{\varepsilon}^2pS_0}{2i} A_n(z) \exp\left(i[(k_x+nq)x+k_{z}z-(\omega+n\Omega)t]\right) \cdot \\ &\sum_{n}\left(
-[n\Omega+\omega+\Omega]^2e^{i(qx-\Omega t)} + [n\Omega+\omega-\Omega]^2e^{i(-qx+\Omega t)} \right),
\label{7.35a}
\end{split}
\eeq
drugi odvod polja po času pa 
\beq
\frac{\partial E^{2}}{\partial t^{2}} = - (n\Omega + \omega)^2 
\sum_{n}A_{n}(z)e^{in(qx-\Omega t)}e^{i(k_{x}x+k_{z}z-\omega t)}.
\label{7.35b}
\eeq
Vstavimo izraze (\ref{7.35}), (\ref{7.35a}) in (\ref{7.35b}) v valovno enačbo (\ref{7.33a})
in izenačimo člene z isto časovno in prostorsko frekvenco, na primer
s $k_z z+(k_x+mq)x-(\omega+m\Omega)t$. Tako dobimo 
\begin{eqnarray}
-[k_{z}^{2}+(k_{x}+mq)^{2}]A_{m}+2ik_{z}A_{m}' + \frac{\tilde{\varepsilon}}{c^2}(m\Omega+\omega)^2A_m
=\\ =\frac{\mu_0\varepsilon_0\tilde{\varepsilon}^2pS_0}{2i}(\omega+m\Omega)^{2}(A_{m-1}-A_{m+1}).
\end{eqnarray}
Upoštevamo, da je 
\beq 
k_{x}^{2}+k_{z}^{2}=k^{2}=\frac{\tilde{\epsilon}\omega^2}{c^{2}}
\eeq
in naredimo približek $(\omega +m\Omega)^2 \approx \omega^2$.
Sledi
\begin{equation}
A_{m}'+i\beta_{m}A_{m}+\xi(A_{m+1}-A_{m-1})=0,
\label{7.37}
\end{equation}
kjer sta
\begin{equation}
\beta_{m}=\frac{mq}{k_{z}}(k_{x}+\frac{1}{2}mq)
\label{7.38}
\end{equation}
 in 
\begin{equation}
\xi=-\frac{\tilde{\epsilon} pS_0k^2}{4k_z}.
\label{7.39}
\end{equation}
Reševanje sistema enačb~(\ref{7.37}) je težavno, zato poiščimo rešitve le v treh
pomembnih limitnih primerih. Amplituda vala, ki vpada z leve, naj bo $A_{0}(0)=A_{0}$, 
za ostale pa naj velja $A_{n}(0)=0$.

Najprej privzemimo, da je $L\xi \ll 1$, da je torej velikost $\Delta \epsilon$
majhna in debelina zvočnega snopa ne prevelika. Tedaj je pri vseh
$z$ in za pozitivne $m$ $A_{m+1}\ll A_{m}$ in lahko člen $A_{m+1}$
v enačbi~(\ref{7.37}) izpustimo. S tem zapišemo preprost sistem enačb
\begin{equation}
A_{m}'+i\beta_{m}A_{m}=\xi A_{m-1},
\label{7.40}
\end{equation}
 ki jih lahko zapored integriramo: 
\begin{equation}
A_{m}(z)=\xi e^{-i\beta_{m}z}\int_{0}^{z}A_{m-1}(z')
e^{i\beta_{m}z'}dz'.
\label{7.41}
\end{equation}
Podobne izraze izpeljemo za negativne $m$.

Poglejmo posebej prvi uklonjeni val z amplitudo $A_{1}$. Po predpostavki,
da je $A_{\pm1}\ll A_{0}$, se le malo energije uklanja iz osnovnega
vala in lahko privzamemo, da je $A_{0}(z)$ skoraj konstanta. 
Potem lahko integral v enačbi~(\ref{7.41})
izračunamo
\begin{equation}
A_{1}(d)=A_{0}\xi d\,\frac{\sin\beta_{1}d/2}{\beta_{1}d/2}\, e^{-i\beta_{1}d/2}\;,
\label{7.41a}
\end{equation}
pri čemer je $d$ debelina plasti zvočnega valovanja.
Funkcija $A_{1}(d)$ ima vrh pri $\beta_{1}=0$, to je po enačbi~(\ref{7.38}) pri 
\begin{equation}
k_x+ \frac{q}{2} = k \sin\vartheta + \frac{q}{2} = 0
\qquad \mathrm{ali} \qquad 
2\Lambda\sin\vartheta=-\lambda.
\label{7.43}
\end{equation}
Vidimo, da predstavlja $\beta_{1}=0$ ravno pogoj za Braggovo sipanje vpadnega
vala.

Delež moči uklonjenega vala je potem
\boxeq{7.45}{
\frac{I_{1}}{I_{0}}=\left|\frac{A_{1}}{A_{0}}\right|^{2}=(\xi d)^{2}
\left(\frac{\sin\beta_{1}d/2}{\beta_{1}d/2}\right)^{2}.
}
Če je Braggov pogoj izpolnjen, je $I_{1}/I_{0}=(\xi d)^{2}$,
kar lahko velja le, dokler je $\xi d\ll1$. Kadar intenziteta uklonjenega žarka
tako naraste, da ta pogoj ni več izpolnjen, je treba v računu upoštevati tudi 
zmanjšanje moči vpadnega snopa.

Drug primer naj bo približek, da sta le $A_{0}$ in $A_{1}$ različni od nič, 
opustimo pa omejitev $d\xi\ll 1$. Ta približek je smiseln, saj je 
Braggov pogoj hkrati lahko izpolnjen le za en uklonjen val, na
primer $m=1$. Tedaj so vse ostale amplitude $A_{m, m\ne0,1}$
majhne in ne vplivajo na $A_{1}$. Zaradi velike pretvorbe 
$A_{0}(z)$ ne smemo več obravnavati kot konstante. Upoštevamo 
izpolnjen Braggov pogoj (enačba~\ref{7.43}) in iz sistema enačb~(\ref{7.37})
dobimo\index{Braggov uklon}
\begin{eqnarray}
A_{0}'+\xi A_{1} & = & 0\nonumber \\
A_{1}'-\xi A_{0} & = & 0.
\end{eqnarray}
Ob začetnih pogojih $A_{0}(0)=A_{0}$ in $A_{1}(0)=0$ sta rešitvi gornjih enačb
\beq
A_{0}(d) = A_{0}\cos (\xi d)
\eeq
in
\beq
A_{1}(d) = A_{0}\sin (\xi d).
\eeq
Če je izpolnjen Braggov pogoj, se moč vpadnega vala na razdalji $\pi/(2\xi)$
skoraj vsa pretoči v uklonjeni snop, nato pa zopet nazaj (slika \ref{s7.10}).
Za čim bolj učinkovito delovanje akusto-optičnega modulatorja seveda
želimo doseči ravno take pogoje.
\begin{figure}[h]
\centering
\def\svgwidth{90truemm} 
\input{slike/09_AOcalc.pdf_tex}
\caption{Intenziteta prepuščenega in uklonjenega valovanja na zvočnem valovanju v
odvisnosti od debeline plasti zvočnega valovanja}
\label{s7.10}
\end{figure}

V gornja izraza vstavimo še parameter $\xi$, ki je podan z enačbo~(\ref{7.39}).
Razmerje med močjo uklonjenega in vpadnega snopa je tako
\boxeq{7.48}{
\frac{I_{1}}{I_{0}}=\sin^{2}\left(\frac{\pi n_{0}^{3}pS_{0}d}{2\lambda\cos\vartheta}\right).
}

Amplituda deformacije $S_0$ je povezana z gostoto energijskega toka
zvočnega valovanja 
\begin{equation}
j_{z}=\frac{1}{2}CS_{0}^{2}v_{z},
\label{7.49}
\end{equation}
kjer je $C$ elastična konstanta snovi, $v_z$ pa hitrost zvoka v snovi. 
Iz zveze $v_{z}^{2}=C/\varrho$ izrazimo $C$ z gostoto $\varrho$, s čemer dobimo 
\begin{equation}
S_{0}=\sqrt{\frac{2j_{z}}{\varrho v_{z}^{3}}}.
\label{7.50}
\end{equation}
Praktično je vpeljati merilo uporabnosti neke snovi za akusto-optični modulator. To je koeficient 
\begin{equation}
M=\frac{n_{0}^{6}p^{2}}{\varrho v_{z}^{3}}.
\label{7.51}
\end{equation}
Večja kot je njegova vrednost, bolj izrazit je akusto-optični pojav v dani snovi. 

Poglejmo primer. V kremenu z gostoto $\varrho=2,2\cdot10^{3}$ kg/m$^{3}$ je hitrost zvoka $v_{z}=6000$~m/s,
$\tilde{n}=1,46$ in $p=0,2$. To da $M=8\cdot10^{-16}$~W/m$^{2}$.
Pri gostoti zvočnega toka 10~W/cm$^{2}$ in valovni dolžini svetlobe 633~nm
pride do popolnega prenosa moči v uklonjeni snop pri debelini $d=3$~cm. Gornja gostota
zvočnega toka je kar velika in je ni prav lahko doseči, zato so 
uklonski izkoristki navadno nekaj manjši od 1.

Izračunajmo še kot odklona uklonjenega vala $\theta = 2 \vartheta$  
\begin{equation}
\theta \approx \frac{q}{k}=\frac{\lambda}{\tilde{n}\Lambda}=1,7\cdot10^{-3}.
\label{7.52}
\end{equation}
Uklonski kot je torej precej majhen.

\begin{remark}
Opisani račun izkoristka uklona na zvočnih valovih je uporaben tudi
pri računu izkoristka holograma. V primeru faznega holograma je račun
povsem enak in nam kaže tudi razliko med tankim in debelim hologramom,
kako pa je z izkoristkom amplitudnega holograma, kjer je modulirana
absorpcija v snovi, lahko bralec izračuna sam.\footnote{Glej H. Kogelnik, Bell Syst. Tech. J.
48, 2909 (1969).}
\end{remark}

Oglejmo si še tretji primer. Izhajamo iz sistema enačb~(\ref{7.37}), ki smo ga 
zaenkrat rešili za primer Braggovega odboja oziroma v njegovi bližini. 
Enačbe je preprosto rešiti še v primeru Raman-Nathovega približka. 
\index{Raman-Nathov uklon}Vpeljimo novo neodvisno spremenljivko $\zeta=2\xi z$. 
Zveza~(\ref{7.37})
preide v 
\begin{equation}
2\frac{dA_{m}(\zeta)}{d\zeta}+A_{m+1}(\zeta)-A_{m-1}(\zeta)=\frac{\beta_{m}}{i\xi}A_{m}.
\label{7.53}
\end{equation}
 Člen na desni lahko izpustimo, če je 
\begin{equation}
\frac{\beta_{m}}{\xi}=\left| \frac{4mq}{\tilde{\varepsilon}pS_0k}(\sin\vartheta+\frac{mq}{2k})\right| 
\ll 1,
\label{7.54}
\end{equation}
oziroma če je valovna dolžina zvoka dovolj velika v primerjavi z valovno dolžino svetlobe. Potem 
v enačbi~(\ref{7.53}) prepoznamo rekurzijsko zvezo za Besselove funkcije 
\begin{equation}
2J_{n}'+J_{n+1}-J_{n-1}=0
\label{7.55}
\end{equation}
z rešitvijo $A_{m}(z)=A_{0}J_{m}(2\xi z)$. Kadar je $2\xi d$ ničla funkcije
J$_{0}$, prvič je to pri $2\xi d\approx 2.4$, se vsa energija ukloni iz
vpadnega snopa, vendar se v tem primeru, ko Braggov pogoj ni izpolnjen,
razporedi v mnogo uklonjenih snopov.

\section{Modulacija s tekočimi kristali}

\subsection*{Nematični tekoči kristali}
\index{Tekoči kristali}
\index{Nematik}
Za konec opišimo še modulacijo svetlobe s tekočimi kristali. 
Tekoči kristali so anizotropne kapljevine. To pomeni, da so tekoči kot 
kapljevine, imajo pa določene anizotropne lastnosti kot trdni kristali. 
Tekoče kristale tvorijo podolgovate ali ploščate molekule, 
ki odražajo različne stopnje urejenosti. 

Omejimo se najosnovnejši
primer, to so podolgovate organske molekule v nematični fazi tekočega kristala. 
Navadno so to molekule z relativno togim jedrom iz
dveh ali treh benzenovih obročev, ki imajo na koncih krajše ali daljše
alifatske verige (slika~\ref{fig:5CB}). Značilnost nematične faze je, da
so v njej težišča molekul neurejena, enako kot v navadni tekočini, 
osi molekul pa so v povprečju urejene v določeno smer. Pravimo, da imajo molekule
v nematiku orientacijsko ureditev dolgega dosega. Če nematik segrejemo,
preide v izotropno tekočo fazo, če pa ga ohladimo, neposredno ali prek drugih
tekočekristalnih faz preide v trdno kristalno obliko. 

Smer povprečne urejenosti podolgovatih molekul opišemo z enotskim vektorjem 
$\mathbf{n}$, ki ga imenujemo direktor. Smeri $\mathbf{n}$ in $-\mathbf{n}$ sta 
enakovredni, saj molekule z enako verjetnostjo kažejo v smer $+\mathbf{n}$ kot 
v $-\mathbf{n}$. Stopnja urejenosti v mikroskopski sliki ni prav velika, povprečen
odklon molekul od $\mathbf{n}$ je nekaj deset stopinj, odvisno seveda od temperature.
\begin{figure}[h]
\centering
\def\svgwidth{30truemm} 
\input{slike/09_5CB.pdf_tex}\qquad\qquad
\def\svgwidth{50truemm} 
\input{slike/09_nematik.pdf_tex}
\caption{Molekula enega najbolj razširjenih tekočih kristalov, 4-ciano-4'pentil-bifenila 
ali 5CB (levo) in shematski prikaz nematične faze (desno)}
\label{fig:5CB}
\end{figure}

Molekule so v nematični fazi v povprečju orientacijsko urejene, zato se nematik
obnaša kot enoosen dvolomni kristal\index{Dvolomnost}. Njegova optična os je vzporedna 
z $\mathbf{n}$, lastni vrednosti dielektričnega tenzorja pa sta $\varepsilon_\bot$ in
$\varepsilon_{\parallel}$, ki ustrezata rednemu ($n_o$) in izrednemu ($n_e$) 
lomnemu količniku.  
Ker je optična polarizabilnost benzenovih obročev vzdolž osi molekul precej večja kot
v prečni smeri, je razlika med rednim in izrednim lomnim količnikom v nematiku razmeroma 
velika, navadno med 0,1 in 0,2, seveda spet odvisno od temperature.

V povprečju so molekule urejene v smeri direktorja. Če se smer direktorja lokalno
spremeni, je energija takega deformiranega 
stanja nekoliko večja od energije homogenega urejenega stanja. Tekoči kristal na
drugače orientiran delček snovi zato deluje z navorom v smeri zmanjševanja 
nehomogenosti $\mathbf{n}$. To lastnost, ki je značilna za tekoče kristale,
imenujemo orientacijska elastičnost. Vendar so v makroskopskem vzorcu
nematičnega tekočega kristala elastični navori prešibki,
da bi uredili celoten vzorec, zato se v splošnem smer direktorja $\mathbf{n}$ 
po vzorcu neurejeno spreminja. V optičnih napravah pa potrebujemo urejene vzorce, zato
moramo ureditev vzorca vsiliti. To naredimo z zunanjim električnim ali magnetnim poljem, 
ali pa vzorce pripravimo dovolj tanke, da ureditev vsilijo mejne površine. 

Poglejmo najprej, kako nastane urejen vzorec v tankih plasteh. Če površino,
ki je v stiku s tekočim kristalom, ustrezno pripravimo (prevlečemo s posebnimi 
plastmi ali mehansko obdelamo), se molekule tekočega kristala tik ob površini uredijo
v dani smeri (slika~\ref{s7.20a}).
Tako na primer podrgnjena tanka plast najlona uredi $\mathbf{n}$ ob površini v smeri
drgnjenja vzporedno s površino. Po drugi strani pa tanka plast lecitina ali 
surfaktanta silana uredi direktor pravokotno na površino. Ti dve snovi imata namreč
polarno glavo, ki se adsorbira na stekleno površino, in alifatsko verigo, 
ki stoji približno pravokotno na površino. Zato se tudi alifatski repi molekul
tekočega kristala uredijo pravokotno na steklo. V obeh primerih, 
vzporedni (planarni) ali pravokotni (homeotropni) ureditvi ob steni, 
se urejenost zaradi orientacijske elastičnosti
ohranja tudi stran od stene, tako da lahko brez težav naredimo urejene
vzorce debeline do kakih 200~$\mu$m. Pri večjih debelinah so elastični
navori prešibki in v vzorcu nastanejo defekti.
\begin{figure}[h]
\centering
\def\svgwidth{50truemm} 
\input{slike/09_planarno.pdf_tex}\qquad
\def\svgwidth{54truemm} 
\input{slike/09_homeo.pdf_tex}
\caption{Ureditev tekočega kristala navadno vsilimo z urejevalno površino. Dva primera
sta planarna ureditev (levo), kjer je direktor vzporeden z urejevalno površino, in 
homeotropna ureditev (desno), kjer je direktor pravokoten mejno ploskev.}
\label{s7.20a}
\end{figure}
 
Na ureditev molekul tekočega kristala vpliva zunanje električno ali magnetno polje.
Zaradi urejenosti molekul električna (in magnetna) susceptibilnost nematičnega tekočega
kristala ni skalar, temveč ima dve različni lastni vrednosti,
eno v za smer vzporedno z $\mathbf{n}$, drugo za pravokotno nanj. Zato je
elektrostatična energija odvisna od kota med zunanjim poljem $\mathbf{E}$
in direktorjem $\mathbf{n}$. Gostoto električne energije\index{Gostota energije} 
zapišemo kot 
\begin{equation}
w_{el}=-\frac{1}{2}\mathbf{E}\cdot \mathbf{D}.
\label{lcwe}
\end{equation}
Električno polje lahko razstavimo na del, ki je vzporeden z $\mathbf{n}$, in del, ki je
pravokoten nanj
\beq
\mathbf{E} = (\mathbf{E} \cdot \mathbf{n}) \mathbf{n} + \left( \mathbf{E} - 
(\mathbf{E} \cdot \mathbf{n}) \mathbf{n} \right).
\eeq
Potem je 
\beq
\mathbf{D} = \varepsilon \varepsilon_\bot \mathbf{E} + \varepsilon_0 \varepsilon_a
(\mathbf{E}\cdot\mathbf{n})\mathbf{n},
\eeq
pri čemer je $\varepsilon_a = \varepsilon_\parallel - \varepsilon_\bot$ anizotropni
del dielektrične konstante. Anizotropni del energije je tako do konstante
\boxeq{7.56}{
w_a = -\frac{1}{2}\epsilon_{0}\epsilon_{a}(\mathbf{E}\cdot\mathbf{n})^{2},
}
Če je $\epsilon_{a}>0$, se molekule tekočega kristala uredijo v smeri 
zunanjega polja, v nasprotnem primeru pa pravokotno nanj.

Struktura tekočekristalnega vzorca je tako odvisna od orientacijske
elastičnosti, robnih pogojev, ki jih določimo z obdelavo mejne
površine, in od jakosti ter smeri zunanjega električnega ali magnetnega polja.

\subsection*{Tekočekristalni prikazovalnik}
\index{Tekočekristalni prikazovalnik}
Vzemimo tanko plast tekočega kristala med dvema površinama, ki vsiljujeta
vzporedno planarno ureditev. Vzorec je urejen in homogen, optična os leži v ravnini 
plasti. Če dodamo na površini še prozorni elektrodi, lahko
z zunanjo napetostjo spreminjamo orientacijo molekul v plasti in tako tudi 
smer optične osi. Dovolj velika napetost zasuče $\mathbf{n}$ in optična os
se postavi pravokotno na stene, razen tik ob površini. Pri debelini okoli 10~$\mu$m
je potrebna napetost nekaj voltov.

Ta pojav lahko izkoristimo za izdelavo preprostega optičnega preklopnika. 
Naj debelina plasti $d$ ustreza debelini ploščice $\lambda/2$\index{Ploščica $\lambda/2$} 
za izbrano valovno dolžino svetlobe 
\begin{equation}
d(n_{e}-n_{o})=(2N+1)\frac{\lambda}{2},
\label{7.57}
\end{equation}
kjer je $N$ celo število, $n_e$ izredni in $n_o$ redni 
lomni količnik. Ker je v nematikih $n_{e}-n_{o}\sim0,1$, 
je ustrezna debelina $d$ nekaj $\mu$m. Tak vzorec damo med dva prekrižana 
polarizatorja s prepustno smerjo pod kotom 45$^\circ$ glede na $\mathbf{n}$
oziroma optično os. Vzorec, ki deluje kot ploščica $\lambda/2$, 
polarizacijo svetlobe z izbrano valovno dolžino zasuče za 90$^\circ$ in 
svetloba prehaja skozi analizator. Ko
priključimo napetost, se optična os obrne v smeri polja. Polarizacija vpadne 
svetlobe se pri prehodu skozi plast ohrani in 
analizator je ne prepusti. Z električnim poljem smo torej preklopili iz
stanja, ki prepušča svetlobo, v stanje, ki svetlobe ne prepusti.
Vendar ima tak preklopnik nekaj slabosti. Prepustnost je odvisna
od valovne dolžine svetlobe in od temperature, poleg tega mora biti debelina 
plasti povsod povsem enaka. Zato se v praksi uporablja zasukan nematik. 

Zasukan  nematik nastane tako, da površini, ki vsiljujeta planarno ureditev,
zasučemo za kot 90$^\circ$ eno glede na drugo (slika~\ref{LCD1}\,a), zato se $\mathbf{n}$ 
v plasti zvezno zavrti.
Pokazali bomo, da polarizacija svetlobe, ki je ob vstopu 
v plast polarizirana v smeri urejanja, pri prehodu skozi plast približno
sledi $\mathbf{n}$ in je ob izstopu iz plasti pravokotna
na vpadno polarizacijo. Ko priključimo električno polje, se optična os 
obrne v smer pravokotno na plast tekočega kristala (slika~\ref{LCD1}\,b). V tem primeru 
se polarizacija ne zasuče in analizator svetlobe ne prepusti. Plast med prekrižanima
polarizatorjema brez polja torej prepušča svetlobo, s poljem pa ne. Pri tem
delovanje prikazovalnika ni dosti odvisno niti od debeline plasti niti
od valovne dolžine. 
\begin{figure}[h!]
\centering
\def\svgwidth{105truemm} 
\input{slike/09_LCD1.pdf_tex}
\caption{a) Zasukana nematična celica. 
Polarizacija (P) približno sledi smeri zasukanega direktorja in analizator (A) 
prepusti svetlobo. b) Ko priključimo električno polje, se tekočekristalne molekule
zasučejo v smer polja. Polarizacija svetlobe (P) se ohranja in analizator (A) 
je ne prepusti. }
\label{LCD1}
\end{figure}

Slabost uporabe tekočekristalnih preklopnikov je njihova diskretnost. Pri 
kristalnem elektro-optičnem modulatorju je mogoče doseči tudi vmesne
prepustnosti, medtem ko s tekočimi kristali na opisan način
dosežemo le zaprto in odprto stanje.

\begin{remark}
Tekočekristalni zasloni, ki jih uporabljamo v praksi, so precej bolj zapleteni.
Najpreprostejši so črno-beli prikazovalniki, ki delujejo z odbito svetlobo (npr. 
v žepnih računalih), zato imajo za analizatorjem odbojno površino. Večina 
sodobnih prikazovalnikov (npr. računalniški ali telefonski zasloni) pa za 
osvetlitev uporablja LED ali fluorescenčna svetila. 
Barve dosežemo z barvnimi filtri (rdečim, modrim in zelenim) na vsakem 
pikslu posebej, natančno krmiljenje pikslov pa s tankoplastnimi 
tranzistorji ({\it Thin film transistors}, TFT). Tekočekristalne zaslone lahko 
z dodatnimi plastmi naredimo tudi občutljive na dotik.
\begin{figure}[h!]
\centering
\includegraphics[width=63truemm]{slike/LCD.jpg}
\caption{ZAMENJAJ Z LASTNO SLIKO!}
\end{figure}
\end{remark}

Pokazati moramo še, da polarizacija svetlobe približno sledi zasuku optične osi. 
Sredstvo naj bo lokalno enoosno, optična os naj se suče v smeri pravokotno
glede na smer širjenja svetlobe (glej sliko~\ref{LCD1}\,a). 
Poleg zasukane nematične celice je pomemben primer
snovi s takimi lastnostmi holesteričen tekoči kristal, ki je zelo
podoben nematičnim , le da se $\mathbf{n}$ spontano suče okoli
smeri, pravokotne na $\mathbf{n}$.

Optična os sredstva naj leži v ravnini $xy$ in naj se enakomerno
suče, ko se premikamo vzdolž osi $z$. Kot med optično osjo in osjo
$x$ lahko zapišemo 
\begin{equation}
\varphi=qz.
\label{7.58}
\end{equation}
Zanimajmo se le za širjenje svetlobe v smeri $z$. 
Tedaj potrebujemo le del dielektričnega
tenzorja v $xy$ ravnini. 

\begin{definition}
Pokaži, da se dielektrični tenzor v zasukani nematični plasti zapiše kot
\begin{equation}
\varepsilon (z)=\left[\begin{array}{cc}
\bar{\varepsilon}+\frac{1}{2}\varepsilon_{a}\cos(2qz) & \frac{1}{2}\varepsilon_{a}\sin(2qz)\\
\frac{1}{2}\varepsilon_{a}\sin(2qz) & \bar{\varepsilon}-\frac{1}{2}\varepsilon_{a}\cos(2qz)
\end{array}\right],
\label{7.59}
\end{equation}
kjer  je $z$ razdalja od plasti, v kateri je direktor
obrnjen v smeri $x$, povprečna vrednost $\bar{\varepsilon}$ pa  
\begin{equation}
\bar{\varepsilon}=\frac{\varepsilon_{\parallel}+\varepsilon_{\perp}}{2}.
\label{7.60}
\end{equation}
\end{definition}

Iz Maxwellovih enačb~(enačbe~\ref{eq:Maxwell1}--\ref{eq:Maxwell4}) 
hitro uvidimo, da je valovna enačba
za valovanje s frekvenco $\omega$ oblike
\begin{equation}
\frac{d^{2}\mathbf{E}}{dz^{2}}+\frac{\omega^{2}}{c^{2}} \epsilon
(z)\mathbf{E}=0
\label{7.61}
\end{equation}
ali v komponentah, upoštevajoč tenzor dielektričnosti~(\ref{7.59})
\beq
\frac{d^{2}E_{x}}{dz^{2}} + 
(\beta^{2}+\alpha^{2}\cos(2qz))E_{x}+\alpha^{2}E_{y}\sin(2qz) = 0
\label{7.62a}
\eeq
in 
\beq
\frac{d^{2}E_{y}}{dz^{2}} +
\alpha^{2}E_{x}\sin(2qz)+(\beta^{2}-\alpha^{2}\cos(2qz))E_{y} = 0,
\label{7.62b}
\eeq
kjer je $\alpha^{2}=\epsilon_{a}\omega^{2}/(2c^{2})$ in 
$\beta^{2}=\bar{\epsilon}\omega^{2}/c^{2}$. Dobili smo torej sistem
dveh sklopljenih diferencialnih enačb. 

Za reševanje je ugodno vpeljati krožni polarizaciji 
$E_{+}=E_{x}+iE_{y}$ in $E_{-}=E_{x}-iE_{y}$.
Enačbi~(\ref{7.62a}) in (\ref{7.62b}) potem preieta v 
\beq
-\frac{d^{2}E_{+}}{dz^{2}}=\beta^{2}E_{+}+\alpha^{2}E_{-}e^{2iqz}
\label{lcm1}
\eeq
in 
\beq
-\frac{d^{2}E_{-}}{dz^{2}}=\alpha^{2}E_{+}e^{-2iqz}+\beta^{2}E_{-}.
\label{lcm2}
\eeq

\begin{remark}
Iščemo torej lastne rešitve valovne enačbe v sredstvu s periodično modulacijo
lomnega količnika. Matematično podoben problem je iskanje lastnih
funkcij elektronov v kristalu. Za te vemo, da morajo 
biti morajo produkt periodične funkcije s periodo kristalne
mreže in faktorja $\exp(ikz)$ oziroma $\exp(-ikz)$. Matematiki pravijo tej trditvi Floquetov
izrek\footnote{Francoski matematik Achille Marie Gaston Floquet, 1847--1920.}. 
\end{remark}

Lastne rešitve poiščimo v obliki 
\begin{eqnarray}
E_{+} & = & Ae^{i(k+q)z} \label{7.65a} \qquad \mathrm{in}\\
E_{-} & = & Be^{i(k-q)z}.
\label{7.65}
\end{eqnarray}
Nastavek reši sistem enačb~(\ref{lcm1}) in (\ref{lcm2}), 
če $A$ in $B$ rešita sistem homogenih linearnih enačb 
\beq
[(k+q)^{2}-\beta^{2}]A-\alpha^{2}B  =  0 
\eeq
in
\beq
-\alpha^{2}A+[(k-q)^{2}-\beta^{2}]B  =  0.
\eeq
 Sistem je netrivialno rešljiv, če je determinanta koeficientov enaka
nič
\begin{equation}
(k^{2}+q^{2}-\beta^{2})^{2}-4k^{2}q^{2}-\alpha^{4}=0.
\label{7.66}
\end{equation}
Spomnimo se, da sta $\beta$ in $\alpha$ sorazmerna z $\omega$,
zato dobljena enačba predstavlja disperzijsko relacijo -- zvezo med 
$\omega$ in $k$ -- za svetlobo v zavitem sredstvu. 

V splošnem je iskanje rešitev gornje enačbe zapleten problem, vendar 
za razlago delovanja zasukane nematične celice zadošča približek 
$q\ll\beta$ in $\alpha$, ko je torej perioda sukanja optične osi
velika v primeri z valovno dolžino svetlobe. Tedaj lahko $q$ v disperzijski
zvezi (enačba~\ref{7.66}) zanemarimo in dobimo
\begin{equation}
k^{2}=\left\{ \begin{matrix}\beta^{2}+\alpha^{2}=\frac{\omega^{2}}{c^{2}}\epsilon_{\parallel} 
\\               \beta^{2}-\alpha^{2}=\frac{\omega^{2}}{c^{2}}\epsilon_{\bot} 
              \end{matrix}\right.
\label{7.67}
\end{equation}
 Ti vrednosti ustrezata velikosti valovnega vektorja za izredni
in redni val v navadnem enoosnem kristalu. Vstavimo ju v ~(\ref{7.65a}) ali
(\ref{7.65}) in za polarizaciji lastnih valov dobimo $B=\pm A$.

Izračunajmo obe kartezični komponenti električnega polja za prvo rešitev:
\begin{align}
E_{x} &=  \frac{1}{2}(E_{+}+E_{-})  =  \frac{1}{2}Ae^{ikz}(e^{iqz}+e^{-iqz})  =  Ae^{ikz}\cos qz\\
E_{y} & = \frac{1}{2i}(E_{+}-E_{-})  =  \frac{1}{2i}Ae^{ikz}(e^{iqz}-e^{-iqz})  =  Ae^{ikz}\sin qz.
\label{7.68}
\end{align}
Polarizacija torej res sledi optični osi. Druga rešitev da val,
ki je polariziran pravokotno na lokalno optično os in se prav tako
suče z njo. Pri tem se prvi val širi s fazno hitrostjo $c/n_{e}$, torej kot
izredni val, drugi pa s $c/n_{o}$, to je kot redni val. Če na zasukano
nematično celico vpada svetloba, ki je polarizirana ali paralelno ali pravokotno
na optično os ob meji, se pojavi na izhodni strani polarizacija, zasukana
za pravi kot. V primeru, da vpadna polarizacija ne sovpada z eno od
lastnih osi, jo moramo razstaviti na obe lastni in po prehodu skozi
tekoči kristal zopet sestaviti, s čemer seveda v splošnem nastane eliptična
polarizacija.

\begin{remark}
Disperzijsko zvezo (enačba~\ref{7.66}) lahko tudi numerično rešimo (slika~\ref{gap}). 
Spomnimo se, da je parameter $\beta$ sorazmeren frekvenci $\omega$. Vidimo, da pri 
vsaki frekvenci, razen v ozkem območju med $\omega_{-}$
in $\omega_{+}$, recimo mu frekvenčna reža, obstajajo štiri realne
rešitve za $k$, po dve za valovanji v pozitivni in v negativni smeri.
V območju reže je en par rešitev imaginaren. Vsaki vrednosti $k$
pripada neko razmerje amplitud $A$ in $B$, ki ga izračunamo
iz enačb~(\ref{7.65a}) in (\ref{7.65}) in ki določa polarizacijo lastnega vala. Polarizacije
lastnih valov so v splošnem eliptične in pri dani frekvenci med
seboj niso pravokotne, saj zapisani sistem enačb ne predstavlja čisto 
navadnega problema lastnih vektorjev simetrične matrike. 
V območju frekvenčne reže le en par rešitev predstavlja
potujoč val, drug pa polje, ki eksponentno pojema v sredstvo. Zato
se svetloba s frekvenco v reži in z ustrezno polarizacijo, ki vpada
na holesteričen tekoči kristal, totalno odbije. Pojav je povsem analogen
Braggovemu odboju na kristalih in daje holesterikom značilen obarvan
videz.
\begin{figure}[h]
\centering
\def\svgwidth{70truemm} 
\input{slike/09_gap.pdf_tex}
\caption{Rešitve disperzijske zveze (enačba~\ref{7.66}) v zasukanem nematiku. Razen na ozkem 
frekvenčnem območju ima štiri rešitve za vsak $\beta$ oziroma za vsako frekvenco, saj je 
$\beta \propto \omega$. }
\label{gap}
\end{figure}
\end{remark}

\section{*Izračun preklopa v tekočem kristalu}

V prejšnjem poglavju smo omenili, da lahko z dovolj velikim zunanjim poljem 
molekule tekočega kristala, razen tik ob površini, obrnemo v smeri polja. 
Izračunajmo jakost polja, ki je potrebna za ta zasuk. 

Energija nematičnega tekočega kristala je najnižja, kadar je direktor $\mathbf{n}$
povsod obrnjen v isto smer. Povečanje energije zaradi krajevne odvisnosti $\mathbf{n}$
v splošnem zapišemo z orientacijsko elastično energijo oziroma Frankovo prosto 
energijo\footnote{Angleški fizik Sir Frederick Charles Frank, 1911--1998.}
\boxeq{7.70}{
F_{e}=\frac{1}{2}\int\left\{ K_{1}(\nabla\cdot\mathbf{n})^{2}+K_{2}
[\mathbf{n}\cdot(\nabla\times\mathbf{n})]^{2}+K_{3}
[\mathbf{n}\times(\nabla\times\mathbf{n})]^{2}\right\} dV.
}
Pri tem so $K_{1}$, $K_{2}$ in $K_{3}$ so tri Frankove elastične
konstante. Prvi člen predstavlja povečanje energije zaradi deformacije v obliki 
pahljače, drugi zaradi zasuka, tretji pa zaradi upogiba (slika~\ref{s7.20}).
\begin{figure}[h]
\centering
\def\svgwidth{140truemm} 
\input{slike/09_KKK.pdf_tex}
\caption{Trije načini deformacije ureditve tekočega kristala: pahljačasta deformacija,
zasuk in upogib}
\label{s7.20}
\end{figure}

V zunanjem električnem polju se energija tekočega kristala dodatno spremeni. 
Navadno je neodvisna električna količina električna poljska jakost, saj je polje posledica
zunanje napetosti na elektrodah. Ustrezni člen v termodinamičnem
potencialu je tedaj (enačbi~\ref{lcwe} in~\ref{7.56})
\beq
w_{el} = -\frac{1}{2} \mathbf{D}\cdot\mathbf{E} = -\frac{1}{2} 
\left( \varepsilon_0 \varepsilon_\bot \mathbf{E}\cdot\mathbf{E} + 
\varepsilon_{0}\varepsilon_{a}(\mathbf{E}\cdot\mathbf{n})^{2}\right).
\eeq
Prvi člen je neodvisen od $\mathbf{n}$, zato ni pomemben. Prosta energija
nematičnega tekočega kristala v električnem polju je tako 
\begin{equation}
F=F_{0}+F_{e}-\frac{1}{2}\varepsilon_{0}\varepsilon_{a}
(\mathbf{E}\cdot \mathbf{n})^{2},
\label{7.72}
\end{equation}
 kjer $F_{0}$ predstavlja del proste energije, ki je neodvisen od $\mathbf{n}$
in $\mathbf{E}$. V ravnovesju je prosta energija najmanjša. Kadar je
$\epsilon_{a}>0$, se zato skuša $\mathbf{n}$ postaviti vzporedno s
poljem. Da lahko z minimizacijo $F$ izrazimo $\mathbf{n}(\mathbf{r})$, moramo
poznati še robne pogoje.

Poglejmo primer. Naj bo nematični tekoči kristal med dvema vzporednima
steklenima ploščama v razmiku $d$. Na obeh ploščah naj bo $\mathbf{n}$ vzporeden
s površino in vsiljeni smeri naj bosta enaki, tako da je brez zunanjega 
polja $\mathbf{n}$ povsod enako usmerjen. Naj bo to smer $x$.
Na stekleni plošči dodamo elektrodi, ki ustvarjata polje pravokotno na 
prvotno smer direktorja, naj bo to smer $z$.
Ko priključimo polje, je energijsko ugodnejše, če
se molekule vsaj delno zasučejo v smeri polja. Ta zasuk opišemo s
komponento vektorja $\mathbf{n}$ v smeri $z$
\begin{equation}
\mathbf{n}(z)=(n_{x}(z),0,n_{z}(z)).
\label{7.73}
\end{equation}
Robna pogoja, katerima mora direktor zadostiti,
sta $n_{z}(0)=n_{z}(d)=0$. Približno rešitev zato iščemo z nastavkom 
\begin{equation}
n_{z}(z)=a\sin (qz), \qquad q=\frac{\pi}{d},
\label{7.74}
\end{equation}
ki ni nič drugega kot prvi člen razvoja prave rešitve v Fourierevo vrsto.
Ker je direktor enotski vektor, potem velja
\beq
n_x = \sqrt{1-a^2\sin^2(qz)} \approx 1 - \frac{a^2}{2}\sin^2(qz)
\eeq
Vzdolž smeri $x$ in $y$ se direktor ne spreminja, zato velja
\begin{equation}
\nabla\times\mathbf{n}=(0,-\frac{dn_{x}}{dz},0)
\label{7.75}
\end{equation}
 in 
\begin{equation}
\mathbf{n}\times(\nabla\times\mathbf{n})=(n_{z}\frac{dn_{x}}{dz},0,
-n_{x}\frac{dn_{x}}{dz}).
\label{7.76}
\end{equation}
Površinska gostota proste energije je tako 
\begin{eqnarray}
F & = & \frac{1}{2}\int\left[K_{1}\left(\frac{dn_{z}}{dz}\right)^{2}+K_{3}(n_x^2+n_{z}^{2})
\left(\frac{dn_{x}}{dz}\right)^{2}-
\epsilon_{0}\epsilon_{a}(n_{z}E)^{2}\right]dz=\nonumber \\
 & = & \frac{1}{2}\int_{0}^{d}
 [K_{1}q^{2}a^{2}\cos^{2}(qz)+K_{3}q^{2}a^{4}\sin^{2}(qz)\cos^2(qz)-
 \epsilon_{0}\epsilon_{a}E^2a^{2}\sin^{2}(qz)]dz=\nonumber \\
 & = & \frac{\pi}{4q}a^2\left( K_{1}q^{2}+\frac{1}{4}K_{3}q^{2}a^2-\epsilon_{0}\epsilon_{a}E^2\right).
\end{eqnarray}
V našem primeru so integral lahko izračunali, saj smo vstavili nastavek (enačba~\ref{7.74}).
Sicer bi morali uporabiti Euler-Lagrangevo metodo za minimizacijo proste energije, ki 
jo poznamo iz variacijskega računa.

Poiščimo amplitudo deformacije $a$, pri kateri je prosta energija
najmanjša. Tedaj mora biti $a$ rešitev enačbe 
\begin{equation}
2(K_{1}q^{2}-\epsilon_{0}\epsilon_{a}E)a+K_{3}q^{2}a^{3}=0.
\label{7.78}
\end{equation}
 Rešitvi sta 
\begin{equation}
a=0 \qquad \mathrm{in} \qquad a^{2}=2\frac{\epsilon_{0}\epsilon_{a}E^2-K_{1}q^{2}}{K_{3}q^{2}}.
\label{7.79}
\end{equation}
 Pri majhnih poljih, ko je $\epsilon_{0}\epsilon_{a}E<K_{1}q^{2}$,
je fizikalno smiselna le prva rešitev, torej brez deformacije, pri velikih poljih pa je  
stabilna druga rešitev. Ko večamo polje, deformacija
v sredini plasti hitro naraste, tako da se $\mathbf{n}$ postavi skoraj
popolnoma v smer zunanjega polja. Tedaj naša rešitev seveda ni dobra,
saj smo pri računu privzeli, da je $n_{z}\ll1$. Prehodu iz nedeformiranega
stanja v deformirano stanje pravimo tudi Frederiksov prehod\footnote{Ruski fizik
Vsevolod Konstantinovič Frederiks, tudi Fr\'{e}edericksz, 1885--1944.}. Na njem
temelji preklapljanje optičnih prikazovalnikov na nematične tekoče kristale.

Izračunajmo še kritično jakost električnega polja, pri kateri pride do prehoda v deformirano fazo.
To se zgodi pri 
\beq
\epsilon_{0}\epsilon_{a}E_c^2-K_{1}q^{2} = 0
\eeq
oziroma
\boxeq{FreeE}{
E_c = \frac{\pi}{d}\sqrt{\frac{K_1}{\varepsilon_0\varepsilon_a}}.
}

Poglejmo še, kako narašča amplituda deformacije v bližini prehoda. Iz enačbe~(\ref{7.79})
sledi 
\beq
a = \sqrt{\frac{2 \varepsilon_0 \varepsilon_a}{K_3 q^2 }(E^2-E_c^2)}.
\eeq
Pogosto naredimo približek enakih konstant, kjer privzamemo, da so vse Frankove 
elastične konstante enake vrednosti. V tem približku je 
\beq
a \approx \sqrt{\frac{2(E^2-E_c^2)}{E_c^2}}
\eeq
in torej korensko narašča s naraščajočim poljem (slika~\ref{Fred}). Tak prehod je torej
fazni prehod drugega reda, saj količina, ki opisuje prehod (amplituda deformacije $a$)
zvezno preide iz vrednosti $a=0$ v končno vrednost. 
\begin{figure}[h]
\centering
\def\svgwidth{80truemm} 
\input{slike/09_Fred.pdf_tex}
\caption{Kvalitativno obnašanje amplitude deformacije ob Frederiksovem prehodu}
\label{Fred}
\end{figure}

\begin{definition}
Izračunaj Frederiksov prehod v zasukani nematični celici (kot zasuka med zgornjo in spodnjo 
mejno ploskvijo naj bo $\pi/2$) in pokaži, da je kritično polje za prehod enako
\beq
E_c =  \frac{\pi}{d}\sqrt{\frac{K_1}{\varepsilon_0\varepsilon_a}}
\sqrt{1 + \frac{K_3-2K_2}{4K_1}}.
\eeq
Namig: uporabi nastavek $\varphi = z \pi/2d$ in $\vartheta = a \sin(\pi z/d)$. 
\end{definition}

