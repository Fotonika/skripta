\chapterimage{Mavrica.jpg} % Chapter heading image

\chapter{Modulacija svetlobe}

V optičnih napravah pogosto želimo spreminjati lastnosti svetlobnega
valovanja. En tak primer smo že spoznali pri obravnavi laserja,
ko smo za preklop kvalitete potrebovali element, ki hitro spreminja prepustnost. 
Še bolj pomembna je modulacija valovanja pri optičnem prenosu informacij.

Svetlobno valovanje lahko moduliramo na več načinov. Z ustreznim moduliranjem
lomnega količnika lahko valovanju spreminjamo amplitudo ali frekvenco oziroma fazo. 
\begin{figure}[h]
\centering
\def\svgwidth{140truemm} 
\input{slike/09_AMFM.pdf_tex}
\caption{Amplitudno moduliran signal (levo) in fazno oziroma frekvenčno moduliran signal (desno)
}
\label{fig:amfm}
\end{figure}

Optični modulatorji za delovanje izkoriščajo različne pojave. V tem poglavju bomo spoznali dva
najpomembnejša, to sta elektro-optični in elasto-optični pojav. Pri prvem 
dosežemo spremembo lomnega količnika snovi z električnim poljem, pri drugem pa z 
deformacijo, ki se navadno pojavi v zvočnem valu, zato takim modulatorjem pravimo tudi
akusto-optični. Na koncu bomo spoznali poseben zelo pomemben primer elektro-optičnih modulatorjev
na osnovi tekočih kristalov.

\section{Elektro-optični pojav}
Če na snov priključimo električno polje (napetost), se njen dielektrični tenzor lahko spremeni.
Omejimo se na statično zunanje polje oziroma na polje, katerega frekvenca 
je bistveno manjša od optične frekvence. Omejitev na nizko frekvenco je potrebna zato, 
da optično polje še lahko obravnavamo linearno. Kako je v nasprotnem primeru, ko je frekvenca
polja primerljiva z optično frekvenco, smo na široko obravnavali v prejšnjem poglavju o
nelinearni optiki (poglavje~\ref{chap:NLO}).

Iz zgodovinskih razlogov namesto spremembe dielektričnega tenzora pišemo 
spremembo dielektričnega inverznega tenzorja 
\beq
\underline{b}=\underline{\epsilon}^{-1}.
\eeq
Ker so spremembe razmeroma majhne, lahko spremembo komponente $b_{ij}$ zapišemo 
kot potenčno vrsto zunanjega polja $E$
\begin{equation}
\delta b_{ij}=r_{ijk}E_{k}+q_{ijkl}E_{k}E_{l}.
\label{7.1}
\end{equation}
Prvi člen, linearen z ozirom na zunanje polje, opisuje linearni elektro-optični
ali Pockelsov pojav\footnote{Nemški fizik Friedrich Carl Alwin Pockels, 1865--1913.}. 
Tenzor tretjega ranga $r_{ijk}$, ki lastnost snovi, imenujemo tako elektro-optični tenzor 
ali tudi Pockelsov tenzor. Kvadratnemu elektro-optičnemu pojavu pravimo Kerrov 
pojav\footnote{Škotski fizik John Kerr, 1824--1907}. Pockelsov tenzor je 
različen od nič v snoveh brez centra inverzije, Kerrov pa je šibkejši, a v vseh snoveh
različen od nič. Značilne vrednosti Pockelsovega
tenzorja so okoli $r \sim 10^{-12} - 10^{-10}$~m/V, Kerrovega pa 
$q \sim 10^{-24}$~m$^2$/V$^2$.

Za uporabo trdnih kristalov je pomemben
predvsem linearni člen, zato se bomo osredotočili le na prvi člen in zapisali
\boxeq{eq:Pockels}{
\delta b_{ij}=r_{ijk}E_{k}. 
}

Preden si pobliže ogledamo Pockelsov tenzor, zapišimo še,
kako se z $\delta b_{ij}$ izrazijo spremembe komponent dielektričnega
tenzorja. Vzemimo koordinatni sistem, v katerem je nemoten dielektrični
tenzor $\underline{\tilde{\varepsilon}}$ diagonalen. Ker so spremembe majhne, velja 
\begin{equation}
\underline{\varepsilon} = \underline{\tilde{\varepsilon}} + \delta \underline{\varepsilon}=
(\underline{b}+\delta \underline{b})^{-1}=\left(\underline{b}(1+\underline{b}^{-1}
\delta \underline{b})\right)^{-1}=(1+\underline{b}^{-1}\delta \underline{b})^{-1}\underline{b}^{-1}
\approx \underline{b}^{-1}-\underline{b}^{-1}\delta \underline{b}\, \underline{b}^{-1}.
\label{7.2}
\end{equation}
Sprememba dielektričnega tenzorja je tako
\beq
 \delta \underline{\varepsilon}= -\underline{b}^{-1}\delta \underline{b}\, \underline{b}^{-1}
 = -\underline{\tilde{\varepsilon}}\, \delta \underline{b}\, \underline{\tilde{\varepsilon}}
\eeq
oziroma
\begin{equation}
\delta\epsilon_{ij}=-\tilde{\epsilon}_{ik}\delta b_{kl}\tilde{\epsilon}_{lj}
=-\tilde{\epsilon}_{ii}\tilde{\epsilon}_{jj}\delta b_{ij}.
\label{7.3}
\end{equation}
V zadnjem zapisu smo upoštevali, da je nemoten dielektrični tenzor diagonalen.

\subsection*{Pockelsov tenzor}
Simetrija pomembno vpliva na obliko tenzorjev, ki opisujejo lastnosti
snovi. Pockelsov tenzor $r$ je tenzor tretjega ranga, zato je lahko različen
od nič le v kristalih brez centra inverzije. Pri inverziji namreč električno 
polje spremeni predznak, odziv pa bi moral biti na inverzijo neobčutljiv. 

Simetrija tudi v primeru, ko ni centra inverzije, navadno močno
zmanjša število neodvisnih komponent $r_{ijk}$. Pockelsov tenzor
je po definiciji simetričen v prvih dveh indeksih
\beq
r_{ijk} = r_{jik},
\eeq
zato ima v najmanj simetričnem primeru triklinskega kristala 18 neodvisnih komponent,
v kristalih z višjo simetrijo pa še manj. 

Za primer poglejmo, kako štirištevna simetrijska os zmanjša število komponent. 
Naj bo simetrijska os v smeri $z$, električno polje pa vzporedno z njo.
Rotacija za $\pi/2$ okoli osi $z$ je simetrijska operacija in prevede
os $x$ v $y$, zato mora biti $r_{xxz}=r_{yyz}$. Poglejmo še zvezo 
\begin{equation}
\delta b_{xy}=r_{xyz}E_{z}.
\label{7.4}
\end{equation}
Pri rotaciji za $\pi/2$ gre $x$ v $y$, $y$ pa v $-x$, zato gre
$\delta b_{xy}$ v $-\delta b_{yx}$ in dobimo 
\begin{equation}
-\delta b_{xy}=r_{xyz}E_{z}.
\label{7.5}
\end{equation}
Enačbi~(\ref{7.4}) in~(\ref{7.5}) sta lahko hkrati izpolnjeni, le če je $r_{xyz}=0$.
Rotacija za $\pi$ prevede zvezo $\delta b_{xz}=r_{xzz}E_{z}$ v $-\delta b_{xz}=r_{xzz}E_{z}$,
zato je tudi $r_{xzz}=0$ in z enakim razmislekom tudi $r_{yzz}=0$.

Obrnimo zdaj polje v smer osi $x$. Pri rotaciji za $\pi$ gre $E_{x}$
v $-E_{x}$, $\delta b_{xx}$ pa se ne spremeni. Od tod sledi $r_{xxx}E_{x}=-r_{xxx}E_{x}$ in 
posledično $r_{xxx}=0$. Rotacija za $\pi/2$ zvezo $\delta b_{yz}=r_{yzx}E_{x}$
prevede v $-\delta b_{xz}=r_{yzx}E_{y}$. Po definiciji pa velja $\delta b_{xz}=r_{xzy}E_{y}$,
od koder sledi $r_{yzx}=-r_{xzy}$.

Podobno ravnamo še s preostalimi komponentami in tako ugotovimo, da
so v tetragonalni grupi, ki vsebuje le štirištevno simetrijsko os, štiri neodvisne
komponente elektro-optičnega tenzorja: $r_{xxz}=r_{yyz}$, $r_{zzz}$,
$r_{yzx}=-r_{xzy}$ in $r_{xzx}=r_{yzy}$, vse ostale komponente pa
so enake nič. Nekaj primerov Pockelsovih tenzorjev pri različnih kristalnih simetrijah
je podanih v tabeli~(\ref{table:Pockels}).

\begin{table}[h!]
 \centering
\begin{tabular}{|c|c|c|c|} \hline  
      Kristal & Grupa & Neničelne komponente tenzorja $r$ & Vrednost ($10^{-12}$~m/V)\\ \hline
      BaTiO$_3$\index{BaTiO$_3$} & 4mm & $r_{xzx} = r_{yzy} = r_{zxx} = r_{zyy} = 
      r_{51} = r_{42}$  &
	    (pri 1,55~$\mu$m) $r_{51} = 800$ \\
	      & & $r_{zzx} = r_{zzy} = r_{31} = r_{32}$ &  $r_{31} = 8$ \\
	      & & $r_{zzz} = r_{33}$ & $r_{33} = 28$ \\ \hline
      KDP\index{KDP} & 
      $\overline{4}$2m & $r_{yzx} = r_{zyx} = r_{xzy} = r_{zxy} = r_{41} = r_{52}$  &
	    $r_{41} = 8,77$ \\
	    & & $r_{xyz} = r_{yxz} = r_{63}$ &  $r_{63} = -10,3$ \\ \hline
      GaAs\index{GaAs}\index{ZnTe} &  $\overline{4}$3m&
	  $r_{yzx} = r_{zyx} = r_{xzy} = r_{zxy} = r_{xyz} = r_{yxz}$  & (pri 10,6~$\mu$m) $r_{41} = 1,5$ \\
	ZnTe  & &   $= r_{41} = r_{52}=r_{63}$  &(pri 3,4~$\mu$m) $r_{41} = 4,2$ 
	    \\ \hline
      LiNbO$_3$\index{LiNbO$_3$} & 3m & $r_{xzx} = r_{zxx} = r_{yzy} = r_{zyy} = r_{51} = r_{42}$  &
	    $r_{51} = 32,6$ \\
	     & & $r_{xxz} = r_{yyz} = r_{13} = r_{23}$ &  $r_{13} = 9,6$ \\
	      & & $r_{zzz} = r_{33}$ & $r_{33} = 30,9$ \\
	    & &  $r_{yyy} = - r_{xxy} = -r_{xyx} = -r_{yxx}  = $ & \\
	    & &  $=r_{22} =  -r_{12} =-r_{61} $  &
	    $r_{22}  = 6,8$ \\
\hline 
\end{tabular}
  \caption{Koeficienti Pockelsovega tenzorja za nekaj izbranih snovi. Če ni navedeno drugače, veljajo
  vrednosti pri valovni dolžini okoli 600~nm.}
\label{table:Pockels}
\end{table}

Podobno kot smo to naredili pri nelinearni susceptibilnosti (poglavje~\ref{Chap:Chi}), 
tudi elektro-optični tenzor pogosto zapišemo le z dvema komponentama. 
Prva dva indeksa, v katerih je $r_{ijk}$ simetričen, združimo
v enega z vrednostmi od 1 do 6 po dogovoru $xx=1$, $yy=2$, $zz=3$,
$yz=4$, $zx=5$ in $xy=6$. Tako postane $r_{ijk}$ matrika velikosti
$6\times3$, simetrični tenzor drugega ranga $b_{ij}$ pa šetkomponenten
vektor.

\section{Longitudinalna modulacija}
Poglejmo na primeru, kako električno polje spremeni lastnosti elektro-optičnega kristala
in kako vpliva na svetlobo, ki potuje skozi tak kristal. V praksi se navadno uporabljajo 
kristali, ki so dvolomni že brez zunanjega polja. 
Kot primer vzemimo kristal KH$_{2}$PO$_{4}$ (KDP), ki ima tetragonalno 
simetrijo ($\bar{4}2m$). Kot razberemo iz tabele~(\ref{table:Pockels}) ima 
elektro-optični tenzor dve nedodvisni komponenti: $r_{41} = r_{52}=10,3 \times 10^{-11}$~m/V
in $r_{63}=8,77 \times 10^{-12}$~m/V.

Kristal naj bo odrezan po kristalografskih oseh, svetloba naj skozi kristal potuje 
v smeri optične osi, to je smeri $z$, v isti smeri pa na kristal priključimo
polje $E_z$ (t.i. longitudinalna modulacija). 
\begin{figure}[h]
\centering
\def\svgwidth{80truemm} 
\input{slike/09_AMshema.pdf_tex}
\caption{Shema longitudinalne modulacije signala. Ker je polje priključeno v smeri
potovanja svetlobe, morata biti elektrodi transparentni.}
\label{fig:amshema}
\end{figure}
Inverzni tenzor dielektričnosti brez priključenega polja zapišemo kot
\beq
\underline{\tilde{b}} = 
\left[\begin{array}{ccc}
1/n_o^2 & 0& 0\\
0 & 1/n_o^2& 0\\
0 & 0&  1/n_e^2
\end{array}\right],
\label{7.8}
\eeq
pri čemer sta $n_o$ in $n_e$ redni in izredni lomni količnik. Ko priključimo 
polje, se tenzor dielektričnosti spremeni zaradi Pockelsovega pojava. Sprememba
inverznega tenzorja dielektričnosti je po enačbi~(\ref{eq:Pockels})
\begin{align}
\delta b_{xx} & =r_{xxz}E_z = 0\\
\delta b_{xy} & = \delta b_{yx} = r_{xyz}E_z = r_{63}E_z\\
\delta b_{xz} & = \delta b_{zx} =r_{xzz}E_z = 0\\
\delta b_{yy} & =r_{yyz}E_z = 0\\
\delta b_{yz} & = \delta b_{zy} =r_{yzz}E_z = 0\\
\delta b_{zz} & =r_{zzz}E_z = 0.
\end{align}
Vidimo, da je večina členov enaka nič, se pa zaradi električnega polja v smeri $z$ pojavi 
izvendiagonalna komponenta dielektričnega tenzorja 
\beq
\underline{b} = 
\left[\begin{array}{ccc}
1/n_o^2 & 0& 0\\
0 & 1/n_o^2 & 0\\
0 & 0& 1/n_e^2
\end{array}\right] + \left[\begin{array}{ccc}
 0& r_{63}E_z& 0\\
r_{63}E_z & 0 & 0\\
0 & 0&  0
\end{array}\right] = \left[\begin{array}{ccc}
1/n_o^2 & r_{63}E_z& 0\\
r_{63}E_z& 1/n_o^2 & 0\\
0 & 0&  \frac{1}{n_e^2}
\end{array}\right].
\label{7.8a}
\eeq
Če želimo izračunati, kako se po takem kristalu širi vpadni svetlobni
snop, moramo gornji dielektrični tenzor diagonalizirati. Lastne vrednosti novega tenzorja
in pripadajoče nove lastne osi so
\begin{align}
\lambda_1 &= \frac{1}{n_o^2}+ r_{63}E_z \quad \mathrm{in} \quad \mathbf{e}_1' = \frac{1}{\sqrt{2}}(1,1,0)\\
\lambda_2 &= \frac{1}{n_o^2}- r_{63}E_z \quad \mathrm{in} \quad \mathbf{e}_2' = \frac{1}{\sqrt{2}}(-1,1,0)\\
\lambda_3 &= \frac{1}{n_e^2} \quad \mathrm{in} \quad \mathbf{e}_3' = (0,0,1).
\end{align}
Vidimo, da so nove lastne osi zasukane za kot $45~^\circ$ glede na prvotne osi sistema.
V novem koordinatnem sistemu je inverzni dielektrični tenzor diagonalen in enak
\beq
\underline{b} = 
\left[\begin{array}{ccc}
1/n_o^2 + r_{63}E_z& 0& 0\\
0 & 1/n_o^2 - r_{63}E_z& 0\\
0 & 0& 1/n_e^2
\end{array}\right].
\eeq
\begin{figure}[h]
\centering
\def\svgwidth{60truemm} 
\input{slike/09_AMindikatrisa.pdf_tex}
\caption{Optično enoosni kristal postane pod napetostjo dvoosen. Indikatrisa, ki je pravokotno
na optično os brez polja krožnica, se pod vplivom napetosti spremeni v elipso. }
\label{fig:amn}
\end{figure}
Spomnimo se, da svetloba potuje skozi kristal vzdolž osi $z$. Brez zunanjega električnega
polja je kristal enoosen z optično osjo v smeri $z$. Lomni količnik je torej neodvisen od
polarizacije vpadnega valovanja in je enak $n_o$. Ko priključimo polje, postane kristal
optično dvoosen, saj so vse tri lastne vrednosti tenzorja dielektričnosti različne. Za žarek, 
ki potuje vzdolž osi $z$, torej obstajata dve lastni smeri $x'$ in $y'$ z ustreznima
novima lastnima količnikoma, ki ju izrazimo kot
\beq
\frac{1}{n_x'^2} = \frac{1}{n_o^2}+ r_{63}E_z \quad \mathrm{in} \quad 
\frac{1}{n_y'^2} = \frac{1}{n_o^2}- r_{63}E_z. 
\eeq
Za vse druge smeri polarizacije vhodne svetlobe, dobimo po preletu kristala eliptično polariziran žarek.

Za vsa eksperimentalno dosegljiva polja velja, da je $rE\ll1/n^2$, 
zato lahko gornja izraza razvijemo za majhne popravke. Dobimo
\beq
n_x' = \sqrt{\frac{n_o^2}{1+ n_o^2 r_{63}E_z}} \approx n_o \sqrt{1- n_o^2 r_{63}E_z}
\eeq
in 
\boxeq{EOnx}{
n_x'\approx n_o - \frac{1}{2}n_o^3 r_{63}E_z.
}
Podobno izpeljemo še za drugo lastno vrednost
\boxeq{EOny}{
n_y'\approx n_o + \frac{1}{2}n_o^3 r_{63}E_z.
}
Različni lastni polarizaciji potujeta vzdolž osi $z$ z različnima hitrostima. 
Med njima zato pride do fazne razlike, ki je enaka
\beq
\Delta \phi = k_0 n_y' L - k_0 n_x' L = \frac{\omega}{c_0}L 
n_o^3 r_{63}E_z.
\eeq
Kristal torej doda vpadnemu valovanju dodatni fazni zamik, ki je odvisen od električne poljske
jakosti $E_z$. Vpeljemo še karakteristično napetost $U_\pi$, pri kateri je dodatna 
fazna razlika enaka $\pi$ in kristal deluje kot ploščica $\lambda/2$
\beq
U_\pi = \frac{\pi c_o}{\omega n_o^3 r_{63}} = \frac{\lambda}{2 n_o^3 r_{63}}.
\eeq
Za kristal KDP je $\pi$-vrednost napetosti pri valovni dolžini $633$~nm enaka $8970$~V. 
Izračunana napetost je precej velika. Velike delovne napetosti
so značilne za kristalne elektro-optične modulatorje in so njihova
glavna pomanjkljivost. 

\section{Transverzalna modulacija}
Iz praktičnih razlogov je navadno preprosteje uporabili transverzalno modulacijo.
Tudi to postavitev obravnavajmo na primeru, za zgled vzemimo kristal LiNbO$_3$.


Poglejmo na primeru, kako električno polje spremeni lastnosti elektro-optičnega kristala
in kako vpliva na svetlobo, ki potuje skozi tak kristal. V praksi se navadno uporabljajo 
kristali, ki so dvolomni že brez zunanjega polja. 
Zato kot primer vzemimo kristal KH$_{2}$PO$_{4}$ (KDP), ki ima tetragonalno 
simetrijo ($\bar{4}2m$). Kot razberemo iz tabele~(\ref{table:Pockels}) ima 
elektro-optični tenzor dve nedodvisni komponenti: $r_{41} = r_{52}=10,3 \times 10^{-11}$~m/V
in $r_{63}=8,77 \times 10^{-12}$~m/V.




 
 Ker je kristal že sam po sebi dvolomen, povzroči zunanje
polje le majhno dodatno fazno razliko. Dolžina kristala mora biti
taka, da velja $k_{0}L(n_{r}-n_{i})=2N\pi$, če naj bo modulator brez
zunanjega polja zaprt. Pri tem nastopi težava. Pogoj je lahko zaradi
temperaturnega raztezanja in odvisnosti lomnih količnikov od temperature
izpolnjen le pri eni temperaturi, poleg tega bi bil tak modulator
tudi zelo občutljiv na to, da se svetloba širi natanko v smeri $y^{\prime}$.
Zato dvolomnost nemotenega kristala kompenziramo tako, da vzamemo
dva enako dolga kristala in ju postavimo zapored tako, da sta optični
osi med seboj pravokotni, kot kaže slika \ref{s7.3}. Modulacijska
napetost na drugem kosu mora imeti nasprotni predznak. Tedaj se fazna
razlika med obema polarizacijama zaradi naravne dvolomnosti odšteje,
zaradi modulacijske napetosti pa sešteje.





Polarizator na izhodni strani prepusti le projekcijo obeh lastnih
polarizacij: 
\begin{equation}
E_{izh}=\frac{1}{\sqrt{2}}E_{vh}(\frac{1}{\sqrt{2}}-\frac{1}{\sqrt{2}}e^{i\phi})\;,\label{7.16}
\end{equation}
 kjer je $E_{vh}$ amplituda svetlobnega vala za vhodnim polarizatorjem.
Gostota prepuščenega svetlobnega toka bo torej 
\begin{equation}
j_{izh}=\frac{1}{4}j_{vh}|1-e^{i\phi}|^{2}=\frac{1}{2}j_{vh}(1-\cos\phi)\;.\label{7.17}
\end{equation}
 Ko je napetost na kristalu nič, je $\phi=0$ in je tudi $j_{izh}=)$,
kot pričakujemo, saj sta analizator in polarizator prekrižana, kristal
pa je brez polja optično izotropen. Prepustnost je največja, ko
je $\phi=\pi$.


Včasih želimo, da je zveza med modulacijsko napetostjo in izhodno
gostoto toka linearna. Za to mora modulator delovati v okolici $\phi=\pi/2$.
Namesto visoke stalne napetosti lahko uporabimo med polarizatorjem
in kristalom še ploščico $\lambda/4$, ki nam da zahtevani stalni
fazni premik med rednim in izrednim valom.


\section{Amplitudna modulacija}
Poglejmo, kako lahko elektro-optični pojav izkoristimo za modulacijo
amplitude svetlobnega snopa. Osnovna zamisel je, da z električnim
poljem tako spremenimo dvolomnost primerno izbranega in odrezanega
kristala, da se spremeni polarizacija vpadnega vala, zaradi česar
se spremeni tudi svetlobna moč, ki jo prepusti analizator za kristalom.
Poglejmo si to kar na primeru.



\section{Fazna in frekvenčna modulacija}

Amplitudno modulacijo svetlobe smo dobili tako, da smo z zunanjim
poljem spremenili fazi lastnih valov, zaradi česar je postalo linearno
polarizirano vpadno valovanje po prehodu kristala eliptično polarizirano.
Spremembo polarizacije smo z analizatorjem prevedli v spremembo amplitude.
Včasih pa želimo modulirati fazo vpadne svetlobe. To naredimo tako,
da odstranimo izhodni polarizator, vhodno polarizacijo pa usmerimo
tako, da je vpadna svetloba lastno valovanje, za katero je sprememba
lomnega količnika zaradi modulacijskega polja večja. V primeru ZnTe,
ki smo ga obravnavali v prejšnjem razdelku, je to izredno valovanje,
polarizirano v smeri modulacijskega polja. Dodatna faza na izhodni
strani kristala je 
\begin{equation}
\phi=k_{0}L(n_{i}-n_{0})=\frac{\omega n_{0}^{3}rLU}{\sqrt{3}cd}.
\label{7.20}
\end{equation}

Če je sprememba faze linearna funkcija časa, to je, če jo lahko zapišemo
v obliki $\phi=\omega_{1}t+\phi_{0}$, predstavlja koeficient $\omega_{1}$
spremembo frekvence vpadne svetlobe. Linearno naraščajoča modulacijska
napetost da torej spremembo frekvence, kar v optiki pogosto potrebujemo.
Dosegljive spremembe frekvence $\omega_{1}$ so seveda dokaj majhne,
do nekaj sto MHz. Omejene so z možno hitrostjo spreminjanja napetosti.
Napetost seveda tudi ne more neomejeno naraščati. Kadar se napetost
vrača na nič, pride do frekvenčnega premika v nasprotni smeri, ki pa ga
lahko zanemarimo, če je čas vračanja kratek primeri s časom naraščanja.

Poglejmo še, kakšen je spekter svetlobe, če je fazna modulacija periodična:
\begin{equation}
U=U_{m}\sin\omega_{m}t\;.\label{7.21}
\end{equation}
 Polje izhodne svetlobe bo tedaj 
\begin{equation}
E_{izh}=E_{vh}\cos(\omega t+\delta\sin\omega_{m}t)\;,\label{7.22}
\end{equation}
 kjer je $\delta=\omega n_{0}^{3}rU_{m}L/(\sqrt{3}cd)$. Z uporabo
identitet 
\begin{eqnarray}
\cos(\delta\sin x) & = & \mbox{J}_{0}(\delta)+2\mbox{J}_{2}(\delta)\cos2x+\ldots\nonumber \\
\sin(\delta\sin x) & = & 2\mbox{J}_{1}(\delta)\sin x+2\mbox{J}_{3}\sin3x+\ldots
\end{eqnarray}
 je izhodno polje mogoče zapisati v obliki 
\begin{eqnarray}
E_{izh} & = & E_{vh}[\mbox{J}_{0}(\delta)\cos\omega t\pm\mbox{J}_{1}(\delta)\cos(\omega\pm\omega_{m})t+\nonumber \\
 &  & +\mbox{J}_{2}(\delta)\cos(\omega\pm2\omega_{m})t\pm\ldots]
\end{eqnarray}
 Periodična fazna modulacija torej da v spektru stranske pasove, odmaknjene
od osnovne frekvence $\omega$ za modulacijsko frekvenco. Njihova
velikost je podana s kvadratom Besselovih funkcij parametra $\delta$.
Če je ta majhen, se lahko zadovoljimo le s prvim členom.

\section{Modulacija pri visokih frekvencah}

Pogosto je pomembna hitrost elektrooptične modulacije. Zato polejmo,
kaj se zgodi pri visokih modulacijskih frekvencah.

Elektro-optični pojav pri nizkih frekvencah ima dva prispevka: direktnega,
kjer zunanje polje vpliva neposredno na elektronsko polarizabilnost,
in posrednega preko piezoelektričnega pojava. Snovi, ki nimajo centra
inverzije, so tudi piezoelektrične in se v zunanjem električnem polju
deformirajo. Deformacija pa povzroči spremembo lomnega količnika,
o čemer bomo podrobneje govorili v enem od naslednjih oddelkov. Celotno
spremembo tenzorja $b_{ij}$ lahko zapišemo 
\begin{eqnarray}
\delta b_{ij} & = & r_{ijk}^{\ast}E_{k}+p_{ijlm}S_{lm}\nonumber \\
 & = & r_{ijk}^{\ast}E_{k}+p_{ijlm}\pi_{lmk}E_{k}
\end{eqnarray}
 Tu je $S_{lm}=\pi_{lmk}E_{k}$ piezoelektrično povzročena deformacija.
Pri nizkih frekvencah sta oba prispevka primerljivo velika in je efektivni
elektro-optični tenzor $r_{ijk}=r_{ijk}^{\ast}+p_{ijlm}\pi_{lmk}$.
Pri dovolj velikih frekvencah deformacija kristala ne more več slediti
modulacijski napetosti in ostane le direktni prispevek $r_{ijk}^{\ast}$.
To se zgodi nad akustičnimi resonancami kristala. Pri akustičnih resonancah,
to je, kadar modulacija v kristalu vzbudi stoječe zvočno valovanje,
pa se piezoelektrični prispevek resonančno poveča.

Pogoj za akustično resonanco je, da je dimenzija kristala mnogokratnik
polovice valovne dolžine akustičnega vala v kristalu. Uporabne dimenzije
kristalov so reda velikosti centimeter, hitrost zvočnih valov je okoli
5000~m/s, tako da so resonance v področju od nekaj sto kHz do
nekaj deset MHz. Mogoče jih je tudi izkoristiti za povečanje elektro-optičnega
efekta pri izbrani frekvenci.

Pri visokih frekvencah postane pomembna tudi električna vezava modulatorja.
Kristal predstavlja neko kapacitivno breme. Njegova impedanca pada
z rastočo frekvenco, zato je vedno večji del padca napetosti na notranjem
uporu izvora napetosti. Pomagamo si lahko tako, da vzporedno s kristalom
vežemo še tuljavo, tako da je resonančna frekvenca $1/(L_{t}C)$ nastalega
nihajnega kroga enaka željeni modulacijski frekvenci $\omega_{m}$.
Tedaj je večina padca napetosti na kristalu in tuljavi. Da resonanca
ni preostra in da je na voljo dovolj širok pas modulacijskih frekvenc,
vežemo vzporedno s kristalom še upor z upornostjo $R$. Širina modulacijskega
pasu je 
\begin{equation}
\Delta\omega_{m}=\frac{1}{RC}\;.\label{7.24}
\end{equation}
 Na uporu se troši moč 
\begin{equation}
P=\frac{1}{2}\frac{U^{2}}{R}=\frac{1}{2}U^{2}C\Delta\omega_{m}=\frac{\epsilon\epsilon_{0}La}{2d}U^{2}C\Delta\omega_{m}\;,\label{7.25}
\end{equation}
 kjer je $a$ velikost kristala v prečni smeri. Naj bo $U$ ravno
napetost, ki da fazno razliko $\pi$. Potem sledi z uporabo enačbe
\ref{7.18} 
\begin{equation}
P=\frac{A}{L}\,\frac{\epsilon\epsilon_{0}\Delta\omega_{m}\lambda^{2}}{3n_{0}^{6}r^{2}}\;,\label{7.26}
\end{equation}
 kjer je $A$ prečni presek kristala. Potrebna moč je odvisna od lastnosti
modulatorja in širine modulacijskega pasu. Pri širini modulacijskega
pasu 1~MHz in preseku kristala 1~cm$^{2}$ je potrebna moč nekaj
deset W, kar je za visokonapetosten in hiter izvor že znatna moč.

\section{Elastooptični pojav}

Dielektrične lastnosti in lomni količnik so odvisne tudi od deformacije
snovi. Podobno kot pri elektro-optičnem pojavu lahko spremembo obratnega
dielektričnega tenzorja zapišemo 
\begin{equation}
\delta b_{ij}=p_{ijkl}S_{kl}\;.\label{7.27}
\end{equation}
 $S_{kl}$ je tenzor defomacije snovi: 
\begin{equation}
S_{kl}=\frac{1}{2}\left({\frac{\partial u_{k}}{\partial x_{l}}}+{\frac{\partial u_{l}}{\partial x_{k}}}\right)\;,\label{7.28}
\end{equation}
 $p_{ijkl}$ pa \textit{elastooptični tenzor}. Ta je različen od nič
v vsaki snovi, ker povezuje dva simetrična tenzorja drugega ranga.
Popisuje tudi spremembo dielektrične konstante in lomnega količnika
zaradi spremembe gostote snovi. Je simetričen v prvem in drugem paru
indeksov, tako da ima v najbolj splošnem primeru 36 neodvisnih komponent.
Simetrija danega kristala seveda to število zmanjša.

Podobno kot pri elektro-optičnem pojavu lahko iz \ref{7.27} izrazimo
spremembo dielektričnega tenzorja 
\begin{equation}
\delta\epsilon_{ij}=-\epsilon_{ii}\epsilon_{jj}p_{ijkl}S_{kl}\;,\label{7.29}
\end{equation}
 kjer smo že predpostavili, da je nemoteni $\epsilon$ diagonalen.

Dvojni lom, ki se pojavi v deformirani snovi, izkoriščamo za študij
mehanskih napetosti v modelih, ki so izdelani iz prozorne plastične
snovi. Nas bo v nadaljevanju zanimal uklon svetlobe na periodični
modulaciji lomnega količnika, ki nastane zaradi zvočnih valov v snovi.


\section{Braggov uklon na zvočnih valovih}

V plasti prozorne snovi vzbudimo stoječe zvočno valovanje. V zgoščini
je lomni količnik nekoliko večji, zato je optična pot na takem mestu
skozi plast daljša. Ravno svetlobno valovanje, ki pada na plast, po
izstopu iz plasti ne bo imelo več povsod enake faze, valovno čelo
bo periodično modulirano s periodo valovne dolžine zvoka. V veliki
oddaljenosti od plasti bomo poleg osnovnega snopa dobili še uklonjene
snopa v smereh, za katere velja 
\begin{equation}
\Lambda\sin\theta=\pm2N\pi\;,\label{7.29}
\end{equation}
 kjer smo z $\Lambda$ označili valovno dolžino zvoka v snovi. Zgoščine
in razredčine izginejo vsake pol zvočne periode, zato je tudi intenziteta
uklonjenih snopov modulirana z dvojno frekvenco zvoka.

V splošnem je delež uklonjene svetlobe neuporabno majhen. Znaten postane
le tedaj, kadar je za enega od uklonjenih valov izpolnjen Braggov
pogoj, to je, kadar velja 
\begin{equation}
\vec{k}_{0}\pm\vec{q}=\vec{k}_{1}\;,\label{7.30}
\end{equation}
 kjer je $\vec{k}_{0}$ valovni vektor vpadne svetlobe, $\vec{k}_{1}$
valovni vektor uklonjenega svetlobnega snopa, $\vec{q}$ pa valovni
vektor zvočnega vala. Znak plus velja, kadar potuje zvok proti projekciji
$\vec{k}_{0}$ na $\vec{q}$. Enačba \ref{7.30} je pogoj za ohranitev
gibalne količine fotona pri sipanju na zvočnem valu. Lahko jo prepišemo
še nekoliko drugače. Frekvenca zvočnega vala je dosti nižja od frekvence
svetlobe, zato se frekvenca svetlobe pri sipanju le malo spremeni
in sta $vec{k}_{0}$ in $\vec{k}_{1}$ po velikosti skoraj enaka.
Tedaj je $q=2k_{0}\sin\theta_{B}/2$ (glej sliko \ref{7.7}), od koder
zapišmo Braggov pogoj v obliki 
\begin{equation}
2\Lambda\sin\frac{\theta_{B}}{2}=\lambda_{0}\;.\label{7.31}
\end{equation}
 Obenem mora biti vpadni kot na zvočni val enak izhodnemu, torej na
zvočnem valu se Braggovo sipana svetloba zrcalno odbije. Razmere so
povsem analogne Braggovemu sipanju rentgenske svetlobe na kristalnih
ravninah. Kadar je izpolnjen Braggov pogoj, je mogoče doseči, da se
vsa vapdna svetloba siplje, kot bomo pokazali nekoliko kasneje.

Če je zvočni val potujoč, kar smo v gornjem razmišljanju že privzeli
s tem, ko smo mu pripisali natanko določen valovni vektor $\vec{q}$,
se spremeni tudi frekvenca sipanega vala zaradi Dopplerjevega premika
pri odboju zvočnem valu, ki potuje s hitrostjo $v_{z}$. Upoštevati
moramo le projekcijo na smer vapdne in odbite svetlobe, zato je 
\begin{equation}
\frac{\Delta\omega}{\omega}=\pm\frac{2v_{z}\sin\theta_{B}/2}{c}=
\pm\frac{2\Omega\Lambda\sin\theta_{B}/2}{c}=\pm\frac{\Omega}{\omega};.\label{7.32}
\end{equation}
 Upoštevali smo, da velja Braggov pogoj \ref{7.31}. Sprememba frekvence
sipane svetlobe je kar enaka frekvenci zvočnega vala. To sledi tudi
iz zahteve, da se mora pri sipanju na zvočnem valu ohraniti energija
vpadnega fotona in kvanta zvo\textquotedbl{}nega valovanja - fonona,
ki se pri sipanju absorbira (znak plus, zvočni val potuje proti projekciji
$\vec{k}_{0}$ na $\vec{q}$) ali nastane.

Kadar se pojavi v snovi stoječe zvočno valovanje, lahko sipanje obravnavamo
kot vsoto sipanja na dveh valovanjih z valovnima vektorjema $\vec{q}$
in $-\vec{q}$. Smer Braggovo sipanega vala je obakrat enaka, frekvenca
pa se enkrat poveča, drugič zmanjša za $\Omega$. Zato se pojavi utripanje
sipanega vala s frekvenco $2\Omega$.

Braggovo sipanje svetlobe na zvočnih valovih se uporablja v več optičnih
napravah. Najpomembnejše je uklanjanje svetlobe iz vpadne smeri, pri
čemer ima uklonjeni snop še spremenjeno frekvenco. Z vklaplanjem in
izklaplanjem zvočnega vala, ki ga vzbujamo s piezoelektričnim elementom,
na katerega pritisnemo izmenično napetost, lahko moduliramo intenziteto
direktnega svetlobnega snopa. To potrebujemo na primer pri preklaplanju
kvalitete laserskega resonatorja. S spreminjanjem zvočne frekvence
pa lahko spreminjamo smer uklonjenega snopa, pri čemer pa smo precej
omejeni s tem, da mora biti približno izpolnjen Braggov pogoj.

Druga uporaba je spreminjanje frekvence svetlobe. Možne so spremembe
do nekaj sto MHz, kar je ravno primerno za uporabo v laserskih merilnikih
hitrosti, kjer merimo frekvenco utripanja med svetlobo, odbito od
merjenega predmeta, in referenčno svetlobo. Če ima referenčna svetloba
isto frekvenco kot merilni snop, ni mogoče določiti predznaka hitrosti
predmeta, če pa referenčni svetlobi nekoliko spremenimo frekvenco,
se pojavi utripanje tudi tedaj, ko predmet miruje. Frekvenca utripanja
se poveča ali zmanjša glede na predznak hitrosti predmeta.

Tretja pomembna uporaba je kombinacija obeh gornjih za uklepanje faz
v laserskem resonatorju. Če je v Braggovem elementu prisotno stoječe zvočno
valovanje, je amplituda direktnega snopa modulirana s frekvenco zvoka.
Če je frekvenca zvoka ravno enaka razmiku frekvenc laserskih nihanj,
lahko nastanejo uklenjene faze vzbujenih nihanj in s tem kratke, periodične
sunke svetlobe, kot smo videli v petem poglavju.

Zanimiva je tudi možnost, da napravimo s pomočjo Braggovega elementa
hiter frekvenčni analizator električnih signalov. Shemo kaže slika
\ref{s7.8}. Piezoelektrični element vzbujamo z električnim siganlom,
ki ima neznan spekter. Enak spekter imajo tudi vzbujeni zvočni valovi.
Vsakemu valu določene frekvence ustreza določen kot odklona svetlobnega
snopa. Za Braggovim elementom postavimo lečo. Vsak delni uklonjeni
snop da v goriščni ravnini svetlo točko, katere položaj je odvisen
od kota odklona in s tem od frekvence zvočnega vala. Spekter zaznamo
z vrstičnim detektorjem. Na celo napravo lahko pogledamo tudi takole:
Braggova celica frekvenčni spekter zvočnih valov prevede v prostorski
spekter prepuščene svetlobe. Prostorski spekter svetlobe pa lahko
analiziramo z lečo, ki nam v goriščni ravnini na desni da prostorsko
Fourierovo transformacijo svetlobnega snopa na levi strani leče.

Nismo še ugotovili, kolikšen je delež uklonjene svetlobe. Rešiti moramo
valovno enačbo v nehomogenem sredstvu, kar je dokaj težaven problem
in se moramo zateči k približkom. Ena možnost je, da uporabimo običajno
uklonsko teorijo, kjer privzamemo, da lahko vpliv periodične modulacije
lomnega količnika snovi upoštevamo s spremembo faze svetlobnega vala
na izhodu iz snovi: $\delta\phi=\delta nk_{0}L$, kjer je $L$ debelina
plasti. (Naloga) Vendar je tak račun dober le v primeru, kadar je
prečna variacija faze majhna in je debelina $L$ majhna.

Primernejša je metoda \textit{sklopljenih valov}. Paralelen snop zvočnega
valovanja z valovnim vektorjem $vec{q}$ naj potuje v smeri $x$.
Širina snopa naj bo $L$. Nanj pod kotom $\phi$ glede na os $z$
vpada ravno svetlobno valovanje z valovnim vektorjem $\vec{k}=(k_{1},0,k_{3})=k(\sin\phi,0,\cos\phi)$.
Vse valovanje, vpadno na levi zvočnega snopa in izhodno na desni,
obravnavajmo znotraj snovi, da nam ni treba upoštevati še loma, ki
le zaplete izraze. Zaradi periodične modulacije lomnega količnika
so ravni valovi, katerih $x$-komponente valovnih vektorjev se razlikujejo
za $q$, med seboj sklopljeni, zato njihove amplitude niso konstantne,
temveč se v smeri $z$ počasi spreminjajo, kar želimo v različnih
približkih izračunati.

Privzemimo, da se v snovi zaradi zvočnih valov spremeni le velikost
dielektrične konstante, ki jo lahko zapišemo v obliki 
\begin{equation}
\epsilon=\epsilon^{\prime}+\epsilon_{1}\sin(qx-\Omega t)\;,\label{7.33}
\end{equation}
 kjer je sprememba $\epsilon_{1}$ povezana z amplitudo deformacije
$S_{0}$ v zvočnem valu :$\epsilon_{1}=-\epsilon^{2}pS_{0}$. Izpustili
smo indekse tenzorjev.

Valovna enačba ima obliko 
\begin{equation}
\nabla^{2}E=\frac{\epsilon^{\prime}}{c^{2}}{\frac{\partial E^{2}}
{\partial t^{2}}}+\mu_{0}{\frac{\partial P_{1}^{2}}{\partial t^{2}}}\;.\label{7.33}
\end{equation}
 Na levi strani smo zanemarili, da $\nabla\cdot\mathbf{E}\ne0$, če je
$\epsilon$ funkcija kraja. Da s tem nismo zagrešili znantne napake,
naj bralec ugotovi sam (Naloga). $P_{1}=\epsilon_{0}\epsilon_{1}\sin(qx-\Omega t)E$
predstavlja dodatno polarizacijo snovi, ki nastane zaradi zvočnega
vala.

Enačbo \ref{7.33} brez $P_{1}$ rešijo ravni valovi. Učinek člena
s $P_{1}$ je, da ravnemu valu z valovnim vektorjem $\vec{k}$ in
frekvenco $\omega$ primeša val z valovnim vektorjem $\vec{k}\pm\vec{q}$
in frekvenco $\omega\pm\Omega$. Zato iščimo rešitve v obliki vsote
ravnih valov, torej Fourierove vrste 
\begin{equation}
E=\sum_{n}A_{n}(z)e^{in(qx-\Omega t)}e^{i(k_{1}x+k_{3}z-\omega t)}\;.\label{7.34}
\end{equation}
 Zaradi sklopitve preko $P_{1}$ moramo dovoliti, da so amplitude
$A_{n}$ funkcije $z$. Če je $\epsilon_{1}$ dovolj majhen, se $A_{n}(z)$
le počasi spreminjajo.

Izračunajmo 
\begin{equation}
\nabla^{2}E=\sum_{n}\left\{ -[k_{3}^{2}+(k_{1}+nq)^{2}]A_{n}+2ik_{3}A_{n}^{\prime}\right\} \, e^{i[(k_{1}+nq)x+k_{3}z-(\omega+n\Omega)t]}\;.\label{7.35}
\end{equation}
 Člen z $A_{n}^{\prime\prime}$ lahko izpustimo, če je le $k_{3}A_{n}^{\prime}>>A_{n}^{\prime\prime}$,
to je, če se $A_{n}$ spreminjajo počasi v primerjavi z exp$(ik_{3}z)$.
Vstavimo izraza \ref{7.34} in \ref{7.35} v valovno enačbo \ref{7.33}
in zahtevajmo, da je vsak člen vsote po $n$ posebej enak nič. Tako
dobimo 
\begin{eqnarray}
-[k_{3}^{2}+(k_{1}+nq)^{2}]A_{n}+2ik_{3}A_{n}^{\prime}=\nonumber \\
\quad=-\frac{\epsilon^{\prime}}{c^{2}}(\omega+n\Omega)^{2}[A_{n}-
\frac{\epsilon_{1}}{2i\epsilon^{\prime}}(A_{n-1}-A_{n+1}) & \;.
\end{eqnarray}
 Upoštevamo, da je $k_{1}^{2}+k_{3}^{2}=\epsilon^{\prime}(\omega/c)^{2}=k^{2}$
in $\Omega=v_{z}q$. Ker je $v_{z}<<c$, zanemarimo člene reda $v_{z}/c$,
in dobimo 
\begin{equation}
A_{n}^{\prime}+i\beta_{n}A_{n}+\xi(A_{n+1}-A_{n-1})=0\;,\label{7.37}
\end{equation}
 kjer je 
\begin{equation}
\beta_{n}=\frac{nq}{k_{3}}(k_{1}+nq)\label{7.38}
\end{equation}
 in 
\begin{equation}
\xi=\frac{\epsilon_{1}}{4\epsilon^{\prime}}\,\frac{k^{2}}{k_{3}}\;.\label{7.39}
\end{equation}
 Reševanje sistema \ref{7.37} je težavno. Rešitve poiščimo le v treh
pomembnih limitnih primerih. Naj bo amplituda vala, ki vpada z leve,
$A_{0}(0)=A_{0}$, ostale $A_{n}(0)$ pa nič.

Najprej privzemimo, da je $L\xi<<1$, da je torej $\epsilon_{1}$
majhen in debelina zvočnega snopa ne prevelika. Tedaj je pri vseh
$z$ in za pozitivne $n$ $A_{n+1}<<A_{n}$ in lahko člen $A_{n+1}$
v enačbi \ref{7.37} ispustimo. S tem zapišemo preprost sistem enačb
\begin{equation}
A_{n}^{\prime}+i\beta_{n}A_{n}=\xi A_{n}\;,\label{7.40}
\end{equation}
 ki jih lahko zapored integriramo: 
\begin{equation}
A_{n}(z)=\xi e^{-i\beta_{n}z}\int_{0}^{z}A_{n-1}(z^{\prime})
e^{i\beta_{n}z^{\prime}}dz^{\prime}\;.\label{7.41}
\end{equation}
 Podobne izraze izpeljemo za negativne $n$, to je za uklonjene valove,
ki se jim frekvenca pri sipanju zmanjša.

Poglejmo posebej prvi uklonjeni val z amplitudo $A_{1}$. Po predpostavki,
da je $A_{\pm1}<<A_{0}$, se le malo energije uklanja iz osnovnega
vala in je $A_{0}(z)$ skoraj konstanta. Potem lahko integral v \ref{7.41}
izračunamo: 
\begin{equation}
A_{1}(L)=A_{0}\xi L\,\frac{\sin\beta_{1}L/2}{\beta_{1}L/2}\, e^{-i\beta_{1}L/2}\;.\label{7.41}
\end{equation}
 $A_{1}(L)$ ima vrh pri $\beta_{1}=0$, to je pri 
\begin{equation}
\frac{q}{\cos\phi}(k\sin\phi+\frac{q}{2})=0\label{7.42}
\end{equation}
 ali 
\begin{equation}
2k\sin\phi=-q\;.\label{7.43}
\end{equation}
 Prečna komponenta valovnega vektorja sipanega vala je tedaj 
\begin{equation}
k_{1}+q=k\sin\phi-2k\sin\phi=-k\sin\phi\;.\label{7.44}
\end{equation}
 Smer sipanega vala je torej pod kotom $-\phi$ glede na os $z$,
simetrično z vpadnim valom. Če še označimo $\theta=2\phi$, vidimo,
da predstavlja $\beta_{1}=0$ ravno pogoj za Braggovo sipanje vpadnega
vala.

Razmere pri Braggovem sipanju je vredno pogledati še nekoliko podrobneje.
Ker je hitrost zvoka mnogo manjša od hitrosti svetlobe, je $\Omega/c<<q$
in sta velikosti vpadnega in sipanega valovnega vektorja enaki. Komponenti
$x$ se razlikujeta za $q$. Kadar je Braggov pogoj izpolnjen, velja
tudi da je sipani valovni vektor $\vec{k}_{s}=\vec{k}+\vec{q}$, kot
je razvidno iz slike \ref{s7.9} ali kot se lahko prepričamo s kratkim
računom. Braggov pogoj je torej enakovreden zahtevi, da se morajo
pri sipanju ohraniti valovni vektorji. V kvantni mehaniki je $\hbar\vec{k}$
gibalna količina fotona, $\hbar\vec{q}$ pa gibalna količina kvanta
zvočnega valovanja fonona. Braggov pogoj je torej primer ohranitve
gibalne količine. Če se ta ne ohranja, je sipanje neučinkovito in
le oscilira okoli majhne vrednosti, kar opisuje faktor $\sin(\xi L/2)$.

Delež moči uklonjenga vala je 
\begin{equation}
\frac{I_{1}}{I_{0}}=\left|\frac{A_{1}}{A_{0}}\right|^{2}=(\xi L)^{2}
\left(\frac{\sin\beta_{1}L/2}{\beta_{1}L/2}\right)^{2}\;.\label{7.45}
\end{equation}
 Kadar je Braggov pogoj izpolnjen, je $I_{0}/I_{1}=(\xi L)^{2}$,
kar lahko velja le, dokler je $\xi L<<1$. Da se izognem tej omejitvi,
moramo upoštevati tudi zmanjšanje moči vpadnega snopa.

Braggov pogoj je hkrati lahko izpolnjen le za en uklonjen val, na
primer $n=1$. Zato so tedaj vse ostale amplitude $A_{n}$, $n\ne0,1$
majhne in ne vplivajo na $A_{1}$. To nam omogoča drug približek,
ki je za uporabo zelo pomemben. Opustimo omejitev $L\xi<<1$, vendar
zahtevajmo, da sta le $A_{0}$ in $A_{1}$ različni od nič. Sedaj
$A_{0}(z)$ ne smemo več obravnavati kot konstante. Iz sistema \ref{7.37}
sledi 
\begin{eqnarray}
A_{0}^{\prime}+\xi A_{1} & = & 0\nonumber \\
A_{1}^{\prime}-\xi A_{0} & = & 0\;.
\end{eqnarray}
 Začetni pogoji so $A_{0}(0)=A_{0}$ in $A_{1}(0)=0$. Rešitev enačb
\ref{7.46} je 
\begin{eqnarray}
A_{0}(L) & = & A_{0}\cos\xi L\nonumber \\
A_{1}(L) & = & A_{0}\sin\xi L\;.
\end{eqnarray}
 Če je izpolnjen Braggov pogoj, se moč vpadnega vala na razdalji $\pi\xi/2$
skoraj vsa pretoči v uklonjeni snop, nato pa zopet nazaj (slika \ref{s7.10}).
Za čim bolj učinkovito delovanje akustooptičnega modulatorja seveda
želimo doseči ravno take pogoje.

Razmerje med močjo uklonjenega in vpadnega snopa je 
\begin{equation}
\frac{I_{1}}{I_{0}}=\sin^{2}\left(\frac{\pi n_{0}^{3}pS_{0}L}{2\lambda\cos\phi}\right)\;.\label{7.48}
\end{equation}
 kjer smo upoštevali, da je $\xi=\pi\epsilon_{1}/(2n_{0}\lambda\cos\phi)$
in $\epsilon_{1}=n_{0}^{4}pS_{0}$, kjer je $S_{0}$ amplituda deformacije
v zvočnem valu. Izrazimo jo lahko z gostoto moči zvočnega vala 
\begin{equation}
j_{z}=\frac{1}{2}CS_{0}^{2}v_{z}\;,\label{7.49}
\end{equation}
 kjer je $C$ elastična konstanta snovi. Iz $v_{z}^{2}=C/\rho$ izrazimo
še $C$, s čemer dobimo 
\begin{equation}
S_{0}=\sqrt{\frac{2j_{z}}{\rho v_{z}^{3}}}\;.\label{7.50}
\end{equation}
 Merilo uporabnosti neke snovi za akusto-optični modulator je 
\begin{equation}
M=\frac{n_{0}^{6}p^{2}}{\rho v_{z}^{3}}\;.\label{7.51}
\end{equation}

Poglejmo primer. V kremenu $\rho=2,2\cdot10^{3}$ kg/m$^{3}$, $v_{z}=6000$~m/s,
$n_{0}=1,46$ in $p=0,2$, tako da je $M=8\cdot10^{-16}$~W/m$^{2}$.
Pri gostoti zvočnega toka 10~W/cm$^{2}$ in valovni dolžini 633~nm
je za popolen prenos močiv uklonjeni snop $L=3$~cm. Gornja gostota
zvočnega toka je kar velika in je ni prav lahko doseči, zato so običajni
uklonski izkoristki nekaj manjši od 1.

Kot odklona uklonjenega vala $2\phi$ je 
\begin{equation}
2\phi=\frac{q}{k}=\frac{\lambda}{n_{0}\Lambda}=1,7\cdot10^{-3}\;.\label{7.52}
\end{equation}
 Kot je torej precej majhen.

Opisani račun izkoristka uklona na zvočnih valovih je uporaben tudi
pri računu izkoristka holograma. V primeru faznega holograma je račun
čisto enak in nam kaže tudi razliko med tankim in debelim hologramom,
kako pa je z izkoristkom amplitudnega holograma, kjer je modulirana
absorpcija v snovi, lahko bralec izračuna sam (Naloga).

Enačbe \ref{7.37} je enostavno rešiti še v {it Raman- Nathovem približku},
ki sicer ni posebno pomemben za uporabo, je pa zanimiv. Vpeljimo novo
neodvisno spremenljivko $\zeta=2\xi z$. Zveza \ref{7.37} preide
v 
\begin{equation}
2\frac{dA_{n}(\zeta)}{d\zeta}+A_{n+1}(\zeta)-A_{n-1}(\zeta)=\frac{\beta_{n}}{\xi}A_{n}\;.\label{7.53}
\end{equation}
 Člen na desni lahko izpustimo, če je 
\begin{equation}
\frac{\beta_{n}}{\xi}=\frac{4nq}{k}\frac{\epsilon^{\prime}}{\epsilon_{1}}(\sin\phi-\frac{nq}{2k})<<1\;,\label{7.54}
\end{equation}
 to je, če je pri danem $\epsilon_{1}$ valovna dolžina zvoka dovolj
velika v primeri z valovno dolžino svetlobe. Iz \ref{7.53} sledi
tedaj rekurzijska zveza za Besselove funkcije 
\begin{equation}
2\mbox{J}_{n}^{\prime}+\mbox{J}_{n+1}-\mbox{J}_{n-1}=0\label{7.55}
\end{equation}
 in je $A_{n}(z)=A_{0}\mbox{J}_{n}(2\xi z)$. Kadar je $2\xi L$ ničla
J$_{0}$, prvič je to pri $2\xi L=2.4$, se vsa energija ukloni iz
vpadnega snopa, vendar gre v tem primeru, ko Braggov pogoj ni izpolnjen,
v mnogo uklonjenih snopov.

\section{Modulacija s tekočimi kristali}

Tekoči kristali so anizotropne kapljevine in so vmesna faza med običajnimi
izotropnimi kapljevinami in kristali. Stopnja njihove urejenosti je
lahko različna. Tekoče kristale tvorijo podolgovate ali ploščate molekule,
vendar so za sedaj praktično pomembne le faze, ki jih tvorijo podolgovate
organske molekule. Tudi med njimi so bili do nedavnega v uporabi le
optične naprave z nematskimi tekočimi kristali, zato se zanimajmo
le zanje. Običajno jih tvorijo molekule z relativno togim jedrom iz
dveh ali treh benzenovih obročev, ki imajo na koncih krajše ali daljše
alifatske verige (slika \ref{s7.11}). Težišča molekul so v nematski
fazi neurejena, enako kot v navadni tekočini, osi molekul pa so v
povprečju urejene v določeno smer. Smer povprečne urejenosti opišemo
z enotnim vektorjem $\vec{n}$, ki mu rečemo {it direktor}. Smeri
$\vec{n}$ in $-\vec{n}$ sta enakovredni, z drugimi besedami, molekule
z enako verjetnostjo kažejo v smeri $+\vec{n}$ kot v $-\vec{n}$.
Stopnja urejenosti v mikroskopski sliki ni prav velika, povprečen
odklon molekul od $\vec{n}$ je nekaj deset stopinj.

Nematski tekoči kristal se obnaša kot enoosen dvolomni kristal z optično
osjo vzporedno z $\vec{n}$. Ker je optična polarizabilnost benzenovih
obročev vzdolž osi molekul precej večja od prečne, je razlika med
rednim in izrednim lomnim količnikom razmeroma velika, navadno med
0,1 in 0,2.

V makroskopskem vzorcu nematskega tekočega kristala se smer $\vec{n}$
po vzorcu neurejeno spreminja, če posebej ne poskrbimo, da ima povsod
isto smer. Energija takega deformiranega stanja je nekoliko večja
od energije homogenega stanja, zaradi česar na delček tekočega kristala
okolica deluje z navorom, ki deluje v smeri zmanjševanja nehonogenosti
$\vec{n}$. Temu pojavu, zna\textquotedbl{}ilnemu za tekoče kristale,
pravimo orientacijska elastičnost. Nekoliko podrobneje je opisan v
Dodatku na koncu poglavja. Vendar so ti elastični navori prešibki,
da bi uredili razsežne vzorce. Urejene vzorce, kakršne potrebujemo
za uporabo v optičnih elementih, lahko pripravimo s pomočjo zunanjega
polja ali pa v dovolj tankih plasteh, kjer mejne površine ustrezno
pripravimo.

Zunanje električno ali magnetno polje na tekoče kristale tudi deluje
z navorom. Električna in magnetna susceptibilnost nematskega tekočega
kristala nista skalarja, temveč imata dve različni lastni vrednosti,
eno v za smer vzporedno z $\vec{n}$, drugo za pravokotno. Zato je
elektrostatična energija odvisna od kota med zunanjim poljem $vec{E}$
in $\vec{n}$. Od kota odvisni del energije lahko zapišemo v obliki
\begin{equation}
w_{a}=-\frac{1}{2}\epsilon_{0}\epsilon_{a}(\mathbf{E}\cdot\vec{n})^{2}\;,\label{7.56}
\end{equation}
 kjer je $\epsilon_{a}=\epsilon_{\parallel}-\epsilon_{\perp}$ anizotropni
del dielektrične konstante. Če je $\epsilon_{a}>0$, se molekule skušajo
postaviti v smer polja, če je $\epsilon_{a}<0$, pa pravokotno na
polje.

Urejen vzorec je mogoče narediti tudi v tankih plasteh. Če površino,
ki je v stiku s tekočim kristalom, prevlečemo z primerno pastjo, se
molekule tekočega kristala tik ob povšini na določen način uredijo.
Tanka plast najlona, ki jo podrgnemo v željeni smeri, povzroči, da
je $\vec{n}$ ob površini paralelen s površino v smeri drgnenja. Drgnenje
deloma uredi verige najlona, zato se tudi molekule tekočega kristala
raje uredijo v isti smeri. Da je $\vec{n}$ pravokoten na površino,
dosežemo na primer s tanko plastjo lecitina. Ta ima polarno glavo,
ki se adsorbira na stekleno površino, in alifatsko verigo, ki stoji
približno pravokotno na površino. Zato se tudi alifatski repi molekul
tekočega kristala uredijo enako. V obeh primerih, paralelni ali pravokotni
orientaciji ob steni, se urejenost zaradi orientacijske elastičnosti
ohranja tudi stran od stene, tako da lahko brez težav naredimo urejene
vzorce debeline do kakih 200~$\mu$m. Pri večjih debelinah so elastični
navori prešibki in ostanejo v vzorcu defekti.

Struktura tekočekristalnega vzorca je tako odvisna od orientacijske
elastičnosti, od robnih pogojev, ki jih določimo z obdelavo mejne
površine, in od zunanjega električnega ali magnetnega polja.

Preprost elektro-optični modulator ali kazalnik na osnovi nematskih
tekočih kristalov lahko naredimo takole. Vzemimo vzorec tekočega kristala
med dvema stekloma, na katerih sta prozorni elektrodi. Ureditev tekočega
kristala naj bo vzporedna s površino stekel. Tudi optična os ima isto
smer. Dovolj velika napetost zasuče $\vec{n}$ in s tem optično os
pravokotno na stene, razen tik ob površini. Pri debelini 10~$\mu$m
je potrebna napetost nekaj voltov. Nekaj več o tem preklopu najde
bralec v Dodatku na koncu poglavja.

Naj debelina $h$ obravnavane plasti za izbrano valovno dolžino svetlobe
ustreza pogoju 
\begin{equation}
h(n_{i}-n_{r})=(2N+1)\frac{\lambda}{2}\;,\label{7.57}
\end{equation}
 kjer je $N$ celo število. Plast torej deluje kot ploščica $\lambda/2$.
Ker je $n_{i}-n_{r}\simeq0,1$, je potrebna debelina nekaj $\mu$m.
Vzorec damo med prekrižana polarizatorja s prepustno smerjo 45$^{o}$
na $\vec{n}$. Ploščica polarizacijo svetlobe z izbrano valovno dolžino
zasuče za 90$^{o}$, tako da gre svetlobe tudi skozi analizator. Ko
vključimo napetost, se optična os obrne, polarizacija svetlobe se
pri prehodu skozi plast ohrani in analizator je ne prepusti.

Tak preklopnik ima nekaj slabih lastnosti. Prepustnost je odvisna
od valovne dolžine svetlobe in od temperature in debelina preklopnika
mora biti povsod enaka. Zato se v praksi uporablja zasukana nematska
plast.

Obrnimo eno od stekel za 90$^{o}$, tako da sta smeri urejanja na
obeh mejah med seboj pravokotni. Tedaj se smer $\vec{n}$ v plasti
zvezno zavrti od ene do druge površine, kot kaže slika \ref{s7.13.}.
Polarizacija svetlobe, ki je ob vstopu v plast polarizirana v smeri
urejanja, skozi plast približno sledi smeri $\vec{n}$, kot bomo pokazali
nekoliko kasneje, in je ob izstopu iz plasti polarizirana pravokotno
na vpadno polarizacijo. Z električnim poljem preklopimo optično os
pravokotno na plast in polarizacija se ne zasuče. Plast med prekrižanima
polarizatorjema brez polja prepušča svetlobo, v polju pa ne. Pri tem
delovanje kazalnika ni dosti odvisno niti od debeline plasti niti
od valovne dolžine. Kadar želimo, da kazalnik deluje v odbiti svetlobi,
damo za zadnji analizator še odbojno površino.

Pri kristalnem elektro-optičnem modulatorju smo lahko dobili tudi vmesne
prepustnosti, medtem ko s tekočimi kristali na opisan način lahko
dobimo le zaprto in odprto stanje, vmesna stanja pa je zelo težko
kontrolirati.

Pokažimo še, da polarizacija v sredstvu, ki je lokalno enoosno in
se optična os suče v pravokotni smeri, v določenih pogojih res približno
sledi optični osi. Poleg zasukane nematske celice je pomemben primer
snovi s takimi lastnostmi holesteričen tekoči kristal, ki je zelo
podoben nematskim, le da se $\vec{n}$ spontano počasi suče okoli
smeri, pravokotne na $\vec{n}$.

Optična os sredstva naj leži v ravnini $xy$ in naj se enakomerno
suče, ko se premikamo vzdolž osi $z$. Kot med optično osjo in osjo
$x$ lahko zapišemo 
\begin{equation}
\phi=qz\;.\label{7.58}
\end{equation}
 Pri tem je $2\pi/q$ perioda sukanja optične osi. Zanimajmo se le
za širjenje svetlobe v smeri $z$. Tedaj potrebujemo le del dielektričnega
tenzorja v $xy$ ravnini. Njegova oblika je 
\begin{equation}
\boldsymbol{\epsilon}(z)=\left[\begin{array}{cc}
\bar{\epsilon}+\frac{1}{2}\epsilon_{a}\cos2\phi & \frac{1}{2}\epsilon_{a}\sin2\phi\\
\frac{1}{2}\epsilon_{a}\sin2\phi & \bar{\epsilon}+\frac{1}{2}\epsilon_{a}\cos2\phi
\end{array}\right]\;,\label{7.59}
\end{equation}
 kjer je 
\begin{equation}
\bar{\epsilon}=\frac{\epsilon_{\parallel}+\epsilon_{\perp}}{2}\label{7.60}
\end{equation}
 Valovna enačba za valovanje s frekvenco $\omega$ je 
\begin{equation}
\frac{d^{2}\mathbf{E}}{dz^{2}}=-\frac{\omega^{2}}{c^{2}}\boldsymbol{\epsilon}(z)\mathbf{E}\label{7.61}
\end{equation}
 ali v komponentah 
\begin{eqnarray}
-\frac{d^{2}E_{x}}{dz^{2}} & = & (\beta^{2}+\alpha^{2}\cos2qz)E_{x}+\alpha^{2}E_{y}\sin2qz\nonumber \\
-\frac{d^{2}E_{y}}{dz^{2}} & = & \alpha^{2}E_{x}\sin2qz+(\beta^{2}+\alpha^{2}\cos2qz)E_{y}\;,
\end{eqnarray}
 kjer je $\alpha^{2}=\epsilon_{a}\omega^{2}/(2c^{2})$ in $\beta^{2}=\bar{\epsilon}\omega^{2}/c^{2}$.

Ugodno je vpeljati krožni polarizaciji $E_{+}=E_{x}+iE_{y}$ in $E_{-}=E_{x}-iE_{y}$.
Enačbi \ref{7.62} preieta v 
\begin{eqnarray}
-\frac{d^{2}E_{+}}{dz^{2}}=\beta^{2}E_{+}+\alpha^{2}E_{-}e^{2iqz}\nonumber \\
-\frac{d^{2}E_{-}}{dz^{2}}=\alpha^{2}E_{+}e^{-2iqz}+\beta^{2}E_{-}\;.
\end{eqnarray}


Iščemo lastne rešitve valovne enačbe v sredstvu s periodično modulacijo
lomnega količnika. Matematično podoben problem je iskanje lastnih
funkcij elektronov v kristalu. Za te vemo, da morajo imeti Blochovo
obliko, to je, biti morajo produkt periodične funkcije s periodo kristalne
mreže in faktorja exp($ikz$). Matematiki pravijo tej trditvi Floquetov
izrek. Veljati mora tudi v našem primeru. Lastne valove torej poiščimo
v obliki 
\begin{eqnarray}
E_{+} & = & Ae^{i(k+q)z}\nonumber \\
E_{-} & = & Be^{i(k-q)z}\;.
\end{eqnarray}
 Nastavek reši sistem \ref{63}, če $A$ in $B$ rešita sistem homogenih
linearnih enačb 
\begin{eqnarray}
[(k+q)^{2}-\beta^{2}]A-\alpha^{2}B & = & 0\nonumber \\
-\alpha^{2}A+[(k-q)^{2}-\beta^{2}]B & = & 0\;.
\end{eqnarray}
 Sistem je netrivialno rešljiv, če je determinanta kkoeficientov enaka
nič: 
\begin{equation}
(k^{2}+q^{2}-\beta^{2})^{2}-4k^{2}q^{2}-\alpha^{4}=0\;.\label{7.66}
\end{equation}
 $\beta$ in $\alpha$ sta sorazmerna z $\omega$. Dobljena enačba
tako predstavlja disperzijsko relacijo - zvezo med $\omega$ in $k$
- za svetlobo v zavitem sredstvu. Prikazana je na sliki \ref{s7.14}.
Vsaki vrednosti $\omega$, razen v ozkem območju med $\omega_{-}$
in $\omega_{+}$, recimo mu frekvenčna špranja, pripadajo 4 realne
rešitve za $k$, po dve za valovanji v pozitivni in negativni smeri.
V območju špranje je en par rešitev imaginaren. Vsaki vrednosti $k$
pripada neko razmerje koeficientov $A$ in $B$, ki ga izračunamo
iz enčb \ref{7.65} in ki določa polarizacijo lastnega vala. Polarizacije
lastnih valov so v splošnem eliptične in med pri dani frekvenci med
seboj niso pravokotne, kar je posledica tega, da sistem \ref{7.65}
ne predstavlja čisto navadnega problema lastnih vektorjev simetrične
matrike. V območju frekvenčne špranje le en par rešitev predstavlja
potujoč val, drug pa polje, ki eksponento pojema v sredstvo. Zato
se svetloba s frekvenco v špranji in z ustrezno polarizacijo, ki vpada
na holesteričen tekoči kristal, totalno odbije. Pojav je povsem analogen
Braggovemu odboju na kristalih in daje holesterikom značilen obarvan
videz. Več o zanimivih podrobnostih optike holesteričnih tekočih kristalov
najde bralec v \ref{degennes} in nalogah k temu poglavju.(naloga)

Za razlago delovanja zasukane nematske celice zadošča primer, ko je
$q<<\beta$ in $\alpha$, ko je torej perioda sukanja optične osi
velika v primeri z valovno dolžino svetlobe. Tedaj lahko $q$ v disperzijski
zvezi kar zanemarimo, prvi popravek je šele reda $q^{2}$, in dobimo
\begin{equation}
k^{2}=\left\{ \begin{matrix}\beta^{2}+\alpha^{2}=\frac{\omega^{2}}{c^{2}}\epsilon_{\parallel}\end{matrix}\right.\label{7.67}
\end{equation}
 Ti vrednosti, ki ustrezata velikosti valovnega vektorja za izredni
in redni val v običajnem enoosnem kristalu, postavimo v eno od enačb
\ref{7.65} in izračunamo še polarizaciji lastnih valov: $B=\pm A$.

Izračunajmo obe kartezični komponenti električnega polja za prvo rešitev:
\begin{equation}
\begin{array}{lclclcl}
E_{x} & = & \frac{1}{2}(E_{+}+E_{-}) & = & \frac{1}{2}Ae^{ikz}(e^{iqz}+e^{-iqz}) & = & Ae^{ikz}\cos qz\\
E_{y} & = & \frac{1}{2}(E_{+}-E_{-}) & = & \frac{1}{2i}Ae^{ikz}(e^{iqz}-e^{-iqz}) & = & Ae^{ikz}\sin qz
\end{array}\;.\label{7.68}
\end{equation}
 Polarizacija torej res kar sledi optični osi. Druga rešitev da val,
ki je polariziran pravokotno na lokalno optično os in se prav tako
suče z njo. Prvi val se širi s fzano hitrostjo $c/n_{i}$, torej kot
izredni val, drugi pa s $c/n_{r}$, to je kot redni val. Če na zasukano
nematsko celico vpada svetloba, polarizirana ali paralelno ali pravokotno
na optično os ob meji, se pojavi na izhodni strani polarizacija, zasukano
za pravi kot. V primeru, da vpadna polarizacija ne sovpada z eno od
lastnih, jo moramo seveda razstaviti na obe lastni in po prehodu skozi
tekoči kristal zopet sestaviti, s čemer seveda na splošno nastane eliptično
polarizacijo.


\section{Dodatek}

Energija nematskega tekočega kristala je najnižja, kadar ima $\vec{n}$
povsod isto smer. Povečanje energije zaradi krajevne odvisnosti $\vec{n}$
je v najnižjem redu sorazmerno s $(\nabla\vec{n}(\vec{r}))^{2}$.
Najsplo\textquotedbl{}nejši izraz za {it orientacijsko elastično
energijo} dobimo tako, da tvorimo vse neodvisne člene, v katerih
nastopa $(\nabla\vec{n}(\vec{r}))^{2}$ in ki so invariantni na simetrijske
operacije nematske faze. Sledi\cite{degennes} 
\begin{equation}
F_{e}=\int\left\{ K_{1}(\nabla\cdot\vec{n})^{2}+K_{2}[\vec{n}\times(\nabla\times\vec{n})]^{2}+K_{3}[\vec{n}\cdot(\nabla\times\vec{n})]^{2}\right\} dV\label{7.70}
\end{equation}
 $K_{1}$, $K_{2}$ in $K_{3}$ so tri Frankove orientacijske elastične
konstante. Prvi člen predstavlja deformacijo v obliki pahljače, drugi
upogib, tretji pa zasuk (slika \ref{s7.20})

V zunanjem električnem polju se energija spremeni. Običajno je neodvisna
električna količina električna poljska jakost, ker je polje posledica
predpisanih napetosti na fiksnih elektrodah. Ustrezni člen v termodinamičnem
potencialu je tedaj $-\vec{D}\cdot\mathbf{E}/2$. $\mathbf{E}$ razstavimo
na komponento, paraleleno in pravokotno z $\vec{n}$. Dielektrični
tenzor ima v smeri $\vec{n}$ vrednost $\epsilon_{\parallel}$, v
pravokotni smeri pa $\epsilon_{\perp}$. Tako je 
\begin{eqnarray}
\vec{D} & = & \epsilon_{0}\underline{\epsilon}\mathbf{E}=\epsilon_{0}\epsilon_{\parallel}(\vec{n}\cdot\mathbf{E})\vec{n}+\epsilon_{0}\epsilon_{\perp}[\mathbf{E}-(\vec{n}\cdot\mathbf{E})\vec{n}]\nonumber \\
 & = & \epsilon_{0}\epsilon_{a}(\vec{n}\cdot\mathbf{E})\vec{n}-\epsilon_{0}\epsilon_{\perp}\mathbf{E}\;,
\end{eqnarray}
 kjer je $\epsilon_{a}=\epsilon_{\parallel}-\epsilon_{\perp}$. Drugi
člen je neodvisen od $\vec{n}$, zato ga lahko izpustimo. Prosta energija
nematičnega tekočega kristala v električnem polju je tako 
\begin{equation}
F=F_{0}+F_{e}-\frac{1}{2}(\vec{n}\cdot\mathbf{E})^{2}\;.\label{7.72}
\end{equation}
 $F_{0}$ predstavlja del proste energije, ki je neodvisen od $\vec{n}$
in $\mathbf{E}$. V ravnovesju je prosta energija najmanjša. Kadar je
$\epsilon_{a}>0$, se zato skuša $\vec{n}$ postaviti vzporedno s
poljem. Da lahko z minimizacijo $F$ izrazimo $\vec{n}(\vec{r})$, moramo
poznati še robne pogoje.

Poglejmo primer. Naj bo nematski kristal med dvema paralelnima steklenima
ploščama v razmiku $h$. Na obeh ploščah naj bo $\vec{n}$ vzporeden
s površino v isti smeri. Brez zunanjega polja je $\vec{n}$ povsod
enako usmerjen. V dovolj velike zunanjem polju, pravokotnem na plošči,
naj bo to smer $z$, dobi $\vec{n}$ komponento v smeri $z$: 
\begin{equation}
\vec{n}(z)=(n_{1}(z),0,n_{3}(z))\;.\label{7.73}
\end{equation}
 Deformacija $n_{3}$ naj bo majhna. Tedaj je $n_{1}=1n_{3}^{2}/2$.
Robna pogoja sta $n_{3}(0)=n_{3}(h)=0$. Približno rešitev, ki ustreza
robnima pogojema, poiščemo z nastavkom 
\begin{equation}
n_{3}(z)=a\sin qz\;,\mbox{\hskip1cm}q=\frac{\pi}{h}\;.\label{7.74}
\end{equation}
 Ta nastavek je prvi člen razvoja prave rešitve v Fourierovo vrsto.
Velja 
\begin{equation}
\nabla\times\vec{n}=(0,-\frac{dn_{1}}{dz},0)\label{7.75}
\end{equation}
 in 
\begin{equation}
\vec{n}\times(\nabla\times\vec{n})=(n_{3}\frac{dn_{1}}{dz},0,n_{1}\frac{dn_{3}}{dz})\simeq(0,0,n_{3}\frac{dn_{3}}{dz})\label{7.76}
\end{equation}
 do drugega reda v $n_{3}$. Prosta energija je tako 
\begin{eqnarray}
F & = & \frac{1}{2}\int\left[K_{1}\left(\frac{dn_{3}}{dz}\right)^{2}+K_{2}n_{3}^{2}\left(\frac{dn_{3}}{dz}\right)^{2}-\epsilon_{0}\epsilon_{a}(n_{3}E)^{2}\right]dz=\nonumber \\
 & = & \frac{1}{2}\int_{0}^{h}[K_{1}q^{2}a^{2}\cos^{2}qz+K_{3}q^{2}a^{4}\sin^{2}qz\cos^{2}qz-\epsilon_{0}\epsilon_{a}Ea^{2}\sin^{2}qz]dz=\nonumber \\
 & = & \frac{\pi}{4q}[K_{1}q^{2}a^{2}+\frac{1}{4}K_{3}q^{2}a^{4}-\epsilon_{0}\epsilon_{a}Ea^{2}]\;.
\end{eqnarray}
 Iščemo amplitudo deformacije $a$, pri kateri je prosta energija
najmanjša. Tedaj mora biti $a$ rešitev enačbe 
\begin{equation}
(K_{1}q^{2}-\epsilon_{0}\epsilon_{a}E)a+\frac{1}{2}K_{3}q^{2}a^{3}=0\;.\label{7.78}
\end{equation}
 Rešitvi sta 
\begin{equation}
a=0\mbox{\hskip1cm in \hskip1cm}a^{2}=2\frac{\epsilon_{0}\epsilon_{a}E-K_{1}q^{2}}{K_{3}q^{2}}\;.\label{7.79}
\end{equation}
 Pri majhnih poljih, ko je $\epsilon_{0}\epsilon_{a}E<K_{1}q^{2}$,
je fizikalno smiselna le prva rešitev, ko defomacije ni. Pri večjih
poljih pa je stabilna druga rešitev. Ko večamo polje, deformacija
v sredini plasti hitro naraste, tako da se $\vec{n}$ postavi skoraj
popolnoma v smer zunanjega polja. Tedaj naša rešitev seveda ni dobra,
saj smo računali, kot da je $n_{3}<<1$. Prehodu iz nedefromiranega
stanja v deformirano stanje pravimo tudi Fredericksov prehod. Na njem
je osnovano preklaplanje optičnih kazalnikov na nematske tekoče kristale.

V primeru zasukane nematske celice je prehod podoben, le račun je
nekoliko bolj zapleten, ker ima deformacija vse tri komponente, tudi
zasuk (Naloga).