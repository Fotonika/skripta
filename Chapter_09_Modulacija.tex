\chapterimage{AOModulator.jpg} % Chapter heading image

\chapter{Modulacija svetlobe}

V optičnih napravah pogosto želimo spreminjati lastnosti svetlobnega
valovanja. Tak primer smo že spoznali pri obravnavi laserja, kjer za 
preklop dobrote potrebujemo element, ki hitro spreminja prepustnost. 
Še pomembnejša je modulacija valovanja pri optičnem prenosu informacij.

Svetlobno valovanje lahko moduliramo na več načinov. Z ustreznim moduliranjem
lomnega količnika lahko valovanju spreminjamo amplitudo\index{Elektro-optična modulacija!amplitudna} 
ali frekvenco oziroma fazo\index{Elektro-optična modulacija!frekvenčna}
\index{Elektro-optična modulacija!fazna}. 
\begin{figure}[h]
\centering
\def\svgwidth{140truemm} 
\input{slike/09_AMFM.pdf_tex}
\caption{Amplitudno (levo) in fazno oziroma frekvenčno moduliran signal (desno)
}
\label{fig:amfm}
\end{figure}

Delovanje optičnih modulatorjev temelji na različnih pojavih. V tem poglavju bomo 
podrobneje spoznali dva načina, to sta elektro-optični in elasto- oziroma akusto-optični pojav. 
Pri prvem pride do spremembe lomnega količnika snovi pod vplivom zunanjega električnega polja, 
pri drugem pa zaradi mehanske deformacije. Kadar mehansko deformacijo povzroči zvočno valovanje, 
takim modulatorjem pravimo akusto-optični. Na koncu bomo spoznali poseben zelo pomemben 
primer elektro-optičnih modulatorjev na osnovi tekočih kristalov.

\section{Elektro-optični pojav}
Elektro-optični pojav\index{Elektro-optični pojav} opisuje spremembe optičnih lastnosti 
snovi (dielektričnosti in lomnega količnika) pod vplivom zunanjega električnega polja. 
Omejimo se na statično zunanje polje oziroma
na polje, katerega frekvenca je bistveno manjša od optične frekvence. Omejitev na nizko 
frekvenco je potrebna zato, da optično polje še lahko obravnavamo linearno. 
Kako je v nasprotnem primeru, ko je frekvenca polja primerljiva z optično frekvenco, 
smo na široko obravnavali v poglavju o nelinearni optiki (poglavje~\ref{chap:NLO}).

Namesto dielektričnega tenzorja navadno vpeljemo inverzni dielektrični tenzor
\beq
\underline{b}=\underline{\epsilon}^{-1}.
\eeq
Izračunajmo zvezo med spremembo inverznega tenzorja $\delta b_{ij}$ 
in spremembo dielektričnega tenzorja $\delta \varepsilon_{ij}$. 
Če so spremembe majhne, velja 
\begin{equation}
\underline{\varepsilon} = \underline{\tilde{\varepsilon}} + \delta \underline{\varepsilon}=
(\underline{b}+\delta \underline{b})^{-1}=\left(\underline{b}(1+\underline{b}^{-1}
\delta \underline{b})\right)^{-1}=(1+\underline{b}^{-1}\delta \underline{b})^{-1}\underline{b}^{-1}
\approx \underline{b}^{-1}-\underline{b}^{-1}\delta \underline{b}\, \underline{b}^{-1}.
\label{7.2}
\end{equation}
Sprememba dielektričnega tenzorja je tako
\beq
 \delta \underline{\varepsilon}= -\underline{b}^{-1}\delta \underline{b}\, \underline{b}^{-1}
 = -\underline{\tilde{\varepsilon}}\, \delta \underline{b}\, \underline{\tilde{\varepsilon}}.
\eeq
Če je nemoten dielektrični tenzor $\underline{\tilde{\varepsilon}}$ diagonalen, velja
\begin{equation}
\delta\epsilon_{ij}=-\tilde{\epsilon}_{ik}\delta b_{kl}\tilde{\epsilon}_{lj}
=-\tilde{\epsilon}_{ii}\tilde{\epsilon}_{jj}\delta b_{ij}.
\label{7.3}
\end{equation}

Pri elektro-optičnem pojavu so spremembe tenzorja dielektričnosti zaradi vpliva zunanjega polja razmeroma 
majhne. Spremembo komponente $\delta b_{ij}$ lahko zato zapišemo kot potenčno vrsto zunanjega polja $E$, 
pri čemer upoštevajmo zgolj prva dva člena v razvoju
\boxeq{7.1}{
\delta b_{ij}=r_{ijk}E_{k}+q_{ijkl}E_{k}E_{l}.
}
Prvi člen, linearno sorazmeren zunanjem polju, opisuje linearni elektro-optični
ali Pockelsov\index{Pockelsov pojav} pojav\footnote{Nemški fizik Friedrich Carl Alwin Pockels, 1865--1913.}. 
Tenzor tretjega ranga $r_{ijk}$, ki je lastnost snovi, imenujemo elektro-optični 
tenzor\index{Elektro-optični tenzor}
ali tudi Pockelsov tenzor\index{Pockelsov tenzor|see Elektro-optični tenzor}. 
Pockelsov tenzor je različen od nič v snoveh brez centra inverzije, značilne vrednosti Pockelsovega
tenzorja pa so okoli $r \sim 10^{-12} - 10^{-10}$~m/V,

Kvadratnemu elektro-optičnemu pojavu pravimo Kerrov\index{Kerrov pojav}
pojav\footnote{Škotski fizik John Kerr, 1824--1907.}, tenzorju $q_{ijkl}$ pa Kerrov tenzor\index{Kerrov tenzor}. 
Kerrov pojav je praviloma precej šibkejši od Pockelsovega, vendar je različen od nič v vseh snoveh, ne glede na
njihove simetrijske lastnosti, torej tudi v tekočinah. 
Značilna vrednost Kerrovega tenzorja je $q \sim 10^{-24}$~m$^2$/V$^2$. Navadno ločimo dva primera Kerrovega
pojava: Kerrov elektro-optični pojav pri zunanjih poljih z nizko frekvenco, in optični Kerrov pojav, ki smo 
ga podrobneje spoznali pri obravnavi nelinearnih optičnih pojavov (poglavje~\ref{OKP}).

Za uporabo trdnih kristalov je pomemben
predvsem linearni člen, zato se bomo osredotočili le nanj in zapisali
\boxeq{eq:Pockels}{
\delta b_{ij}=r_{ijk}E_{k}. 
}

\subsection*{Elektro-optični ali Pockelsov tenzor}
Simetrija snovi pomembno vpliva na obliko tenzorjev, ki opisujejo njene lastnosti.
Pockelsov tenzor $r$ je tenzor tretjega ranga, zato je lahko različen
od nič le v kristalih brez centra inverzije. 
Simetrija kristala tudi v primeru, ko ni centra inverzije, močno
zmanjša število neodvisnih komponent $r_{ijk}$. 

Ker je inverzni dielektrični tenzor $b$ simetričen, je v prvih dveh indeksih simetričen
tudi Pockelsov tenzor
\beq
r_{ijk} = r_{jik}.
\eeq
V najmanj simetričnem primeru triklinskega kristala ima tako namesto 27 zgolj 
18 neodvisnih komponent, v kristalih z višjo simetrijo pa še manj. 

Podobno kot pri nelinearni susceptibilnosti (poglavje~\ref{Chap:Chi}) 
tudi elektro-optični tenzor pogosto zapišemo le z dvema komponentama. 
Prva dva indeksa, v katerih je $r_{ijk}$ simetričen, združimo
v enega z vrednostmi od 1 do 6 po dogovoru $xx=1$, $yy=2$, $zz=3$,
$yz=4$, $zx=5$ in $xy=6$. Tako postane $r_{ijk}$ matrika velikosti
$6\times3$, simetrični tenzor drugega ranga $b_{ij}$ pa šestdimenzionalen
vektor.

\begin{definition}
Naj bo $Q$ transformacijska matrika za dano simetrijsko operacijo. Potem za tenzorje
tretjega ranga velja
\beq
r_{ijk} = Q_{ip}Q_{jq}Q_{kr}r_{pqr}.
\eeq
Zapiši transformacijsko matriko $Q$ za vrtenje okoli osi $z$ za $\pi/2$ in pokaži, da so
v primeru štirištevne simetrije od nič različne le komponente $r_{xxz}=r_{yyz}, r_{zzz}, 
r_{yzx}=-r_{xzy}$ in $r_{xzx}=r_{yzy}$. Razmisli in izračunaj, kakšen bi bil tenzor $r$, če bi 
štirištevni simetriji dodali še zrcaljenje čez ravnino $xy$.
\end{definition}

Nekaj primerov Pockelsovih tenzorjev pri različnih kristalnih simetrijah
je podanih v tabeli~(\ref{table:Pockels}).

\begin{table}[h!]
 \centering
\begin{tabular}{|c|c|c|c|} \hline  
      Kristal & Grupa & Neničelne komponente tenzorja $r$ & Vrednost ($10^{-12}$~m/V)\\ \hline
      BaTiO$_3$\index{BaTiO$_3$} & 4mm & $r_{xzx} = r_{yzy} = r_{zxx} = r_{zyy} = 
      r_{51} = r_{42}$  &
	    (pri 1,55~$\mu$m) $r_{51} = 800$ \\
	      & & $r_{xxz} = r_{yyz} = r_{13} = r_{23}$ &  $r_{13} = 8$ \\
	      & & $r_{zzz} = r_{33}$ & $r_{33} = 28$ \\ \hline
      KDP\index{KDP} & 
      $\overline{4}$2m & $r_{yzx} = r_{zyx} = r_{xzy} = r_{zxy} = r_{41} = r_{52}$  &
	    $r_{41} = 8,77$ \\
	    & & $r_{xyz} = r_{yxz} = r_{63}$ &  $r_{63} = -10,3$ \\ \hline
      GaAs\index{GaAs}\index{ZnTe} &  $\overline{4}$3m&
	  $r_{yzx} = r_{zyx} = r_{xzy} = r_{zxy} = r_{xyz} = r_{yxz}$  & (pri 10,6~$\mu$m) $r_{41} = 1,5$ \\
	ZnTe  & &   $= r_{41} = r_{52}=r_{63}$  &(pri 3,4~$\mu$m) $r_{41} = 4,2$ 
	    \\ \hline
      LiNbO$_3$\index{LiNbO$_3$} & 3m & $r_{xzx} = r_{zxx} = r_{yzy} = r_{zyy} = r_{51} = r_{42}$  &
	    $r_{51} = 32,6$ \\
	     & & $r_{xxz} = r_{yyz} = r_{13} = r_{23}$ &  $r_{13} = 9,6$ \\
	      & & $r_{zzz} = r_{33}$ & $r_{33} = 30,9$ \\
	    & &  $r_{yyy} = - r_{xxy} = -r_{xyx} = -r_{yxx}  = $ & \\
	    & &  $=r_{22} =  -r_{12} =-r_{61} $  &
	    $r_{22}  = 6,8$ \\
\hline 
\end{tabular}
  \caption{Koeficienti Pockelsovega tenzorja za nekaj izbranih snovi. Če ni navedeno drugače, veljajo
  vrednosti pri valovni dolžini okoli 600~nm.}
\label{table:Pockels}
\end{table}

\begin{remark}
Komponente elektro-optičnega tenzorja zaradi nazornosti pogosto ponazarjamo grafično. V matriki $6\times 3$
s piko označimo komponente, ki so enake nič, s polnim krožcem neničelne komponente, povezava med 
komponentami pomeni njihovo enakost, prazen krožec in črtkana črta pa označujeta 
neničelno komponento nasprotnega predznaka. Kot primer sta podana prikaza tenzorjev za 
GaAs (levo) in LiNbO$_3$ (desno).
\begin{figure}[h!]
\centering
\def\svgwidth{20truemm} 
\input{slike/09_tensor.pdf_tex}\qquad \qquad
\def\svgwidth{20truemm} 
\input{slike/09_tensor2.pdf_tex}
\end{figure}
\end{remark}

\section{Longitudinalna modulacija}
Poglejmo podrobneje, kako električno polje spremeni optične lastnosti 
elektro-optičnega kristala in kako to vpliva na svetlobo, ki potuje skozi tak kristal.
Navadno se uporabljajo kristali, ki so dvolomni že brez zunanjega polja. 
Kot primer vzemimo kristal KH$_{2}$PO$_{4}$ (KDP)\index{KDP}, ki ima tetragonalno 
simetrijo ($\bar{4}2m$). Kot razberemo iz tabele~(\ref{table:Pockels}) ima 
elektro-optični tenzor dve neodvisni komponenti: $r_{41} = r_{52}=8,77 \times 10^{-12}$~m/V
in $r_{63}= -10,3 \times 10^{-12}$~m/V.

Kristal naj bo odrezan po kristalografskih oseh, svetloba naj skozi kristal potuje 
v smeri optične osi, to je smeri $z$, v isti smeri pa na kristal priključimo
polje $E_z$. Ker je smer električnega polja vzporedna s smerjo širjenja svetlobe, taki 
postavitvi pravimo longitudinalna in pojavu longitudinalna 
modulacija.\index{Elektro-optična modulacija!longitudinalna} 
\begin{figure}[h]
\centering
\def\svgwidth{80truemm} 
\input{slike/09_AMshema.pdf_tex}
\caption{Shema longitudinalne modulacije signala. Ker je polje priključeno v smeri
potovanja svetlobe, morata biti elektrodi transparentni. Z uporabo polarizatorja in 
analizatorja sestavimo amplitudni modulator (glej poglavje~\ref{chap:ampmod}).}
\label{fig:amshema}
\end{figure}

Inverzni tenzor dielektričnosti v odsotnosti zunanjega polja zapišemo kot
\beq
\underline{\tilde{b}} = 
\left[\begin{array}{ccc}
1/n_o^2 & 0& 0\\
0 & 1/n_o^2& 0\\
0 & 0&  1/n_e^2
\end{array}\right],
\label{7.8}
\eeq
pri čemer sta $n_o$ in $n_e$ redni in izredni lomni količnik. Ko priključimo 
polje, se tenzor dielektričnosti spremeni zaradi Pockelsovega pojava. Sprememba
inverznega tenzorja dielektričnosti je po enačbi~(\ref{eq:Pockels})
\begin{align}
\delta b_{xx} & =r_{xxx}E_x + r_{xxy}E_y + r_{xxz}E_z = 0,\nonumber \\
\delta b_{xy} & = \delta b_{yx} = r_{xyx}E_x + r_{xyy}E_y + r_{xyz}E_z = r_{63}E_z,\nonumber\\
\delta b_{xz} & = \delta b_{zx} =r_{xzz}E_z = 0,\nonumber\\
\delta b_{yy} & =r_{yyz}E_z = 0,\nonumber\\
\delta b_{yz} & = \delta b_{zy} =r_{yzz}E_z = 0,\nonumber\\
\delta b_{zz} & =r_{zzz}E_z = 0.
\end{align}
Vidimo, da je večina členov enaka nič, se pa zaradi zunanjega električnega
polja v smeri $z$ pojavi izvendiagonalna komponenta 
\beq
\underline{b} = 
\left[\begin{array}{ccc}
1/n_o^2 & 0& 0\\
0 & 1/n_o^2 & 0\\
0 & 0& 1/n_e^2
\end{array}\right] + \left[\begin{array}{ccc}
 0& r_{63}E_z& 0\\
r_{63}E_z & 0 & 0\\
0 & 0&  0
\end{array}\right] = \left[\begin{array}{ccc}
1/n_o^2 & r_{63}E_z& 0\\
r_{63}E_z& 1/n_o^2 & 0\\
0 & 0&  1/n_e^2
\end{array}\right].
\label{7.8a}
\eeq
Če želimo izračunati, kako se po kristalu pod napetostjo širi vpadni svetlobni
snop, moramo gornji dielektrični tenzor diagonalizirati. Lastne vrednosti novega tenzorja
in pripadajoče nove lastne osi so
\begin{align}
\lambda_1 &= \frac{1}{n_o^2}+ r_{63}E_z \quad \mathrm{in} \quad \mathbf{e}_1 = \frac{1}{\sqrt{2}}(1,1,0)\\
\lambda_2 &= \frac{1}{n_o^2}- r_{63}E_z \quad \mathrm{in} \quad \mathbf{e}_2 = \frac{1}{\sqrt{2}}(-1,1,0)\\
\lambda_3 &= \frac{1}{n_e^2} \quad \mathrm{in} \quad \mathbf{e}_3 = (0,0,1).
\end{align}
Vidimo, da so nove lastne osi zasukane za kot $45~^\circ$ glede na prvotne osi sistema.
V novem koordinatnem sistemu je inverzni dielektrični tenzor diagonalen in enak
\beq
\underline{b} = 
\left[\begin{array}{ccc}
1/n_o^2 + r_{63}E_z& 0& 0\\
0 & 1/n_o^2 - r_{63}E_z& 0\\
0 & 0& 1/n_e^2
\end{array}\right].
\eeq
\begin{figure}[h]
\centering
\def\svgwidth{60truemm} 
\input{slike/09_AMindikatrisa.pdf_tex}
\caption{Optično enoosni kristal postane pod napetostjo dvoosen. Indikatrisa, ki je pravokotno
na optično os brez polja krožnica, se pod vplivom napetosti spremeni v elipso. }
\label{fig:amn}
\end{figure}

Spomnimo se, da potuje svetloba skozi kristal vzdolž osi $z$. Brez zunanjega električnega
polja je kristal enoosen z optično osjo v smeri $z$. Lomni količnik je torej neodvisen od
polarizacije vpadnega valovanja in je enak $n_o$. Ko priključimo polje, postane kristal
optično dvoosen, saj so vse tri lastne vrednosti tenzorja dielektričnosti različne. Za žarek, 
ki potuje vzdolž osi $z$, torej obstajata dve lastni smeri $x'$ in $y'$ z ustreznima
novima lastnima količnikoma, ki ju izrazimo kot
\beq
\frac{1}{n_{x'}^2} = \frac{1}{n_o^2}+ r_{63}E_z \quad \mathrm{in} \quad 
\frac{1}{n_{y'}^2} = \frac{1}{n_o^2}- r_{63}E_z. 
\eeq
Kadar polarizacija vpadnega valovanja ne sovpada z novimi lastnimi osmi $x'$ ali $y'$, je 
svetloba po preletu kristala v splošnem eliptično polarizirana.

Za vsa eksperimentalno dosegljiva polja velja, da je $rE\ll1/n^2$, 
zato lahko gornja izraza razvijemo za majhne popravke
\beq
n_{x'} = \sqrt{\frac{n_o^2}{1+ n_o^2 r_{63}E_z}} \approx n_o \sqrt{1- n_o^2 r_{63}E_z}.
\eeq
Sledi
\boxeq{EOnx}{
n_{x'}\approx n_o - \frac{1}{2}n_o^3 r_{63}E_z.
}
Podobno izpeljemo še za drugo lastno vrednost
\boxeq{EOny}{
n_{y'}\approx n_o + \frac{1}{2}n_o^3 r_{63}E_z.
}
Različni lastni polarizaciji potujeta vzdolž osi $z$ z različnima hitrostma. Ko 
prepotujeta dolžino kristala $L$, pride med njima do fazne razlike
\beq
\Delta \phi = k_0 n_{y'} L - k_0 n_{x'} L = \frac{\omega}{c_0}L 
n_o^3 r_{63}E_z.
\label{phiAM}
\eeq
Prelet kristala torej doda vpadnemu valovanju fazni zamik, ki je odvisen od električne poljske
jakosti $E_z$. 

Vpeljemo še karakteristično napetost $U_\pi$, pri kateri je dodatna \index{$\pi$-napetost}
fazna razlika enaka $\pi$ in kristal deluje kot ploščica $\lambda/2$\index{Ploščica $\lambda/2$}
\beq
U_\pi = \frac{\pi c_o}{\omega n_o^3 r_{63}} = \frac{\lambda}{2 n_o^3 r_{63}}.
\eeq
Za kristal KDP je $\pi$-vrednost napetosti pri valovni dolžini $633$~nm okoli  $9000$~V. 
Izračunana napetost je precej velika. Velike delovne napetosti
so značilne za kristalne elektro-optične modulatorje in so njihova
glavna pomanjkljivost. 

\section{Transverzalna modulacija}
\index{Elektro-optična modulacija!transverzalna}
Iz praktičnih razlogov je navadno preprosteje priključiti električno polje v smeri, ki 
je pravokotna na smer širjenja svetlobe. Taki postavitvi pravimo transverzalna in pojavu
transverzalna modulacija\index{Elektro-optična modulacija!transverzalna}.

Tudi to postavitev obravnavajmo na primeru. Za zgled vzemimo kristal LiNbO$_3$, ki 
ima trigonalno simetrijo (3m) in po tabeli~(\ref{table:Pockels}) štiri 
neodvisne komponente: $r_{51}=r_{42}, r_{13}=r_{23}, r_{33}$ in $r_{22}=-r_{12}=-r_{61}$.

\begin{figure}[h]
\centering
\def\svgwidth{80truemm} 
\input{slike/09_TMshema.pdf_tex}
\caption{Shema transverzalne modulacije signala}
\label{fig:tmshema}
\end{figure}
\pagebreak
Naj se svetloba širi vzdolž osi $z$, ki je hkrati tudi optična os, 
električno polje pa priključimo v smeri $y$ (slika~\ref{fig:tmshema}). 
Krajši račun pokaže, da je inverzni dielektrični tenzor pod vplivom polja enak
\beq
\underline{b} = 
 \left[\begin{array}{ccc}
1/n_o^2  -r_{22}E_y& 0& 0\\
0& 1/n_o^2+r_{22}E_y& r_{51}E_y\\
0 & r_{51}E_y&  1/n_e^2
\end{array}\right].
\label{7.8b}
\eeq
Tudi v tem primeru tenzor diagonaliziramo in poiščemo nove lastne vrednosti.
Ob privzetku, da je sprememba zaradi električnega polja majhna ($rE\ll1$),
so nove lastne vrednosti enake
\begin{align}
\lambda_1 &\approx \frac{1}{n_o^2}-r_{22}E_y \\
\lambda_2 &\approx \frac{1}{n_o^2}+ r_{22}E_y \\
\lambda_3 &\approx \frac{1}{n_e^2},
\end{align}
kar ustreza lomnim količnikom 
\begin{align}
n_{x'} &\approx n_o(1+\frac{1}{2}n_o^2r_{22}E_y)\\
n_{y'} &\approx n_o(1-\frac{1}{2}n_o^2r_{22}E_y)\\
n_z' &\approx n_e.
\end{align}
Kako pa je z novimi lastnimi osmi? Hitro ugotovimo, da se tudi pri 
priključenem polju os $x$ ohranja. Pojavi se torej zasuk okoli osi $x$,
ki ga označimo s kotom $\vartheta$. Račun pokaže, da je za smiselne
vrednosti električnega polja ta kot zelo majhen ($\vartheta~
\approx~r_{51}E_y/(1/n_o^2-1/n_e^2) \sim~1$~mrad),
tako da lahko v približku rečemo, da se lastne osi ohranjajo. 

Če potuje svetloba vzdolž osi $z$, sta torej lomna količnika za 
polarizaciji v smeri $x$ in $y$ približno enaka $n_{x'}$ in $n_{y'}$, fazna razlika med 
polarizacijama po preletu kristala z dolžino $L$ pa je 
\beq
\Delta \phi = k_0 n_{y'} L - k_0 n_{x'} L = \frac{\omega}{c_0}L 
n_o^3 r_{22}E_y.
\label{fazaTM}
\eeq
Karakteristična $\pi$-napetost \index{$\pi$-napetost}je tako
\beq
U_\pi = \frac{\lambda d}{2 Ln_o^3 r_{22}},
\eeq
pri čemer moramo ločiti med $L$, ki je dolžina kristala v smeri $z$, in $d$, ki je  
širina v prečni smeri v kateri priključimo napetost. 
Za izbran kristal ($d=5$~mm, $L=1$~cm) je $\pi$-vrednost 
napetosti pri valovni dolžini $633$~nm okoli $2000$~V. 

\begin{remark}
Transverzalno modulacijo lahko dosežemo tudi tako, da se žarek širi vzdolž 
osi $y$, električno polje pa priključimo vzdolž optične osi $z$.
V tem primeru se lastne osi ohranijo in kristal ostane optično enoosen. Vendar 
pa ima tudi ta rešitev določene slabosti. Ker je kristal že sam po sebi dvolomen, 
povzroči zunanje polje le majhno dodatno fazno razliko, zato je najbolje, če je dolžina 
kristala taka, da velja $k_{0}L(n_{o}-n_{e})=2N\pi$. Pri tem pa nastopi težava. 
Pogoj je lahko zaradi temperaturnega raztezanja in odvisnosti lomnih količnikov od temperature
izpolnjen le pri eni temperaturi, poleg tega se mora svetloba širiti natančno v smeri $y$.
Zato dvolomnost nemotenega kristala navadno kompenziramo, tako da postavimo 
dva enako dolga kristala zapored, pri čemer sta optični
osi med seboj pravokotni, modulacijska napetost na drugem kristalu pa ima
nasproten predznak. Tedaj se fazna razlika med obema polarizacijama zaradi 
naravne dvolomnosti odšteje, zaradi modulacijske napetosti pa sešteje.
\end{remark}

\section{Amplitudna modulacija}
\label{chap:ampmod}
\index{Elektro-optična modulacija!amplitudna}
Poglejmo, kako lahko elektro-optični pojav izkoristimo za modulacijo
amplitude svetlobnega snopa. Pod vplivom polja pride v kristalu do
faznega zamika med polarizacijama, ki je sorazmeren napetosti 
(enačbi~\ref{phiAM} in \ref{fazaTM}).
Če za tak kristal postavimo analizator, lahko z napetostjo spreminjamo 
prepuščeno moč svetlobe -- dobili smo amplitudni modulator.

Vrnimo se k prvemu primeru longitudinalne\index{Elektro-optična modulacija!longitudinalna}
modulacije (slika~\ref{fig:amshema}).
Naj bo vpadna električna poljska jakost $E_0$ polarizirana v smeri $y$. 
Ko priključimo napetost, os $y$ ni več lastna os, ampak sta lastni osi zasukani 
za kot $45~^\circ$ glede na prvotni lastni osi (slika~\ref{fig:amn}). Vpadno 
valovanje razstavimo na komponenti $x'$ in $y'$
\beq
\mathbf{E}_0 = E_0\, \mathbf{e}_y = \frac{E_0}{\sqrt{2}}\left(\mathbf{e}_{x'} + \mathbf{e}_{y'}\right).
\eeq
Po prehodu skozi kristal pride med njima do fazne razlike $\Delta \phi$ 
(enačba~\ref{phiAM}), zato je polje $\mathbf{E}_1$ ob izstopu iz kristala
\beq
\mathbf{E}_1 = \frac{E_0}{\sqrt{2}}\left(e^{ik_0 n_{x'}L}\mathbf{e}_{x'} + e^{ik_0 n_{y'}L}\mathbf{e}_{y'}\right)
= \frac{E_0}{\sqrt{2}}e^{ik_0 n_{x'}L}\left(\mathbf{e}_{x'} + 
e^{i\Delta\phi}\mathbf{e}_{y'}\right).
\eeq
Analizator na izhodni strani je obrnjen v smeri $x$, to je pravokotno
na smer vpadne polarizacije, in prepusti le projekcijo obeh lastnih polarizacij
na os $x$
\begin{equation}
\mathbf{E}_2= \mathbf{E}_1 \cdot \mathbf{e}_x = 
\frac{E_0}{\sqrt{2}}e^{ik_0 n_{x'}L}
\left(\frac{1}{\sqrt{2}} -\frac{1}{\sqrt{2}} e^{i\Delta\phi}\right)\mathbf{e}_x .
\label{7.16}
\end{equation}
Gostota prepuščenega svetlobnega toka ob vpadnem toku $j_0$ je tako 
\begin{equation}
j=\frac{1}{4}j_{0}\left|1-e^{i\Delta\phi}\right|^{2}=\frac{1}{2}j_{0}(1-\cos\Delta\phi).
\label{7.17}
\end{equation}
Preoblikujemo izraz in zapišemo prepustnost takega modulatorja ob upoštevanju 
enačbe~(\ref{phiAM})
\boxeq{AMfinal}{
T = \frac{j}{j_0} = \sin\left(\frac{\Delta\phi}{2}\right)^2 = 
\sin\left(\frac{\pi n_o^3 r_{63}U}{\lambda}\right)^2 .
}
\begin{figure}[h]
\centering
\def\svgwidth{90truemm} 
\input{slike/09_AMT.pdf_tex}
\caption{Prepustnost amplitudnega modulatorja v odvisnosti od faznega zamika $\Delta \phi$, 
ki je sorazmeren priključeni napetosti $U$. Če pred vzorec dodamo ploščico $\lambda/4$, 
se pojavi stalni fazni zamik in odvisnost prepustnosti od priključene napetosti
je približno  linearna.}
\label{fig:amt}
\end{figure}

Ko je napetost na kristalu enaka nič, je $\Delta \phi=0$ in tudi intenziteta 
prepuščene svetlobe $j=0$. To je pričakovano, saj sta analizator in polarizator prekrižana, 
vpadni žarek pa se širi vzdolž lastne osi kristala.
Prepustnost doseže največjo vrednost, ko je $\Delta \phi=\pi$, kar je ravno pri 
$\pi$-napetosti\index{$\pi$-napetost}. Ko torej napetost povečamo z 0 na $U_\pi$, se
prepustnost modulatorja spremeni z 0 na 1 (slika~\ref{fig:amt}).

Pogosto želimo, da je zveza med modulacijsko napetostjo in izhodno
gostoto toka linearna. To lahko dosežemo, če modulator deluje v okolici $\Delta\phi=\pi/2$
(slika~\ref{fig:amt}).
Ena rešitev bi bila dodati stalno visoko napetost, signal pa modulirati okoli
te vrednosti. Precej bolj praktična rešitev je, da med polarizator
in kristal dodamo ploščico $\lambda/4$\index{Ploščica $\lambda/4$}, ki da zahtevan stalni
fazni premik med rednim in izrednim valom. Potem lahko z razmeroma majhno napetostjo
naredimo linearen amplitudni modulator\index{Elektro-optična modulacija!linearna}.

\section{Fazna in frekvenčna modulacija}

Amplitudno modulacijo svetlobe smo dobili tako, da smo z zunanjim
poljem spremenili fazi lastnih valov, zaradi česar je postalo linearno
polarizirano vpadno valovanje po prehodu kristala eliptično polarizirano.
Spremembo polarizacije smo z analizatorjem prevedli v spremembo amplitude.

Včasih pa želimo modulirati fazo vpadne svetlobe. To naredimo tako,
da odstranimo izhodni polarizator, vhodno polarizacija pa izberemo tako, da je vzporedna 
eni od novih lastnih osi kristala. 

Če se vrnemo k primeru longitudinalnega modulatorja (slika~\ref{fig:amshema}), 
ki je za fazni modulator brez analizatorja, izberemo polarizacijo 
vpadnega valovanja v smeri nove lastne osi $x'$. Faznemu zamiku po preletu skozi
kristal dodajmo še časovni faktor
\beq
\phi =  k_0 n_{x'} L -\omega_0 t= \frac{\omega_0}{c_0}L \left(n_o -
\frac{1}{2}n_o^3 r_{63}\frac{U}{L}\right)-\omega_0 t.
\eeq
Fazni zamik je torej odvisen od priključene napetosti.

Obravnavajmo dva primera spreminjajoče se napetosti. V prvem primeru naj bo 
napetost linearna funkcija časa 
\beq
U= U_0 + \frac{\Delta U}{\Delta t}t.
\eeq
Celotna faza prepuščenega valovanja je potem
\beq
\phi = \frac{\omega_0}{c_0}L n_o - \frac{\omega_0 n_o^3 r_{63}}{2c_0}\left( U_0 + 
\frac{\Delta U}{\Delta t}t\right) - \omega_0 t.
\eeq
Spomnimo se, da ni trenutna frekvenca valovanja nič drugega kot negativni odvod faze po času
\beq
\omega = -\frac{d\phi}{dt}.
\eeq
V našem primeru dobimo 
\beq
\omega = \omega_0 + \frac{\omega_0 n_o^3 r_{63}}{2c_0}\frac{\Delta U}{\Delta t} = 
\omega_0 + \Delta \omega.
\eeq
Linearno naraščajoča modulacijska napetost da torej konstanten premik frekvence, kar v optiki 
pogosto potrebujemo. Dosegljive spremembe frekvence so seveda dokaj majhne,
do nekaj sto MHz. Omejene so z možno hitrostjo spreminjanja napetosti.
Napetost seveda tudi ne more neomejeno naraščati. Kadar se napetost
vrača na nič, pride do frekvenčnega premika v nasprotni smeri, ki pa ga
lahko zanemarimo, če je čas vračanja kratek primeri s časom naraščanja.

Poglejmo še drug primer, pri katerem je priključena napetost periodična. Zapišemo jo kot
\begin{equation}
U=U_{0}\sin(\omega_{m}t).
\label{7.21}
\end{equation}
Faza izhodnega valovanja bo tedaj 
\beq
\phi = \frac{\omega_0}{c_0}L n_o - \frac{\omega_0 n_o^3 r_{63}}{2c_0} U_0\sin(\omega_{m}t)
- \omega_0 t.
\eeq
Konstantni člen lahko izpustimo in zapišemo polje  
\beq
E = E_0 \cos\left( \omega_0 t + \frac{\omega_0 n_o^3 r_{63}}{2c_0} U_0\sin(\omega_{m}t)\right)
= E_0 \cos\left( \omega_0 t + \delta \sin(\omega_{m}t)\right),
\eeq
pri čemer je
\beq
\delta = \frac{\omega_0 n_o^3 r_{63}}{2c_0} U_0.
\eeq
Z uporabo Jacobi-Angerjevih identitet 
\begin{align}
\cos\left(\delta\sin x\right)  &=J_0(\delta)+2J_2(\delta)\cos2x+
2J_4(\delta)\cos4x \ldots\nonumber \\
\sin\left(\delta\sin x\right) &=2J_1(\delta)\sin x+2J_3\sin3x+
2J_5\sin5x+\ldots
\end{align}
je izhodno polje mogoče zapisati v obliki 
\begin{equation}
HELP!
% \begin{split}
% E = E_{0}\left[ J_0(\delta)\cos(\omega_0 t)+ J_1(\delta) \cos(\omega_0+\omega_m )t-
% J_1 (\delta) \cos(\omega_0 -\omega_m)t + \\ J_2(\delta)\cos(\omega +2\omega_m)t + 
% J_2(\delta)\cos(\omega -2\omega_m)t+ \\
% J_3(\delta)\cos(\omega +3\omega_m)t - J_3(\delta)\cos(\omega -3\omega_m)t+
% \right]
% \end{split}
\end{equation}
Periodična fazna modulacija torej da v spektru stranske pasove, odmaknjene
od osnovne frekvence $\omega_0$ za modulacijsko frekvenco. Njihova
velikost je podana s kvadratom Besselovih funkcij parametra $\delta$.
Če je ta majhen, se lahko zadovoljimo le s prvim členom.
\begin{definition}
Izpelji E. 
\end{definition}

\begin{remark}
Uporaba elektro-optičnega pojava: za modulacijo svetlobe, DC do $10^{10}$~Hz, uklanjanje žarkov
Q-switching, shutters... Deflektor iz dveh prizm, z smer kontra obrnjena, žarek se odklanja
sorazmerno z napetostjo. machzender deflektor, Davis, 494 in 495.
\end{remark}

% \section{*Modulacija pri visokih frekvencah}
% Do zdaj smo obravnavali elektro-optično modulacijo pri statičnih poljih ali 
% poljih, ki so se le počasi spreminjali s časom. Poglejmo še,
% kaj se zgodi pri visokih modulacijskih frekvencah.
% 
% Elektro-optični pojav pri nizkih frekvencah ima dva prispevka: direktnega,
% kjer zunanje polje vpliva neposredno na elektronsko polarizabilnost,
% in posrednega preko piezoelektričnega pojava. Snovi, ki nimajo centra
% inverzije, so tudi piezoelektrične in se v zunanjem električnem polju
% deformirajo. Deformacija pa povzroči spremembo lomnega količnika,
% o čemer bomo podrobneje govorili v enem od naslednjih oddelkov. Celotno
% spremembo tenzorja $b_{ij}$ lahko zapišemo 
% \begin{eqnarray}
% \delta b_{ij} & = & r_{ijk}^{\ast}E_{k}+p_{ijlm}S_{lm}\nonumber \\
%  & = & r_{ijk}^{\ast}E_{k}+p_{ijlm}\pi_{lmk}E_{k}
% \end{eqnarray}
%  Tu je $S_{lm}=\pi_{lmk}E_{k}$ piezoelektrično povzročena deformacija.
% Pri nizkih frekvencah sta oba prispevka primerljivo velika in je efektivni
% elektro-optični tenzor $r_{ijk}=r_{ijk}^{\ast}+p_{ijlm}\pi_{lmk}$.
% Pri dovolj velikih frekvencah deformacija kristala ne more več slediti
% modulacijski napetosti in ostane le direktni prispevek $r_{ijk}^{\ast}$.
% To se zgodi nad akustičnimi resonancami kristala. Pri akustičnih resonancah,
% to je, kadar modulacija v kristalu vzbudi stoječe zvočno valovanje,
% pa se piezoelektrični prispevek resonančno poveča.
% 
% Pogoj za akustično resonanco je, da je dimenzija kristala mnogokratnik
% polovice valovne dolžine akustičnega vala v kristalu. Uporabne dimenzije
% kristalov so reda velikosti centimeter, hitrost zvočnih valov je okoli
% 5000~m/s, tako da so resonance v področju od nekaj sto kHz do
% nekaj deset MHz. Mogoče jih je tudi izkoristiti za povečanje elektro-optičnega
% efekta pri izbrani frekvenci.
% 
% Pri visokih frekvencah postane pomembna tudi električna vezava modulatorja.
% Kristal predstavlja neko kapacitivno breme. Njegova impedanca pada
% z rastočo frekvenco, zato je vedno večji del padca napetosti na notranjem
% uporu izvora napetosti. Pomagamo si lahko tako, da vzporedno s kristalom
% vežemo še tuljavo, tako da je resonančna frekvenca $1/(L_{t}C)$ nastalega
% nihajnega kroga enaka željeni modulacijski frekvenci $\omega_{m}$.
% Tedaj je večina padca napetosti na kristalu in tuljavi. Da resonanca
% ni preostra in da je na voljo dovolj širok pas modulacijskih frekvenc,
% vežemo vzporedno s kristalom še upor z upornostjo $R$. Širina modulacijskega
% pasu je 
% \begin{equation}
% \Delta\omega_{m}=\frac{1}{RC}\;.\label{7.24}
% \end{equation}
%  Na uporu se troši moč 
% \begin{equation}
% P=\frac{1}{2}\frac{U^{2}}{R}=\frac{1}{2}U^{2}C\Delta\omega_{m}=
% \frac{\epsilon\epsilon_{0}La}{2d}U^{2}C\Delta\omega_{m}\;,\label{7.25}
% \end{equation}
%  kjer je $a$ velikost kristala v prečni smeri. Naj bo $U$ ravno
% napetost, ki da fazno razliko $\pi$. Potem sledi z uporabo enačbe
% \ref{7.18} 
% \begin{equation}
% P=\frac{A}{L}\,\frac{\epsilon\epsilon_{0}\Delta\omega_{m}\lambda^{2}}{3n_{0}^{6}r^{2}}\;,\label{7.26}
% \end{equation}
%  kjer je $A$ prečni presek kristala. Potrebna moč je odvisna od lastnosti
% modulatorja in širine modulacijskega pasu. Pri širini modulacijskega
% pasu 1~MHz in preseku kristala 1~cm$^{2}$ je potrebna moč nekaj
% deset W, kar je za visokonapetosten in hiter izvor že znatna moč.

\section{Elasto-optični pojav}
Elasto-optični pojav je pojav, pri katerem dielektrične lastnosti snovi in njen lomni količnik
spremenimo z mehansko deformacijo. Podobno kot pri elektro-optičnem pojavu lahko 
spremembo inverznega dielektričnega tenzorja zapišemo 
\begin{equation}
\underline{b} = \underline{\tilde{b}}+ \Delta\underline{b},
\end{equation}
pri čemer je $\Delta\underline{b}$ sprememba zaradi deformacije snovi. Zapišemo jo kot
\begin{equation}
 \Delta b_{ij}=p_{ijkl}S_{kl}.
\label{7.27}
\end{equation}
Sorazmerna je s tenzorjem defomacije snovi oziroma 
Greenovim tenzorjem\footnote{Angleški matematični fizik George Green, 1793--1841.} v
linearnem približku
\begin{equation}
S_{kl}=\frac{1}{2}\left({\frac{\partial u_{k}}{\partial x_{l}}}+{\frac{\partial u_{l}}{\partial x_{k}}}\right),
\label{7.28}
\end{equation}
pri čemer je $\mathbf{u}$ vektor deformacije. Vpeljali smo še sorazmernostni faktor
$p_{ijkl}$, ki ga imenujemo \textit{elasto-optični tenzor} in popisuje tudi spremembo 
dielektrične konstante in lomnega količnika zaradi spremembe gostote snovi. 
Tenzor $p$ je različen od nič v vsaki snovi, ker povezuje dva simetrična tenzorja 
drugega ranga. Posledično je simetričen v prvem in drugem paru indeksov
\beq
p_{ijkl} = p_{jikl} = p_{ijlk} =p_{jilk}.
\eeq
V najbolj splošnem primeru ima tako 36 neodvisnih komponent, simetrija dane
snovi pa to število še zmanjša. Če vpeljemo skrajšan zapis indeksov ($xx = 1,
yy=2, zz = 3, yz = 4, zx = 5, xy = 6$), zapišemo tenzor izotropne snovi kot
\beq
\underline{p}_{\textrm{izo}} = 
\left[\begin{array}{cccccc}
p_{11} & p_{12}& p_{12}&0&0&0\\
p_{12} & p_{11}& p_{12}&0&0&0\\
p_{12} & p_{12}& p_{11}&0&0&0\\
0 & 0& 0&p_{44}&0&0\\
0 & 0& 0&0&p_{44}&0\\
0 & 0& 0&0&0&p_{44}
\end{array}\right],
\label{tenzorp}
\eeq
pri čemer je $p_{44}= \frac{1}{2}(p_{11}-p_{12})$. Koeficienti tenzorja so 
brezdimenzijski, njihova tipična vrednost pa je $p\sim0,1$.
\begin{definition}
Yariv 442, izračunaj lastne osi in lastne lomne količnike če longitudinalen zvok v smeri x.
\end{definition}

Podobno kot pri elektro-optičnem pojavu lahko iz enačbe~(\ref{7.27}) izrazimo
spremembo dielektričnega tenzorja 
\begin{equation}
\Delta\epsilon_{ij}=-\tilde{\epsilon}_{ii}\tilde{\epsilon}_{jj}p_{ijkl}S_{kl},
\label{7.29}
\end{equation}
kjer smo že predpostavili, da je nemoteni $\epsilon$ diagonalen. Pogosto namesto zapletenih
računov s tenzorji uporabimo kar približek z efektivnim koeficientom elasto-optičnega pojava. 
Efektivni koeficient za vodo je tako $p = 0,31$.

Dvojni lom, ki se pojavi v deformirani snovi, izkoriščamo za študij
mehanskih napetosti v modelih, ki so izdelani iz prozorne plastične
snovi. Nas bo v nadaljevanju zanimal uklon svetlobe na periodični
modulaciji lomnega količnika, ki nastane zaradi zvočnega valovanja v snovi. Takemu pojavu
pravimo tudi akusto-optični pojav.

\section{Uklon svetlobe na zvočnem valovanju}

Vzbudimo v plasti prozorne snovi zvočno valovanje z valovno dolžino $\Lambda$, ki potuje v smeri $x$.
To naredimo tako, da na eno stran snovi priključimo piezoelektrik, 
ki se pod izmenično napetostjo periodično krči in razteza s frekvenco $\Omega$, 
na drugo stran pa akustični reflektor. Na ta način lahko v snovi vzbudimo tudi stoječe
valovanje. Zaradi zvočnega valovanja se v snovi periodično spreminja gostota in 
z njo lomni količnik
\beq
n = n_0 + \Delta n \sin \left(\frac{2\pi}{\Lambda} x- \Omega t\right).
\eeq
V zgoščini je lomni količnik nekoliko večji kot v razredčini, zato je optična pot na takem mestu
skozi plast daljša. Ravno svetlobno valovanje, ki vpada na plast, po
izstopu nima povsod enake faze, valovno čelo pa je periodično modulirano s periodo 
valovne dolžine zvočnega valovanja. Zvočno valovanje v snovi torej deluje kot 
optična fazna mrežica. Tipična frekvenca, s katero vzbujamo elastično
deformacijo, je okoli $\Omega=50$~MHz, ustrezna valovna dolžina pa okoli $\Lambda = 100~\mu$m. 
Frekvence, ki so v uporabi, pa navadno sežejo od nekaj MHz prek 10 GHz. Vsa ta valovanja bomo 
imenovali zvočno valovanje. 

\begin{figure}[h]
\centering
\def\svgwidth{70truemm} 
\input{slike/09_AOshema.pdf_tex}
\caption{Vpadna svetloba se na stoječem zvočnem valovanju v snovi uklanja.}
\label{fig:ao}
\end{figure}

Oglejmo si dva limitna primera. V prvem primeru je debelina snovi $d$, v kateri vzbujamo
zvočno valovanje, zelo majhna. Takrat modulator deluje 
kot tanka uklonska mrežica in pojavi se veliko 
uklonskih vrhov. Kote, pod katerimi se pojavijo ojačitve, izračunamo po preprosti enačbi
\beq
\Lambda (\sin \vartheta - \sin \beta ) = N \lambda,
\eeq
pri čemer je $\lambda$ valovna dolžina svetlobe v snovi, $N$ pa celo število. Takemu pojavu 
pravimo Raman-Nathov uklon\footnote{Indijski fizik in nobelovec Sir Chandrasekhara 
Venkata Raman, 1888--1970, 
in indijski fizik N. S. Nagendra Nath.}. Opazimo ga pri razmeroma nizkih zvočnih frekvencah 
(pod $\Omega\sim10$~MHz) in majhnih debelinah (pod $d\sim 1$~cm) pri poljubnem vpadnem 
kotu $\vartheta$.
\begin{figure}[h]
\centering
\def\svgwidth{50truemm} 
\input{slike/09_AO_1.pdf_tex}\qquad
\def\svgwidth{65truemm} 
\input{slike/09_AO_2.pdf_tex}
\caption{Ob vpadu na tanek modulator se pojavi veliko uklonskih vrhov, na debelem modulatorju
pa zgolj en uklonjen vrh, pa še ta le ob izpolnjenem Braggovem pogoju. }
\label{fig:ao_bragg}
\end{figure}

V nasprotnem limitnem primeru se svetloba uklanja na ravnih zvočnih valovih in modulator deluje 
kot debela uklonska mrežica. V splošnem je delež uklonjene svetlobe na taki mrežici neuporabno majhen. 
Znaten postane le tedaj, kadar je za enega od uklonjenih valov izpolnjen Braggov
pogoj\footnote{Angleška znanstvenika in nobelova nagrajenca Sir William Henry Bragg, 1862--1942,
in Sir William Lawrence Bragg, 1890--1971.}
\begin{equation}
2 \Lambda\sin\theta=\pm N\lambda.
\label{7.29a}
\end{equation}

Poglejmo natančneje, kako pridemo do gornjega pogoja. Zapišimo pogoj
za ohranitev gibalne količine fotona pri sipanju na zvočnem valu
\begin{equation}
\mathbf{k}_{0}\pm\mathbf{q}=\mathbf{k}_{1},
\label{7.30}
\end{equation}
kjer je $\mathbf{k}_{0}$ valovni vektor vpadne svetlobe, $\mathbf{k}_{1}$
valovni vektor uklonjenega svetlobnega snopa, $\mathbf{q}$ pa valovni
vektor zvočnega vala. Znak plus velja, kadar potuje zvok proti projekciji
$\mathbf{k}_{0}$ na $\mathbf{q}$, negativen predznak pa ob potovanju zvoka v nasprotno smer. 
Ker je frekvenca zvočnega vala dosti nižja od frekvence svetlobe, se frekvenca svetlobe 
pri sipanju le malo spremeni in $vec{k}_{0}$ in $\mathbf{k}_{1}$ sta po velikosti skoraj enaka.
Tedaj je $q=2k_{0}\sin\vartheta$ (glej sliko~\ref{fig:Bragg}), od koder sledi Braggov pogoj 
(enačba~\ref{7.30}). Obenem mora biti vpadni kot na zvočni val enak izhodnemu, kar pomeni, da se na
zvočnem valu Braggovo sipana svetloba zrcalno odbije. Razmere so 
povsem analogne Braggovemu sipanju rentgenske svetlobe na kristalnih
ravninah. Kadar je izpolnjen Braggov pogoj, je mogoče doseči, da se
vsa vpadna svetloba siplje, kot bomo pokazali v nadaljevanju.

\begin{remark}
 Poskusimo še bolj natančno oceniti, kdaj je v veljavi Raman-Nathov in kdaj Braggov režim. 
Izhajajmo iz pogoja, da je razširitev žarka na debelini prehoda skozi zvočno valovanje dovolj
 majhna, da se snop ne širi iz območja zgoščine v območje razredčine, da se torej ne razširi za 
 več kot za $\Lambda/2$. Tako zapišemo
 divergenco kot
 \beq
 \delta\vartheta \sim \frac{\lambda}{w_0} \sim \frac{2\lambda}{\Lambda},
 \eeq
 ki ne sme presegati razširitve
 \beq
 \delta \vartheta \sim \frac{\Lambda}{2d}.
 \eeq
Sledi kriterij za debelino $d$, pri kateri preidemo iz enega v drug režim
\beq
d \sim \frac{\Lambda^2}{4 \lambda}.
\eeq
Če tako svetloba z $\lambda = 1~\mu$m vpade na kristal, v katerem je vzbujeno zvočno
valovanje z valovno dolžino $\Lambda = 0,2$~mm in frekvenco $\Omega = 150$~MHz, je mejna 
debelina $d \sim 1$~cm.
\end{remark}

Če je zvočno valovanje potujoče, kar smo v gornjem razmišljanju že privzeli
s tem, ko smo mu pripisali natanko določen valovni vektor $\mathbf{q}$,
se spremeni tudi frekvenca sipanega vala zaradi Dopplerjevega premika
pri odboju na zvočnem valovanju, ki potuje s hitrostjo $v_{z}$. Upoštevati
moramo le projekcijo na smer vpadne in odbite svetlobe, zato je 
\begin{equation}
\frac{\Delta\omega}{\omega}=\pm\frac{2v_{z}\sin\vartheta}{c}=
\pm\frac{2\Omega\Lambda\sin\vartheta}{2 \pi c}=\pm\frac{\Omega}{\omega},
\label{7.32}
\end{equation}
pri čemer smo uporabili Braggov pogoj (enačba~\ref{7.29a}). Sprememba frekvence
sipane svetlobe je torej kar enaka frekvenci zvočnega valovanja. To sledi tudi
iz zahteve, da se mora pri sipanju na zvočnem valovanju ohraniti energija
vpadnega fotona in kvanta zvočnega valovanja - fonona,
ki se pri sipanju absorbira ali pri njem nastane.

Malenkost drugačno je obnašanje, ko v snovi vzbudimo stoječe zvočno valovanje. 
Takrat lahko sipanje obravnavamo kot vsoto sipanja na dveh valovanjih z valovnima 
vektorjema $\mathbf{q}$ in $-\mathbf{q}$. Smer Braggovo sipanega vala je obakrat enaka, 
frekvenca pa se enkrat poveča, drugič zmanjša za $\Omega$. Zato se pojavi utripanje
sipanega vala s frekvenco $2\Omega$.

\subsection*{Uporaba akusto-optičnih modulatorjev}
Spoznali smo, da lahko z zvočnim valovanjem spreminjamo smer svetlobe iz vpadne smeri.
Bistvena razlika od navadnih uklonskih mrežic je dinamičnost akusto-optičnih modulatorjev, 
saj lahko s spreminjajočim zvočnim valovanjem spreminjamo uklonski kot svetlobe,
pri čemer pa smo omejeni s tem, da mora biti približno izpolnjen Braggov pogoj.
S kombinacijo dveh med seboj pravokotnih akusto-optičnih modulatorjev lahko žarek zelo hitro
premikamo po ravnini, kar s pridom uporabljamo v različnih optičnih napravah. 

Z vklapljanjem in izklaplanjem zvočnega valovanja, ki ga vzbujamo s piezoelektričnim elementom,
na katerega pritisnemo izmenično napetost, lahko moduliramo intenziteto
direktnega svetlobnega snopa. To potrebujemo na primer pri preklapljanju
kvalitete laserskega resonatorja.

Druga uporaba je spreminjanje frekvence svetlobe. Možne so spremembe
do nekaj 100~MHz, kar je ravno primerno za uporabo v laserskih merilnikih
hitrosti, kjer merimo frekvenco utripanja med svetlobo, odbito od
merjenega predmeta, in referenčno svetlobo. Če ima referenčna svetloba
isto frekvenco kot merilni snop, ni mogoče določiti predznaka hitrosti
predmeta, če pa referenčni svetlobi nekoliko spremenimo frekvenco,
se pojavi utripanje tudi tedaj, ko predmet miruje. Frekvenca utripanja
se poveča ali zmanjša glede na predznak hitrosti predmeta.

Tretja pomembna uporaba je kombinacija obeh gornjih za uklepanje faz
v laserskem resonatorju. Če je v Braggovem elementu prisotno stoječe zvočno
valovanje, je amplituda direktnega snopa modulirana s frekvenco zvoka.
Kadar je frekvenca zvoka ravno enaka razmiku frekvenc laserskih nihanj,
lahko nastanejo uklenjene faze vzbujenih nihanj in s tem kratki, periodični
sunki svetlobe.

Zanimiva je tudi uporaba Braggovega elementa za idelavo
hitrega frekvenčnega analizatorja električnih signalov. Shemo kaže slika~(\ref{fig:AORF}). 
Piezoelektrični element vzbujamo z električnim signalom,
ki ima neznan spekter. Enak spekter imajo tudi vzbujeni zvočni valovi, 
pri čemer vsakemu valu določene frekvence ustreza določen kot odklona svetlobnega
snopa. Za Braggovim elementom postavimo lečo. Vsak delni uklonjeni
snop da v goriščni ravnini svetlo točko, katere položaj je odvisen
od kota odklona in torej od frekvence zvočnega vala. Spekter zaznamo
z vrstičnim detektorjem. Na celo napravo lahko pogledamo tudi takole:
Braggova celica frekvenčni spekter zvočnih valov prevede v prostorski
spekter prepuščene svetlobe. Prostorski spekter svetlobe pa lahko
analiziramo z lečo, ki nam v goriščni ravnini na desni da prostorsko
Fourierevo transformacijo svetlobnega snopa na levi strani leče.

Akusto-optični modulatorji so nepogrešljivi pri hitrem in 
natančnem usmerjanju laserskega žarka, na primer v optičnih pincetah, optičnih čitalcih ali 
optičnih litografskih zapisovalnikih.

\section{Račun akusto-optičnega pojava}

Izračunajmo še delež uklonjene svetlobe. Za to moramo rešiti 
valovno enačbo v nehomogenem sredstvu, kar je dokaj težaven problem
in se moramo zateči k približkom. Poslužili se bomo metode sklopljenih valov. 

Naj vzporeden snop zvočnega valovanja s širino $d$ in valovnim vektorjem $vec{q}$ potuje v smeri $x$.
Nanj pod kotom $\vartheta$ glede na os $z$ vpada ravno svetlobno valovanje z valovnim vektorjem 
$\mathbf{k}=(k_{x},0,k_{z})= k(\sin\vartheta,0,\cos\vartheta)$.
Vse valovanje, vpadno na levi od zvočnega snopa in izhodno na njegovi desni,
obravnavajmo znotraj snovi, da nam ni treba upoštevati še loma, ki
le zaplete izraze. 

Privzemimo, da se v snovi zaradi zvočnih valov spremeni le velikost
dielektrične konstante. Ob upoštevanju zveze med spremembo dielektričnosti in deformacijo
v zvočnem valu~(enačba~\ref{7.29}) lahko spremembo dielektričnosti
zapišemo kot  
\begin{equation}
\epsilon=\tilde{\epsilon}+\Delta\epsilon = 
\tilde{\varepsilon} -\tilde{\epsilon}^{2}pS_{0}\sin(qx-\Omega t).
\label{7.33}
\end{equation}
Zaradi spremembe dielektričnosti pride do pojava
dodatne polarizacije $\Delta P$
\beq
\Delta P = \varepsilon_0 \Delta \varepsilon E = - \varepsilon_0 \tilde{\varepsilon}^2 p S_0 \sin(qx-\Omega t)E,
\eeq
ki nastane zaradi zvočnega vala. Upoštevati jo moramo pri zapisu valovne enačbe, 
podobno kot smo to naredili pri nelinearni optiki (enačba~\ref{8.3})
\begin{equation}
\nabla^{2}E-\frac{\tilde{\epsilon}}{c^{2}}{\frac{\partial E^{2}}
{\partial t^{2}}}=\mu_{0}{\frac{\partial^2 \Delta P}{\partial t^{2}}},
\label{7.33a}
\end{equation}
pri čemer smo privzeli, da je $\nabla\cdot\mathbf{E}\approx 0$, čeprav je
$\epsilon$ funkcija kraja. 

Enačbo~(\ref{7.33a}) brez dodane polarizacije $\Delta P$ rešijo ravni valovi 
z valovnim vektorjem $\mathbf{k}$ in frekvenco $\omega$. Tej rešitvi se 
primešajo valovi z valovnim vektorjem $\mathbf{k}\pm n\mathbf{q}$
in frekvenco $\omega\pm n\Omega$. Zato iščemo rešitve v obliki vsote
ravnih valov, torej Fourierove vrste
\begin{equation}
E=\sum_{n}A_{n}(z)e^{in(qx-\Omega t)}e^{i(k_{x}x+k_{z}z-\omega t)}.
\label{7.34}
\end{equation}
Zaradi sklopitve preko $\Delta P$ moramo dovoliti, da so amplitude
$A_{n}$ funkcije $z$. Če je $\Delta\epsilon$ dovolj majhen, se $A_{n}(z)$
le počasi spreminjajo.

Izračunajmo 
\begin{equation}
\nabla^{2}E=\sum_{n}\left( -[k_{z}^{2}+(k_{x}+nq)^{2}]A_{n}(z)+2ik_{z}A_{n}'(z)\right) \, e^{i[(k_x+nq)x+k_{z}z-(\omega+n\Omega)t]}.
\label{7.35}
\end{equation}
Člen z $A_{n}''$ lahko izpustimo, če je le $k_{z}A_{n}'\gg A_{n}''$ oziroma 
kadar se $A_{n}$ spreminjajo počasi v primerjavi z exp$(ik_{z}z)$. Drugi odvod 
polarizacije po času da
\beq
\frac{\partial^2 \Delta P}{\partial t^2} =-\frac{\varepsilon_0 \tilde{\varepsilon}^2pS_0}{2i}\sum_{n}\left(
-[n\Omega+\omega+\Omega]^2e^{i(qx-\Omega t)} + [n\Omega+\omega-\Omega]^2e^{i(-qx+\Omega t)} \right) \, 
A_n(z) e^{i[(k_x+nq)x+k_{z}z-(\omega+n\Omega)t]}.
\label{7.35a}
\eeq
Vstavimo izraze (\ref{7.34}), (\ref{7.35}) in (\ref{7.35a}) v valovno enačbo (\ref{7.33a})
in izenačimo člene, ki nihajo z isto časovno in prostorsko frekvenco, na primer
z $k_z z+(k_x+mq)x-(\omega+m\Omega)t$. Tako dobimo pogoj
\begin{eqnarray}
-[k_{z}^{2}+(k_{x}+mq)^{2}]A_{m}+2ik_{z}A_{m}' + \frac{\tilde{\varepsilon}}{c^2}(m\Omega+\omega)^2A_m
=\frac{\mu_0\varepsilon_0\tilde{\varepsilon}^2pS_0}{2i}(\omega+m\Omega)^{2}(A_{m-1}-A_{m+1}).
\end{eqnarray}
Upoštevamo, da je 
\beq 
k_{x}^{2}+k_{z}^{2}=\frac{\tilde{\epsilon}\omega^2}{c^{2}}=k^{2}
\eeq
in naredimo približek $(\omega +m\Omega)^2 \approx \omega^2$.
Sledi
\begin{equation}
A_{m}'+i\beta_{m}A_{m}+\xi(A_{m+1}-A_{m-1})=0,
\label{7.37}
\end{equation}
kjer sta
\begin{equation}
\beta_{m}=\frac{mq}{k_{z}}(k_{x}+\frac{1}{2}mq)
\label{7.38}
\end{equation}
 in 
\begin{equation}
\xi=-\frac{\tilde{\epsilon} pS_0k^2}{4k_z}.
\label{7.39}
\end{equation}
 Reševanje sistema enačb~(\ref{7.37}) je težavno. Rešitve poiščimo le v treh
pomembnih limitnih primerih. Naj bo amplituda vala, ki vpada z leve,
$A_{0}(0)=A_{0}$, ostale $A_{n}(0)$ pa nič.

Najprej privzemimo, da je $L\xi \ll 1$, da je torej velikost $\Delta \epsilon$
majhna in debelina zvočnega snopa ne prevelika. Tedaj je pri vseh
$z$ in za pozitivne $m$ $A_{m+1}\ll A_{m}$ in lahko člen $A_{m+1}$
v enačbi~(\ref{7.37}) izpustimo. S tem zapišemo preprost sistem enačb
\begin{equation}
A_{m}'+i\beta_{m}A_{m}=\xi A_{m-1},
\label{7.40}
\end{equation}
 ki jih lahko zapored integriramo: 
\begin{equation}
A_{m}(z)=\xi e^{-i\beta_{m}z}\int_{0}^{z}A_{m-1}(z')
e^{i\beta_{m}z'}dz'.
\label{7.41}
\end{equation}
 Podobne izraze izpeljemo za negativne $m$, to je za uklonjene valove,
ki se jim frekvenca pri sipanju zmanjša.

Poglejmo posebej prvi uklonjeni val z amplitudo $A_{1}$. Po predpostavki,
da je $A_{\pm1}\ll A_{0}$, se le malo energije uklanja iz osnovnega
vala in je $A_{0}(z)$ skoraj konstanta. Potem lahko integral v enačbi~(\ref{7.41})
izračunamo
\begin{equation}
A_{1}(L)=A_{0}\xi L\,\frac{\sin\beta_{1}L/2}{\beta_{1}L/2}\, e^{-i\beta_{1}L/2}\;.
\label{7.41a}
\end{equation}
Funkcija $A_{1}(L)$ ima vrh pri $\beta_{1}=0$, to je pri 
\begin{equation}
\beta_1 = k_x+ \frac{q}{2} = k \sin\vartheta + \frac{q}{2} = 0
\label{7.42}
\end{equation}
ali 
\begin{equation}
2\Lambda\sin\vartheta=-\lambda.
\label{7.43}
\end{equation}
Vdimo, da predstavlja $\beta_{1}=0$ ravno pogoj za Braggovo sipanje vpadnega
vala.

Razmere pri Braggovem sipanju je vredno pogledati še nekoliko podrobneje.
Ker je hitrost zvoka mnogo manjša od hitrosti svetlobe, je $\Omega/c\ll q$
in sta velikosti vpadnega in sipanega valovnega vektorja praktično enaki. 
Komponenti $x$ se razlikujeta za $q$. Kadar je Braggov pogoj izpolnjen, velja
tudi, da je sipani valovni vektor $\mathbf{k}_{s}=\mathbf{k}+\mathbf{q}$.
Braggov pogoj je torej primer ohranitve gibalne količine. Če se ta ne ohranja, 
je sipanje neučinkovito in le oscilira okoli majhne vrednosti, kar opisuje 
faktor $\sin(\beta L/2)/(\beta L/2)$.

Delež moči uklonjenga vala je 
\begin{equation}
\frac{I_{1}}{I_{0}}=\left|\frac{A_{1}}{A_{0}}\right|^{2}=(\xi L)^{2}
\left(\frac{\sin\beta_{1}L/2}{\beta_{1}L/2}\right)^{2}\;.\label{7.45}
\end{equation}
Če je Braggov pogoj izpolnjen, je $I_{0}/I_{1}=(\xi L)^{2}$,
kar lahko velja le, dokler je $\xi L\ll1$. Kadar intenziteta uklonjenega žarka
tako naraste, da ta pogoj ni več izpolnjen, je treba v računu upoštevati tudi 
zmanjšanje moči vpadnega snopa.

Braggov pogoj je hkrati lahko izpolnjen le za en uklonjen val, na
primer $m=1$. Zato so tedaj vse ostale amplitude $A_{m}$, $m\ne0,1$
majhne in ne vplivajo na $A_{1}$. To nam omogoča drug približek,
ki je za uporabo zelo pomemben. Opustimo omejitev $L\xi\ll 1$, vendar
zahtevajmo, da sta le $A_{0}$ in $A_{1}$ različni od nič. Sedaj
$A_{0}(z)$ ne smemo več obravnavati kot konstante. Iz sistema enačb~(\ref{7.37})
sledi, pri čemer upoštevamo izpolnjen Braggov pogoj (enačba~\ref{7.42}).
\begin{eqnarray}
A_{0}'+\xi A_{1} & = & 0\nonumber \\
A_{1}'-\xi A_{0} & = & 0.
\end{eqnarray}
Ob začetnih pogojih $A_{0}(0)=A_{0}$ in $A_{1}(0)=0$ je rešitev enačb~(\ref{7.46}) 
\begin{eqnarray}
A_{0}(L) & = & A_{0}\cos\xi L\nonumber \\
A_{1}(L) & = & A_{0}\sin\xi L\;.
\end{eqnarray}
Če je izpolnjen Braggov pogoj, se moč vpadnega vala na razdalji $\pi/(2\xi)$
skoraj vsa pretoči v uklonjeni snop, nato pa zopet nazaj (slika \ref{s7.10}).
Za čim bolj učinkovito delovanje akusto-optičnega modulatorja seveda
želimo doseči ravno take pogoje.

Razmerje med močjo uklonjenega in vpadnega snopa je tako
\begin{equation}
\frac{I_{1}}{I_{0}}=\sin^{2}\left(\frac{\pi n_{0}^{3}pS_{0}L}{2\lambda\cos\phi}\right).
\label{7.48}
\end{equation}
Amplitudo deformacije v zvočnem valu $S_0$ izrazimo še z gostoto moči zvočnega vala 
\begin{equation}
j_{z}=\frac{1}{2}CS_{0}^{2}v_{z},
\label{7.49}
\end{equation}
kjer je $C$ elastična konstanta snovi, $v_z$ pa hitrost zvoka v snovi. 
Iz zveze $v_{z}^{2}=C/\rho$ izrazimo $C$ z gostoto, s čemer dobimo 
\begin{equation}
S_{0}=\sqrt{\frac{2j_{z}}{\rho v_{z}^{3}}}\;.\label{7.50}
\end{equation}
Praktično je vpeljati merilo uporabnosti neke snovi za akusto-optični modulator. To je koeficient 
\begin{equation}
M=\frac{n_{0}^{6}p^{2}}{\rho v_{z}^{3}}.
\label{7.51}
\end{equation}

Poglejmo primer. V kremenu z gostoto $\rho=2,2\cdot10^{3}$ kg/m$^{3}$ je hitrost zvoka $v_{z}=6000$~m/s,
$n_{0}=1,46$ in $p=0,2$. To da $M=8\cdot10^{-16}$~W/m$^{2}$.
Pri gostoti zvočnega toka 10~W/cm$^{2}$ in valovni dolžini svetlobe 633~nm
pride do popolnega prenosa moči v uklonjeni snop pri debelini $L=3$~cm. Gornja gostota
zvočnega toka je kar velika in je ni prav lahko doseči, zato so 
uklonski izkoristki navadno nekaj manjši od 1.

Izračunajmo še kot odklona uklonjenega vala $\theta = 2 \vartheta$  
\begin{equation}
\theta \approx \frac{q}{k}=\frac{\lambda}{n_{0}\Lambda}=1,7\cdot10^{-3}
.\label{7.52}
\end{equation}
Uklonski kot je torej precej majhen.

\begin{remark}
Opisani račun izkoristka uklona na zvočnih valovih je uporaben tudi
pri računu izkoristka holograma. V primeru faznega holograma je račun
čisto enak in nam kaže tudi razliko med tankim in debelim hologramom,
kako pa je z izkoristkom amplitudnega holograma, kjer je modulirana
absorpcija v snovi, lahko bralec izračuna sam.
\end{remark}

Vrnimo se k sistemu enačb~(\ref{7.37}), ki smo jo zaenkrat rešili za primer Braggovega odboja
oziroma v njegovi bližini. Enačbe je preprosto rešiti še v enem primeru, v tako imenovanem
Raman- Nathovem približku. Vpeljimo novo neodvisno spremenljivko $\zeta=2\xi z$. Zveza~(\ref{7.37})
preide v 
\begin{equation}
2\frac{dA_{m}(\zeta)}{d\zeta}+A_{m+1}(\zeta)-A_{m-1}(\zeta)=\frac{\beta_{m}}{i\xi}A_{m}.
\label{7.53}
\end{equation}
 Člen na desni lahko izpustimo, če je 
\begin{equation}
\frac{\beta_{m}}{\xi}=\left| \frac{4mq}{\tilde{\varepsilon}pS_0k}(\sin\vartheta+\frac{mq}{2k})\right| 
\ll 1,
\label{7.54}
\end{equation}
oziroma če je valovna dolžina zvoka dovolj velika v primerjavi z valovno dolžino svetlobe. 
V enačbi~(\ref{7.53}) prepoznamo rekurzijsko zvezo za Besselove funkcije 
\begin{equation}
2J_{n}'+J_{n+1}-J_{n-1}=0
\label{7.55}
\end{equation}
z rešitvijo $A_{m}(z)=A_{0}J_{m}(2\xi z)$. Kadar je $2\xi L$ ničla funkcije
J$_{0}$, prvič je to pri $2\xi L\approx 2.4$, se vsa energija ukloni iz
vpadnega snopa, vendar se v tem primeru, ko Braggov pogoj ni izpolnjen,
razporedi v mnogo uklonjenih snopov.

\section{Modulacija s tekočimi kristali}

Za konec opišimo še modulacijo svetlobe s tekočimi kristali. 
Tekoči kristali so anizotropne kapljevine in so vmesna faza med navadnimi
izotropnimi kapljevinami in kristali. Tekoče kristale tvorijo podolgovate ali 
ploščate molekule, ki imajo različne stopnje urejenosti. Omejimo se najosnovnejši
primer, to je podolgovate organske molekule v nematski fazi tekočega kristala. 
Navadno so to molekule z relativno togim jedrom iz
dveh ali treh benzenovih obročev, ki imajo na koncih krajše ali daljše
alifatske verige (slika~\ref{s7.11}). V nematski fazi so težišča molekul neurejena, 
enako kot v navadni tekočini, osi molekul pa so v povprečju urejene v določeno smer. 
Smer povprečne urejenosti opišemo z enotnim vektorjem $\mathbf{n}$, ki mu rečemo direktor. 
Smeri $\mathbf{n}$ in $-\mathbf{n}$ sta enakovredni, saj molekule
z enako verjetnostjo kažejo v smeri $+\mathbf{n}$ kot v $-\mathbf{n}$.
Stopnja urejenosti v mikroskopski sliki ni prav velika, povprečen
odklon molekul od $\mathbf{n}$ je nekaj deset stopinj.

Ker so molekule v povprečju urejene, se nematski tekoči kristal obnaša kot enoosen 
dvolomni kristal, njegova optična os pa je vzporedna z $\mathbf{n}$. 
Ker je optična polarizabilnost benzenovih obročev vzdolž osi molekul precej večja kot
v prečni smeri, je razlika med rednim in izrednim lomnim količnikom razmeroma velika, navadno med
0,1 in 0,2.

V makroskopskem vzorcu nematskega tekočega kristala se v splošnem smer
direktorja $\mathbf{n}$ po vzorcu neurejeno spreminja. Energija takega deformiranega 
stanja je nekoliko večja od energije homogenega urejenega stanja,
zato na delček tekočega kristala okolica deluje z navorom v smeri zmanjševanja 
nehonogenosti $\mathbf{n}$. Temu pojavu, značilnemu za tekoče kristale,
pravimo orientacijska elastičnost. Ti ti elastični navori so prešibki,
da bi uredili razsežne neurejene vzorce. Urejene vzorce, kakršne potrebujemo
v optičnih elementih, pripravimo v zunanjem polju ali pa dovolj tanke, da mejne 
površine vsilijo ureditev v celem vzorcu. Poglejmo, kako.

Na ureditev molekul tekočega kristala vpliva zunanje električno ali magnetno polje.
Električna in magnetna susceptibilnost nematskega tekočega
kristala nista skalarja, temveč imata dve različni lastni vrednosti,
eno v za smer vzporedno z $\mathbf{n}$, drugo za pravokotno nanj. Zato je
elektrostatična energija odvisna od kota med zunanjim poljem $vec{E}$
in $\mathbf{n}$. Kotno odvisni del energije lahko zapišemo v obliki
\begin{equation}
w_{a}=-\frac{1}{2}\epsilon_{0}\epsilon_{a}(\mathbf{E}\cdot\mathbf{n})^{2},
\label{7.56}
\end{equation}
kjer je $\epsilon_{a}=\epsilon_{\parallel}-\epsilon_{\perp}$ anizotropni
del dielektrične konstante. Če je $\epsilon_{a}>0$, se molekule skušajo
zasukati v smer polja, če je $\epsilon_{a}<0$, pa pravokotno nanj.

Urejen vzorec je mogoče narediti tudi v tankih plasteh. Če površino,
ki je v stiku s tekočim kristalom, prevlečemo z primerno pastjo, se
molekule tekočega kristala tik ob povšini na določen način uredijo.
Tanka plast najlona, ki jo podrgnemo v željeni smeri, povzroči, da
je $\mathbf{n}$ ob površini vzporeden s površino v smeri drgnenja. Drgnenje
deloma uredi verige najlona, zato se tudi molekule tekočega kristala
raje uredijo v isti smeri. Da je $\mathbf{n}$ pravokoten na površino,
dosežemo na primer s tanko plastjo lecitina ali surfaktanta silana. 
Ta imata polarno glavo, ki se adsorbira na stekleno površino, in alifatsko verigo, 
ki stoji približno pravokotno na površino. Zato se tudi alifatski repi molekul
tekočega kristala uredijo enako. V obeh primerih, vzporedni ali pravokotni
orientaciji ob steni, se urejenost zaradi orientacijske elastičnosti
ohranja tudi stran od stene, tako da lahko brez težav naredimo urejene
vzorce debeline do kakih 200~$\mu$m. Pri večjih debelinah so elastični
navori prešibki in v vzorcu nastanejo defekti.

Struktura tekočekristalnega vzorca je tako odvisna od orientacijske
elastičnosti, od robnih pogojev, ki jih določimo z obdelavo mejne
površine, in od zunanjega električnega ali magnetnega polja.

Preprost elektro-optični modulator ali kazalnik na osnovi nematskih
tekočih kristalov naredimo takole. Vzemimo vzorec tekočega kristala
med dvema stekloma, na katerih sta prozorni elektrodi. Ureditev tekočega
kristala naj bo vzporedna s površino stekel. Tudi optična os ima isto
smer. Dovolj velika napetost zasuče $\mathbf{n}$ in s tem optično os
pravokotno na stene, razen tik ob površini. Pri debelini okoli 10~$\mu$m
je potrebna napetost nekaj voltov.

Naj debelina $h$ obravnavane plasti za izbrano valovno dolžino svetlobe
ustreza pogoju 
\begin{equation}
h(n_{i}-n_{r})=(2N+1)\frac{\lambda}{2}\;,\label{7.57}
\end{equation}
 kjer je $N$ celo število. Plast torej deluje kot ploščica $\lambda/2$.
Ker je $n_{i}-n_{r}\simeq0,1$, je potrebna debelina nekaj $\mu$m.
Vzorec damo med prekrižana polarizatorja s prepustno smerjo 45$^{o}$
na $\mathbf{n}$. Ploščica polarizacijo svetlobe z izbrano valovno dolžino
zasuče za 90$^{o}$, tako da gre svetlobe tudi skozi analizator. Ko
vključimo napetost, se optična os obrne, polarizacija svetlobe se
pri prehodu skozi plast ohrani in analizator je ne prepusti.

Tak preklopnik ima nekaj slabih lastnosti. Prepustnost je odvisna
od valovne dolžine svetlobe in od temperature in debelina preklopnika
mora biti povsod enaka. Zato se v praksi uporablja zasukana nematska
plast.

Obrnimo eno od stekel za 90$^{o}$, tako da sta smeri urejanja na
obeh mejah med seboj pravokotni. Tedaj se smer $\mathbf{n}$ v plasti
zvezno zavrti od ene do druge površine, kot kaže slika \ref{s7.13.}.
Polarizacija svetlobe, ki je ob vstopu v plast polarizirana v smeri
urejanja, skozi plast približno sledi smeri $\mathbf{n}$, kot bomo pokazali
nekoliko kasneje, in je ob izstopu iz plasti polarizirana pravokotno
na vpadno polarizacijo. Z električnim poljem preklopimo optično os
pravokotno na plast in polarizacija se ne zasuče. Plast med prekrižanima
polarizatorjema brez polja prepušča svetlobo, v polju pa ne. Pri tem
delovanje kazalnika ni dosti odvisno niti od debeline plasti niti
od valovne dolžine. Kadar želimo, da kazalnik deluje v odbiti svetlobi,
damo za zadnji analizator še odbojno površino.

Pri kristalnem elektro-optičnem modulatorju smo lahko dobili tudi vmesne
prepustnosti, medtem ko s tekočimi kristali na opisan način lahko
dosežemo le zaprto in odprto stanje, vmesna stanja pa je zelo težko
kontrolirati.

Pokažimo še, da polarizacija v sredstvu, ki je lokalno enoosno in
se optična os suče v pravokotni smeri, v določenih pogojih res približno
sledi optični osi. Poleg zasukane nematske celice je pomemben primer
snovi s takimi lastnostmi holesteričen tekoči kristal, ki je zelo
podoben nematskim, le da se $\mathbf{n}$ spontano počasi suče okoli
smeri, pravokotne na $\mathbf{n}$.

Optična os sredstva naj leži v ravnini $xy$ in naj se enakomerno
suče, ko se premikamo vzdolž osi $z$. Kot med optično osjo in osjo
$x$ lahko zapišemo 
\begin{equation}
\phi=qz\;.\label{7.58}
\end{equation}
 Pri tem je $2\pi/q$ perioda sukanja optične osi. Zanimajmo se le
za širjenje svetlobe v smeri $z$. Tedaj potrebujemo le del dielektričnega
tenzorja v $xy$ ravnini. Njegova oblika je 
\begin{equation}
\boldsymbol{\epsilon}(z)=\left[\begin{array}{cc}
\bar{\epsilon}+\frac{1}{2}\epsilon_{a}\cos2\phi & \frac{1}{2}\epsilon_{a}\sin2\phi\\
\frac{1}{2}\epsilon_{a}\sin2\phi & \bar{\epsilon}+\frac{1}{2}\epsilon_{a}\cos2\phi
\end{array}\right]\;,\label{7.59}
\end{equation}
 kjer je 
\begin{equation}
\bar{\epsilon}=\frac{\epsilon_{\parallel}+\epsilon_{\perp}}{2}\label{7.60}
\end{equation}
 Valovna enačba za valovanje s frekvenco $\omega$ je 
\begin{equation}
\frac{d^{2}\mathbf{E}}{dz^{2}}=-\frac{\omega^{2}}{c^{2}}\boldsymbol{\epsilon}(z)\mathbf{E}\label{7.61}
\end{equation}
 ali v komponentah 
\begin{eqnarray}
-\frac{d^{2}E_{x}}{dz^{2}} & = & (\beta^{2}+\alpha^{2}\cos2qz)E_{x}+\alpha^{2}E_{y}\sin2qz\nonumber \\
-\frac{d^{2}E_{y}}{dz^{2}} & = & \alpha^{2}E_{x}\sin2qz+(\beta^{2}+\alpha^{2}\cos2qz)E_{y}\;,
\end{eqnarray}
 kjer je $\alpha^{2}=\epsilon_{a}\omega^{2}/(2c^{2})$ in $\beta^{2}=\bar{\epsilon}\omega^{2}/c^{2}$.

Ugodno je vpeljati krožni polarizaciji $E_{+}=E_{x}+iE_{y}$ in $E_{-}=E_{x}-iE_{y}$.
Enačbi \ref{7.62} preieta v 
\begin{eqnarray}
-\frac{d^{2}E_{+}}{dz^{2}}=\beta^{2}E_{+}+\alpha^{2}E_{-}e^{2iqz}\nonumber \\
-\frac{d^{2}E_{-}}{dz^{2}}=\alpha^{2}E_{+}e^{-2iqz}+\beta^{2}E_{-}\;.
\end{eqnarray}


Iščemo lastne rešitve valovne enačbe v sredstvu s periodično modulacijo
lomnega količnika. Matematično podoben problem je iskanje lastnih
funkcij elektronov v kristalu. Za te vemo, da morajo imeti Blochovo
obliko, to je, biti morajo produkt periodične funkcije s periodo kristalne
mreže in faktorja exp($ikz$). Matematiki pravijo tej trditvi Floquetov
izrek. Veljati mora tudi v našem primeru. Lastne valove torej poiščimo
v obliki 
\begin{eqnarray}
E_{+} & = & Ae^{i(k+q)z}\nonumber \\
E_{-} & = & Be^{i(k-q)z}\;.
\end{eqnarray}
 Nastavek reši sistem \ref{63}, če $A$ in $B$ rešita sistem homogenih
linearnih enačb 
\begin{eqnarray}
[(k+q)^{2}-\beta^{2}]A-\alpha^{2}B & = & 0\nonumber \\
-\alpha^{2}A+[(k-q)^{2}-\beta^{2}]B & = & 0\;.
\end{eqnarray}
 Sistem je netrivialno rešljiv, če je determinanta kkoeficientov enaka
nič: 
\begin{equation}
(k^{2}+q^{2}-\beta^{2})^{2}-4k^{2}q^{2}-\alpha^{4}=0\;.\label{7.66}
\end{equation}
 $\beta$ in $\alpha$ sta sorazmerna z $\omega$. Dobljena enačba
tako predstavlja disperzijsko relacijo - zvezo med $\omega$ in $k$
- za svetlobo v zavitem sredstvu. Prikazana je na sliki \ref{s7.14}.
Vsaki vrednosti $\omega$, razen v ozkem območju med $\omega_{-}$
in $\omega_{+}$, recimo mu frekvenčna špranja, pripadajo 4 realne
rešitve za $k$, po dve za valovanji v pozitivni in negativni smeri.
V območju špranje je en par rešitev imaginaren. Vsaki vrednosti $k$
pripada neko razmerje koeficientov $A$ in $B$, ki ga izračunamo
iz enčb \ref{7.65} in ki določa polarizacijo lastnega vala. Polarizacije
lastnih valov so v splošnem eliptične in med pri dani frekvenci med
seboj niso pravokotne, kar je posledica tega, da sistem \ref{7.65}
ne predstavlja čisto navadnega problema lastnih vektorjev simetrične
matrike. V območju frekvenčne špranje le en par rešitev predstavlja
potujoč val, drug pa polje, ki eksponento pojema v sredstvo. Zato
se svetloba s frekvenco v špranji in z ustrezno polarizacijo, ki vpada
na holesteričen tekoči kristal, totalno odbije. Pojav je povsem analogen
Braggovemu odboju na kristalih in daje holesterikom značilen obarvan
videz. Več o zanimivih podrobnostih optike holesteričnih tekočih kristalov
najde bralec v \ref{degennes} in nalogah k temu poglavju.(naloga)

Za razlago delovanja zasukane nematske celice zadošča primer, ko je
$q<<\beta$ in $\alpha$, ko je torej perioda sukanja optične osi
velika v primeri z valovno dolžino svetlobe. Tedaj lahko $q$ v disperzijski
zvezi kar zanemarimo, prvi popravek je šele reda $q^{2}$, in dobimo
\begin{equation}
k^{2}=\left\{ \begin{matrix}\beta^{2}+\alpha^{2}=\frac{\omega^{2}}{c^{2}}\epsilon_{\parallel}\end{matrix}\right.\label{7.67}
\end{equation}
 Ti vrednosti, ki ustrezata velikosti valovnega vektorja za izredni
in redni val v običajnem enoosnem kristalu, postavimo v eno od enačb
\ref{7.65} in izračunamo še polarizaciji lastnih valov: $B=\pm A$.

Izračunajmo obe kartezični komponenti električnega polja za prvo rešitev:
\begin{equation}
\begin{array}{lclclcl}
E_{x} & = & \frac{1}{2}(E_{+}+E_{-}) & = & \frac{1}{2}Ae^{ikz}(e^{iqz}+e^{-iqz}) & = & Ae^{ikz}\cos qz\\
E_{y} & = & \frac{1}{2}(E_{+}-E_{-}) & = & \frac{1}{2i}Ae^{ikz}(e^{iqz}-e^{-iqz}) & = & Ae^{ikz}\sin qz
\end{array}\;.\label{7.68}
\end{equation}
 Polarizacija torej res kar sledi optični osi. Druga rešitev da val,
ki je polariziran pravokotno na lokalno optično os in se prav tako
suče z njo. Prvi val se širi s fzano hitrostjo $c/n_{i}$, torej kot
izredni val, drugi pa s $c/n_{r}$, to je kot redni val. Če na zasukano
nematsko celico vpada svetloba, polarizirana ali paralelno ali pravokotno
na optično os ob meji, se pojavi na izhodni strani polarizacija, zasukano
za pravi kot. V primeru, da vpadna polarizacija ne sovpada z eno od
lastnih, jo moramo seveda razstaviti na obe lastni in po prehodu skozi
tekoči kristal zopet sestaviti, s čemer seveda na splošno nastane eliptično
polarizacijo.


\section{Dodatek}

Energija nematskega tekočega kristala je najnižja, kadar ima $\mathbf{n}$
povsod isto smer. Povečanje energije zaradi krajevne odvisnosti $\mathbf{n}$
je v najnižjem redu sorazmerno s $(\nabla\mathbf{n}(\mathbf{r}))^{2}$.
Najsplo\textquotedbl{}nejši izraz za {it orientacijsko elastično
energijo} zapišemo tako, da tvorimo vse neodvisne člene, v katerih
nastopa $(\nabla\mathbf{n}(\mathbf{r}))^{2}$ in ki so invariantni na simetrijske
operacije nematske faze. Sledi\cite{degennes} 
\begin{equation}
F_{e}=\int\left\{ K_{1}(\nabla\cdot\mathbf{n})^{2}+K_{2}[\mathbf{n}\times(\nabla\times\mathbf{n})]^{2}+K_{3}[\mathbf{n}\cdot(\nabla\times\mathbf{n})]^{2}\right\} dV\label{7.70}
\end{equation}
 $K_{1}$, $K_{2}$ in $K_{3}$ so tri Frankove orientacijske elastične
konstante. Prvi člen predstavlja deformacijo v obliki pahljače, drugi
upogib, tretji pa zasuk (slika \ref{s7.20})

V zunanjem električnem polju se energija spremeni. Običajno je neodvisna
električna količina električna poljska jakost, ker je polje posledica
predpisanih napetosti na fiksnih elektrodah. Ustrezni člen v termodinamičnem
potencialu je tedaj $-\mathbf{D}\cdot\mathbf{E}/2$. $\mathbf{E}$ razstavimo
na komponento, paraleleno in pravokotno z $\mathbf{n}$. Dielektrični
tenzor ima v smeri $\mathbf{n}$ vrednost $\epsilon_{\parallel}$, v
pravokotni smeri pa $\epsilon_{\perp}$. Tako je 
\begin{eqnarray}
\mathbf{D} & = & \epsilon_{0}\underline{\epsilon}\mathbf{E}=\epsilon_{0}\epsilon_{\parallel}(\mathbf{n}\cdot\mathbf{E})\mathbf{n}+\epsilon_{0}\epsilon_{\perp}[\mathbf{E}-(\mathbf{n}\cdot\mathbf{E})\mathbf{n}]\nonumber \\
 & = & \epsilon_{0}\epsilon_{a}(\mathbf{n}\cdot\mathbf{E})\mathbf{n}-\epsilon_{0}\epsilon_{\perp}\mathbf{E}\;,
\end{eqnarray}
 kjer je $\epsilon_{a}=\epsilon_{\parallel}-\epsilon_{\perp}$. Drugi
člen je neodvisen od $\mathbf{n}$, zato ga lahko izpustimo. Prosta energija
nematičnega tekočega kristala v električnem polju je tako 
\begin{equation}
F=F_{0}+F_{e}-\frac{1}{2}(\mathbf{n}\cdot\mathbf{E})^{2}\;.\label{7.72}
\end{equation}
 $F_{0}$ predstavlja del proste energije, ki je neodvisen od $\mathbf{n}$
in $\mathbf{E}$. V ravnovesju je prosta energija najmanjša. Kadar je
$\epsilon_{a}>0$, se zato skuša $\mathbf{n}$ postaviti vzporedno s
poljem. Da lahko z minimizacijo $F$ izrazimo $\mathbf{n}(\mathbf{r})$, moramo
poznati še robne pogoje.

Poglejmo primer. Naj bo nematski kristal med dvema paralelnima steklenima
ploščama v razmiku $h$. Na obeh ploščah naj bo $\mathbf{n}$ vzporeden
s površino v isti smeri. Brez zunanjega polja je $\mathbf{n}$ povsod
enako usmerjen. V dovolj velike zunanjem polju, pravokotnem na plošči,
naj bo to smer $z$, dobi $\mathbf{n}$ komponento v smeri $z$: 
\begin{equation}
\mathbf{n}(z)=(n_{1}(z),0,n_{3}(z))\;.\label{7.73}
\end{equation}
 Deformacija $n_{3}$ naj bo majhna. Tedaj je $n_{1}=1n_{3}^{2}/2$.
Robna pogoja sta $n_{3}(0)=n_{3}(h)=0$. Približno rešitev, ki ustreza
robnima pogojema, poiščemo z nastavkom 
\begin{equation}
n_{3}(z)=a\sin qz\;,\mbox{\hskip1cm}q=\frac{\pi}{h}\;.\label{7.74}
\end{equation}
 Ta nastavek je prvi člen razvoja prave rešitve v Fourierovo vrsto.
Velja 
\begin{equation}
\nabla\times\mathbf{n}=(0,-\frac{dn_{1}}{dz},0)\label{7.75}
\end{equation}
 in 
\begin{equation}
\mathbf{n}\times(\nabla\times\mathbf{n})=(n_{3}\frac{dn_{1}}{dz},0,n_{1}\frac{dn_{3}}{dz})\simeq(0,0,n_{3}\frac{dn_{3}}{dz})\label{7.76}
\end{equation}
 do drugega reda v $n_{3}$. Prosta energija je tako 
\begin{eqnarray}
F & = & \frac{1}{2}\int\left[K_{1}\left(\frac{dn_{3}}{dz}\right)^{2}+K_{2}n_{3}^{2}\left(\frac{dn_{3}}{dz}\right)^{2}-\epsilon_{0}\epsilon_{a}(n_{3}E)^{2}\right]dz=\nonumber \\
 & = & \frac{1}{2}\int_{0}^{h}[K_{1}q^{2}a^{2}\cos^{2}qz+K_{3}q^{2}a^{4}\sin^{2}qz\cos^{2}qz-\epsilon_{0}\epsilon_{a}Ea^{2}\sin^{2}qz]dz=\nonumber \\
 & = & \frac{\pi}{4q}[K_{1}q^{2}a^{2}+\frac{1}{4}K_{3}q^{2}a^{4}-\epsilon_{0}\epsilon_{a}Ea^{2}]\;.
\end{eqnarray}
 Iščemo amplitudo deformacije $a$, pri kateri je prosta energija
najmanjša. Tedaj mora biti $a$ rešitev enačbe 
\begin{equation}
(K_{1}q^{2}-\epsilon_{0}\epsilon_{a}E)a+\frac{1}{2}K_{3}q^{2}a^{3}=0\;.\label{7.78}
\end{equation}
 Rešitvi sta 
\begin{equation}
a=0\mbox{\hskip1cm in \hskip1cm}a^{2}=2\frac{\epsilon_{0}\epsilon_{a}E-K_{1}q^{2}}{K_{3}q^{2}}\;.\label{7.79}
\end{equation}
 Pri majhnih poljih, ko je $\epsilon_{0}\epsilon_{a}E<K_{1}q^{2}$,
je fizikalno smiselna le prva rešitev, ko defomacije ni. Pri večjih
poljih pa je stabilna druga rešitev. Ko večamo polje, deformacija
v sredini plasti hitro naraste, tako da se $\mathbf{n}$ postavi skoraj
popolnoma v smer zunanjega polja. Tedaj naša rešitev seveda ni dobra,
saj smo računali, kot da je $n_{3}<<1$. Prehodu iz nedefromiranega
stanja v deformirano stanje pravimo tudi Fredericksov prehod. Na njem
je osnovano preklaplanje optičnih kazalnikov na nematske tekoče kristale.

V primeru zasukane nematske celice je prehod podoben, le račun je
nekoliko bolj zapleten, ker ima deformacija vse tri komponente, tudi
zasuk (Naloga).