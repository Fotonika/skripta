\chapterimage{slike/Navy.jpg} 
\chapter{Laser}

V prejšnjih poglavjih smo spoznali resonatorje, pojasnili proces ojačevanja svetlobe in 
opisali črpanje, ki je potrebno za vzdrževanje obrnjene zasedenosti v ojačevalnem sredstvu. 
V tem poglavju bomo vsa ta
spoznanja združili in komponente sestavili v eno samo napravo -- laser. Zapisali
bomo zasedbene enačbe, pojasnili delovanje laserjev in spoznali prednosti 
laserske svetlobe pred svetlobo iz navadnih svetil. Opisali bomo način delovanja 
sunkovnih laserjev in na koncu predstavili semiklasični model laserja. 

\section{Laser}
\index{Laser}\index{Spontano sevanje}
\index{Optično ojačevanje}
V sredstvu, v katerem med dvema nivojema dosežemo obrnjeno 
zasedenost, se svetloba z ustrezno valovno dolžino ojačuje. 
Postavimo tako sredstvo v optični resonator. \index{Resonator} 
Na začetku nastaja predvsem spontano izsevana svetloba in 
v resonatorju se vzbujajo tista lastna nihanja, katerih frekvenca je blizu frekvence
atomskega prehoda. Resonator poskrbi, da se svetloba odbija nazaj v ojačevalno 
sredstvo. Če v njem vzdržujemo obrnjeno zasedenost, se svetloba ob prehodu skozi 
sredstvo ojačuje.
V začetku je ojačenje za
izbrana nihanja veliko, z naraščajočo intenziteto pa se ojačenje zmanjšuje.
\index{Izgube v resonatorju} 
Ko se ojačenje na prelet izenači z izgubami, sistem preide v stacionarno stanje in 
seva močno koherentno svetlobo. Tak
izvor svetlobe imenujemo laser. Beseda laser je nastala iz kratice za {\it Light
Amplification by Stimulated Emission of Radiation} --  ojačenje svetlobe s
stimuliranim sevanjem.\footnote{R. G. Gould, {\it The Ann Arbor Conference on Optical Pumping: 
the University of Michigan}, ur. P. A. Franken in R. H. Sands (1959).}

\begin{figure}[h]
\centering
\def\svgwidth{90truemm} 
\input{slike/06_shema.pdf_tex}
\caption{Shema laserja s ključnimi deli: ojačevalno sredstvo, črpalni mehanizem in resonator.
Odbojnost izhodnega zrcala $\mathcal{R}_1<1$, da svetloba lahko zapusti resonator.}
\label{fig:shemalaserja}
\end{figure}
\vglue-6truemm
\begin{remark}
Kot klasično analogijo za laser vzamemo klarinet, ki je sestavljen iz 
cevi in ustnika. Cev deluje kot resonator, v katerem nastane 
stoječi zvočni val, pri čemer je frekvenca stoječega vala določena z 
dolžino cevi in s številom vozlov. Naloga ustnika je dovajanje energije 
in s tem vzdrževanje konstantne amplitude nihanja. To glasbenik doseže s 
pihanjem v ustnik in tresenjem prožnega jezička, ki s tresljaji proizvaja 
zvok. Tresenje jezička je približno periodično in vsebuje mnogo različnih 
frekvenc, tudi take, ki ustrezajo frekvenci stoječih valov v cevi. 
Ko amplituda tlaka v cevi naraste nad neko mejo, nastopi zanimiv
pojav. Nihanje tlaka v gornjem koncu cevi povratno deluje na ustnik
in ga sili, da niha s frekvenco najmočneje vzbujenega stoječega vala v cevi,
druge frekvence pa zamrejo. Moč pihanja gre le še v
nihanje jezička s pravo frekvenco in ojačuje nihanje zračnega stolpca. 
Tako se s povratno zvezo med nihanjem jezička in stoječim valovanjem v cevi
vzdržuje stoječe valovanje s konstantno amplitudo. 
\end{remark}

V grobem ima laser tri ključne sestavne dele: ojačevalno sredstvo, 
črpalni mehanizem in resonator, ki je v najpreprostejšem primeru sestavljen iz dveh 
ukrivljenih zrcal (slika~\ref{fig:shemalaserja}). Črpalni \index{Črpanje}
mehanizem vzdržuje obrnjeno zasedenost v ojačevalnem sredstvu, medtem ko resonator 
omogoča, da se svetloba med številnimi prehodi skozi ojačevalno sredstvo dovolj ojači.
Odbojnost vsaj enega od zrcal 
mora biti manjša od 1, da skozenj lahko izhaja svetloba.

Zaenkrat se omejimo na najpreprostejši model laserja in 
privzamemo, da frekvenca le enega resonatorskega lastnega nihanja sovpada s 
frekvenco prehoda aktivne snovi. Ta predpostavka v večini laserjev ni
avtomatično izpolnjena, vendar jo je pogosto mogoče doseči z dodatnimi elementi 
v resonatorju. Aktivno snov oziroma ojačevalno sredstvo v laserju stalno 
črpamo in s tem vzdržujemo obrnjeno zasedenost. 

Naj bo $W$ energija svetlobnega valovanja v resonatorju. Zaradi izgub skozi
zrcali, absorpcije in sipanja v resonatorju se energija na en obhod 
resonatorja zmanjša za (enačba~\ref{eq:Lambda})
\begin{equation}
\Delta W_{\rm izgube}=-\Lambda W=-\left(1-{\cal {R}}_{1}+1-{\cal {R}}_{2}
+2\alpha L\right)W,
\label{5.1}
\end{equation}
kjer so $\Lambda $ celotne izgube, $\alpha$ izgube na enoto poti zaradi
absorpcije in sipanja, $L$ je dolžina resonatorja ter \index{Izgube v resonatorju}
${\cal {R}}_{1}$ in ${\cal {R}}_{2}$ odbojnosti
zrcal. V ojačevalnem sredstvu, v katerem vzdržujemo obrnjeno zasedenost,
pride do ojačevanja s stimuliranim sevanjem. Energija nihanja resonatorja 
se tako na en obhod po enačbi (\ref{eq:djG}) poveča za 
\index{Ojačevanje s stimuliranim sevanjem|see{Stimulirano sevanje}}
\begin{equation}  
\Delta W_{\rm oja\check{c}enje}=\frac{G}{1+W/W_s}\,W\, 2L'.
\label{5.2}
\end{equation}
Vpeljali smo saturacijsko energijo $W_s=Vj_s/c$ in $G$ kot koeficient ojačenja.
Dolžino ojačevalnega sredstva,
ki se v splošnem razlikuje od dolžine resonatorja $L$, smo označili z $L'$.
\index{Saturacijska energija}
Zapis sicer pogosto poenostavimo in vzamemo $L'=L$, vendar tukaj zaradi
jasnosti obdržimo ločen zapis. Privzeli smo tudi, da je ojačenje na en
obhod dovolj majhno, da enačbe (\ref{eq:djG}) ni treba integrirati.

V stacionarnem stanju se zmanjšanje energije zaradi izgub ravno izenači 
s povečanjem energije zaradi ojačenja. Zapišemo
\beq
|\Delta W_{\rm izgube}|=|\Delta W_{\rm oja\check{c}enje}|,
\eeq
od koder sledi
\begin{equation}  
\Lambda\, W=\frac{G\,2L'}{1+W/W_s}\,W.
\label{5.3}
\end{equation}
\begin{figure}[h]
\centering
\def\svgwidth{140truemm} 
\input{slike/06_stacio.pdf_tex}
\caption{Za majhne vrednosti ojačenja $G$ ima enačba~(\ref{5.3}) eno samo 
rešitev, to je pri $W=0$ (a). Pri večjih ojačenjih ima tudi neničelno rešitev (b).}
\label{fig:stacio}
\end{figure}

Enačba~(\ref{5.3}) ima pri majhnem ojačenju $G$ eno samo rešitev, to je 
$W=0$ (slika~\ref{fig:stacio}). Pri večjih vrednostih ojačenja $G$ obstaja še neničelna rešitev
za energijo svetlobnega nihanja 
\boxeq{5.4}{ 
W=W_s \left(\frac{G}{G_\mathrm{pr}}-1\right),
}
pri čemer smo vpeljali ojačenje na pragu delovanja\index{Prag delovanja laserja}
\boxeq{5.5}{
G_\mathrm{pr} = \frac{\Lambda}{2L'}.
}
Združimo obe rešitvi: energija svetlobe v laserju je pri ojačenju, ki je manjše
od ojačenja na pragu delovanja $G_\mathrm{pr}$, enaka
nič. Nad pragom energija svetlobe linearno narašča z ojačenjem $G$ (slika~\ref{fig:energija}).
Ojačenje je seveda odvisno od stopnje obrnjene zasedenosti, ki je povezana
z močjo črpanja.
\begin{figure}[h]
\centering
\def\svgwidth{60truemm} 
\input{slike/06_energija.pdf_tex}
\caption{Odvisnost energije svetlobe v laserju $W$ od ojačenja $G$. 
Pod pragom $G_\mathrm{pr}$ je energija enaka nič, 
nad pragom pa linearno narašča z ojačenjem.}
\label{fig:energija}
\end{figure}

Izhodna moč laserja je enaka energiji, ki zapusti\index{Izhodna moč laserja}
resonator skozi izhodno zrcalo, deljeni s časom obhoda resonatorja $2L/c$ 
\boxeq{5.6}{
P=(1-{\cal {R}}_{1})\frac{c}{2L}\,W.
}
Ker so vsi predfaktorji v enačbi konstantni, je izhodna moč kar sorazmerna
energiji svetlobe v resonatorju. Odvisnost izhodne moči laserja od črpanja je 
tako do konstante enaka energiji, prikazani na sliki~\ref{fig:energija}. 

Gostoto svetlobnega toka, ki izhaja iz laserja, zapišemo 
\begin{equation}
 j = \frac{P}{S} = \frac{1}{2} (1-{\cal {R}}_{1}) j_s \left(\frac{G}{G_\mathrm{pr}}-1\right).
\end{equation}
Zapisana enačba seveda velja za ojačenja nad pragom, $G>G_\mathrm{pr}$.

\begin{naloga}
Izračunaj izhodno moč iz laserja pri dani dolžini resonatorja ($L=L'$), 
odbojnosti enega zrcala ${\cal {R}}_{2}=1$, 
notranjih izgubah na enoto dolžine $\alpha$ in ojačenju $G$. Pokaži, da
je izhodna moč največja pri odbojnosti izhodnega zrcala 
\beq
{\cal {R}}_1 = 1-2\alpha L \left(\sqrt{\frac{G}{\alpha}}-1\right).
\eeq
\end{naloga}

\section{Zasedbene enačbe}
\index{Zasedbene enačbe}
Za podrobnejši opis delovanja laserja zapišemo zasedbene enačbe. 
Enačbam za zasedenost atomskih nivojev \index{Trinivojski sistem}
v trinivojskem sistemu (enačbe~\ref{4.39.1}--\ref{4.39}) dodamo še enačbo za 
energijo lastnega nihanja v resonatorju. Še naprej obravnavamo primer, ko je 
vzbujeno le eno lastno resonatorsko nihanje, in opazujemo prehode  med prvim 
in drugim vzbujenim atomskim stanjem (slika~\ref{fig:3nivojski}\,b). 

Preden zapišemo enačbe, napravimo še nekaj poenostavitev. Najprej privzamemo, 
da je razpadni čas spodnjega atomskega stanja $|1\rangle$, 
ki ga določa koeficient $A_{10}$, dosti krajši od razpadnega
časa zgornjega stanja $|2\rangle$. Tedaj vsi atomi iz spodnjega stanja zelo 
hitro preidejo v osnovno stanje in $N_1 \approx 0$, če le ni preveč
stimuliranega sevanja. Zanemarimo tudi spontano sevanje iz drugega vzbujenega
nivoja $A_{20} \approx 0$. Celoten sistem potem opišemo z dvema 
\index{Obrnjena zasedenost}
spremenljivkama: prva je $N_2$, ki označuje zasedenost drugega vzbujenega
stanja in hkrati približno obrnjeno zasedenost; druga je $n$, ki pove število
fotonov v izbranem lastnem nihanju resonatorja. Število fotonov določa energijo 
polja v resonatorju, ki je enaka $W = \hslash\omega n$ (enačba~\ref{4.11}). Ustrezna
gostota energije je $w = \hslash\omega n/V$, pri čemer je $V$ volumen resonatorja.

Zasedbeni enačbi sta
\begin{equation}
\frac{dN_2}{dt}=rN-A_{21}N_2-B_{21}gN_2\frac{\hslash \omega}{V}\,n
=rN-A_{21}N_2-\frac{\sigma c}{V}\, N_2\,n
\label{5.7}
\end{equation}
in \index{Izgube v resonatorju}\index{Življenjski čas nihanj}
\begin{equation}
\frac{dn}{dt}=\frac{\sigma c}{V}\, N_2\,(n+1)-\frac{2}{\tau}\,n.
\label{5.8}
\end{equation}
Prva enačba sledi neposredno iz enačbe~(\ref{4.39}) ob upoštevanju zgoraj navedenih
poenostavitev. Drugo  dobimo s sledečim razmislekom. Energija svetlobe 
oziroma število fotonov v resonatorju se povečuje predvsem 
zaradi stimuliranega sevanja, opisanega z zadnjim členom v enačbi~(\ref{5.7}).
Po drugi strani vemo, da je verjetnost za prehod atoma iz višjega v nižje stanje z 
izsevanjem fotona v izbrano stanje elektromagnetnega polja sorazmerna z 
$n+1$ (enačba~\ref{4.59}), kjer je $n$ število fotonov v izbranem stanju. 
Če torej namesto $n$ v zadnjem členu enačbe~(\ref{5.7}) pišemo $n+1$, 
opišemo poleg stimuliranega sevanja tudi prispevek spontanega sevanja. Dodamo še 
člen, s katerim popišemo zmanjševanje energije svetlobe v resonatorju zaradi 
izgub, kar opišemo z razpadnim časom $\tau/2$ (enačba~\ref{eq:dW}). 
\index{Ojačenje s stimuliranim sevanjem|see{Stimulirano sevanje}}
\index{Koeficient ojačenja}

Enačbi~(\ref{5.7} in \ref{5.8}) predstavljata sistem dveh sklopljenih diferencialnih enačb za 
časovni razvoj števila fotonov v resonatorskem stanju in za zasedenost 
zgornjega atomskega stanja. Enačbi sta nelinearni in nimata  
analitične rešitve. Vseeno lahko nekaj povemo o takem sistemu.

Poglejmo najprej stacionarne rešitve, za katere velja $dN_2/dt=0$ in 
$dn/dt=0$. Iz enačbe~(\ref{5.7}) izrazimo $N_{2}$ in ga vstavimo v enačbo~(\ref{5.8}).
Sledi 
\begin{equation}
\frac{2}{\tau }n\left(A_{21}V+\sigma c\,n\right)=
\sigma c\, r\,N\,(n+1).
\label{5.9}
\end{equation}
Enačbo zapišemo bolj pregledno, če vpeljemo koeficient ojačenja $G$ (enačba~\ref{4.44})
in ojačenje na pragu $G_\mathrm{pr}$ (enačbi~\ref{taulambda} in \ref{5.5})
\beq
G_\mathrm{pr}\, n\, \left(1+\frac{\sigma c}{VA_{21}}n \right)= G(n+1).
\label{5.9.a}
\eeq
Vpeljemo brezdimenzijsko konstanto $p$, pri čemer upoštevamo zvezo
med Einsteinovimi koeficienti (enačba~\ref{4.27})
\begin{equation}
p=\frac{VA_{21}}{\sigma c} = 
\frac{VA_{21}}{B_{21}\hslash \omega g}=\frac{V\omega ^{2}}{\pi
^{2}c^{3}g}\approx
\frac{V\omega ^{2}}{\pi ^{2}c^{3}}\Delta \omega.  
\label{5.10}
\end{equation}
V zadnjem koraku smo privzeli, da je $g\approx 1/\Delta \omega $. 
Parameter $p$ je približno enak produktu 
gostote stanj elektromagnetnega polja v resonatorju (enačba~\ref{4.4}),
širine atomskega prehoda in volumna, torej kar številu vseh stanj 
v frekvenčnem intervalu atomskega prehoda. To število je navadno precej 
veliko $p \sim 10^{8}$--$10^{10}$. 

\begin{naloga}
Primerjaj izraz za $p$ (enačba~\ref{5.10}) z izrazom za saturacijsko 
gostoto toka $j_s$ (enačba~\ref{eq:jsatg}). Pokaži, da velja
\begin{equation}
p = \frac{W_s}{\hslash \omega}.
\end{equation}
Parameter $p$ je torej enak številu fotonov v resonatorju, pri katerem pride 
do nasičenja ojačenja, če je frekvenca nihanja resonatorja blizu 
centra atomske črte. 
\end{naloga}

Enačbo~(\ref{5.9.a}) prepišemo in dobimo
\begin{equation}
\frac{n^2}{p}-\left(\frac{G}{G_\mathrm{pr}}-1\right)\,n-\frac{G}{G_\mathrm{pr}}=0
\label{5.11}
\end{equation}
s pozitivno rešitvijo 
\begin{equation}
n=\frac{p}{2}\left( \left(\frac{G}{G_\mathrm{pr}}-1\right)+\sqrt{\left(\frac{G}{G_\mathrm{pr}}
-1\right)^{2}+ \frac{4G}{p\,G_\mathrm{pr}}}\right).
\label{5.12}
\end{equation}
Ker je $p$ zelo veliko število, lahko koren razvijemo, če le ni ojačenje
preveč blizu praga, ko je $G/G_\mathrm{pr}\sim 1$. 
\index{Prag delovanja laserja} 

Pod pragom je $G<G_\mathrm{pr}$ in 
\begin{equation}
n\approx \frac{p}{2}\left( \left(\frac{G}{G_\mathrm{pr}}-1\right)+\left(1
-\frac{G}{G_\mathrm{pr}}\right)+\frac{2G}{p(G_\mathrm{pr}-G)}\right) =\frac{G}{G_\mathrm{pr}-G}.
\label{5.13}
\end{equation}

Nad pragom je
število fotonov 
\boxeq{5.14}{
n\approx p\left(\frac{G}{G_\mathrm{pr}}-1\right) = \frac{W_s}{\hslash \omega} \left(\frac{G}{G_\mathrm{pr}}-1\right).
}
Rezultat, ki je po pričakovanju skladen z enačbo~(\ref{5.4}), si oglejmo podrobneje. 
Pod pragom so izgube večje od črpanja in gre praktično vsa moč, ki jo dovedemo v sistem, 
s spontanim sevanjem v veliko število stanj elektromagnetnega polja. 
Število fotonov v izbranem resonatorskem nihanju je tako okoli 
ena vse do neposredne bližine praga. Nad pragom povsem prevlada stimulirano sevanje 
v eno samo izbrano nihanje resonatorja in število fotonov je reda velikosti $p$ (slika~\ref{fig:p}).
Prehod čez prag je zaradi velikega $p$ tako hiter, da ga ni mogoče izmeriti.
Izjema so polprevodniški laserji, \index{Laser!polprevodniški}katerih 
volumen -- in posledično tudi $p$ -- 
je tako majhen, da je mogoče opaziti zvezen prehod čez prag.
\begin{figure}[h]
\centering
\def\svgwidth{50truemm} 
\input{slike/06_p.pdf_tex}
\caption{Odvisnost števila fotonov v resonatorju od ojačenja za $p=10^5$. 
Pod pragom je število fotonov majhno, ob pragu skokovito naraste in tudi 
pri močnem črpanju ostaja reda velikosti $p$.}
\label{fig:p}
\end{figure}

Izračunajmo še stacionarno zasedenost zgornjega atomskega nivoja. Iz 
enačbe~(\ref{5.8}) sledi
\boxeq{5.15}{
N_2=\frac{2V}{\tau \sigma c}\frac{n}{n+1} \approx \frac{2V}{\tau \sigma c}.
}
Na pragu je po enačbi (\ref{5.12}) $n=\sqrt{p}$. 
Sledi 
\begin{equation}  
N_{\rm 2pr}=\frac{2V}{\tau \sigma c}\frac{\sqrt{p}}{\sqrt{p}+1}.
\label{5.16}
\end{equation}
Ker je $p$ zelo veliko število, lahko zasedenost zgornjega
nivoja (oziroma obrnjena zasedenost) narašča le do bližine praga. Nad pragom ostaja praktično
konstantna in skoraj natanko enaka kot na pragu. Tega ni težko razumeti. Nad
pragom je število fotonov v resonatorju veliko in linearno narašča 
z močjo črpanja. S tem se povečuje hitrost praznjenja zgornjega atomskega 
stanja s stimuliranim sevanjem, kar ravno izniči učinek povečanja črpanja. 
V stacionarno delujočem laserju obrnjene zasedenosti torej ni mogoče povečati 
nad vrednost na pragu $N_{\rm 2pr}$. To ima pomembne praktične posledice, kot bomo
videli v nadaljevanju.

\begin{remark}
Obravnava laserja z zasedbenimi enačbami je seveda zelo groba. Nismo
upoštevali, da je prostorska odvisnost polja v delujočem laserju 
drugačna od lastnega nihanja praznega resonatorja. Poleg tega smo
privzeli, da so atomi lahko le v lastnih energijskih stanjih, kar je res le
v primeru stacionarnih stanj brez zunanjega, časovno odvisnega polja
svetlobe. Bolj podroben pristop je semiklasični model, pri katerem 
za opis svetlobe uporabimo klasično valovno enačbo, za atome
pa kvantno mehaniko (glej razdelek~\ref{chap:semiklasicni}). Ta model
zadošča za opis skoraj vseh pojavov v laserjih razen vpliva spontanega sevanja. 
Za dosledno obravnavo tega je treba svetlobo opisati s pomočjo 
kvantne elektrodinamike.
\end{remark}

Povzemimo na kratko, kaj smo spoznali o delovanju 
enofrekvenčnega laserja. Pri dovolj velikem ojačenju, ki ravno pokriva izgube 
resonatorja, je v stacionarnem stanju energija in s tem amplituda 
izbranega lastnega nihanja resonatorja različna od nič. Frekvenca svetlobe je
določena z izbranim lastnim nihanjem resonatorja, ki določa tudi prostorsko
odvisnost valovanja v resonatorju in obliko izhodnega snopa. V navadnem stabilnem 
resonatorju je polje po obliki zelo blizu Gaussovemu snopu, zato je tak tudi izhodni snop.
Gaussova prostorska odvisnost izhodnega snopa je morda najpomembnejša lastnost
laserjev. Gaussov snop se, kot vemo, najmanj širi zaradi uklona in ga je mogoče
zbrati v piko reda velikosti valovne dolžine. Laser se tako najbolj 
približa idealno točkastemu izvoru svetlobe.

\section{Spektralna širina enega laserskega nihanja}
\index{Spektralna črta}
Povejmo še nekaj o spektralni širini svetlobe enofrekvenčnega laserja. 
Če bi se lastno stanje 
elektromagnetnega polja v resonatorju obnašalo kot klasično 
harmonsko nihalo, bi bil spekter laserja neskončno ozek. Vendar 
imajo laserji končno spektralno širino -- v idealnem primeru zaradi
kvantizacije elektromagnetnega polja, v praksi pa zaradi zunanjih motenj.
Poskusimo oceniti razširitev zaradi vpliva kvantizacije. Zaradi nje
je poleg stimuliranega sevanja vedno prisotno tudi spontano sevanje. To 
predstavlja kvantni šum, ki povzroči razširitev spektra. 

Predstavimo amplitudo nihanja $E(t)$ na izbranem mestu v resonatorju kot kompleksno 
število, ki ga v kompleksni ravnini določata dolžina $|E(t)|$ in faza
$\varphi$ (slika~\ref{fig:fazor}). 
Pri tem fazo določimo glede na neko začetno izbrano fazo. 
Ker je energija svetlobe sorazmerna s številom fotonov, je dolžina $|E(t)|$
sorazmerna s korenom iz števila fotonov v izbranem lastnem nihanju. 
Stimulirano sevanje, ki ravno pokriva izgube resonatorja, vzdržuje
dolžino $|E(t)|$ praktično konstantno, nespremenjena ostaja 
tudi faza. Spontano sevanje velikosti amplitude nihanja ne spreminja dosti, 
vendar stohastična narava spontano izsevanih fotonov vpliva na njeno fazo.
Majhen prispevek spontanega sevanja zaradi spreminjajoče se faze
skrajša koherenčni čas in določa spodnjo mejo za širino spektralne črte. 

\begin{figure}[h]
\centering
\def\svgwidth{70truemm} 
\input{slike/06_fazor.pdf_tex}
\caption{Amplituda polja v resonatorju in njena sprememba zaradi 
spontanega sevanja}
\label{fig:fazor}
\end{figure}

Pri spontani emisiji se izseva en foton s poljubno fazo. Prispevek h kompleksni
amplitudi ima torej dolžino 1 in poljubno smer (slika~\ref{fig:fazor}). Zanima
nas povprečje kvadrata spremembe faznega kota pri enem spontano izsevanem fotonu
\begin{equation}
\overline{(\Delta \varphi_{1})^{2}}=\overline{\left(\frac{\cos\psi}{\sqrt{n} }\right)^2}
=\frac{1}{2\overline{n}},
\label{5.17}
\end{equation}
pri čemer je kot $\psi$ označen na sliki~\ref{fig:fazor}. 
Zaporedne spontane emisije so med seboj neodvisne, zato izračunamo
povprečni kvadrat spremembe faze pri $m$ emisijah tako, da seštejemo
povprečne kvadrate za posamezne fotone 
\begin{equation}
\overline{\Delta \varphi_{m}^{2}}=m\overline{\Delta \varphi_{1}^{2}}=
\frac{m}{2\overline{n}}.
\label{5.18}
\end{equation}
Ocenimo še število spontano izsevanih fotonov na časovno enoto.
Vemo, da stimulirano sevanje ravno pokrije izgube resonatorja, zato je
stimulirano izsevanih fotonov na časovno enoto $2\overline{n}/\tau $. Vemo tudi, 
da je razmerje med verjetnostjo za stimulirano in spontano sevanje enako \index{Stimulirano sevanje}
številu fotonov v danem stanju polja (enačba~\ref{4.56}), zato je število 
spontanih sevanj na časovno enoto kar $2/\tau $.\index{Spontano sevanje}
Tako je število spontano izsevanih fotonov v času $t$ enako $m=2t/\tau $ in 
\begin{equation}
\overline{\Delta \varphi^{2}(t)}=\frac{t}{\overline{n}\tau }.
\label{5.19}
\end{equation}
Čas $t_{p}$, v katerem se faza znatno spremeni, je torej
velikostnega reda 
\begin{equation}
t_{p}\sim \overline{n}\tau =\frac{W}{\hslash \omega }\,\tau =\frac{P}{\hslash
\omega }\tau ^{2}.
\label{5.20}
\end{equation}
Ker je število fotonov v izbranem nihanju nad pragom zelo veliko ($\sim 10^9$ v majhnem 
He-Ne laserju) in $\tau~\sim 10^{-7}$, je karakteristični\index{Laser!He-Ne}
čas za fazno razširitev idealnega laserja $t_p \sim 100~\si{s}$. 

Iz enačbe (\ref{5.20}) vidimo tudi, da je spektralna širina, ki je
podana z $1/t_{p}$, obratno sorazmerna z izhodno močjo laserja. Spodnjo
mejo za spektralno širino pri dani izhodni moči laserja 
podaja Schawlow-Townseva limita\footnote{A. 
L. Schawlow in C. H. Townes, Phys. Rev. {\bf 112}, 1940 (1958). }
\beq
\Delta \nu_\mathrm{min} = \frac{ \pi h \nu}{P} \Delta \nu_R^2,
\eeq
pri čemer $\Delta \nu_R$ predstavlja širino nihanja praznega 
resonatorja.\footnote{Natančnejši izračun odstopa od preprosto izpeljanega za 
faktor 2. V zapisanem izrazu je že upoštevan pravilen predfaktor.}
V neposredni bližini praga, kjer je $\overline{n}\sim 1$, je 
spektralna širina približno enaka širini nihanja praznega resonatorja.

Dejanski laserji seveda nimajo niti približno tako ozkega spektra, kot smo ga
pravkar ocenili. Vemo, da je frekvenca laserja določena z dolžino resonatorja 
($\nu=N c/2L$), pri čemer je $N$ zelo veliko celo število. Že majhna
sprememba dolžine resonatorja povzroči spremembo frekvence laserja, pri 
znatnejši spremembi dolžine lahko preskoči tudi vzbujeno
nihanje v resonatorju, torej se spremeni število $N$. Dolžina resonatorja se 
spreminja predvsem zaradi zunanjih mehanskih motenj in zaradi spreminjanja
temperature. Če se posebej ne potrudimo s konstrukcijo resonatorja, so 
fluktuacije frekvence kar reda velikosti razmika med sosednimi stanji 
resonatorja, to je reda velikosti $\sim 100~\si{MHz}$. 
Fluktuacije dolžine je mogoče zmanjšati s skrbno konstrukcijo, 
temperaturno stabilizacijo in uporabo materialov z majhnim toplotnim raztezkom. 
Na tak način je mogoče narediti laser z efektivno spektralno širino pod $\sim 1~\si{MHz}$.

\begin{remark}
Zapisali smo, da lahko s posebno konstrukcijo laserjev dosežemo
spektralno širino pod $\sim 1~\si{MHz}$. Vendar najmanjša spektralna širina
znaša $\sim 10~\si{mHz}$, kar je še 8 velikostnih redov manj! Za to je potreben prav poseben 
laser iz monokristalov silicija, hlajen na $-150~\si{\celsius}$. Fluktuacije
dolžine resonatorja so pogojene s termičnimi fluktuacijami v odbojnih plasteh, 
ki znašajo okoli $10^{-17}\si{\metre}$. Koherenčna dolžina takega laserja je več
milijonov kilometrov.\footnote{D. G. Matei et al., Phys. Rev. Lett. {\bf 118}, 263202 (2017).} 
\end{remark}

Velja opozoriti, da je narava spektralne razširitve v laserju 
drugačna kot v navadnih svetilih. V drugem poglavju smo videli, da 
intenziteta svetlobe navadnega svetila fluktuira na časovni skali 
koherenčnega časa, ki je obraten spektralni širini (razdelek~\ref{chap:kns}). 
Šum navadnih svetil je torej amplitudno moduliran šum. Pri \index{Šum}
enofrekvenčnem laserju je drugače. Amplituda in intenziteta 
izhodne svetlobe sta konstantni, fluktuira le frekvenca oziroma
faza. Šum laserja je torej v obliki frekvenčne modulacije.

\section{Primerjava laserjev in navadnih svetil}
Primerjajmo enofrekvenčni laser, v katerem je vzbujeno eno lastno nihanje
resonatorja Gaussove oblike, z navadnim nekoherentnim izvorom svetlobe.

Svetlobni snop iz laserja ima dve takoj očitni odliki: je zelo\index{Laser!enofrekvenčni}
usmerjen in zelo enobarven. Prva lastnost je posledica tega, da je
lastno nihanje stabilnega resonatorja Gaussove oblike in je zato tak tudi
izhodni snop. Divergenca Gaussovega snopa je posledica uklona in je 
najmanjša možna. Valovne fronte izhodne svetlobe so gladke in na dani oddaljenosti ves 
čas enake, zato je laserski snop prostorsko idealno koherenten. 
Koherenten Gaussov snop lahko z ustrezno optiko zberemo v piko velikosti
valovne dolžine, s čimer dosežemo že pri razmeroma majhni izhodni moči zelo veliko
gostoto svetlobnega toka. To je zelo uporabno v tehnologiji, na primer za natančno in
čisto obdelavo materialov, ter v medicini za
zahtevne kirurške posege.

Kako pa je z navadnimi svetili? V njih atomi sevajo neodvisno, zato
izsevana svetloba ni prostorsko koherentna. Valovna fronta na danem 
mestu je nepravilna in se znotraj koherenčnega časa znatno spremeni. 
Vendar tudi iz svetlobe navadnega nekoherentnega svetila lahko pridobimo
koherenten snop, če na dano razdaljo od svetila postavimo zaslonko, ki
je manjša od koherenčne ploskve na tistem mestu (glej 
razdelek~\ref{Prostorska-koherenca}). Ocenimo moč tako dobljenega
koherentnega snopa za zaslonko.\index{Koherenčna ploskev}

Svetilo naj ima svetlost\footnote{Svetilnost je moč, izsevana v dan 
prostorski kot; svetlost je svetilnost na enoto ploskve
$B= dP/Sd\Omega$.} $B$.  
Moč koherentnega snopa za zaslonko, ki prepušča svetlobo skozi prostorski kot 
$\Delta\Omega$, je (slika~\ref{fig:svetlost})\index{Svetilnost}\index{Svetlost}
\begin{equation}
P=BS_{0}\Delta \Omega =\frac{BS_{0}S_{c}}{z^{2}}\sim \frac{BS_{0}}{z^{2}}\,
\frac{\lambda ^{2}z^{2}}{S_{0}}=B\lambda ^{2}.
\label{5.21}
\end{equation}
Pri tem je $S_{0}$ površina svetila, $z$ oddaljenost zaslonke od svetila in
$S_{c}$ velikost koherenčne ploskve, za katero smo uporabili oceno 
(enačba~\ref{eq:koherencna-ploskev}). Da iz $S_0=1~\si{mm}^2$ velikega svetila 
dosežemo koherenten snop svetlobe z valovno dolžino okoli $550~\si{nm}$, 
mora biti premer zaslonke, ki jo postavimo $1~\si{m}$ od izvora, 
okoli $0,6~\si{mm}$. V tem primeru znaša 
moč, ki prehaja skozi zaslonko, pri svetlosti $100~\si{W/cm^{2}}$ 
le približno $3\cdot10^{-7}~\si{W}$.
Pri podobni divergenci snopa je močno navadno svetilo torej štiri rede
velikosti šibkejše od šibkih laserjev z močjo $1~\si{mW}$. 
\begin{figure}[h]
\centering
\def\svgwidth{100truemm} 
\input{slike/06_svetlost.pdf_tex}
\caption{K izračunu moči koherentnega snopa svetlobe iz nekoherentnega svetila. $S_0$
je površina svetila, $z$ oddaljenost do zaslona, $\Delta \Omega$ prostorski kot odprtine
 in $S_c$ velikost koherenčne ploskve na zaslonu.}
\label{fig:svetlost}
\end{figure}

Druga odlična lastnost svetlobe iz enofrekvenčnega laserja je njena zelo majhna
spektralna širina, ki je lahko z nekaj truda pod $1~\si{kHz}$. Po drugi strani so emisijske 
črte v plinu zaradi Dopplerjeve razširitve široke vsaj nekaj $\si{GHz}$, 
pa še to le v razmeroma redkem in hladnem plinu, kjer je svetlost majhna.

Primerjajmo spektralno gostoto moči laserja in navadnih svetil. Majhen He-Ne
laser seva $1~\si{mW}$ v približno $10^{7}~\si{Hz}$ in spektralna gostota
moči je $dP/d\nu \sim 10^{-10}~\si{W/Hz}$. Po drugi strani zelo svetla 
živosrebrna svetilka seva v močno razširjene spektralne črte s širino okoli 
$10~\si{nm}$, kar ustreza $\sim 10^{13}~\si{Hz}$. \index{Spektralna gostota moči}
Spektralna gostota v koherentnem snopu, ki smo ga pripravili iz
navadne svetilke, je tako le okoli $3\cdot 10^{-20}~\si{W/Hz}$. Šolski
He-Ne laser torej prekaša najmočnejše nekoherentno svetilo za 10\index{Laser!He-Ne}
velikostnih redov. Z laserji je seveda mogoče doseči znatno večje
moči, v sunkih tipično do okoli $10^{12}$ W, tako da po spektralni gostoti moči v
koherentnem snopu laserji prekašajo običajna svetila za $20$--$25$
velikostnih redov. Verjetno v zgodovini težko najdemo kak drug izum, 
ki je prinesel tolikšno izboljšavo v neki bistveni količini, in tako ni 
nenavadno, da je iznajdba laserjev v začetku 60-ih let dvajsetega stoletja povzročila preporod optike.

\section{*Stabilizacija frekvence laserja na nasičeno absorpcijo}
\label{chap:stabilizacija}
\index{Nasičena absorpcija!{stabilizacija frekvence}}\index{Spektralna črta}
Laserska svetloba ima končno spektralno širino, ki 
je v realnih sistemih pogosto pogojena s fluktuacijami dolžine resonatorja ali 
temperature ojačevalnega sredstva. Eden od načinov 
zoženja spektralne črte je zato aktivna stabilizacija
dolžine resonatorja. Ideja je sledeča: svetlobo iz laserja primerjamo
z nekim standardom in iz razlike določimo spremembo dolžine. S
piezoelektričnim nosilcem nato premaknemo eno od zrcal in popravimo 
dolžino resonatorja. 

Poglavitna težava je najti dovolj stabilen primerjalni standard za frekvenco. 
Ena možnost je, da izhodno svetlobo usmerimo skozi pomožni 
interferometer, \index{Fabry-Perotov interferometer}
ki je skoraj v resonanci z laserjem in ima dovolj ozek vrh 
prepustnosti\footnote{R. W. P. Drever et al., Appl. Phys. B $\mathbf{31}$, 97 (1983).}.
Že majhen 
premik frekvence laserja povzroči spremembo prepuščenega svetlobnega toka. 
Na prvi pogled je videti, da z uporabo interferometra nismo pridobili, 
saj je tudi resonančna frekvenca interferometra stabilna le toliko, kot je 
stabilna njegova dolžina. Vendar je z izolacijo in temperaturno stabilizacijo 
mogoče ohranjati dolžino praznega resonatorja -- interferometra -- znatno
natančneje kot dolžino resonatorja, v katerem se nahaja aktivno sredstvo, ki mu 
dovajamo energijo.

Druga možnost je stabilizacija laserja na primerno molekularno absorpcijsko
črto\footnote{A. Brillet in P. Cerez, J. Phys. Col. $\mathbf{42}$, C8-73 (1981).}.
Te so v razredčenem plinu lahko zelo ozke, zato je tudi spekter 
laserja lahko izredno ozek, pod $1~\si{kHz}$. 
Pri tem moti Dopplerjeva razširitev absorpcijske črte, vendar se ji
izognemo s pojavom nasičenja 
absorpcije. \index{Nasičena absorpcija}V poglavju
(\ref{NasabsNehom}) smo spoznali, da se pri dvakratnem prehodu
monokromatskega snopa svetlobe skozi plin v nasprotnih smereh pojavi v
sredini Dopplerjevo razširjene črte Lambova vdolbina,\index{Lambova vdolbina} ki 
ima obliko homogeno razširjene črte (slika~\ref{fig:Lamb}). 
Homogena širina je navadno mnogo manjša od Dopplerjeve in
je zato vdolbina uporabna kot frekvenčni standard. 

S spreminjanjem dolžine resonatorja spreminjamo frekvenco laserja in 
ko se ta znotraj homogene širine približa sredini absorpcijske črte (pri $\omega_0$), 
se absorpcija zmanjša in moč laserja poveča. To naredimo tako, da je 
eno od zrcal pritrjeno na piezoelektrični nosilec, na katerega je priključena
izmenična napetost s krožno frekvenco $\Omega$. Zaradi periodičnega spreminjanja dolžine
laserja se spreminja izhodna moč. Kadar je krožna frekvenca laserja 
enaka $\omega_0$, se moč spreminja z dvojno krožno frekvenco modulacije $2\Omega$. 
Kadar pa srednja krožna frekvenca laserja odstopa od $\omega_0$, se izhodna moč 
pri odmiku v zrcala v eno stran spremeni drugače kot v drugo, kar pomeni, 
da je v izhodnem signalu tudi komponenta s krožno frekvenco $\Omega$. Da držimo 
srednjo krožno frekvenco laserja enako $\omega_0$, moramo torej meriti komponento 
izhodne moči pri modulacijski frekvenci in jo s povratno zanko ohranjati na nič.

Napravimo še kvantitativno oceno opisane stabilizacijske sheme. Odvisnost
izhodne moči od krožne frekvence laserja $\omega$ lahko približno zapišemo v
obliki (enačba~\ref{eq:spekter-primer})
\begin{equation}  
\label{5.40}
P(\omega)=P_0 + \frac{P_1\gamma^2}{(\omega- \omega_0)^2+\gamma^2},
\end{equation}
pri čemer je $P_0$ moč laserja brez saturacijskega vrha pri $\omega_0$, $P_1$ 
povečanje moči pri $\omega_0$ in $\gamma$ homogena širina črte. Privzeli  
smo, da se ojačenje laserja in nehomogeno razširjeni del absorpcije ne 
spreminjata znatno znotraj homogene širine absorberja in je zato moč $P_0$ 
približno konstantna. Krožno frekvenco laserja moduliramo, tako da je 
\begin{equation}  
\label{5.41}
\omega=\omega_0+\Delta\omega+b \sin \Omega t.
\end{equation}
Z $\Delta\omega$ smo označili odstopanje srednje krožne frekvence laserja od
sredine absorpcijske črte $\omega_0$ in z $b$ amplitudo modulacije. 

Če sta $b$ in $\Delta\omega$ majhna v
primerjavi z $\gamma$, lahko imenovalec v enačbi~(\ref{5.40})
razvijemo
\begin{equation}  
\label{5.42}
P=P_0+P_1 \left(1-\frac{\Delta\omega^2}{\gamma^2} +\frac{2}{\gamma^2} b
\Delta\omega \sin\Omega t - \frac{b^2}{\gamma^2}\sin^2\Omega t\right).
\end{equation}
Amplituda signala pri krožni frekvenci modulacije $\Omega$ je 
\begin{equation}
P(\Omega) = \frac{2P_1 b \Delta\omega}{\gamma^2}. 
\end{equation}
Najmanjša moč, ki jo ločeno zaznamo na detektorju, je določena s šumom 
meritve. Kot bomo videli v poglavju (\ref{chap:sum}), je najmanjša 
sprememba svetlobne moči, ki jo lahko izmerimo, enaka 
\begin{equation}  
\label{5.43}
P_N\approx \sqrt{\hslash\omega P \frac{1}{\tau}},
\end{equation}
pri čemer je $P$ celotna vpadna svetlobna moč, $\tau$ pa čas
meritve. Vzemimo za primer He-Ne laser\index{Laser!He-Ne}. Stabiliziramo ga tako,
da vanj dodamo jodovo celico, v kateri pride do nasičene absorpcije.
Povprečna moč laserja naj bo $P_0 = 10~\si{mW}$ in $P_1= 0,1~\si{mW}$. 
Širina absorpcijske črte je $\gamma= 10^6~\si{s}^{-1}$. Izberimo amplitudo 
modulacije $b=10^5~\si{s}^{-1}$ in $\tau=10^{-4}~\si{s}$. Časovna konstanta 
$\tau$ ne sme biti prevelika, določa namreč, kako hitro prilagajamo dolžino 
laserja. Za najmanjšo zaznavno moč pri $\Omega$ dobimo 
$P_N(\Omega) = 5\times10^{-9}~\si{W}$.
Najmanjše merljivo odstopanje krožne frekvence laserja je tedaj 
\begin{equation}  
\label{5.44}
\Delta\omega_N=\frac{P_N \gamma^2}{2P_1 b}=250~\si{s}^{-1}.
\end{equation}
Takšno in še boljšo stabilnost frekvence se danes lahko doseže. 

\begin{remark}
Na absorpcijsko črto stabiliziranega laserja navadno ne uporabljamo
direktno, temveč z njim kontroliramo drug laser. Del izhodne svetlobe iz
obeh laserjev zmešamo na detekcijski fotodiodi. V signalu dobimo utripanje,
ki je enako razliki frekvenc obeh laserjev. S spreminjanjem dolžine drugega
laserja skrbimo, da je frekvenca utripanja konstantna. Na ta način lahko v
ozkem frekvenčnem intervalu še spreminjamo frekvenco drugega laserja.
Z opazovanjem utripanja med dvema stabiliziranima laserjema določamo tudi
njuno stabilnost.
\end{remark}

\section{Večfrekvenčni laser}
Do zdaj smo obravnavali laserje, v katerih je bilo vzbujeno eno samo stoječe
valovanje. Vendar je ojačevalna širina večine aktivnih (ojačevalnih) sredstev 
navadno večja od razlike med frekvencami posameznih \index{Laser!večfrekvenčni}
nihanj resonatorja. V plinih, na primer, je ojačevalna širina zaradi 
Dopplerjevega pojava vsaj nekaj $\si{GHz}$. Po drugi strani so lastne frekvence resonatorja 
pri $30~\si{cm}$ dolgem resonatorju razmaknjene za $500~\si{MHz}$ in kaj  
lahko se zgodi, da ojačenje v laserju za več lastnih nihanj hkrati 
presega ojačenje na pragu. Takrat je v resonatorju vzbujenih več lastnih nihanj in 
svetloba iz takega večfrekvenčnega laserja ni monokromatska,
temveč je sestavljena iz množice ozkih črt znotraj ojačevalnega pasu.
Izsevana svetloba ni bistveno bolj monokromatska od ustrezne spektralne 
komponente svetlečega plina. Ostaja seveda prostorsko koherentna.

Za holografijo, interferometrijo in nekatere spektroskopske metode
potrebujemo ozko spektralno črto. Zato moramo poskrbeti, da je v laserju vzbujeno le
eno lastno nihanje resonatorja, najbolje tisto, ki je najbližje vrhu ojačenja
aktivnega sredstva. To dosežemo tako, da za vsa ostala nihanja povečamo izgube,
na primer s Fabry-Perotovim etalonom, \index{Fabry-Perotov interferometer}
ki ga vstavimo v laserski resonator
(slika~\ref{fig:FPres}). 

Njegova prepustnost je odvisna od krožne frekvence 
svetlobe $\omega$, debeline ploščice $L$, lomnega količnika $n$, odbojnosti na stenah 
${\cal R}$ in nagiba glede na os resonatorja $\varphi$ (enačba~\ref{eq:FP}) 
\begin{equation}
T=\frac{1}{1+\frac{4{\cal {R}}}{(1-{\cal {R}})^{2}}\sin^{2}(\frac{n\omega}{c}L\cos \varphi )}.
\label{5.22}
\end{equation}

\vglue-5truemm
\begin{figure}[h]
\centering
\def\svgwidth{70truemm} 
\input{slike/06_FPres.pdf_tex}
\caption{Shema laserja z vstavljenim Fabry-Perotovim etalonom}
\label{fig:FPres}
\end{figure}

Debelino etalona in njegov nagib 
izberemo tako, da vrh prepustnosti sovpada z izbranim lastnim nihanjem resonatorja. 
Izgube za ostala nihanja, ki bi sicer bila ojačena, so znatno večje in laser
sveti pri eni sami izbrani frekvenci. Ta proces je shematsko prikazan na 
sliki~\ref{fig:FPmodes}: izmed vseh možnih stanj v resonatorju (a) svetijo le tista, za katere
je ojačenje nad pragom (b). Ko v resonator dodamo Fabry-Perotov etalon z razmeroma velikim
razmikom med sosednjimi vrhovi prepustnosti (c), je ojačeno eno samo nihanje (d). 
Ker zadošča že zmerno povečanje izgub, je odbojnost sten etalona navadno dokaj nizka, 
pod 0,5. 

\begin{figure}[h]
\centering
\def\svgwidth{80truemm} 
\input{slike/06_FPmodes.pdf_tex}
\caption{Lastne frekvence resonatorja (a) in frekvenčna odvisnost ojačenja z označenim 
pragom delovanja (b). Z zeleno so označene tiste lastne frekvence, ki se v laserju ojačujejo. 
Ko dodamo Fabry-Perotov etalon z dano prepustnostjo (c), povečamo
izgube za vse načine, ki bi sicer bili ojačeni, razen za enega. 
Tako dosežemo delovanje laserja pri eni sami frekvenci (d).}
\label{fig:FPmodes}
\end{figure}
Opisali smo, kako v laserju dosežemo delovanje pri enem samem longitudinalnem nihanju.
Poleg tega je treba omejiti tudi ojačenje višjih prečnih nihanj. To navadno 
dosežemo z zaslonkami, saj imajo snopi višjih redov večji efektivni polmer od osnovnega. 
\begin{remark}
Nagib etalona omogoča natančno spreminjanje izbrane frekvence, poleg tega
 je nujen, da se neprepuščena svetloba odbije ven iz smeri osi resonatorja. Če bi 
bila os etalona vzporedna z osjo resonatorja, bi se pojavile dodatne resonance, 
kar bi močno motilo delovanje laserja. Namesto Fabry-Perotovega etalona se uporabljajo
tudi prizme in uklonske mrežice.
\end{remark}

\section{Relaksacijske oscilacije}
\index{Relaksacijske oscilacije}
Stacionarno delovanje laserjev smo že dodobra spoznali. Za obravnavo
nestacionarnega delovanja  se moramo vrniti k obravnavi zasedbenih enačb 
(\ref{5.7}) in (\ref{5.8}), ki jih je na splošno treba reševati numerično. 
Vseeno lahko nekaj povemo o obnašanju laserja
v bližini stacionarnega delovanja. 

Spet se omejimo na enofrekvenčni laser, v katerem zasedenost vzbujenega stanja
in število fotonov opišemo z enačbama (\ref{5.7})
in (\ref{5.8}). Najprej zaradi preglednosti vpeljemo
brezdimenzijski čas $t^{\prime}=t A$ in $\tau^{\prime}=\tau A$, kar pomeni, da merimo 
čas v enotah življenjskega časa laserskega nivoja. Ponovno uporabimo parameter
$p=VA/(B\hslash\omega g)$ (enačba~\ref{5.10}), ki pomeni število lastnih stanj 
elektromagnetnega polja v volumnu $V$ in znotraj spektralne širine laserskega prehoda. 
Enačbi zapišemo kot
\begin{align}  
\frac{d N_2}{d t^{\prime}}&=-\frac{nN_2}{p}-N_2+N_{20} \label{5.23a} \qquad \mathrm{in}\\
\frac{d n}{d t^{\prime}}& =  \frac{nN_2}{p}-\frac{2}{\tau^{\prime}}n.
\label{5.23}
\end{align}
Pri tem smo vpeljali konstanto $N_{20}= rN/A$, ki ima tudi nazoren pomen.
Predstavlja zasedenost, ki bi jo dobili pri danem stacionarnem črpanju, če v
izbranem stanju ne bi bilo fotonov in s tem ne stimuliranega sevanja. Meri torej 
moč črpanja. V enačbi~(\ref{5.23})
smo zanemarili prispevek spontanega sevanja, za katerega smo že ugotovili,
da se pozna le do praga.

Poiščimo približne rešitve sistema sklopljenih diferencialnih enačb z 
linearizacijo. Naj laser najprej deluje stacionarno, nato se v nekem trenutku  
nekoliko izmakne iz stacionarnega stanja. To se lahko zgodi, na primer, če v nekem 
trenutku spremenimo moč črpanja. Trenutno zasedenost $N_2$ in število fotonov $n$
zapišemo v obliki 
\begin{equation}  
N_2= N_{2s}+x \qquad \mathrm{in} \qquad n=n_s+y,
\label{5.24}
\end{equation}
kjer sta $N_{2s}$ in $n_s$ vrednosti v stacionarnem stanju. Zanju velja 
\begin{equation}  
N_{2s}=\frac{2p}{\tau^{\prime}}\qquad \mathrm{in}\qquad  
n_s=p\frac{N_{20}-N_{2s}}{N_{2s}}=p(a-1).
\label{5.26}
\end{equation}
Prva enačba je v skladu s tem, da je stacionarna zasedenost 
enaka zasedenosti na pragu, ta pa je odvisna od izgub resonatorja. 
Razmerje $a=N_{20}/N_{2s}$ je mera za moč črpanja in
mora biti v delujočem laserju večje od 1. V večini praktičnih primerov
doseže vrednosti $a \sim 5$.

Vstavimo nastavka (enačbi~\ref{5.24}) v enačbi (\ref{5.23a} in \ref{5.23}). 
Dobimo sistem enačb
\begin{align}  
\frac{d x}{d t^{\prime}} &=-\frac{n_sN_{2s}}{p}-N_{2s}+N_{20}- \frac{1}{p}
(n_sx+N_{2s}y+xy)-x \qquad \mathrm{in}\\
\frac{d y}{d t^{\prime}} &= \frac{n_sN_{2s}}{p}-\frac{2}{\tau^{\prime}}n_s
+ \frac{1}{p}(n_s x+N_{2s} y+xy)-\frac{2}{\tau^{\prime}}y.
\label{5.27}
\end{align}

Ker sta $x$ in $y$ majhna v primerjavi s stacionarnimi vrednostmi, lahko
mešani produkt $xy$ zanemarimo. Vsi členi, ki vsebujejo le stacionarne vrednosti,
dajo ravno 0, saj smo jih tako določili. Če upoštevamo še izraza 
za stacionarni vrednosti (enačbi~\ref{5.26}), 
sta linearizirani diferencialni enačbi za odmika od stacionarnih vrednosti 
\begin{equation}
\frac{dx}{dt^{\prime }} =-a\,x-\frac{2}{\tau ^{\prime }}\,y   \qquad \mathrm{in} \qquad
\frac{dy}{dt^{\prime }} =(a-1)\,x.
\label{5.28}
\end{equation}
Rešitev sistema diferencialnih enačb s konstantnimi
koeficienti poiščemo z nastavkom
\begin{equation}
x=x_{0}e^{\lambda t^{\prime }} \quad \mathrm{in} \quad 
y=y_{0}e^{\lambda t^{\prime }}.
\label{5.29}
\end{equation}
Dobimo homogen sistem linearnih enačb 
\begin{align}
(a+\lambda )x_{0}+\frac{2}{\tau ^{\prime }}y_{0} &=0  \label{5.30} \qquad \mathrm{in}\\
-(a-1)x_{0}+\lambda y_{0} &=0,
\end{align}
ki ima netrivialno rešitev le, če je njegova determinanta enaka nič
\begin{equation}
\lambda ^{2}+a\lambda +\frac{2}{\tau ^{\prime }}(a-1)=0.  
\label{5.301}
\end{equation}
Rešitvi sta 
\begin{equation}
\lambda =-\frac{a}{2}\pm \sqrt{\frac{a^{2}}{4}-\frac{2}{\tau ^{\prime }}(a-1)}.
\label{5.31}
\end{equation}
Obnašanje rešitve je odvisno od velikosti brezdimenzijskega relaksacijskega
časa nihanja resonatorja $\tau ^{\prime }=A\tau $. 

Kratek račun pokaže, da je 
za $\tau ^{\prime }>2$ izraz pod korenom za vse vrednosti $a$ pozitiven in laser 
se eksponentno vrača v stacionarno stanje. Za $\tau' <2$ je koren v določenem območju
parametra $a$ imaginaren in laser se v stacionarno stanje vrača z
dušenim nihanjem, ki mu pravimo relaksacijske 
oscilacije.\index{Relaksacijske oscilacije}
Primer takega nihanja 
pri vključitvi laserja kaže slika~\ref{fig:relax}. Opazimo, da sta tako 
frekvenca relaksacijskih oscilacij kot karakteristični čas dušenja oscilacij
odvisna od parametra $a$, torej od moči črpanja\footnote{O. Svelto in D. C. Hanna, 
{\it Principles of Lasers}, peta izdaja, 
Springer (2010).}.
\begin{figure}[h]
\centering
\def\svgwidth{80truemm} 
\input{slike/06_relax.pdf_tex}
\caption{Relaksacijske oscilacije števila fotonov $n$ po vklopu laserja. Po začetnem
nihanju se število fotonov ustali pri stacionarni vrednosti $n_s$.}
\label{fig:relax}
\end{figure}

Poglejmo primer. Vzemimo vrednost $\tau \sim 10^{-7}~\si{s}$ in razpadno
konstanto laserskega nivoja $A \sim 10^5/\si{s}$. Tedaj je $\tau^{\prime}\sim 10^{-2}$ 
in relaksacijske oscilacije se pojavijo pri vseh dosegljivih vrednostih črpanja 
nad pragom, to je za $a>1$. Ker $a$ v praksi ni nikoli dosti večji od $3$--$5$, 
je krožna frekvenca oscilacij v brezdimenzijskih enotah približno
enaka $\omega^{\prime}_r\sim 1/\sqrt{\tau^{\prime}}$. Ko preidemo
nazaj na enote časa, dobimo $\omega_r\sim \sqrt{A/\tau}$. Krožna frekvenca 
relaksacijskih oscilacij je v tem primeru velikostnega reda geometrijske 
sredine med razpadnima konstantama nihanja resonatorja in atomskega stanja. 
Tipične vrednosti frekvence relaksacijskih oscilacij so $\sim 10^5~\si{Hz}$, 
karakteristični čas dušenja pa $\sim 10^{-5}~\si{s}$.
\begin{remark}
Relaksacijske oscilacije so praktično pomembne, saj določajo zgornjo mejo
hitrosti, s katero lahko izhodna moč laserja sledi modulaciji črpanja.
Poleg tega se pri tej frekvenci pojavi resonanca, pri kateri se šum črpanja
ojačeno prenaša v šum izhodne moči. 
\end{remark}

\section{Sunkovni laserji}
Kadar potrebujemo veliko izhodno moč svetlobe iz laserja pri zmerni povprečni porabi 
energije, zvezno delujoči laserji, ki smo jih obravnavali do zdaj, 
niso primerni. Uporabiti moramo sunkovne laserje, 
ki v kratkem časovnem intervalu delujejo z zelo veliko izhodno močjo. 
\index{Sunkovni laser}

Poglejmo primer. 
Sunkovni laser naj oddaja svetlobo v $10~\si{ns}$ dolgih sunkih 
s povprečno energijo $1~\si{J}$ in s ponovitvijo $1000/\si{s}$.
Povprečna moč, s katero deluje tak laser, je $1~\si{kW}$ in moč, ki jo 
dosega v sunkih, $100~\si{MW}$. Po drugi strani ima močan zvezno delujoč laser  
izhodno moč $10~\si{kW}$. Tako pri desetkrat nižji povprečni moči laserja
dosežemo moči, ki so za $5$--$10$ velikostnih redov večje od moči 
zvezno delujočih laserjev. 

V grobem ločimo dva načina sunkovnega delovanja laserjev. V prvem primeru
periodično spreminjamo črpalno moč, medtem ko v drugem  periodično 
spreminjamo izgube v resonatorju. V slednjo skupino uvrščamo laserje na 
preklop dobrote in laserje, ki uklepajo faze valovanj.   

Obravnavajmo najprej način, pri katerem periodično moduliramo črpanje.
To dosežemo na primer z bliskavico ali drugim sunkovnim laserjem. V 
intervalu \index{Sunkovni laser!{preklop črpanja}}\index{Črpanje}
črpanja je ojačenje večje od izgub in svetloba izhaja iz laserja, ko pa črpanja 
ni, so izgube prevelike in laser ne sveti (slika~\ref{fig:Gswitch}\,a). Za dosego velike
obrnjene zasedenosti v ojačevalnem sredstvu mora biti sprememba črpanja zelo hitra, 
da v času preklopa števila fotonov še ne poveča znatno. 
Najbolj pogosto se modulacija črpanja uporablja
v polprevodniških laserjih, saj v njih črpanje poteka z električnim tokom, ki 
ga je zelo preprosto modulirati z razmeroma velikimi frekvencami. 
\begin{figure}[h]
\centering
\def\svgwidth{140truemm} 
\input{slike/06_pulseG.pdf_tex}
\caption{Delovanje sunkovnega laserja s periodično moduliranim črpanjem (a). Laser sveti,
kadar je črpanje dovolj veliko in ojačanje $G$ nad pragom $G_{\mathrm{pr}}$. 
Pri hitrem preklapljanju črpanja se lahko pojavijo neželene oscilacije izhodne moči $P$ in 
izhodni sunek se popači (b).}
\label{fig:Gswitch}
\end{figure}

Pri modulaciji črpanja se lahko pojavi težava. Ko ob močnem črpanju 
obrnjena zasedenost znatno preseže zasedenost praga (v nestacionarnem stanju 
je to mogoče), laser posveti in v kratkem času zasedenost pade nazaj pod prag. 
Če tedaj črpanje še traja, zasedenost naraste in laser ponovno posveti. 
To se lahko večkrat ponovi. Razmiki med zaporednimi sunki\index{Relaksacijske oscilacije}
so reda velikosti periode relaksacijskih oscilacij in so lahko precej
nepravilni  (slika~\ref{fig:Gswitch}\,b). Pri takem delovanju
razpoložljiva energija črpanja ne izstopa iz laserja v obliki enega samega lepo oblikovanega sunka, 
ampak v zaporedju sunkov. Omenjene oscilacije močno omejujejo uporabnost tega pristopa.

\section{Delovanje v sunkih s preklopom dobrote}
\index{Preklop dobrote|see{Sunkovni laser}}
\label{qswitch}
Namesto modulacije črpanja lahko v sunkovnih laserjih periodično spreminjamo 
izgube.\index{Sunkovni laser!{preklop dobrote}} Čim večje so namreč izgube, višji 
je prag za delovanje laserja. \index{Obrnjena zasedenost}\index{Izgube v resonatorju}
Posledično lahko dosežemo večjo stopnjo obrnjene zasedenosti in v sistemu atomov shranimo več 
energije kot pri majhnih izgubah.\index{Sunkovni laser}
Ko je enkrat ustvarjena velika obrnjena zasedenost, izgube zelo hitro zmanjšamo. 
Optično ojačenje je veliko in energija svetlobe v kratkem času močno naraste. 
S tem se tudi obrnjena zasedenost hitro zniža na vrednost močno pod pragom.
Predstavljamo si lahko, da dobimo prvi nihaj relaksacijskih oscilacij, le da
je začetno stanje daleč od stacionarnega in zato linearni približek ne drži več.
Iz laserja dobimo kratek in zelo močan sunek svetlobe\footnote{F. J.
McClung in R. W. Hellwarth, J. Appl. Phys. $\mathbf{33}$, 828 (1962).}. Energija
sunka je skoraj tolikšna, kot je bila energija obrnjene zasedenosti. 
Tipične dolžine sunkov, ki jih dosežemo na ta način, so $t \sim 10~\si{ns}$, sunki
pa se ponavljajo s frekvenco $\nu \sim 1$--$100~\si{kHz}$.
Dogajanje v laserju kaže slika~\ref{fig:pulseQ}.

\begin{remark}
V elektrotehniki se izgube resonatorjev podajajo z dobroto $Q$, to je razmerjem
frekvence lastnega nihanja in njegove širine. Ker s povečanjem izgub spremenimo 
širino črte in z njo dobroto, opisano tehniko imenujemo preklop 
dobrote.\index{Dobrota resonatorja}
\end{remark}
\vglue-5truemm
\begin{figure}[h]
\centering
\def\svgwidth{90truemm} 
\input{slike/06_pulseQ.pdf_tex}
\caption{Delovanje laserja v režimu preklopa dobrote. Izgube $\Lambda$ (a), relativno 
ojačenje $G/G_\mathrm{pr}$ (b), 
zasedenost višjega nivoja $N_2$ (c) in moč izsevane svetlobe $P$ (d) v odvisnosti od časa $t$.}
\label{fig:pulseQ}
\end{figure}

Izgube resonatorja je mogoče spreminjati na več načinov. Najpreprosteje
je mehansko, na primer z vrtečo prizmo ali zrcalom. Tedaj je resonator uglašen le v kratkem
intervalu, ko žarek vpada pravokotno na zrcalo resonatorja. Bolj razširjen način 
je z vgradnjo elektro-optičnega ali akusto-optičnega modulatorja, o katerih
bomo govorili v nadaljevanju (poglavje~\ref{chap:modulacija}). Na kratko povejmo, 
da z njimi izgube preklapljamo zelo hitro, saj to naredimo s spreminjajočo se 
električno napetostjo.

Kot smo že povedali, sistema nelinearnih enačb za zasedenost vzbujenega stanja $N_2$ 
in število fotonov $n$
(enačbi~\ref{5.7} in \ref{5.8}) ne moremo analitično rešiti. Vendar lahko na hitro
naredimo nekaj ocen. Dolžina izhodnega sunka je odvisna od hitrosti, 
s katero se izprazni zgornji laserski nivo. To se ne more zgoditi
hitreje kot v nekaj preletih sunka skozi resonator. Trajanje sunka je torej
vsaj nekajkrat $2L/c$, to je za $15~\si{cm}$ dolg resonator vsaj nekaj $\si{ns}$.

Ocenimo še hitrost naraščanja števila fotonov na začetku in 
njegovega upadanja na koncu sunka. Še enkrat zapišemo enačbi za zasedenost in za število
fotonov, pri čemer upoštevamo, da nas zanima le dogajanje v času sunka,
ki je zelo kratek v primerjavi z atomskim razpadnim časom, zato 
ustrezni člen v enačbi (\ref{5.7}) zanemarimo. Navadno je tudi črpanje prešibko, da
bi med sunkom samim znatno vplivalo na zasedenost, zato tudi člen $rN$
izpustimo. Seveda je črpanje upoštevano v tem, da je v sistemu velika začetna 
zasedenost $N_{20}$. Tako ostane 
\begin{equation}
 \frac{d N_2}{d t}=-\frac{\sigma c}{V}\,n\,N_2 
 \label{5.32a}
\end{equation}
in 
\begin{equation}
 \frac{d n}{d t}=\frac{\sigma c}{V}\,n\,N_2 - \frac{2}{\tau}\,n.
 \label{5.32}
\end{equation}
Na začetku sunka je $N_2$ velik in se ne razlikuje dosti od začetne vrednosti $N_{20}$. Takrat 
je vrednost  $n$ majhna, zato drugi člen v enačbi~(\ref{5.32}) lahko  
zanemarimo. Izračunamo, da število fotonov
na začetku sunka narašča približno eksponentno
\begin{equation}  
n(t)=n_0e^{t\,N_{20}\,\sigma c/V}= n_0e^{t/\tau_r},
\label{5.33}
\end{equation}
ker je $t_r = V/N_{20} \sigma c$.
Začetnega števila fotonov ne poznamo, vendar vemo, da je velikostnega reda 1,
saj predstavlja spontano emisijo. Da $n$ znatno naraste, recimo nad 
$10^{10}$ fotonov, je potreben čas blizu $25~\tau_r$.
Proti koncu sunka $N_2$ pojema zaradi sevanja svetlobe in v enačbi~(\ref{5.32}) lahko 
zanemarimo člen z $N_2$. Ostane samo še drugi člen, ki da preprosto rešitev
\begin{equation}  
n(t)=\tilde{n}_0e^{-2t/\tau}.
\label{5.33a}
\end{equation}
Eksponentno pojemanje števila fotonov na koncu sunka je torej določeno z izgubami
v resonatorju (enačba~\ref{taulambda}). 

Dogajanja v vmesnih časih ne moremo enostavno popisati, vendar lahko najdemo
medsebojno zvezo med $n$ in $N_2$. 
Izrazimo $dt$ iz enačbe (\ref{5.32a}) in ga vstavimo v enačbo~(\ref{5.32}).
Dobimo
\begin{equation}
dn=-dN_{2}+\frac{\tilde{N}_2}{N_{2}}dN_{2}\;\;,  \label{5.341}
\end{equation}
kjer smo zapisali $\tilde{N}_{2}=2V/(\sigma c\,\tau)$.
Enačbo brez težav integriramo
\boxeq{5.351}{
n=N_{20}-N_{2}+\tilde{N}_{2}\ln \frac{N_{2}}{N_{20}}.
}
Pri tem smo privzeli, da je na začetku sunka $n=0$ in $N_{2}=N_{20}$. 
Najprej izračunamo, kolikšna je zasedenost na koncu sunka 
$N_{2k}$. Izhajamo iz pogoja, da je število fotonov na koncu sunka $n=0$, 
kar da transcendentno enačbo za $N_{2k}$
\beq
\ln \frac{N_{2k}}{N_{20}} = \frac{N_{2k}}{\tilde{N}_2}- \frac{N_{20}}{\tilde{N}_2}.
\label{xaenacba}
\eeq
Enačbo preprosto numerično rešimo (slika~\ref{fig:Qeq}).

\begin{figure}[h]
\centering
\def\svgwidth{90truemm} 
\input{slike/06_Qeq.pdf_tex}
\caption{Rešitev enačbe~(\ref{xaenacba}), ki pove, kolikšna je
zasedenost vzbujenega stanja $N_{2k}$ na koncu sunka pri dani začetni 
zasedenosti $N_{20}$. Večja kot je začetna zasedenost, bolj se 
izprazni vzbujeni nivo.
}
\label{fig:Qeq}
\end{figure}

Če vpeljemo parametra $x=N_{2k}/\tilde{N}_2$ in $a=N_{20}/\tilde{N}_{2}$, se
enačba prepiše v 
\beq
\ln \frac{x}{a}= x-a.
\eeq
Za vrednosti $a<1$ enačba nima rešitev in 
v $\tilde{N}_{2}$ prepoznamo zasedenost vzbujenega stanja na pragu delovanja. Kadar je začetna 
zasedenost $N_{20}$ le malo nad pragom, končna zasedenost $N_{2k}$ ne pade 
dosti pod prag, zato je izraba energije slabša. Pri večjih začetnih vrednostih 
$N_{20}$ pade končna zasedenost praktično na nič. Za $a=2$, na primer,
je $x=0,41$, medtem ko je že pri $a=4$ vrednost $x$ le še 0,08. 

Ko enkrat poznamo začetno in končno zasedenost, lahko izračunamo 
celotno energijo v sunku $W=\hslash \omega (N_{20}-N_{2k})$. Kadar je začetna vrednost
dovolj nad pragom ($N_{20}\gg\tilde{N}_{2}$), končno zasedenost zanemarimo ($N_{k}\ll\tilde{N}_{2}$) in 
celotna energija svetlobe, izsevane v sunku, je 
\boxeq{QW}{
W\approx\hslash \omega\,N_{20}.
}
\begin{naloga}
\label{nalpmax}
Trenutna moč svetlobe, ki izhaja iz laserja, je enaka 
\begin{equation}
P=n \hslash \omega \frac{2}{\tau}.
\end{equation}
Pokaži, da je izsevana moč največja natanko tedaj, ko pade zasedenost na vrednost pri 
pragu ($N_{2}=\tilde{N}_{2}$). 
\end{naloga}

Največja izsevana moč iz laserja je \index{Izhodna moč laserja}
\beq
P_{\rm max}=\frac {n_{\rm max} \hslash \omega}{2L/c}\left(1-\cal{R}\right) = 
\frac {2 n_{\rm max} \hslash \omega}{\tau},
\eeq
pri čemer smo z $n_\textrm{max}$ označili število fotonov v vrhu sunka. 
Ko vstavimo še vrednost za $n_{\rm max}$ pri $N_{2}=\tilde{N}_{2}$ (glej 
nalogo~\ref{nalpmax}), dobimo
\beq
P_{\rm max}=\frac {2\hslash \omega}{\tau} \left(N_{20}-\tilde{N}_{2}-\tilde{N}_{2}
\ln (N_{20}/\tilde{N}_{2})\right).
\eeq
Ker je navadno $N_{20}\gg \tilde{N}_2$, je $n_{\rm max} \approx N_{20}$
in 
\boxeq{QP}{
P_{\rm max} \approx \frac{2}{\tau} \hslash \omega N_{20}.
}

Poglejmo primer. Naj bo presek za stimulirano sevanje $\sigma=B\hslash \omega g/c$ 
okoli $10^{-19}~\si{cm}^{2}$ in začetna gostota zasedenosti $N_{20}/V=10^{19}~\si{cm}^{-3}$.
Tedaj je $\tau_{r}=30~\si{ps}$ in čas naraščanja
sunka okoli $\sim 1~\si{ns}$. Število fotonov se nato zmanjšuje s
karakterističnim razpadnim časom resonatorja $\tau /2\sim 2L/(c(1-{\cal R}))$. 
Za dosego kratkih sunkov svetlobe je zato v laserjih s preklopom dobrote odbojnost 
izhodnega zrcala navadno dokaj nizka, recimo $0,5$. Pri dolžini resonatorja 
$L=15~\si{cm}$ je $\tau=4~\si{ns}$.
Celotno trajanje sunka je v izbranem primeru $~\sim~10~\si{ns}$, pri
čemer traja okoli $100~\si{ns}$ od preklopa dobrote do trenutka, ko sunek zraste iz šuma
spontanega sevanja. Energija v sunku je blizu $N_{20}\hslash \omega $, to pri
aktivnem volumnu $0,5~\si{cm}^3$ znaša $\sim~1~\si{\joule}$. Ocenimo, da je
moč v vrhu sunka $\sim~100~\si{MW}$.

\section{Uklepanje faz}
\index{Uklepanje faz}\index{Sunkovni laser}
\label{chap:Uklepanje}
Krajše sunke kot s preklopom dobrote je mogoče dobiti z uklepanjem 
faz\footnote{L. E. Hargrove, R. L. Fork in M. A. Pollack, Appl. Phys. Lett. $\mathbf{5}$, 4 (1964).}.
Pri tem gre za povsem drugačen način, ki je prav presenetljiva manifestacija 
koherentnosti laserske svetlobe. Spoznali smo že, da je v laserju navadno 
vzbujenih več lastnih nihanj hkrati, pri čemer so njihove krožne frekvence 
enakomerno razmaknjene za $\Delta \omega =\pi c/L$ 
(enačba~\ref{eq:delta-omega-resonator}). Celotno električno
polje v neki točki v laserju zapišemo kot vsoto 
\begin{equation}
E(t)=\sum_{m=-N/2}^{N/2}A_{m}e^{-i(\omega _{0}+m\Delta \omega )t+i\varphi
_{m}(t)},
\label{5.342}
\end{equation}
pri čemer je $N$ število vseh vzbujenih nihanj. Upoštevali smo, da ima vsako
vzbujeno nihanje lahko poljubno fazo $\varphi _{m}(t)$, ki je na splošno predvsem
zaradi zunanjih motenj slučajna funkcija časa. Zaradi tega se
celotno polje v resonatorju slučajno spreminja, kar močno zmanjšuje uporabnost
takega laserja.

Denimo, da nekako dosežemo enake faze vseh nihanj, ki naj bodo vse enake nič. 
Poleg tega zaradi enostavnosti
računa privzamemo, da so enake tudi vse amplitude vzbujenih nihanj $A_{m}= A_0$. Tedaj
postane vsota (\ref{5.342}) geometrijska in jo brez težav seštejemo 
\begin{equation}
E(t)=A_{0}\,e^{-i\omega _{0}t}\frac{\sin (N\Delta \omega t/2)}{\sin(\Delta
\omega t/2)}.
\label{5.352}
\end{equation}
Moč izhodne svetlobe ima časovno odvisnost (slika~\ref{s5.10})\index{Izhodna moč laserja}
\begin{equation}
P(t)=P_{0}\,\frac{\sin ^{2}(N\Delta \omega t/2)}{\sin ^{2}(\Delta \omega t/2)}.
\label{5.36}
\end{equation}

\begin{naloga}
\label{naloga:mlock}
Pokaži, da izhodno moč iz laserja v primeru enakih faz vzbujenih lastnih nihanj
zapišemo z enačbo~(\ref{5.36}). Pokaži še, da je razmik med posameznimi sunki (vrhovi
moči) enak $T=2L/c$, dolžina posameznega sunka $\tau = T/N$, vrednost moči v vrhu 
sunka $N^{2}P_{0}$ in povprečna moč $NP_{0}$.
\end{naloga}

Izhodna svetloba predstavlja periodično zaporedje sunkov, 
ki si sledijo s periodo $T=2\pi /\Delta \omega =2L/c$, kar je enako času obhoda
svetlobe v resonatorju (glej nalogo~\ref{naloga:mlock}).  Izračunamo lahko tudi 
dolžino sunkov
\begin{equation}
\tau=\frac{T}{N}=\frac{2\pi }{N\Delta \omega }=\frac{2\pi }{\Delta\omega_{G}} = 
\frac{1}{\Delta\nu_{G}}.
\label{5.37}
\end{equation}
\begin{figure}[h]
\centering
\def\svgwidth{130truemm} 
\input{slike/06_pulseML.pdf_tex}
\caption{Valovanja z različno lastno frekvenco, a enako fazo, se seštejejo v 
posamezne vrhove (a). 
Časovna odvisnost izhodne moči večfrekvenčnega laserja z uklenjenimi fazami $P$ je 
v obliki izrazitih sunkov (b).}
\label{s5.10}
\end{figure}

Ker je $N$ število vzbujenih lastnih nihanj, ki se med seboj po krožni 
frekvenci razlikujejo za $\Delta \omega$, je produkt $N\Delta \omega$ ravno enak
širini ojačenja $\Delta \omega_{G}$. Dolžina sunka je torej obratno sorazmerna s širino
ojačevanja aktivnega sredstva. Za dosego zelo kratkih sunkov z metodo uklepanja
faz je treba uporabiti laser s kar se da veliko širino ojačevanja. 

Poglejmo primer. V Ti:safir laserju z dolžino resonatorja $L=1,5~\si{m}$ \index{Laser!Ti:safir}je 
širina ojačenja $\Delta \nu_G = 100~\si{THz}$. Iz takega laserja izhajajo
sunki, dolgi $\tau = 1/\Delta \nu_G = 10~\si{fs}$, med posameznima sunkoma 
poteče $T = 2L/c = 10~\si{ns}$. Število fazno uklenjenih lastnih načinov je 
$N = T/\tau = 10^6$.

Premislimo, kakšna je prostorska odvisnost električnega polja v
resonatorju. Polje na danem mestu opisuje enačba~(\ref{5.352}). Krajevno 
odvisnost dobimo, če v enačbi~(\ref{5.352}) parameter $t$ zamenjamo s $(t-z/c)$. To
predstavlja svetlobni paket, ki potuje sem in tja med zrcaloma
resonatorja. Na izhodnem zrcalu se vsakič del sunka odbije in del zapusti
resonator (slika~\ref{fig.5.11}). Razmik med sunki, ki izhajajo iz
resonatorja, je $2L$, prostorska dolžina posameznega sunka pa $\tau c=2L/N$.
\begin{figure}[h]
\centering
\def\svgwidth{120truemm} 
\input{slike/06_MLvsota.pdf_tex}
\caption{Prostorska odvisnost fazno uklenjenih sunkov. Pri vsakem odboju del
sunka zapusti resonator. Razmik med zaporednima sunkoma je dvakratnik dolžine resonatorja.}
\label{fig.5.11}
\end{figure}
\vglue-5truemm
\begin{remark}
V našem računu predpostavka, da so vse amplitude $A_{m}$ enake, ni ključnega 
pomena. Če vzamemo, da so amplitude oblike 
$A_{m}=A_{0}\exp (-(m\Delta \omega /\Delta \omega_{G})^{2})$, 
kar je bolj realistično (slika~\ref{fig:FPmodes}), vsote (enačba~\ref{5.342}) 
ne znamo točno sešteti. Lahko jo
približno pretvorimo v integral, ki je Fouriereva transformiranka Gaussove
funkcije (pri prehodu z diskretne vsote na integral seveda izgubimo
periodičnost zaporedja sunkov). Ta je zopet Gaussova funkcija, katere
širina je obratna vrednost širine prvotne funkcije, prav podobno, kot
smo dobili zgoraj. Odvisnost amplitud vzbujenih lastnih nihanj od $m$ 
vpliva torej le na točno obliko sunkov.
\end{remark}
\newpage
Pri uklepanju faz je bistvena predpostavka, da so vse faze $\varphi_m$ enake. 
V večfrekvenčnih laserjih so resonatorska stanja na splošno med seboj
neodvisna, zato so njihove faze poljubne in se zaradi motenj lahko še spreminjajo.
Za dosego istih faz posameznih nihanj moramo poskrbeti posebej. Tako ujemanje
oziroma uklepanje faz lahko dosežemo na več načinov. V grobem ločimo dva načina:
aktivno in pasivno uklepanje faz.

Pri aktivnem uklepanju faz moduliramo izgube v resonatorju
s frekvenco,\index{Uklepanje faz!aktivno} ki je ravno enaka razliki 
frekvenc med lastnimi resonatorskimi nihanji (slika~\ref{fig:aktpas}\,a). 
To dosežemo tako, da v resonator dodamo
modulator, na primer akusto-optični modulator (glej poglavje~\ref{chap:modulacija}).
Ko v akusto-optičnem modulatorju vzbudimo stoječe zvočno valovanje, se 
svetloba na njem uklanja in izgube so velike. Stoječe valovanje periodično izginja
in takrat se uklonske izgube zmanjšajo. Če se to dogaja v časovnih\index{Akusto-optični pojav}
razmikih, ki so enaki $T=2L/c$, se v resonatorju ojačuje le kratek sunek svetlobe. 
Navadno zadošča razmeroma majhna sinusna modulacija izgub, pri kateri je relativna 
prepustnost v minimumu za nekaj desetink manjša od največje. Ta način se uporablja
v šibkejših sunkovnih laserjih, na primer v Nd:YAG laserjih.\index{Laser!Nd:YAG}
\begin{figure}[h]
\centering
\def\svgwidth{80truemm} 
\input{slike/06_aktpas.pdf_tex}
\caption{Izgube v resonatorju in izhodna moč pri aktivnem (a) in pasivnem (b) uklepanju faz}
\label{fig:aktpas}
\end{figure}
\vskip-5truemm

Kako se pri modulaciji izgub faze uklenejo, lahko uvidimo še drugače.
Modulacija amplitude $m$-tega lastnega nihanja povzroči, da se v spektru nihanja
pojavita še stranska pasova pri krožnih frekvencah $(\omega_0 + m\Delta \omega) \pm \Delta\omega$. Ta
se ravno ujemata z obema sosednjima nihanjema in se konstruktivno
prištejeta, če imata enako fazo. Podobno tudi naprej za ostala nihanja. 
S tem se zmanjšajo izgube in delovanje laserja z uklenjenimi fazami ima najnižji prag. 

Pri pasivnem uklepanju faz dodanega elementa ne krmilimo od zunaj, 
ampak\index{Uklepanje faz!pasivno} je njegova prepustnost odvisna od intenzitete
izbranega nihajnega načina (slika~\ref{fig:aktpas}\,b). 
Ena vrsta pasivnih elementov je plast raztopljenega
barvila, ki močno absorbira svetlobo pri majhni gostoti svetlobnega toka, pri veliki 
gostoti pa pride do nasičenja absorpcije (glej\index{Nasičena absorpcija}
razdelek~\ref{chap:NasAbs}) in praktično vsa vpadna svetloba je 
prepuščena\footnote{E. P. Ippen, C. V. Shank in A. Dienes, Appl. Phys. Lett. $\mathbf{21}$, 348 (1972).}. 
V laserju je po vklopu prisotno predvsem 
spontano sevanje, ki se pri enem prehodu skozi aktivno snov deloma ojači. 
Barvilo najmanj absorbira fluktuacijo z največjo intenziteto. Pri dovolj 
velikem ojačenju se ta izbrana fluktuacija ojačuje, ostale pa ne, zato
se pojavi fazno uklenjeni sunek. Vzbujeni atomi se morajo čim hitreje
vrniti v osnovno stanje, da se svetloba lahko spet absorbira. To pomeni, da
 mora biti relaksacijski čas barvila  krajši od časa
obhoda $T$, tipično je $\sim \si{ps}$. 

Drugi način pasivnega uklepanja faz je z uporabo nelinearne optike 
(glej \index{Kerrov pojav!optični}
razdelek~\ref{OKP}). Z optičnim Kerrovim pojavom se snop svetlobe, ki vpade na 
optično nelinearno sredstvo, zoža, pri čemer je njegova širina odvisna od 
intenzitete svetlobe. Z dodatkom zaslonke poskrbimo, 
da je prepuščen le zelo močen svetlobni sunek, šibkejši, ki imajo večji polmer,
pa so zadušeni. Ta način pogosto
uporabimo v Ti:safirnih laserjih,\index{Laser!Ti:safir} s čimer dosežemo zelo
kratke izhodne sunke.

Z uklepanjem faz je mogoče dobiti sunke, krajše od $100~\si{fs}$. 
Tak sunek traja le še nekaj deset optičnih period. S posebnimi prijemi 
jih lahko še skrajšamo na okoli $10~\si{fs}$ (glej razdelek~\ref{kompdisp}). 
Zelo kratki svetlobni sunki so uporabni za študij hitre molekularne dinamike 
in kratkoživih vzbujenih elektronskih stanj v trdnih snoveh. Z uporabo fazno
uklenjenih sunkov svetlobe se je časovna ločljivost merilnih metod povečala za 
več redov velikosti.

\section{*Frekvenčni glavnik in absolutna meritev frekvence laserja}
\index{Frekvenčni glavnik}\index{Uklepanje faz}\index{Sunkovni laser}
Iz fazno uklenjenega laserja izhajajo zelo kratki sunki svetlobe, ki 
so sestavljeni iz velikega števila vzbujenih lastnih nihanj. Spekter 
izhodne svetlobe tako vsebuje veliko število spektralnih črt, ki\index{Spekter}
so med seboj enakomerno razmaknjene (slika~\ref{fig:comb}). Zaradi podobnosti spektra
z glavnikom imenujemo tak izvor svetlobe frekvenčni glavnik\footnote{Za odkritje 
sta John L. Hall in Theodor W. H\"ansch leta 2005 prejela 
Nobelovo nagrado.}. Značilna širina ojačenega
območja je $300~\si{THz}$, razmik med posameznimi črtami pa $100~\si{MHz}$, tako 
da tipični glavnik vsebuje okoli več milijonov spektralnih črt\footnote{
J. L. Hall, Rev. Mod. Phys. $\mathbf{78}$, 1279 (2006) in 
T. W. H\"ansch, Rev. Mod. Phys. $\mathbf{78}$, 1297 (2006).}. 
\begin{figure}[h]
\centering
\def\svgwidth{110truemm} 
\input{slike/06_comb.pdf_tex}
\caption{Spekter svetlobe iz frekvenčnega glavnika}
\label{fig:comb}
\end{figure}

Frekvence izhodne svetlobe na splošno zapišemo kot 
\begin{equation}
\nu = \nu_0 + m\Delta \nu,
\label{eq:comb}
\end{equation}
pri čemer je $m$ naravno število, $\Delta \nu = c/2L$ razlika med dvema zaporednima 
frekvencama in $\nu_0$ frekvenčni zamik. S stabilizacijo $\Delta \nu$ in 
frekvenčnega zamika $\nu_0$ so frekvence izhodne svetlobe  
natančno določene, zato frekvenčni glavnik uporabljamo kot referenco 
za izredno natančno določanje frekvenc. Ko svetloba z neznano frekvenco
interferira s svetlobo iz frekvenčnega glavnika, nastopi utripanje -- pojavi
se signal pri razliki obeh frekvenc. Zaradi velikega števila lastnih nihanj 
in posledično majhnih razlik v frekvencah je frekvenca utripanja navadno v 
radijskem območju. To pa znamo zelo natančno izmeriti in tako določiti neznano 
vpadno frekvenco. Na ta način lahko merimo frekvence z natančnostjo do 
$10^{-18}$.

V primeru, da so sunki svetlobe iz laserja povsem periodični in se ujemajo tako
v amplitudi kot v fazi, so frekvence izhodne svetlobe kar 
večkratniki $\Delta \nu$. V praksi se zaradi Gouyeve faze,\index{Gouyeva faza}
disperzije \index{Disperzija}in nelinearnosti pojavi zamik
v fazi $\Delta \varphi$ in vrh ovojnice sunka na splošno ne sovpada z vrhom amplitude
nihanja (slika~\ref{fig:comb2}). V izrazu za
frekvenco izhodne svetlobe se zato pojavi dodatni zamik $\nu_0 \neq 0$ (enačba~\ref{eq:comb}). 
Za absolutno
določitev frekvence moramo ta zamik seveda natančno poznati. 
\begin{figure}[h]
\centering
\def\svgwidth{110truemm} 
\input{slike/06_comb2.pdf_tex}
\caption{Časovna odvisnost amplitude izhodne svetlobe iz frekvenčnega glavnika. V splošnem
vrh ovojnice sunka ne sovpada z vrhom amplitude nihanja, vendar zamik $\nu_0$ znamo izmeriti.}
\label{fig:comb2}
\end{figure}

Zamik $\nu_0$ lahko izmerimo
z interferometrom, v katerem med seboj primerjamo valovanje pri 
osnovni in pri podvojeni frekvenci\footnote{Tak interferometer imenujemo $f$--$2f$
interferometer, kar nakazuje na osnovno ($f$)  in podvojeno ($2f$) frekvenco.}.
Valovanje pri osnovni frekvenci najprej frekvenčno podvojimo. To dosežemo 
z nelinearnimi optičnimi pojavi, ki omogočajo generacijo valovanja pri frekvenci, 
ji je enaka dvakratni frekvenci vpadnega valovanja 
(glej razdelek~\ref{chap:SHG}). Podvojeno frekvenco $2(\nu_0 + 
m\Delta \nu)= 2\nu_0 + 2m\Delta \nu$ nato primerjamo s frekvenco pri 
točno dvakrat večjem $m$, to je pri $\nu_0 + 2m\Delta \nu$. Pri razliki med tema
dvema frekvencama, ki je ravno $\nu_0$, se na detektorju pojavi utripanje.
Zamik $\nu_0$ na ta način izmerimo, hkrati ga s povratno zanko
vzdržujemo konstantnega. Ko določimo oba parametra $\nu_0$ in $\Delta \nu$, 
poznamo absolutne vrednosti frekvenc posameznih spektralnih črt v frekvenčnem 
glavniku z relativno natančnostjo okoli $10^{-18}$.

\subsection*{Definicija metra}
Danes je meter definiran kot pot, ki jo svetloba v vakuumu prepotuje v \index{Definicija metra}
$1/299\,792\,458~\si{s}$.  Vendar ni bilo vedno tako. Do šestdesetih let dvajsetega
stoletja je bil meter definiran z dolžino prametra, to je palice iz platine in iridija, 
pri atmosferskem tlaku in temperaturi taljenja ledu. 

Leta 1960 so definicijo 
izboljšali in jo vpeljali glede na svetlobo kriptonove svetilke. Meter je bil 
definiran kot $1\,650\, 763,73$ valovnih dolžin svetlobe, ki jo izseva kriptonov
izotop $^{86}$Kr med nivojema $2p_{10}$ in $5p_5$. Vendar je širina črte
kriptonove svetilke razmeroma velika, zato je bil meter 
definiran le z relativno negotovostjo $4 \times 10^{-9}$ oziroma absolutno 
negotovostjo $4~\si{nm}$. 

Z odkritjem laserjev so hitro spoznali, da definicija s kriptonovo svetlobo ni 
najprimernejša, saj se je izkazalo, da je njen spekter razmeroma širok in asimetričen. 
S heterodinsko tehniko \index{Heterodinska detekcija}so primerjali valovno dolžino 
svetlobe \index{Laser!He-Ne}
iz He-Ne laserja, stabiliziranega z metanom ($3,39~\si{\micro\meter}$), z osnovno
cezijevo uro in določili absorpcijsko črto z natančnostjo cezijeve ure.
Frekvenca izhodne svetlobe je bila določena na $88,376181627(50)~\si{THz}$.
Po drugi strani so lahko izmerili valovno dolžino izsevane svetlobe s primerjavo 
z valovno dolžino kriptonove svetlobe in tako določili hitrost svetlobe na
$299\, 792\, 458~\si{m/s}$.

Leta 1983 so definicijo metra vnovič spremenili in jo prek hitrosti svetlobe v
vakuumu vezali na enoto sekunde. Ta definicija velja še danes. Sekunda je določena
z nihaji cezijevih atomov in to omogoča določitev metra z relativno 
negotovostjo $10^{-13}$. 

Z novo definicijo izbira snovi, ki izseva svetlobo, ni več ključnega
pomena. Priporočena realizacija definicije metra je s He-Ne laserjem, 
stabiliziranim z jodovo celico, ki z absorpcijo še zmanjša spektralno širino črte.
V tem primeru je meter $1\,579\,800,762042(33)$ valovnih dolžin izsevane svetlobe. 
Laser je tako postal sekundarni standard za dolžino -- vendar je laser pri tem le pomožna 
naprava, standard je ustrezni molekularni prehod. 

\section{*Semiklasični model laserja}
\label{chap:semiklasicni}
\index{Semiklasični model}
Doslej smo laserje obravnavali le z modelom zasedbenih enačb. Ta je zelo
grob, saj smo pri tem zanemarili nekaj pomembnih pojavov. Svetlobo v 
resonatorju smo opisali s celotno energijo oziroma številom fotonov in se za
njeno valovno naravo nismo menili. Privzeli smo, da sta frekvenca
delujočega laserja in oblika polja lastnega nihanja v njem enaki kot v
praznem resonatorju. Aktivno snov smo opisali z zasedenostjo zgornjega
in spodnjega atomskega nivoja in s tem izpustili možnost, da se zaradi
interakcij z elektromagnetnim poljem atomi nahajajo v nestacionarnem,
mešanem stanju.\index{Zasedbene enačbe}

Navedene pomanjkljivosti delno odpravimo s tem, da elektromagnetno polje v
resonatorju obravnavamo klasično z valovno enačbo, atome aktivne snovi pa
kvantno in upoštevamo, da se pokoravajo Schr\"odingerjevi enačbi. S tem dobimo 
semiklasični model laserja. Za še natančnejši opis bi morali obravnavati
kvantno tudi svetlobo.

Aktivna snov naj bo množica enakih dvonivojskih 
atomov s stanjema $|1\rangle$ in $|2\rangle$, ki imata energiji $E_1$ in $E_2$.
Atomi s svetlobo interagirajo z dipolno interakcijo oblike $H = -eE(t)\hat{x}$, 
kjer je $E(t)$ polje v resonatorju, ki naj bo zaradi preprostosti 
polarizirano v smeri osi $x$. Časovno odvisno stanje atomov
zapišemo v obliki 
\begin{equation}  \label{5.45}
|\psi\rangle=c_1(t)|1\rangle\exp(-iE_1t/\hslash)+
c_2(t)|2\rangle\exp(-iE_2t/\hslash).
\end{equation}
Iz Schr\"odingerjeve enačbe (enačba~\ref{eq:sk-S}) dobimo za časovna odvoda
koeficientov $c_1(t)$ in $c_2(t)$ zvezi
(glej enačbi~\ref{eq:c1c2})
\begin{equation}
\frac{d c_1}{dt}=-\frac{i}{\hslash} \mathcal{V} E(t) e^{-i\omega_0 t}\, c_2 
\qquad \mathrm{in} \qquad
\frac{d c_2}{dt}=-\frac{i}{\hslash} \mathcal{V} E(t) e^{i\omega_0 t}\, c_1,
\label{5.46}
\end{equation}
pri čemer je $\mathcal{V} = -e\langle1|\hat{x}|2\rangle$ realen in $\omega_0=(E_2-E_1)/\hslash$.

Električni dipolni moment atoma v stanju ${\psi}$ je 
\begin{equation}  
\label{5.47}
p=e\langle\psi|\hat{x}|\psi\rangle=-
(c_1^{\ast}c_2e^{-i \omega_0t}+c_1c_2^{\ast}e^{i \omega_0 t}) \mathcal{V}.
\end{equation}
Razdelimo $p$ na dva dela in zapišemo
\begin{equation}  
\label{5.48}
p=p^+ + p^-=-\mathcal{V}\left(\eta(t)+\eta^{\ast}(t)\right),
\end{equation}
kjer smo vpeljali $\eta(t)=c_1^{\ast}c_2e^{-i \omega_0 t}$.

Zanima nas, kako se dipolni moment spreminja s časom. Zadošča, da vemo, kako se 
s časom spreminja parameter $\eta(t)$. Njegov časovni odvod izrazimo
iz enačb (\ref{5.46}) in dobimo
\begin{equation}  
\label{5.49}
\frac{d\eta}{dt}=- i \omega_0\eta+\frac{i}{\hslash}\,\mathcal{V}E(t) \left(|c_2|^2-|c_1|^2\right).
\end{equation}
Spomnimo, da je $|c_i|^2$ verjetnost za zasedenost stanja $|i\rangle$. Izraz v oklepaju
na desni strani enačbe~(\ref{5.49}) torej meri razliko zasedenosti obeh stanj, ki jo označimo
s $\zeta$. Podobno kot zgoraj izrazimo spreminjanje razlike zasedenosti s časovni odvodom 
\begin{equation}  
\label{5.50}
\frac{d\zeta}{dt}=\frac{2i}{\hslash}\, \mathcal{V} E(t)\left(\eta- \eta^{\ast}\right).
\end{equation}
S tem smo iz Schr\"odingerjeve enačbe dobili enačbi za časovni razvoj
dipolnega momenta in obrnjene zasedenosti, vendar ju moramo še dopolniti.

Naj bo atom na začetku v vzbujenem stanju $|2\rangle$ in naj bo $E(t)=0$. Začetna
vrednost obrnjene zasedenosti je tako $\zeta(0)=1$. Po enačbi (\ref{5.50}) naj bi 
bil odvod obrnjene zasedenosti enak nič in $\zeta$ konstantna. 
Vendar vemo, da se atom, ki je v vzbujenem stanju, sčasoma vrne v
osnovno stanje. Verjetnost za tak spontan prehod na časovno enoto smo označili z $A$ (glej 
razdelek~\ref{chap:ASSS}).

Poleg tega moramo upoštevati še črpanje, s katerim\index{Črpanje}
vzdržujemo obrnjeno zasedenost in s tem lasersko delovanje. Za podroben
opis črpanja bi morali v Hamiltonov operator dodati ustrezne člene in
morda upoštevati še druga stanja atomov, vendar nas take podrobnosti 
ne zanimajo. Če bi ne bilo črpanja, bi bila stacionarna vrednost
v odsotnosti laserskega polja
\begin{equation}
 \zeta_{\mathrm{stac}}= |c_2|^2-|c_1|^2 = -1.
\end{equation}
Zaradi črpanja zavzame obrnjena zasedenost neko vrednost $-1<\zeta_0<1$, 
odvisno od moči črpanja. Enačbi (\ref{5.50}) dodamo ustrezen člen
\begin{equation}  
\label{5.51}
\frac{d\zeta}{dt}=A\left(\zeta_0-\zeta\right)+\frac{2i}{\hslash}\,\mathcal{V}E(t)\left(\eta-\eta^{\ast}\right),
\end{equation}
ki opisuje vpliv črpanja in spontane prehode v nižje stanje. 

Dopolnimo še enačbo za časovno spreminjanje električnega dipola 
(enačba~\ref{5.49}). Pri $E(t)=0$ da zapisana enačba časovno odvisnost 
$\eta \propto e^{-i \omega_0 t}$, to je brez dušenja. Vendar vemo, da električna
polarizacija v mešanem stanju razpada vsaj zaradi spontanega sevanja, lahko
pa še zaradi drugih vplivov, na primer trkov. Označimo
koeficient dušenja polarizacije snovi, ki meri tudi spektralno širino svetlobe, 
izsevane pri prehodu $2\rightarrow 1$, z $\gamma$. Zapišemo
dopolnjeno enačbo
\begin{equation}  
\label{5.52}
\frac{d\eta}{dt}=- \left(i \omega_0+\gamma\right)\eta+\frac{i}{\hslash}\,\mathcal{V} E(t) \zeta
\end{equation}
in kompleksno konjugirano enačbo
\begin{equation}  
\label{5.52c}
\frac{d\eta^*}{dt}=-\left(-i \omega_0+\gamma\right)\eta^*-\frac{i}{\hslash}\,\mathcal{V}E(t) \zeta.
\end{equation}
Dobili smo sistem sklopljenih diferencialnih enačb, ki opisuje časovno
spreminjanje obrnjene zasedenosti in dipolnega momenta atoma. 
\vglue-2truemm
\begin{remark}
 Enačbe~(\ref{5.51}), (\ref{5.52}) in (\ref{5.52c}) pogosto imenujemo \index{Maxwell-Blochove enačbe}
 Maxwell-Blochove ali optične Blochove enačbe\footnote{Švicarsko-ameriški fizik 
 in nobelovec Felix Bloch, 1905--1983.}. Osnovne Blochove enačbe opisujejo gibanje 
 jedrskega magnetnega  momenta v elektromagnetnem polju, zato so jih najprej uporabili za opis 
 jedrske magnetne in elektronske spinske resonance. 
\end{remark}
\vglue-2truemm
Za opis potrebujemo še enačbo za polje $E(t)$. To naredimo klasično, tako da
jakost električnega polja zadošča valovni enačbi. Upoštevati moramo, 
da je v snovi tudi električna polarizacija različna od nič. 
Valovna enačba v skalarni obliki je tedaj\index{Valovna enačba}
\begin{equation}  
\label{5.54}
\nabla^2 E-\frac{1}{c^2}\frac{\partial^2 E}{\partial t^2}=\mu_0 \frac{\partial^2 P}{\partial t^2},
\end{equation}
pri čemer je v primeru, da so vsi atomi enakovredni, električna polarizacija enaka
\begin{equation}  
\label{5.53}
P=\frac{N}{V}p = -\frac{N}{V}\,\mathcal{V}\left(\eta+\eta^{\ast}\right)=P^+ + P^-,
\end{equation}
kjer smo z $V$ označili volumen in z $N$ število atomov.

Namesto mikroskopske količine $\zeta$ uvedemo gostoto obrnjene
zasedenosti $Z=(N/V)\zeta$ in enačbe (\ref{5.51}), (\ref{5.52}) in (\ref{5.52c})
prepišemo v obliko 
\begin{eqnarray}
\frac{dZ}{dt} &=& A\left(Z_0-Z\right)+\frac{2i}{\hslash}E\left(P^- - P^+\right), \label{5.56} \\
\frac{dP^+}{dt}&=&\left(-i \omega_0-\gamma\right)P^{+}-\frac{i}{\hslash} \mathcal{V}^2 E  Z \qquad 
\mathrm{in} \label{5.56b}\\
\frac{dP^-}{dt}&=&\left(i \omega_0-\gamma\right)P^{-}+\frac{i}{\hslash} \mathcal{V}^2 E  Z.\label{5.56c} 
\end{eqnarray}
Opozorimo še, da je prehod z enačb (\ref{5.51}--\ref{5.52c}) na (\ref{5.56}--\ref{5.56c}) 
mogoč le, kadar so vsi atomi enakovredni, to je, kadar ni nehomogene razširitve 
spektra.\footnote{Primer nehomogene razširitve je opisan v npr.  
H. Haken, {\it Laser Theory}, Springer (1984).}

Enačbe (\ref{5.56}--\ref{5.56c}), skupaj z valovno enačbo (enačba~\ref{5.54}) podajajo
semiklasični opis interakcij svetlobe s snovjo. Iz izpeljave je razvidno, da
je v semiklasičnem opisu spontano sevanje obravnavano pomanjkljivo, le s fenomenološkim
nastavkom. To je moč popraviti tako, da tudi elektromagnetno polje kvantiziramo. 
Kljub tej pomanjkljivosti je s semiklasičnim modelom mogoče zelo podrobno opisati 
večino pojavov v laserjih in tudi druge pojave širjenja svetlobe po snovi. Je pa 
reševanje zapisanega sistema nelinearnih parcialnih diferencialnih enačb  
na splošno zelo težavno.

\subsection*{Primer enofrekvenčnega laserja}\index{Laser!enofrekvenčni}
Semiklasične enačbe pobliže spoznajmo na najenostavnejšem primeru. To naj bo laser, 
v katerem je vzbujeno le eno resonatorsko nihanje, označimo ga z $n$.
Polje ima tedaj obliko
\begin{equation}  
\label{5.57}
E(\mathbf{r},t)= E_n(t)u_n(\mathbf{r}),
\end{equation}
kjer je $u_n(\mathbf{r})$ krajevni del lastnega nihanja resonatorja, ki
zadošča enačbi 
\begin{equation}  
\label{5.58}
\nabla^2 u_n+\frac{\omega_n^2}{c^2}u_n=0.
\end{equation}
Funkcija $E_n(t)$ opisuje časovno odvisnost. Za laser v stacionarnem
delovanju je periodična, vendar njena krožna frekvenca $\Omega$ ni nujno enaka lastni
krožni frekvenci praznega resonatorja $\omega_n$. Krožno frekvenco 
$\Omega$ moramo še izračunati.

Razvijmo še električno polarizacijo po lastnih funkcijah $u_n(\mathbf{r})$. 
Ker so lastne funkcije med seboj ortogonalne, preide valovna enačba (\ref{5.54}) 
ob upoštevanju enačbe~(\ref{5.58}) v 
\begin{equation}  
\label{5.59}
\omega_n^2 E_n+\frac{d^2 E_n}{dt^2}= 
-\frac{1}{\epsilon_0}\frac{d^2P_n}{dt^2}.
\end{equation}

Razstavimo $E_n(t)$ na dva dela: 
\begin{equation}  \label{5.60}
E_n(t)=E_n^+(t)+E_n^-(t)=A^+(t)e^{-i
\omega_nt}+A^-(t)e^{i \omega_nt}.
\end{equation}
Dejanska krožna frekvenca laserja je blizu $\omega_n$, zato pričakujemo,
da se bosta amplitudi $A^{\pm}(t)$ v primerjavi z $e^{-i \omega_nt}$ le
počasi spreminjali. Izračunajmo 
\begin{equation}  
\label{5.61}
\frac{d^2 E_n^+}{dt^2}=-\omega_n^2 E_n^+ - 2i \omega_n 
\frac{dA^+}{dt} e^{-i \omega_nt} \approx 
-\omega_n^2 E_n^+-2i \omega_n\left(\frac{dE_n^+}{dt}+
i \omega_nE_n^+\right).
\end{equation}
Izpustili smo drugi odvod $A$ po času in s tem napravili približek počasi spreminjajoče se amplitude.

Električna polarizacija snovi je približno periodična s krožno frekvenco $\omega_0 \approx \omega_n$.
Tudi amplituda polarizacije se le počasi spreminja, zato lahko privzamemo, da je 
drugi odvod $P_n^+$ po času približno enak $-\omega_0^2 P_n^+$. Z uporabo tega
približka in enačbe~(\ref{5.61}) preide valovna enačba (enačba~\ref{5.54}) za izbrano nihanje v 
\begin{equation}  
\label{5.62}
\frac{dE_n^+}{dt}=-i \omega_n E_n^++
\frac{i \omega_0}{2\epsilon_0}P_n^+.
\end{equation}

Upoštevajmo še, da je polje v praznem resonatorju dušeno, in ustrezno popravimo enačbo
\begin{equation}  
\label{5.63}
\frac{dE_n^+}{dt}=\left(-i \omega_n-\frac{1}{\tau}\right) E_n^+ 
+\frac{i \omega_0}{2\epsilon_0}P_n^+.
\end{equation}
Kadar v resonatorju ni snovi, je dobljena enačba enaka enačbi (\ref{3.36}).

Enačbe (\ref{5.56}--\ref{5.56c}) so nelinearne, zato jih ni moč kar tako
prepisati za razvoj po lastnih nihanjih resonatorja. Pri enačbah za
razvoj polarizacije (enačbi~\ref{5.56b} in \ref{5.56c}) v zadnjem členu na desni 
nastopa produkt komponente polja $E_n$ in obrnjene zasedenosti $Z$. Bistveni
prispevek je od krajevnega povprečja $\overline{Z}$, ki se s časom le počasi spreminja.
Seveda vsebuje $Z$ tudi krajevno odvisne komponente, ki so pomembne 
predvsem zato, ker sklapljajo različna lastna nihanja resonatorja, vendar to
presega našo obravnavo. Iz enačbe~(\ref{5.56b}) dobimo
\begin{equation}  
\label{5.64}
\frac{d P_n^+}{dt}=\left(-i \omega_0-\gamma\right)P_n^{+}-\frac{i}{\hslash}
\mathcal{V}^2 E_n^+ \overline{Z}.
\end{equation}
Povprečje $\overline{Z}$ izrazimo iz enačbe (\ref{5.56}). V zadnjem členu nastopajo produkti 
$E^{\pm}P^{\pm}=E_n^{\pm}P_n^{\pm} u_n^2\left(\mathbf{r}\right)$,
ki jih moramo prostorsko povprečiti. Funkcije $u_n\left(\mathbf{r}\right)$ so
normirane, tako da je $\int u_n^2\left(\mathbf{r}\right) \,dV=V$ in $\overline{u_n^2\left(\mathbf{r}\right)}=1$.
Sledi
\begin{equation}  
\label{5.65}
\frac{d\overline{Z}}{dt}= A\left(\overline{Z}_0-\overline{Z}\right)+ \frac{2i}{\hslash}\left(E_n^+
+E_n^-\right)\left(P_n^- - P_n^+\right),
\end{equation}
kjer je $\overline{Z}_0$ povprečje nenasičene zasedenosti $Z_0$. V zadnjem
členu nastopajo produkti, ki nihajo s krožnimi frekvencami $\omega_n-
\omega_0$ in $\omega_n+ \omega_0$. Frekvenci sta si zelo blizu,
zato je njuna vsota znatno večja od razlike. Hitro spreminjajoče se člene 
$E_n^+P_n^+$ in $E_n^- P_n^-$ izpustimo, 
saj skoraj nič ne vplivajo na valovanje blizu $\omega_n$. Časovno odvisnost 
$\overline{Z}$ tako zapišemo 
\begin{equation}  
\label{5.66}
\frac{d\overline{Z}}{dt}= A\left(\overline{Z}_0-\overline{Z}\right)+\frac{2i}{\hslash}\left(E_n^+
P_n^- - E_n^- P_n^+\right).
\end{equation}
Enačbe (\ref{5.63}), (\ref{5.64}) in (\ref{5.66}) predstavljajo skupaj s konjugirano
kompleksnimi enačbami za $E_n^-$ in $P_n^-$ zaključen
sistem, ki opisuje delovanje enofrekvenčnega laserja. Uporabimo jih za
izračun frekvence izhodne svetlobe.

Naj bo stanje stacionarno. Tedaj polje zapišemo v obliki $E_{n
}^{+}=E_{0}e^{-i\Omega t}$, kjer je $E_{0}$ realna konstanta, krožna frekvenca
svetlobe $\Omega$ pa je blizu $\omega _{0}$ in $\omega _{n }$. V
stacionarnem stanju ima električna polarizacija enako časovno odvisnost in 
$P_{n }^{+}=P_{0}e^{-i\Omega t}$. Tedaj je v enačbi~(\ref{5.66}) drugi
oklepaj konstanten in povprečna gostota obrnjene zasedenosti $\overline{Z}$ 
od časa neodvisna. 

Sistem enačb (\ref{5.63}), (\ref{5.64}) in (\ref{5.66}) 
tako da 
\begin{eqnarray}
\left(i\left(\Omega - \omega_n\right)-\frac{1}{\tau}\right) E_{0}+\frac{i\omega _{0}}
{2\epsilon _{0}}\,P_{0} &=&0,  \label{5.67} \\
\left(i\left(\Omega-\omega_{0}\right)-\gamma\right)P_{0}-\frac{i}{\hslash}\mathcal{V}^{2}
E_{0}\overline{Z} &=&0 \qquad \mathrm{in}\label{5.67b}\\
A\left(\overline{Z}_{0}-\overline{Z}\right)+\frac{2i}{\hslash }\,E_{0}\left(P_{0}^{*}-P_{0}\right) &=&0.
\end{eqnarray}

Najprej iz druge enačbe izrazimo $P_0$, ga vstavimo v tretjo in izračunamo $\overline{Z}$. Dobimo
\begin{equation}  
\label{5.68}
\overline{Z}=\overline{Z}_0\left(1+\frac{4\mathcal{V}^2}{\hslash^2 A}\,E_0^2\, \frac{\gamma}
{\left(\omega_0-\Omega\right)^2+\gamma^2}\right)^{-1}.
\end{equation}
Ta izraz že poznamo. V njem prepoznamo Einsteinov
koeficient $B$ (enačba~\ref{4.60}),
$E_0^2$ je sorazmeren gostoti energije polja v resonatorju, medtem ko 
zadnji ulomek v oklepaju podaja obliko homogeno razširjene atomske
črte (\ref{eq:homogenasirina}). Zapišemo
\begin{equation}  
\label{5.69}
\overline{Z}=\overline{Z}_0\left(1+\frac{2B}{Ac}\,g\left(\omega_0- \Omega\right)j\right)^{-1}.
\end{equation}
To je enako izrazu za nasičenje zasedenosti stanj, ki smo ga
izpeljali iz zasedbenih enačb v četrtem poglavju (enačba~\ref{4.33}).

Vstavimo $P_0$ iz prve enačbe sistema (\ref{5.67}) v drugo (\ref{5.67b})
\begin{equation}  
\label{5.70}
E_0\left(-i\Omega+i\omega_n+\frac{1}{\tau}\right) \left(i\Omega- i\omega_0
-\gamma\right)=-\frac{V^2 \omega_0}{2\hslash\epsilon_0}\,E_0\,\overline{Z}.
\end{equation}
V delujočem laserju je $E_0\ne 0$, zato ga lahko krajšamo. Vrednost $\overline{Z}$ je
realna, tako da mora biti imaginarni del leve strani enak nič. Dobimo
\begin{equation}  
\label{5.71}
\left(\Omega- \omega_n\right)\gamma+\left(\Omega- \omega_0\right)\frac{1}{\tau} = 0.
\end{equation}
Od tod lahko izračunamo krožno frekvenco delujočega laserja 
\begin{equation}  \label{5.72}
\Omega=\frac{\omega_n\gamma+ \omega_0\frac{1}{\tau}}{\gamma + \frac{1}{\tau}}.
\end{equation}
Krožna frekvenca torej ni enaka krožni frekvenci praznega resonatorja $\omega_n$,
temveč je premaknjena proti centru atomske črte $\omega_0$. Premik je
odvisen od razmerja širine atomske črte in izgub v resonatorju.

Opisani primer uporabe semiklasičnih enačb je zelo preprost. Prava moč
modela se pokaže pri obravnavi večfrekvenčnega laserja, na primer pri
računu uklepanja faz laserskih nihanj, kar presega okvir te 
knjige.\footnote{Glej npr. Y. Khanin, {\it Fundamentals of Laser Dynamics}, Cambridge International Science
Publishing (2006).} 
