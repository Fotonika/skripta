\chapterimage{slike/Trinivojski.png} 
\chapter{Laser}

V tem poglavju bomo opisali delovanje laserjev in spoznali njihove prednosti
pred navadnimi svetili. Zapisali bomo zasedbene enačbe in razložili delovanje 
sunkovnih laserjev ter spoznali nekaj 
, sprektralno širino, relaksacijske oscilacije, razložili delovanje sunkovnih
laserjev, meritev, stabilizacija frekvence, glavnik..*****

\section{Laser}

V prejšnjem poglavju smo spoznali, da se v sredstvu z obrnjeno zasedenostjo
med dvema nivojema pojavi optično ojačevanje. Zdaj pa postavimo tako snov 
v optični resonator. Na začetku dobimo predvsem spontano izsevano svetlobo, ki se
odbija med zrcaloma resonatorja in se pri prehodu skozi snov ojačuje. Tako
se vzbujajo nihanja resonatorja z nihajnimi frekvencami blizu frekvence
atomskega prehoda, pri katerih snov ojačuje. Energija nihanj z dovolj
majhnimi izgubami narašča, dokler se ojačenje ne izenači z izgubami.
Sistem preide v stacionarno stanje in seva močno koherentno svetlobo. Tak
izvor svetlobe imenujemo laser. Beseda laser je nastala iz kratice za {\it Light
Amplification by Stimulated Emission of Radiation} --  ojačenje svetlobe s
stimulirano emisijo sevanja.
\begin{figure}[h]
\centering
\def\svgwidth{110truemm} 
\input{slike/06_shema.pdf_tex}
\caption{Shema laserja, ki ga sestavljajo ojačevalno sredstvo, črpalni mehanizem
in resonator.}
\label{fig:shemalaserja}
\end{figure}
\vglue-0.5truecm
V grobem ima laser tri ključne sestavne dele: ojačevalno sredstvo, 
črpalni mehanizem in zrcali, ki tvorita resonator (slika~\ref{fig:shemalaserja}). Črpalno
sredstvo vzdržuje obrnjeno zasedenost v ojačevalnem sredstvu, resonator pa zagotavlja,
da je dovolj valovanja, da stimulirano sevanje prevlada nad spontanim.
Odbojnost vsaj enega od zrcal mora biti manjša od 1, zato da skozenj izhaja svetloba.

\begin{remark}
Kot klasično analogijo za laser vzemimo klarinet, ki je sestavljen iz 
cevi in iz ustnika. Cev deluje kot zvočni resonator, v katerem nastane 
stojni zvočni val, pri čemer je frekvenca stoječega vala določena z 
dolžino cevi in s številom vozlov. Naloga ustnika je dovajati energijo 
in s tem vzdrževati konstantno amplitudo nihanja. To glasbenik doseže s 
pihanjem v ustnik in tresenjem prožnega jezička, ki s tresljaji poizvaja 
zvok. Tresenje jezička je približno periodično in vsebuje mnogo različnih 
frekvenc, tudi take, ki ustrezajo stoječim valovom v cevi. 
Ko amplituda tlaka v cevi naraste nad neko mejo, pride do zanimivega
pojava. Nihanje tlaka v gornjem koncu cevi povratno deluje na ustnik
in ga sili, da niha s frekvenco najbolj vzbujenega stoječega vala v cevi,
nihanje jezička z drugimi frekvencami pa zamre. Moč pihanja gre le še v
nihanje jezička s pravo frekvenco in ojačuje nihanje zračnega stolpca. 
Tako s povratno zvezo med nihanjem jezička in stoječim valovanjem v cevi
vzdržujemo stoječe valovanje s konstantno amplitudo. 
\end{remark}

Pri osnovnem opisu delovanja laserja se omejimo na najpreprostejši model in 
privzemimo, da je le eno resonatorsko nihanje tako, da njegova frekvenca
sovpada s frekvenco prehoda aktivne snovi. Ta privzetek v večini laserjev ni
avtomatično izpolnjen, vendar ga je pogosto mogoče doseči z dodatnimi elementi 
v resonatorju. Aktivno snov oziroma ojačevalno sredstvo v laserju stalno 
črpamo in s tem vzdržujemo obrnjeno zasedenost. 

Naj bo $W$ energija svetlobnega valovanja v resonatorju. Zaradi izgub skozi
zrcali in zaradi absorpcije ter sipanja v resonatorju se energija na en obhod 
resonatorja zmanjša za (enačba~\ref{eq:Lambda})
\begin{equation}
\Delta W_{\rm izgube}=-\Lambda W=-\left((1-{\cal {R}}_{1})+(1-{\cal {R}}_{2})
-2\alpha L\right)W,
\label{5.1}
\end{equation}
kjer so $\Lambda $ celotne izgube, $\alpha$ so izgube na enoto poti zaradi
absoprcije in sipanja, $L$ je dolžina resonatorja, 
${\cal {R}}_{1}$ in ${\cal {R}}_{2}$ pa sta odbojnosti
obeh zrcal. V ojačevalnem sredstvu, v katerem vzdržujemo obrnjeno zasedenost,
pride do ojačevanja s stimuliranim sevanjem. Enegija nihanja resonatorja 
se tako na na en obhod po enačbi (\ref{eq:djG}) poveča za 
\begin{equation}  
\Delta W_{\rm oja\check{c}enje}=\frac{G}{1+W/W_s}\,W\, 2L'.
\label{5.2}
\end{equation}
Namesto saturacijske gostote svetlobnega toka $j_s$ smo vpeljali
saturacijsko energijo $W_s=Vj_s/c$, $L'$ pa označuje dolžino ojačevalnega
sredstva. Račun pogosto poenostavimo, tako da vzamemo $L'=L$, tukaj pa zaradi
jasnosti obdržimo ločen zapis. Privzeli smo tudi, da je ojačenje na en
obhod dovolj majhno, da enačbe (\ref{eq:djG}) ni treba integrirati.

V stacionarnem stanju se izgube izenačijo z ojačenjem
\beq
|\Delta W_{\rm izgube}|=|\Delta W_{\rm oja\check{c}enje}|,
\eeq
od koder sledi
\begin{equation}  
\Lambda\, W=\frac{G\,2L'}{1+W/W_s}\,W.
\label{5.3}
\end{equation}
Ta enačba ima dve rešitvi za energijo svetlobnega nihanja. 
Prva je $W=0$, druga pa  
\boxeq{5.4}{ 
W=W_s \left(\frac{G}{G_{\rm pr}}-1\right),
}
pri čemer je 
\boxeq{5.5}{
G_{\rm pr} = \frac{\Lambda}{2L'}.
}
Vidimo, da je druga rešitev za energijo nihanja pozitivna le v primeru,
da je ojačenje večje od ojačenja praga $G> G_{\rm pr}$. Združimo 
obe rešitvi in zapišemo, da je energija svetlobe v laserju pod pragom enaka
nič, nad pragom pa linearno narašča z ojačenjem (slika~\ref{fig:energija}).
Ojačenje je seveda odvisno od stopnje obrnjene zasedenosti, ta pa je povezana
z močjo optičnega črpanja.
\begin{figure}[h]
\centering
\def\svgwidth{80truemm} 
\input{slike/06_energija.pdf_tex}
\caption{Odvisnost energije svetlobe v laserju od ojačenja}
\label{fig:energija}
\end{figure}

Izhodna moč laserja je enaka deležu energije, ki na en prelet zapusti
resonator skozi izhodno zrcalo, deljenim s časom obhoda resonatorja $2L/c$ 
\boxeq{5.6}{
P=(1-{\cal {R}}_{1})\frac{c}{2L}\,W.
}
Ker so vsi faktorji v gornji enačbi konstantni, je izhodna moč kar sorazmerna
z energijo svetlobe v resonatorju. Odvisnost izhodne moči od črpanja je 
tako do konstante enaka, kot je prikazana na sliki (\ref{fig:energija}). 

\begin{definition}
Izračunaj izhodno moč iz zrcala pri dani dolžini resonatorja $L=L'$, 
odbojnosti enega zrcala ${\cal {R}}_{2}=1$, 
notranjih izgubah na enoto dolžine $\alpha$ in ojačenju $G$ ter pokaži, da
je izhodna moč največja pri 
\beq
{\cal {R}}_1 = 1-2\alpha L \left(\sqrt{\frac{G}{\alpha}}-1\right).
\eeq
\end{definition}

\section{Zasedbene enačbe}
Za podrobnejši opis se vrnemo k enačbam za zasedenost atomskih nivojev 
(enačbe~\ref{4.39.1}--\ref{4.39}), ki jim dodamo še enačbo za energijo 
lastnega valovanja v resonatorju. Še naprej obravnavajmo primer, ko je 
vzbujeno le eno resonatorsko stanje, opazujemo pa prehode med prvim 
in drugim vzbujenim stanjem (slika~\ref{fig:3nivojski}\,b). 

Brez škode lahko enačbe za zasedenost precej poenostavimo. Privzamemo, 
da je razpadni čas spodnjega laserskega stanja $|1\rangle$, 
ki ga določa koeficient $A_{10}$, dosti krajši od razpadnega
časa zgornjega stanja $|2\rangle$. Tedaj vsi atomi iz spodnjega stanja zelo 
hitro preidejo v osnovno stanje in $N_1\simeq 0$, če le ni preveč
stimuliranega sevanja. Zasedenost atomskih stanj lahko torej popišemo z eno
samo spremenljivko $N_2$. Zanemarimo tudi spontano sevanje iz drugega vzbujenega
nivoja $A_{20} \approx 0$. Energijo v izbranem stanju resonatorja zapišemo s
številom fotonov $n$, tako da je gostota energije polja $n\hbar\omega/V$,
kjer je $V$ volumen resonatorja. S tem dobimo za zasedenost 
\begin{eqnarray}  
\frac{dN_2}{dt}&=&rN-A_{21}N_2-B_{21}g(\omega)N_2\frac{\hbar \omega}{V}\,n \label{5.7.1}\\
&=&rN-A_{21}N_2-\sigma \frac{c}{V}\, N_2\,n. 
\label{5.7}
\end{eqnarray}

Energija svetlobe oziroma število fotonov v resonatorju se povečuje predvsem 
zaradi stimuliranega sevanja. V prejšnjem poglavju pa smo spoznali, da je verjetnost 
za prehod atoma iz višjega v nižje stanje z izsevanjem fotona v izbrano 
stanje elektromagnetnega polja sorazmerna z $n+1$ (enačba~\ref{4.59}), 
kjer je $n$ število fotonov v izbranem stanju. Če torej namesto $n$ v
zadnjem členu enačbe~(\ref{5.7}) pišemo $n+1$, opišemo poleg stimuliranega
sevanja tudi prispevek spontanega sevanja. Upoštevati moramo še, da 
se energija svetlobe v resonatorju zmanjšuje zaradi izgub skozi zrcali
in absorpcije ter sipanja, kar opišemo z razpadnim časom $\tau/2$
(enačba~\ref{eq:dW}). Sledi 
\begin{equation}
\frac{dn}{dt}=\sigma \frac{c}{V}\, N_2\,(n+1)-\frac{2}{\tau}\,n.
\label{5.8}
\end{equation}

Gornji dve enačbi predstavljata sistem dveh diferencialnih enačb za 
časovni razvoj števila fotonov v resonatorskem stanju in za zasedenost 
gornjega atomskega stanja. Sklopljeni enačbi sta nelinearni in ju ne znamo 
analitično rešiti. Vseeno pa lahko nekaj povemo o takem sistemu.

Poglejmo najprej stacionarne rešitve, za katere velja $\dot{N}_{2}=0$ in 
$\dot{n}=0$. Iz enačbe~(\ref{5.7}) izrazimo $N_{2}$ in ga vstavimo v enačbo~(\ref{5.8}).
Sledi 
\begin{equation}
\frac{2}{\tau }n\left(A_{21}V+\sigma c\,n\right)=
\sigma c\, r\,N\,(n+1).
\label{5.9}
\end{equation}
Enačbo zapišemo še bolj pregledno, če vpeljemo koeficient ojačenja $G$ (enačba~\ref{4.44})
in ojačenje na pragu $G_{\rm pr}$ (enačbi~\ref{taulambda} in \ref{5.5})
\beq
G_{\rm pr}\, n\, \left(1+\frac{\sigma c}{VA_{21}}n \right)= G(n+1)
\eeq
Vpeljemo še brezdimenzijsko konstanto $p$, pri čemer upoštevamo zvezo
med Einsteinovimi koeficienti (enačba~\ref{4.27})
\begin{equation}
p=\frac{VA_{21}}{\sigma c} = 
\frac{VA_{21}}{B_{21}\hbar \omega g}=\frac{V\omega ^{2}}{\pi
^{2}c^{3}g}\simeq 
\frac{V\omega ^{2}}{\pi ^{2}c^{3}}\Delta \omega.  
\label{5.10}
\end{equation}
V zadnjem izrazu smo privzeli, da je v $g(\omega-\omega_0)\simeq
1/\Delta \omega $. Parameter $p$ je torej približno enak produktu 
gostote stanj elektromagnetnega polja v resonatorju (enačba~\ref{4.4}) 
in širine atomskega prehoda (ter volumna), torej kar številu vseh stanj 
v frekvenčnem intervalu atomskega prehoda. To število je navadno precej 
veliko $p \sim 10^{8}$--$10^{10}$. S primerjavo izraza za $p$ 
(enačba~\ref{5.10}) z izrazom za saturacijsko gostoto toka (enačba~\ref{eq:jsatg})
vidimo, da je $p$ tudi število fotonov v resonatorju, pri katerem pride 
do nasičenja ojačenja, če je frekvenca nihanja resonatorja blizu 
centra atomske črte. 

Enačbo~(\ref{5.9}) zapišemo 
\begin{equation}
\frac{1}{p}\,n^{2}-(\frac{G}{G_{\rm pr}}-1)\,n-\frac{G}{G_{\rm pr}}=0
\label{5.11}
\end{equation}
s pozitivno rešitvijo 
\begin{equation}
n=\frac{p}{2}\left( \left(\frac{G}{G_{\rm pr}}-1\right)+\sqrt{\left(\frac{G}{G_{\rm pr}}
-1\right)^{2}+ \frac{4G}{pG_{\rm pr}}}\right).
\label{5.12}
\end{equation}
Ker je $p$ zelo veliko število, lahko koren razvijemo, če le ni ojačenje
preveč blizu praga, ko je $G/G_{\rm pr}\simeq 1$. Pod pragom je $G<G_{\rm pr}$ 
in 
\begin{equation}
n\approx \frac{p}{2}\left( \left(\frac{G}{G_{\rm pr}}-1\right)+\left(1
-\frac{G}{G_{\rm pr}}\right)+\frac{2G}{p(G_{\rm pr}-G)}\right) =\frac{G}{G_{\rm pr}-G}.
\label{5.13}
\end{equation}
Pri razvoju korena smo upoštevali, da mora biti pozitiven. Nad pragom je
število fotonov 
\begin{equation}
n\approx p\left(\frac{G}{G_{\rm pr}}-1\right).
\label{5.14}
\end{equation}

Poglejmo rezultat podrobneje. Pod pragom je število fotonov v izbranem
resonatorskem nihanju okoli ena do neposredne bližine praga, kjer hitro
naraste in doseže takoj nad pragom red velikosti $p$ (slika~\ref{fig:p}). 
Pod pragom gre praktično vsa moč črpanja, ki jo dovajamo v sistem, 
preko spontanega sevanja v veliko število stanj elektromagnetnega polja. 
Nad pragom povsem prevlada stimulirano sevanje v eno samo izbrano nihanje resonatorja. 
Prehod preko praga je zaradi velikega $p$ tako hiter, da ga ni mogoče izmeriti; 
izjema so polvodniški laserji, katerih volumen -- in posledično $p$ -- je zelo majhen, 
da je mogoče opaziti zvezen prehod preko praga.
\begin{figure}[h]
\centering
\def\svgwidth{70truemm} 
\input{slike/06_p.pdf_tex}
\caption{Odvisnost števila fotonov v resonatorju od ojačenja za $p=10^5$.}
\label{fig:p}
\end{figure}

Iz enačbe (\ref{5.8}) izračunamo še zasedenost zgornjega atomskega
nivoja v stacionarnem stanju
\begin{equation}  
N_2=\frac{2V}{\tau \sigma c}\frac{n}{n+1}.
\label{5.15}
\end{equation}
Na pragu je po enačbi (\ref{5.12}) $n=\sqrt{p}$. 
Sledi 
\begin{equation}  
N_{\rm 2pr}=\frac{2V}{\tau \sigma c}\frac{\sqrt{p}}{\sqrt{p}+1}.
\label{5.16}
\end{equation}
Ker je tudi $\sqrt{p}$ veliko število, iz gornjih enačb sledi, da 
obrnjena zasedenost narašča do bližine praga, nad pragom pa je praktično
konstantna in skoraj natanko enaka kot na pragu. To ni težko razumeti. Nad
pragom je število fotonov v resonatorju veliko in linearno narašča 
z močjo črpanja. S tem se povečuje hitrost praznjenja gornjega atomskega 
stanja s stimuliranim sevanjem, kar ravno izniči učinek povečanja črpanja. 
V stacionarno delujočem laserju torej ni mogoče povečati obrnjene zasedenosti 
nad vrednost na pragu $N_{\rm 2pr}$, kar ima pomembne praktične posledice, kot bomo
videli v nadaljevanju.

\begin{remark}
Obravnava laserja z zasedbenimi enačbami je seveda zelo groba. Nismo
upoštevali, da je prostorska odvisnost polja v delujočem laserju 
drugačna od lastnega stanja praznega resonatorja. Poleg tega smo
privzeli, da so atomi lahko le v lastnih energijskih stanjih, kar je res le
v primeru stacionarnih stanj brez zunanjega, časovno odvisnega polja
svetlobe. Bolj podroben pristop je semiklasični model, pri katerem 
za opis svetlobe uporabimo klasično valovno enačbo, za atome
pa kvantno mehaniko (glej poglavje~\ref{chap:semiklasicni}). Ta model
zadošča za opis skoraj vseh pojavov v laserjih razen vpliva spontanega sevanja. 
Za dosledno obravnavo tega je treba svetlobo opisati s pomočjo 
kvantne elektrodinamike, kar presega okvir te knjige.
\end{remark}

Povzemimo na kratko, kaj smo ugotovili o delovanju enofrekvenčnega laserja. 
Pri dovolj velikem ojačenju s stimuliranim sevanjem, ki pokriva izgube 
resonatorja, je v stacionarnem stanju energija in s tem amplituda 
izbranega lastnega nihanja resonatorja različna od nič. Frekvenca svetlobe je
določena z izbranim lastnim stanjem resonatorja, ki določa tudi prostorsko
odvisnost valovanja v resonatorju in izhodnega snopa. V navadnem stabilnem 
resonatorju ima polje obliko zelo blizu Gaussovega snopa, zato je tak tudi izhodni snop.
Gaussova prostorska odvisnost izhodnega snopa je morda najpomembnejša lastnost
laserjev. Gaussov snop se najmanj širi zaradi uklona in ga je mogoče
zbrati v piko približne velikosti valovne dolžine in se tako najbolj
približa idealno točkastemu izvoru svetlobe.

\section{Spektralna širina enega laserskega nihanja}

Poglejmo, kaj lahko povemo o spektralni 
širini svetlobe enofrekvenčnega laserja. Če bi se lastno stanje 
elektromagnetnega polja v resonatorju obnašalo kot klasično 
harmonsko nihalo, bi bil spekter laserja neskončno ozek. Vendar 
imajo laserji končno spektralno širino -- v idealnem primeru zaradi
kvantizacije elektromagnetnega polja, v praksi pa zaradi zunanjih motenj.
Poskusimo najprej oceniti razširitev zaradi vpliva kvantizacije. Zaradi nje
imamo poleg stimuliranega vedno prisotno tudi spontano sevanje. To 
predstavlja kvantni šum, ki povzroči razširitev spektra. 

Predstavimo amplitudo nihanja $E(t)$ na izbranem mestu v resonatorju kot kompleksno 
število, ki ga v kompleksni ravnini določata dolžina in faza (slika~\ref{fig:fazor}). 
Pri tem fazo določimo glede na neko začetno izbrano fazo. 
Ker je energija svetlobe sorazmerna s številom fotonov, je dolžina 
kar sorazmerna s korenom iz števila fotonov v nihanju. Dolžina se praktično
ohranja, saj jo vzdržuje stimulirano sevanje, ki ravno pokriva
izgube resonatorja. Pri tem ostaja nespremenjena tudi faza. Vendar se
faza spreminja zaradi majhnega prispevka spontanega sevanja.

\begin{figure}[h]
\centering
\def\svgwidth{70truemm} 
\input{slike/06_fazor.pdf_tex}
\caption{Amplituda polja v resonatorju in njena sprememba zaradi 
spontanega sevanja}
\label{fig:fazor}
\end{figure}

Pri spontani emisiji se izseva en foton s poljubno fazo. Prispevek h kompleksni
amplitudi ima torej dolžino 1 in poljubno smer (slika~\ref{fig:fazor}). Zanima
nas povprečni kvadrat spremembe faznega kota pri enem spontano izsevanem fotonu
\begin{equation}
\overline{\Delta \varphi_{1}^{2}}=\overline{\left(\frac{\cos\psi}{\sqrt{n} }\right)^2}
=\frac{1}{2\overline{n}},
\label{5.17}
\end{equation}
Zaporedne spontane emisije so med seboj neodvisne, zato izračunamo
povprečni kvadrat spremembe faze pri $N$ emisijah kar tako, da seštejemo
povprečne kvadrate za posamezne fotone 
\begin{equation}
\overline{\Delta \varphi_{m}^{2}}=m\overline{\Delta \varphi_{1}^{2}}=
\frac{m}{2\overline{n}}.
\label{5.18}
\end{equation}

Ocenimo še število spontano izsevanih fotonov na časovno enoto.
Vemo, da stimulirano sevanje ravno pokrije izgube resonatorja, zato je
stimulirano izsevanih fotonov na časovno enoto $2\overline{n}/\tau $. Vemo tudi, 
da je razmerje med verjetnostjo za stimulirano in spontano sevanje enako 
številu fotonov v danem stanju polja (enačba~\ref{4.56}), zato je število 
spontanih sevanj na časovno enoto kar $2/\tau $.
Tako je število spontano izsevanih fotonov v času $t$ enako $m=2t/\tau $ in 
\begin{equation}
\overline{\Delta \varphi^{2}(t)}=\frac{t}{\overline{n}\tau }.
\label{5.19}
\end{equation}
Čas $t_{p}$, v katerem se faza znatno spremeni, je torej
velikostnega reda 
\begin{equation}
t_{p}\sim \overline{n}\tau =\frac{W}{\hbar \omega }\,\tau =\frac{P}{\hbar
\omega }\tau ^{2}.
\label{5.20}
\end{equation}
Ker je število fotonov v nihanju nad pragom zelo veliko ($10^9$ v majhnem 
He-Ne laserju), $\tau$ pa je reda velikosti $10^{-7}$, je karakteristični
čas za fazno razširitev idealnega laserja $t_p \sim 100~\si{s}$. 
Iz enačbe (\ref{5.20}) vidimo tudi, da je spektralna širina, ki je
podana z $1/t_{p}$, obratno sorazmerna z izhodno močjo laserja. 
V neposredni bližini praga, kjer je $\overline{n}\sim 1$, pa je 
spektralna širina približno enaka širini nihanj praznega resonatorja.

Dejanski laserji seveda nimajo niti približno tako ozkega spektra, kot smo ga
pravkar ocenili. Vemo, da je frekvenca laserja določena z dolžino resonatorja 
($\omega=n\pi c/L$), pri čemer je $n$ zelo veliko celo število. Že majhna
sprememba dolžine resonatorja povzroči spremembo frekvence laserja, pri 
znatnejši spremembi dolžine pa lahko pride tudi do preskoka vzbujenega
stanja resonatorja, to je števila $n$. Dolžina resonatorja se 
spreminja predvsem zaradi zunanjih mehanskih motenj in zaradi spreminjanja
temperature. Če se posebej ne potrudimo s konstrukcijo resonatorja, so 
fluktuacije frekvence kar reda velikosti razmika med sosednimi stanji 
resonatorja, to je reda velikosti $\sim 100~\si{MHz}$. 
Fluktuacije dolžine je mogoče zmanjšati s skrbno konstrukcijo, 
temperaturno stabilizacijo in uporabo materialov z majhnim toplotnim raztezkom. 
Na tak način je mogoče dobiti laser z efektivno spektralno širino pod $\sim 1\si{MHz}$.

Tu velja opozoriti, da je narava spektralne razširitve v laserju 
drugačna kot v navadnih svetilih. V drugem poglavju smo videli, da 
intenziteta svetlobe navadnega svetila fluktuira na časovni skali 
koherenčnega časa, ki je obraten spektralni širini (poglavje~\ref{chap:kns}). 
Šum navadnih svetil je torej amplitudno moduliran šum. Pri 
enofrekvenčnem laserju je drugače. Amplituda in s tem intenziteta 
izhodne svetlobe je konstantna, fluktuira le frekvenca oziroma
faza. Šum laserja je torej v obliki frekvenčne modulacije.

\begin{remark}
Omenili smo, da lahko s posebno konstrukcijo laserjev dosežemo
spektralno širino pod $\sim 1\si{MHz}$. Vendar najmanjša dosežena spektralna širina
znaša $\sim 10\si{mHz}$, kar je še 8 velikostnih redov manj! Gre za prav poseben 
laser, ki je kar se da izoliran od okolice. Njegov resonator je zgrajen iz 
monokristalov silicija, hlajenega na $-150~\si{\celsius}$. Fluktuacije
dolžine resonatorja so tako pogojene s termičnimi fluktuacijami 
v odbojnih plasteh, ki znašajo okoli $10^{-17}\si{\metre}$. 
Koherenčna dolžina takega laserja je več
milijonov kilometrov\footnote{Phys. Rev. Lett. {\bf 118}, 263202 (2017).}. 
\end{remark}

\section{Primerjava laserjev in navadnih svetil}
Primerjajmo enofrekvenčni laser, v katerem je vzbujeno le eno osnovno
stanje resonatorja Gaussove oblike z navadnimi nekoherentnimi svetili. 

Svetlobni snop iz laserja, ima dve takoj očitni odliki. Je zelo
usmerjen in zelo enobarven. Prva lastnost je posledica tega, da je
lastno stanje stabilnega resonatorja Gaussove oblike in je zato tak tudi
izhodni snop. Divergenca takega snopa je posledica uklona in je 
najmanjša možna. Valovne fronte so gladke in na dani razdalji ves 
čas enake, zato je laserski snop prostorsko idealno koherenten. 
Koherenten Gaussov snop lahko z ustrezno optiko zberemo v piko velikosti
valovne dolžine, s čimer dosežemo že pri majhni izhodni moči zelo veliko
gostoto svetobnega toka. To je zelo uporabno v tehnologiji za natančno in
čisto obdelavo materialov ter v medicini, kjer laserje uporabljajo za
zahtevne kirurške posege.

Kako pa je z navadnimi svetili? V njih vsak atom sveti neodvisno, zato
izsevana svetloba ni prostorsko koherentna. Valovna fronta na danem 
mestu je nepravilna in se v koherenčnem času znatno spremeni. 
Vendar tudi iz svetlobe navadnega nekoherentnega svetila lahko pripravimo
koherenten snop, če na dano razdaljo od svetila postavimo zaslonko, ki
je manjša od koherenčne ploskve na tistem mestu (glej 
razdelek~\ref{Prostorska-koherenca}). Ocenimo moč snopa za zaslonko. 

Svetilo naj ima svetlost $B$\footnote{Svetilnost je moč, izsevana v dan 
prostoski kot $I = dP/d\Omega$; svetlost pa je svetilnost na enoto ploskve
$B=I/S = dP/Sd\Omega$.}. Pri najsvetlejših nekoherentnih izvorih, 
to so živosrebrne svetilke, doseže $B$ vrednost do $100~\si{W/cm^{2}}$. 
Moč snopa za zaslonko, ki prepušča svetlobo skozi prostorski kot $\Delta\Omega$, 
je
\begin{equation}
P=BS_{0}\Delta \Omega =\frac{BS_{0}S_{c}}{z^{2}}\sim \frac{BS_{0}}{z^{2}}\,
\frac{\lambda ^{2}z^{2}}{S_{0}}=B\lambda ^{2}.
\label{5.21}
\end{equation}
Pri tem je $S_{0}$ površina svetila, $z$ oddaljenost od zaslonke, 
$S_{c}$ pa velikost koherenčne ploskve, za katero smo uporabili oceno 
(enačba~\ref{eq:koherencna-ploskev}). Pri svetlosti $100~\si{W/cm^{2}}$ 
znaša moč, ki preide skozi zaslonko, le približno $3\cdot10^{-7}~\si{W}$.
Pri enaki divergenci žarka je torej močno navadno svetlilo štiri rede
velikosti šibkejše od zelo šibkih laserjev z močjo $1~\si{mW}$. 

Druga odlična lastnost svetlobe iz enofrekvenčnega laserja je zelo majhna
spektralna širina. Z nekaj truda je ta lahko pod $1~\si{kHz}$, emisijske 
črte v plinu pa so zaradi Dopplerjeve razširitve široke vsaj nekaj $\si{GHz}$, 
pa še to le v razmeroma redkem in hladnem plinu, kjer je svetlost majhna.
Primerjajmo spektralno gostoto moči laserja in navadnih svetil. Majhen He-Ne
laser seva $1~\si{mW}$ v približno $10^{7}~\si{Hz}$, tako da je spektralna gostota
moči $dP/d\nu \sim 10^{-10}~\si{W/Hz}$. Po drugi strani zelo svetla 
živosrebrna svetilka seva v močno razširjene spektralne črte s širino okoli 
$10~\si{nm}$, kar ustreza $\sim 10^{13}~\si{Hz}$. 
Spektralna gostota v koherentnem snopu, ki ga pripravimo iz
take svetilke, bo tako le okoli $3\cdot 10^{-20}~\si{W/Hz}$. Šolski
He-Ne laser tako prekaša najmočnejše nekoherentno svetilo za 10
velikostnih redov. Z laserji je seveda mogoče doseči znatno večje
moči, v sunkih tja do $10^{12}$ W, tako da po spektralni gostoti moči v
koherentnem snopu laserji prekašajo običajna svetila preko 20
velikostnih redov. Verjetno v zgodovini težko najdemo še kak drug izum, 
ki je prinesel tolikšno izboljšavo v neki bistveni količini in tako ni 
čudno, da je prihod laserjev v začetku 60-ih let povzročil preporod optike.

\begin{remark}
Med laserji in navadnimi svetili je še ena pomembna, a manj opazna
razlika. Z ustreznim interferometrom lahko snop iz nekoherentnega
svetila filtriramo, tako da dobimo enako ozko spektralno širino kot 
iz laserja, seveda z znatno manjšo močjo. Vendar je narava spektralne razširitve
različna. Šum laserja je v obliki frekvenčne modulacije, 
šum nekohernetnega svetila pa je v obliki amplitudne modulacije.
\end{remark}

\section{Večfrekvenčni laser}
Do zdaj smo obravnavali laserje, v katerih je bilo vzbujeno eno samo
valovanje. Vendar je ojačevalna širina večine aktivnih sredstev 
navadno večja od razlike med frekvencami posameznih 
stanj resonatorja. V plinih, na primer, je ojačevalna širina zaradi 
Dopplerjevega pojava vsaj nekaj $\si{GHz}$, lastne frekvence resonatorja 
pa so pri $30~\si{cm}$ dolgem resonatorju razmaknjene za $500~\si{MHz}$. 
Tako se lahko zgodi, da ojačenje v laserju za več nihanj
presega ojačenje na pragu in vzbujenih je več nihanj hkrati.
Svetloba iz takega večfrekvenčnega laserja ni več monokromatska,
temveč je sestavljena iz množice ozkih črt znotraj ojačevalnega pasu.
Izsevana svetloba tako ni bistveno bolj monokromatska od ustrezne spektralne 
komponente svetlečega plina. Ostaja pa prostorsko koherentna.

Za holografijo, interferometrijo in nekatere spektroskopske uporabe
potrebujemo ozko spektralno črt. Zato moramo poskrbeti, da je vzbujeno le
eno nihanje resonatorja, najbolje tisto, ki je najbližje vrhu ojačenja
ativnega sredstva. To dosežemo tako, da za vsa ostala nihanja povečamo izgube, 
na primer s Fabry-Perotovim interferometrom, ki ga vstavimo v laserski resonator
(slika~\ref{fig:FPres}). Njegova prepustnost v odvisnosti od frekvence $\omega$, debeline $L$,
prepustnosti zrcal ${\cal R}$ in nagiba glede na os resonatorja $\varphi$ 
je podana z enačbo~(\ref{eq:FP}) 
\begin{equation}
T=\frac{1}{1+\frac{4{\cal {R}}}{(1-{\cal {R}})^{2}}\sin^{2}(\frac{\omega}{c}L\cos \varphi )}
\label{5.22}
\end{equation}
in jo kaže slika~(\ref{fig:Fabry-Perot}). Razmik med zrcaloma in nagib interferometra
izberemo tako, da vrh prepustnosti sovpada z izbranim stanjem resonatorja. 
Izgube za ostala nihanja, ki bi sicer bila ojačana, se tako povečajo in laser
sveti le pri eni sami izbrani frekvenci. Na sliki 
(\ref{fig:FPmodes}) so prikazana ojačana lastna stanja resonatorja in prepustnost
Fabry-Perotovega interferometra, ki vodi do enega samega ojačanega stanja. 
Ker zadošča že zmerno povečanje izgub, je reflektivnost zrcal interferometra 
običajno dokaj nizka, pod 0,5. 

\begin{figure}[h]
\centering
\def\svgwidth{90truemm} 
\input{slike/06_FPres.pdf_tex}
\caption{Shema laserja, v katerega vstavimo Fabry-Perotov interferometer. Tako dosežemo,
da laser deluje pri eni sami izbrani frekvenci.}
\label{fig:FPres}
\end{figure}

\begin{figure}[h]
\centering
\def\svgwidth{100truemm} 
\input{slike/06_FPmodes.pdf_tex}
\caption{Lastne frekvence resonatorja (a) in frekvenčna odvisnost ojačenja z označenim 
pragom ojačenja (b). Z rdečo so označene tiste lastne frekvence, ki se v laserju ojačujejo. 
Ko dodamo Fabry-Perotov interferometer s dano prepustnostjo (c), povečamo
izgube za vse načine, ki bi sicer bili ojačani, razen za enega. 
Tako dosežemo delovanje laserja pri eni sami frekvenci (d).}
\label{fig:FPmodes}
\end{figure}

\begin{remark}
Nagib interferometra omogoča natančno spreminjanje izbrane frekvence, poleg tega
pa je nujen, da se neprepuščena svetloba odbije ven iz smeri osi resonatorja. Če bi 
bila os interferometra vzporedna z osjo resonatorja, bi se pojavile dodatne resonance, 
kar bi močno motilo delovanje laserja.
\end{remark}

\section{Relaksacijske oscilacije}
Stacionarno delovanje laserjev smo že dodobra spoznali. Za obravnavo
nestacionarnega delovanja pa moramo reševati sistem diferencialnih enačb 
(\ref{5.7}) in (\ref{5.8}), kar gre v splošnem le numerično.

Za začetek si oglejmo, kako se obnaša laser v bližini stacionarnega stanja. 
Spet se omejimo na enofrekvenčni laser, ki ga opišemo z enačbama (\ref{5.7})
in (\ref{5.8}). Te zaradi preglednosti zapišimo nekoliko drugače. Najprej vpeljimo 
brezdimnezijski čas $t^{\prime}=t A$ in $\tau^{\prime}=\tau A$, kar pomeni, da merimo 
čas v enotah življenskega časa laserskega nivoja. Uporabimo spet parameter
$p=VA/(B\hbar\omega g)$ (enačba~\ref{5.10}), ki pomeni število stanj 
elektromagnetnega polja v volumnu $V$ in znotraj spektralne širine laserskega nivoja. 
Enačbi potem zapišemo 
\begin{eqnarray}  
\frac{d N_2}{d t^{\prime}}&=&-\frac{nN_2}{p}-N_2+N_{20} \label{5.23a} \\
\frac{d n}{d t^{\prime}}& = & \frac{nN_2}{p}-\frac{2}{\tau^{\prime}}n.
\label{5.23}
\end{eqnarray}
Pri tem smo vpeljali konstanto $N_{20}= rN/A$, ki ima tudi nazornen pomen.
Predstavlja zasedenost, ki bi jo dobili pri danem stacionarnem črpanju, če v
izbranem stanju ne bi bilo fotonov in s tem ne stimuliranega sevanja. Meri torej 
moč črpanja. V enačbi za hitrost spreminjanja števila fotonov
smo zanemarili prispevek spontanega sevanja, za katerega smo že ugotovili,
da se pozna le do praga.

Pri reševanju nelinearnih diferencialnih enačb pogosto poiščemo približke rešitev
z linearizacijo. Naj laser najprej deluje stacionarno, v nekem trenutku pa se 
nekoliko izmakne iz stacionarnega stanja. To se lahko zgodi, na primer, če
spremenimo moč črpanja. Trenutno zasedenost $N_2$ in število fotonov $n$
lahko zapišemo v obliki 
\begin{equation}  
N_2= N_{2s}+x \qquad {\mathrm in} \qquad n=n_s+y,
\label{5.24}
\end{equation}
kjer sta $N_{2s}$ in $n_s$ vrednosti v stacionarnem stanju. Zanju velja 
\begin{equation}  
N_{2s}=\frac{2p}{\tau^{\prime}}
\label{5.25}
\end{equation}
in 
\begin{equation} 
n_s=p\frac{N_{20}-N_{2s}}{N_{2s}}=p(a-1).
\label{5.26}
\end{equation}
Enačba (\ref{5.25}) je v skladu s tem, da je stacionarna zasedenost 
enaka zasedenosti na pragu, ta pa je odvisna od izgub resonatorja. 
Razmerje $a=N_{20}/N_{2s}$ je mera za moč črpanja in
mora biti v delujočem laserju večje od 1. V večini praktičnih primerov
doseže $a$ vrednosti do 3 ali 5.

Vstavimo nastavka (enačbi~\ref{5.24}) v enačbi (\ref{5.23a} in \label{5.23}). 
Dobimo sistem enačb
\begin{eqnarray}  
\frac{d x}{d t^{\prime}} &=&-\frac{n_sN_{2s}}{p}-N_{2s}+N_{20}- \frac{1}{p}
(n_sx+N_{2s}y+xy)-x \\
\frac{d y}{d t^{\prime}} &=& \frac{n_sN_{2s}}{p}-\frac{2}{\tau^{\prime}}n_s
+ \frac{1}{p}(n_s x+N_{2s} y+xy)-\frac{2}{\tau^{\prime}}y
\label{5.27}
\end{eqnarray}

Ker sta $x$ in $y$ majhna v primeri s stacionarnimi vrednostmi, lahko
zanemarimo mešan produkt $xy$. Vsi členi, ki vsebujejo le stacionarne vrednosti,
dajo ravno 0, saj smo jih tako določili. Če upoštevamo še izraza 
za stacionarni vrednosti (enačbi~\ref {5.25} in \ref{5.26}), 
zapišemo linearizirani diferencialni enačbi za odmika od stacionarnih vrednosti 
\begin{eqnarray}
\frac{dx}{dt^{\prime }} &=&-a\,x-\frac{2}{\tau ^{\prime }}\,y  \label{5.28}
\\
\frac{dy}{dt^{\prime }} &=&(a-1)\,x.
\end{eqnarray}
Linearne sisteme diferencialnih enačb s konstantnimi
koeficienti rešujemo tako, da poiščemo rešitev v obliki eksponentne funkcije 
\begin{equation}
x=x_{0}e^{\lambda t^{\prime }} \qquad {\mathrm in } \qquad 
y=y_{0}e^{\lambda t^{\prime }}.
\label{5.29}
\end{equation}
Dobljeni homogeni sistem linearnih enačb 
\begin{eqnarray}
(a+\lambda )x_{0}+\frac{2}{\tau ^{\prime }}y_{0} &=&0  \label{5.30} \\
-(a-1)x_{0}+\lambda y_{0} &=&0
\end{eqnarray}
ima netrivialno rešitev le, če je njegova determinanta enaka nič
\begin{equation}
\lambda ^{2}+a\lambda +\frac{2}{\tau ^{\prime }}(a-1)=0.  
\label{5.301}
\end{equation}
Rešitvi sta 
\begin{equation}
\lambda =-\frac{a}{2}\pm \sqrt{\frac{a^{2}}{4}-\frac{2}{\tau ^{\prime }}(a-1)}.
\label{5.31}
\end{equation}
Obnašanje rešitve je odvisno od velikosti brezdimenzijskega relaksacijskega
časa nihanja resonatorja $\tau ^{\prime }=A\tau $. Kratek račun pokaže, da je 
za $\tau ^{\prime }>2$ izraz pod korenom pozitiven pri vseh $a$ in laser 
se vrača v stacionarno stanje eksponentno. Za $\tau' <2$ pa je koren v nekem območju
parametra $a$ imaginaren in laser se vrača v stacionarno stanje z
dušenim nihanjem, ki mu pravimo relaksacijske oscilacije.

\begin{figure}[h]
\centering
\def\svgwidth{90truemm} 
\input{slike/06_relax.pdf_tex}
\caption{Relaksacijske oscilacije intenzitete laserja po vklopu}
\label{fig:relax}
\end{figure}

V navadnih resonatorjih je $\tau \sim 10^{-7}~\si{s}$. Razpadna
konstanta laserskega nivoja $A$ je navadno dokaj majhna, ker je le v takih
primerih lahko doseči obrnjeno zasedenost. Vzemimo $A \sim 10^5/\si{s}$,
lahko pa je ta vrednost še dosti manjša. Tedaj je $\tau^{\prime}\sim 10^{-2}$ 
in relaksacijske oscilacije se pojavijo pri vseh dosegljivih vrednostih črpanja 
nad pragom, to je za $a>1$. Ker $a$ v praksi ni nikoli dosti večji od 3, 
je frekvenca oscilacij v brezdimenzijskih enotah v grobem
enaka $\omega^{\prime}_r\sim 1/\sqrt{\tau^{\prime}}$. Ko preidemo
nazaj na prave enote časa, dobimo $\omega_r\sim \sqrt{A/\tau}$. Frekvenca 
relaksacijskih oscilacij je v tem primeru velikostnega reda geometrijske 
sredine med razpadnima konstantama nihanja resonatorja in atomskega stanja. 
Primer takega nihanja pri vključitvi laserja kaže slika~(\ref{fig:relax}).

\begin{remark}
Relaksacijske oscilacije so praktično pomembne, saj določajo gornjo mejo
hitrosti, s katero lahko izhodna moč laserja sledi modulaciji črpanja.
Poleg tega se pri tej frekvenci pojavi resonanca, pri kateri se šum črpanja
ojačano prenaša v šum izhodne moči. 
\end{remark}

\section{Sunkovni laserji}
Kadar potrebujemo veliko izhodno moč laserja pri zmerni povprečni porabi 
energije, zvezno delujoči laserji, ki smo jih obravnavali do zdaj, 
niso primerni. Poslužiti se moramo sunkovnih laserjev, 
ki v kratkem časovnem intervalu 
delujejo z zelo veliko izhodno močjo. 

Poglejmo primer. Zvezno delujoč laser z izhodno močjo $10~\si{kW}$ bi pri
$0,1~\%$ izkoristku poteboval črpalno moč $10~\si{MW}$. Po drugi strani 
sunkovni laser, ki seva svetlobo v $10~\si{ns}$ dolgih sunkih s 
povprečno energijo $1~\si{J}$, deluje z močjo $100~\si{MW}$. Pri 
ponovitvni sunkov $1000/\si{s}$ je tako potrebna črpalna moč le $1~\si{kW}$.
Poznamo tri načine sunkovnega delovanja laserjev. V prvem primeru 
periodično spreminjamo črpalno moč, v drugih dveh (preklop dobrote in 
uklepanje faz) pa periodično spreminjamo izgube v resonatorju. 

Poglejmo najprej najpreprostejši način, pri katerem moduliramo črpanje.
To dosežemo na primer z bliskavko ali drugim sunkovnim laserjem.
Tipično je frekvenca modulacije črpanja $\nu \sim 20~\si{Hz}$.
V takem laserju je črpanje periodično (slika~\ref{fig:pulseG}). Vendar
se pri tem lahko pojavi težava. Ko ob močnem črpanju obrnjena zasedenost znatno
preseže zasedenost praga (v nestacionarnem stanju je to mogoče), laser
posveti in v kratkem času zasedenost pade nazaj pod prag. Če tedaj
črpanje še traja, čez čas zasedenost dovolj naraste in laser
ponovno posveti. To se lahko večkrat ponovi. Razmiki med zaporednimi sunki
so reda velikosti periode relaksacijskih oscilacij, so pa lahko precej
nepravilni. Pri takem režimu delovanja v posameznem sunku ne dobimo
razpoložljive energije črpanja v enem samem lepo oblikovanem sunku, kar
je za vrsto uporab zelo pomembno. 

\begin{figure}[h]
\centering
\def\svgwidth{90truemm} 
\input{slike/06_pulseG.pdf_tex}
\caption{Delovanje sunkovnega laserja z moduliranim črpanjem}
\label{fig:pulseG}
\end{figure}

\section{Delovanje v sunkih s preklopom dobrote}
Namesto modulacije črpanja lahko v sunkovnih laserjih periodično spreminjamo 
izgube. Čim večje so namreč izgube, višji je prag za delovanje laserja in 
večjo stopnjo inverzije lahko dosežemo. Tako v sistemu atomov shranimo več 
energije, ki se lahko izseva iz sistema preko stimuliranega sevanja. 
Ko je enkrat ustvarjena velika obrnjena zasedenost, izgube zelo hitro zmanjšamo. 
Optično ojačenje je veliko in energija svetlobe v kratkem času močno naraste. 
S tem se tudi obrnjena zasedenost hitro zniža na vrednost močno pod pragom.
Predstavljamo si lahko, da dobimo prvi nihaj relaksacijskih oscilacij, le da
je začetno stanje daleč od stacionarnega in zato linearni približek ne drži več.
Iz laserja dobimo kratek in zelo močan sunek svetlobe, pri čemer je energija
sunka skoraj tolikšna kot je bila energija obrnjene zasedenosti. Tipične
dolžine sunkov, ki jih dosežemo na ta način, so $t \sim 10~\si{ns}$, sunki
pa se ponavljajo s frekvenco $\nu \sim 1$--$100~\si{kHz}$.
Dogajanje v laserju kaže slika~(\ref{fig:pulseQ}).

\begin{remark}
V elektrotehniki se izgube resonatorjev podajajo z dobroto $Q$, to je razmerjem
frekvence lastnega stanja in njegove širine. Ker s povečanjem izgub spremenimo 
širino črte in z njo dobroto, opisano tehniko imenujemo preklop dobrote. 
\end{remark}

\begin{figure}[h]
\centering
\def\svgwidth{90truemm} 
\input{slike/06_pulseQ.pdf_tex}
\caption{Izgube ($\Lambda$), relativno ojačenje ($G/G_{\rm pr}$), zasedenost višjega 
nivoja ($N_2$) in izsevana moč laserja ($P$) v odvisnosti od časa, kadar laser 
deluje v režimu preklopa dobrote.}
\label{fig:pulseQ}
\end{figure}

Izgube resonatorja je mogoče spreminjati na več načinov. Najpreprosteje
je vrteti eno od ogledal. Tedaj je resonator uglašen le v kratkem
trenutku, ko je ogledalo pravokotno na os. Metoda je dokaj uspešna, a
zastarela. Boljši in danes najbolj razširjen način je z vgradnjo
elektro-optičnega ali akusto-optičnega modulatorja, o katerih
bomo govorili v nadaljevanju (poglavje~\ref{chap:modulacija}). Na kratko povejmo, 
da lahko z njimi električno krmilimo izgube z visoko frekvenco.

Kot smo že povedali, nelinearnih laserskih enačb za zasedenost in število fotonov
(enačbi~\ref{5.7}) in (\ref{5.8}) ne moremo analitično rešiti. Preden jih podrobneje
pogledamo, napravimo nekaj ocen. Dolžina sunka je odvisna od hitrosti, 
s katero se izprazni zgornji laserski nivo. To se ne more zgoditi
hitreje kot v nekaj preletih sunka skozi resonator. Trajanje sunka je torej
vsaj nekajkrat $2L/c$, to je za $15~\si{cm}$ dolg resonator vsaj nekaj $\si{ns}$.

Ocenimo še hitrost naraščanja števila fotonov na začetku in 
njegovega upadanja na koncu sunka. Zapišimo najprej še enkrat enačbi za zasedenost in število
fotonov, pri čemer upoštevajmo, da nas zanima le dogajanje v času sunka,
ki je zeko kratek v primerjavi z atomskim razpadnim časom, zato 
ustrezni člen v enačbi (\ref{5.7}) zanemarimo. Navadno je tudi črpanje prešibko, da
bi med sunkom znatno vplivalo na zasedenost, zato lahko člen $rN$
izpustimo. S črpanjem seveda ustvarimo začetno zasedenost $N_{20}$. Tako
ostane 
\begin{eqnarray}  
\frac{d N_2}{d t}&=&-\frac{\sigma c}{V}\,n\,N_2 \label{5.32a}\\
\frac{d n}{d t}&=&\frac{\sigma c}{V}\,n\,N_2 - \frac{2}{\tau}\,n.
\label{5.32}
\end{eqnarray}
Na začetku sunka je $n$ majhen, $N_2$ pa velik in njegova vrednost
se ne razlikuje dosti od začetne vrednosti $N_{20}$. Število fotonov
na začetku sunka tako narašča približno eksponentno
\begin{equation}  
n(t)=n_0e^{t\,N_{20}\,\sigma c/V}= n_0e^{t/\tau_r}.
\label{5.33}
\end{equation}
Začetnega števila fotonov ne poznamo, vemo pa, da je velikostnega reda 1,
saj predstavlja spontano emisijo. Da $n$ naraste na znatno
vrednost, recimo več od $10^{10}$ fotonov, je potreben čas blizu $30~\tau_r$.

Proti koncu sunka $N_2$ pojema zaradi sevanja svetlobe in $N_2 \to 0$. Ostane
samo še en člen, ki da preprosto rešitev
\begin{equation}  
n(t)=\tilde{n}_0e^{-2t/\tau}.
\label{5.33a}
\end{equation}
Eksponentno pojemanje števila fotonov na koncu sunka je torej določeno z izgubami
resonatorja (enačba~\ref{taulambda}). 

Dogajanja v vmesnih časih ne moremo enostavno popisati, lahko pa najdemo
medsebojno zvezo med $n$ in $N_2$, če iz enačb eliminiramo čas. 
Izrazimo $dt$ iz enačbe (\ref{5.32a}) in ga vstavimo v enačbo~(\ref{5.32}).
Sledi
\begin{equation}
dn=-dN_{2}+\frac{\tilde{N}_2}{N_{2}}dN_{2}\;\;,  \label{5.341}
\end{equation}
kjer smo zapisati $\tilde{N}_{2}=2V/(\sigma c\,\tau)$.
Enačbo brez težav integriramo
\boxeq{5.351}{
n=N_{20}-N_{2}+\tilde{N}_{2}\ln \frac{N_{2}}{N_{20}}.
}
Pri tem smo privzeli, da je na začetku sunka $n=0$ in $N_{2}=N_{20}$. 

Iz dobljene zveze najprej izračunamo, kolikšna je končna zasedenost $N_{2k}$. Na
koncu mora biti zopet $n=0$, kar da transcendentno enačbo za $N_{2k}$
\beq
\ln \frac{N_{2k}}{N_{20}} = \frac{N_{2k}}{N_{20}}- \frac{N_{20}}{\tilde{N}_2}.
\eeq
Enačba ima obliko
\beq
\ln \frac{x}{a}= x-a,
\eeq
kjer je $x=N_{2k}/N_{20}$ in $a=N_{20}/\tilde{N}_{2}$, in jo lahko preprosto
numerično rešimo. Izkaže se, da kadar je začetna zasedenost $N_{20}$
le malo nad pragom, tudi končna zasedenost $N_{2k}$ ne pade dosti pod prag, 
zato je izraba energije slabša. Pri večjih začetnih vrednostih $N_{20}$ pa 
pade končna zasedenost praktično na nič. Za $a=2$, na primer,
je $x=0,41$, medtem ko je že pri $a=4$ vrednost $x$ le še 0,08. 

Ko poznamo začetno in končo vrednost zasedenosti, lahko izračunamo 
celotno energijo sunka $W=\hbar \omega (N_{20}-N_{2k})$. Če je začetna vrednost
dovolj nad pragom, lahko končno zasedenost zanemarimo in je 
\boxeq{QW}{
W\approx\hbar \omega\,N_{20}.
}

Trenutna moč, ki izhaja iz laserja, je dana s $P=(\hbar \omega 2/\tau )n$.
Največja je v vrhu sunka, ki je določen z $dn/dN_{2}=0$. Ta enačba ima
rešitev pri $N_{2}=\tilde{N}_{2}$, vrh sunka je torej natanko tedaj, ko
pade zasedenost $N_2$ na prag $\tilde{N}_2$. 

Izsevana moč je tedaj 
\beq
P_{\rm max}=\frac {n_{\rm max} \hbar \omega}{2L/c}\left(1-\cal{R}\right) = 
\frac {2 n_{\rm max} \hbar \omega}{\tau}.
\eeq
Ko vstavimo še vrednost za $n_{\rm max}$, dobimo
\beq
P_{\rm max}=\frac {2\hbar \omega}{\tau} \left(N_{20}-\tilde{N}_{2}-\tilde{N}_{2}
\ln (N_{20}/\tilde{N}_{2})\right).
\eeq
Ker je navadno $N_{20}\gg \tilde{N}_2$, je $n_{\rm max} \approx N_{20}$
in 
\boxeq{QP}{
P_{\rm max} \approx \frac{2 \hbar \omega N_{20}}{\tau}.
}

Vzemimo za primer neodimov laser. Presek za stimulirano sevanje 
$\sigma=B\hbar \omega g/c$ je okoli $10^{-19}\si{cm}^{2}$. 
Naj bo začetna gostota zasedenosti $N_{20}/V=10^{19}\si{cm}^3$. 
Tedaj je $\tau _{r}=3$ ns. 
PREVERI!!!!

"Stevilo fotonov bo tedaj upadalo s
krakterističnim razpadnim časom resonatorja $\tau /2\simeq 2L/c(1-{\cal R%
})^{-1}$. V laserjih s preklopom dobrote je odbojnost izhodnega zrcala
navadno dokaj nizka, recimo 0,5. Pri $L=15\;\mbox{cm}$ je tako $\tau =4$ ns.
Celotno trajanje sunka je v izbranem primeru tako približno 10 ns, pri
čemer traja okoli 100 ns od preklopa dobrote, da sunek zraste iz šuma
spontanega sevanja. Energija sunka je blizu $N_{20}\hbar \omega $, to je pri
aktivnem volumnu 0,5 cm$^{3}$ nekaj desetink joula. Od tod lahko ocenimo
še, da je moč v vrhu sunka velikostnega reda 10 MW.

Naloga: Oceni, kolikšna je dosegljiva zasedenost pri dani dolžini Nd:YAG
paličke.

\section{Uklepanje faz}
\label{chap:Uklepanje}
"Se dosti krajše sunke kot s preklopom dobrote je mogoče dobiti na
povsem drug način, ki je prav presentljiva manifestacija koherentnosti
laserske svetlobe. Naj v laserju niha več nihanj hkrati. Njihove frekvence
so enakomerno razmaknjene za $\Delta \omega =\pi c/L$. Celotno električno
polje v neki točki v laserju je 
\begin{equation}
E(t)=\sum_{m=-N/2}^{N/2}A_{m}e^{i[(\omega _{0}+m\Delta \omega )t+\varphi
_{m}(t)]}\;\;.  \label{5.342}
\end{equation}
$N$ je število vseh vzbujenih nihanj. Upoštevali smo, da ima vsako
nihanje lahko poljubno fazo $\varphi _{m}(t)$, ki je v splošnem predvsem
zaradi zunanjih motenj slučajna funkcija časa. Zaradi tega se tudi
celotno polje slučajno spreminja, kar močno zmanjšuje uporabnost
takega laserja tam, kjer je potrebna časovna koherenca.

\begin{figure}[tbp]
\label{s5.10} \vskip 5cm
\caption{"Casovna odvisnost moči mnogofrekvenčnega laserja z enakimi
fazami }
\end{figure}

Denimo, da so faze vseh nihanj enake. Poleg tega zaradi enostavnosti
računa privzemimo še, da so tudi vse amplitude $A_{m}$ enake. Tedaj
postane vsota \ref{5.342} geometrijska in jo lahko brez težav seštejemo: 
\begin{equation}
E(t)=A_{0}e^{i\omega _{0}t}\frac{\sin (N\Delta \omega t/2)}{sin(\Delta
\omega t/2)}\;\;.  \label{5.352}
\end{equation}
Trenutna moč izhodne svetlobe ima časovno odvisnost 
\begin{equation}
P(t)=P_{0}\frac{\sin ^{2}(N\Delta \omega t/2)}{\sin ^{2}(\Delta \omega t/2)}
\label{5.36}
\end{equation}
in jo kaže slika \ref{5.10}. Predstavlja periodično zaporedje sunkov, ki
si slede s periodo $T=2\pi /\Delta \omega =2L/c$, kar je enako času obhoda
svetlobe v resonatorju. Konstanta $P_{0}$ je moč posameznega nihanja.
Moč v vrhu sunka je tako $N^{2}P_{0}$, povprečna moč pa $NP_{0}$.
Računsko je pojav enak kot uklon na mrežici in lahko rečemo, da imamo
opravka z interferenco v času. Dolžina sunkov je 
\begin{equation}
\tau _{ML}=\frac{T}{N}=\frac{2\pi }{N\Delta \omega }=\frac{2\pi }{\Delta
\omega _{G}}\;\;,  \label{5.37}
\end{equation}
ker je $N$ ravno število nihanj znotraj širine ojačevanja $\Delta
\omega _{G}$. Dolžina sunka je torej obratno sorazmerna s širino
ojačevanja aktivnega sredstva.

Premislimo še, kakšna je prostorska odvisnost električnega polja v
resonatorju. Polje na danem mestu opisuje enačba \ref{5.352}. "Se
krajevno odvisnost dobimo, če v \ref{5.352} zamenjamo $t$ s $(t-z/c)$. To
pa predstavlja svetlobni paket, ki potuje sem in tja med ogledali
resonatorja. Na izhodnem ogledalu se ga vsakič nekaj odbije, nekaj pa gre
ven iz resonatorja (slika \ref{5.11}). Razmik med sunki, ki izhajajo iz
resonatorja, je $2L$, prostorska dolžina posameznega sunka pa $\tau
_{ML}c=2L/N$.

\begin{figure}[tbp]
\label{5.11} \vskip 5cm
\caption{Prostorska odvisnost fazno uklenjenih sunkov.}
\end{figure}

V našem računu predpostavka, da so vse amplitude $A_{m}$ enake, ni prav
nič bistvena za osnovne ugotovitve. "Ce vzamemo realističen primer, da
so amplitude oblike $A_{m}=A_{0}\exp [(m\Delta \omega /\Delta \omega
_{G})^{2}]$, vsote \ref{5.342} ne znamo točno sešteti, lahko pa jo
približno pretvorimo v integral, ki je Fourierova transformacija Gaussove
funkcije (Pri prehodu z diskretne vsote na integral seveda izgubimo
periodičnost zaporedja sunkov.). Ta je zopet Gaussova funkcija, katere
širina je obratna vrednost širine prvotne funkcije, prav podobno, kot
smo dobili zgoraj. Odvisnost amplitud nihanj od $m$ vpliva torej le na
točno obliko sunkov, osnovne ugotovitve pa se ne spremene. (Naloga).

Pač pa je predpostavka, da so vse faze $\varphi_m$ enake, bistvena. V naši
dosedanji sliki mnogofrekvenčnih laserjev so resonatorska stanja med seboj
neodvisna, zato so faze poljubne in se zaradi motenj lahko še spreminjajo.
Da dobijo za vsa nihanja isto vrednost, moramo poskrbeti posebej. Tako {\it %
uklepanje faz} je mogoče doseči na več načinov. Ena možnost je, da
moduliramo izgube resonatorja s frekvenco, ki je ravno enaka razliki
frekvenc med resonatorskimi stanji. To ni težko razložiti. Naj bo
modulator tak, da je večino časa zaprt, le v razmikih $T=2L/c$ naj bo
kratek čas odprt. Postavimo ga tik ob eno ogledalo. Tedaj se v resonatorju
očitno lahko uspešno ojačuje le kratek sunek, kakršen je na sliki \ref
{***}. Izgube za vsa nihanja bodo majhne le tedaj, kadar bodo vse faze
enake. V praksi ni potrebno, da je modulacija tako izrazita. Običajno
zadošča sinusna modulacija izgub, kjer je relativna prepustnost v minimumu
za nekaj destink manjša od maksimalne.

Kako pri modulaciji pride do uklepanja faz, lahko uvidimo še drugače.
Modulacija amplitude posameznega nihanja povzroči, da se v spektru nihanja
pojavita še stranska pasova pri frekvencah $\omega_m \pm \Delta\omega$. Ta
se ravno pokrivata z obema sosednjima nihanjema in se konstruktivno
prištejeta, če imata enako fazo. S tem pa so tudi izgube manjše in ima
delovanje laserja z uklenjenimi fazami najnižji prag. Zadnji razmislek nam
tudi pove, da ni dobra le amplitudna modulacija, temveč tudi fazna (ali
frekvenčna), saj se tudi tedaj pojavijo stranski pasovi.

Za modulacijo se najpogosteje uporabljajo akustooptični modulatorji, pri
katerih izkoriščamo uklon svetlobe na stoječih zvočnih valovih v
primernem kristalu (Glej 7. poglavje). Frekvenca zvočnega vala mora biti
enaka polovici zahtevane modulacijske frekvence, za 1,5~m dolg laser torej
50~MHz.

Poleg opisanega aktivnega postopka je mogoče faze ukleniti tudi tako, da v
resonator postavimo plast barvila, ki močno absorbira svetlobo laserja pri
majnhi gostoti toka, pri veliki gostoti toka pa pride do nasičenja
absorpcije (glej razdelek 4.5), zato postane barvilo prozorno. Na začetku
imamo v laserju predvsem spontano sevanje, ki se pri enem prehodu skozi
aktivno snov deloma ojači. Barvilo najmanj absorbira največjo fluktuacijo.
Pri dovolj velikem ojačenju bo ta rastla in spet dobimo fazno uklenjeni
sunek. Ker mora po prehodu sunka absorpcija v barvilu zopet hitro narasti,
mora biti relaksacijski čas barvila zelo kratek, v območju pikosekund.

Z uklepanjem faz je danes mogoče dobiti sunke z dolžino pod 100~fs ($%
10^{-13}$~s). Tak sunek traja le še nekaj deset optičnih period. S
posebnimi prijemi jih lahko še skrajšajo na okoli 10~fs. "Ce je potrebna
večja energija sunkov, jih ojačijo, kar ne pokvari mnogo osnovnega sunka.
Zelo kratke svetlobne sunke danes na široko uporabljajo za študij hitre
molekularne dinamike in kratkoživih vzbujenih elektronskih stanj v
polvodnikih in mnogih drugih snoveh. Z njimi se je časovna ločljivost
povečala za nekaj redov velikosti \cite{pikosekunde}.

\section{*Stabilizacija frekvence laserja na na\-sičeno absorpcijo}
\label{chap:stabilizacija}

"Se ožjo lasersko črto lahko dobimo z aktivno stabilizacijo dolžine
resonatorja. Ideja je taka: frekvenco svetlobe, ki izhaja iz laserja,
primerjamo z nekim standardom in iz razlike ugotovimo spremembo dolžine
resonatorja. Eno od obeh zrcal je nameščeno na piezoelektričnem nosilcu,
ki mu z električno napetostjo lahko spreminjamo dolžino in tako popravimo
dolžino resonatorja.

Pogalvitna težava je seveda, kako najti dovolj stabilen primerjalni
standard za frekvenco. Ena možnost je, da izhodno svetlobo spustimo skozi
konfokalni interferometer, ki je skoraj v resonanci z laserjem in ima dovolj
ozek vrh prepustnosti. Majhen premik frekvence laserja bo povzročil, da se
bo spremenil skozi interferometer prepuščeni svetlobni tok. Na prvi pogled
je videti, da s tem nismo nič pridobili, saj bo resonančna frekvenca
interferometra stabilna tudi le toliko, kot je stabilna njegova dolžina.
Vendar je z izolacijo in temperaturno stabilizacijo možno držati dolžino
praznega resonatorja - interferometra - mnogo natančneje kot dolžino
laserja, v katerem imamo aktivno sredstvo, ki mu moramo dovajati energijo.

Druga možnost je stabilizirati laser na primerno molekularno absorpcijsko
črto. Te so lahko zelo ozke, zato je tudi spekter laserja lahko izredno
ozek, pod 1 kHz. Pri tem moti Dopplerjeva raziširitev absorpcijske črte,
ki pa se ji je mogoče izogniti. Kako to napravimo in kako je bila s tem
omogočena nova definicija metra, si bomo pogledali v razdelku ????.


V drugem razdelku smo videli, da je efektivna spektralna širina
eno\-frekvenčnega laserja odvisna od fluktuacij dolžine optične poti
svetlobe pri preletu resonatorja. Na to lahko poleg spreminjanja
geometrijske dolžine vpliva še spreminjanje lomnega količnika. "Ce se
posebej ne potrudimo, laser sveti nekje blizu vrha ojačevalnega pasu, pri
čemer frekvenca pleše za znaten del razmika med resonatorskimi stanji. V
šolskem He-Ne laserju je to na primer nekaj deset MHz.

Bistveno manjšo širino lahko dosežemo z aktivno stabilizacijo dolžine
resonatorja. Pri tem je pogalvitni problem, kako dobiti primerjalni
standard. S stabilizacijo na pomožni interferometer, ki smo jo na kratko
opisali v drugem razdelku, lahko dobimo zelo ozko črto, ki pa ima le toliko
natančno določeno frekvenco, kot poznamo dolžino interferometra. Včasih,
na primer za natančna interferometrična merjenja dolžin, s tem nismo
zadovoljni in potrebujemo drug, absoluten standard.

Tak standard za frekvenco so ozki prehodi v primernem razredčenem plinu.
Vendar naletimo na težavo. Zaradi Dopplerjevega pojava so absorpcijske
črte močno razširjene. Pomaga nam pojav nasičenja absorpcije, o katerem
smo govorili v razdelku 4.9. Tam smo videli, da se pri dvakratnem prehodu
monokromatskega snopa svetlobe skozi plin v nasprotnih smereh pojavi v
sredini Dopplerjevo razširjene črte vdolbina, ki ima obliko homogeno
razširjene črte. Homogena širina je lahko mnogo manjša od Dopplerjeve in
je zato vdolbina uporabna kot frekvenčni standard.

V laserski resonator postavimo poleg aktivnega sredstva še celico s
primernim plinom, ki ima absorpcijsko črto v bližini vrha ojačenja
aktivnega sredstva. Za He-Ne laser pri 633 nm so to na primer pare ioda.
Zaradi absorpcije se povečajo izgube v laserju in izhodna moč se zmanjša.
Spreminjajmo sedaj dolžino resonatorja in s tem frekvenco laserja. Ko se ta
približa na homogeno širino centru absorpcijske črte pri $\omega_0$, se
absorpcija zmanjša in s tem se moč laserja poveča. Odvisnost moči
laserja z absorberjem od dolžine kaže slika \ref{s5.12}. Povečanje moči
v vrhu običajno ni prav veliko, manj od procenta.

\begin{figure}[tbp]
\label{s5.12} \vskip 5cm
\caption{Odvisnost moči laserja z nasičenim absorberjem od frekvence}
\end{figure}

Shema stabilizacije na nasičeno absorpcijo je prikazana na sliki \ref{****}%
. Eno od zrcal resonatorja je na piezoelektričnem nosilcu. Nanj vodimo
izmenično napetost s frekvenco $\Omega$ in s tem moduliramo frekvenco
laserja, da se vozi preko absorpcijske vdolbine pri $\omega_0$. Zaradi tega
se spreminja tudi izhodna moč laserja, ki jo opazujemo s fotodiodo. Kadar
je srednja frekvenca laserja enaka $\omega_0$, se moč zmanjša simetrično
pri odmikih navzgor in navzdol od $\omega_0$ in se zato spreminja z dvojno
frekvenco modulacije $2\Omega$. Kadar pa je srednja frekvenca laserja
nekoliko odmaknjena od $\omega_0$, se izhodna moč pri odmiku v zrcala v eno
stran spremeni drugače kot v drugo, kar pomeni, da je v signalu s fotodiode
tudi komponenta s frekvenco $\Omega$. Da držimo srednjo frekvenco laserja
enako $\omega_0$, moramo torej meriti komponento izhodne moči pri
modulacijski frekvenci in s povratno zanko skrbeti, da je ta enaka nič.

\begin{figure}[tbp]
\label{s5.13} \vskip 7cm
\caption{Shema stabilizacije laserja na nasičeno absoprcijo}
\end{figure}

Komponento signala s frekvenco $\Omega$ zaznamo s faznim detektorjem, ki
deluje tako, da signal množi z refenčno modulacijsko napetostjo. V
produktu dobimo istosmerno komponento, ki je sorazmerna signalu pri
frekvenci $\Omega$ in ki jo izločimo z nizkopasovnim filtrom. Izhod iz
faznega detektorja je tako sorazmeren odmiku srednje frekvence laserja od $%
\omega_0$. Preko primernega ojačevalnika ga vodimo na piezoelektrični
nosilec zrcala in tako popravljamo dolžino laserja.

Napravimo kvantitativno oceno opisane stabilizacijske sheme. Odvisnost
izhodne moči od frekvence laserja $\omega$ lahko približno zapišemo v
obliki 
\begin{equation}  \label{5.40}
P(\omega)=P_0 + \frac{P_1\gamma^2}{(\omega- \omega_0)^2+\gamma^2}\;\;.
\end{equation}
$P_0$ je moč laserja brez saturacijskega vrha pri $\omega_0$, $P_1$ pa
povečanje moči pri $\omega_0$. Predpostavili smo, da se ojačenje laserja
in nehomogeno razširjeni del absorpcije ne spreminjata mnogo preko homogene
širine absorberja in je zato $P_0$ približno konstantna. Frekvenco laserja
moduliramo: 
\begin{equation}  \label{5.41}
\omega=\omega_0+\Delta\omega+a \sin \Omega t\;\;.
\end{equation}
Z $\Delta\omega$ smo označili odstopanje srednje frekvence laserja od
centra absorpcijske črte $\omega_0$. "Ce sta $a$ in $\Delta\omega$ majhna v
primeri s homogeno širino $\gamma$, lahko imenovalec v enačbi \ref{5.40}
razvijemo: 
\begin{equation}  \label{5.42}
P(\omega)=P_0+P_1 [1-\frac{1}{\gamma^2}\Delta\omega^2 +\frac{2}{\gamma^2} a
\Delta\omega \sin\Omega t - \frac{a^2}{\gamma^2}\sin^2\Omega t ]\;\;.
\end{equation}
Amplituda signala pri $\Omega$ je $2P_1 a \Delta\omega/\gamma^2$. Najmanjša
razlika, ki jo lahko zaznamo, je določena s šumom meritve. Kot bomo videli
v poglavju o detekciji svetlobe, je osnovni izvor šuma fotodiode Poissonov
šum števila parov elektron-vrzel, ki nastanejo zaradi fotoefekta v p-n
spoju. Najmanjša sprememba svetlobne moči, ki jo lahko izmerimo, je (glej
10??. poglavje) 
\begin{equation}  \label{5.43}
P_N\simeq \sqrt{\hbar\omega P \frac{1}{\tau}}\;,
\end{equation}
kjer je $P$ celotna svetlobna moč, ki vpada na diodo, $\tau$ pa čas
meritve, ki je v našem primeru določen s časovno konstanto nizkopasovnega
filtra na izhodu faznega detektorja.

Vzemimo na primer He-Ne laser, stabiliziran na iodove pare. Povprečna moč
laserja $P_0$ naj je 10 mW in $P_1=0.1 $ mW. "Sirina absorpcijske črte $%
\gamma= 10^6$ s$^{-1}$. Izberimo amplitudo modulacije $a=10^5$ s$^{-1}$ in $%
\tau=10^{-4}$ s. "Casovna konstanta $\tau$ ne sme biti prevelika, določa
namreč, kako hitro popravljamo dolžino laserja. Gornje vrednosti nam dajo
za najmanjšo zaznavno moč pri $\Omega$ $P_N=0.5\times10^{-8}$ W.
Najmanjše merljivo odstopanje frekvence laserja je tedaj 
\begin{equation}  \label{5.44}
\Delta\omega_N=\frac{P_N \gamma^2}{2P_1 a}=2.5\times 10^3\;s^{-1}\;.
\end{equation}
Takšno in še boljšo stabilnost frekvence tudi zares dosežejo. Pozoren
bralec bo opazil, da je $\Delta\omega_N<0.01 \gamma$, to je, položaj
absorpcijskega vrha je na opisan način mogoče določiti z natančnostjo
nekaj tisočink celotne širine.

Na absorpcijsko črto stabiliziranega laserja navadno ne uporabljamo
direktno, temveč z njim kontroliramo drug laser. Del izhodne svetlobe iz
obeh laserjev zmešamo na detekcijski fotodiodi. V signalu dobimo utripanje,
ki je enako razliki frekvenc obeh laserjev. S spreminjanjem dolžine drugega
laserja skrbimo, da je frekvenca utripanja konstantna. Na ta način lahko v
ozkem frekvenčnem intervalu še spreminjamo frekvenco drugega laserja.

Z merjenjem utripanja med dvema stabiliziranima laserjema ugotavljajo tudi
njihovo stabilnost.

\section{*Absolutna meritev frekvence laserja in definicija metra}

Najnatančneje merljiva količina je čas odnosno frekvenca. Frekvence
laserja, ki sveti v vidnem področju seveda ni mogoče direktno prešteti.
Pač pa je v začetku sedemdesetih let uspelo s heterodinsko tehniko, ki se
v mikrovalovni tehniki pogosto uporablja, napraviti primerjavo
stabiliziranega He-Ne laserja z osnovno cezijevo uro in tako določiti
frekvenco absorpcijske črte metana pri 3.39 $\mu$m z isto natan"nostjo, kot
jo ima cezijeva ura.

S heterodinsko tehniko primerjamo frekvenci dveh ali več valovanj tako, da
jih zmešamo na primernem nelinearnem elementu, običajno neki diodi. Zaradi
nelinearnosti dobimo v odzivu diode različne mnogokratnike vpadnih
frekvenc, njihove vsote in razlike. Od teh je kaktera lahko dovolj nizka, da
jo lahko direktno preštejemo.

Za primerjanje frekvenc nad mikrovalvnim področjem je potreben ustrezen
mešalni element. Polvodniške diode nehajo biti uporabne pri približno 20
GHz. Za višje frekvence uporabijo diode kovina-izolator-kovina, ki jih
sestavljajo oksidirana površina niklja, ki se je dotika ostra volframska
konica. Taka dioda deluje kot uporaben mešalni element do frekvenc okoli
200 THz, to je skoraj do vidnega področja.

\begin{figure}[tbp]
\label{s5.14} \vskip 15cm
\caption{Primerjalna veriga za meritev frekvence He-Ne laserja}
\end{figure}

Za primerjavo He-Ne laserja, stabiliziranega na metan pri 3.39 $\mu$m, z
osnovno cezijevo uro je bilo potrebno zgraditi celo verigo vmesnih
primerjav, ki jo kaže slika \ref{*****}. Frekvenco CO$_2$ laserja dobimo na
primer iz utripanja med frekvencama CO$_2$ laserja pri 10.2 $\mu$m in pri
9.3 $\mu$m, trikratnikom frekvence HCN laserja in še klistrona s frekvenco
20 GHz. Na ta način so izmerili, da je frekvenca CH$_4$ črte na katero je
stabiliziran He-Ne laser, 88.376181627 THz.

Valovno dolžino laserja dobimo z interferometrično primerjavo z
dolžinskim standardom, ki je bil do leta 1984 določen z neko kriptonovo
črto. Iz znane frekvence in valovna dolžine določimo hitrost svetlobe.
Zaradi relativno velike širine črte kriptonove svetilke je bil po starem
meter definiran le z relativno natančnostjo 10$^{- 8}$, kar je pomenilo, da
tudi hitrost svetlobe ne more biti določena bolj natančno. Meritev
frekvence laserja pa je dosti natančnejša. Zato je bilo smiselno opustiti
meter kot osnovno enoto in raje definirati hitrost svetlobe kot pretvornik
med sekundo in metrom. Z njeno vrednost so vzeli, kar so dobili z naboljšo
primerjavo stabiliziranega laserja in kriptonove črte: $c=299 792
458\;m/s\;. $ Na metan ali jod stabilizirani laser je postal sekundarni
standard za dolžino. Laser je pravzaprav pri tem le pomožna naprava;
standard je ustrezni molekularni prehod.

\section{*Semiklasični model laserja}
\label{chap:semiklasicni}
Doslej smo laserje obravnavali le z modelom zasedbenih enačb. Ta je zelo
grob, saj smo zanemarili nekaj pomembnih pojavov. Svetlobo v resonatorju smo
opisali smo opisali le s celotno energijo ali številom fotonov in se za
njeno valovno naravo nismo menili. Kar privzeli smo, da je frekvenca
delujočega laserja in oblika polja v njem enaka kot za lastno stanje
praznega resonatorja. Aktivno snov smo opisali le z zasedenostjo zgornjega
in spodnjega laserskega stanja in smo s tem izpustili možnost, da se zaradi
sodelovanja z elektromagnetnim poljem atomi nahajajo v nestacionarnem,
mešanem stanju.

Gornje pomanjkljivosti odpravimo s tem, da elektromagnetno polje v
resonatorju obravnavamo z valovno enačbo, za atome aktivne snovi pa
upoštevamo, da se pokoravajo Schroedingerjevi enačbi. S tem dobimo {\it %
semiklasični model} laserja. Za še natančnejši opis pa moramo tudi
svetlobo obravnavati kvantno, kar presega okvir te knjige.

Aktivna snov naj bo še naprej kar najenostavnejša, to je množica enakih
dvonivojskih atomov s stanji $|1\rangle$ in $|2\rangle$, ki imata energiji $%
W_1$ in $W_2$. Atomi s svetlobo sodelujejo preko dipolne interakcije oblike $%
e\hat{x}E(t)$, kjer je $E(t)$ polje v resonatorju, ki naj bo zaradi
preprostosti polarizirano v smeri osi $x$. "Casovno odvisno stanje atomov
zapišimo v obliki 
\begin{equation}  \label{5.45}
|\psi\rangle=c_1(t)|1\rangle\exp(-iW_1t/\hbar)+
c_2(t)|2\rangle\exp(-iW_2t/\hbar)\;.
\end{equation}
Iz Schroedingerjeve enačbe dobimo za koeficienta $c_1(t)$ in $c_2(t)$ 
\begin{equation}  \label{5.46}
\dot{c}_1=\frac{1}{i\hbar}\,E(t)v_{12}e^{-i\omega_0 t}c_2 \dot{c}_2=\frac{1}{%
i\hbar}\,E(t)v_{12}e^{i\omega_0 t}c_1\;,
\end{equation}
kjer je $\omega_0=(W_2-W_1)/\hbar$ in $v_{12}=e\langle 1|\hat{x}||2\rangle$.

Električni dipolni moment atoma v stanju $ket{\psi}$ je 
\begin{equation}  \label{5.47}
p=-e\langle\psi|\hat{x}|\psi\rangle=-(c_1^{\ast}c_2e^{-i \omega_0
t}+c_1c_2^{\ast}e^{i \omega_0 t}) v_{12}\;.
\end{equation}
Razdelimo $p$ na dva dela: 
\begin{equation}  \label{5.48}
p=p^++p^-=v_{12}[\eta(t)+\eta^{\ast}(t)]\;,
\end{equation}
kjer smo vpeljali $\eta(t)=c_1^{\ast}c_2e^{-i \omega_0 t}$.

Zanima nas, kako se dipolni moment spreminja s časom. Zato s pomočjo
enačb \ref{5.46} izrazimo 
\begin{equation}  \label{5.49}
\dot{\eta}=- i \omega_0\eta-\frac{1}{i\hbar}\,E(t)v_{12} (|c_2|^2-|c_1|^2)\;.
\end{equation}
$|c_i|^2$ je verjetnost za zasedenost stanja $|i\rangle$. Izraz v oklepaju
na desni strani gornje enačbe torej meri razliko zasednosti obeh stanj;
označimo ga z $\zeta$. Podobno kot zgoraj izrazimo časovni odvod 
\begin{equation}  \label{5.50}
\dot{\zeta}=\frac{2v_{12}}{i\hbar}\,E(t)(\eta^{\ast}- \eta)\;.
\end{equation}
S tem smo iz Schroedingerjeve enačbe dobili enačbe za časovni razvoj
diponega momenta in obrnjene zasedenosti, ki pa jih moramo še dopolniti.
Naj bo atom na začetku v stanju $|2\rangle$ in naj bo $E(t)=0$. Začetna
vrednost $\zeta(0)=1$ in po enačbi \ref{5.50} naj bi bila $\zeta(t)$
konstantna. Vemo pa, da se atom, ki je v vzbujenem stanju, sčasoma vrne v
osnovno stanje. Verjetnost za prehod na časovno enoto smo označili z $A$.
Poleg tega moramo na nek način upoštevati še črpanje, s katerim
vzdržujemo obrnjeno zasedenost in s tem lasersko delovanje. Za podroben
opis črpanja bi morali v Hamiltonov operator dodati ustrezne člene in
morda upoštevati še druga stanja atomov, vendar nas take podrobnosti na
tem mestu ne zanimajo. Zaradi črpanja stacionarna vrednost $\zeta$ v
odsotnosti laserskega polja $E(t)$ ni -1, temveč zavzame neko vrendost $%
\zeta_0$ med -1 in 1, odvisno od moči črpanja. Tako lahko enačbo \ref
{5.50} popravimo: 
\begin{equation}  \label{5.51}
\dot{\zeta}=A(\zeta_0-\zeta)+ \frac{2v_{12}}{i\hbar}\,E(t)(\eta^{\ast}-\eta)%
\;,
\end{equation}
kjer prvi člen popisuje spontane prehode v nižje stanje in vpliv črpanja.

Podobno dopolnimo še enačbo \ref{5.49}. Pri $E(t)=0$ da časovno odvisnost 
$\eta$ oblike $e^{-i \omega_0 t}$, to je brez dušenja. Vema pa, da
polarizacija v mešanem stanju razpada vsaj zaradi spontanega sevanja, lahko
pa še zaradi drugih vplivov, na primer trkov z drugimi atomi. Označimo
koeficient dušenja polarizacije z $\gamma$, ki meri tudi spektralno širino
svetlobe, ki jo sevajo atomi pri prehodu $2\rightarrow 1$. Tako imamo 
\begin{equation}  \label{5.52}
\dot{\eta}=- (i \omega_0\eta+\gamma)- \frac{1}{i\hbar}\,E(t)v_{12} \zeta \;.
\end{equation}
Tej enačbi moramo dodati še konjugirano kompleksno enačbo. Enačbe \ref
{5.51} in \ref{5.52} pogosto imenujejo Blochove enačbe. Najprej so jih
uporabili za obravnavo jedrske magnetne resonance.

Potrebujemo še enačbo za polje $E(t)$. Zanj dobimo iz Maxwellovih enačb
valovno enačbo, kjer moramo upoštevati, da imamo tudi od nič različno
polarizacijo snovi, ki je v primeru, da so vsi atomi enakovredni, podana z 
\begin{equation}  \label{5.53}
P=\frac{N}{V}\,v_{12}(\eta+\eta^{\ast})=P^++P^-\;.
\end{equation}
Valovna enačba je tedaj \cite{empolje} 
\begin{equation}  \label{5.54}
\nabla^2 E-\frac{1}{c^2}\ddot{E}=\mu_0 \ddot{P}\;.
\end{equation}

Namesto mikroskopske količine $\zeta$ lahko uvedemo še gostoto obrnjene
zasedenosti $Z=(N/V)\zeta$, pa lahko enačbi \ref{5.51} in \ref{5.52}
prepišemo v obliko 
\begin{eqnarray}  \label{5.56}
\dot{P}^{\pm}&=&(\mp i \omega_0-\gamma)P^{\pm}+ \frac{v_{12}^2}{i\hbar} E\,Z
\\
\dot{Z}&=&A(Z_0-Z)-\frac{2}{i\hbar}E(P^--P^+)\;.
\end{eqnarray}
Prehod od enačb \ref{5.51} in \ref{5.52} na \ref{5.55} je mogoč le, kadar
so vsi atomi enakovredni, to je, kadar ni nehomogene razširitve. Kako je v
primeru nehomogene razširitve, si bralec lahko ogleda v \cite{haken2}.

Enačbe \ref{5.51}, \ref{5.52} ali \ref{5.55}, skupaj z \ref{5.54} dajejo
semiklasični opis sodelovanja svetlobe in snovi. Iz izpeljave je vidno, da
je v njem spontano sevanje obravnavano pomankljivo, le s fenomenološkim
na\-stavkom, kar je moč popraviti tako, da tudi elektromagnetno polje
kvantiziramo. Kljub tej pomanjkljivosti je s semiklasičnim modelom mogoče
zelo podrobno obravnavati večino pojavov v laserjih in tudi druge probleme
širjenja svetlobe po snovi. Reševanje zapisanega sistema nelinearnih
parcialnih diferencialnih enačb pa je v splošnem zelo težavno.

Da bomo semiklasične enačbe le nekoliko pobliže spoznali, na kratko
poglejmo najenostavnejši primer, to je laser, v katerem je vzbujeno le eno
resonatorsko stanje. Polje ima tedaj obliko 
\begin{equation}  \label{5.57}
E(\vec{r},t)=E_{\lambda}(t)u_{\lambda}(\vec{r})\;,
\end{equation}
kjer je $u_{\lambda}(\vec{r})$ krajevni del lastnega stanja resonatorja, ki
zadošča enačbi 
\begin{equation}  \label{5.58}
\nabla^2 u_{\lambda}- \frac{\omega_{\lambda}^2}{c^2}u_{\lambda}=0\;.
\end{equation}
$E_{\lambda}(t)$ opisuje časovno odvisnost, ki je za laser v stacionarnem
delovanju periodična, vendar frekvenca ni nujno kar enaka lastni frekvneci
praznega resonatorja $\omega_{\lambda}$, temveč jo moramo še izračunati.

Tudi polarizacijo lahko razvijemo po lastnih funkcijah $u_{\lambda}(\vec{r})$%
. Ker so te med seboj ortogonalne, preide valovna enačba \ref{5.54} v 
\begin{equation}  \label{5.59}
\omega_{\lambda}^2 E_{\lambda}-\ddot{E_{\lambda}}= \frac{1}{\epsilon_0}\ddot{%
P}_{\lambda}\;.
\end{equation}

Razstavimo $E_{\lambda}(t)$ na dva dela: 
\begin{equation}  \label{5.60}
E_{\lambda}(t)=E_{\lambda}^+(t)+E_{\lambda}^-(t)=A^+(t)e^{-i
\omega_{\lambda}t}+A^-(t)e^{i \omega_{\lambda}t}\;.
\end{equation}
Dejanska frekvenca laserja je blizu $\omega_{\lambda}$, zato pričakujemo,
da bosta amplitudi $A^{\pm}(t)$ v primerjavi z $e^{-i \omega_{\lambda}t}$ le
počasni funkciji časa. Izračunajmo 
\begin{eqnarray}  \label{5.61}
\ddot{E}_{\lambda}^+&=&-\omega_{\lambda}^2 E_{\lambda}^+-2i \omega_{\lambda} 
\dot{A}^+ e^{-i \omega_{\lambda}t} + \ddot{A}^+ e^{-i \omega_{\lambda}t} 
\nonumber \\
&\simeq&-\omega_{\lambda}^2 E_{\lambda}^+-2i \omega_{\lambda}(\dot{E}%
_{\lambda}^++i \omega_{\lambda}E_{\lambda}^+)
\end{eqnarray}
V drugi vrstici smo izpustili člen z $\ddot{A}^+$, ker pričakujemo, da je
majhen. S tem smo napravili približek {\it počasne amplitude}.

Polarizacija snovi je približno periodična s frekvenco $\omega_0$, z
amplitudo, ki je tudi počasna funkcija časa. Zato je $\ddot{P}%
_{\lambda}^+\simeq- \omega_0^2 P_{\lambda}^+$. Pri drugem odvodu polja po
času smo potrebovali en člen več, ker se člen $-\omega_{\lambda}^2
E_{\lambda}^+$ na levi strani enačbe \ref{5.54} odšteje. Z uporabo tega
približka in enačb \ref{5.58} in \ref{5.61} preide valovna enačba \ref
{5.54} za eno nihanje v 
\begin{equation}  \label{5.62}
\dot{E}_{\lambda}^+=-i \omega_{\lambda} E_{\lambda}^++\frac{i \omega_0}{%
2\epsilon_0}P_{\lambda}^+\;.
\end{equation}

Doslej nismo upoštevali, da je polje v praznem resonatorju dušeno, zato
moramo gornjo enačbo še popraviti: 
\begin{equation}  \label{5.63}
\dot{E}_{\lambda}^+=(-i \omega_{\lambda}-\frac{1}{\tau}) E_{\lambda}^++\frac{%
i \omega_0}{2\epsilon_0}P_{\lambda}^+\;.
\end{equation}
Kadar v reosnatorju ni snovi, je dobljena enačba enaka kot enačba \ref
{3.45}.

Enačbi \ref{5.55} in \ref{5.56} sta nelinearni, zato ju n moč kar tako
prepisati za primer razvoja po lastnih stanjih resonatorja. Pri enačbi za
razvoj polarizacije \ref{5.55} imamo v zadnjem členu na desni produkt
komponente polja $E_{\lambda}$ in obrnjene zasedenosti $Z$, od katere
bistveno prispeva le krajevno povprečje $\bar{Z}$, ki se tudi s časom le
počasi spreminja. Seveda vsebuje $Z$ tudi krajevno odvisne komponente, ki
pa so pomembne predvsem zato, ker sklaplajo različna lastna stanja
resonatorja, kar presega našo trenutno obravnavo. Tako imamo 
\begin{equation}  \label{5.64}
\dot{P}_{\lambda}^+=(-i \omega_0- \gamma)P_{\lambda}^++\frac{v_{12}^2} {%
i\hbar}\,E_{\lambda}^+\bar{Z}\;.
\end{equation}

Enačbo za $\bar{Z}$ dobimo iz \ref{5.56}. V zadnjem členu imamo produkte $%
E^{\pm}P^{\pm}=E_{\lambda}^{\pm}P_{\lambda}^{\pm} u_{\lambda}^2(\vec{r})$,
kar moramo prostorsko povprečiti. Funkcije $u_{\lambda}(\vec{r})$ naj so
normalizirane tako, da je $\int u_{\lambda}^2(\vec{r}) \,dV=V$. Tako imamo $%
\overline{u_{\lambda}^2(\vec{r})}=1$ in 
\begin{equation}  \label{5.65}
\dot{\bar{Z}}= A(\bar{Z}_0-\bar{Z})- \frac{2}{i\hbar}(E_{\lambda}^++
E_{\lambda}^-)(P_{\lambda}^- - P_{\lambda}^+)\;,
\end{equation}
kjer je $\bar{Z}_0$ povprečje nenasičene zasedenosti $Z_0$. V zadnjem
členu nastopajo produkti, ki nihajo s frekvencami $\omega_{\lambda}-
\omega_0$ in $\omega_{\lambda}+ \omega_0$. Obe frekvenci sta si zelo blizu,
zato je njuna vsota mnogo večja od razlike. "Cleni $E_{\lambda}^+
P_{\lambda}^+$ in $E_{\lambda}^- P_{\lambda}^-$ se torej zelo hitro
spreminjajo in skoraj nič ne vplivajo na valovanje blizu $\omega_{\lambda}$%
, zato jih izpustimo. S tem je časovna odvisnost $\bar{Z}$ podana z 
\begin{equation}  \label{5.66}
\dot{\bar{Z}}= A(\bar{Z}_0-\bar{Z})- \frac{2}{i\hbar}(E_{\lambda}^+
P_{\lambda}^- - E_{\lambda}^- P_{\lambda}^+)\;.
\end{equation}
Enačbe \ref{5.63}, \ref{5.64} in \ref{5.66}, skupaj s konjugirano
kompleksnimi enačbami za $E_{\lambda}^-$ in $P_{\lambda}^-$, so zaključen
sistem, ki opisuje delovanje enofrekvenčnega laserja. Uporabimo jih za
izračun frekvence izhodne svetlobe.

Naj bo stanje stacionarno. Tedaj lahko polje zapišemo v obliki $E_{\lambda
}^{+}=E_{0}e^{-i\Omega t}$, kjer je $E_{0}$ realna konstanta, frekvenca
svetlobe $\omega $ pa je blizu $\omega _{0}$ in $\omega _{\lambda }$. V
stacionarnem stanju mora imeti polarizacija enako časovno odvisnost: $%
P_{\lambda }^{+}=P_{0}e^{-i\Omega t}$. Tedaj je v enačbi \ref{5.66} drugi
oklepaj konstanten in mora biti tudi $\bar{Z}$ v stacionarnem stanju od
časa neodvisna. Sistem enačb \ref{5.63}, \ref{5.64} in \ref{5.66} nam
tako da 
\begin{eqnarray}
-[i(\omega _{\lambda }-\Omega )+\frac{1}{\tau }]E_{0}-\frac{i\omega _{0}}{%
2\epsilon _{0}}\,P_{0} &=&0  \nonumber  \label{5.67} \\
-[i(\omega _{0}-\Omega )+\gamma ]P_{0}+\frac{v_{12}^{2}}{i\hbar }\,E_{0}\bar{%
Z} &=&0  \nonumber \\
A(\bar{Z}_{0}-\bar{Z})-\frac{2}{i\hbar }\,E_{0}(P_{0}^{*}-P_{0}) &=&0\;.
\end{eqnarray}

Najprej izračunamo $P_0$ iz druge enačbe, ga postavimo v tretjo in
izračunamo $\bar{Z}$: 
\begin{equation}  \label{5.68}
\bar{Z}=\bar{Z}_0\left[1+\frac{v_{12}^2}{\hbar^2 A}\,E_0^2\, \frac{2\gamma}{%
(\omega_0-\Omega)^2+\gamma^2}\right]^{-1}
\end{equation}
Ta izraz že poznamo. $\pi v_{12}^2/(\epsilon_0\hbar^2)$ je Einsteinov
koeficient $B$. $E_0^2$ je sorazmern gostoti energije polja v resonatorju,
zadnji ulomek v oklepaju pa nam podaja obliko homogeno razširjene atomske
črte: 
\begin{equation}  \label{5.69}
\bar{Z}=\bar{Z}_0\left[1+\frac{2B}{A}\,g(\omega_0- \Omega)w\right]^{-1}
\end{equation}
To je natanko enako izrazu za nasičenje zasedenosti stanj, ki smo ga
izpeljali iz zasedbenih enačb v četrtem poglavju.

Postavimo $P_0$ iz prve enačbe sistema \ref{5.67} v drugo: 
\begin{equation}  \label{5.70}
E_0[i(\Omega-\omega_{\lambda})+\frac{1}{\tau}] [i(\Omega- \omega_0)
+\gamma]=-\frac{v_{12}^2 \omega_0}{2\hbar\epsilon_0}\,E_0\,\bar{Z}\;.
\end{equation}
V delujočem laserju je $E_0\ne 0$, zato lahko krajšamo. $\bar{Z}$ je
realen, tako da mora biti imaginarni del leve strani enak nič: 
\begin{equation}  \label{5.71}
(\Omega- \omega_{\lambda})\gamma+(\Omega- \omega_0)\frac{1}{\tau} = 0 \;.
\end{equation}
Od tod lahko izračunamo frekvenco laserja 
\begin{equation}  \label{5.72}
\Omega=\frac{\omega_{\lambda}\gamma+ \omega_0\frac{1}{\tau}}{\gamma + \frac{1%
}{\tau}}\;.
\end{equation}
Frekvenca torej ni enaka frekvenci praznega resonatorja $\omega_{\lambda}$,
temveč je premaknjena proti centru atomske črte $\omega_0$. Premik je
odvisen od razmerja širine atomske črte in izgub resonatorja.

Bralec lahko sam iz enačbe \ref{5.70} izračuna še energijo svetlobe v
resonatorju in rezultat primerja s tistim, ki smo ga dobili z uporabo
zasedbenih enačb.

Gornji primer uporabe polklasičnih enačb je zelo preprost. Prava moč
modela se pokaže pri obravnavi mnogofrekvenčnega laserja, na primer pri
računu uklepanja faz laserskih nihanj, kar pa presega okvir te knjige. Več
bo bralec našel v \cite{haken2}.
