\chapter{Laser}

Stoječe valovanje v optičnem resonatorju se obnaša kot dušeno
harmonično nihalo. Iz nihala lahko napravimo oscilator, to je napravo, ki
samostojno niha brez zunajega vzbujanja, če nam dano nihalo uspe povezati v
povratno zvezo z ustreznim ojačevalnikom, ki pokriva izgube nihala zaradi
dušenja. Primeri so ure z mehanskimi nihali, oscilatorji z električnim
nihajnim krogom v elektroniki, pa vrsta glasbenih inštrumentov.

Ugotovili smo, da je ojačevanje svetlobe mogoče dobiti v sredstvu z
obrnjeno zasedenostjo med dvema nivojema. Postavimo tako snov v optični
resonator. Na začetku dobimo predvsem spontano sevano svetlobo, ki se
odbija med zrcaloma resonatorja in se pri prehodu skozi snov ojačuje. Tako
se vzbujajo nihanja resonatorja z nihajnimi frekvencami blizu frekvence
atomskega prehoda, pri katerem snov ojačuje. Energija nihanj z dovolj
majhnimi izgubami bo rasla, dokler se ojačevanje ne bo izenačilo z
izgubami. Tak izvor svetlobe je laser. Beseda je nastala iz kratice za Light
Amplification by Stimulated Emission of Radiation.

Kot analogija za laser so posebno zanimiva glasbila, predvsem pihala.
Vzemimo na primer klarinet. To glasbilo razdelimo na dve bistveni enoti:
cev, v kateri lahko nastane stojni zvočni val, in ustnik, katerega naloga
je dovajati energijo in s tem vzdrževati konstantno amplitudo nihanja.

Frekvenca stoječega vala, to je nihanja zračnega stolpca v piščali, je
določena z dolžino cevi ( pravzaprav le do prve odprte tonske luknjice,
vendar te podrobnosti za nas niso pomembne ) in s številom vozlov
stoječega vala v cevi. Na gornjem koncu, pri ustniku, lahko predpostavimo,
da je cev zaprta, zato imamo tam hrbet nihanja pritiska. Cev je torej
zvočni resonator.

Ustnik je polobel zaključek cevi, ki ga skoraj povsem zapira ploščat,
prožen jeziček. Ko glasbenik piha v ustnik, se ta trese in s tem proizvaja
zvok. "Ce ustnik ni nataknjen na cev klarineta, je nastali zvok bolj
nekakšen šum kot pa lep, dobro določen ton. Tresenje jezička je le
približno periodično in vsebuje mnogo frekvenc.

Ko ustnik nataknemo na cev in pihnemo vanj, začne tresenje jezička
vzbujati tiste stoječe valove v cevi, katerih frekvence so vsebovane v
spektru tresenja jezička. Dokler je vzbujanje šibko, se ne zgodi nič
posebnega, ko pa amplituda tlaka v cevi dovolj naraste, pride do čisto
novega pojava. Nihanje tlaka v gornjem koncu cevi povratno deluje na ustnik
in ga sili, da niha s frekvenco najbolj vzbujenega stoječega vala v cevi,
nihanje jezička z drugimi frekvencami pa zamre. Moč pihanja gre le še v
nihanje jezička s pravo frekvenco in ojačuje nihanje zračnega stolpca. S
pihanjem v ustnik lahko torej zaradi povratne zveze med nihanjem jezička in
nihanji zračnega stolpca v cevi vzdržujemo stoječe valovanje s konstantno
amplitudo. Dovedena moč se seveda izseva kot zvok na odprtem koncu
klarineta.

\begin{figure}[tbp]
\label{s5.1} \vskip 5cm
\caption{Osnovna shema laserja}
\end{figure}

Vrnimo se k svetlobi in si najprej oglejmo najpreprostejši model laserja.
Privzemimo, da je le eno resonatorsko nihanje tako, da njegova frekvenca
sovpada s frekvenco prehoda aktivne snovi, ki jo konstantno črpamo in s tem
vzdržujemo obrnjeno zasedenost. Ta privzetek v večini laserjev ni
avtomatično izpolnjen, vendar je pogosto z dodatnimi elementi v resonatorju
mogoče doseči, da je vzbujeno le eno nihanje, kot bomo videli nekoliko
kasneje.

Naj bo $W$ energija svetlobnega valovanja v resonatorju. Zaradi izgub skozi
zrcali in zaradi absorpcije in sipanja se energija na en prelet resonatorja
zmanjša za 
\begin{equation}
\Delta W_{izgube}=-\Lambda W=-[\alpha L+(1-\frac{1}{2}{\cal {R}}_{1})+\frac{1%
}{2}(1-{\cal {R}}_{2})]\,W\;\;,  \label{5.1}
\end{equation}
kjer so $\Lambda $ celotne izgube, $\alpha $ izgube na enoto poti zaradi
absoprcije in sipanja, ${\cal {R}}_{1}$ in ${\cal {R}}_{2}$ pa odbojnosti
obeh zrcal. Vsaj eno od obeh zrcal mora imeti odbojnost manj od 1, če naj
laser nekaj svetlobe tudi izseva.

Zaradi ojačevanja s stimuliranim sevanjem snovi z obrnjeno zasedenostjo se
energija nihanja resonatorja na en prelet po enačbi \ref{4.43} poveča za 
\begin{equation}  \label{5.2}
\Delta W_{oj}=\frac{LGW}{1+W/W_s}\;\;.
\end{equation}
Upoštevali smo, da pride pri veliki energiji svetlobe do nasičenja, pri
čemer smo namesto saturacijske gostote svetlobnega toka $j_s$ vpeljali
saturacijsko energijo $W_s=Vj_s/c$. Privzeli smo tudi, da je ojačenje na en
prelet dovolj majhno, da nam enačbe \ref{4.43} ni treba integrirati.

V stacionarnem stanju se morajo izgube izenačiti z ojačenjem: \newline
$-\Delta W_{izgube}=\Delta W_{oj}$, od koder dobimo, da je ali $W=0$ ali 
\begin{equation}  \label{5.3}
\Lambda=\frac{LG}{1+W/W_s}\;\;.
\end{equation}
Energija svetlobnega nihanja je torej 
\begin{equation}  \label{5.4}
W=\frac{LG-\Lambda}{\Lambda}\,W_s
\end{equation}
in je pozitivna le, če je ojačenje $G$ večje od praga 
\begin{equation}  \label{5.5}
G_{pr}=\frac{\Lambda}{L}\;\;.
\end{equation}
Ojačenje je seveda odvisno od stopnje obrnjene zasedenosti, ki jo lahko
spreminjamo z močjo optičnega črpanja. Pogosto, na primer pri
trinivojskem sistemu, obravnavanem v razdelku 4.7, je kar sorazmerno z
močjo črpanja. Energija svetlobe v laserju je pod pragom nič, nad pragom
pa je linearna funkcija ojačenja, kot kaže slika \ref{s5.2}.

\begin{figure}[tbp]
\label{s5.2} \vskip 5cm
\caption{Odvisnost energije svetlobe v laserju od ojačenja}
\end{figure}

Izhodno moč laserja dobimo tako, da delež energije, ki gre na en prelet
skozi izhodno zrcalo, delimo s časom preleta preko resonatorja $L/c$: 
\begin{equation}
P_{izh}=\frac{1}{2}(1-{\cal {R}}_{1})\frac{c}{L}\,W\;\;.  \label{5.6}
\end{equation}
"Ce upoštevamo en. \ref{5.1} in \ref{5.4}, vidimo, da je izhodna moč v
primeru, da imamo le izgube skozi izhodno zrcalo, največja, če je
reflektivnost čim bliže 1. Ker imamo vedno še druge izgube, je izhodna
moč največja pri ${\cal {R}}_{1}<1$ (Naloga).

\section{Zasedbene enačbe}

Za podrobnejšo sliko se moramo vrniti k enačbam za zasedenost atomskih
nivojev \ref{4.39}, ki jim dodamo še enačbo za energijo lastnega valovanja
v resonatorju. "Se naprej se omejimo na primer, ko je vzbujeno le eno
resonatorsko stanje.

Zaradi enostavnosti obravnave lahko enačbe \ref{4.39} brez škode precej
poenostavimo. Privzemimo, da je razpadni čas spodnjega laserskega stanja $%
|1\rangle$, ki ga določa koeficient $A_{10}$, dosti krajši od razpadnega
časa zgornjega stanja $|2\rangle$. Tedaj je $N_1\simeq0$, če le ni preveč
stimuliranega sevanja, in lahko zasedenost atomskih stanj popišemo z eno
samo spremenljivko $N_2$. Energijo v izbranem stanju resonatorja zapišimo s
števi\-lom fotonov $n$, tako da je gostota energije polja $n\hbar\omega/V$,
kjer je $V$ volumen resonatorja. S tem dobimo za zasedenost 
\begin{equation}  \label{5.7}
\frac{dN_2}{dt}=- \frac{1}{V}\hbar\omega B_{21}g(\omega)n\,N_2-N_2A_{21}+rN
\end{equation}

Energija svetlobe v resonatorju se povečuje zaradi stimuliranega sevanja.
Atomi z gornjega stanja prehajajo še s spontanim seva\-njem, ki ga nekaj
tudi gre v izbrano nihanje resonatorja. V razdelku 4.8 prejšnjega poglavja
smo videli, da je verjetnost za prehod atoma z višjega na nižje stanje z
izsevanjem fotona v izbrano stanje elektromagnetnega polja sorazmerno z $n+1$%
, kjer je $n$ število fotonov v stanju. Od tod sledi, da upoštevamo
prispevek spontanega sevanja s tem, da v prvem členu desnega dela izraza 
\ref{5.7} pišemo $n+1$ namesto $n$. Energija resonatorja se zmanjšuje
zaradi izgub skozi zrcala in absorpcije in sipanja, kar smo v tretjem
poglavju opisali z razpadnim časom $\tau/2=\Lambda c/L$. Tako imamo 
\begin{equation}  \label{5.8}
\frac{dn}{dt}=\frac{1}{V}\hbar\omega B_{21}g(\omega)(n+1)\, N_2-\frac{2}{\tau%
}\,n\;\;.
\end{equation}

Dobili smo sistem dveh diferencialnih enačb za časovni razvoj števila
fotonov v resonatorskem stanju in za zasedenost atomskega stanja. Enačbi
sta nelinearni in ju ne znamo analitično rešiti.

Kot je pri nelinearnih diferencialnih enačbah običajno, poglejmo
najprej, kakšne so stacionarne, to je, od časa neodvisne rešitve $\dot{%
N}_{2}=0$ in $\dot{n}=0$. Izrazimo $N_{2}$ iz \ref{5.7} in postavimo v \ref
{5.8}. Dobimo kvadratno enačbo za število fotonov: 
\begin{equation}
\frac{2}{V\tau }n(A_{21}+B_{21}\hbar \omega g(\omega )n)-\frac{1}{V}%
B_{21}\hbar \omega g(\omega )rN(n+1)=0\;\;.  \label{5.9}
\end{equation}
Preden zapišemo rešitve dobljene enačbe, jo še nekoliko
preoblikujmo. Ulomek $\frac{B_{21}\hbar \omega g(\omega )rN}{VcA_{21}}$ je
koeficient ojačenja, ki smo ga v prejšnjem poglavju označili z $G$. $%
2/c\tau $ so izgube,ki morajo biti enake ojačenju na pragu $G_{pr}$.
Vpeljimo še brezdimenzijsko konstanto $p$: 
\begin{equation}
p=\frac{VA_{21}}{B_{21}\hbar \omega g(\omega )}=\frac{V\omega ^{2}}{\pi
^{2}c^{3}g(\omega )}=\frac{VA_{21}}{B_{21}\hbar \omega g(\omega )}\simeq 
\frac{V\omega ^{2}}{\pi ^{2}c^{3}}\Delta \omega \;\;.  \label{5.10}
\end{equation}
V zadnjem izrazu smo upoštevali, da je v maksimumu $g(\omega )\simeq
=1/\Delta \omega $. $p$ je torej približno produkt gostote stanj
elektromagnetnega polja v resonatorju in širine atomskega prehoda, torej
kar število vseh stanj v frekvenčnem intervalu atomskega prehoda. To je
navadno precj veliko, okoli $10^{8}$ do $10^{10}$. S primerjavo izraza \ref
{4.34} za saturacijsko gostoto toka vidimo še, da je $p$ tudi število
fotonov v resonatorju, pri katerem pride do nasičenja ojačenja, če je
frekvenca nihanja resonatorja blizu centra atomske črte. S tem en. \ref
{5.9} dobi obliko 
\begin{equation}
\frac{1}{p}\,n^{2}-(\frac{G}{G_{pr}}-1)n-\frac{G}{G_{pr}}=0  \label{5.11}
\end{equation}
s pozitivno rešitvijo 
\begin{equation}
n=\frac{p}{2}\left[ (\frac{G}{G_{pr}}-1)+\sqrt{(\frac{G}{G_{pr}}-1)^{2}+%
\frac{4G}{pG_{pr}}}\right] \;\;.  \label{5.12}
\end{equation}
Ker je $p$ veliko število, lahko koren razvijemo, če le ni ojačenje
preveč blizu praga, ko je $\frac{G}{G_{pr}}\simeq 1$. Pod pragom je $%
G<G_{pr}$ in je 
\begin{equation}
n\simeq \frac{p}{2}\left[ (\frac{G}{G_{pr}}-1)+(1-\frac{G}{G_{pr}})+\frac{2G%
}{p(G_{pr}-G)}\right] =\frac{G}{G_{pr}-G}\;\;.  \label{5.13}
\end{equation}
Pri razvoju korena smo upoštevali, da mora biti pozitiven. Nad pragom je
število fotonov 
\begin{equation}
n\simeq p(\frac{G}{G_{pr}}-1)\;\;.  \label{5.14}
\end{equation}

Poglejmo, kaj smo dobili. Pod pragom je število fotonov v izbranem
resonatorskem nihanju blizu ena do neposredne bližine praga, kjer hitro
naraste in doseže takoj nad pragom red velikosti $p$. Moč črpanja, s
katero spreminjamo koeficient ojačenja $G$, gre pod pragom preko spontanega
sevanja skoraj vsa v veliko število stanj elektromagnetnega polja, nad
pragom pa povsem prevlada stimulirano sevanje v eno samo izbrano nihanje
resonatorja. Obnašanje števila fotonov $n$ pri spremi\-njanju ojačenja $G$
kaže slika \ref{s5.3}. Prehod preko praga je zaradi velikega $p$ v večini
laserjev tako hiter, da ga ni mogoče izmeriti; izjema so polvodniški
laserji, katerih volumen je zelo majhen, zato je tudi $p$ dovolj majhen, da
je mogoče opaziti zvezen prehod preko praga.

\begin{figure}[tbp]
\label{s5.3} \vskip 5cm
\caption{Odvisnost števila fotonov od ojačenja}
\end{figure}

Iz enačbe \ref{5.8} lahko izračunamo še zasedenost gornjega atomskega
nivoja v stacionarnem stanju: 
\begin{equation}  \label{5.15}
N_2=\frac{2V}{\tau B_{21}\hbar\omega g(\omega)}\frac{n}{n+1}\;\;.
\end{equation}
Na pragu je po enačbi \ref{5.12} $n=\sqrt{p}$. Tako dobimo 
\begin{equation}  \label{5.16}
N_{2pr}=\frac{2V}{\tau B_{21}\hbar\omega g(\omega)}\frac{\sqrt{p}}{\sqrt{p}+1%
}\;\;.
\end{equation}
Ker je tudi $\sqrt{p}$ veliko število, sledi iz gornjih enačb, da obrnjena
zasedenost narašča do bližine praga, nad pragom pa je praktično
konstantna in skoraj natanko enaka kot na pragu. To ni težko razumeti. Nad
pragom je število fotonov v resonatorju veliko in linearno (v našem
preprostem modelu) narašča s povečevanjem moči črpanja, zato se
povečuje tudi hitrost praznenja gornjega atomskega stanja s stimuliranim
sevanjem, kar ravno izniči učinek povečanja črpanja. V stacionarno
delujočem laserju torej ni mogoče povečati obrnjene zasedenosti nad
vrednost na pragu $N_{2pr}$, kar ima pomembne praktične posledice, kot bomo
videli nekoliko kasneje.

V laserju nad pragom je število fotonov zelo veliko, zato prispevka
spontanega sevanja v enačbi \ref{5.8} v nadaljevanju ne bomo upoštevali.

Obravnava laserja z zasedbenimi enačbami je seveda zelo groba. Nismo
upoštevali, da je svetloba v resonatorju valovanje, ki uboga valo\-vno
enačbo. S tem smo privzeli, da je prostorska odvisnost polja v delujočem
laserju enaka kot za lastno stanje praznega resonatorja. Poleg tega smo
privzeli, da so atomi lahko le v lastnih energijskih stanjih, kar je res le
v primeru stacionarnih stanj brez zunanjega, časovno odvisnega polja
svetlobe. "Ce za opis svetlobe uporabimo klasično valovno enačbo, za atome
pa kvantno mehaniko, dobimo {\it polklasični približek} vedenja laserja,
ki je mnogo zahtevnejši od zasedbenih enačb, opiše pa skoraj vse pojave v
laserjih, razen vpliva spontanega sevanja. Za dosledno in podrobno obravnavo
tega pa je potrebno tudi svetlobo opisati s pomočjo kvantne elektrodinamike.

Povzemimo, kaj smo z modelom zasedbenih enačb ugotovili o delovanju
enofrekvenčnega laserja. Pri dovolj velikem ojačenju s stimuliranim
sevanjem, ki pokriva izgube resonatorja, je v stacionarnem stanju energija
in s tem amplituda izbranega lastnega nihanja resonatorja različna od nič.
Del valovanja izhaja skozi eno ali obe zrcali. Frekvenca svetlobe je
določena z izbranim lastnim stanjem resonatorja. To določa tudi prostorsko
odvisnost valovanja v resonatorju in izhodnega snopa. Ugotovili smo, da ima
v običajnem stabilnem resonatorju polje obliko zelo blizu Gaussovega snopa,
zato je tak tudi izhodni snop.

Lepa prostorska odvisnost izhodnega snopa je morda najpomembnejša lastnost
laserjev. Gaussov snop se najmanj širi zaradi uklona in ga je mogoče
zbrati v piko približne velikosti valovne dolžine. S tem se tudi najbolj
približa idealno točkastemu izvoru svetlobe. Te lastnosti so zelo pomembne
za uporabo; na njih so osnovani optični komunikacijski sistemi, čitanje
optičnih diskov, laserski obdelovalni stroji...

\section{Spektralna širina enega laserskega nihanja}

Spektralni širini svetlobe enofrekvenčnega laserja je treba posvetiti
nekaj več pozornosti. Videli smo, da je amplituda svetlobe laserja
konstantna. Spektralna širina klasičnega harmonskega nihala s stalno
amplitudo je nič, frekvenca nihanja je natanko določena. "Ce bi se lastno
stanje elektromagnetnega polja v resonatorju obnašalo povsem enako kot
klasično harmonsko nihalo, bi bil tudi spekter laserja neskončno ozek.
Laserji imajo seveda končno spektralno širino, v idealnem primeru zaradi
kvantizacije elektromagnetnega polja, v praksi pa zaradi zunanjih motenj.

Poskusimo najprej oceniti razširitev zaradi vpliva kvantizacije. Zaradi nje
imamo poleg stimuliranega tudi spontano sevanje. To je kvantni šum, ki
povzroči majhno razširitev spektra. Stroga obravnava tega pojava je
mogoča le z dosledno uporabo kvantne elektrodinamike, kar seže izven
okvira te knjige, zato se zadovoljimo le z grobo oceno.

\begin{figure}[tbp]
\label{s5.4} \vskip 5cm
\caption{Amplituda polja v resonatorju in prispevek spontanega sevanja.}
\end{figure}

Amplitudo svetlobe na izbranem mestu v resonatorju predstavimo kot število
v kompleksni ravnini (Slika \ref{s5.4}). Kot, ki ga amplituda tvori z realno
osjo, je faza glede na neko izbrano začetno fazo. Energija svetlobe je
sorazmerna s številom fotonov; velikost amplitude polja zato lahko merimo
kar s korenom iz števila fotonov v nihanju. Velikost amplitude je
praktično konstantna, vzdržuje jo stimulirano sevanje, ki ravno pokriva
izgube resonatorja. Pri tem ostaja  tudi faza nespremenjena. Pač pa se
faza spremeni zaradi majhnega prispevka spontanega sevanja.

Pri spontani emisiji se foton izseva s poljubno fazo; prispevek k kompleksni
amplitudi na sliki \ref{s5.4} ima torej dolžino 1 in poljubno smer. Zanima
nas povprečni kvadrat spremembe faznega kota pri enem spontano izsevanem
fotonu: 
\begin{equation}
\overline{\delta \phi _{1}^{2}}=\frac{1}{\overline{n}}\,\overline{\cos
^{2}\psi }=\frac{1}{2\overline{n}}\;\;,  \label{5.17}
\end{equation}
kjer smo s $\psi $ označili slučajno fazno razliko med celotnim poljem
in spontano izsevanim fotonom, indeks 1 pa označuje, da gre za spremembo
za 1 foton. Zaporedne spontane emisije so med seboj neodvisne, zato dobimo
povprečni kvadrat spremembe faze pri $m$ emisijah kar tako, da seštejemo
povprečne kvadrate za posamezne fotone: 
\begin{equation}
\ \overline{\delta \phi _{m}^{2}}=m\overline{\delta \phi _{1}^{2}}=\frac{m}{2%
\overline{n}}\;\;.  \label{5.18}
\end{equation}

Prešteti moramo še, koliko je spontano izsevanih fotonov na enoto
časa. Stimulirano sevanje ravno pokrije izgube resonatorja, zato je
stimulirano izsevnih fotonov na enoto časa $2\overline{n/}\tau $. Iz
razdelka 4.8, enačba \ref{4.56} vemo, da je razmerje med verjetnostjo za
stimulirano in spontano sevanje enako številu fotonov v danem stanju
polja, zato je število spontanih sevanj na časovno enoto kar $2/\tau $.
Tako imamo v poljubnem času $m=2t/\tau $ in 
\begin{equation}
\overline{\delta \phi ^{2}(t)}=\frac{t}{\overline{n}\tau }\;\;.  \label{5.19}
\end{equation}
"Cas $t_{p}$, v katerem se bo faza znatno spremenila, recimo za 1, je torej
velikostnega reda 
\begin{equation}
t_{p}\simeq 2\overline{n}\tau =\frac{W}{\hbar \omega }\,\tau =\frac{P}{\hbar
\omega }\tau ^{2}\;\;.  \label{5.20}
\end{equation}
To lezenje faze nam seveda da spektralno širino $1/t_{p}$.

"Stevilo fotonov v laserskem nihanju je nad pragom zelo veliko, recimo $10^9$
v majhnem He-Ne laserju. $\tau$ je reda velikosti $10^{-7}$, tako da je
karakteristični čas - koherenčni čas - idealnega laserja mnogo sekund.
Iz enačbe \ref{5.20} vidimo še, da je spektralna širina obratno
sorazmerna z izhodno močjo laserja. V neposredni bližini praga, kjer je $%
\overline{n}\simeq 1$, je spektralna širina približno enaka širini nihanj
praznega resonatorja.

Dejanski laserji seveda nimajo niti približno tako ozkega spektra, kot smo
pravkar ocenili. Frekvenca laserja je določena z dolžino resonatorja: $%
\omega=n\pi c/L$, pri čemer je $n$ zelo veliko celo število. Majhna
sprememba dolžine resonatorja povzroči tudi spremembo frekvence laserja,
pri znatnejši spremembi dolžine pa lahko pride tudi do preskoka vzbujenega
stanja resonatorja, to je števila $n$. Dolžina resonatorja se lahko
spreminja predvsem zaradi zunanjih mehanskih motenj in zaradi spreminjanja
temperature. Zato se spreminja tudi frekvenca laserja in če se posebej ne
potrudimo s konstrukcijo resonatorja, so te fluktuacije frekvence kar reda
velikosti razmika med sosednimi stanji resonatorja, to je reda velikosti 100
MHz.

Tu velja opozoriti, da je narava spektralne razširitve v laserju tako v
idealnem kot v praktičnem primeru drugačna kot v navadnih svetilih. V
prvem poglavju smo videli, da intenziteta svetlobe navadnega svetila
fluktuira na časovni skali koherenčnega časa, ki je obraten spektralni
širini. Po elektrotehniško lahko rečemo, da je taka svetloba amplitudno
moduliran šum. Pri enofrekvenčnem laserju je drugače. Amplituda in s tem
intenziteta izhodne svetlobe je konstantna, fluktuira le frekvenca oziroma
faza. "Sum laserja je torej v obliki frekvenčne modulacije.

Fluktuacije dolžine in s tem frekvence laserja je mogoče zmanjšati. S
skrbno konstrukcijo in uporabo materialov z majhnim toplotnim raztezkom se
zmanjša vpliv vibracij in temperaturnih sprememb v okolici. Poleg tega
lahko okolico temperaturno stabiliziramo in laser postavimo na vibracijsko
izolirano mizo. Na tak način je mogoče dobiti laser z efektivno spektralno
širino pod 1 MHz.

"Se ožjo lasersko črto lahko dobimo z aktivno stabilizacijo dolžine
resonatorja. Ideja je taka: frekvenco svetlobe, ki izhaja iz laserja,
primerjamo z nekim standardom in iz razlike ugotovimo spremembo dolžine
resonatorja. Eno od obeh zrcal je nameščeno na piezoelektričnem nosilcu,
ki mu z električno napetostjo lahko spreminjamo dolžino in tako popravimo
dolžino resonatorja.

Pogalvitna težava je seveda, kako najti dovolj stabilen primerjalni
standard za frekvenco. Ena možnost je, da izhodno svetlobo spustimo skozi
konfokalni interferometer, ki je skoraj v resonanci z laserjem in ima dovolj
ozek vrh prepustnosti. Majhen premik frekvence laserja bo povzročil, da se
bo spremenil skozi interferometer prepuščeni svetlobni tok. Na prvi pogled
je videti, da s tem nismo nič pridobili, saj bo resonančna frekvenca
interferometra stabilna tudi le toliko, kot je stabilna njegova dolžina.
Vendar je z izolacijo in temperaturno stabilizacijo možno držati dolžino
praznega resonatorja - interferometra - mnogo natančneje kot dolžino
laserja, v katerem imamo aktivno sredstvo, ki mu moramo dovajati energijo.

Druga možnost je stabilizirati laser na primerno molekularno absorpcijsko
črto. Te so lahko zelo ozke, zato je tudi spekter laserja lahko izredno
ozek, pod 1 kHz. Pri tem moti Dopplerjeva raziširitev absorpcijske črte,
ki pa se ji je mogoče izogniti. Kako to napravimo in kako je bila s tem
omogočena nova definicija metra, si bomo pogledali v razdelku ????.

\section{Primerjava laserjev in običajnih svetil}

"Cas je, da povzamemo, kar smo doslej dognali o lastnostih laserjev in jih
primerjamo z običajnimi svetili. Kot ves čas doslej obravnavajmo
enofrekvenčni laser, v katerem je vzbujeno le eno osnovno stanje
resonatorja, ki je Gaussove oblike.

Svetlobni snop, ki izhaja iz laserja, ima dve takoj očitni odliki. Je zelo
usmerjen in zelo enobarven. Prva lastnost je seveda posledica tega, da je
lastno stanje stabilnega resonatorja Gaussove oblike in je zato tak tudi
izhodni snop. Kot smo videli v 3. poglavju, je divergenca takega snopa samo
posledica uklona in je najmanjša možna. Valovne fronte so gladke in na
dani razdalji ves čas enake, laserski snop je torej prostorsko idealno
koherenten. Je tudi najboljši približek točkastega svetila, kar je v
neposredni zvezi s prostorsko koherenco.

Koherenten Gaussov snop lahko z ustrezno optiko zberemo v piko velikosti
valovne dolžine, s čimer dosežemo že pri skromni moči zelo veliko
gostoto svetobnega toka. To izkoriščajo v tehnologiji za natančno in
čisto obdelavo materialov in v medicini, kjer laserje uporabljajo za
zahtevne kirurške posege.

Kako pa je z običajnimi svetili? V njih vsak atom sveti po svoje, zato
nimamo prostorske koherence. Valovna fronta na danem mestu je nepravilna in
se v koherenčnem času znatno spremeni. Svetlost površine svetila je
neodvisna od smeri (Lambertov zakon). Osvetljenost slike, ki jo dobimo s
poljubnim optičnim sistemom, ne more biti večja od izsevane gostote
svetlobnega toka.

\begin{figure}[tbp]
\label{s5.5} \vskip 5cm
\caption{Zaslonke za pripravo prostorsko koherentnega snopa iz
nekoherentnega svetila}
\end{figure}

Tudi iz svetlobe običajnega nekoherentnega svetila lahko pripravimo
koherenten snop. Na neki razdalji od svetila moramo postaviti zaslonko, ki
je manjša od koherenčne ploskve na tistem mestu (glej razdelek 1.3).
Ocenimo, kolikšna je moč snopa za zaslonko. Svetilo naj ima svetlost%
\footnote{%
Svetlost je svetilnost na enoto ploskve} $B$. Pri najsvetlejših
nekoherentnih izvorih, to so živosrebrne svetilke, doseže $B$ vrednost
do 100 W/cm$^{2}$. Moč snopa za zaslonko je (slika \ref{s5.5}) 
\begin{equation}
P=BS_{0}\Delta \Omega =\frac{BS_{0}S_{c}}{z^{2}}\simeq \frac{BS_{0}}{z^{2}}\,%
\frac{\lambda ^{2}z^{2}}{S_{0}}=B\lambda ^{2}\;\;.  \label{5.21}
\end{equation}
Tu je $S_{0}$ površina svetila, $z$ oddlajenost zaslonke, $S_{c}$ pa
velikost koherenčne ploskve, za katero smo uporabili oceno \ref{1.8} iz
prvega poglavja. Pri svetlosti 100 W/cm$^{2}$ dobimo, da je moč
koherentnega snopa le približno $3\cdot 10^{-7}$ W, kar je štiri rede
velikost manj od prav šibkih laserjev z močjo 1 mW.

Druga odlična lastnost svetlobe iz enofrekvenčnega laserja je zelo majhna
spektralna širina. Z nekaj truda je ta lahko pod 1 kHz, emisijske črte v
plinu pa so zaradi Dopplerjeve razširitve široke vsaj nekaj GHz, pa še to
le v razmeroma redkem in hladnem plinu, kjer je svetlost majhna.

Različne širine spektrov običajne svetilke in laserja lahko
upoštevamo tako, da kot primerjalno količino vzamemo namesto celotne
moči v koherentnem snopu spektralno gostoto moči. Majhen šolski He-Ne
laser seva 1 mW v približno $10^{7}$ Hz, tako da je spektralna gostota
moči $dP/d\nu \simeq 10^{-10}$ W/Hz. Zelo svetla živosrebrna svetilka
seva v močno razširjene spektralne črte s širino okoli 10 nm $\simeq
10^{13}$ Hz. Spektralna gostota v koherentnem snopu, ki ga pripravimo iz
take svetilke, bo tako le okoli $3\cdot 10^{-20}$ W/Hz. Majhen šolski
He-Ne laser tako prekaša najmočnejše nekoherentno svetilo za skoraj 10
velikostnih redov. Z laserji je seveda mogoče doseči mnogo večje
moči, v sunkih tja do $10^{12}$ W, tako da po spektralni gostoti moči v
koherentnem snopu laserji prekašajo običajna svetila do preko 20
velikostnih redov. Za toliko je tudi večja dosegljiva gostota moči na
enoto ploskve in frekvence, kar je za vrsto uporab laserjev, posebno v
spektroskopiji, odločilnega pomena. Najbrž v zgodovini težko najdemo
še kak drug izum, ki je prinesel tolikšno izboljšavo v neki bistveni
količini in tako ni čudno, da je prihod laserjev v začetku 60-tih let
povzročil novo rojstvo optike.

Med laserji in običajnimi svetili je še ena pomembna, a manj opazna
razlika. Z ustreznim interferometrom lahko tudi snop iz nekoherentnega
svetila filtriramo, tako da dobimo enako spektralno širino kot iz laserja,
sveda z mnogo manjšo močjo. Vendar je narava spektralne razširitve
različna. "Sum laserja je v obliki frekvenčne modulacije, pri čemer je
amplituda svetlobnega vala konstantna, šum nekohernetnega svetila pa je v
obliki amplitudne modulacije.

\section{Mnogofrekvenčni laser}

Doslej smo obravnavali le laser, v katerem je bilo vzbujeno eno samo stanje
resonatorja. Ojačevalna širina večine aktivnih sredstev je precej
velika. V plinih je na primer zaradi Dopplerjevega pojava vsaj nekaj GHz.
Lastne frekvence resonatorja so navadno mnogo manj narazen, pri 30 cm dolgem
resonatorju je razmik 500 MHz. Tako se prav lahko zgodi, da je ojačenje v
laserju dovolj veliko za več nihanj hkrati. Vzbujena bodo vsa tista
nihanja, za katere je ojačenje večje od ojačenja na pragu $G_{pr}$.
Svetloba iz takega mnogofrekvenčnega laserja ni več monokromatska,
temveč je sestavljena iz množice ozkih črt znotraj ojačevalnega pasu
in tako ni mnogo bolj monokromatska kot ustrezna spektralna komponenta
svetlečega plina. Ostaja pa prostorsko koherentna.

\begin{figure}[tbp]
\label{s5.6} \vskip 5cm
\caption{ Frekvenčna odvisnost ojačenja, položaj lastnih frekvenc
resonatorja in ojačenje na pragu.}
\end{figure}

Za holografijo, interferometrijo in nekatere spektroskopske uporabe
potrebujemo ozko spektralno črt. Zato moramo poskrbeti, da bo vzbujeno le
eno nihanje resonatorja, najbolje tisto, ki je najbližje vrhu ojačenja
ativnega sredstva, kar napravimo tako, da povečamo izgube za vsa ostala
nihanja. Ena možnost je, da v resonator postavimo še Fabry-Perotov
interferometer, kot kaže slika \ref{s5.6}. Njegova prepustnost v
odvisnosti od frekvence, razmika med zrcali $L_{f}$ in nagiba glede na os
resonatorja je podana z enačbo (glej 3. poglavje) 
\begin{equation}
T=\frac{1}{1+\frac{4{\cal {R}}}{(1-{\cal {R}})^{2}}\sin ^{2}(\frac{\omega }{c%
}L_{f}\cos \phi )}  \label{5.22}
\end{equation}
in jo kaže slika \ref{s5.7}. Izrazitost vrhov prepustnosti je odvisna od
reflektivnosti zrcal interferometra. Z ustrezno izbiro razmika med zrcali in
nagiba lahko dosežemo, da vrh prepustnosti ravno sovpada z izbranim
stanjem resonatorja. Izgube za ostala nihanja, vsebovane v $G_{pr}$, so se s
tem povečale in laser sveti le pri eni sami izbrani frekvenci. Na sliki 
\ref{s5.7} so prikazane povečane izgube, lastna stanja 
\begin{figure}[tbp]
\label{s5.7} \vskip 7cm
\caption{Frekvenčna odvisnost ojačenja, položaj lastnih frekvenc
resonatorja in ojačenje na pragu, kadar je v resonator vgrajen
Fabry-Perotov interferometer.}
\end{figure}
resonatorja in ojačenje aktivnega sredstva. Ker zadošča že zmerno
povečanje izgub, je reflektivnost zrcal interferometra običajno dokaj
nizka, pod 0,5.

Naloga: Izberi ustrezne parametre interferometra.

Nagib interferometra je potreben tudi zato, da se neprepuščena svetloba
odbije ven iz smeri osi resonatorja. "Ce bi bila os interferometra vzporedna
z osjo resonatorja, bi nastale dodatne, neželjene resonance z osnovnimi
zrcali resonatorja, kar bi močno motilo delovanje laserja.

\section{Relaksacijske oscilacije}

Doslej smo obravnavali le stacionarno delovanje laserjev. Včasih želimo
izhodno moč modulirati s spreminjanjem črpanja. Pogosto laserji delujejo v
sunkih. V nekaterih aktivnih sredstvih je mogoče doseči obrnjeno
zasedenost le za kratek čas, črpanje v sunkih je včasih enostavnješe,
recimo z bliskavko pri trdnih laserjih, dostikrat pa tudi želimo dobiti iz
laserja kratke in močne koherentne sunke svetlobe. Za obravnavo
nestacionarnega delovanja moramo seveda reševati sistem diferencialnih
enačb \ref{5.7} in \ref{5.8}, kar gre v splošnem le numerično.

Preden se lotimo sunkov, poglejmo, kako se obnaša laser, ki je blizu
stacionarnega stanja. Spet se omejimo na enofrekvenčni laser, ki ga
opišemo z enačbami \ref{5.7} in \ref{5.8}. Te zaradi preglednosti
zapišimo nekoliko drugače. Vpeljimo brezdimnezijski čas $t^{\prime}=t A$
in $\tau^{\prime}=\tau A$, čas torej merimo v enotah življenskega časa
laserskega nivoja. Upoštevajmo še, da je $VA/(B\hbar\omega g(\omega_0))=p$%
, kjer je $p$ število stanj elektromagnetnega polja v volumnu $V$ in
znotraj spektralne širine laserskega nivoja. Tako imamo 
\begin{eqnarray}  \label{5.23}
\frac{d N_2}{d t^{\prime}}&=&-\frac{1}{p}\,nN_2-N_2+N_{20} \\
\frac{d n}{d t^{\prime}}& = & \frac{1}{p}\,nN_2-\frac{2}{\tau^{\prime}}n
\;\;.
\end{eqnarray}
Pri tem smo vpeljali konstanto $N_{20}= rN/A$, ki ima tudi nazornen pomen:
je zasedenost, ki bi jo dobili pri danem stacionarnem črpanju, če v
izbranem stanju ne bi bilo fotonov in s tem ne stimuliranega sevanja in
torej meri moč črpanja. V enačbi za hitrost spreminjanja števila fotonov
smo zanemarili prispevek spontanega sevanja, za katerega smo že ugotovili,
da se pozna le do praga.

Pri nelinearnih diferencialnih enačbah lahko pogosto dobimo upora\-bne
približke z linearizacijo. Naj laser najprej deluje stacionarno, v nekem
trenutku pa ga nekoliko izmaknemo iz stacionarnega stanja, recimo tako, da
spremenimo moč črpanja. Trenutno zasedenost $N_2$ in število fotonov $n$
lahko zapišemo v obliki 
\begin{equation}  \label{5.24}
N_2= N_{2s}+x \mbox{\hskip 1cm in \hskip 1cm} n=n_s+y\;\;,
\end{equation}
kjer sta $N_{2s}$ in $n_s$ vrednosti $N_2$ in $n_s$ v stacionarnem stanju.
Zanju velja 
\begin{equation}  \label{5.25}
N_{2s}=\frac{2p}{\tau^{\prime}}
\end{equation}
in 
\begin{equation}  \label{5.26}
n_s=p\frac{N_{20}-N_{2s}}{N_{2s}}=p(a-1)\;\;.
\end{equation}
Kot smo ugotovili že v drugem razdelku, je stacionarna zasedenost enaka
zasedenosti na pragu, ki je odvisna od izgub resonatorja, kar kaže tudi
enačba \ref{5.25}. Razmerje $a=N_{20}/N_{2s}$ je mera za moč črpanja in
mora biti v delujočem laserju večje od 1. V večini praktičnih primerov
doseže $a$ vrednosti do 3 ali 5.

Postavimo sedaj nastavka \ref{5.24} v enačbi \ref{5.23}. Dobimo 
\begin{eqnarray}  \label{5.27}
\frac{d x}{d t^{\prime}} &=&-\frac{1}{p}n_sN_{2s}-N_{2s}+N_{20}- \frac{1}{p}
(n_sx+N_{2s}y+xy)-x \\
\frac{d y}{d t^{\prime}} &=& \frac{1}{p}n_sN_{2s}-\frac{2}{\tau^{\prime}}n_s
+ \frac{1}{p}(n_s x+N_{2s} y+xy)-\frac{2}{\tau^{\prime}}y
\end{eqnarray}

Ker sta $x$ in $y$ majhna v primeri s stacionarnimi vrednostmi, lahko
zanemarimo produkt $xy$. Vsi členi, ki vsebujejo le stacionarne vrednosti,
dajo ravno 0, saj smo jih tako določili. "Ce upoštevamo še izraza \ref
{5.25} in \ref{5.26}, dobimo željeni linearizirani diferencialni enačbi
za odmika od stacionarnih vrednosti 
\begin{eqnarray}
\frac{dx}{dt^{\prime }} &=&-a\,x-\frac{2}{\tau ^{\prime }}\,y  \label{5.28}
\\
\frac{dy}{dt^{\prime }} &=&(a-1)\,x\;\;.
\end{eqnarray}
Kot smo pri linearnih sistemih diferencialnih enačb s konstantnimi
koeficienti vajeni, poiščemo rešitev v obliki eksponentne funkcije 
\begin{equation}
x=x_{0}e^{\lambda t^{\prime }}\mbox{\hskip 1cm in \hskip 1cm}%
y=y_{0}e^{\lambda t^{\prime }}\;\;.  \label{5.29}
\end{equation}
Dobljeni homogeni sistem linearnih enačb 
\begin{eqnarray}
(a+\lambda )x_{0}+\frac{2}{\tau ^{\prime }}y_{0} &=&0  \label{5.30} \\
-(a-1)x_{0}+\lambda y_{0} &=&0
\end{eqnarray}
bo imel netrivialno rešitev le, če bo njegova determinanta nič: 
\begin{equation}
\lambda ^{2}+a\lambda +\frac{2}{\tau ^{\prime }}(a-1)=0\;\;.  \label{5.301}
\end{equation}
Rešitvi sta 
\begin{equation}
\lambda =-\frac{a}{2}\pm \sqrt{\frac{a^{2}}{4}-\frac{2}{\tau ^{\prime }}(a-1)%
}\;\;.  \label{5.31}
\end{equation}
Narava rešitve je odvisna od velikosti brezdimenzijskega relaksacijskega
časa nihanja resonatorja $\tau ^{\prime }=A\tau $. Za $\tau ^{\prime }>2$
je izraz pod korenom pozitiven pri vseh $a$ in laser se vrača v
stacionarno stanje eksponentno. Za $\tau <2$ pa je koren v nekem območju
parametra $a$ imaginaren in laser se vrača v stacionarno stanje z
dušenim nihanjem, ki mu pravimo {\it relaksacijske oscilacije}.

\begin{figure}[tbp]
\label{s5.8} \vskip 5cm
\caption{Relaksacijske oscilacije}
\end{figure}

V običajnih resonatorjih je $\tau$ velikostnega reda $10^{- 7}$~s. Razpadna
konstanta laserskega nivoja $A$ je navadno dokaj majhna, ker je le v takih
primerih lahko doseči obrnjeno zasedenost, tipična vrednost je recimo $%
10^5 $ s$^{-1}$, lahko pa je še dosti manjša. Tedaj je $%
\tau^{\prime}\simeq 10^{-2}$ in imamo relaksacijske oscilacije pri vseh
dosegljivih vrednostih črpanja nad pragom, to je za $a>1$. Ker $a$ v praksi
ni nikoli dosti večji od 3, je frekvenca oscilacij $\omega^{\prime}_r$ v
brezdimenzijskih enotah približno $1/\sqrt{\tau^{\prime}}$. "Ce preidemo
nazaj na prave enote časa, dobimo $\omega_r\simeq \sqrt{A/\tau}$. V tem
primeru je torej frekvenca relaksacijskih oscilacij velikostnega reda
geometrijske sredine med razpadnima konstantama nihanja resonatorja in
atomskega stanja. Primer takega nihanja pri vključitvi laserja kaže slika 
\ref{s5.8}.

Relaksacijske oscilacije so praktično pomembne, ker določajo gornjo mejo
hitrosti, s katero lahko izhodna moč laserja sledi modulaciji črpanja.
Poleg tega imamo pri tej frekvenci resonanco, pri kateri se šum črpanja
ojačano prenaša v šum izhodne moči.

\section{Delovanje v sunkih s preklopom dobrote}

Veliko trenutno moč dobimo iz laserjev, kadar delujejo v kratkih sunkih.
Pogosto tedaj tudi ojačevalno sredstvo črpamo v sunkih. Pri tem se
pojavi težava. Ko ob močnem črpanju obrnjena zasedenost znatno
preseže zasedenost praga (v nestacionarnem stanju je to mogoče), laser
posveti in v kratkem času zasedenost pade nazaj pod prag. "Ce tedaj
črpanje še traja, čez čas zasedenost zopet dovolj naraste in laser
ponovno posveti. To se lahko večkrat ponovi. Razmiki med zaporednimi sunki
so reda velikosti periode relaksacijskih oscilacij, so pa lahko precej
nepravilni. Pri takem režimu delovanja v posameznem sunku ne dobimo
razpoložljive energije črpanja v enem samem lepo oblikovanem sunku, kar
je za vrsto uporab zelo pomembno.

Opisani težavi se je moč izogniti. V sistemu atomov v stanju obrnjene
zasedenosti je shranjena energija, ki se preko stimuliranega sevanja lahko
pretvori v koherenten svetlobni sunek. "Cim večja je stopnja inverzije, tem
več energije je na voljo. Ugotovili pa smo, da obrnjene zasedenosti v
resonatorju z danimi izgubami ni mogoče povečati znatno nad prag. Pri
velikih izgubah je prag visoko in je shranjene energije več. Zato postopamo
takole. Najprej držimo izgube resonatorja velike in ustvarimo veliko
obrnjeno zasedenost. Nato dovolj hitro izgube zmanjšamo. Optično ojačenje
je veliko in energija svetlobe v kratkem času močno naraste. S tem se tudi
obrnjena zasedenost hitro zniža na vrednost močno pod pragom.
Predstavljamo si lahko, da dobimo prvi nihaj relaksacijskih oscilacij, le da
je začetno stanje daleč od stacionarnega in je zato linearni približek
zelo slab. Iz laserja dobimo kratek in zelo močan sunek svetlobe. Energija
sunka je skoraj tolikšna kot je bila energija obrnjene zasedenosti.
Elektrotehniki izgube resonatorjev podajajo z dobroto, to je razmerjem
frekvence lastnega stanja in njegove širine, zato opisano tehniko imenujemo 
{\it preklop dobrote}. Dogajanje kaže slika \ref{s5.9}.

\begin{figure}[tbp]
\label{s5.9} \vskip 10cm
\caption{Obrnjena zasedenost in energija v laserju pri preklopu dobrote}
\end{figure}

Izgube resonatorja je mogoče spreminjati na mnogo načinov. Najenostavneje
je vrteti eno od ogledal. Tedaj dobimo uglašen resonator le v kratkem
trenutku, ko je ogledalo pravokotno na os. Metoda je dokaj uspešna, a
zastarela. Boljši in danes največkrat uporabljani način je, da v
resonator vgradimo elektrooptični ali akustooptični modulator, o katerih
bomo govorili kasneje. Z njimi lahko električno krmilimo izgube.

Kot smo že povedali, nelinearnih laserskih enačb ne moremo analitično
rešiti. Zato najprej napravimo nekaj ocen. Trajanje sunka je odvisno od
hitrosti, s katero se izprazni gornji laserski nivo. To se ne more zgoditi
hitreje kot v nekaj preletih sunka skozi resonator. Trajanje sunka je torej
vsaj nekajkrat $2L/c$, to je za 15 cm dolg resonator vsaj 10 ns.

Ocenimo lahko še tudi hitrost naraščanja na začetku in upadanja na koncu
sunka. Zapišimo najprej še enkrat enačbi za zasedenost in število
fotonov, pri čemer upoštevajmo, da nas zanima le dogajanje v času sunka,
ki je zeko kratek v primeri atomskim razpadnim časom, zato zanemarimo
ustrezni člen v enačbi \ref{5.7}. Navadno je tudi črpanje prešibko, da
bi med sunkom znatno vplivalo na zasedenost, zato lahko člen $rN$
izpustimo. S črpanjem seveda ustvarimo začetno zasedenost $N_{20}$. Tako
nam ostane 
\begin{eqnarray}  \label{5.32}
\frac{d N_2}{d t}&=&-\frac{1}{V}B\hbar\omega g(\omega)\,n\,N_2 \\
\frac{d n}{d t}&=&\frac{1}{V}B\hbar\omega g(\omega)\,n\,N_2 - \frac{2}{\tau}%
\,n\;.
\end{eqnarray}
Na začetku sunka se vrednost $N_2$ ne razlikuje dosti od začetne vrednosti 
$N_{20}$, tako da število fotonov narašča približno eksponentno: 
\begin{equation}  \label{5.33}
n(t)=n_0e^{\frac{1}{V}B\hbar\omega g(\omega)t}= n_0e^{t/\tau_r}\;\;.
\end{equation}
Začetnega števila fotonov ne poznamo, vemo pa, da je velikostnega reda 1,
saj ga dobimo zaradi spontane emisije. Da bo $n$ narastel na znatno
vrednost, recimo več od 10$^{10}$ fotonov, je potreben čas blizu 30 $%
\tau_r $.

Vzemimo za primer neodimov laser. Presek za stimulirano sevanje $\sigma
(\omega )=\frac{1}{c}B\hbar \omega g(\omega )$ (razdelek 4.4) je okoli 10$%
^{-19}$ cm$^{2}$. Naj bo začetna gostota zasedenosti $N_{20}/V=10^{19}/%
\mbox{cm}^{3}$. Tedaj je $\tau _{r}=3$ ns. Proti koncu sunka pade obrnjena
zasedenost precej pod prag in za grobo oceno lahko predpostavimo, da
ojačevanja ni več. "Stevilo fotonov bo tedaj upadalo s
krakterističnim razpadnim časom resonatorja $\tau /2\simeq 2L/c(1-{\cal R%
})^{-1}$. V laserjih s preklopom dobrote je odbojnost izhodnega zrcala
navadno dokaj nizka, recimo 0,5. Pri $L=15\;\mbox{cm}$ je tako $\tau =4$ ns.
Celotno trajanje sunka je v izbranem primeru tako približno 10 ns, pri
čemer traja okoli 100 ns od preklopa dobrote, da sunek zraste iz šuma
spontanega sevanja. Energija sunka je blizu $N_{20}\hbar \omega $, to je pri
aktivnem volumnu 0,5 cm$^{3}$ nekaj desetink joula. Od tod lahko ocenimo
še, da je moč v vrhu sunka velikostnega reda 10 MW.

Iz enačb \ref{5.32} sicer ne moremo izračunati časovnih odvisnosti $%
N_{2}$ in $n$, lahko pa najdemo njuno medsebojno zvezo. Izrazimo iz prve $dt$
in ga postavimo v drugo enačbo. Dobimo 
\begin{equation}
dn=-dN_{2}+\frac{N_{2p}}{N_{2}}dN_{2}\;\;,  \label{5.341}
\end{equation}
kjer smo upoštevali, da je $N_{2p}=2V/(B\hbar \omega g(\omega )\tau )$.
Enačbo brez težav integriramo: 
\begin{equation}
n=N_{20}-N_{2}+N_{2p}\ln \frac{N_{2}}{N_{20}}\;\;.  \label{5.351}
\end{equation}
Vzeli smo, da je na začetku $N_{2}=N_{20}$ in $n=0$. Iz dobljene zveze
lahko najprej izračunamo, kolikšna je končna zasedenost $N_{2k}$. Na
koncu mora biti zopet $n=0$, kar nam da transcendentno enačbo za $N_{2k}$.
Ima obliko $(x/a)=\exp (x-a)$, kjer je $x=N_{2k}/N_{20}$ in $a=N_{20}/N_{2p}$%
. Kadar je začetna zasedenost le malo nad pragom, tudi končna ne pade
dosti pod prag, zato je izraba energije slabša. Pri večjih začetnih
vrednostih $N_{20}$ pa pade končna skoraj na nič. Za $a=2$, na primer,
je $x=0,41$, medtem ko je že pri $a=4$ $x$ le še 0,08. S tem lahko
izračunamo celotno energijo sunka: $W=\hbar \omega (N_{20}-N_{2k})$.

Trenutna moč, ki izhaja iz laserja, je dana s $P=(2\hbar \omega /\tau )n$.
Največja bo v vrhu sunka, ki je določen z $dn/dN_{2}=0$. Ta enačba ima
očitno rešitev pri $N_{2}=N_{2p}$, vrh sunka je torej natanko tedaj, ko
pade zasedenost na prag. Moč je tedaj $P_{max}=(2\hbar \omega /\tau
)[N_{20}-N_{2p}-N_{2p}\ln (N_{20}/N_{2p})]$.

S preklopom dobrote dobimo zelo kratke in močne svetlobne sunke. Vendar je
največja energija omejena. "Ce je začetna obrnjena zasedenost dovolj
velika, postane ojačenje tolikšno, da se svetloba že v enem preletu
aktivnega sredstva močno ojači in izprazni laserski nivo. To omejuje
nadaljne črpanje. Večje moči dosežejo tako, da sunke iz laserja ojačijo
s prehodom skozi enako aktivno snov, kot je v laserju. Z večstopenjskim
ojačevanjem je mogoče doseči zelo velike energije v nanosekundnih sunkih,
do 100 kJ. Da se ojačevalniki med seboj ne motijo, jih je treba ločiti s
Faradayevimi izolatorji.

Naloga: Oceni, kolikšna je dosegljiva zasedenost pri dani dolžini Nd:YAG
paličke.

\section{Uklepanje faz}
\label{chap:Uklepanje}
"Se dosti krajše sunke kot s preklopom dobrote je mogoče dobiti na
povsem drug način, ki je prav presentljiva manifestacija koherentnosti
laserske svetlobe. Naj v laserju niha več nihanj hkrati. Njihove frekvence
so enakomerno razmaknjene za $\Delta \omega =\pi c/L$. Celotno električno
polje v neki točki v laserju je 
\begin{equation}
E(t)=\sum_{m=-N/2}^{N/2}A_{m}e^{i[(\omega _{0}+m\Delta \omega )t+\varphi
_{m}(t)]}\;\;.  \label{5.342}
\end{equation}
$N$ je število vseh vzbujenih nihanj. Upoštevali smo, da ima vsako
nihanje lahko poljubno fazo $\varphi _{m}(t)$, ki je v splošnem predvsem
zaradi zunanjih motenj slučajna funkcija časa. Zaradi tega se tudi
celotno polje slučajno spreminja, kar močno zmanjšuje uporabnost
takega laserja tam, kjer je potrebna časovna koherenca.

\begin{figure}[tbp]
\label{s5.10} \vskip 5cm
\caption{"Casovna odvisnost moči mnogofrekvenčnega laserja z enakimi
fazami }
\end{figure}

Denimo, da so faze vseh nihanj enake. Poleg tega zaradi enostavnosti
računa privzemimo še, da so tudi vse amplitude $A_{m}$ enake. Tedaj
postane vsota \ref{5.342} geometrijska in jo lahko brez težav seštejemo: 
\begin{equation}
E(t)=A_{0}e^{i\omega _{0}t}\frac{\sin (N\Delta \omega t/2)}{sin(\Delta
\omega t/2)}\;\;.  \label{5.352}
\end{equation}
Trenutna moč izhodne svetlobe ima časovno odvisnost 
\begin{equation}
P(t)=P_{0}\frac{\sin ^{2}(N\Delta \omega t/2)}{\sin ^{2}(\Delta \omega t/2)}
\label{5.36}
\end{equation}
in jo kaže slika \ref{5.10}. Predstavlja periodično zaporedje sunkov, ki
si slede s periodo $T=2\pi /\Delta \omega =2L/c$, kar je enako času obhoda
svetlobe v resonatorju. Konstanta $P_{0}$ je moč posameznega nihanja.
Moč v vrhu sunka je tako $N^{2}P_{0}$, povprečna moč pa $NP_{0}$.
Računsko je pojav enak kot uklon na mrežici in lahko rečemo, da imamo
opravka z interferenco v času. Dolžina sunkov je 
\begin{equation}
\tau _{ML}=\frac{T}{N}=\frac{2\pi }{N\Delta \omega }=\frac{2\pi }{\Delta
\omega _{G}}\;\;,  \label{5.37}
\end{equation}
ker je $N$ ravno število nihanj znotraj širine ojačevanja $\Delta
\omega _{G}$. Dolžina sunka je torej obratno sorazmerna s širino
ojačevanja aktivnega sredstva.

Premislimo še, kakšna je prostorska odvisnost električnega polja v
resonatorju. Polje na danem mestu opisuje enačba \ref{5.352}. "Se
krajevno odvisnost dobimo, če v \ref{5.352} zamenjamo $t$ s $(t-z/c)$. To
pa predstavlja svetlobni paket, ki potuje sem in tja med ogledali
resonatorja. Na izhodnem ogledalu se ga vsakič nekaj odbije, nekaj pa gre
ven iz resonatorja (slika \ref{5.11}). Razmik med sunki, ki izhajajo iz
resonatorja, je $2L$, prostorska dolžina posameznega sunka pa $\tau
_{ML}c=2L/N$.

\begin{figure}[tbp]
\label{5.11} \vskip 5cm
\caption{Prostorska odvisnost fazno uklenjenih sunkov.}
\end{figure}

V našem računu predpostavka, da so vse amplitude $A_{m}$ enake, ni prav
nič bistvena za osnovne ugotovitve. "Ce vzamemo realističen primer, da
so amplitude oblike $A_{m}=A_{0}\exp [(m\Delta \omega /\Delta \omega
_{G})^{2}]$, vsote \ref{5.342} ne znamo točno sešteti, lahko pa jo
približno pretvorimo v integral, ki je Fourierova transformacija Gaussove
funkcije (Pri prehodu z diskretne vsote na integral seveda izgubimo
periodičnost zaporedja sunkov.). Ta je zopet Gaussova funkcija, katere
širina je obratna vrednost širine prvotne funkcije, prav podobno, kot
smo dobili zgoraj. Odvisnost amplitud nihanj od $m$ vpliva torej le na
točno obliko sunkov, osnovne ugotovitve pa se ne spremene. (Naloga).

Pač pa je predpostavka, da so vse faze $\varphi_m$ enake, bistvena. V naši
dosedanji sliki mnogofrekvenčnih laserjev so resonatorska stanja med seboj
neodvisna, zato so faze poljubne in se zaradi motenj lahko še spreminjajo.
Da dobijo za vsa nihanja isto vrednost, moramo poskrbeti posebej. Tako {\it %
uklepanje faz} je mogoče doseči na več načinov. Ena možnost je, da
moduliramo izgube resonatorja s frekvenco, ki je ravno enaka razliki
frekvenc med resonatorskimi stanji. To ni težko razložiti. Naj bo
modulator tak, da je večino časa zaprt, le v razmikih $T=2L/c$ naj bo
kratek čas odprt. Postavimo ga tik ob eno ogledalo. Tedaj se v resonatorju
očitno lahko uspešno ojačuje le kratek sunek, kakršen je na sliki \ref
{***}. Izgube za vsa nihanja bodo majhne le tedaj, kadar bodo vse faze
enake. V praksi ni potrebno, da je modulacija tako izrazita. Običajno
zadošča sinusna modulacija izgub, kjer je relativna prepustnost v minimumu
za nekaj destink manjša od maksimalne.

Kako pri modulaciji pride do uklepanja faz, lahko uvidimo še drugače.
Modulacija amplitude posameznega nihanja povzroči, da se v spektru nihanja
pojavita še stranska pasova pri frekvencah $\omega_m \pm \Delta\omega$. Ta
se ravno pokrivata z obema sosednjima nihanjema in se konstruktivno
prištejeta, če imata enako fazo. S tem pa so tudi izgube manjše in ima
delovanje laserja z uklenjenimi fazami najnižji prag. Zadnji razmislek nam
tudi pove, da ni dobra le amplitudna modulacija, temveč tudi fazna (ali
frekvenčna), saj se tudi tedaj pojavijo stranski pasovi.

Za modulacijo se najpogosteje uporabljajo akustooptični modulatorji, pri
katerih izkoriščamo uklon svetlobe na stoječih zvočnih valovih v
primernem kristalu (Glej 7. poglavje). Frekvenca zvočnega vala mora biti
enaka polovici zahtevane modulacijske frekvence, za 1,5~m dolg laser torej
50~MHz.

Poleg opisanega aktivnega postopka je mogoče faze ukleniti tudi tako, da v
resonator postavimo plast barvila, ki močno absorbira svetlobo laserja pri
majnhi gostoti toka, pri veliki gostoti toka pa pride do nasičenja
absorpcije (glej razdelek 4.5), zato postane barvilo prozorno. Na začetku
imamo v laserju predvsem spontano sevanje, ki se pri enem prehodu skozi
aktivno snov deloma ojači. Barvilo najmanj absorbira največjo fluktuacijo.
Pri dovolj velikem ojačenju bo ta rastla in spet dobimo fazno uklenjeni
sunek. Ker mora po prehodu sunka absorpcija v barvilu zopet hitro narasti,
mora biti relaksacijski čas barvila zelo kratek, v območju pikosekund.

Z uklepanjem faz je danes mogoče dobiti sunke z dolžino pod 100~fs ($%
10^{-13}$~s). Tak sunek traja le še nekaj deset optičnih period. S
posebnimi prijemi jih lahko še skrajšajo na okoli 10~fs. "Ce je potrebna
večja energija sunkov, jih ojačijo, kar ne pokvari mnogo osnovnega sunka.
Zelo kratke svetlobne sunke danes na široko uporabljajo za študij hitre
molekularne dinamike in kratkoživih vzbujenih elektronskih stanj v
polvodnikih in mnogih drugih snoveh. Z njimi se je časovna ločljivost
povečala za nekaj redov velikosti \cite{pikosekunde}.

\section{*Stabilizacija frekvence laserja na na\-sičeno absorpcijo}

V drugem razdelku smo videli, da je efektivna spektralna širina
eno\-frekvenčnega laserja odvisna od fluktuacij dolžine optične poti
svetlobe pri preletu resonatorja. Na to lahko poleg spreminjanja
geometrijske dolžine vpliva še spreminjanje lomnega količnika. "Ce se
posebej ne potrudimo, laser sveti nekje blizu vrha ojačevalnega pasu, pri
čemer frekvenca pleše za znaten del razmika med resonatorskimi stanji. V
šolskem He-Ne laserju je to na primer nekaj deset MHz.

Bistveno manjšo širino lahko dosežemo z aktivno stabilizacijo dolžine
resonatorja. Pri tem je pogalvitni problem, kako dobiti primerjalni
standard. S stabilizacijo na pomožni interferometer, ki smo jo na kratko
opisali v drugem razdelku, lahko dobimo zelo ozko črto, ki pa ima le toliko
natančno določeno frekvenco, kot poznamo dolžino interferometra. Včasih,
na primer za natančna interferometrična merjenja dolžin, s tem nismo
zadovoljni in potrebujemo drug, absoluten standard.

Tak standard za frekvenco so ozki prehodi v primernem razredčenem plinu.
Vendar naletimo na težavo. Zaradi Dopplerjevega pojava so absorpcijske
črte močno razširjene. Pomaga nam pojav nasičenja absorpcije, o katerem
smo govorili v razdelku 4.9. Tam smo videli, da se pri dvakratnem prehodu
monokromatskega snopa svetlobe skozi plin v nasprotnih smereh pojavi v
sredini Dopplerjevo razširjene črte vdolbina, ki ima obliko homogeno
razširjene črte. Homogena širina je lahko mnogo manjša od Dopplerjeve in
je zato vdolbina uporabna kot frekvenčni standard.

V laserski resonator postavimo poleg aktivnega sredstva še celico s
primernim plinom, ki ima absorpcijsko črto v bližini vrha ojačenja
aktivnega sredstva. Za He-Ne laser pri 633 nm so to na primer pare ioda.
Zaradi absorpcije se povečajo izgube v laserju in izhodna moč se zmanjša.
Spreminjajmo sedaj dolžino resonatorja in s tem frekvenco laserja. Ko se ta
približa na homogeno širino centru absorpcijske črte pri $\omega_0$, se
absorpcija zmanjša in s tem se moč laserja poveča. Odvisnost moči
laserja z absorberjem od dolžine kaže slika \ref{s5.12}. Povečanje moči
v vrhu običajno ni prav veliko, manj od procenta.

\begin{figure}[tbp]
\label{s5.12} \vskip 5cm
\caption{Odvisnost moči laserja z nasičenim absorberjem od frekvence}
\end{figure}

Shema stabilizacije na nasičeno absorpcijo je prikazana na sliki \ref{****}%
. Eno od zrcal resonatorja je na piezoelektričnem nosilcu. Nanj vodimo
izmenično napetost s frekvenco $\Omega$ in s tem moduliramo frekvenco
laserja, da se vozi preko absorpcijske vdolbine pri $\omega_0$. Zaradi tega
se spreminja tudi izhodna moč laserja, ki jo opazujemo s fotodiodo. Kadar
je srednja frekvenca laserja enaka $\omega_0$, se moč zmanjša simetrično
pri odmikih navzgor in navzdol od $\omega_0$ in se zato spreminja z dvojno
frekvenco modulacije $2\Omega$. Kadar pa je srednja frekvenca laserja
nekoliko odmaknjena od $\omega_0$, se izhodna moč pri odmiku v zrcala v eno
stran spremeni drugače kot v drugo, kar pomeni, da je v signalu s fotodiode
tudi komponenta s frekvenco $\Omega$. Da držimo srednjo frekvenco laserja
enako $\omega_0$, moramo torej meriti komponento izhodne moči pri
modulacijski frekvenci in s povratno zanko skrbeti, da je ta enaka nič.

\begin{figure}[tbp]
\label{s5.13} \vskip 7cm
\caption{Shema stabilizacije laserja na nasičeno absoprcijo}
\end{figure}

Komponento signala s frekvenco $\Omega$ zaznamo s faznim detektorjem, ki
deluje tako, da signal množi z refenčno modulacijsko napetostjo. V
produktu dobimo istosmerno komponento, ki je sorazmerna signalu pri
frekvenci $\Omega$ in ki jo izločimo z nizkopasovnim filtrom. Izhod iz
faznega detektorja je tako sorazmeren odmiku srednje frekvence laserja od $%
\omega_0$. Preko primernega ojačevalnika ga vodimo na piezoelektrični
nosilec zrcala in tako popravljamo dolžino laserja.

Napravimo kvantitativno oceno opisane stabilizacijske sheme. Odvisnost
izhodne moči od frekvence laserja $\omega$ lahko približno zapišemo v
obliki 
\begin{equation}  \label{5.40}
P(\omega)=P_0 + \frac{P_1\gamma^2}{(\omega- \omega_0)^2+\gamma^2}\;\;.
\end{equation}
$P_0$ je moč laserja brez saturacijskega vrha pri $\omega_0$, $P_1$ pa
povečanje moči pri $\omega_0$. Predpostavili smo, da se ojačenje laserja
in nehomogeno razširjeni del absorpcije ne spreminjata mnogo preko homogene
širine absorberja in je zato $P_0$ približno konstantna. Frekvenco laserja
moduliramo: 
\begin{equation}  \label{5.41}
\omega=\omega_0+\Delta\omega+a \sin \Omega t\;\;.
\end{equation}
Z $\Delta\omega$ smo označili odstopanje srednje frekvence laserja od
centra absorpcijske črte $\omega_0$. "Ce sta $a$ in $\Delta\omega$ majhna v
primeri s homogeno širino $\gamma$, lahko imenovalec v enačbi \ref{5.40}
razvijemo: 
\begin{equation}  \label{5.42}
P(\omega)=P_0+P_1 [1-\frac{1}{\gamma^2}\Delta\omega^2 +\frac{2}{\gamma^2} a
\Delta\omega \sin\Omega t - \frac{a^2}{\gamma^2}\sin^2\Omega t ]\;\;.
\end{equation}
Amplituda signala pri $\Omega$ je $2P_1 a \Delta\omega/\gamma^2$. Najmanjša
razlika, ki jo lahko zaznamo, je določena s šumom meritve. Kot bomo videli
v poglavju o detekciji svetlobe, je osnovni izvor šuma fotodiode Poissonov
šum števila parov elektron-vrzel, ki nastanejo zaradi fotoefekta v p-n
spoju. Najmanjša sprememba svetlobne moči, ki jo lahko izmerimo, je (glej
10??. poglavje) 
\begin{equation}  \label{5.43}
P_N\simeq \sqrt{\hbar\omega P \frac{1}{\tau}}\;,
\end{equation}
kjer je $P$ celotna svetlobna moč, ki vpada na diodo, $\tau$ pa čas
meritve, ki je v našem primeru določen s časovno konstanto nizkopasovnega
filtra na izhodu faznega detektorja.

Vzemimo na primer He-Ne laser, stabiliziran na iodove pare. Povprečna moč
laserja $P_0$ naj je 10 mW in $P_1=0.1 $ mW. "Sirina absorpcijske črte $%
\gamma= 10^6$ s$^{-1}$. Izberimo amplitudo modulacije $a=10^5$ s$^{-1}$ in $%
\tau=10^{-4}$ s. "Casovna konstanta $\tau$ ne sme biti prevelika, določa
namreč, kako hitro popravljamo dolžino laserja. Gornje vrednosti nam dajo
za najmanjšo zaznavno moč pri $\Omega$ $P_N=0.5\times10^{-8}$ W.
Najmanjše merljivo odstopanje frekvence laserja je tedaj 
\begin{equation}  \label{5.44}
\Delta\omega_N=\frac{P_N \gamma^2}{2P_1 a}=2.5\times 10^3\;s^{-1}\;.
\end{equation}
Takšno in še boljšo stabilnost frekvence tudi zares dosežejo. Pozoren
bralec bo opazil, da je $\Delta\omega_N<0.01 \gamma$, to je, položaj
absorpcijskega vrha je na opisan način mogoče določiti z natančnostjo
nekaj tisočink celotne širine.

Na absorpcijsko črto stabiliziranega laserja navadno ne uporabljamo
direktno, temveč z njim kontroliramo drug laser. Del izhodne svetlobe iz
obeh laserjev zmešamo na detekcijski fotodiodi. V signalu dobimo utripanje,
ki je enako razliki frekvenc obeh laserjev. S spreminjanjem dolžine drugega
laserja skrbimo, da je frekvenca utripanja konstantna. Na ta način lahko v
ozkem frekvenčnem intervalu še spreminjamo frekvenco drugega laserja.

Z merjenjem utripanja med dvema stabiliziranima laserjema ugotavljajo tudi
njihovo stabilnost.

\section{*Absolutna meritev frekvence laserja in definicija metra}

Najnatančneje merljiva količina je čas odnosno frekvenca. Frekvence
laserja, ki sveti v vidnem področju seveda ni mogoče direktno prešteti.
Pač pa je v začetku sedemdesetih let uspelo s heterodinsko tehniko, ki se
v mikrovalovni tehniki pogosto uporablja, napraviti primerjavo
stabiliziranega He-Ne laserja z osnovno cezijevo uro in tako določiti
frekvenco absorpcijske črte metana pri 3.39 $\mu$m z isto natan"nostjo, kot
jo ima cezijeva ura.

S heterodinsko tehniko primerjamo frekvenci dveh ali več valovanj tako, da
jih zmešamo na primernem nelinearnem elementu, običajno neki diodi. Zaradi
nelinearnosti dobimo v odzivu diode različne mnogokratnike vpadnih
frekvenc, njihove vsote in razlike. Od teh je kaktera lahko dovolj nizka, da
jo lahko direktno preštejemo.

Za primerjanje frekvenc nad mikrovalvnim področjem je potreben ustrezen
mešalni element. Polvodniške diode nehajo biti uporabne pri približno 20
GHz. Za višje frekvence uporabijo diode kovina-izolator-kovina, ki jih
sestavljajo oksidirana površina niklja, ki se je dotika ostra volframska
konica. Taka dioda deluje kot uporaben mešalni element do frekvenc okoli
200 THz, to je skoraj do vidnega področja.

\begin{figure}[tbp]
\label{s5.14} \vskip 15cm
\caption{Primerjalna veriga za meritev frekvence He-Ne laserja}
\end{figure}

Za primerjavo He-Ne laserja, stabiliziranega na metan pri 3.39 $\mu$m, z
osnovno cezijevo uro je bilo potrebno zgraditi celo verigo vmesnih
primerjav, ki jo kaže slika \ref{*****}. Frekvenco CO$_2$ laserja dobimo na
primer iz utripanja med frekvencama CO$_2$ laserja pri 10.2 $\mu$m in pri
9.3 $\mu$m, trikratnikom frekvence HCN laserja in še klistrona s frekvenco
20 GHz. Na ta način so izmerili, da je frekvenca CH$_4$ črte na katero je
stabiliziran He-Ne laser, 88.376181627 THz.

Valovno dolžino laserja dobimo z interferometrično primerjavo z
dolžinskim standardom, ki je bil do leta 1984 določen z neko kriptonovo
črto. Iz znane frekvence in valovna dolžine določimo hitrost svetlobe.
Zaradi relativno velike širine črte kriptonove svetilke je bil po starem
meter definiran le z relativno natančnostjo 10$^{- 8}$, kar je pomenilo, da
tudi hitrost svetlobe ne more biti določena bolj natančno. Meritev
frekvence laserja pa je dosti natančnejša. Zato je bilo smiselno opustiti
meter kot osnovno enoto in raje definirati hitrost svetlobe kot pretvornik
med sekundo in metrom. Z njeno vrednost so vzeli, kar so dobili z naboljšo
primerjavo stabiliziranega laserja in kriptonove črte: $c=299 792
458\;m/s\;. $ Na metan ali jod stabilizirani laser je postal sekundarni
standard za dolžino. Laser je pravzaprav pri tem le pomožna naprava;
standard je ustrezni molekularni prehod.

\section{*Semiklasični model laserja}

Doslej smo laserje obravnavali le z modelom zasedbenih enačb. Ta je zelo
grob, saj smo zanemarili nekaj pomembnih pojavov. Svetlobo v resonatorju smo
opisali smo opisali le s celotno energijo ali številom fotonov in se za
njeno valovno naravo nismo menili. Kar privzeli smo, da je frekvenca
delujočega laserja in oblika polja v njem enaka kot za lastno stanje
praznega resonatorja. Aktivno snov smo opisali le z zasedenostjo zgornjega
in spodnjega laserskega stanja in smo s tem izpustili možnost, da se zaradi
sodelovanja z elektromagnetnim poljem atomi nahajajo v nestacionarnem,
mešanem stanju.

Gornje pomanjkljivosti odpravimo s tem, da elektromagnetno polje v
resonatorju obravnavamo z valovno enačbo, za atome aktivne snovi pa
upoštevamo, da se pokoravajo Schroedingerjevi enačbi. S tem dobimo {\it %
semiklasični model} laserja. Za še natančnejši opis pa moramo tudi
svetlobo obravnavati kvantno, kar presega okvir te knjige.

Aktivna snov naj bo še naprej kar najenostavnejša, to je množica enakih
dvonivojskih atomov s stanji $|1\rangle$ in $|2\rangle$, ki imata energiji $%
W_1$ in $W_2$. Atomi s svetlobo sodelujejo preko dipolne interakcije oblike $%
e\hat{x}E(t)$, kjer je $E(t)$ polje v resonatorju, ki naj bo zaradi
preprostosti polarizirano v smeri osi $x$. "Casovno odvisno stanje atomov
zapišimo v obliki 
\begin{equation}  \label{5.45}
|\psi\rangle=c_1(t)|1\rangle\exp(-iW_1t/\hbar)+
c_2(t)|2\rangle\exp(-iW_2t/\hbar)\;.
\end{equation}
Iz Schroedingerjeve enačbe dobimo za koeficienta $c_1(t)$ in $c_2(t)$ 
\begin{equation}  \label{5.46}
\dot{c}_1=\frac{1}{i\hbar}\,E(t)v_{12}e^{-i\omega_0 t}c_2 \dot{c}_2=\frac{1}{%
i\hbar}\,E(t)v_{12}e^{i\omega_0 t}c_1\;,
\end{equation}
kjer je $\omega_0=(W_2-W_1)/\hbar$ in $v_{12}=e\langle 1|\hat{x}||2\rangle$.

Električni dipolni moment atoma v stanju $ket{\psi}$ je 
\begin{equation}  \label{5.47}
p=-e\langle\psi|\hat{x}|\psi\rangle=-(c_1^{\ast}c_2e^{-i \omega_0
t}+c_1c_2^{\ast}e^{i \omega_0 t}) v_{12}\;.
\end{equation}
Razdelimo $p$ na dva dela: 
\begin{equation}  \label{5.48}
p=p^++p^-=v_{12}[\eta(t)+\eta^{\ast}(t)]\;,
\end{equation}
kjer smo vpeljali $\eta(t)=c_1^{\ast}c_2e^{-i \omega_0 t}$.

Zanima nas, kako se dipolni moment spreminja s časom. Zato s pomočjo
enačb \ref{5.46} izrazimo 
\begin{equation}  \label{5.49}
\dot{\eta}=- i \omega_0\eta-\frac{1}{i\hbar}\,E(t)v_{12} (|c_2|^2-|c_1|^2)\;.
\end{equation}
$|c_i|^2$ je verjetnost za zasedenost stanja $|i\rangle$. Izraz v oklepaju
na desni strani gornje enačbe torej meri razliko zasednosti obeh stanj;
označimo ga z $\zeta$. Podobno kot zgoraj izrazimo časovni odvod 
\begin{equation}  \label{5.50}
\dot{\zeta}=\frac{2v_{12}}{i\hbar}\,E(t)(\eta^{\ast}- \eta)\;.
\end{equation}
S tem smo iz Schroedingerjeve enačbe dobili enačbe za časovni razvoj
diponega momenta in obrnjene zasedenosti, ki pa jih moramo še dopolniti.
Naj bo atom na začetku v stanju $|2\rangle$ in naj bo $E(t)=0$. Začetna
vrednost $\zeta(0)=1$ in po enačbi \ref{5.50} naj bi bila $\zeta(t)$
konstantna. Vemo pa, da se atom, ki je v vzbujenem stanju, sčasoma vrne v
osnovno stanje. Verjetnost za prehod na časovno enoto smo označili z $A$.
Poleg tega moramo na nek način upoštevati še črpanje, s katerim
vzdržujemo obrnjeno zasedenost in s tem lasersko delovanje. Za podroben
opis črpanja bi morali v Hamiltonov operator dodati ustrezne člene in
morda upoštevati še druga stanja atomov, vendar nas take podrobnosti na
tem mestu ne zanimajo. Zaradi črpanja stacionarna vrednost $\zeta$ v
odsotnosti laserskega polja $E(t)$ ni -1, temveč zavzame neko vrendost $%
\zeta_0$ med -1 in 1, odvisno od moči črpanja. Tako lahko enačbo \ref
{5.50} popravimo: 
\begin{equation}  \label{5.51}
\dot{\zeta}=A(\zeta_0-\zeta)+ \frac{2v_{12}}{i\hbar}\,E(t)(\eta^{\ast}-\eta)%
\;,
\end{equation}
kjer prvi člen popisuje spontane prehode v nižje stanje in vpliv črpanja.

Podobno dopolnimo še enačbo \ref{5.49}. Pri $E(t)=0$ da časovno odvisnost 
$\eta$ oblike $e^{-i \omega_0 t}$, to je brez dušenja. Vema pa, da
polarizacija v mešanem stanju razpada vsaj zaradi spontanega sevanja, lahko
pa še zaradi drugih vplivov, na primer trkov z drugimi atomi. Označimo
koeficient dušenja polarizacije z $\gamma$, ki meri tudi spektralno širino
svetlobe, ki jo sevajo atomi pri prehodu $2\rightarrow 1$. Tako imamo 
\begin{equation}  \label{5.52}
\dot{\eta}=- (i \omega_0\eta+\gamma)- \frac{1}{i\hbar}\,E(t)v_{12} \zeta \;.
\end{equation}
Tej enačbi moramo dodati še konjugirano kompleksno enačbo. Enačbe \ref
{5.51} in \ref{5.52} pogosto imenujejo Blochove enačbe. Najprej so jih
uporabili za obravnavo jedrske magnetne resonance.

Potrebujemo še enačbo za polje $E(t)$. Zanj dobimo iz Maxwellovih enačb
valovno enačbo, kjer moramo upoštevati, da imamo tudi od nič različno
polarizacijo snovi, ki je v primeru, da so vsi atomi enakovredni, podana z 
\begin{equation}  \label{5.53}
P=\frac{N}{V}\,v_{12}(\eta+\eta^{\ast})=P^++P^-\;.
\end{equation}
Valovna enačba je tedaj \cite{empolje} 
\begin{equation}  \label{5.54}
\nabla^2 E-\frac{1}{c^2}\ddot{E}=\mu_0 \ddot{P}\;.
\end{equation}

Namesto mikroskopske količine $\zeta$ lahko uvedemo še gostoto obrnjene
zasedenosti $Z=(N/V)\zeta$, pa lahko enačbi \ref{5.51} in \ref{5.52}
prepišemo v obliko 
\begin{eqnarray}  \label{5.56}
\dot{P}^{\pm}&=&(\mp i \omega_0-\gamma)P^{\pm}+ \frac{v_{12}^2}{i\hbar} E\,Z
\\
\dot{Z}&=&A(Z_0-Z)-\frac{2}{i\hbar}E(P^--P^+)\;.
\end{eqnarray}
Prehod od enačb \ref{5.51} in \ref{5.52} na \ref{5.55} je mogoč le, kadar
so vsi atomi enakovredni, to je, kadar ni nehomogene razširitve. Kako je v
primeru nehomogene razširitve, si bralec lahko ogleda v \cite{haken2}.

Enačbe \ref{5.51}, \ref{5.52} ali \ref{5.55}, skupaj z \ref{5.54} dajejo
semiklasični opis sodelovanja svetlobe in snovi. Iz izpeljave je vidno, da
je v njem spontano sevanje obravnavano pomankljivo, le s fenomenološkim
na\-stavkom, kar je moč popraviti tako, da tudi elektromagnetno polje
kvantiziramo. Kljub tej pomanjkljivosti je s semiklasičnim modelom mogoče
zelo podrobno obravnavati večino pojavov v laserjih in tudi druge probleme
širjenja svetlobe po snovi. Reševanje zapisanega sistema nelinearnih
parcialnih diferencialnih enačb pa je v splošnem zelo težavno.

Da bomo semiklasične enačbe le nekoliko pobliže spoznali, na kratko
poglejmo najenostavnejši primer, to je laser, v katerem je vzbujeno le eno
resonatorsko stanje. Polje ima tedaj obliko 
\begin{equation}  \label{5.57}
E(\vec{r},t)=E_{\lambda}(t)u_{\lambda}(\vec{r})\;,
\end{equation}
kjer je $u_{\lambda}(\vec{r})$ krajevni del lastnega stanja resonatorja, ki
zadošča enačbi 
\begin{equation}  \label{5.58}
\nabla^2 u_{\lambda}- \frac{\omega_{\lambda}^2}{c^2}u_{\lambda}=0\;.
\end{equation}
$E_{\lambda}(t)$ opisuje časovno odvisnost, ki je za laser v stacionarnem
delovanju periodična, vendar frekvenca ni nujno kar enaka lastni frekvneci
praznega resonatorja $\omega_{\lambda}$, temveč jo moramo še izračunati.

Tudi polarizacijo lahko razvijemo po lastnih funkcijah $u_{\lambda}(\vec{r})$%
. Ker so te med seboj ortogonalne, preide valovna enačba \ref{5.54} v 
\begin{equation}  \label{5.59}
\omega_{\lambda}^2 E_{\lambda}-\ddot{E_{\lambda}}= \frac{1}{\epsilon_0}\ddot{%
P}_{\lambda}\;.
\end{equation}

Razstavimo $E_{\lambda}(t)$ na dva dela: 
\begin{equation}  \label{5.60}
E_{\lambda}(t)=E_{\lambda}^+(t)+E_{\lambda}^-(t)=A^+(t)e^{-i
\omega_{\lambda}t}+A^-(t)e^{i \omega_{\lambda}t}\;.
\end{equation}
Dejanska frekvenca laserja je blizu $\omega_{\lambda}$, zato pričakujemo,
da bosta amplitudi $A^{\pm}(t)$ v primerjavi z $e^{-i \omega_{\lambda}t}$ le
počasni funkciji časa. Izračunajmo 
\begin{eqnarray}  \label{5.61}
\ddot{E}_{\lambda}^+&=&-\omega_{\lambda}^2 E_{\lambda}^+-2i \omega_{\lambda} 
\dot{A}^+ e^{-i \omega_{\lambda}t} + \ddot{A}^+ e^{-i \omega_{\lambda}t} 
\nonumber \\
&\simeq&-\omega_{\lambda}^2 E_{\lambda}^+-2i \omega_{\lambda}(\dot{E}%
_{\lambda}^++i \omega_{\lambda}E_{\lambda}^+)
\end{eqnarray}
V drugi vrstici smo izpustili člen z $\ddot{A}^+$, ker pričakujemo, da je
majhen. S tem smo napravili približek {\it počasne amplitude}.

Polarizacija snovi je približno periodična s frekvenco $\omega_0$, z
amplitudo, ki je tudi počasna funkcija časa. Zato je $\ddot{P}%
_{\lambda}^+\simeq- \omega_0^2 P_{\lambda}^+$. Pri drugem odvodu polja po
času smo potrebovali en člen več, ker se člen $-\omega_{\lambda}^2
E_{\lambda}^+$ na levi strani enačbe \ref{5.54} odšteje. Z uporabo tega
približka in enačb \ref{5.58} in \ref{5.61} preide valovna enačba \ref
{5.54} za eno nihanje v 
\begin{equation}  \label{5.62}
\dot{E}_{\lambda}^+=-i \omega_{\lambda} E_{\lambda}^++\frac{i \omega_0}{%
2\epsilon_0}P_{\lambda}^+\;.
\end{equation}

Doslej nismo upoštevali, da je polje v praznem resonatorju dušeno, zato
moramo gornjo enačbo še popraviti: 
\begin{equation}  \label{5.63}
\dot{E}_{\lambda}^+=(-i \omega_{\lambda}-\frac{1}{\tau}) E_{\lambda}^++\frac{%
i \omega_0}{2\epsilon_0}P_{\lambda}^+\;.
\end{equation}
Kadar v reosnatorju ni snovi, je dobljena enačba enaka kot enačba \ref
{3.45}.

Enačbi \ref{5.55} in \ref{5.56} sta nelinearni, zato ju n moč kar tako
prepisati za primer razvoja po lastnih stanjih resonatorja. Pri enačbi za
razvoj polarizacije \ref{5.55} imamo v zadnjem členu na desni produkt
komponente polja $E_{\lambda}$ in obrnjene zasedenosti $Z$, od katere
bistveno prispeva le krajevno povprečje $\bar{Z}$, ki se tudi s časom le
počasi spreminja. Seveda vsebuje $Z$ tudi krajevno odvisne komponente, ki
pa so pomembne predvsem zato, ker sklaplajo različna lastna stanja
resonatorja, kar presega našo trenutno obravnavo. Tako imamo 
\begin{equation}  \label{5.64}
\dot{P}_{\lambda}^+=(-i \omega_0- \gamma)P_{\lambda}^++\frac{v_{12}^2} {%
i\hbar}\,E_{\lambda}^+\bar{Z}\;.
\end{equation}

Enačbo za $\bar{Z}$ dobimo iz \ref{5.56}. V zadnjem členu imamo produkte $%
E^{\pm}P^{\pm}=E_{\lambda}^{\pm}P_{\lambda}^{\pm} u_{\lambda}^2(\vec{r})$,
kar moramo prostorsko povprečiti. Funkcije $u_{\lambda}(\vec{r})$ naj so
normalizirane tako, da je $\int u_{\lambda}^2(\vec{r}) \,dV=V$. Tako imamo $%
\overline{u_{\lambda}^2(\vec{r})}=1$ in 
\begin{equation}  \label{5.65}
\dot{\bar{Z}}= A(\bar{Z}_0-\bar{Z})- \frac{2}{i\hbar}(E_{\lambda}^++
E_{\lambda}^-)(P_{\lambda}^- - P_{\lambda}^+)\;,
\end{equation}
kjer je $\bar{Z}_0$ povprečje nenasičene zasedenosti $Z_0$. V zadnjem
členu nastopajo produkti, ki nihajo s frekvencami $\omega_{\lambda}-
\omega_0$ in $\omega_{\lambda}+ \omega_0$. Obe frekvenci sta si zelo blizu,
zato je njuna vsota mnogo večja od razlike. "Cleni $E_{\lambda}^+
P_{\lambda}^+$ in $E_{\lambda}^- P_{\lambda}^-$ se torej zelo hitro
spreminjajo in skoraj nič ne vplivajo na valovanje blizu $\omega_{\lambda}$%
, zato jih izpustimo. S tem je časovna odvisnost $\bar{Z}$ podana z 
\begin{equation}  \label{5.66}
\dot{\bar{Z}}= A(\bar{Z}_0-\bar{Z})- \frac{2}{i\hbar}(E_{\lambda}^+
P_{\lambda}^- - E_{\lambda}^- P_{\lambda}^+)\;.
\end{equation}
Enačbe \ref{5.63}, \ref{5.64} in \ref{5.66}, skupaj s konjugirano
kompleksnimi enačbami za $E_{\lambda}^-$ in $P_{\lambda}^-$, so zaključen
sistem, ki opisuje delovanje enofrekvenčnega laserja. Uporabimo jih za
izračun frekvence izhodne svetlobe.

Naj bo stanje stacionarno. Tedaj lahko polje zapišemo v obliki $E_{\lambda
}^{+}=E_{0}e^{-i\Omega t}$, kjer je $E_{0}$ realna konstanta, frekvenca
svetlobe $\omega $ pa je blizu $\omega _{0}$ in $\omega _{\lambda }$. V
stacionarnem stanju mora imeti polarizacija enako časovno odvisnost: $%
P_{\lambda }^{+}=P_{0}e^{-i\Omega t}$. Tedaj je v enačbi \ref{5.66} drugi
oklepaj konstanten in mora biti tudi $\bar{Z}$ v stacionarnem stanju od
časa neodvisna. Sistem enačb \ref{5.63}, \ref{5.64} in \ref{5.66} nam
tako da 
\begin{eqnarray}
-[i(\omega _{\lambda }-\Omega )+\frac{1}{\tau }]E_{0}-\frac{i\omega _{0}}{%
2\epsilon _{0}}\,P_{0} &=&0  \nonumber  \label{5.67} \\
-[i(\omega _{0}-\Omega )+\gamma ]P_{0}+\frac{v_{12}^{2}}{i\hbar }\,E_{0}\bar{%
Z} &=&0  \nonumber \\
A(\bar{Z}_{0}-\bar{Z})-\frac{2}{i\hbar }\,E_{0}(P_{0}^{*}-P_{0}) &=&0\;.
\end{eqnarray}

Najprej izračunamo $P_0$ iz druge enačbe, ga postavimo v tretjo in
izračunamo $\bar{Z}$: 
\begin{equation}  \label{5.68}
\bar{Z}=\bar{Z}_0\left[1+\frac{v_{12}^2}{\hbar^2 A}\,E_0^2\, \frac{2\gamma}{%
(\omega_0-\Omega)^2+\gamma^2}\right]^{-1}
\end{equation}
Ta izraz že poznamo. $\pi v_{12}^2/(\epsilon_0\hbar^2)$ je Einsteinov
koeficient $B$. $E_0^2$ je sorazmern gostoti energije polja v resonatorju,
zadnji ulomek v oklepaju pa nam podaja obliko homogeno razširjene atomske
črte: 
\begin{equation}  \label{5.69}
\bar{Z}=\bar{Z}_0\left[1+\frac{2B}{A}\,g(\omega_0- \Omega)w\right]^{-1}
\end{equation}
To je natanko enako izrazu za nasičenje zasedenosti stanj, ki smo ga
izpeljali iz zasedbenih enačb v četrtem poglavju.

Postavimo $P_0$ iz prve enačbe sistema \ref{5.67} v drugo: 
\begin{equation}  \label{5.70}
E_0[i(\Omega-\omega_{\lambda})+\frac{1}{\tau}] [i(\Omega- \omega_0)
+\gamma]=-\frac{v_{12}^2 \omega_0}{2\hbar\epsilon_0}\,E_0\,\bar{Z}\;.
\end{equation}
V delujočem laserju je $E_0\ne 0$, zato lahko krajšamo. $\bar{Z}$ je
realen, tako da mora biti imaginarni del leve strani enak nič: 
\begin{equation}  \label{5.71}
(\Omega- \omega_{\lambda})\gamma+(\Omega- \omega_0)\frac{1}{\tau} = 0 \;.
\end{equation}
Od tod lahko izračunamo frekvenco laserja 
\begin{equation}  \label{5.72}
\Omega=\frac{\omega_{\lambda}\gamma+ \omega_0\frac{1}{\tau}}{\gamma + \frac{1%
}{\tau}}\;.
\end{equation}
Frekvenca torej ni enaka frekvenci praznega resonatorja $\omega_{\lambda}$,
temveč je premaknjena proti centru atomske črte $\omega_0$. Premik je
odvisen od razmerja širine atomske črte in izgub resonatorja.

Bralec lahko sam iz enačbe \ref{5.70} izračuna še energijo svetlobe v
resonatorju in rezultat primerja s tistim, ki smo ga dobili z uporabo
zasedbenih enačb.

Gornji primer uporabe polklasičnih enačb je zelo preprost. Prava moč
modela se pokaže pri obravnavi mnogofrekvenčnega laserja, na primer pri
računu uklepanja faz laserskih nihanj, kar pa presega okvir te knjige. Več
bo bralec našel v \cite{haken2}.