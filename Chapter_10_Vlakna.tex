\chapterimage{slike/Fibri.jpg} % Chapter heading image

\chapter{Optična vlakna}
Moderna komunikacijska tehnologija zahteva vedno hitrejši prenos
vedno večje količine podatkov. Navadne kovinske vodnike
so zato v računalniških in telefonskih povezavah nadomestila optična 
vlakna, ki jih odlikujejo majhne izgube, neobčutljivost na elektromagnetne
in medsebojne motnje ter zmožnost prenosa izjemno velike količine podatkov. 
V tem poglavju bomo opisali mehanizme prenosa podatkov po optičnih vodnikih
in spoznali omejitve pri prenosu,
predvsem disperzijo in izgube, ter načine, kako se z njimi 
spopadamo.

\section{Planparalelni vodnik}
\subsection*{Geometrijska razlaga}
\index{Optični vodnik}
\index{Optično vlakno}
\index{Optični vodnik!planparalelni}
Klasično pojasnimo delovanje optičnih vlaken s totalnim odbojem~\index{Totalni odboj}
na meji med dvema dielektrikoma. Kadar prehaja svetloba iz snovi 
z večjim lomnim količnikom\index{Lomni količnik} v sredstvo z manjšim lomnim količnikom,
se pri vpadnih kotih, ki so večji od kritičnega kota, totalno odbije. 
\begin{figure}[h]
\centering
\def\svgwidth{110truemm} 
\input{slike/10_Vodnik.pdf_tex}
\caption{Klasična razlaga valovnega vodnika}
\label{fig:vodnik}
\end{figure}

Najpreprostejši optični vodnik je planparalelna plast 
dielektrika, ki je obdana s snovjo z manjšim lomnim količnikom (slika~\ref{fig:vodnik}). 
Plasti z večjim lomnim količnikom ($n_1$) rečemo sredica\index{Optični vodnik!sredica}, 
okoliški snovi z lomnim količnikom $n_2<n_1$ pa plašč\index{Optični vodnik!plašč}. 
Žarek potuje po vodniku, če je vpadni kot 
na mejo med sredico in plaščem $\vartheta$ večji od kota totalnega odboja $\vartheta_c$, 
za katerega velja
\boxeq{10totalni}{
\sin\vartheta_{c}=\frac{n_{2}}{n_{1}}.
}
Obstaja torej največji vpadni kot $\alpha_{\rm max}$, pod katerim 
svetloba vpada v vodnik in ostane v njem ujeta.
Z njim povezana je numerična odprtina (apertura) vlakna\index{Numerična odprtina}, 
ki jo izračunamo kot 
\begin{equation}
NA = \sin \alpha_{\rm max} = n_1 \sin \beta_{\rm max} = 
n_1 \sin(\pi/2-\vartheta_c) =
n_1 \cos\vartheta_c = n_1 \sqrt{1-\sin^2\vartheta_c}.
\label{10NAa}
\end{equation}
Upoštevajoč enačbo~(\ref{10totalni}) numerično odprtino zapišemo kot 
\boxeq{10NA}{
NA = \sqrt{n_1^2-n_2^2}.
}
Ker je razlika lomnih količnikov v vodnikih razmeroma majhna,
tipično le nekaj stotink, je tudi numerična odprtina optičnih 
vodnikov navadno
$NA \lesssim 0,1 $. Kot, pod katerim lahko vpada svetloba
v vodnik (ali vlakno), da se vanj ujame, je zato zelo majhen, tipično le nekaj stopinj.

\subsection*{Valovni opis}
Za natančen opis širjenja svetlobe po vodnikih ali vlaknih\footnote{Dogovorimo se, da bomo 
besedo vlakno uporabljali za cilindrične strukture, besedo vodnik pa za planparalelne 
strukture in njim podobne.}, ki imajo navadno polmer
sredice od nekaj do nekaj deset mikrometrov, geometrijska optika ne
zadošča. Rešiti moramo Maxwellove enačbe (enačbe~\ref{eq:Maxwell1}--\ref{eq:Maxwell4}) 
z ustreznimi robnimi pogoji (enačbe~\ref{eq:robni-pogoji}--\ref{eq:robni-pogoji5}),
kar je za cilindrična vlakna dokaj dolg in zapleten račun. Zato določimo najprej 
osnovne značilnosti valovanja, ki se širi po planparalelnem vodniku.

Glede na smer polarizacije električne poljske jakosti 
ločimo dva primera (slika~\ref{fig:TETM}). Če je smer električne poljske
jakosti valovanja vzporedna z mejnima ploskvama (smer $y$), 
govorimo o transverzalnem električnem (TE) valovanju\index{Polarizacija!TE}. 
V nasprotnem primeru, ko je 
z mejnima ploskvama vzporedna magnetna poljska jakost in 
leži električna poljska jakost v ravnini $xz$, 
govorimo o transverzalnem magnetnem (TM) valovanju\index{Polarizacija!TM}.
\begin{figure}[h]
\centering
\def\svgwidth{140truemm} 
\input{slike/10_TETM.pdf_tex}
\caption{Polarizaciji TE (levo) in TM (desno) valovanj v valovnem vodniku}
\label{fig:TETM}
\end{figure}

Geometrijskemu žarku, ki pod kotom potuje po sredici in se na njeni meji
odbija, ustreza v valovni sliki val, ki ima prečno komponento valovnega
vektorja\index{Valovni vektor} $k_{x}$ različno od nič. Ker je valovanje v prečni smeri 
omejeno na sredico končne debeline (naj bo debelina sredice enaka $a$), lahko
$k_{x}$ zavzame le diskretne vrednosti. Te so v grobem približno enake $N\pi/a$, pri čemer je $N$
celo število. Pravimo, da vsak $N$ določa en rod valovanj v vodniku. Po drugi strani 
pa obstaja največji $k_x$, za katerega približno velja
\begin{equation}
k_{x \mathrm{max}} \approx k_0 \sin\alpha_{\rm max} = 
k_0 n_1 \cos\vartheta_c = k_0 \sqrt{n_1^2 -n_2^2}.
\end{equation}
Število rešitev za $k_x$ je torej omejeno in točno določeno, odvisno
pa je od razlike lomnih količnikov in od debeline vodnika oziroma polmera vlakna. 
V nadaljevanju bomo spoznali, da v optičnih vlaknih en rod vselej obstaja,
za razliko od dielektričnih in kovinskih vodnikov, kakršne
poznamo iz mikrovalovne tehnike, po katerih se pod določeno frekvenco
valovanje ne more širiti. Enorodovna optična vlakna, torej vlakna, po katerih se širi
en sam rod, imajo še posebej lepe lastnosti za uporabo v komunikacijskih
sistemih\index{Optični vodnik!enorodovni}\index{Optično vlakno!enorodovno}.

Povejmo še nekaj o hitrosti valovanja v vodniku.
Naj bo $\beta$ velikost komponente valovnega vektorja vzdolž smeri $z$. Odvisnost polja
od koordinate vzdolž vodnika je tako $\exp (i\beta z)$. Po drugi strani za velikost
valovnega vektorja v sredici vodnika velja
\begin{equation}
k_1 = n_{1}\frac{\omega}{c_0}=\sqrt{\beta^{2}+k_{x}^{2}}
\label{9.0}.
\end{equation}
Za dano vrednost $k_{x}$ torej zveza med valovnim številom $\beta$
in frekvenco $\omega$ ni linearna. Fazna hitrost\index{Hitrost valovanja!fazna} 
valovanja $v_{f}=\omega/\beta$ je
odvisna od frekvence in pride do pojava disperzije\index{Disperzija}. Grupna 
hitrost\index{Hitrost valovanja!grupna} $v_{g}=d\omega/d\beta$ 
se zaradi nelinearne zveze razlikuje od fazne hitrosti in njena frekvenčna odvisnost 
ima pomembne posledice za uporabo vlaken pri prenosu podatkov. Več o tem bomo spoznali 
v nadaljevanju poglavja. 

\section{Račun lastnih rodov v planparalelnem vodniku}
\index{Optični vodnik!lastni rodovi}
\index{Optični vodnik!planparalelni}
Poiščimo rešitve valovne enačbe v planparalelnem vodniku. 
To je preprost dvodimenzionalen model optičnega vlakna, ki je sestavljen iz 
plasti prozornega dielektrika z lomnim količnikom $n_1$ in plašča z lomnim količnikom $n_2$.
Zaradi enostavnosti privzamemo, da je plašč na obeh straneh sredice enak.
Sredica naj bo debela $a$, izhodišče koordinatnega sistema
si izberemo na sredini plasti. Ločimo tri območja, kjer rešujemo valovno enačbo:
območje II označuje sredico, območji I in III pa sta v plašču nad sredico oziroma pod njo. 

\begin{figure}[h]
\centering
\def\svgwidth{120truemm} 
\input{slike/10_VodnikRacun.pdf_tex}
\caption{K izračunu lastnih rodov v simetričnem planparalelnem vodniku}
\label{fig:vodnikracun}
\end{figure}

Krajevni del valovne enačbe, ki jo rešujemo, opisuje Helmholtzeva enačba\index{Helmholtzeva enačba} 
(enačba~\ref{eq:Helmholtz})
\begin{equation}
\nabla^{2}\mathbf{E}+n^2(x)\,k_{0}^{2}\,\mathbf{E}=0,
\label{9.1}
\end{equation}
kjer je $k_{0}=\omega/c$, $n(x)$ pa nezvezno spremeni vrednost ob prehodu iz sredice v plašč. 
Nastavek za rešitev naj bo oblike 
\begin{equation}
{\mathbf E}(x,z)=\mathbf{e}\psi\left(x\right)\, e^{i\beta z}.
\label{9.2}
\end{equation}
Omejimo se le na primer TE polarizacije\index{Optični vodnik!TE rodovi} (za izračun lastnih rodov 
TM polariziranega
valovanja glej nalogo~\ref{naloga:TM}). Vstavimo nastavek (enačba~\ref{9.2}) v enačbo
(\ref{9.1}) in zapišemo
\begin{equation}
\frac{d^{2}{\bf \psi}}{dx^{2}}+\left(k_{0}^{2}n_1^{2}-\beta^{2}\right){\bf \psi}=0
\qquad \textrm{v sredici oziroma obmo\v cju II} 
\label{9.3a}
\end{equation}
in 
\begin{equation}
\frac{d^{2}{\bf \psi}}{dx^{2}}+\left(k_{0}^{2}n_2^{2}-\beta^{2}\right){\bf \psi}=0
\qquad \textrm{v plašču oziroma obmo\v cjih I in III.} 
\label{9.3b}
\end{equation}
Iz zveze~(enačba~\ref{9.0}) sledi $k_0^2n_1^2-\beta^2=k_x^2$, zato lahko rešitve prve enačbe
zapišemo v obliki
\begin{equation}
\psi_{\mathrm{II}}(x) = C \cos(k_x x)+D \sin(k_x x),
\end{equation}
rešitve v plašču pa so oblike
\begin{equation}
\psi_{\mathrm{I}}(x) = A \exp(-\kappa x)+B \exp(\kappa x),\\
\psi_{\mathrm{III}}(x) = F \exp(-\kappa x)+G \exp(\kappa x),
\end{equation}
pri čemer smo vpeljali $\kappa^2= \beta^2-k_0^2n_2^2$.

Da valovanje ostane ujeto v vlakno, mora biti $\kappa$ realno število.
Le tako namreč dosežemo eksponentno pojemanje električne poljske jakosti 
z oddaljenostjo od sredice,
sicer je valovanje v vseh treh območjih oscilatorno in ni ujeto v vlakno. 

Iz zahteve, da sta $k_x$ in $\kappa$ realna, sledi pogoj za valovno 
število\index{Valovno število} $\beta$
\boxeq{vlaknobeta}{
k_0 n_2 < \beta < k_0 n_1.
}

Poleg tega zahteva po končnosti rešitve da pogoj, da je v območju I (pri $x>a/2$) koeficient $B=0$, 
v območju III (pri $x<-a/2$) pa $F=0$. Hitro ugotovimo, da so zaradi simetrije rešitve
lahko le sode ali lihe funkcije. 

\subsection*{Sode rešitve}
\index{Optični vodnik!sodi rodovi}
Poglejmo najprej sode rešitve. V sredici bo tako različen od nič samo $C$, 
v območjih I in III pa bosta amplitudi enaki in $A = G$ (slika~\ref{fig:TESodi}\,a). 
Rešitev je tako oblike 
\begin{align}
\psi_{\mathrm{I}}(x) =&~ A \exp(-\kappa x), \\
\psi_{\mathrm{II}}(x) =&~ C \cos(k_x x),\\
\psi_{\mathrm{III}}(x) =&~ A \exp(\kappa x).
\end{align}
Zvezo med koeficientoma $A$ in $C$ določimo z upoštevanjem robnih pogojev. Na meji
med sredico in plaščem morata biti tangencialni komponenti 
električne in magnetne poljske jakosti zvezni (enačbi~\ref{eq:robni-pogoji4} in 
\ref{eq:robni-pogoji5}). Iz tega  
izluščimo pogoj, da se za TE valovanje
na meji ohranja amplituda električne poljske jakosti. Pri $x = a/2$ tako zapišemo
\begin{equation}
A \exp(-\kappa a/2) = C \cos(k_x a/2).
\end{equation}
Drugi pogoj dobimo iz zveze $\nabla\times{\bf E}=i\omega\mu_{0}{\bf H}$, ki izhaja
neposredno iz Maxwellove enačbe~(\ref{eq:Maxwell2}). Ker se na meji ohranja
tangencialna komponenta ${\bf H}$, to je v tem primeru $H_z$, se posledično ohranja 
odvod električne poljske jakosti $dE_y/dx$. 
Pri $x = a/2$ tako velja
\begin{equation}
-A \kappa \exp(-\kappa a/2) = -C k_x \sin(k_x a/2).
\end{equation}
Enačba, ki določa rešitev $k_x$, sledi iz zahteve, da sta gornja robna 
pogoja hkrati izpolnjena. Za sode načine sekularno enačbo zapišemo 
kot\index{Sekularna enačba!sodi rodovi}
\boxeq{sekular1}{
\frac{\kappa}{k_x} = \tan \frac{k_x a}{2}.
}
Pri tem je parameter $\kappa$ odvisen od $k_x$ preko 
\boxeq{kappak}{
\kappa^{2}+ k_x^{2}=k_{0}^{2}\left(n_{1}^{2}-n_{2}^{2}\right).
}

\subsection*{Lihe rešitve}
\index{Optični vodnik!lihi rodovi}
Oglejmo si še lihe rešitve v planparalelnem vodniku. V sredici je od nič različen
le $D$, polji v plašču pa sta nasprotno enaki in $A = -G$ (slika~\ref{fig:TESodi}\,b). Sledi
\begin{align}
\psi_{\mathrm{I}}(x) =&~ A \exp(-\kappa x),\\
\psi_{\mathrm{II}}(x) =&~ D \sin(k_x x),\\
\psi_{\mathrm{III}}(x) =&~ -A \exp(\kappa x).
\end{align}
Z upoštevanjem zveznosti električne poljske jakosti in njenega odvoda na 
meji med sredico in plaščem zapišemo robna pogoja pri $x=a/2$
\begin{equation}
A \exp(-\kappa a/2) = D \sin(k_x a/2)
\end{equation}
in 
\begin{equation}
-\kappa A \exp(-\kappa a/2) = D k_x \cos(k_x a/2).
\end{equation}
Ustrezna sekularna enačba za lihe rešitve je\index{Sekularna enačba!lihi rodovi}
 \boxeq{sekular2}{
-\frac{k_x}{\kappa} = \tan \frac{k_x a}{2}.
}
\begin{figure}[h]
\centering
\def\svgwidth{140truemm} 
\input{slike/10_TESodi.pdf_tex} 
\caption{Prečne oblike električne poljske jakosti za sode (a) in lihe (b) rodove v 
simetričnem planparalelnem valovnem vodniku. Modra barva označuje sredico, beli del 
pa plašč vodnika. 
}
\label{fig:TESodi}
\end{figure}

\begin{remark}
Če ne prej, je bralec ob slikah~(\ref{fig:TESodi}) zagotovo opazil podobnost s kvantnim 
delcem, ujetim v končni enodimenzionalni potencialni jami. Svetloba, ujeta v vodnik ali
vlakno, ustreza vezanim stanjem delca, numerična odprtina pa je tisti parameter, 
ki določa globino potencialne jame. Pri majhnih vrednostih obstaja samo ena rešitev 
za vezano stanje, pri globlji jami je rešitev več. Podobno kot v kvantni mehaniki
tudi v tem primeru ena rešitev za vezano stanje vedno obstaja. 
\end{remark}

Sekularnih enačb za lastne rodove (enačbi~\ref{sekular1} in \ref{sekular2}) ne moremo rešiti 
analitično. Rešujemo jih numerično, zelo nazorna pa je tudi grafična predstavitev
(slika~\ref{fig:TEsec}). Najprej narišemo desno stran enačb~(\ref{sekular1}) 
in~(\ref{sekular2}), to je $\tan (k_x a/2)$, v odvisnosti
od $k_xa$ (črna črta). Nato narišemo levi strani enačb, pri čemer $\kappa$ izrazimo iz enačbe~(\ref{kappak}). Rdeča krivulja ustreza sodim rešitvam in modra lihim. 
Presečišča rdeče oziroma modre krivulje s črnimi
označujejo rešitve za $k_xa$. Število presečišč da število rodov, ki se lahko razširjajo po takem vodniku. 

V predstavljenem primeru je rodov za izbrano TE polarizacijo pet: trije sodi in dva liha. 
Z grafa razberemo 
še eno pomembno lastnost. Ne glede na to, kako tanek je vodnik, vedno obstaja vsaj ena rešitev za $k_x$, 
saj rdeča krivulja vedno v vsaj eni točki seka črno. Vodniku, v katerem se širi en sam rod valovanja, 
pravimo enorodovni vodnik (in vlaknu enorodovno vlakno)\index{Optični vodnik!enorodovni}, 
sicer je vodnik večrodoven\index{Optični vodnik!večrodovni}. Za tipično enorodovno vlakno velja 
$a\lesssim 5~\si{\micro\meter}$, za večrodovno z okoli 20 rodovi pa $a\sim 50~\si{\micro\meter}$.
\begin{figure}[h]
\centering
\def\svgwidth{90truemm} 
\input{slike/10_TEsekularna.pdf_tex}
\caption{K izračunu prečnih komponent valovnega vektorja $k_x$ v planparalelnem valovnem vodniku
za TE polarizacijo. V predstavljenem primeru je vodnik petrodoven.}
\label{fig:TEsec}
\end{figure}

Ocenimo število dovoljenih rodov še z izračunom. S slike~(\ref{fig:TEsec}) vidimo, da je največja 
vrednost prečne komponente valovnega vektorja $k_x$, pri kateri valovanje še potuje po vodniku, omejena z vrednostjo, 
pri kateri $\kappa$ pade na nič. Do te vrednosti je po ena rešitev na vsak interval dolžine $\pi/a$, izmenično
soda in liha. Celotno število rodov za eno polarizacijo je tako\index{Optični vodnik!število rodov}
\begin{equation}
N \approx \frac{k_{x\mathrm{max}}}{\pi/a}  = \frac{k_0\, a\, NA }{\pi} = \frac{V}{\pi},
\end{equation}
pri  čemer smo vpeljali normirano frekvenco\index{Normirana frekvenca}
\begin{equation}
V = k_0 a NA.
\end{equation}
Ko enkrat izračunamo dovoljene vrednosti $k_x$, končno poznamo celotno električno poljsko
jakost v vodniku in izven njega. Za primer s slike~(\ref{fig:TEsec}) so lastni rodovi
narisani na sliki~(\ref{fig:TESodi}).

\begin{definition}
\label{naloga:TM}
Ponovi izračun za TM valovanje\index{Optični vodnik!TM rodovi} in pokaži, da sta sekularni enačbi enaki 
\begin{equation}
\frac{\kappa}{k_x} \left(\frac{n_1}{n_2}\right)^2= 
\tan \frac{k_x a}{2} \qquad \mathrm{in} \qquad -\frac{k_x}{\kappa} \left(\frac{n_2}{n_1}\right)^2= 
\tan \frac{k_x a}{2}.
\end{equation}
Namig: Zapiši enačbe za magnetno poljsko jakost ${\bf H}$ in upoštevaj ustrezne robne pogoje.
\end{definition}

Podoben račun lahko naredimo tudi za TM valovanje (glej nalogo~\ref{naloga:TM}). Zaradi drugačnih
robnih pogojev se sekularni enačbi razlikujeta od tistih za TE polarizacijo. Razlika je v
faktorjih $(n_1/n_2)^2$, ki pa so v tipičnem vodniku zelo blizu ena. Zato se tudi rešitve 
za prečno komponento $k_x$ le malo razlikujejo. Bolj pomembna je ugotovitev, da je število
dovoljenih rodov za TM polarizacijo enako številu dovoljenih rodov za TE polarizacijo, 
saj je največji $k_x$ v obeh primerih določen z istim pogojem 
($\kappa = 0$). Celotno število lastnih rodov, ki se širi v danem vodniku, je torej sestavljeno 
iz TE sodih in lihih ter TM sodih in lihih rodov.

Ugotovili smo, da je električna poljska jakost tudi izven sredice vodnika različna od nič. 
Poglejmo še, kako je z energijskim tokom. Čeprav se velika večina pretaka po sredici, 
delež, ki se pretaka po plašču, ni vedno zanemarljiv. To posebej velja za višje rodove. 
Delež energijskega toka, ki se pretaka po sredici, izračunamo\index{Gostota energijskega toka}
z integralom
\boxeq{confinement}{
\Gamma = \frac{\int_{-a/2}^{a/2}j\, dS}{\int_{-\infty}^{\infty}j\, dS}.
}
\begin{definition}
Pokaži, da je razmerje med energijskim tokom, ki se pretaka po plašču, in energijskim tokom, 
ki se pretaka po sredici vodnika, za sode rodove enako
\begin{equation}
\Gamma = \frac{n_2}{n_1}\frac{2 k_x}{\kappa} \frac{\cos^2(k_x a/2)}{k_xa + \sin(k_xa)},
\end{equation}
in za lihe rodove
\begin{equation}
\Gamma = \frac{n_2}{n_1}\frac{2 k_x}{\kappa} \frac{\cos^2(k_x a/2)}{k_xa - \sin(k_xa)}.
\end{equation}
\end{definition}

\section{Cilindrično vlakno}
\label{chap:Cilinder}
\index{Optično vlakno}
Do zdaj smo obravnavali ravninski valovni vodnik. V praksi svetlobo
navadno usmerjamo po optičnih vlaknih, ki imajo cilindrično geometrijo.
Najpreprostejša struktura, ki je analogna primeru planparalelne
plasti, je cilindrično vlakno, v katerem je lomni količnik cilindrične
sredice konstanten in nekoliko večji od lomnega količnika plašča. Navadno je 
$n_1 - n_2 \sim 0,001$. Pogosto se uporablja
bolj zapletene konstrukcije, pri katerih se lomni količnik sredice spreminja z
oddaljenostjo od osi. Z zapletenejšo geometrijo namreč zmanjšamo disperzijo v vlaknu in s tem
povečamo zmogljivost prenašanja velike količine podatkov na dolge razdalje -- 
najzmogljivejša komercialna vlakna zmorejo prenos več deset terabitov na 
sekundo\footnote{Za doprinos k razvoju in uporabi optičnih vlaken je leta 2009 Charles
K. Kao prejel Nobelovo nagrado.}.

Račun za širjenje svetlobe po cilindričnem vlaknu s homogeno sredico
je podoben kot za planparalelni vodnik, vendar je precej bolj
zapleten. V cilindrični geometriji namreč ni delitve na čiste električne in 
magnetne transverzalne valove, saj so robni pogoji sklopljeni. V splošnem se rešitve izražajo 
v obliki kombinacij Besslovih funkcij. Izkaže se, da je osnovni rod, ki se  širi po
cilindričnem vlaknu, po obliki zelo podoben osnovnemu Gaussovemu snopu, zato je sklopitev
laserskih snopov v optična vlakna zelo učinkovita.
Tudi v cilindričnih vlaknih obstaja končno število vodenih valov, odvisno od polmera sredice in
lomnih količnikov sredice in plašča. Če je polmer zadosti majhen (razlika lomnih
količnikov navadno je), obstaja le eno vodeno valovanje in optično vlakno je 
enorodovno.\index{Optično vlakno!enorodovno} Sicer je vlakno večrodovno\index{Optično 
vlakno!večrodovno}.

\subsection*{Valovna enačba v cilindričnem vlaknu}
\index{Valovna enačba}
Točen izračun za rodove v cilindričnem vlaknu presega okvire tega učbenika, zato
si oglejmo le izhodiščne enačbe in rešitve.\footnote{Točen izračun lahko bralec poišče npr. v C. C. Davis, 
{\it Lasers and Electro-optics}, Cambridge University Press.} Za jakost električnega in magnetnega polja velja 
Helmholtzeva enačba~(enačba~\ref{eq:Helmholtz})\index{Helmholtzeva enačba}
\begin{equation}
\nabla^2 \mathbf{E} + n^2(r)\, k_0^2\, \mathbf{E} = 0,
\end{equation}
pri čemer je $n(r<a)=n_1$ lomni količnik sredice in $n(r>a)=n_2$ 
lomni količnik plašča, ki je dovolj debel, da njegova debelina ne 
vpliva na potovanje svetlobe. $\mathbf{E}$ in $\mathbf{H}$ sta vektorja in ima
tri komponente, ki pa so med seboj odvisne. Izračunajmo naprej $E_z$ z nastavkom
\begin{equation}
E_z = R(r)e^{i \nu \varphi}e^{i \beta z},
\end{equation}
pri čemer je $\nu$ celo število zaradi zahteve po enoličnosti rešitve pri spremembi
kota za $2\pi$. Za $R(r)$ v sredici vlakna velja enačba
\begin{equation}
r^2 R(r)'' + r R(r)' + (k_s^2r^2 - \nu^2)R(r) = 0,
\label{10BS}
\end{equation}
kjer je 
\begin{equation}
k_s^2=k_0^2n_1^2- \beta^2,
\label{eq:ks}
\end{equation}
in v plašču
\begin{equation}
r^2 R(r)'' + r R(r)' + (-\kappa^2r^2 - \nu^2)R(r) = 0,
\label{10BP}
\end{equation}
kjer je 
\begin{equation}
\kappa^2=\beta^2-k_0^2n_2^2.
\end{equation}
V enačbah (\ref{10BS}) in (\ref{10BP}) prepoznamo Besslovo diferencialno enačbo. 
Upoštevajoč le funkcije, ki na izbranem območju ne divergirajo, zapišemo rešitev v sredici kot
\begin{equation}
E_z (r, \varphi, z) = A J_\nu(k_sr)\sin(\nu \varphi)e^{i \beta z} \quad  \mathrm{in} \quad 
H_z (r, \varphi, z) = B J_\nu(k_sr)\cos(\nu \varphi)e^{i \beta z}.
\end{equation}
Podobno zapišemo tudi rešitve v plašču
\begin{equation}
E_z (r, \varphi, z)= C K_\nu(\kappa r)\sin(\nu \varphi)e^{i \beta z} \quad \mathrm{in} \quad 
H_z (r, \varphi, z)= D K_\nu(\kappa r)\cos(\nu \varphi)e^{i \beta z}.
\end{equation}
Pri tem so $A,B,C$ in $D$ konstante, $J_\nu(x)$ je Besslova funkcija prve vrste reda 
$\nu$, $K_\nu(x)$ pa je modificirana Besslova funkcija druge vrste reda $\nu$ 
(slika~\ref{fig:J01}). 
\begin{figure}[h]
\centering
\def\svgwidth{140truemm} 
\input{slike/10_Bessel1.pdf_tex} 
\caption{Besslove funkcije: (a) Besslove funkcije prve vrste 
$J_0(x)$ (črna), $J_1(x)$ (rdeča) in $J_2(x)$ (modra), 
ki predstavljajo oblike rešitev v sredici vlakna, in (b)
modificirane Besslove funkcije druge vrste $K_0(x)$ (črna), $K_1(x)$ (rdeča) in $K_2(x)$ (modra), 
ki prestavljajo rešitev v plašču vlakna.}
\label{fig:J01}
\end{figure}

Ko enkrat poznamo komponenti $E_z$ in $H_z$, lahko z uporabo Maxwellovih enačb izračunamo še 
preostale komponente. Nato z upoštevanjem robnih pogojev zapišemo štiri enačbe za 
pet neznank ($A,B,C,D$ in $\beta$),
tako da ostane ena spremenljivka (amplituda polja) prosta. Na ta način izračunamo celotni 
jakosti električnega in magnetnega polja v vlaknu in podobno kot pri valovnem vodniku 
tudi tukaj dobimo sekularno enačbo, ki jo moramo rešiti numerično. Pri vsakem $\nu$ obstaja 
več rešitev, ki jih zato označujemo z indeksoma $\nu$ in $m$. 

\subsection*{TE in TM rodovi}
Najprej si oglejmo rešitve, pri katerih je $\nu=0$ in so tako neodvisne od kota $\varphi$. 
V klasični sliki so to žarki, ki potujejo po osi vlakna. Iz robnih pogojev sledi, da 
gre za transverzalne TE rodove, za katere velja $E_z=0$, $E_r=0$ in $E_\varphi \propto J_1(k_sr)$.
Električno poljsko jakost za TE zapišemo kot \index{Optično vlakno!TE rodovi}
\begin{equation}
\mathbf{E} \propto \mathbf{e}_\varphi \, J_1(k_s r).
\end{equation}
Podobno lahko prepoznamo tudi TM rodove, pri katerih je $H_z=0$, $H_r=0$ in $H_\varphi \propto J_1(k_sr)$.
Ustrezna električna poljska jakost za TM rodove je\index{Optično vlakno!TM rodovi}
\begin{equation}
\mathbf{E} \propto \mathbf{e}_r \, J_1(k_s r).
\end{equation}
Amplitudi električne poljske jakosti sta za TE in TM rodove enaki, zato
sta enaki tudi sliki gostote svetlobnega toka (slika~\ref{fig:TE01}). Opazimo, da je v osi
vlakna gostota svetlobnega toka enaka nič, zato sklepamo, da to niso osnovni načini 
širjenja svetlobe po cilindričnem vlaknu. 
\begin{figure}[h]
\centering
\def\svgwidth{100truemm} 
\input{slike/10_TE01.pdf_tex}
\caption{Gostota svetlobnega toka in električna poljska jakost za TE$_{01}$ in TM$_{01}$ rod
v vlaknu}
\label{fig:TE01}
\end{figure}

Podobno kot smo zapisali sekularno enačbo v valovnem vodniku (enačbi~\ref{sekular1}
in~\ref{sekular2}), tudi tukaj zapišemo enačbo za dovoljene vrednosti $k_s$. 
Ob približku, da se lomna količnika\index{Sekularna enačba!TE rodovi}
sredice in plašča le malo razlikujeta, je poenostavljena enačba za TE valovanje
\boxeq{sekFiber}{
\frac{J_1(k_sa)}{k_sa\,J_0(k_sa)}=-\frac{K_1(\kappa a)}{\kappa a\, K_0(\kappa a)},
}
pri čemer velja zveza $\kappa^2+k_s^2=(NA)^2k_0^2$. Zaporedne rešitve enačbe ustrezajo rodovom TE$_{0m}$. 
Če želimo izračunati še valovne vektorje za rodove TM$_{0m}$, moramo levo stran enačbe~(\ref{sekFiber}) pomnožiti
z $(n_1/n_2)^2$. Ker je tudi ta faktor približno enak ena, se rešitve enačb med seboj le malo razlikujejo.
\begin{figure}[h]
\centering
\def\svgwidth{90truemm} 
\input{slike/10_SekFiber.pdf_tex}
\caption{K izračunu prečnih komponent valovnega vektorja $k_s$ v 
cilindričnem vlaknu za TE polarizacijo.
Leva stran sekularne enačbe je narisana s črno in desna z rdečo.}
\label{fig:TEsecFib}
\end{figure} 

Zapisano sekularno enačbo rešujemo numerično. Lahko se je lotimo tudi grafično in 
na sliki~(\ref{fig:TEsecFib}) poiščemo presečišča krivulj. Primer na sliki 
ima dve rešitvi. S slike tudi uvidimo, da pri dovolj majhnem polmeru 
vlakna $a$ enačba nima rešitev. Takrat se namreč
črne krivulje 'raztegnejo' proti desni in rdeča krivulja divergira preden 
doseže drugo vejo črne krivulje.

Zapišimo to ugotovitev še matematično. Prvi pol leve strani sekularne enačbe nastopi pri 
$J_0 (k_s a)  = 0$, to je pri $k_s a= 2,405$. 
Po drugi strani da desna stran enačbe realne rešitve le za $k_s a \le k_0 a\,NA = V$.
Pogoj za polmer vlakna, pri katerem se TE$_{01}$ (ali TM$_{01}$) 
valovanje z dano valovno dolžino sploh širi po vlaknu, je 
\boxeq{10_cutoff}{
a \geq \frac{2,405}{k_0 NA}.
}

\subsection*{Hibridni HE in EH rodovi}
\index{Optično vlakno!HE rodovi}
\index{Optično vlakno!EH rodovi}
Poglejmo še rešitve, pri katerih $\nu \neq 0$. V tem primeru je 
vseh šest komponent električnega in magnetnega polja valovanja različnih od nič in vsi rodovi
imajo tudi komponento v smeri $z$. Take rodove imenujemo hibridni rodovi in jih 
označimo s HE, če je $E_z$ razmeroma velik ali vsaj primerljiv z $E_r$ in $E_\varphi$, 
oziroma z EH, če je $H_z$ po velikosti primerljiv s $H_r$ in $H_\varphi$ ali večji od njiju. 

Sekularna enačba za hibridne rodove je precej bolj zapletena in je ne bomo zapisali. 
Oglejmo si le njihovo obliko (slika~\ref{fig:HE11}). Najpomembnejši hibridni rod je HE$_{11}$, 
ki je sorazmeren z $J_0(k_sr)$ in zato v središču različen od nič. 
To je osnovni rod, za katerega rešitev sekularne enačbe vedno obstaja in se
zato širi po še tako tankem vlaknu. 
\begin{figure}[h]
\centering
\def\svgwidth{85truemm} 
\input{slike/10_HE11.pdf_tex}\\
\def\svgwidth{85truemm} 
\input{slike/10_HE21.pdf_tex} \\
\def\svgwidth{85truemm} 
\input{slike/10_HE31.pdf_tex} \\
\def\svgwidth{85truemm} 
\input{slike/10_EH11.pdf_tex}
\caption{Intenziteta in električna poljska jakost za rodove
HE$_{11}$, HE$_{21}$, HE$_{31}$ in EH$_{11}$}
\label{fig:HE11}
\end{figure}

Po obliki je osnovni HE$_{11}$ rod zelo podoben Gaussovemu profilu $\exp(-r^2/w^2)$,
zato ga lahko razmeroma dobro opišemo z Gaussovim približkom. 
Pri tem efektivni polmer snopa
$w$ izračunamo po Marcusejevi\index{Marcusejeva formula}
\index{Gaussov snop!efektivni polmer} formuli\footnote{D. Marcuse, Bell Syst. Tech. J. 56, 703 (1977).}
\begin{equation} 
w = (0,65 + \frac{1,619}{V^{3/2}}+\frac{2,879}{V^{6}})\,a,
\label{Marcuse}
\end{equation}
pri čemer je $V = k_0 a\,NA $. Podobnost profila osnovnega
HE$_{11}$ z Gaussovo funkcijo omogoča zelo dobro sklopitev Gaussovih
snopov, ki izhajajo iz laserja, v cilindrična vlakna.

Na sliki~(\ref{fig:HE11}) je poleg osnovnega HE$_{11}$ roda še nekaj primerov višjih rodov. Opazimo, 
da imajo vsi rodovi, razen osnovnega, v izhodišču minimum. Poleg tega opazimo tudi podobnost med 
oblikami posameznih rodov, do katere pride zaradi majhne razlike med lomnima količnikoma sredice in plašča
($n_1 \approx n_2$). V takem primeru se sekularne enačbe poenostavijo, nekateri rodovi so 
med seboj degenerirani in dajo enako rešitev. Poleg rodov z enako obliko in različno polarizacijo so tako 
med seboj degenerirani še HE$_{\nu+1,m}$ in EH$_{\nu-1,m}$ rodovi. Degenerirane 
rodove lahko združimo v linearne kombinacije teh valov in nastanejo pretežno linearno polarizirani 
rodovi. 

\subsection*{LP rodovi}
\index{Optično vlakno!LP rodovi}
Za praktično uporabo so najpomembnejši linearno polarizirani rodovi. Taki rodovi niso
točne rešitve valovne enačbe v cilindrični geometriji, ampak jih zapišemo kot linearno 
kombinacijo lastnih rodov, ki so zaradi majhne razlike med lomnima količnikoma sredice
in plašča degenerirani. Tudi te rodove označimo z dvema indeksoma: prvi določa število azimutalnih
vozlov, drugi pa število radialnih vrhov. Poglejmo nekaj primerov (slika~\ref{fig:LP}).

Osnovni LP$_{01}$ je kar približno enak osnovnemu HE$_{11}$ rodu. Električna poljska jakost je 
\begin{equation}
\mathbf{E}_\mathrm{LP01} \propto { \mathbf{e}_x \brace \mathbf{e}_y} \, J_0(k_s a),
\end{equation}
saj ima dve možni smeri polarizacije. V splošnem so LP$_{0m}$ zelo podobni rodovom HE$_{1m}$. 

Višje rodove, na primer LP$_{11}$ sestavimo kot linearno kombinacijo 
TE$_{01}$ ali TM$_{01}$ in HE$_{21}$.
Električna poljska jakost v LP$_{11}$ je tako 
\begin{equation}
\mathbf{E}_\mathrm{LP11} \propto { \mathbf{e}_x \brace \mathbf{e}_y} \, J_1(k_s a)
{ \cos\varphi \brace \sin\varphi},
\end{equation}
kar opisuje štiri različne oblike rodov LP$_{11}$.

Tudi LP$_{21}$ rodovi, ki nastanejo kot kombinacija HE$_{31}$
in EH$_{11}$ rodov, imajo štiri oblike
\begin{equation}
\mathbf{E}_\mathrm{LP21} \propto { \mathbf{e}_x \brace \mathbf{e}_y} \, J_2(k_s a)
{ \cos 2\varphi \brace \sin 2\varphi}.
\end{equation}
\begin{figure}[h!]
\centering
\def\svgwidth{93truemm} 
\input{slike/10_LP01.pdf_tex} \\
\def\svgwidth{93truemm} 
\input{slike/10_LP02.pdf_tex} \\
\def\svgwidth{93truemm} 
\input{slike/10_LP11a.pdf_tex} \\
\def\svgwidth{93truemm} 
\input{slike/10_LP11b.pdf_tex} \\
\def\svgwidth{93truemm} 
\input{slike/10_LP21a.pdf_tex} \\
\def\svgwidth{93truemm} 
\input{slike/10_LP21b.pdf_tex} \\
\caption{Gostota svetlobnega toka in smeri električne poljske jakosti za približno linearne rodove
LP$_{01}$, LP$_{02}$, LP$_{11}$ in LP$_{21}$.}
\label{fig:LP}
\end{figure}
Linearno polarizirani LP rodovi imajo precejšno uporabno vrednost. To so 
namreč rodovi, ki jih v vlaknu vzbudimo, ko nanj posvetimo s polarizirano 
lasersko svetlobo. Zavedati pa se moramo, da to niso lastni rodovi vlakna, 
ampak njihove linearne kombinacije, ki po vlaknu potujejo z malenkost različnimi
hitrostmi. Polarizacija svetlobe se zato vzdolž vlakna rahlo spreminja.

\begin{definition}
Pokaži, da je približno število dovoljenih rodov v cilindričnem vlaknu pri 
izbrani normirani frekvenci $V = k_0a\, NA $ enako
\begin{equation} 
N = \frac{4 V^2}{\pi^2}.
\end{equation}
Namig: Upoštevaj asimptotični razvoj Besslovih funkcij za velike argumente
\begin{equation}
J_\nu(x) \approx \sqrt{\frac{2}{\pi x}}\cos\left(x - \frac{\nu \pi }{2}- \frac{\pi}{4}\right).
\end{equation}
\end{definition}

\subsection*{Cilindrično vlakno s paraboličnim profilom lomnega količnika}
Čeprav je račun lastnih načinov v cilindričnem vlaknu zapleten, lahko 
razmeroma enostavno poiščemo rešitve za vlakno, v katerem je dielektrična 
konstanta kvadratna funkcija radialne koordinate $r$\index{Optično vlakno!parabolični profil}. 
\begin{equation}
\varepsilon (r) = \varepsilon_1 - C r^2,
\end{equation}
pri čemer je sprememba navadno majhna. Zapišemo še enačbo z lomnimi količniki
in vpeljemo brezdimenzijski parameter $\Delta$
\begin{equation}
n^2\left(r<a\right)=n_{1}^{2}- \Delta^2 \frac{r^2}{a^2},
\label{9.15}
\end{equation}
pri čemer $a$ označuje polmer vlakna.
Enačbo lahko tudi razvijemo za majhno razliko $\Delta$ in za vse smiselne vrednosti $r$
ima tudi lomni količnik parabolični profil. Parabolična
sredica je seveda omejena, okoli nje pa je plašč s konstantnim
lomnim količnikom $n_2 \approx n_1-\Delta^2/2n_1$ (slika~\ref{fig:GRIN}). 
Tipičen polmer sredice $a$ je nekaj deset mikrometrov, plašča pa približno petkrat toliko.
\begin{figure}[h]
\centering
\def\svgwidth{90truemm} 
\input{slike/10_GRIN.pdf_tex} 
\caption{Parabolični profil lomnega količnika sredice zmanjša disperzijo v vlaknu. Plašč
 je praviloma bistveno debelejši od sredice vlakna.}
\label{fig:GRIN}
\end{figure}

Električno poljsko jakost za izbrano polarizacijo zapišemo v obliki 
\begin{equation}
E=E_{0}\psi(x,y)\, e^{i\beta z} e^{-i\omega t}.
\label{9.16}
\end{equation}
Pri tem smo zanemarili, da zaradi odvisnosti od prečnih koordinat in pogoja $\nabla\cdot{\bf D}=0$
polje ne more imeti povsod iste smeri; za bolj natančen račun bi morali v gornji obliki 
zapisati vektorsko polje. Vstavimo približni
nastavek~(enačba~\ref{9.16}) in krajevno odvisnost lomnega količnika~(enačba~\ref{9.15})
v valovno enačbo (enačba~\ref{eq:valovna-skalarna}) in dobimo 
\begin{equation}
\nabla_{\perp}^{2}\psi+\left[k_{0}^{2}\left(n_{1}^{2}-\Delta^{2}\frac{r^{2}}{a^2}\right)-
\beta^{2}\right]\,\psi=0.
\label{9.17}
\end{equation}
Rešitve lahko zapišemo v obliki
\begin{equation}
\psi(x,y) = X(x)Y(y),
\end{equation}
od koder sledita dve neodvisni enačbi
\begin{equation}
X'' - \frac{k_0^2 \Delta^2}{a^2}\,X\,x^2 - \lambda_1 X = 0 \qquad \mathrm{in} \qquad
Y'' - \frac{k_0^2 \Delta^2}{a^2}\,Y\,y^2 - \lambda_2 Y = 0,
\label{eq:XY}
\end{equation}
pri čemer sta $\lambda_1$ in $\lambda_2$ konstanti. 
Opazimo, da sta enačbi popolnoma enaki enačbama za krajevni del lastnih funkcij 
harmonskega oscilatorja v kvantni mehaniki. Rešitev posamezne enačbe je tako 
produkt Gaussove in Hermitove funkcije\index{Harmonski oscilator}
\begin{equation}
X_n(x) = e^{-\xi^2 x^2/2} H_n(\xi x),
\label{eq:GH}
\end{equation}
pri čemer je $\xi = \sqrt{k_0 \Delta/a}$.
\begin{definition}
Uporabi nastavek (enačba~\ref{eq:GH}) in pokaži, da reši enačbo~(\ref{eq:XY}). Pri tem si pomagaj z 
diferencialno enačbo za Hermitove polinome
\begin{equation}
\left( \frac{d^2}{dx^2}-2x\frac{d}{dx}+2n \right) H_n(x) = 0.
\end{equation}
\end{definition}
Lastne vrednosti enačbe so oblike
\begin{equation}
\beta_{mn}^{2}=n_{1}^{2}k_{0}^{2}\left(1-\frac{2\Delta}{k_{0}n_{1}^2a}\left(m+n+1\right)\right).
\label{9.19}
\end{equation}
Drugi člen v oklepaju je navadno zelo majhen, zato lahko izraz razvijemo in 
\begin{equation}
\beta_{mn}=n_{1}k_{0}\left(1-\frac{\Delta}{k_{0}n_{1}^2 a}\left(m+n+1\right)\right)
= n_{1}k_{0} - \frac{\Delta \left(m+n+1\right)}{n_{1} a}.
\end{equation}
Ob privzetku, da je parameter $\Delta$ neodvisen od frekvence, je grupna 
hitrost\index{Hitrost valovanja!grupna}
\begin{equation}
v_{g}=\left(\frac{d\beta_{mn}}{d\omega}\right)^{-1}=\frac{c_{0}}{n_{1}}
\label{9.21}
\end{equation}
in torej enaka za vse rodove. To je pomembna značilnost vlakna s paraboličnim profilom
lomnega količnika. V dejanskem vlaknu je seveda taka odvisnost mogoča
le v omejenem območju sredice, zato je tudi gornja analiza le približna
in velja dobro za tiste rodove, ki se ne raztezajo dosti izven sredice.

Neodvisnost grupne hitrosti od roda je praktično zelo pomembna. 
Grupna hitrost namreč določa čas potovanja svetlobnega sunka, ki
lahko predstavlja en bit informacije. Če se po vlaknu širi več
rodov z različnimi grupnimi hitrostmi, se sunek po prehodu skozi
vlakno razširi, kar -- kot bomo podrobneje videli v naslednjem razdelku -- omejuje 
uporabno dolžino vlakna. Temu se sicer lahko izognemo z uporabo enorodovnih vlaken,
ki pa so dražja, poleg tega morata divergenca in polmer svetlobnega snopa 
natančno ustrezati značilnostim enorodovnega vlakna, da se izognemo izgubam. 
Zato se za krajše zveze uporabljajo večrodovna vlakna, ki imajo sredico s 
približno paraboličnim profilom lomnega količnika.

\section{Disperzija}
\label{chap:Disperzija}
Pri prenosu velike količine podatkov na daljavo je zelo pomembno, da
se oblika svetlobnih sunkov, ki prenašajo informacijo, čim manj spremeni.
Na obliko sunka močno vpliva disperzija, to je odvisnost fazne in grupne (skupinske) hitrosti
valovanja od frekvence. Zaradi disperzije se kratki sunki, ki potujejo po vlaknu, podaljšajo in 
tako omejujejo količino informacije, ki jo lahko prenašamo po vlaknu dane dolžine\index{Disperzija}.
\begin{figure}[h]
\centering
\def\svgwidth{120truemm} 
\input{slike/10_disperzija.pdf_tex} 
\caption{Zaradi disperzije se širina sunkov svetlobe, ki potujejo skozi vlakno, 
močno poveča, zato jih na izhodu iz vlakna ne moremo več ločiti.}
\label{fig:disp}
\end{figure}

V splošnem poznamo pri potovanju svetlobe po optičnih vlaknih tri 
vrste disperzije: materialno, valovodno in rodovno. 
Do materialne disperzije pride zaradi odvisnosti lomnega količnika vlakna od 
valovne dolžine svetlobe. Valovodna disperzija se pojavi zaradi nelinearne zveze 
med valovnim številom $\beta$ in frekvenco valovanja. Rodovna
disperzija pa je posledica različnih hitrosti različnih rodov. Govorimo lahko še o 
polarizacijski disperziji, saj smo že spoznali, da je hitrost valovanja svetlobe 
v vodniku odvisna tudi od polarizacije valovanja. Poglejmo si disperzije podrobneje. 

\subsection*{Materialna disperzija}
\index{Disperzija!materialna}
Vzemimo najprej enorodovno vlakno in naj bo svetloba v vlaknu modulirana
v obliki kratkih sunkov, ki nosijo informacijo. Kratki sunki svetlobe
potujejo z grupno hitrostjo 
\begin{equation}
v_{g}=\frac{d\omega}{d\beta}=\left(\frac{d\beta}{d\omega}\right)^{-1} = \frac{c_0}{n_g},
\label{9.51}
\end{equation}
pri čemer smo vpeljali grupni lomni količnik $n_g$.\index{Lomni količnik!grupni}
Ker je sunek končno dolg, ima končno spektralno širino $\Delta \lambda=\lambda_{\mathrm{max}}-\lambda_{\mathrm{min}}$. 
O materialni disperziji govorimo, kadar je lomni količnik vlakna odvisen od
valovne dolžine svetlobe in zato različne spektralne komponente po vlaknu potujejo z različnimi
hitrostmi. Dolžino sunka $\tau$ po prehodu skozi vlakno dolžine $L$ zapišemo kot
\begin{equation}
\tau_m = \frac{L}{v_g(\lambda_{\mathrm{max}})} - \frac{L}{v_g(\lambda_{\mathrm{min}})} = 
\frac{L}{c_0}\left(n_g(\lambda_{\mathrm{max}}) - n_g(\lambda_{\mathrm{min}}) \right) 
= \frac{L}{c_0}\frac{d n_g}{d \lambda}\Delta \lambda.
\label{MatD}
\end{equation}
Za enorodovno vlakno velja $k_x \approx 0$ in $\beta \approx n_1 \omega/c_0$. 
Sledi
\begin{equation}
n_g = c_0 \frac{d\beta}{d\omega} = n_1 + \omega \frac{d n_1}{d\omega} = n_1 - \lambda 
\frac{dn_1}{d\lambda}
\end{equation}
in
\begin{equation}
\frac{dn_g}{d\lambda} = -\lambda\frac{d^2n_1}{d\lambda^2}.
\end{equation}
To vstavimo v izraz za dolžino sunka~(enačba~\ref{MatD}) in lahko zapišemo
\boxeq{eq:dmat1}{
\tau_m = D_m L\, \Delta \lambda,
}
pri čemer je $D_m$ koeficient materialne disperzije
\boxeq{eq:dmat}{
D_m = - \frac{\lambda}{c_0}\frac{d^2n_1}{d\lambda^2}.
}
Navadno ga izrazimo v enotah $\si{\pico\second/\nano\meter\, \kilo\meter}$, 
njegova vrednost pa je lahko pozitivna ali negativna. 
V snoveh, ki jih uporabljamo za optična vlakna, je $D_m$ reda 
$10~\si{\pico\second/\nano\meter\, \kilo\meter}$, lahko pa seže
tudi do več $100~\si{\pico\second/\nano\meter\, \kilo\meter}$, odvisno seveda od valovne dolžine
in izbrane snovi. 

Materialno disperzijo lahko zmanjšamo na več načinov. Lahko uporabimo čim bolj enobarven
vir svetlobe, da zmanjšamo $\Delta \lambda$. Za snovi, ki so v uporabi, 
lahko celo izberemo valovno dolžino, pri kateri je koeficient materialne disperzije enak nič.
Za SiO$_2$\index{SiO$_2$} je to pri okoli $1300$--$1500~\si{\nano\meter}$, odvisno od dopiranja stekla. Še najbolj uporabna je rešitev, pri kateri z materialno 
disperzijo izničimo vpliv drugih disperzij in 
na ta način zmanjšamo skupno disperzijo v vlaknu.

\subsection*{Valovodna disperzija}
\index{Disperzija!valovodna}
Spomnimo se, da v vlaknu velja zveza med prečno $k_s$ in vzdolžno komponento $\beta$ 
valovnega vektorja (enačba~\ref{eq:ks})
\begin{equation}
\beta^2 + k_s^2 = k_0^2 n_1^2 = \left( \frac{\omega}{c_0}\right)^2n_1^2.
\end{equation}
Pri tem moramo $k_s$ izračunati numerično iz sekularne enačbe (ki je podobna enačbi~\ref{sekFiber}). 
Rešitev je odvisna od valovne dolžine svetlobe oziroma njene frekvence. Valovno število  
\begin{equation}
\beta = \sqrt{\left( \frac{\omega}{c_0}\right)^2n_1^2 - k_s(\omega)^2}
\end{equation}
je tako nelinearna funkcija frekvence, zaradi česar pride do disperzije. Če naredimo podoben 
račun kot pri materialni disperziji, je razširitev začetnega kratkega sunka enaka
\begin{equation}
\tau_v = \frac{L}{v_g(\omega_{\mathrm{max}})} - \frac{L}{v_g(\omega_{\mathrm{min}})} = 
L\frac{d \beta}{d\omega}(\omega_{\mathrm{max}}) - L\frac{d \beta}{d\omega}(\omega_{\mathrm{min}})=
L \frac{d^2\beta}{d\omega^2}\Delta \omega.
\label{ValD}
\end{equation}
Upoštevamo zvezo med valovno dolžino in frekvenco in zapišemo 
\begin{equation}
\tau_v=-\frac{2 \pi c_0}{\lambda^2}\frac{d^2\beta}{d\omega^2}L \Delta \lambda= D_v\,L\, \Delta \lambda,
\end{equation}
pri čemer je  $D_v$ koeficient valovodne disperzije
\boxeq{Dvalkoef}{
D_v = - \frac{2 \pi c_0}{\lambda^2}\frac{d^2\beta}{d\omega^2}.
}
Prispevek valovodne disperzije je praviloma najmanjši, reda 
$1$--$10~\si{\pico\second/\nano\meter\,\kilo\meter}$. 
Znaten postane v enorodovnih vlaknih v območju, kjer je materialna disperzija 
zelo majhna ali celo enaka nič. 
V vlaknih s homogeno sredico se valovodni disperziji ne moremo
izogniti, lahko pa jo pri dani valovni dolžini izničimo z materialno~(slika~\ref{fig:MatVal}). 
\begin{figure}[h]
\centering
\def\svgwidth{90truemm} 
\input{slike/10_Zero.pdf_tex} 
\caption{Odvisnost disperzije od valovne dolžine v SiO$_2$ vlaknu.\index{SiO$_2$} $D_m$ 
označuje materialno disperzijo, $D_v$ valovodno, $D$ pa je vsota obeh. Pri valovni dolžini
okoli $1450~\si{\nano\meter}$ se materialna in valovodna disperzija odštejeta in skupna disperzija
je praktično enaka nič.}
\label{fig:MatVal}
\end{figure}
\begin{remark}
Na valovodno disperzijo je mogoče vplivati tudi s konstrukcijo vlakna. Pokazali smo že, da
v idealnem primeru v vlaknu s paraboličnim profilom lomnega količnika disperzije ni. 
V praksi je sredica sestavljena iz več plasti z različnimi lomnimi količniki in različnimi
debelinami, s čimer se prispevek valovodne disperzije spremeni. Na ta način lahko 
položaj ničle celotne disperzije premaknemo k valovni dolžini izvora oziroma k 
valovni dolžini, pri kateri je v vlaknu najmanj absorpcije in izgub.
\end{remark}

Količina podatkov, ki jih lahko prenašamo po enorodovnem vlaknu, je kar približno obratno 
sorazmerna s širino izhodnih sunkov svetlobe. Pri celotni disperziji 
$5~\si{\pico\second/\nano\meter\,\kilo\meter}$ in 
spektralni širini $1~\si{\nano\meter}$ je tako v $100~\si{\kilo\meter}$ 
dolgem vlaknu najvišja frekvenca modulacije
okoli $2~\si{\giga\hertz}$. V nadaljevanju bomo videli, da je pri prenosu podatkov
v vlaknih poglavitni omejujoči faktor ravno disperzija in ne absorpcija. 

\subsection*{Rodovna disperzija}
\index{Disperzija!rodovna}
\index{Optično vlakno!večrodovno}
Do zdaj smo obravnavali disperzijo v enorodovnih vlaknih, v katerih
lahko disperzijo močno zmanjšamo. V večrodovnih vlaknih pa
je poglavitni vzrok širjenja sunkov rodovna disperzija. Do nje pride zaradi razlike 
v hitrostih posameznih rodov. 

Obravnavajmo vlakno, v katerem se širi več rodov. Osnovni rod ima najmanjšo
vrednost in uporabimo kar približek $k_s \approx 0$ in $\beta \approx k_0 n_1$. 
Zadnji še dovoljeni rod ima\index{Lomni količnik!grupni}
največjo vrednost $k_s \approx NA\, k_0$ in $\beta \approx k_0 n_2$. Grupna
lomna količnika za prvi in zadnji rod sta tako
\begin{equation}
n_{g0} = c_0 \left(\frac{d\beta}{d\omega}\right) = c_0 \frac{d(k_0n_1)}{d\omega} = 
n_1 + \omega \frac{dn_1}{d\omega}
\end{equation}
in
\begin{equation}
n_{gN} = c_0 \left(\frac{d\beta}{d\omega}\right) = c_0 \frac{d(k_0n_2)}{d\omega} = 
n_2 + \omega \frac{dn_2}{d\omega}.
\end{equation}
Kratek račun pokaže, da je razširitev sunka zaradi rodovne disperzije 
\boxeq{DispRod}{
\tau_r \approx 
\frac{L}{c_0} (n_1-n_2).
}
Za $1~\si{\kilo\meter}$ dolgo vlakno z razliko lomnih količnikov $\Delta n = 0,05$  je tako
največja frekvenca modulacije okoli $10~\si{\mega\hertz}$, kar je znatno manj od enorodovnih vlaken.
Čeprav lahko disperzijo zmanjšamo s paraboličnim profilom lomnega količnika, so
večrodovna vlakna za prenos podatkov na dolge razdalje praktično neuporabna. 

V večrodovnem vlaknu je treba upoštevati prispevke vseh treh disperzij. Materialna
in valovodna sta obe odvisni od valovne dolžine in zato medsebojno korelirani, rodovna pa 
je odvisna predvsem od zgradbe vlakna in je od prvih dveh praktično neodvisna. Ko na 
večrodovno vlakno posvetimo s sunkom, katerega spekter je Gaussove oblike, je dolžina 
sunka \index{Disperzija!podaljšanje sunka}po prehodu skozi vlakno
\boxeq{Dtotal}{
\tau = \sqrt{(\tau_m +\tau_v)^2 + \tau_r^2}.
}

\begin{remark}
 Omenili smo tudi polarizacijsko disperzijo,\index{Disperzija!polarizacijska} 
 do katere pride zaradi različnih hitrosti valovanj
 z različnima polarizacijama. V idealnem cilindričnem vlaknu potujeta obe polarizaciji
 z enako hitrostjo. V realnem vlaknu pa 
 pride na nečistočah in asimetrijah v vlaknu do različnih hitrosti za različni polarizaciji. 
 Ker so nečistoče slučajno in neodvisno razporejene, tako disperzijo zelo težko odpravimo.
\end{remark}

\section{*Potovanje kratkega sunka po enorodovnem vlaknu}
\label{chap:sunvl}
\subsection*{Podaljšanje sunka zaradi disperzije}
\index{Optično vlakno!enorodovno}
\index{Disperzija!podaljšanje sunka}
Poglejmo si podrobneje, kako po enorodovnem vlaknu ali drugem
sredstvu z disperzijo potuje kratek sunek valovanja z dano začetno obliko.
Sunek zapišimo kot  
\begin{equation}
E\left(x, y, z, t\right)=\psi\left(x,y\right)\, a\left(z,t\right),
\label{9.61}
\end{equation}
kjer je $\psi\left(x,y\right)$ lastna rešitev prečnega dela valovne
enačbe, ki določa  zvezo $\beta\left(\omega\right)$. 
Funkcija $a\left(z,t\right)$ opisuje obliko sunka v smeri $z$ in njegovo
širjenje. Razvijmo jo pri $z=0$ po ravnih valovih z ustreznimi amplitudnimi faktorji 
\begin{equation}
a\left(0,t\right)=\int \tilde{A}(\omega)\, e^{- i\omega t}d\omega.
\label{9.62}
\end{equation}
Ko sunek potuje vzdolž osi $z$, moramo vsaki frekvenčni komponenti pripisati
ustrezen fazni faktor $i \beta (\omega) z$. Tako dobimo 
\begin{equation}
a\left(z,t\right)=\int \tilde{A}(\omega)\, e^{i \beta (\omega) z - i\omega t}d\omega.
\label{9.62f}
\end{equation}
Osnovni sunek naj bo približno monokromatičen s frekvenco $\omega_{0}$,
kar pomeni, da je mnogo daljši od optične periode. Potem lahko lahko $\beta(\omega)$
razvijemo okoli $\omega_{0}$, pri čemer vpeljemo razliko frekvenc $\Omega = \omega - \omega_0$
\begin{equation}
\beta(\omega_0 + \Omega) \approx \beta(\omega_{0})
+\frac{d\beta}{d\omega}\,\Omega+\frac{1}{2}\,\frac{d^{2}\beta}{d\omega^{2}}\,\Omega^{2}.
\label{9.62c}
\end{equation}
Enačbo~(\ref{9.62f}) tako prepišemo v 
\begin{eqnarray}
a\left(z,t\right)&=&\int \tilde{A}(\Omega)\, e^{i \beta (\omega_0 + \Omega)z - 
i(\omega_0 + \Omega) t}d\Omega\\
& = & e^{i \beta (\omega_0)z - i\omega_0 t} A(z,t).
\label{9.62b}
\end{eqnarray}
Funkcija $A(z, t)$ očitno predstavlja prostorsko in časovno odvisnost ovojnice sunka. Z upoštevanjem
razvoja (enačba~\ref{9.62c}) jo zapišemo kot 
\begin{equation}
 A(z,t) = \int \tilde{A}(\Omega)\, \exp \left(i \frac{d\beta}{d\omega}\, 
 \Omega\,z + \frac{i}{2}\frac{d^2 \beta}{d\omega^2}
 \,\Omega^2\, z - i\, \Omega\, t\right) d\Omega.
\end{equation}
Vpeljemo še grupno hitrost $v_g = d\omega/d\beta$ in ovojnico zapišemo kot\index{Hitrost valovanja!grupna}
\begin{equation}
 A(z,t) = \int \tilde{A}(\Omega)\, \exp \left(i \Omega \left(\frac{z}{v_g}-t\right)+ 
 \frac{i}{2}\frac{d}{d\omega}
 \left(\frac{1}{v_g}\right)\Omega^2 z\right) d\Omega.
 \label{9.ovoj}
\end{equation}
V primeru, da je grupna hitrost neodvisna od frekvence, da torej ni disperzije grupne hitrosti, 
je drugi člen v eksponentu enak nič. Takrat je ovojnica enaka 
\begin{equation}
  A(z,t) = \int \tilde{A}(\Omega)\, e^{i \Omega \left(\frac{z}{v_g}-t\right)} d\Omega =
  A\left(0,t-\frac{z}{v_g}\right),
\end{equation}
kar je ravno enako obliki signala pri $z = 0$. Oblika sunka se ob odsotnosti 
disperzije ohranja in sunek 
poljubne začetne oblike nepopačen potuje po vlaknu z grupno hitrostjo. 

Vrnimo se k enačbi~(\ref{9.ovoj}). Odvajajmo jo po $z$ 
\begin{equation}
 \frac{\partial A(z,t)}{\partial z} = \left( i\Omega \frac{1}{v_g}+ \frac{i}{2}
\frac{d}{d\omega} \left(\frac{1}{v_g}\right) \Omega^2 \right) A(z,t).
\label{9.67}
\end{equation}
Desno stran izraza lahko zapišemo s časovnimi odvodi ovojnice in dobimo 
\begin{equation}
\frac{\partial A(z,t)}{\partial z} = -\frac{1}{v_{g}}\,\frac{\partial A (z,t)}{\partial t}
-\frac{i}{2}\,\frac{d^{2}\beta}{d\omega^{2}}\,\frac{\partial^{2}A\left(z,t\right)}{\partial t^{2}}
\label{9.68} 
\end{equation}
oziroma
\begin{equation}
\left(\frac{\partial}{\partial z} + \frac{1}{v_{g}}\,\frac{\partial }{\partial t}\right)A (z,t) = 
-\frac{i}{2}\,\frac{d^{2}\beta}{d\omega^{2}}\,\frac{\partial^{2}A\left(z,t\right)}{\partial t^{2}}.
\label{9.68b} 
\end{equation}
Enačbo lahko nekoliko poenostavimo z vpeljavo novih neodvisnih spremenljivk
\begin{eqnarray}
\tau & = & t-\frac{z}{v_{g}}\nonumber \\
\zeta & = & z.
\label{9.70}
\end{eqnarray}
Za vrh sunka, ki naj ima pri $t=0$ koordinato $z=0$ in se giblje
z grupno hitrostjo, je vselej $\tau=0$. Spremenljivka $\tau$ torej predstavlja
čas v točki $z=\zeta$ , merjen od trenutka, ko tja
prispe osrednji del sunka. Z novima spremenljivkama se enačba~(\ref{9.68b})
prepiše v 
\begin{equation}
\frac{d^{2}\beta}{d\omega^{2}}\,\frac{\partial^{2}A}{\partial\tau^{2}}-
2\, i\,\frac{\partial A}{\partial\zeta}=0
\label{9.71}
\end{equation}
Ta enačba ima isto obliko kot obosna valovna enačba, ki smo jo v
drugem poglavju uporabili za obravnavo koherentnih \index{Obosna valovna enačba}
snopov (enačba~\ref{eq:obosna-valovna-enacba}). Spomnimo se, da 
obosno valovno enačbo rešijo Gaussovi snopi (enačba~\ref{eq:gaussov-snop}). 

Podobnost med pojavoma seže dlje od formalne oblike. Pri snopih, ki so omejeni 
v prečni smeri, disperzija fazne in grupne hitrosti po prečnih komponentah valovnega
vektorja povzroča spreminjanje prečnega preseka snopa. Pri časovno
omejenih sunkih v sredstvu s frekvenčno disperzijo se namesto preseka sunka
spreminja njegova vzdolžna oblika. Poglejmo, kako.

Enačba~(\ref{9.71}) je zelo podobna obosni valovni enačbi, le da ima tukaj vlogo 
prečne koordinate čas $\tau$. Po analogiji s snopi lahko sklepamo, da se najmanj 
širi ravno sunek z Gaussovo časovno odvisnostjo. Celotnega računa
ni treba ponavljati, namesto tega kar v izrazu za Gaussove snope 
(enačba~\ref{eq:gaussov-snop}) napravimo ustrezno zamenjavo spremenljivk. 
Iz enačbe~(\ref{9.71}) razberemo, da valovnemu številu $k$ pri snopih  
ustreza parameter $\mu=(d^{2}\beta/d\omega^{2})^{-1}$. Poleg tega vpeljemo
trajanje sunka $\sigma$, ki ustreza polmeru Gaussovega snopa $w$, in parameter
$b$, ki ustreza krivinskemu radiju $R$. Oba parametra sta seveda odvisna od $\zeta$, 
tako kot sta parametra $w$ in $R$ odvisna od $z$. 

Na podlagi analogije zapišemo obliko podaljšanega Gaussovega sunka\index{Gaussov sunek}
\begin{equation}
A\left(\zeta,\tau\right)=\frac{A_{0}}{\sqrt{1+\frac{\zeta^{2}
}{\zeta_{0}^{2}}}}\exp\left(-\frac{\tau^{2}}{\sigma^{2}}\right)\exp
\left(-i\frac{\mu\tau^{2}}{2b}\right)e^{i\phi\left(\zeta\right)}.
\label{9.72}
\end{equation}
Pri tem za dolžino sunka $\sigma$ velja enaka zveza kot za polmer 
Gaussovega snopa (enačba~\ref{eq:w})
\boxeq{9.73}{
\sigma^{2}=\sigma_{0}^{2}\left(1+\left(\frac{\zeta}{\zeta_{0}}\right)^{2}\right).
}
Tu je $\sigma_{0}$ trajanje sunka pri $\zeta=0$, to je na začetku,
kjer je sunek najkrajši. Dodatna skupna faza $\phi\left(\zeta\right)$
ni posebno pomembna, pač pa je zanimiv drugi eksponentni faktor v
enačbi~(\ref{9.72}). V njem smo z $b=\zeta\left(1+\zeta_{0}^{2}/\zeta^{2}\right)$
označili količino, ki je analogna krivinskemu radiju valovnih front
v primeru Gaussovih snopov (enačba~\ref{eq:R}).
\begin{figure}[h]
\centering
\def\svgwidth{110truemm} 
\input{slike/10_Gausstau.pdf_tex}
\caption{Primerjava krajevne širitve Gaussovega snopa in časovne širitve Gaussovega sunka}
\label{fig:Gausstau}
\end{figure}

Odvod faze po $\tau$ predstavlja spremembo
frekvence glede na centralno frekvenco sunka $\omega_{0}$
\begin{equation}
\omega-\omega_{0}=\frac{\mu\tau}{b}.
\label{9.74}
\end{equation}
Za pozitivne vrednosti $\mu$ je frekvenca na začetku sunka,
to je pri $\tau<0$, manjša od $\omega_0$, z naraščajočim časom pa se 
linearno povečuje proti koncu sunka. Obnašanje je torej podobno tistemu, 
ki ga poznamo iz nelinearne optike
(slika~\ref{fig:chirp}\,a). 

\begin{remark}
Pri $\zeta=0$ je sunek najkrajši možen pri dani spektralni
širini. Lahko si mislimo, da je sunek najkrajši,
to je omejen s Fourierevo transformacijo spektra, kadar se
vse frekvenčne komponente seštejejo z isto fazo, to je pri $\zeta=0$.
Da nastanejo najkrajši sunki, kadar je faza vseh delnih valov enaka,
smo spoznali že pri fazno uklenjenih sunkih iz večfrekvenčnih laserjev
(poglavje~\ref{chap:Uklepanje}).\index{Uklepanje faz}
Pri potovanju sunka se zaradi disperzije faze frekvenčnih komponent
različno spreminjajo in sunek se podaljša. Pri tem je pomemben  
drugi odvod fazne hitrosti po frekvenci. Linearno spreminjanje faze 
namreč ne povzroči razširitve, temveč le razliko med grupno in fazno hitrostjo.
\end{remark}

\begin{definition}
\label{naloga:pulzdisperzija}
Naj bo vpadni sunek svetlobe Gaussove oblike $E(x,y, z=0, t) = 
\psi(x,y) e^{-at^2-i \omega_0 t}$. Pokaži, da je ustrezna funkcija $\tilde{A}(\Omega)$ oblike
\begin{equation}
\tilde{A}(\Omega) = \frac{1}{\sqrt{4 \pi a}}e^{-\Omega^2/4a},
\end{equation}
nato pa z neposredno integracijo (enačba~\ref{9.ovoj}) pokaži, 
da je rezultat enak ovojnici, zapisani z enačbo~(\ref{9.72}), pri čemer je
\begin{equation}
\zeta_0 = \frac{\mu}{2 a}.
\end{equation}
\end{definition}

\begin{definition}
Uporabi enačbo~(\ref{9.73}) in pokaži, da je najmanjše podaljšanje sunka svetlobe
pri dani dolžini vlakna enako
\begin{equation}
\tau_v (L) = \sqrt{\frac{2 \ln 2}{a}}\sqrt{1 + \frac{4 a^2 L^2}{\mu^2}}
= \tau_v(0)\sqrt{1 + \left(\frac{4 \ln 2}{\tau_v(0)^2 \mu}L\right)^2}
\end{equation}
in za velike dolžine enako izrazu, ki smo ga izračunali pri 
valovodni disperziji (enačba~\ref{ValD}).
\end{definition}

\subsection*{Kompenzacija disperzije}
\index{Disperzija!kompenzacija}
Razširitev sunka zaradi pozitivne disperzije je pri $\mu > 0$ mogoče kompenzirati
s parom vzporednih uklonskih mrežic, kot kaže slika~(\ref{fig:comp}).\index{Uklonska mrežica}
Prva mrežica različne frekvenčne komponente razkloni, druga pa ponovno
zbere, vendar so pri tem dolžine optičnih poti za različne komponente različno dolge.
Pokazali bomo, da je celoten učinek enak kot pri razširjanju sunka po sredstvu z negativno
disperzijo. Vzporednost uklonskih mrežic zagotavlja vzporednost izhodnih žarkov,
vendar so različne komponente vpadne svetlobe med seboj razmaknjene (slika~\ref{fig:comp}\,a).
V praksi zato uporabimo ali dva para uklonskih mrežic ali pa 
zrcalo, ki svetlobo usmeri po isti poti nazaj. 
\begin{figure}[h]
\centering
\def\svgwidth{120truemm} 
\input{slike/10_comp.pdf_tex}
\caption{Kompenzacija disperzije z uklonskima mrežicama (a) in shema z
oznakami (b).}
\label{fig:comp}
\end{figure}

Naj na par vzporednih uklonskih mrežic vpada ravni val pod kotom $\alpha$, odbije
pa naj se pod kotom $\beta = \beta(\omega)$ (slika~\ref{fig:comp}\,b). 
Pot, ki jo prepotuje žarek od vpada na mrežico 
do izhoda iz sistema (med točkama $A$ in $B$), je enaka 
\begin{equation}
P = \frac{L}{\cos\beta} \left(1+\cos(\alpha + \beta)\right).
\end{equation}
Zaradi uklona velja zveza $\sin\,\alpha - \sin\,\beta = \lambda/\Lambda$,
pri čemer je $\lambda$ valovna dolžina svetlobe, $\Lambda$ pa perioda uklonske mrežice. 
Pri fazi moramo upoštevati še fazni premik na drugi mrežici
\begin{equation}
\Phi_m=\frac{2\pi}{\Lambda} \, L \, \tan\,\beta.
\end{equation}
Celotna sprememba faze je tako $\Phi = \omega P/c + \Phi_m$.

\begin{definition}
\label{nal:dispk}
Pokaži, da je drugi odvod faze po kotni frekvenci enak
\begin{equation}
\frac{d^2 \Phi}{d \omega^2} = - \frac{L\, c\, q^2}
{\sqrt{\omega^2 - (\omega\, \sin\alpha + cq)^2}^{3/2}},
\label{eq:10faza}
\end{equation}
pri čemer je $q = 2 \pi/\Lambda$.
\end{definition}
Račun v nalogi (\ref{nal:dispk})
pokaže, da je disperzija, ki je določena z drugim odvodom faze po kotni hitrosti 
(enačba~\ref{eq:10faza}), vedno negativna. Par vzporednih uklonskih mrežic
torej deluje kot sredstvo
z negativno disperzijo in sunek, ki se je razširil zaradi potovanja
po sredstvu s pozitivno disperzijo, lahko skrajša do meje,
določene s širino spektra. 

\begin{remark}
Postopek kompenzacije disperzije se uporablja za pridobivanje zelo 
kratkih sunkov. Sunku iz fazno uklenjenega barvilnega\index{Laser!organska barvila} 
ali Ti:safirnega\index{Laser!Ti:safir}\index{Uklepanje faz}
laserja se najprej v nelinearnem sredstvu razširi spekter, pri čemer
se sunek tudi časovno podaljša. Razširjen sunek se nato s parom mrežic skrajša
za faktor $10$--$100$ glede na prvotno dolžino sunka. Tako nastanejo sunki,
dolgi le okoli $10~\si{\femto\second}$, kar je le še nekaj optičnih period. 
V vmesni stopnji, ko je sunek podaljšan, ga lahko tudi dodatno ojačimo, česar
s prvotnim kratkim in že tako razmeroma močnim sunkom ne bi mogli narediti. Ojačan
sunek nato s parom uklonskih mrežic ponovno zberemo in nastane zelo kratek
zelo močen sunek svetlobe.
\end{remark}

\section{Izgube in ojačenje v optičnih vlaknih}
\index{Izgube v optičnih vlaknih}
Pri prenosu informacij z optičnimi vlakni je poleg disperzije, ki signal popači,
treba upoštevati tudi izgube, ki signal oslabijo. 
Do izgub pride predvsem zaradi absorpcije svetlobe v snovi,\index{Absorpcija}
Rayleighovega sipanja na termičnih fluktuacijah gostote, sipanja na nečistočah, 
izgub na stiku med vlakni in izgub zaradi upognjenosti vlakna. Za prenos na dolge
razdalje je tako potreben razmeroma močen signal, ki pa ne sme biti premočen,
saj lahko v vlaknu pride do nelinearnih optičnih pojavov. V praksi zato 
optični signal, ki potuje po čezoceanskih vlaknih, ojačujemo. 

Pri izdelavi optičnih vlaken se najpogosteje uporablja kremenovo steklo, ki 
ima najmanjšo absorpcijo svetlobe v bližnjem infrardečem območju 
($1300$--$1500~\si{\nano\meter}$).
Navadno mu dodamo primesi, s čemer dosežemo želen lomni količnik in 
zmanjšanje disperzije. 

Za merilo izgub v vlaknu vpeljemo atenuacijski 
koeficient\index{Atenuacijski koeficient}
\boxeq{dB}{
A [dB] = -10 \log_{10}\frac{j(z)}{j(0)},
}
pri čemer je $j(z)$ intenziteta svetlobe po prepotovani razdalji $z$. Če se po 
kilometru signal zmanjša na polovico, so izgube $3~\si{\decibel/\kilo\meter}$.
Dobra vlakna dosegajo pri valovni dolžini $1,55~\si{\micro\meter}$ izgube okoli 
$0,2~\si{\decibel/\kilo\meter}$. 
Za primerjavo: navadno steklo ima pri vidni svetlobi atenuacijski koeficient okoli 
$1000~\si{\decibel/\kilo\meter}$.

\begin{figure}[h]
\centering
\def\svgwidth{90truemm} 
\input{slike/10_FibAbs.pdf_tex} 
\caption{Izgube v vlaknu v odvisnosti od valovne dolžine: vijolična črta -- UV absorpcija, 
črna črta -- IR absorpcija, zelena črta -- absorpcija na OH ionih in modra črta --
izgube zaradi Rayleighovega sipanja. Z rdečo črto je označena skupna absorpcija.}
\label{FibAbs}
\end{figure}
Slika~(\ref{FibAbs}) prikazuje značilno odvisnost izgub od valovne dolžine 
za dobro enorodovno vlakno iz kremenovega stekla. \index{Optično vlakno!enorodovno}
Celotne izgube (rdeča črta)
so sestavljene iz vrste različnih prispevkov. 
Pri kratkih valovnih dolžinah je absorpcija velika zaradi elektronskih prehodov
v steklu (vijolična črta).\index{Ultravijolična svetloba} 
Širina reže za SiO$_2$\index{SiO$_2$} je namreč okoli $8,9$~eV, 
kar ustreza valovni dolžini\index{Infrardeče valovanje}
okoli $140~\si{\nano\meter}$. Pri velikih valovnih dolžinah pride do absorpcije zaradi
vibracijskih prehodov (črna črta). Čeprav so ti prehodi pri nižjih frekvencah, 
so vrhovi zelo široki in sežejo do okoli $1500~\si{\nano\meter}$. 
Absorpcija na nečistočah lahko ob pazljivi izdelavi postane skoraj v celotnem 
območju praktično zanemarljiva. 
Najbolj problematična nečistoča je voda oziroma OH$^{-}$ ioni, ki imajo velik dipolni
moment in izrazito absorpcijo pri $1380~\si{\nano\meter}$ (zelena črta). Zelo pomemben prispevek k 
absorpciji, posebej pri krajših valovnih dolžinah, je sipanje na fluktuacijah 
gostote (Rayleighovo sipanje),\index{Izgube v optičnih vlaknih!Rayleighovo sipanje} 
saj je sorazmerno z $\lambda^{-4}$ (modra črta). 

S slike je razvidno, da so skupne izgube najmanjše
okoli $1,55~\si{\micro\meter}$, zato se to območje največ uporablja za prenos signalov
na velike razdalje. Izgube so tako majhne, da omogočajo prenos signala 
do nekaj sto kilometrov brez vmesnega ojačevanja. Teh izgub na vlaknih se  
ne bo dalo več kaj dosti izboljšati, saj so že zdaj na meji,
določeni s termičnimi fluktuacijami. Pri dolžini optičnih zvez tako izgube niso več glavna
omejitev, ampak je to popačitev signala zaradi disperzije.

\begin{remark}
Pri prenosu signalov z optičnimi vlakni vpeljemo različne pasove, ki ustrezajo 
različnim valovnim dolžinam. Pri valovnih dolžinah $1260$--$1360~\si{\nano\meter}$ je tako imenovani
pas O (O -- {\it original}), ki so ga sprva uporabljali zaradi razpoložljivih virov svetlobe
in nizke disperzije. Sledita pas E ($1360$--$1460~\si{\nano\meter}$) in pas S ($1460$--$1530~\si{\nano\meter}$). 
Najširše uporabljan je pas C (C -- {\it conventional}) pri valovnih dolžinah $1530$--$1565~\si{\nano\meter}$,
sledita mu še pas L ($1565$--$1625~\si{\nano\meter}$) in pas U ($1625$--$1675~\si{\nano\meter}$).

Po optičnem vlaknu lahko prenašamo več informacij, če za vsako posebej uporabimo
drugo valovno dolžino. Temu procesu pravimo multipleksiranje po valovni dolžini
(WDM -- {\it wavelength-division multiplexing}).\index{Multipleksiranje}
Na ta način dosežemo vzporeden prenos podatkov in hitrosti do 100~Tb/s.
Shematsko je tak način prenosa podatkov prikazan na sliki~(\ref{WDM}).
Oddajniki (O) oddajo sunke svetlobe, ki se rahlo razlikujejo v valovni dolžini. 
Z multiplekserjem (M) signale iz različnih kanalov zberemo in jih usmerimo v 
enorodovno vlakno. Vlakno prenaša signal, vmes ga po potrebi ojačimo (A), 
nato z demultiplekserjem (DM) signal razstavimo na posamezne kanale, ki jih 
zaznamo z ločenimi detektorji (D).
\begin{figure}[h]
\centering
\def\svgwidth{120truemm} 
\input{slike/10_WDM.pdf_tex} 
\caption{Shematski prikaz prenosa več signalov hkrati po enorodovnem vlaknu}
\label{WDM}
\end{figure}
\end{remark}

\subsection*{*Izgube v ukrivljenem vlaknu}
\index{Izgube v optičnih vlaknih!ukrivljeno vlakno}
Pri vseh primerih do zdaj smo upoštevali, da je vlakno ravno 
oziroma da so mejne ploskve valovnega vodnika vzporedne. 
Kadar je vlakno ukrivljeno, del valovanja uhaja v plašč in 
izgube pri prenosu se povečajo. Te izgube tipično postanejo znatne, 
kadar je krivinski radij ukrivljenega vlakna centimeter ali manj. 
Poglejmo si pojav podrobneje na planparalelnem vodniku.
\index{Optični vodnik!planparalelni}

Naj bo vodnik dvodimenzionalna plast debeline $a$ z lomnim količnikom
$n_{1}$, ki je obdana s snovjo z lomnim količnikom $n_{2}$. Vodnik naj
zdaj ne bo raven, temveč ukrivljen s krivinskim radijem $R$, tako da tvori 
del kolobarja z notranjim radijem $R-a/2$ in zunanjim radijem $R+a/2$, pri 
čemer je $R\gg a$ (slika~\ref{fig:bend}). 
\begin{figure}[h]
\centering
\def\svgwidth{60truemm} 
\input{slike/10_Krivina.pdf_tex} 
\caption{K izračunu izgub v ukrivljenem vodniku}
\label{fig:bend}
\end{figure}

Zapišimo Helmholtzevo enačbo (enačba~\ref{eq:Helmholtz}) v cilindrični \index{Helmholtzeva enačba}
geometriji
\begin{equation}
\frac{1}{r}\,\frac{\partial}{\partial r}\, r\,\frac{\partial E}{\partial r}
+\frac{1}{r^{2}}\,\frac{\partial^{2}E}{\partial\varphi^{2}}+k_{0}^{2}n^{2}\left(r\right)\, E=0,
\label{9.31}
\end{equation}
pri čemer ima $n\left(r\right)$ vrednost $n_{1}$ v sredici in $n_{2}$ v plašču. 
Pri tem ne pozabimo, da $r$ ni več radialna koordinata vlakna, ampak
označuje oddaljenost od središča krivine. Zanimajo nas rešitve oblike 
\begin{equation}
E(r, \varphi) =\psi\left(r\right)\, e^{im\varphi}
\label{9.32}
\end{equation}
kjer bomo privzeli, da je $\psi\left(r\right)$ znatna le v sredici. 

Naj bo $z=R\varphi$ dolžina loka vzdolž sredine sredice. Tedaj je faza nastavka
(enačba~\ref{9.32}) enaka $m\varphi = m z/R$ in valovno število $\beta = m/R$.
Ker je valovna dolžina svetlobe dosti manjša od $R$, je $m$ zelo veliko število. 
Funkcija $\psi$ zadošča enačbi 
\begin{equation}
\frac{d^{2}\psi}{dr^{2}}+\frac{1}{r}\,\frac{d\psi}{dr}+\left(k_{0}^{2}\, 
n^{2}\left(r\right)-\frac{m^{2}}{r^{2}}\right)\psi=0.
\label{9.33}
\end{equation}
Rešitve za $\psi$ so kombinacije Besslovih funkcij reda $m$, kar
pa zaradi velikosti $m$ ni posebno zanimivo. 

Dosti več bomo izvedeli, če se problema lotimo malo drugače. Namesto $r$
in $\varphi$ vpeljemo koordinati $x=r-R$ in $z=R\varphi$.
S tem preidemo nazaj na koordinate planparalelne plasti in iščemo popravke valovne
enačbe v sredici (\ref{9.3a}), ki so reda $1/R.$ Privzamemo, da je $1/r\approx1/R$ in 
\begin{equation}
\frac{m^{2}}{r^{2}}=\frac{m^{2}}{\left(R+x\right)^{2}}\approx\frac{m^{2}}
{R^{2}}\,\left(1-2\,\frac{x}{R}\right)=\beta^{2}\left(1-2\,\frac{x}{R}\right).
\label{9.34}
\end{equation}
Z upoštevanjem gornjih zvez enačbo (\ref{9.33}) nadomestimo s približno enačbo za prečno obliko
polja 
\begin{equation}
\frac{d^{2}\psi}{dx^{2}}+\left(k_{0}^{2}\, n^{2}\left(r\right)-\beta^{2}\right)\,\psi+\frac{1}{R}\,
\left(\frac{d\psi}{dx}+2\,\beta^{2}x\,\psi\right)=0.
\label{9.35}
\end{equation}
Člen, ki vsebuje prvi odvod $d \psi/d x$, lahko odpravimo z nastavkom 
\begin{equation}
\psi(x) = e^{-x/2R} \zeta(x)
\end{equation}
in dobimo 
\begin{equation}
\frac{d^{2}\zeta}{dx^{2}}+\left(k_{0}^{2}\, n^{2}\left(r\right)-\beta^{2}-\frac{1}{4R^2}\right)\,\zeta
+ \frac{2\beta^{2}}{R}\,x\,\zeta=0.
\label{9.35}
\end{equation}
Gornja enačba je zelo podobna enačbi za izračun lastnih rodov v planparalelnem
vodniku (\ref{9.3a}), pri čemer se $\beta^2$ poveča za $1/4R^2$. Poleg tega je prisoten
dodatni člen, ki je linearno sorazmeren z $x$. Če ponovno naredimo analogijo
med lastnimi načini v valovnem vodniku in stanji delca, ujetega v končno potencialno jamo, 
je ta člen enake oblike kot potencialna energija delca v konstantnem zunanjem električnem 
polju (slika~\ref{fig:tunel}). Podobno kot lahko ujeti delci uhajajo iz potencialne jame
(tunelirajo), uhaja tudi svetloba iz ukrivljenega vlakna.

Izgube (oziroma delež prepuščene 
svetlobe) so eksponentno odvisne od krivinskega radija vlakna 
\begin{equation}
A \propto e^{-CR},
\end{equation}
pri čemer je $C$ konstanta, odvisna od lomnih količnikov sredice in plašča, od polmera vlakna 
ter od valovne dolžine potujoče svetlobe.
\begin{figure}[h]
\centering
\def\svgwidth{120truemm} 
\input{slike/10_Tunel.pdf_tex} 
\caption{Lastni načini širjenja svetlobe po ravnem vodniku so analogni stanjem 
delca v končni potencialni jami (levo), načini širjenja po ukrivljenem vodniku pa 
so podobni stanjem delca v konstantnem zunanjem električnem polju (desno). Zaradi
spremenjenega potenciala delci tunelirajo, svetloba pa uhaja iz ukrivljenega vlakna.}
\label{fig:tunel}
\end{figure}

\subsection*{Izgube na spoju dveh vlaken}
\index{Izgube v optičnih vlaknih!spoj dveh vlaken}
Omenili smo že, da pride do izgub tudi na spoju dveh vlaken. 
V idealnem primeru sta dve stikajoči vlakni povsem enaki in se sredici povsem ujemata. 
Čim se pojavijo majhna odstopanja v velikosti ali poravnavi sredice, na stiku pride 
do izgub, na tipičnem spoju vlaken do okoli $0,2$--$0,5~\si{\decibel}$.

Omejimo se na spoj enorodovnih vlaken, v katerih krajevni del 
električne poljske jakosti osnovnega roda zapišemo kot
\begin{equation}
E(r, \varphi, z)=\psi(r, \varphi) e^{i\beta z}.
\end{equation} 

Za izračun sklopitve med dvema vlaknoma vpeljemo prekrivalni integral,\index{Prekrivalni integral}
ki pove, kolikšen delež moči iz prvega vlakna 
se sklopi v osnovni rod v drugem vlaknu. Sklopitveni faktor je
\boxeq{10:overlap}{
\eta = \frac{|\int \psi_1(r, \varphi) \psi_2^*(r, \varphi) r\, dr\, d\varphi|^2}
{\left(\int |\psi_1|^2 r\, dr\, d\varphi \right) \left(\int |\psi_2|^2 r\, dr\, d\varphi \right)},
}
pri čemer z indeksom $1$ označimo osnovni rod v prvem vlaknu, z indeksom $2$ pa 
v drugem. Točen izračun sklopitve je v splošnem precej zapleten, saj vsebuje integrale
Besslovih funkcij. Zato račun poenostavimo, tako 
da prečno odvisnost osnovnega roda v cilindričnem vlaknu nadomestimo z 
Gaussovo funkcijo in uporabimo efektivni polmer snopa (enačba~\ref{Marcuse}).\index{Gaussov 
snop!efektivni polmer}

Izračunajmo za primer sklopitveni faktor in izgube na stiku dveh vlaken z rahlo 
različnima polmeroma. Po Marcusejevi formuli najprej določimo efektivna polmera Gaussovih snopov
v obeh vlaknih $w_1$ in $w_2$. Prečni profil v prvem vlaknu je tako\index{Marcusejeva formula}
\begin{equation}
\psi_1 = A_1 e^{-r^2/w_1^2},
\end{equation}
v drugem pa 
\begin{equation}
\psi_2 = A_2 e^{-r^2/w_2^2}.
\end{equation}
Vstavimo gornja nastavka v enačbo~(\ref{10:overlap}) in zapišemo
\begin{equation}
\eta = \frac{|\int A_1 \, e^{-r^2/w_1^2}\, A_2\, e^{-r^2/w_2^2}\, 2 \pi\, r\, dr|^2}
{\left(\int A_1^2 \,e^{-2r^2/w_1^2} \, 2 \pi \, r\, dr \right) \left( A_2^2\, 
e^{-2r^2/w_2^2}\, 2 \pi \, r\, dr \right)},
\end{equation}
od koder sledi
\begin{equation}
\eta = \frac{4 w_1^2 w_2^2}{(w_1^2+w_2^2)^2}.
\end{equation}
Kadar sta polmera vlaken enaka, je prekrivanje popolno in $\eta = 1$. Z naraščajočo razliko
med polmeroma pa vrednost $\eta$ pojema. Pri tem
ni pomembno, ali ima večji polmer prvo ali drugo vlakno, v obeh primerih pride do izgube 
signala. Intuitivno razumemo, da se signal izgubi pri prehodu iz večjega v manjše vlakno, 
obratno pa je tudi res, saj se v širšem končnem vlaknu poleg osnovnega vzbudijo
tudi višji rodovi. 

Pri prehodu iz vlakna z $w = 10~\si{\micro\meter}$ v vlakno
s polmerom $w = 8~\si{\micro\meter}$ (ali obratno), je sklopitveni faktor enak $0,95$. 
To je enako razmerju prepuščene intenzitete svetlobe, zato lahko izračunamo izgube kar z
enačbo~(\ref{dB}). Izračunana sklopitev ustreza izgubam $0,21~\si{\decibel}$. 

\begin{definition}
Pokaži, da je sklopitveni faktor za dve enaki vzporedni vlakni, ki sta iz osi izmaknjeni
za $\Delta$, enak
\begin{equation}
\eta = \exp \left( - \frac{\Delta^2}{w^2}\right).
\end{equation}

\end{definition}

\subsection*{Ojačevanje v vlaknih}
\index{Optično ojačevanje!v vlaknih}
Zaradi znatnih izgub pri prenosu signalov na več tisoč kilometrov dolge razdalje 
je treba signal med prenosom ojačevati. To lahko naredimo elektronsko, kjer optični signal 
pretvorimo v električnega, tega ojačimo in ga nato pretvorimo nazaj v optičnega. 
Precej bolj priročna rešitev je optično ojačevanje v vlaknu samem. 

V ta namen se najpogosteje uporablja vlakno, dopirano z erbijevimi 
ioni\footnote{EDFA - {\it Erbium-doped fiber amplifier}, \index{Optično vlakno!dopirano z erbijem}
\index{Erbij}
ojačevalnik na vlakno, dopirano z erbijem}. 
Na določenih razdaljah (na okoli $100~\si{\kilo\meter}$) svetlobo iz navadnega vlakna 
sklopimo v vlakno, v katerem so erbijevi ioni.
S črpalnim laserjem erbijeve ione vzbudimo, da dosežemo obrnjeno zasedenost. \index{Obrnjena zasedenost}
Ko na dopirani del vlakna vpade svetlobni sunek z valovno dolžino okoli 
$1550~\si{\nano\meter}$, pride do stimulirane emisije in \index{Stimulirano sevanje}
sunek se ojači. Gre za povsem enak princip ojačevanja svetlobe, kot ga poznamo iz 
delovanja laserja, le da tukaj svetloba ni ujeta v resonator, ampak se postopoma 
ojačuje vzdolž vlakna. Pri tem se intenziteta črpalnega laserja postopoma zmanjšuje,
kar omejuje dolžino, na kateri se signal ojačuje.
Spektralna širina ojačenja je zaradi sklopitev z ioni v steklu 
razmeroma široka, tudi $40~\si{\nano\meter}$. To pomeni, da se hkrati ojačujejo signali različnih 
valovnih dolžin, kar je še posebej uporabno pri prenosu več signalov naenkrat.

V praksi se uporabljajo vlakna, v katerih je delež erbijevih ionov okoli $\sim 10^{-4}$. 
Črpalni laser je polprevodniški laser, ki 
deluje pri valovni dolžini $900~\si{\nano\meter}$ ali $1,48~\si{\nano\meter}$ 
z močjo okoli $20$--$100~\si{\milli\watt}$. Na ta način lahko v $10$--$30~\si{\meter}$
dolgih odsekih dosežemo več 1000-kratno ojačenje ($30$--$40~\si{\decibel}$).

\section{Sklopitev svetlobe v vlakna}
\index{Sklopitev v optično vlakno}
Do zdaj smo govorili o svetlobi, ki potuje po valovnem vodniku ali optičnem vlaknu. Kako pa 
svetlobo sploh sklopimo v vodnik? Poznamo več načinov sklopitve, obravnavali bomo 
čelno sklopitev, bočni sklopitvi s prizmo in s periodično strukturo ter  
sklopitev med dvema vzporednima vodnikoma. Prvi način navadno uporabljamo pri 
cilindričnih vlaknih, ostale pa najpogosteje pri planarnih valovodnih strukturah.

\subsection*{Čelna sklopitev}
\index{Sklopitev v optično vlakno!čelna sklopitev}
Pri izračunu izgub na spoju dveh vlaken smo vpeljali prekrivalni integral (enačba~\ref{10:overlap}), 
ki pove, kolikšen del osnovnega roda, ki izhaja iz prvega vlakna, se sklopi v osnovni 
rod drugega. Povsem podobno pristopamo tudi pri izračunu izkoristka sklopitve svetlobe v optično vlakno.
\index{Prekrivalni integral}
Izračunati moramo, kolikšen delež vpadne svetlobe $E(r, \varphi)$ se sklopi z izbranim rodom 
vlakna $n,m$. To naredimo tako, da vpadni val razvijemo po lastnih rodovih vlakna, iz ortogonalnosti
rodov pa sledi prekrivalni integral, ki ga moramo seveda ustrezno normirati. Zapišemo ga kot
\begin{equation}
\eta = \frac{|\int E(r, \varphi) E^*_{n,m}(r, \varphi) r\, dr\, d\varphi|^2}
{\left(\int |E(r, \varphi)|^2 r\, dr\, d\varphi \right) \left(\int |E_{n,m}(r, \varphi)|^2 
r\, dr\, d\varphi \right)},
\end{equation}
oziroma v kartezičnih koordinatah
\begin{equation}
\eta = \frac{|\int E(x,y) E^*_{n,m}(x,y) dx\, dy|^2}
{\left(\int |E(x, y)|^2 dx\,dy \right) \left(\int |E_{n,m}(x, y)|^2 
dx\, dy \right)}.
\end{equation}
Če računamo sklopitev v osnovi rod vlakna, ga tudi tukaj iz praktičnih razlogov nadomestimo 
z Gaussovim profilom.

\begin{remark}
Sklopitev svetlobe v večrodovno vlakno lahko obravnavamo geometrijsko, kot smo to naredili
na začetku poglavja (slika~\ref{fig:vodnik}). Izračunali smo (enačba~\ref{10NAa}), da je največji 
vpadni kot, pod katerim se svetloba še sklopi v vlakno, določen z numerično odprtino 
vlakna $\sin \alpha_{\mathrm{max}}= NA$.
\end{remark}

\subsection*{Bočna sklopitev}
\index{Sklopitev v optično vlakno!bočna sklopitev}
Neposredna sklopitev svetlobe v vodnik preko plašča ni mogoča. Ker je namreč lomni količnik
sredice večji od lomnega količnika plašča, dovolj velikega vstopnega kota, pod katerim bi 
se svetloba ujela v sredico, ni mogoče doseči. Zato je za sklopitev preko stranice treba 
uporabiti drugačen pristop. Navadno uporabimo prizmo ali pa periodično strukturo na vlaknu. 

\begin{figure}[h]
\centering
\def\svgwidth{120truemm} 
\input{slike/10_coupler.pdf_tex} 
\caption{Dva primera bočne sklopitve svetlobe v valovni vodnik: sklopitev s prizmo (a) in sklopitev
s periodično strukturo (b)}
\label{fig:coupler}
\end{figure}
V prvem primeru uporabimo prizmo, kot kaže slika~(\ref{fig:coupler}\,a). Lomni količnik prizme
je pri tem večji od lomnega količnika plašča $n_p > n_2$.
Vhodni žarek vpada na prizmo, se ob prehodu vanjo lomi, nato pa se na stranici, ki je vzporedna
z vodnikom, totalno odbije.\index{Totalni odboj} 
V vmesnem območju med prizmo in sredico vodnika se pojavi evanescentni\index{Evanescentno polje}
val z valovnim vektorjem $\beta  = k_0 n_p \sin \alpha$ v smeri vzporedno z vodnikom. Pogoj za uspešno 
sklopitev je ujemanje vzdolžne komponente valovnega vektorja vpadne svetlobe z valovnim 
vektorjem rodu v vodniku, ki ga želimo vzbuditi. 
S spreminjanjem vpadnega kota lahko torej v vodniku vzbujamo različne rodove. 
Če je razdalja med prizmo in vodnikom dovolj majhna (tipično pod valovno dolžino svetlobe), se v 
valovod ob izpolnjenem pogoju ujemanja faze sklopi znaten delež vpadne svetlobe.

Tudi periodična struktura na valovnem vodniku (slika~\ref{fig:coupler}\,b) deluje na 
ujemanju valovnega vektorja vpadnega vala z valovnim vektorjem ustreznega rodu.
Ko vpade val pod kotom $\alpha$ glede na valovni vodnik, periodična struktura na vodniku 
spremeni njegovo fazo za večkratnik $2 \pi z/\Lambda$, pri čemer je $\Lambda$ perioda strukture.
Če dosežemo, da se nov valovni vektor $\beta = k_0 n_2 \sin \alpha+ 2 \pi/\Lambda$ izenači z valovnim 
vektorjem za izbrani rod v vlaknu, se vpadna svetloba sklopi v vlakno.

\begin{remark}
 Oba opisana načina za sklopitev svetlobe v vlakno lahko uporabimo tudi za sklopitev svetlobe 
 iz vlakna, pri čemer mora biti ravno tako izpolnjen pogoj ujemanja faz. 
 Sklapljanje svetlobe skozi prizmo je uporabno tudi za raziskave tankih plasti snovi. 
 Iz pogoja za ujemanje faz lahko določimo lastnosti tanke plasti, njeno debelino ali njen lomni količnik. 
\end{remark}

\subsection*{Sklopitev med valovodi}
\index{Sklopitev med valovodi}
Ob prenosu signala po optičnem valovodu večina energijskega toka potuje po sredici,
energijski tok pa seže tudi izven nje, v plašč (enačba~\ref{confinement}). 
Če sta dva vzporedna valovoda dovolj blizu, da\index{Evanescentno polje}
se evanescentni električni polji enega in drugega vodnika v plašču prekrivata, pride do 
sklopitve vodnikov in prenosa energijskega toka iz enega vodnika v drugega. 
\begin{figure}[h]
\centering
\def\svgwidth{80truemm} 
\input{slike/10_fcoupler.pdf_tex} 
\caption{Sklopitev med dvema vzporednima vodnikoma}
\label{fig:fcoupler}
\end{figure}

Za podrobnejšo obravnavo bi morali zapisati Maxwellove enačbe z ustreznimi robnimi pogoji 
in jih rešiti za sklopljen primer dveh vzporednih vodnikov. Tak račun je izredno zapleten, 
zato se bomo poslužili približka šibke sklopitve in privzeli, da so rodovi v vodnikih taki, 
kot če bi vodniki ne bili sklopljeni.
Sklopitev torej ne bo spremenila oblike lastnih rodov, bo pa spremenila njihovo amplitudo, ki 
bo tako postala odvisna od vzdolžne koordinate $z$.

Imejmo dva enorodovna vodnika z lomnima količnikoma sredice $n_1$ in $n_2$ in enako 
debelino $a$, med njima in okoli njiju pa naj bo snov z lomnim količnikom $n_0$. Širina 
reže med vodnikoma naj bo $2d$.
Potem zapišemo električni poljski jakosti v prvem in drugem vodniku kot
\begin{eqnarray}
E_1(x,z) &=& A(z) \psi_1(x) e^{i \beta_1 z} \quad \mathrm{in}\\
E_2(x,z) &=& B(z) \psi_2(x) e^{i \beta_2 z},
\label{eq:10_ampl}
\end{eqnarray}
pri čemer se $A(z)$ in $B(z)$ le počasi spreminjata s koordinato $z$. Skupna električna poljska
jakost, ki je v našem približku kar vsota prispevkov $E_1$ in $E_2$, 
mora zadoščati Helmholtzevi enačbi
(enačba~\ref{eq:Helmholtz})
\begin{equation}
\nabla^{2}E(x,z)+k_0^{2}n(x)^2 E(x,z) =0.
\end{equation}
Pri tem smo z $n(x)$ označili prečno odvisnost lomnega količnika.
Vstavimo nastavek za električno poljsko jakost v gornjo enačbo in dobimo
\begin{multline}
A e^{i \beta_1 z}\left(\frac{\partial^2}{\partial x^2}\psi_1(x) - \beta_1^2\psi_1 + k_0^2
n(x)^2 \psi_1 \right)
+ 
B e^{i \beta_2 z}\left(\frac{\partial^2}{\partial x^2}\psi_2(x) - \beta_2^2\psi_2 + k_0^2
n(x)^2 \psi_2 \right)+ \\2 i \beta_1 A' \psi_1 e^{i \beta_1 z}+
2 i \beta_2 B' \psi_2 e^{i \beta_2 z} = 0.
\end{multline}
Pri tem smo člena z drugim odvodom $\partial^2 A/\partial z^2$ in $\partial^2 B/\partial z^2$
zanemarili. Zapišimo enačbi za $\psi$
\begin{eqnarray}
\frac{\partial^2}{\partial x^2}\psi_1(x) + \left(k_0^2n_1(x)^2-\beta_1^2\right) \psi_1 &=&0 \quad \mathrm{in} \\
\frac{\partial^2}{\partial x^2}\psi_2(x) + \left(k_0^2n_2(x)^2-\beta_2^2\right) \psi_2 &=&0.
\end{eqnarray}
Lomna količnika $n_1(x)$ in $n_2(x)$ sta tukaj tudi funkciji prečne koordinate. Naj bo $n_1(x)$  
povsod enak $n_0$ razen v sredici prvega vodnika, kjer je $n_1$, in naj bo 
$n_2(x)$ povsod enak $n_0$, razen v sredici drugega vlakna, kjer je enak $n_2$. Sledi
\begin{multline}
A e^{i \beta_1 z}k_0^2\left(n(x)^2 -n_1(x)^2\right)\psi_1 
+ 
B e^{i \beta_2 z}k_0^2\left(n(x)^2 -n_2(x)^2\right)\psi_2 + \\2 i \beta_1 A' \psi_1 e^{i \beta_1 z}+
2 i \beta_2 B' \psi_2 e^{i \beta_2 z} = 0.
\end{multline}
Enačbo pomnožimo s kompleksno konjugirano vrednostjo $\psi_1^*$ in integriramo po $x$.
Upoštevamo ortonormalnost funkcij $\psi$ in zapišemo 
\begin{equation}
\frac{dA}{dz} = i A K_{11}+i B e^{i(\beta_2-\beta_1)z} K_{12},
\end{equation}
pri čemer sta
\begin{equation}
K_{11}= \frac{k_0^2}{2 \beta_1}\int\psi_1^*\psi_1 (n^2-n_1^2)dx
\end{equation}
in 
\begin{equation}
K_{12}= \frac{k_0^2}{2 \beta_1}\int\psi_1^*\psi_2 (n^2-n_2^2)dx.
\end{equation}
Koeficient $K_{11}$ določa spremembo amplitude v vlaknu zaradi prisotnosti 
drugega vlakna, kar lahko zanemarimo. Tako ostane samo sklopitveni člen
\begin{equation}
\frac{dA}{dz} = i B e^{i(\beta_2-\beta_1)z} K_{12}
\label{10_B}
\end{equation}
in podobno za $B$
\begin{equation}
\frac{dB}{dz} = i A e^{i(-\beta_2+\beta_1)z} K_{21}.
\end{equation}
Prvo enačbo odvajamo, upoštevamo drugo in dobimo
\begin{equation}
\frac{\partial^2 A}{\partial z^2}-i \Delta \beta \frac{\partial A}{\partial z} + K_{12}K_{21}A = 0.
\label{10_part}
\end{equation}
Enačbo rešujemo z nastavkom 
\begin{equation}
A = e^{i \Delta \beta z/2}\left( a_1 e^{i \gamma z} + a_2 e^{-i \gamma z}\right).
\label{10_nastavek}
\end{equation}
\begin{definition}
 Pokaži, da  nastavek (enačba~\ref{10_nastavek}) reši enačbo~(\ref{10_part}) in pokaži,
 da med parametri enačb velja sledeča zveza
 \begin{equation}
 \gamma^2 = K^2 + \frac{\Delta \beta ^2}{4},
 \end{equation}
 pri čemer je $K = \sqrt{K_{12}K_{21}}$ in $\Delta \beta = \beta_2 - \beta_1$. 
 Nato uporabi enačbo~(\ref{10_B}) in pokaži, da je rešitev za amplitudo $B$
 enaka izrazu v enačbi~(\ref{10_BB}).
\end{definition}
Ko poznamo $A$, lahko z uporabo enačbe~(\ref{10_B}) izračunamo še $B$
\begin{equation}
B = \frac{1}{K_{21}}
e^{-i \Delta \beta z/2}\left(\left(\frac{\Delta \beta}{2} +\gamma \right) a_1 e^{i \gamma z} + 
\left(\frac{\Delta \beta}{2} -\gamma \right)a_2 e^{-i \gamma z}\right).
\label{10_BB}
\end{equation}
Naj bo $A(z=0) = A_0$ in $B(z=0)=0$. To pomeni, da potuje svetloba na začetku
le po prvem vlaknu, potem pa se sklopi v drugega. S tema začetnima pogojema zapišemo izraza 
za $A$ in $B$
\begin{eqnarray}
A &=& A_0 e^{i \Delta \beta z/2} \left( \cos(\gamma z) - 
\frac{i \Delta \beta}{2 \gamma}\sin(\gamma z) \right)\\
B &= & A_0 e^{-i \Delta \beta z/2} \frac{i K_{21}}{\gamma}\sin(\gamma z).
\end{eqnarray}
Moč, ki se pretaka po posameznem vlaknu je tako
\begin{eqnarray}
P_1 &=& P_0 \left( \cos^2(\gamma z) + \frac{\Delta \beta^2}{4 \gamma^2}\sin^2(\gamma z) \right)
= P_0 \left( 1 - \frac{K^2}{\gamma^2}\sin^2(\gamma z) \right)\\
P_2 &= & P_0\, \frac{K^2}{\gamma^2}\sin^2(\gamma z),
\end{eqnarray}
pri čemer smo privzeli, da velja $K_{12} \sim K_{21} \sim K$. 
Obe funkciji sta oscilirajoči in svetloba se periodično pretaka med vlaknoma
s periodo $\pi/\gamma$ (slika~\ref{fig:foscil}). Amplituda prenosa je 
odvisna od sklopitvenega faktorja $K$ in ujemanja valovnih vektorjev v obeh vlaknih. Večji koeficient $K$
in manjše odstopanje $\Delta \beta$ vodita v večji prenos svetlobnega toka v drugo vlakno. 
\begin{figure}[h!]
\centering
\def\svgwidth{140truemm} 
\input{slike/10_Coupler.pdf_tex} 
\caption{Prenos svetlobnega toka med dvema sklopljenima vodnikoma. V prvem primeru (a) sta
vodnika različna, v drugem primeru (b) pa sta vodnika enaka in pride do popolnega prenosa.}
\label{fig:foscil}
\end{figure}

Če sta vlakni 
enaki, je $\Delta \beta = 0$ in $\gamma = K$, tako da pride do popolnega prenosa
svetlobnega toka iz enega vlakna v drugo in seveda tudi obratno.
Takrat veljata enačbi
\boxeq{10_couplfib}{
P_1 &= P_0 \cos^2 (\gamma z) \quad \mathrm{in}\\
P_2 &= P_0 \sin^2 (\gamma z).
}

Na ta način lahko v drugo vlakno sklopimo poljuben delež vpadne svetlobe. 
Pri dolžini sklopitve $L = \pi/2 \gamma$ pride do celotnega prenosa svetlobnega toka v drugo vlakno.
Pogosto sklopimo le eno polovico intenzitete vpadne svetlobe. To se zgodi pri dolžini 
$L = \pi/4 \gamma$ in takrat govorimo o 3-dB sklopitvi. \index{Sklopitev med valovodi!3-dB sklopitev}

\begin{remark}
 Pri dani dolžini sklopitve med vlaknoma $L$ je intenziteta svetlobe v drugem vlaknu močno odvisna od
 parametra $\gamma$, to je od sklopitve med vlaknoma in od razlike med valovnima vektorjema 
 v obeh vlaknih. Zgolj z rahlim spreminjanjem parametrov, na primer lomnega količnika enega od vlaken,
 lahko močno vplivamo na delež sklopljenega svetlobnega toka. S priključeno napetostjo in elektro-optičnim
 pojavom v enem izmed vlaken lahko tako zelo natančno spreminjamo delež sklopljene svetlobe. 
\end{remark}

\section{*Vpliv spremembe lomnega količnika vlakna na širjenje svetlobe}

Sprememba lomnega količnika sredice ali plašča vlakna povzroči spremembo
valovnega števila $\beta$ za izbran rod. V enorodovnih vlaknih je to
mogoče izkoristiti za izdelavo senzorjev, na primer temperature ali
tlaka. Spremembo valovnega števila, do katere pride zaradi zunanjih vplivov,
izmerimo preko spremembe faze valovanja na izhodu iz vlakna, to je,
z ustrezno sestavljenim interferometrom. Ker je dolžina vlakna lahko
velika (v nekaj centimetrov velik tulec lahko brez težav navijemo
kilometre vlakna), je celotna sprememba faze velika že pri majhnih
spremembah merjene količine. Sprememba valovnega števila povzroča
tudi neželene spremembe faze in odboje pri prenosu informacij.
V tem razdelku zato poglejmo, kako se spremeni valovno število pri
dani spremembi lomnega količnika in koliko svetlobe se odbije.

Obravnavajmo rod vlakna z vzdolžno komponento valovnega vektorja $\beta_{lm}$ in prečnim
profilom $\psi_{lm}\left(r,\varphi\right).$ Ta mora zadoščati Helmholtzevi enačbi 
(enačba~\ref{eq:Helmholtz})
\begin{equation}
\nabla_{\bot}^{2}\psi_{lm}+\left(\epsilon(r)k_{0}^{2}-\beta_{lm}^{2}\right)\psi_{lm}=0.
\label{9.22}
\end{equation}
Naj se dielektrična konstanta na delu vlakna spremeni za $\delta\epsilon.$
Posledično se spremenita tudi valovno število $\beta=\beta_{lm}+\delta\beta$
in prečna oblika $\psi=\psi_{lm}+\delta\psi.$ Tudi popravljena funkcija
$\psi$ mora zadoščati enačbi (\ref{9.22}), zato za perturbacijo velja
\begin{equation}
\nabla_{\bot}^{2}\delta\psi+\left(\epsilon(r)k_{0}^{2}-\beta_{lm}^{2}\right)\delta\psi+
\delta\epsilon\, k_{0}^{2}\psi_{lm}=2\beta_{lm}\delta\beta\,\psi_{lm},
\label{9.23}
\end{equation}
pri čemer smo zanemarili produkte majhnih količin. Množimo obe strani
enačbe s $\psi_{lm}^{*}$, integrirajmo po preseku vlakna in dobimo
\begin{eqnarray}
 &  & \int\psi_{lm}^{*}\nabla_{\bot}^{2}\delta\psi\,
 dS+\int\left(\epsilon(r)k_{0}^{2}-\beta_{lm}^{2}\right)
 \delta\psi\,\psi_{lm}^{*}+k_{0}^{2}\int\delta\epsilon\,\left|\psi_{lm}\right|^{2}dS\label{9.24}\\
 & = & 2\beta_{lm}\,\delta\beta\int\left|\psi_{lm}\right|^{2}dS.
\end{eqnarray}
Prvi člen na levi preoblikujmo z uporabo zvez $\int(u\,\nabla_{\bot}^{2}v-v\nabla_{\bot}^{2}u)\,
dS=\int\nabla_{\bot}\cdot(u\nabla_{\bot}v-v\nabla_{\bot}u)\, 
dS=\oint (u\,\nabla_{\bot}v-v\,\nabla_{\bot}u)\cdot \mathbf{ds}$.
Ker funkciji $\psi_{lm}$ in $\delta\psi$ opisujeta vodene valove,
morata iti za velike $r$ proti nič, zato je integral po krivulji
v gornji zvezi nič in velja $\int\psi_{lm}^{*}\nabla_{\bot}^{2}\delta\psi\,
dS=\int\delta\psi\nabla_{\bot}^{2}\psi_{lm}^{*}\, dS$.
Funkcija $\psi_{lm}^{*}$ zadošča enačbi (\ref{9.22}), zato se v enačbi (\ref{9.24})
prvi in drugi člen odštejeta. Iskan popravek k valovnem številu je tako 
\begin{equation}
\delta\beta=\frac{k_{0}^{2}\int\delta\epsilon\,\left|\psi_{lm}\right|^{2}dS}{2\,
\beta\int\left|\psi_{lm}\right|^{2}dS}.
\label{9.25}
\end{equation}
\begin{remark}
Ta rezultat je seveda analogen kvantno-mehanskemu rezultatu, ki ga dobimo 
z uporabo teorije motenj za spremembo energije lastnega stanja delca pri majhni 
spremembi Hamiltonovega operatorja. Rezultat je tudi intuitivno razumljiv: v
najnižjem redu je $\delta\beta$ sorazmerna s uteženim povprečjem
$\delta\epsilon$, pri čemer je utež $\psi_{lm}$.
\end{remark}

Sprememba valovnega števila $\beta$ v delu vlakna ne povzroči le spremembe faze, 
ampak tudi delni odboj.
To je le nekoliko druga oblika odboja na (zvezni ali ostri) meji 
dveh dielektrikov ali, splošneje, odboja valovanja na območju,
kjer se spremeni fazna hitrost valovanja.
Odboj na območju vlakna, kjer se spreminja $\beta$, najpreprosteje opišemo 
z enačbo za odboj na meji dveh dielektrikov pri pravokotnem
vpadu. Odbita amplituda je tedaj (enačba~\ref{eq:Fresnel1})
\begin{equation}
E_{r}=\frac{n_{2}-n_{1}}{n_{2}+n_{1}}E_{0}.
\label{9.26}
\end{equation}
Mislimo si, da je sprememba $\beta$ na delu vlakna sestavljena iz
majhnih stopničastih sprememb $\Delta\beta_{i}$ na intervalih $\Delta z$.
Za ravno valovanje je sprememba fazne hitrosti sorazmerna s spremembo
lomnega količnika, zato iz enačbe~(\ref{9.26}) sledi, da je delež odbitega
valovanja na stopničasti spremembi $\Delta\beta_{i}$ enak
\begin{equation}
\Delta E_{i}=\frac{\Delta\beta_{i}}{2\,\beta}\, E_{0}.
\label{9.27}
\end{equation}
Privzeli smo, da je delež odbitega valovanja tako majhen, da ni treba upoštevati 
spremembe amplitude vpadnega vala $E_{0}$. Celotno odbito valovanje je vsota 
prispevkov na posameznih stopnicah $\Delta\beta_{i}$,
pri čemer moramo upoštevati še različne faze delno odbitih valovanj
\begin{equation}
E_{r}=\sum\frac{\Delta\beta_{i}}{2\,\beta}\, e^{2i\beta z_{i}}\, 
E_{0}=\frac{1}{2\,\beta}\sum\frac{d\beta}{dz}\, e^{2i\beta z_{i}}\Delta z\, E_{0}.
\label{9.28}
\end{equation}
Preidemo z vsote na integral in zapišemo amplitudo odbitega valovanja
\begin{equation}
E_{r}=\frac{E_{0}}{2\,\beta}\,\int\frac{d\beta}{dz}\, e^{2i\beta z}dz.
\label{9.29}
\end{equation}
Za primer poglejmo linearno spremembo lomnega količnika in linearno spremembo 
valovnega števila za $\Delta\beta_{0}$ na razdalji $L$. Krajši račun pokaže, da je 
delež intenzitete odbitega valovanja 
\begin{equation}
\frac{I_{r}}{I_{0}}=\left( \frac{\Delta\beta_{0}}{2 \beta}\,\frac{\sin\beta L}{\beta L}\right)^2.
\label{9.30}
\end{equation}
Odbojnost je največja, kadar je $L \ll 1/\beta$,
torej kadar je sprememba $\beta$ ostra stopnica. Čim počasnejša je
sprememba, tem manj je odboja. Kadar je $\sin\beta L=0$, pride do destruktivne 
interference vseh delnih odbojev in odbojnost je enaka nič.

\begin{definition}
Naj se valovno število ob prehodu spreminja kot funkcija erf
\begin{equation}
\beta (z)= \frac{2\Delta \beta_0}{\sqrt{\pi}} \int_0^z e^{-t^2}dt.
\end{equation}
Pokaži, da je amplituda odbitega valovanja v tem primeru enaka
\begin{equation}
\frac{E_r}{E_0} = \frac{\Delta \beta_0}{\beta}e^{-\beta^2}.
\end{equation}
\end{definition}
