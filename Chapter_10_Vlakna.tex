\chapterimage{Mavrica.jpg} % Chapter heading image

\chapter{Optična vlakna}

Optična vlakna vodijo svetlobno valovanje s tem, da se na meji med
sredico z večjim lomnim količnikom in plaščem valovanje totalno odbije.
So torej dielektrični valovni vodniki za svetlobo. Njihove prednosti
pred bakrenimi vodniki v komunikacijski tehnologiji so bistveno večja
količina informacij, ki jo je mogoče prenašati po enem vlaknu, majhne
izgube in neobčutljivost na elektromagnetne motnje. Zato danes v veliki
meri nadomeščajo kovinske vodnike, posebej v medkrajevni telefoniji
in računalniških povezavah.

% \begin{figure}
% \centering
% %\def\svgwidth{\columnwidth}
% \small
% \input{slike/planarni_vodnik.pdf_tex}
% 
% \caption{Shema planarnega valovoda.}
% 
% 
% \end{figure}


Njaenostavnejši model optičnega vodnika je planparalelna plast prozornega
dielektrika z lomnim količnikom $n_{1}$ večjim od od okolice (slika
\ref{sl9.1}). Plasti z večjim lomnim količnikom recimo sredica, okolici
pa plašč vlakna. Žarek je ujet v sredici, če je vpadni kot $\theta$
večji od kota totalnega odboja, za katerega velja $\sin\theta_{c}=n_{0}/n_{1}$.
Količini $\sin\left(\pi/2-\theta_{c}\right)$, ki določa največji
kot divergence svetlobnega snopa, ki je še ujet v vlaknu, pravimo
\textit{numerična odprtina }vlakna. Razlika lomnih količnikov $n_{1}-n_{0}=\Delta n$
je običajno dokaj majhna, to je le nekaj stotink.

Za podroben opis širjenja svetlobe po vlaknih, ki imajo običajno polmer
sredice od nekaj do nekaj deset mikrometrov, geometrijska optika ne
zadošča. Rešiti moramo Maxwellove enačbe z ustreznimi robnimi pogoji,
kar je za praktična vlakna dokaj dolg račun. Ugotovimo najprej, kakšne
so osnovne značilnosti valovanja po vlaknu.

Geometrijskemu žarku, ki potuje po sredici pod kotom in se na meji
odbija, ustreza v valovni sliki val, ki ima prečno komponento valovnega
vektorja različno $k_{\perp}$ od nič. Kot totalnega odboja pri dani
frekvenci določa največjo možno vrednost $k_{\perp\max}\simeq n_{1}k_{0}\cos\theta_{c}=k_{0}\sqrt{n_{1}^{2}-n_{0}^{2}}\simeq k_{0}\sqrt{2n\Delta n}$,
kjer je $k_{0}=\omega/c$, $n$ pa povprečni lomni količnik. Ker je
valovanje v prečni smeri omejeno na sredico dimenzije $a$, ima lahko
$k_{\perp}$ le diskretne vrendosti, ki so približno $N\pi/a$, kjer
je $N$ celo število. Z vsakim $N$ je določen en \textit{rod }valovanj
v vlaknu. Ker je omejen $k_{\perp}$, je število rodov je navzgor
omejeno: $N_{\max}\simeq k_{0}a/\pi\sqrt{2n\Delta n}.$ $N$ je tudi
število vozlov, ki jih ima valovanje v prečni smeri. Število rodov
v danem vlaknu je odvisno od razlike lomnih količnikov in od dimenzije
vlakna. Videli bomo, da v optičnih vlaknih en rod vselej obstaja,
kar je ena od razlik med dielektričnimi in kovinskimi vodniki, kakršne
poznamo iz mikrovalovne tehnike in po katerih se pod določeno frekvenco
ne more širiti nobeno valovanje. Optični vodniki, po katerih se širi
en sam rod, imajo posebej lepe lastnosti za uporabo v komunikacijskih
sistemih.

Fazna hitrost valovanja, ki potuje po valovnem vodniku, je odvisna
od frekvence, imamo torej disperzijo. Razlog je v diskretnih vrednostih
$k_{\perp}$. Naj bo $\beta$ komponenta valovnega vektorja vzdolž
vlakna, recimo ji tudi valovno število, tako da je odvisnost polja
od koordinate vzdolž vlakna $\exp i\beta z$. Veljati mora 
\begin{equation}
n_{1}\frac{\omega}{c}=\sqrt{\beta^{2}+k_{\perp}^{2}}\label{9.0}
\end{equation}
 $.$ Za dano vrednost $k_{\perp}$ torej zveza med valovnim številom
in frekvenco ni linearna, zato je fazna hitrost $v_{f}=\omega/\beta$
odvisna od frekvence. Grupna hitrost $v_{g}=d\omega/d\beta$ je različna
od fazne in tudi odvisna od frekvence, kar ima za uporabo vlaken pomembne
posledice.


\section{Planparalelni vodnik}

Preprost dvodimenzionalen model optičnega vlakna je planparalelna
plast prozornega dielektrika z lomnim količnikom večjim od okolice:
$n_{1}>n_{0}$ (Slika \ref{sl9.1}). Rešitev krajevnega dela valovne
enačbe 
\begin{equation}
\nabla^{2}{\bf E+}n\left(x\right)^{2}k_{0}^{2}{\bf E=}0\label{9.1}
\end{equation}
 kjer je $k_{0}=\omega/c$, iščimo v obliki 
\begin{equation}
{\bf E}={\bf \psi}\left(x\right)\, e^{i\beta z}\label{9.2}
\end{equation}
 Za amplitudo ${\bf \psi}\left(x\right)$ dobimo enačbo 
\begin{equation}
\frac{d^{2}{\bf \psi}}{dx^{2}}+\left[n\left(x\right)^{2}k_{0}^{2}-\beta^{2}\right]\,{\bf \psi}=0\label{9.3}
\end{equation}
 kjer je v plasti (območje II.) lomni količnik $n_{1}$, v okolici
(območji I. in III.) pa $n_{0}$. Robni pogoji so, da sta tangencialni
komponenti električne in magnetne poljske jakosti na meji zvezni.
V planparalelnem primeru lahko valovanja ločimo na transverzalno električna
(TE) in transverzalno magnetna (TM). V prvem primeru je ${\bf E}$
v smeri osi $y,$ v drugem pa ${\bf H}$. Pri TE valovih je tangencialna
komponenta magnetnega polja $H_{z}$. Ker je $\nabla\times{\bf E}=i\omega\mu_{0}{\bf H}$,
dobimo iz zveznosti $H_{z}$, da je na meji zvezen tudi odvod $\partial E_{y}/\partial x$.
S tem postane problem matematično enak iskanju lastnih stanj energije
za delec v končno globoki enodimenzionalni potencialni jami v kvantni
mehaniki, kjer lastni vrednosti energije ustreza $\beta^{2}$. Poglejmo
najprej, kakšne so možne rešitve:
\begin{description}
\item [{a.}] $k_{0}n_{0}<\beta<k_{0}n_{1}$. Tedaj so rešitve oscilatorne
v sredici in eksponentno pojemajoče v plašču. Predstavljajo vodene
valove v plasti. V kvantni mehaniki ustrezajo vezanim stanjem.
\item [{b.}] $\beta<k_{0}n_{0}$. Rešitve so oscilatorne v vseh treh območjih
in opisujejo valove, ki niso ujeti v plasti, temveč z ene strani vpadajo
na plast, se delno skoznjo lomijo, delno pa odbijejo. V rešitvi je
upoštevana tudi interferenca na plasti (Naloga). Taka stanja v kvantni
mehaniki opisujejo prost delec. 
\end{description}
Zanima nas predvsem prvi primer. Začnimo s TE polarizacijo, to je
${\bf \psi(}x{\bf )=(}0,\psi(x),0)$. Naj bo $x=0$ v sredini plasti.
Zaradi simetrije morajo biti rešitve za $\psi\left(x\right)$ ali
sode ali lihe. Sode re\textquotedbl{}itve imajo v sredici obliko 
\begin{equation}
\psi_{II}(x)=C_{1}\cos k_{\perp}x\label{9.4}
\end{equation}
 v območjih I in III pa 
\begin{equation}
\psi_{I,III}\left(x\right)=C_{0}\, e^{\kappa\left(a\mp x\right)}\label{9.41}
\end{equation}
 Tu je $k_{\perp}=\sqrt{n_{1}^{2}k_{0}^{2}-\beta^{2}}$ in $\kappa=\sqrt{\beta^{2}-n_{0}^{2}k_{0}^{2}}$.
Zveznost električne poljske jakosti na meji nam da 
\begin{equation}
C_{1}\cos k_{\perp}a=C_{0}\label{9.5}
\end{equation}
 Iz zveznosti odvoda dobimo 
\begin{equation}
C_{1}k_{\perp}\sin k_{\perp}a=\kappa C_{0}\label{9.6}
\end{equation}
 Zadnji dve enačbi morata biti hkrati izpolnjeni, zato velja zveza
\begin{equation}
k_{\perp}\tan k_{\perp}a=\kappa\label{9.7}
\end{equation}
 Iz izrazov za $k_{\perp}$ in $\kappa$ dobimo še drugo zvezo 
\begin{equation}
k_{\perp}^{2}+\kappa^{2}=k_{0}^{2}\left(n_{1}^{2}-n_{0}^{2}\right)\label{9.8}
\end{equation}
 Vpeljimo brezdimenzijske količine $\xi=ak_{\perp}$ , $\eta=a\kappa$
in $V^{2}=a^{2}k_{0}^{2}\left(n_{1}^{2}-n_{0}^{2}\right)$. Z njimi
lahko gornji enačbi zapišemo v prglednejši obliki 
\begin{eqnarray}
\xi\tan\xi & = & \eta\label{9.9}\\
\xi^{2}+\eta^{2} & = & V^{2}
\end{eqnarray}
 Dovoljene vrednosti $\xi$ in $\eta$ dobimo kot presečišča krogov
z radijem $V$ in funkcije $\xi\tan\xi$, kar je prikazano na sliki
\ref{sl9.2}.

Lihe rešitve imajo v sredici obliko 
\begin{equation}
\psi_{II}(x)=C_{1}\sin k_{\perp}x\label{9.11}
\end{equation}
 v plašču pa 
\begin{equation}
\psi_{I,III}\left(x\right)=\pm C_{0}\, e^{\kappa\left(a\mp x\right)}\label{9.12}
\end{equation}
 Na enak način kot za sode rešitve dobimo iz robnih pogojev, da mora
veljati zveza 
\begin{equation}
k_{\perp}\cot k_{\perp}a=-\kappa\label{9.13}
\end{equation}
 ali 
\begin{equation}
-\xi\cot\xi=\eta\label{9.14}
\end{equation}
 Funkcija $-\xi\cot\xi$ je na sliki \ref{sl9.2} prikazana črtkano.
Lastne vrednosti za lihe rešitve dobimo kot njena presečišča s krogomz
radijem $V$.

Parameter $V$ meri globino potencialne jame; iz slike vidimo, da
eno vodeno valovanje (ali vezano stanje v kvantni mehaniki) vselej
obstaja. Pri velikih vrednostih $V$ je rešitev za dovoljene vrednosti
$\xi$ več. Razen največje so zelo blizu mnogokratniku $\pi/2$, kar
da za dovoljene vrednosti $k_{\perp}$ približno mnogokratnike $\pi/(2a)$,
kot smo pričakovali že v uvodu. Ko smo določili lastne vrednosti $k_{\perp}$,
poznamo za vsak rod tudi disperzijsko zvezo med frekvenco in valovnim
številom: 
\begin{equation}
n_{1}^{2}\frac{\omega^{2}}{c^{2}}=\beta^{2}+k_{\perp}^{2}\label{9.10}
\end{equation}
 Pri tem je tudi lastna vrednost $k_{\perp}$ lahko nekoliko odvisna
od $\omega$, ker je ta vsebovana v $V$.

Osnovna lastna vrednost $\xi_{0}$ je pri $V<<1$ približno kar $V$.
Zato je $k_{\perp}^{2}\simeq k_{0}^{2}(n_{1}^{2}-n_{0}^{2})$ in je
$\beta=n_{0}k_{0}$. V primeru zelo šibkega vodenja je torej fazna
hitrost vodenega vala približno enaka fazni hitrosti v plašču, to
je $c/n_{0}$. To je lahko razumeti; pri majhnem $V$ je namreč tudi
$\kappa$ majhen in se eksponentno pojemajoče valovanje razteza daleč
v plašč. (Naloga:Kakšna je v tej limiti grupna hitrost?). V nasprotnem
primeru, ko je $V>>1$, je $\xi_{0}\simeq\pi/2<<V$. Zato je tudi
$k_{\perp}\simeq\pi/(2a)$ $<<n_{1}k_{0}$ in je približno $\beta\simeq n_{1}k_{0}$.
V tem primeru je torej fazna hitrost taka kot v sredstvu z lomnim
količnikom sredice, kar je v skladu s preprosto predstavo, da se pri
velikem $V$ osnovni vodeni rod širi po sredici vzdolž osi vlakna.

Račun za transverzalno magnetne (TM) valove je zelo podoben, le v
robnem pogoju za tangencialno komponento električnega polja nastopita
dielektrični konstanti sredice in plašča, zato je predvsem najnižja
lastna vrednost za $k_{\perp}$ nekoliko večja kot za TE valove. (Naloga)


\section{Cilindrično vlakno}

Običajna vlakna so seveda tridimenzionalna s cilindrično geometrijo.
Najpreprostejša strukutra, povsem analogna gornjemu primeru planparalene
plasti, je jedro s konstatnim lomnim količnikom, ki je nekoliko večji
od lomnega količnika plašča. Pogoste so tudi zapletenejše konstrukcije,
pri katerih je sredica sestavljena iz več kolobarjev z različnimi
lomnimi količniki. S primerno izbiro je tako mogoče zelo zmanjšati
disperzijo (Slika \ref{sl9.3}).

Račun za širjenje svetlobe po cilindričnem vlaknu s homogeno sredico
je sicer podoben kot za planparalelni primer, vendar je precej bolj
zapleten. Glavna komplikacija je, da v cilindrični geomteriji ni več
delitve na čisto električno in magnento transverzalne valove, zato
postanejo robni pogoji bolj zapleteni. Rešitve se izrazajo v obliki
kombinacij Besselovih funkcij. Podrobnosti si bralec lahko ogleda
v literaturi \cite{vlakna}. Osnovne značilnosti rešite pa ostajajo
enako kot v dvodimenzionalnem primeru. Kot smo ugotovili že na začetku,
obstaja končno število vodenih valov, odvisno od premera sredice in
razlike lomnih količnikov sredice in plašča. Če sta ti količini majhni,
obstaja le eno vodeno valovanje in imamo enorodovno vlakno. Za njegovo
valovno število velja $n_{0}k_{0}<\beta<n_{1}k_{0}$.


\section{Cilindrično vlakno s paraboličnim profilom lomnega količnika}

V treh dimenzijah lahko hitro poiščemo rešitve za vlakno, v katerem
je dielektrična konstanta kvadratna funkcija radialne koordinate $r$:
\begin{equation}
n\left(r\right)^{2}=n_{0}^{2}+n_{2}^{2}\, r^{2}\label{9.15}
\end{equation}
 Parameter $n_{2}$ je v praksi vselej majhen, zato ima za vse smiselne
vrednosti $r$ tudi lomni količnik paraboličen profil. Parabolična
sredica mora sveda biti omejena, okoli nje imamo zopet plašč s konstantnim
lomnim količnikom (Slika \ref{sl9.4}). Tipičen radij sredice je nekaj
deset mikrometrov.

Komponento polja za izbrano polarizacijo napišimo v obliki 
\begin{equation}
E=E_{0}\psi(x,y)\, e^{i\beta z}\label{9.16}
\end{equation}
 Zanemarili smo, da zaradi odvisnosti od prečnih koordinat in $\nabla\cdot{\bf E}=0$
polje ne more imeti povsod iste smeri; če hočemo biti natančni, moramo
v gornji obliki zapisati vektorski potencial. Postavimo približni
nastavek v valovno enačbo. Dobimo enačbo 
\begin{equation}
\nabla_{\perp}^{2}\psi+\left[k_{0}^{2}\left(n_{0}^{2}+n_{2}^{2}r^{2}\right)-\beta^{2}\right]\,\psi=0\label{9.17}
\end{equation}
 ki je popolnoma enaka enačbi za krajevni del lastnih funkcij dvodimenzionalnega
harmonskega oscilatorja v kvantni mehaniki. Rešitve imajo obliko 
\begin{equation}
\psi_{lm}=H_{l}\left(\sqrt{2}\frac{x}{w}\right)\, H_{m}\left(\sqrt{2}\frac{y}{w}\right)\, e^{-(x^{2}+y^{2})/w^{2}},\qquad w^{2}=\frac{\lambda}{\pi n_{2}}\label{9.18}
\end{equation}
 z lastnimi vrednostmi 
\begin{equation}
\beta_{lm}^{2}=n_{0}^{2}k_{0}^{2}\left[1-\frac{2n_{2}}{k_{0}n_{0}}\left(l+m+1\right)\right]\label{9.19}
\end{equation}
 Razmerje $2n_{2}/\left(k_{0}n_{0}\right)$ je majhno, zato je približno
$\beta_{lm}=n_{0}k_{0}\left[1-\frac{n_{2}}{k_{0}n_{0}}\left(l+m+1\right)\right].$Če
je $n_{2}$ neodvisen od frekvence, je grupna hitrost 
\begin{equation}
v_{g}=\left(\frac{d\beta_{lm}}{d\omega}\right)^{-1}=\frac{c_{0}}{n_{0}}\label{9.21}
\end{equation}
 enaka za vse rodove, kar je značilnost vlakna s kvadratnim profilom
lomnega količnika. V dejanskem vlaknu je seveda taka odvisnost možna
le v omejenem območju sredice, zato je tudi gornja analiza le približna
in velja dobro za tiste rodove, ki se ne raztezajo izven sredice.

Neodvisnost grupne hitrosti od reda valovanja v vlaknu je praktično
pomembna. Grupna hitrost določa čas potovanja svetlobnega sunka, ki
lahko predstavlja en bit informacije. Če se po vlaknu lahko širi več
rodov, ki imajo različno grupno hitrost, se sunek po prehodu skozi
vlakno razširi, kar omejuje uporabno dolžino vlakna, kot bomo podrobneje
videli kasneje. Temu se lahko izognemo z uporabo enorodovnih vlaken,
ki pa so dražja in je vanje te\textquotedbl{}je uvesti svetlobni snop,
katerega divergenca in premer morata ustrezati ujetemu valu enorodovnega
vlakna, če naj ne bo izgub. Zato se za krajše zveze uporabljajo mnogorodovna
vlakna, ki imajo sredico s približno paraboličnim profilom lomnega
količnika.


\section{Sprememba lomnega količnika vlakna}

Sprememba lomnega količnika sredice ali plašča vlakna povzroči spremembo
valovnega števila $\beta$ danega roda. V enorodovnih vlaknih je to
mogoče izkoristiti za izdelavo senzorjev, na primer temperature ali
tlaka. Zaradi zunanjih vplivov, ki jih želimo zaznati, se spremeni
lomni količnik vlakna in s tem propagacijska konstanta, kar lahko
izmerimo preko spremembe faze valovanja na izhodu iz vlakna, to je,
z ustrezno sestavljenim interferometrom. Ker je dolžina vlakna lahko
velika, v nekaj centimetrov velik tulec lahko brez težav navijemo
kilometre vlakna, je celotna sprememba faze velika že pri majhnih
spremembah merjene količine. Sprememba valovnega števila povzroča
tudi nezaželjene spremembe faze in odboje pri prenosu informacij.
V tem razdelku zato poglejmo, kako se spremeni valovno število pri
dani spremembi lomenga količnika.

Vzemimo rod vlakna s propagacijsko konstanto $\beta_{lm}$ in prečnim
profilom $\psi_{lm}\left(r,\phi\right).$ Ta mora zadoščati valovni
enačbi 
\begin{equation}
\nabla_{\bot}^{2}\psi_{lm}+\left(\epsilon(r)k_{0}^{2}-\beta_{lm}^{2}\right)\psi_{lm}=0\label{9.22}
\end{equation}
 Naj se dielektrična konstanta vlakna spremeni za $\delta\epsilon.$
Zato se spremenita tudi propagacijska konstanta $\beta=\beta_{lm}+\delta\beta$
in prečna oblika $\psi=\psi_{lm}+\delta\psi.$ Tudi motena funckija
$\psi$ mora zadoščati enačbi \ref{9.22}, zato za perturbacijo velja
\begin{equation}
\nabla_{\bot}^{2}\delta\psi+\left(\epsilon(r)k_{0}^{2}-\beta_{lm}^{2}\right)\delta\psi+\delta\epsilon\, k_{0}^{2}\psi_{lm}=2\beta_{lm}\delta\beta\,\psi_{lm}\label{9.23}
\end{equation}
 kjer smo zanemarili produkte majhnih količin. Množimo obe strani
enačbe s $\psi_{lm}^{*}$ in integrirajmo po preseku vlakna, pri čemer
upoštevajmo, da je $\delta\psi$ ortogonalna na $\psi_{lm}$: 
\begin{eqnarray}
 &  & \int\psi_{lm}^{*}\nabla_{\bot}^{2}\delta\psi\, dS+\int\left(\epsilon(r)k_{0}^{2}-\beta_{lm}^{2}\right)\delta\psi\,\psi_{lm}^{*}+k_{0}^{2}\int\delta\epsilon\,\left|\psi_{lm}\right|^{2}dS\label{9.24}\\
 & = & 2\beta_{lm}\,\delta\beta\int\left|\psi_{lm}\right|^{2}dS
\end{eqnarray}
 Prvi člen na levi preoblikujmo z uporabo zvez $\int(u\,\nabla_{\bot}^{2}v-v\nabla_{\bot}^{2}u)\, dS=\int\nabla_{\bot}\cdot(u\nabla_{\bot}v-v\nabla_{\bot}u)\, dS=\oint{\bf ds\times(}u\,\nabla_{\bot}v-v\,\nabla_{\bot}u)$.
Ker fuknciji $\psi_{lm}$ in $\delta\psi$ opisujeta vodene valove,
morata iti za velike $r$ proti nič, zato je integral po krivulji
v gornji zvezi nič in velja $\int\psi_{lm}^{*}\nabla_{\bot}^{2}\delta\psi\, dS=\int\delta\psi\nabla_{\bot}^{2}\psi_{lm}^{*}\, dS$.
Ker $\psi_{lm}^{*}$ zadošča enačbi \ref{9.22}, se v enačbi \ref{9.24}
prvi in drugi člen uničita in dobimo željeno zvezo 
\begin{equation}
\delta\beta=\frac{k_{0}^{2}\int\delta\epsilon\,\left|\psi_{lm}\right|^{2}dS}{2\,\beta\int\left|\psi_{lm}\right|^{2}dS}\label{9.25}
\end{equation}
 ki je seveda analogna kvantnomehanskemu rezultatu s teorijo motenj
prvega reda za spremembo energije lastnega stanja pri majhni sprmembi
Hamiltonovega operatorja. Rezultat je tudi intuitivno razumljiv: v
najnižjem redu je $\delta\beta$ sorazmerna s uteženim povprečjem
$\delta\epsilon$, kjer je utež $\psi_{lm}$.

Sprememba valovnega števila $\beta$ v delu vlakna, po katerem potuje
svetloba, ne povzroči le spremembe faze, ampak tudi odboj dela valovanja.
Ta pojav je le nekoliko druga oblika odboja na meji (zvezni ali ostri)
dveh dielektrikov, ali splošneje, odboja vsakega valovanja na območju,
kjer se spremeni fazna hitrost valovanja.

Odboj na območju vlakna, kjer se spreminja $\beta$, bomo najlažje
dobili preko formule za odboj na meji dveh dielektrikov pri pravokotnem
vpadu. Odbita amplituda je tedaj 
\begin{equation}
E_{r}=\frac{n_{2}-n_{1}}{n_{2}+n_{1}}E_{0}\label{9.26}
\end{equation}
 Mislimo si, da je sprememba $\beta$ na delu vlakna sestavljena iz
majhnih stopničastih sprememb $\Delta\beta_{i}$ na intervalih $\Delta\dot{z}$.
Za ravno valovanje, za katerega velja enačba \ref{9.26}, je sprememba
lomnega količnika sorazmerna s spremembo fazne hitrosti. Ker tudi
$\beta$ določa fazno hitrost valovanja po vlaknu, je delež odbitega
valovanja na stopničasti spremembi $\Delta\beta_{i}$
\begin{equation}
\Delta E_{i}=\frac{\Delta\beta_{i}}{2\,\beta}\, E_{0}\label{9.27}
\end{equation}
 Predpostavili smo, da je celotni del odbitega valovanja tako majhen,
da ni treba upoštevati spremembe amplitude vpadnega vala $E_{0}$.
Vse odbito valovanje je vsota prispevkov na posameznih stopnicah $\Delta\beta_{i}$,
pri čemer moramo upoštevati še različne faze delnih odbitih valovanj:
\begin{equation}
E_{r}=\sum\frac{\Delta\beta_{i}}{2\,\beta}\, e^{2i\beta z_{i}}\, E_{0}=\frac{1}{2\,\beta}\sum\frac{d\beta}{dz}\, e^{2i\beta z_{i}}\Delta z\, E_{0}\label{9.28}
\end{equation}
 Preidimo z vsote na integral, pa dobimo, da je odbita amplituda 
\begin{equation}
E_{r}=\frac{E_{0}}{2\,\beta}\,\int\frac{d\beta}{dz}\, e^{2i\beta z}dz\label{9.29}
\end{equation}


\textit{Primer:} Naj se valovno število linearno spermeni za $\Delta\beta_{0}$
na razdalji $L$. Tedaj je po gornji formuli 
\begin{equation}
\frac{E_{r}}{E_{0}}=2\,\frac{\Delta\beta_{0}}{L}\,\frac{\sin\beta L/2}{\beta}\label{9.30}
\end{equation}
 Odbojnost je največja, kadar je $L$ majhen v primeri z $1/\beta$,
torej kadar je sprememba $\beta$ ostra stopnica. Čim počasnejša je
sprememba, tem manj je odboja, poleg tega pa odboja ni vsakič, ko
je $\sin\beta L/2=0$, to je, pride do destruktivne interference vseh
delnih odbojev.(Naloga: Odboj na erf stopnici).


\section{Izgube v optičnih vlaknih}

Ena najpomembnejših lastnosti optičnih valken je majhno slabljenje
svetlobnega vala, posebej še v vlaknih iz kremenovega stekla. Najboljša
vlakna imajo danes izgube okoli 0,2 db/km pri valovni dolžini 1,55
$\mu$m. (1 decibel je$\,0,1\,$log$(j_{0}/j).$). Glavni vzroki izgub
so absorpcija in sipanje na nečistočah in sipanje na termičnih fluktuacijah
gostote (Rayleighovo sipanje). Slika \ref{sl9.5} prikazuje tipično
odvisnost izgub od valovne dolžine za dobro enorodovno vlakno iz kremenovega
stekla s primesjo GeO$_{2}.$ Sipanje na fluktuacijah gostote je sorazmerno
z $\lambda^{-4}$, zato dominira pri majhnih valovnih dolžinah, sipanje
na defektih in nepravilnostih je skoraj zanemarljivo, pri valovnih
dolžinah nad 1,6~$\mu$m pa prevlada absorpcija. Vrh med 1,3~$\mu$m
in 1,5~$\mu$m je posledica absorpcije na OH ionih, ki se jih je
v steklu zelo težko znebiti. Iz slike je razvidno, da so izgube najmanjše
okoli 1,55~$\mu$m, zato se to območje največ uporablja za zveze
na velike razdalje. Najboljših modernih vlaken tudi ni več mogoče
kaj dosti izboljšati glede izgub, saj so že dosegla spodnjo mejo,
določeno s termičnimi fluktuacijami.

Pri optičnih zvezah nastanejo še izgube na spojih vlaken, ki so okoli
0,2 db na spoj. Skupne izgube so tako dosti manjše kot v koaksialnem
kablu in je možen prenos signala do nekaj sto kilometrov brez vmesnega
ojačevanja. S tem pri dolžini optičnih zvez izgube niso več glavna
omejitev, ampak je to popačitev signala zaradi disperzije.

V optičnem vlaknu nastanejo izgube tudi, kadar je vlakno ukrivljeno.
Te izgube postanejo znantne, kadar je krivinski radij vlakna centimeter
ali manj. Ta pojav je dovolj zanimiv, da si ga je vredno nekoliko
podrobneje ogledati.

Vzemimo spet dvodimenzionalno plast debeline $2a$ z lomnim količnikom
$n_{1}$ v sredstvu z lomnim količnikom $n_{0},$ ki naj bo ukrivljena
s krivinskim radijem $R$. Taka plast tvori del kolobarja z notranjim
radijem $R-a$ in zunanjim radijem $R+a$, pri čemer je $\dot{R}>>a$
(Slika \ref{sl9.6}). Zapišimo valovno enačbo za dve dimenziji v cilindrični
geometriji: 
\begin{equation}
\frac{1}{r}\,\frac{\partial}{\partial r}\, r\,\frac{\partial E}{\partial r}+\frac{1}{r^{2}}\,\frac{\partial^{2}E}{\partial\phi^{2}}+k_{0}^{2}n^{2}\left(r\right)\, E=0\label{9.31}
\end{equation}
 kjer ima $n\left(r\right)$ vrednost $n_{1}$ v sredici in $n_{0}$
drugje. Zanimajo nas rešitve oblike 
\begin{equation}
E=\psi\left(r\right)\, e^{im\phi}\label{9.32}
\end{equation}
 kjer je $\psi\left(r\right)$ znatna le v sredici. Ker je valovna
dolžina svetlobe dosti manjša od $R$, je $m$ veliko število, ki
je povezano z valovnim številom $\beta$: naj bo $z=R\phi$ dolžina
loka vzdolž sredine sredice. Tedaj je $m\phi=(m/R)\, z$ in je torej
$\beta=m/R$. $\psi$ zadošča enačbi 
\begin{equation}
\frac{d^{2}\psi}{dr^{2}}+\frac{1}{r}\,\frac{d\psi}{dr}+\left[k_{0}^{2}\, n^{2}\left(r\right)-\frac{m^{2}}{r^{2}}\right]\psi=0\label{9.33}
\end{equation}
 Rešitve za $\psi$ so kombinacije Besselovih funkcij reda $m$, kar
pa zaradi velikosti $m$ ni posebno zanimivo. Dosti več bomo izvedeli,
če primerno preoblikujemo valovno enačbo \ref{9.31}. Namesto $r$
in $\phi$ vpeljimo koordinati $x=r-R$ in $z=R\phi$. S tem smo prešli
nazaj na koordinate planparelelne plasti in iščemo popravke valovne
enacbe \ref{9.3} reda $1/R.$ Tako je $1/r\simeq1/R$ in 
\begin{equation}
\frac{m^{2}}{r^{2}}=\frac{m^{2}}{\left(R+x\right)^{2}}\simeq\frac{m^{2}}{R^{2}}\,\left(1-2\,\frac{x}{R}\right)=\beta^{2}\left(1-2\,\frac{x}{R}\right)\label{9.34}
\end{equation}
 S tem dobimo iz enačbe \ref{9.33} približno enačbo za prečno obliko
polja 
\begin{equation}
\frac{d^{2}\psi}{dx^{2}}+\left[k_{0}^{2}\, n^{2}\left(r\right)-\beta^{2}\right]\,\psi+\frac{1}{R}\,\left(\frac{d\psi}{dr}-2\,\beta^{2}x\,\psi\right)=0\label{9.35}
\end{equation}



\section{Disperzija}
\label{chap:Disperzija}
Zaradi disperzije, to je odvisnosti fazne in grupne hitrosti od frekvence,
se sunek svetlobe, ki potuje po vlaknu, podaljšuje. Ta pojav omejuje
količino informacije, ki jo je mogoče prenašati po vlaknu dane dolžine.
Zato je čim manjša disperzija v vlaknih vsaj toliko pomembna kot majhne
izgube.

Vzemimo najprej enorodovno vlakno in naj bo svetloba v vlaknu modulirana
v obliki kratkih sunkov, ki nosijo informacijo. Sunki potujejo z grupno
hitrostjo 
\begin{equation}
v_{g}=\frac{d\omega}{d\beta}=\left(\frac{d\beta}{d\omega}\right)^{-1}\label{9.51}
\end{equation}
 Po enačbi \ref{9.0} je grupna hitrost odvisna od frekvence tako
zaradi eksplicitne korenske zveze med $\omega$ in $\beta$ kot zaradi
odvisnosti lomnih količnikov sredice in plašča od frekvence. Prvemu
prispevku recimo valovodna disperzija, ker je posledica omejitve valovanja
v sredico vlakna, drugemu pa materialna disperzija. Oba prispevka
sta pomembna.

Naj bo dolžina vlakna $L$ in trajanje posameznega sunka $\tau$.
Sunek potuje po vlaknu čas 
\begin{equation}
T=\frac{L}{v_{g}}=L\frac{d\beta}{d\omega}\label{9.52}
\end{equation}
 Svetloba ima končno spektralno širino $\Delta\omega$. Ta mora biti
vsaj 1/$\tau,$ lahko pa je tudi večja, če svetlobni izvor, običajno
polvodniški laser, ni povsem monokromatski. Ker vse spektralne komponente
ne potujejo z isto hitrostjo, se sunek na koncu vlakna razleze za
\begin{equation}
\Delta\tau=\left|\frac{dT}{d\omega}\right|\Delta\omega=L\,\frac{d^{2}\beta}{d\omega^{2}}\,\Delta\omega\label{9.53}
\end{equation}
 Naj bo tudi razmik me zaporednimi sunki $\tau.$ Da se sunki ne bodo
prekrivali, mora biti $\Delta\tau<\tau$ in sme biti najvišja frekvenca
modulacije kvečjemu 
\begin{equation}
v_{\max}=\frac{1}{2\Delta\tau}\label{9.54}
\end{equation}


Ločiti moramo dva mejna primera. Kadar je $\Delta\omega>>1/\tau$,
to je, kadar je spektralna širina izvora mnogo večja od širine zaradi
modulacije, je 
\begin{equation}
v_{\max}=\frac{1}{L\,\Delta\omega}\left(\frac{d^{2}\beta}{d\omega^{2}}\right)^{-1}\label{9.55}
\end{equation}
 Največja gostota informacije, ki jo lahko prenašamo po vlaknu, je
v tem primeru obratno sorazmerna z dolžino vlakna in spektralno širino
laserja.

V nasprotnem primeru, ko je laser dosti bolj monokromatski od razširitve
zaradi same modulacije, imamo $\Delta\omega=1/\tau$. Pri najvišji
možni frekvenci modulacije $v_{\max}$ je $\tau\simeq\Delta\tau$
in je po enačbi \ref{9.53} 
\begin{equation}
(\Delta\tau)^{2}=L\,\frac{d^{2}\beta}{d\omega^{2}}\,\label{9.56}
\end{equation}
 in je 
\begin{equation}
v_{\max}=\sqrt{\frac{1}{L\,}\left(\frac{d^{2}\beta}{d\omega^{2}}\right)^{-1}}\label{9.57}
\end{equation}
 V tem primeru je najvišja frekvenca modulacije obratno sorazmerna
s korenom dolžine vlakna.

Večji del dicperzije grupne hitrosti prinese materialna disperzija,
to je, odvisnost lomnega količnika od frekvence. Primer meritve razširitve
sunka z znano spektralno širino v izbrani dolžini vlakna kaže slika
\ref{sl9.7}. Pri valovni dolžini okoli 1,3 $\mu$m ima materialna
disperzija, to je $d^{2}n/d\omega^{2},$ ničlo, ki je v celotni disperziji
nekoliko premaknjena zaradi prispevka valovodne disperzije. Na valovodno
disperzijo je mogoče vplivati s konstrucijo vlakna. Sredica je lahko
sestavljena iz več plasti z različnimi lomnimi količniki in različnimi
debelinami, s čimer se spremeni prispevek valovodne disperzije in
se položaj ničle celotne disperzije premakne k valovni dolžini izvora.

Iz slike \ref{sl9.7} razberemo, da je tipična disperzija enorodovnega
vlakna okoli 10~ps/km pri spektralni širini 1~nm. Od tod je $d^{2}\beta/d\omega^{2}=\Delta\tau\lambda/\left(\Delta\lambda L\omega\right)\simeq5\cdot10^{-23}$
s$^{2}/$m. Po enačbi \ref{9.57} je v 100 km dolgem vlaknu tedaj
najvišja frekvenca modulacije okoli $10^{9}$ Hz. Pri zmogljivih zvezah
na velike razdalje je torej za največjo dolžino vlakna brez obnovitve
signala disperzija hujaša omejitev kot izgube. Največja možna razdalja
in najvišje frekvenca modulacije sta danes nekaj sto kilometrov in
nekaj GHz.

V mnogorodvnih vlaknih se sunek širi zaradi razli\textquotedbl{}nih
grupnih hitrosti posameznih rodov. Različne grupne hitrosti v valovni
sliki ustrezajo različni dolžini optične poti za žarke, ki potujejo
pod različnimi koti glede na os vlakna. Te razlike so mnogo večje
od disperzije v enorodnih vlaknih, zato je pri dani dolžini vlakna
popačitev signala dosti večja in se mnogorodovna vlakna ne uporabljajo
za dolge zveze.


\section{ Potovanje sunka po enorodovnem vlaknu}

Poglejmo si nekoliko podrobneje, kako po enorodovnem vlaknu ali drugem
sredstvu z disperzijo potuje sunek valovanja z dano začetno obliko.
Denimo, da smo poiskali lastna valovanja in da torej poznamo zvezo
med valovnim številom $\beta$ in frekvenco $\omega$. Zapišimo sunek
v obliki 
\begin{equation}
E\left(z,t\right)=a\left(z,t\right)\,\psi\left(x,y\right)\label{9.61}
\end{equation}
 kjer je $\psi\left(x,y\right)$ lastna rešitev prečenega dela valovne
enačbe, ki določa tudi $\beta\left(\omega\right)$. Funkcijo $a\left(z,t\right)$,
ki opisuje širjenje sunka in njegovo obliko v $z$ smeri, lahko zapišemo
s Fourierovim integralom po frekvencah 
\begin{equation}
a\left(z,t\right)=\int a(\omega,z)\, e^{-i\omega t}d\omega\label{9.62}
\end{equation}
 Sunek naj bo približno monokromatičen s frekvenco $\omega_{0}$,
to pomeni, da je mnogo daljši od optične periode. Pri določeni $\omega$
ima Fourierova amplituda krajevno odvisnost $\exp[i\,\beta\left(\omega\right)\, z]$,
zato zadošča enačbi 
\begin{equation}
\frac{\partial a\left(z,\omega\right)}{\partial z}=i\,\beta\left(\omega\right)\, a\left(z,\omega\right)\label{9.63}
\end{equation}
 Privzeli smo, da je spekter sunka ozek, zato lahko $\beta\left(\omega\right)$
razvijemo okoli $\omega_{0}$: 
\begin{equation}
\frac{\partial a\left(z,\omega\right)}{\partial z}=i\,\left[\beta\left(\omega_{0}\right)+\frac{d\beta}{d\omega}\,\left(\omega-\omega_{0}\right)+\frac{1}{2}\,\frac{d^{2}\beta}{d\omega^{2}}\,\left(\omega-\omega_{0}\right)^{2}\right]\, a\label{9.64}
\end{equation}
 Vpeljimo novo amplitudno funkcijo, ki ne bo vsebovala osnovne odvisnosti
$\exp[i\beta\left(\omega_{0}\right)z]$ 
\begin{equation}
a\left(z,\omega\right)=A\left(z,\omega-\omega_{0}\right)\, e^{i\,\beta\left(\omega_{0}\right)\, z}\label{9.65}
\end{equation}
 Ker je spkter različen od nič le okoli $\omega_{0}$, je prikladno
$A$ pisati kot funkcijo $\omega-\omega_{0}$. Napravimo obratno Fourierovo
transformacijo zadnjega izraza: 
\begin{eqnarray}
a\left(z,t\right) & = & \int_{-\infty}^{\infty}A\left(z,\omega-\omega_{0}\right)\, e^{i\,[\beta\left(\omega_{0}\right)\, z-\omega\,\, t]}\, d\omega\label{9.66}\\
 & = & \int_{-\infty}^{\infty}A\left(z,\omega-\omega_{0}\right)\, e^{-i\left(\omega-\omega_{0}\right)\, t}d\left(\omega-\omega_{0}\right)\,\, e^{i\,[\beta\left(\omega_{0}\right)\, z-\omega\,_{0}\, t]}\nonumber \\
 & = & A\left(z,t\right)\,\, e^{i\,[\beta\left(\omega_{0}\right)\, z-\omega\,_{0}\, t]}\nonumber 
\end{eqnarray}
 Funkcija $A\left(z,t\right)$, katere Fourierova transformacija je
$A\left(z,\omega\right)$, očitno predstavlja prostorsko in časovno
odvisnost ovojnice sunka. Postavimo definicijo \ref{9.65} v enačbo
\ref{9.64}: 
\begin{equation}
\frac{\partial A\left(z,\omega-\omega_{0}\right)}{\partial z}=i\,\left[\frac{d\beta}{d\omega}\,\left(\omega-\omega_{0}\right)+\frac{1}{2}\,\frac{d^{2}\beta}{d\omega^{2}}\,\left(\omega-\omega_{0}\right)^{2}\right]\, A\left(z,\omega-\omega_{0}\right)\label{9.67}
\end{equation}
 Z obratno Fourierovo transformacijo dobimo 
\begin{equation}
\left(\frac{\partial}{\partial z}+\frac{1}{v_{g}}\,\frac{\partial}{\partial t}\right)\, A\left(z,t\right)=-\frac{i}{2}\,\frac{d^{2}\beta}{d\omega^{2}}\,\frac{\partial^{2}A\left(z,t\right)}{\partial t^{2}}\label{9.68}
\end{equation}
 Upoštevali smo, da je 
\begin{equation}
\int\left(i\omega\right)^{n}A\left(z,\omega\right)\, e^{-i\omega t}\, d\omega=\frac{\partial^{n}}{\partial t^{n}}A\left(z,t\right)\label{9.69}
\end{equation}
 in da je $d\beta/d\omega=1/v_{g}$.

Enačba \ref{9.68} opisuje razvoj oblike sunka pri širjenju po vlaknu.
Če ni disperzije grupne hitrosti, to je, če je desna stran enačbe
nič, je rešitev poljubna funkcija $f\left(z-v_{g}t\right)$. Sunek
poljubne začetne oblike potuje po vlaknu nepopačen z grupno hitrostjo.
Disperzija pa povzroči, da se spreminja tudi oblika. Enačbo lahko
še nekoliko poenostavimo z vpeljavo novih neodvisnih spremenljivk
\begin{eqnarray}
\tau & = & t-\frac{z}{v_{g}}\nonumber \\
\zeta & = & z\label{9.70}
\end{eqnarray}
 Za vrh sunka, ki naj ima pri $t=0$ koordinato $z=0$ in se giblje
z grupno hitrostjo, je vselej $\tau=0$. Spremenljivka predstavlja
$\tau$ torej čas v točki $z=\zeta$ , merjen od trenutka, ko tja
prispe center sunka. Z novima spremenljivkama se enačba \ref{9.68}
zapiše 
\begin{equation}
\frac{d^{2}\beta}{d\omega^{2}}\,\frac{\partial^{2}A}{\partial\tau^{2}}+2\, i\,\frac{\partial A}{\partial\zeta}=0\label{9.71}
\end{equation}
 Ta enačba ima isto obliko kot obosna valovna enačba, ki smo jo v
drugem poglavju uporabili za obravnavo koherentnih snopov. Podobnost
seže dlje od formalne oblike. Pri snopih, ki so omejeni v prečni smeri,
disperzija fazne in grupne hitrosti po prečnih komponentah valovnega
vektorja povzroča spreminjanje prečnega preseka snopa, pri časovno
omejenih sunkih v sredstvu s frekvenčno disperzijo pa se spreminja
vzdolžna oblika sunka. Kot se morda bralec spominja, je tudi v praznem
prostoru pri širjenju snopa v okolici grla fazna hitrost funkcija
frekvence. Zato se kratek sunek, ki je omejen v prečni smeri, tudi
v praznem prostoru razširi tako v prečni kot v vzdolžni smeri. (Naloga)

Obosno valovno enačbo rešijo Gaussovi snopi. V en. \ref{9.71} ima
vlogo prečne koordinate $\tau.$ Po analogiji s snopi se bo zaradi
disperzije najmanj širil sunek z Gaussovo časovno odvisnostjo. Računa
nam ni treba ponavljati, kar v izrazu za Gaussove snope napravimo
ustrezno zamenjavo črk. Valovnemu številu $k$ pri snopih na primer
ustreza parameter $\mu=(d^{2}\beta/d\omega^{2})^{-1}$. Tako dobimo
\begin{equation}
A\left(\tau,\zeta\right)=\frac{A_{0}}{\sigma_{0}\sqrt{1+\frac{\zeta^{2}}{\zeta_{0}^{2}}}}\exp\left(-\frac{\tau^{2}}{\sigma^{2}}\right)\exp\left(-i\frac{\mu\tau^{2}}{2b}\right)e^{i\phi\left(\zeta\right)}\label{9.72}
\end{equation}
 kjer je $\sigma$ trajanje sunka, za katerega velja enaka zveza kot
za polmer Gaussovega snopa: 
\begin{equation}
\sigma^{2}=\sigma_{0}^{2}\left(1+\frac{\zeta^{2}}{\zeta_{0}^{2}}\right)\label{9.73}
\end{equation}
 Tu je $\sigma_{0}$ trajanje sunka pri $\zeta=0$, to je na mestu,
kjer je sunek najkrajši. Dodatna skupna faza $\phi\left(\zeta\right)$
ni posebno pomembna, pač pa je zanimiv drugi eksponentni faktor v
enačbi \ref{9.72}. V njem smo z $b=\zeta\left(1+\zeta_{0}^{2}/\zeta^{2}\right)$
označili količino, ki je analogna krivinskemu radiju valovnih front
v primeru Gaussovih snopov. Odvod faze po $\tau$ predstavlja spremembo
frekvence glede na centralno frekvenco sunka $\omega_{0}$: 
\begin{equation}
\omega-\omega_{0}=\frac{\mu\tau}{b}\label{9.74}
\end{equation}
 Za pozitivno disperzijo $\mu$ je frekvenca na prednji strani sunka,
to je pri $\tau<0$, večja in se linearno zmanjšuje proti koncu sunka.
Pri $\zeta=0$ je sunek toliko kratek, kolikor je možno pri dani spektralni
širini. Pri potovanju po vlaknu se zaradi disperzije sunek razširi,
spektralna širina pa ostaja enaka, zato se je del pojavi kot spreminjanje
frekvence znotraj sunka. Lahko si mislimo tudi, da je sunek najkrajši,
to je omejen z Fourierovo transformacijo spektra, tedaj, kadar se
vse frekvenčne komponente seštejejo z isto fazo, to je pri $\zeta=0$.
Da dobimo najkrajše sunke, kadar je faza vseh delnih valov enaka,
smo srečali že pri fazno uklenjenih sunkih iz mnogofrekvenčnih laserjev.
Pri potovanju sunka se zaradi disperzije faze frekvenčnih komponent
različno spreminjajo in sunek se podaljša. Zanimivo je, da je pri
tem pomemben šele drugi odvod fazne hitrosti po frekvenci, ki je sorazmeren
z $\mu$, linearno spreminjanje faze pa ne povzroči razširitve.

Naloga: Pokaži, da je spekter sunka nespremenjen.

Naloga: Pokaži, da je za sunek poljubne začetne oblike razširitev
mogoče zapisati z uklonskim integralom.

Naloga: Pokaži, da iz en. \ref{9.73} sledi podobna ocena za maksimalno
frekvenco modulacije (minimalno razširitev sunka) pri dani dolžini
vlakna, kot jo da en. \ref{9.57}.

Razširitev sunka zaradi disperzije je pri $\dot{\mu}>0$ mogoče kompenzirati
s parom paralelnih uklonskih mrežic, kot kaže slika \ref{sl9.8}.
Prva mrežica različne frekvenčna komponente razkloni, druga pa zopet
zbere, vendar dolžine optičnih poti za različne komponente niso enake.
celoten učinek je enak kot pri razširjanju sunka po sredstvu z negativno
disperzijo. Račun je nekoliko preglednejši, rezultat pa povsem enak,
če namesto refleksijskih mrežic vzamemo transmisijski, kot kaže slika
\ref{sl9.9}. Naj na par vpada raven val pod kotom $\alpha.$ Pred
prvo mre\textquotedbl{}ico je fazni faktor $\exp\left(ik_{1}x\right),$
kjer je $k_{1}=\omega/c\,\sin\alpha$. Pri prehodu skozi mrežico se
polje pomnoži s kompleksno prepustnostjo mrežice, ki povzroči razcep
vala na uklonjene valove. Pri tem se faza za prvi uklonski red poveča
za $qx$, kje je $q=2\pi/\Lambda$ in je $\Lambda\,$perioda mrežice.
Premik do druge mrežice poveča fazo za $k_{3}L$. Za komponento valovnega
vektorja v smeri $z$ velja seveda $k_{3}=\sqrt{\left(\omega/c\right)^{2}-\left(k_{1}+q\right)^{2}}$.
Po prehodu skozi drugo mrežico nas zanima prvi negativni uklonski
red, ki da val v smeri prvotnega vala. Za ta red se faza spremeni
za $-qx$, tako da je celotna sprememba faze 
\begin{equation}
\Phi=L\sqrt{\left(\omega/c\right)^{2}-\left(k_{1}+q\right)^{2}}=
\frac{L}{c}\sqrt{\omega^{2}-\left(\omega\,\sin\alpha+qc\right)^{2}}\label{9.75}
\end{equation}
 Disperzija, ki jo povzroči par mrežic, je določena zdrugim odvodom
faze po frekvenci: 
\begin{equation}
\frac{d^{2}\Phi}{d\omega^{2}}=-\frac{L\, q}{\left[\omega^{2}-\left(\omega\,\sin\alpha+q\, c\right)^{2}\right]^{3/2}}\label{9.76}
\end{equation}
 Drugi odvod je vselej negativen. Par mrežic torej deluje kot sredstvo
z negativno disperzijo. Sunek, ki se je razširil zaradi potovanja
po sredstvu s pozitivno disperzijo, lahko ponovno skrajšamo do meje,
določene s širino spektra. Postopek se uporablja za pridobivanje zelo
kratkih sunkov. Sunku iz fazno uklenjenega barvilnega ali Ti:safirnega
laserja najprej v nelinearnem sredstvu razširijo spekter, pri čemer
se sunek tudi časovno podaljša. O tem najde bralec nekaj več v poglavju
o nelinearni optiki. Razširjen sunek nato s parom mrežic skrajšajo
za faktor 10-100 glede na prvotno dolžino sunka. Tako dobijo sunke
dolge le okoli 10 fs, kar je le še nekaj optičnih period.

\section{Sklopitev v optična vlakna}
