%\chapterimage{slike/Fibri.jpg} % Chapter heading image

\chapter{Optična vlakna}
\label{chap:fibri}
Moderna komunikacijska tehnologija zahteva vedno hitrejši prenos
vedno večje količine podatkov. Navadne kovinske vodnike
so zato v računalniških in telekomunikacijskih povezavah nadomestila optična 
vlakna, ki jih odlikujejo majhne izgube, neobčutljivost na elektromagnetne
in medsebojne motnje ter zmožnost prenosa izjemno velike količine podatkov
na dolge razdalje. V tem poglavju bomo opisali mehanizme prenosa podatkov 
po optičnih vodnikih in spoznali omejitve pri prenosu,
predvsem disperzijo in izgube, ter načine, kako se z njimi spopadamo.

\section{Planparalelni vodnik}
\subsection*{Geometrijski opis}
\index{Optični vodnik}
\index{Optično vlakno}
\index{Optični vodnik!planparalelni}
Klasično pojasnimo delovanje optičnih vlaken s totalnim odbojem~\index{Totalni odboj}
na meji med dvema dielektrikoma. Kadar prehaja svetloba iz snovi 
z večjim lomnim količnikom\index{Lomni količnik} v sredstvo z manjšim lomnim količnikom,
se pri vpadnih kotih, ki so večji od mejnega kota, totalno odbije. 
\begin{figure}[ht]
\centering
\def\svgwidth{110truemm} 
\input{slike/10_Vodnik.pdf_tex}
\caption{Klasična razlaga razširjanja svetlobe po valovnem vodniku. Z modro je označena sredica
vodnika z lomnim količnikom $n_1$ in z zeleno plašč z lomnim količnikom $n_2$. 
Žarek se na meji med sredico in plaščem totalno odbija.}
\label{fig:vodnik}
\end{figure}
\vglue-5truemm
Najpreprostejši optični vodnik je planparalelna plast 
dielektrika, ki je obdana s snovjo z manjšim lomnim količnikom (slika~\ref{fig:vodnik}). 
Plasti z večjim lomnim količnikom ($n_1$) rečemo sredica\index{Optični vodnik!sredica} in 
okoliški snovi z lomnim količnikom $n_2<n_1$ plašč\index{Optični vodnik!plašč}. 
Žarek potuje po vodniku, če je vpadni kot 
na mejo med sredico in plaščem $\vartheta$ večji od mejnega kota totalnega odboja $\vartheta_T$, 
za katerega velja
\boxeq{10totalni}{
\sin\vartheta_{T}=\frac{n_{2}}{n_{1}}.
}
Obstaja največji vpadni kot $\alpha_{\rm max}$, pri katerem se
vpadna svetloba ujame v vlakno.
Z njim povezana je numerična odprtina (apertura) vlakna\index{Numerična odprtina} $NA$, 
ki jo izračunamo kot 
\begin{equation}
NA = \sin \alpha_{\rm max} = n_1 \sin \beta_{\rm max} = 
n_1 \sin(\pi/2-\vartheta_T) =
n_1 \cos\vartheta_T = n_1 \sqrt{1-\sin^2\vartheta_T}.
\label{10NAa}
\end{equation}
Upoštevajoč enačbo~(\ref{10totalni}) numerično odprtino zapišemo kot 
\boxeq{10NA}{
NA = \sqrt{n_1^2-n_2^2}.
}
Razlika med lomnima količnikoma sredice in plašča je navadno majhna,
tipično le nekaj stotink, zato je numerična odprtina optičnih 
vodnikov $NA \lesssim 0,1 $. Kot, pod katerim lahko vpada svetloba
v vodnik (ali vlakno), da se vanj ujame, je tako navadno le nekaj stopinj.

\subsection*{Valovni opis}
Za natančen opis širjenja svetlobe po vodnikih ali vlaknih\footnote{Dogovorimo se, da  
besedo vlakno uporabljamo za cilindrične strukture in besedo vodnik za planparalelne 
ter njim podobne strukture.}, ki imajo debelino ali premer
sredice od nekaj mikrometrov do nekaj deset mikrometrov, geometrijska optika ne
zadošča. Rešiti moramo Maxwellove enačbe (enačbe~\ref{eq:Maxwell1}--\ref{eq:Maxwell4}) 
z ustreznimi robnimi pogoji (enačbe~\ref{eq:robni-pogoji} in \ref{eq:robni-pogoji5}),
kar je za cilindrična vlakna dokaj dolg in zapleten račun. Zato določimo najprej 
osnovne značilnosti valovanja, ki se širi po planparalelnem vodniku.

Glede na smer polarizacije jakosti električnega polja
ločimo dva primera (slika~\ref{fig:TETM}). Če je smer jakosti električnega polja
vzporedna z mejnima ploskvama (smer $y$), 
govorimo o transverzalnem električnem (TE) valovanju\index{Polarizacija!TE}. 
V nasprotnem primeru, ko je 
z mejnima ploskvama vzporedna jakost magnetnega polja in 
leži jakost električnega polja v ravnini $xz$, 
govorimo o transverzalnem magnetnem (TM) valovanju\index{Polarizacija!TM}.
\begin{figure}[ht]
\centering
\def\svgwidth{128truemm} 
\input{slike/10_TETM.pdf_tex}
\caption{Polarizacija TE valovanja (a) in polarizacija TM valovanja (b) v valovnem vodniku}
\label{fig:TETM}
\end{figure}

Geometrijskemu žarku, ki se odbija na mejah stranice, ustreza v 
valovni sliki val, ki ima prečno komponento valovnega
vektorja\index{Valovni vektor} $k_{x}$ različno od nič. Ker je valovanje v prečni smeri 
omejeno na sredico končne debeline (naj bo debelina sredice enaka $a$), lahko
$k_{x}$ zavzame le diskretne vrednosti. Te so v grobem približno enake $N\pi/a$, pri čemer je $N$
celo število. Pravimo, da vsak $N$ določa en rod valovanj v vodniku. Po drugi strani 
obstaja največji $k_x$, za katerega približno velja
\begin{equation}
k_{x \mathrm{max}} \approx k_0 \sin\alpha_{\rm max} = k_0 \sqrt{n_1^2 -n_2^2}.
\end{equation}
Število rešitev za $k_x$ je tako omejeno, točno določeno in odvisno
od razlike lomnih količnikov in od debeline vodnika oziroma polmera vlakna. 
V nadaljevanju bomo spoznali, da v optičnih vlaknih en rod vselej obstaja,
za razliko od kovinskih vodnikov, po katerih se pod določeno frekvenco
valovanje ne more širiti. Enorodovna optična vlakna, torej vlakna, po katerih se širi
en sam rod, imajo še posebej lepe lastnosti za uporabo v komunikacijskih
sistemih\index{Optični vodnik!enorodovni}\index{Optično vlakno!enorodovno}.

Povejmo še nekaj o hitrosti valovanja v vodniku.
Naj bo $\beta$ velikost komponente valovnega vektorja vzdolž smeri $z$ in odvisnost 
polja od koordinate vzdolž vodnika $\exp (i\beta z)$. Velikost
valovnega vektorja v sredici vodnika zapišemo kot
\begin{equation}
k_1 = n_{1}\frac{\omega}{c_0}=\sqrt{\beta^{2}+k_{x}^{2}}
\label{9.0}.
\end{equation}
Vidimo, da za dano vrednost $k_{x}$ zveza med valovnim številom $\beta$
in krožno frekvenco $\omega$ ni linearna. Fazna hitrost\index{Hitrost valovanja!fazna} 
valovanja $v_{f}=\omega/\beta$ je tako
odvisna od krožne frekvence in nastopi disperzija\index{Disperzija}. Grupna ali skupinska
hitrost\index{Hitrost valovanja!grupna} $v_{g}=d\omega/d\beta$ 
se zaradi nelinearne zveze med $\beta$ in $\omega$ 
razlikuje od fazne hitrosti in njena frekvenčna odvisnost 
ima pomembne posledice za uporabo vlaken pri prenosu podatkov. 

\section{Račun lastnih rodov v planparalelnem vodniku}
\index{Optični vodnik!lastni rodovi}
\index{Optični vodnik!planparalelni}
Poiščimo rešitve valovne enačbe v planparalelnem vodniku. 
To je preprost dvodimenzionalen model optičnega vlakna, ki je sestavljen iz 
plasti prozornega dielektrika z lomnim količnikom $n_1$ in plašča z lomnim količnikom $n_2$.
Zaradi enostavnosti privzamemo, da je plašč na obeh straneh sredice enak.
Sredica naj bo debela $a$, izhodišče koordinatnega sistema
si izberemo na sredini plasti. Ločimo tri območja, kjer rešujemo valovno enačbo:
območje II označuje sredico, območji I in III naj bosta v plašču nad sredico oziroma pod njo
(slika~\ref{fig:vodnikracun}).

\begin{figure}[ht]
\centering
\def\svgwidth{120truemm} 
\input{slike/10_VodnikRacun.pdf_tex}
\caption{K izračunu lastnih rodov v simetričnem planparalelnem vodniku}
\label{fig:vodnikracun}
\end{figure}

Krajevni del valovne enačbe, ki jo rešujemo, opisuje Helmholtzeva 
enačba\index{Helmholtzeva enačba} (enačba~\ref{eq:Helmholtz})
\begin{equation}
\nabla^{2}\mathbf{E}+n^2(x)\,k_{0}^{2}\,\mathbf{E}=0,
\label{9.1}
\end{equation}
kjer je $k_{0}=\omega/c$. Pri tem $n(x)$  nezvezno spremeni vrednost ob prehodu iz sredice v plašč. 

Nastavek za rešitev naj bo oblike 
\begin{equation}
{\mathbf E}(x,z)=\mathbf{e}\psi\left(x\right)\, e^{i\beta z},
\label{9.2}
\end{equation}
pri čemer $\mathbf{e}$ označuje enotski vektor v smeri polarizacije valovanja.
Omejimo se le na primer TE polarizacije\index{Optični vodnik!TE rodovi} (za izračun lastnih rodov 
TM polariziranega
valovanja glej nalogo~\ref{naloga:TM}). Vstavimo nastavek (enačba~\ref{9.2}) v enačbo
(\ref{9.1}) in dobimo
\begin{equation}
\frac{d^{2}{\bf \psi}}{dx^{2}}+\left(k_{0}^{2}n_1^{2}-\beta^{2}\right){\bf \psi}=0
\qquad \textrm{v sredici (obmo\v cje II)} 
\label{9.3a}
\end{equation}
in 
\begin{equation}
\frac{d^{2}{\bf \psi}}{dx^{2}}+\left(k_{0}^{2}n_2^{2}-\beta^{2}\right){\bf \psi}=0
\qquad \textrm{v plašču (obmo\v cji I in III).} 
\label{9.3b}
\end{equation}
Iz zveze~(enačba~\ref{9.0}) sledi $k_0^2n_1^2-\beta^2=k_x^2$, zato lahko rešitve prve enačbe
zapišemo v obliki
\begin{equation}
\psi_{\mathrm{II}}(x) = C \cos(k_x x)+D \sin(k_x x),
\end{equation}
rešitve v plašču pa so oblike
\begin{equation}
\psi_{\mathrm{I}}(x) = A \exp(-\kappa x)+B \exp(\kappa x)\quad \mathrm{in}\quad
\psi_{\mathrm{III}}(x) = F \exp(-\kappa x)+G \exp(\kappa x),
\end{equation}
pri čemer smo vpeljali $\kappa^2= \beta^2-k_0^2n_2^2$.
Da valovanje ostane ujeto v vlakno, mora biti $\kappa$ realno število.
Le tako namreč dosežemo eksponentno pojemanje jakosti električnega polja 
z oddaljenostjo od sredice,
sicer je valovanje v vseh treh območjih oscilatorno in ni ujeto v vlakno. 

Iz zahteve, da sta $k_x$ in $\kappa$ realna, sledi pogoj za valovno 
število\index{Valovno število} $\beta$
\boxeq{vlaknobeta}{
k_0 n_2 < \beta < k_0 n_1.
}

Poleg tega zahteva po končnosti rešitve da pogoj, da je v območju I 
(pri $x>a/2$) koeficient $B=0$, 
v območju III (pri $x<-a/2$) pa $F=0$. Hitro ugotovimo, da so zaradi 
simetrije vlakna lastne rešitve lahko le sode ali lihe funkcije. 

\subsection*{Sode rešitve}
\index{Optični vodnik!sodi rodovi}
Poglejmo najprej sode rešitve. V sredici bo različen od nič samo $C$, 
v območjih I in III pa bosta amplitudi enaki in $A = G$ (slika~\ref{fig:TESodi}\,a). 
Dobimo 
\begin{align}
\psi_{\mathrm{I}}(x) =&~ A \exp(-\kappa x), \\
\psi_{\mathrm{II}}(x) =&~ C \cos(k_x x) \qquad \mathrm{in}\\
\psi_{\mathrm{III}}(x) =&~ A \exp(\kappa x).
\end{align}
Zvezo med koeficientoma $A$ in $C$ določimo z upoštevanjem robnih pogojev. Na meji
med sredico in plaščem morata biti tangencialni komponenti 
jakosti električnega in magnetnega polja zvezni (enačbi~\ref{eq:robni-pogoji5}). 
Iz tega izpeljemo pogoj, da se za TE valovanje
na meji ohranja amplituda jakosti električnega polja. Pri $x = a/2$ zapišemo
\begin{equation}
A \exp(-\kappa a/2) = C \cos(k_x a/2).
\label{eq:rps1}
\end{equation}
Drugi pogoj sledi iz zveze $\nabla\times{\bf E}=i\omega\mu_{0}{\bf H}$, ki izhaja
neposredno iz Maxwellove enačbe~(\ref{eq:Maxwell2}). Ker se na meji ohranja
tangencialna komponenta ${\bf H}$, to je v tem primeru $H_z$, se posledično ohranja 
odvod jakosti električnega polja $dE_y/dx$. 
Pri $x = a/2$  velja
\begin{equation}
-A \kappa \exp(-\kappa a/2) = -C k_x \sin(k_x a/2).
\label{eq:rps2}
\end{equation}
Enačbo za $k_x$ izpeljemo iz zahteve, da sta robna 
pogoja  (enačbi~\ref{eq:rps1} in \ref{eq:rps2}) hkrati izpolnjena. Izraza za robna pogoja delimo in dobimo sekularno 
enačbo za sode rešitve \index{Sekularna enačba!sodi rodovi}
\boxeq{sekular1}{
\frac{\kappa}{k_x} = \tan \frac{k_x a}{2}.
}
Rešitve enačbe so diskretne in vsaki vrednosti $k_x$ ustreza 
en sodi rod oziroma sodi lastni način. Pri tem je zveza med $\kappa$ in $k_x$  
\boxeq{kappak}{
\kappa^{2}+ k_x^{2}=k_{0}^{2}\left(n_{1}^{2}-n_{2}^{2}\right).
}

\subsection*{Lihe rešitve}
\index{Optični vodnik!lihi rodovi}
Oglejmo si še lihe rešitve v planparalelnem vodniku. V sredici je od nič različen
le $D$, polji v plašču  sta nasprotno enaki in $A = -G$ (slika~\ref{fig:TESodi}\,b). Sledi
\begin{align}
\psi_{\mathrm{I}}(x) =&~ A \exp(-\kappa x),\\
\psi_{\mathrm{II}}(x) =&~ D \sin(k_x x),\\
\psi_{\mathrm{III}}(x) =&~ -A \exp(\kappa x).
\end{align}
Z upoštevanjem zveznosti jakosti električnega polja in njenega odvoda na 
meji med sredico in plaščem zapišemo robna pogoja pri $x=a/2$
\begin{equation}
A \exp(-\kappa a/2) = D \sin(k_x a/2)
\end{equation}
in 
\begin{equation}
-\kappa A \exp(-\kappa a/2) = D k_x \cos(k_x a/2).
\end{equation}
Ustrezna sekularna enačba za lihe rešitve je\index{Sekularna enačba!lihi rodovi}
 \boxeq{sekular2}{
-\frac{k_x}{\kappa} = \tan \frac{k_x a}{2}.
}
\begin{figure}[ht]
\centering
\def\svgwidth{128truemm} 
\input{slike/10_TESodi.pdf_tex} 
\caption{Prečna odvisnost jakosti električnega polja za sode (a) in lihe (b) rodove v 
simetričnem planparalelnem valovnem vodniku. Modra barva označuje sredico, beli del 
pa plašč vodnika. 
}
\label{fig:TESodi}
\end{figure}
\vglue-3truemm
\begin{remark}
Če ne prej, je bralec ob slikah~\ref{fig:TESodi} zagotovo opazil podobnost s kvantnim 
delcem, ujetim v končni enodimenzionalni potencialni jami. Svetloba, ujeta v vodnik ali
vlakno, ustreza vezanim stanjem delca, numerična odprtina pa je tisti parameter, 
ki določa globino potencialne jame. Pri majhnih vrednostih obstaja ena sama rešitev 
za vezano stanje, pri globlji jami je rešitev več. Podobno kot v kvantni mehaniki
tudi v tem primeru ena rešitev za vezano stanje vedno obstaja.\footnote{Glej npr.
J. Strnad, {\it Fizika, 3. del}, tretja izdaja, DMFA--založništvo (2018).} 
\end{remark}

Sekularnih enačb za lastne rodove (enačbi~\ref{sekular1} in \ref{sekular2}) ne moremo rešiti 
analitično. Rešujemo jih numerično, zelo nazorna je tudi grafična predstavitev
(slika~\ref{fig:TEsec}). Enačbo za sode rodove~(enačba~\ref{sekular1}) pomnožimo s 
$k_xa$ in narišemo funkcijo $k_xa\, \tan (k_xa/2)$ (črne črte). Enačbo za lihe 
rodove~(enačba~\ref{sekular2}) preoblikujemo in narišemo $-k_xa\, \cot (k_xa/2)$ (rdeče črte).
Nato pri danih parametrih $n_1$, $n_2$, $a$ in $k_0$ narišemo krožnico za $\kappa a$, 
ki sledi iz enačbe~(\ref{kappak})
\begin{equation}
 (\kappa a)^2+ (k_x a)^{2}=k_{0}^{2} a^2\left(n_{1}^{2}-n_{2}^{2}\right) = k_0^2\,a^2\,NA^2.
\end{equation}
Število presečišč krivulj s krožnico določa število lastnih rodov v vodniku in lega presečišča
pripadajočo vrednost $k_x$. Na sliki so narisane tri krožnice za 
tri različne debeline vodnika $a$ (pri istih lomnih količnikih in isti valovni 
dolžini svetlobe). V najtanjšem vodniku (zelena črta) je presečišče
le eno in tak vodnik imenujemo enorodovni vodnik\index{Optični vodnik!enorodovni}. 
Z večanjem debeline število presečišč -- in s tem število lastnih rodov -- narašča in taki vodniki so 
večrodovni\index{Optični vodnik!večrodovni}. Lastni rodovi v večrodovnem vodniku
so izmenično sodi in lihi, pri čemer je osnovni rod vedno sod. 
\begin{figure}[ht]
\centering
\def\svgwidth{60truemm} 
\input{slike/10_TEsekularna_nova.pdf_tex}
\caption{K izračunu $k_x$ v valovnem vodniku
za TE polarizacijo. Število presečišč krožnice s krivuljami določa število lastnih rodov 
v vodniku. Zelena krožnica prestavlja enorodovni vodnik, turkizna dvorodovnega in modra
petrodovnega s tremi sodimi in dvema lihima rešitvama.}
\label{fig:TEsec}
\end{figure}

S slike~\ref{fig:TEsec} razberemo še eno pomembno lastnost vodnikov. 
Ne glede na to, kako majhen je polmer krožnice, krožnica vedno seka črno krivuljo. 
To pomeni, da v še tako tankem vodniku vsaj ena rešitev za $k_x$ vedno obstaja
in ta je vedno soda. 

Ocenimo še število lastnih rodov v vodniku. S slike~\ref{fig:TEsec} vidimo, da je 
največja možna vrednost $k_x$ omejena s polmerom krožnice $k_0aNA$, ki ga imenujemo
tudi normirana frekvenca\index{Normirana frekvenca} $V$. Do te vrednosti je 
po ena rešitev na vsak interval dolžine $\pi$, izmenično soda in liha, 
zato je celotno število rodov za eno polarizacijo\index{Optični vodnik!število rodov}
\begin{equation}
N \approx \frac{k_0 a NA}{\pi} = \frac{V}{\pi}.
\end{equation}
Za enorodovni vodnik velja $V < \pi$ oziroma $a< \lambda/2 NA$.
Tipični enorodovni vodnik ima tako debelino $a\lesssim 5~\si{\micro\meter}$, medtem ko je
debelina večrodovnega vodnika z okoli 20 rodovi $a\sim 50~\si{\micro\meter}$.

\begin{naloga}
\label{naloga:TM}
Ponovi izračun za TM valovanje\index{Optični vodnik!TM rodovi}\index{Sekularna enačba!TM rodovi}
in pokaži, da sta sekularni enačbi enaki 
\begin{equation}
\frac{\kappa}{k_x} \left(\frac{n_1}{n_2}\right)^2= 
\tan \frac{k_x a}{2} \qquad \mathrm{in} \qquad -\frac{k_x}{\kappa} \left(\frac{n_2}{n_1}\right)^2= 
\tan \frac{k_x a}{2}.
\end{equation}
Namig: Zapiši enačbe za jakost magnetnega polja ${\bf H}$ in upoštevaj ustrezne robne pogoje.
\end{naloga}

Podoben račun lahko naredimo tudi za TM valovanje (glej nalogo~\ref{naloga:TM}). Zaradi drugačnih
robnih pogojev se sekularni enačbi razlikujeta od tistih za TE polarizacijo. Razlika je v
faktorju $(n_1/n_2)^2$, ki je v tipičnem vodniku zelo blizu ena. Zato se tudi rešitve 
za prečno komponento $k_x$ le malo razlikujejo. Bolj pomembna je ugotovitev, da je število
dovoljenih rodov za TM polarizacijo enako številu dovoljenih rodov za TE polarizacijo, 
saj je največji $k_x$ v obeh primerih določen s polmerom krožnice $V$. 
Vse lastne rodove, ki se širijo v danem planparalelnem vodniku, torej zajamemo z opisom TE sodih in lihih ter 
TM sodih in lihih rodov.

Ugotovili smo, da je jakost električnega polja tudi izven sredice vodnika različna od nič. 
Poglejmo še, kako je z energijskim tokom. Čeprav se velika večina pretaka po sredici, 
delež, ki se pretaka po plašču, ni vedno zanemarljiv. To posebej velja za višje rodove. 
Delež energijskega toka, ki se pretaka po sredici, izračunamo\index{Gostota energijskega toka}
z integralom
\boxeq{confinement}{
\Gamma = \frac{\int_{-a/2}^{a/2}j\, dS}{\int_{-\infty}^{\infty}j\, dS}.
}
\begin{naloga}
Pokaži, da je razmerje med energijskim tokom $P_p$, ki se pretaka po plašču, in energijskim tokom, 
ki se pretaka po sredici vodnika $P_s$, za sode rodove
\begin{equation}
\frac{P_p}{P_s}= \frac{n_2}{n_1}\frac{2 k_x}{\kappa} \frac{\cos^2(k_x a/2)}{k_xa + \sin(k_xa)},
\end{equation}
in za lihe rodove
\begin{equation}
\frac{P_p}{P_s}= \frac{n_2}{n_1}\frac{2 k_x}{\kappa} \frac{\sin^2(k_x a/2)}{k_xa - \sin(k_xa)}.
\end{equation}
\end{naloga}

\section{Cilindrično vlakno}
\label{chap:Cilinder}
\index{Optično vlakno}
V praksi svetlobo navadno usmerjamo po optičnih vlaknih, ki imajo cilindrično geometrijo.
Najpreprostejši primer je cilindrično vlakno, v katerem je lomni količnik 
sredice konstanten in nekoliko večji od lomnega količnika plašča. Navadno je 
$n_1 - n_2 \sim 0,001$. Pogosto se uporablja
bolj zapletene konstrukcije, pri katerih se lomni količnik sredice spreminja z
oddaljenostjo od osi.\footnote{Za doprinos k razvoju in uporabi optičnih vlaken je leta 2009 Charles
K. Kao prejel Nobelovo nagrado.} Z zapletenejšo geometrijo namreč zmanjšamo disperzijo v vlaknu in s tem
povečamo zmogljivost prenašanja velike količine podatkov na dolge razdalje -- 
najzmogljivejša vlakna zmorejo prenos več deset terabitov na 
sekundo.\footnote{D. Hillerkuss et al., Nat. Phot. $\mathbf{5}$, 364 (2011).}

Račun za širjenje svetlobe po cilindričnem vlaknu s homogeno sredico
je podoben kot za planparalelni vodnik, vendar je precej bolj
zapleten. V cilindrični geometriji namreč ni delitve na čiste električne in 
magnetne transverzalne valove, saj so robni pogoji sklopljeni: polje, ki na enem delu
vlakna kaže v radialni smeri, kaže na drugem delu v tangencialni. Na splošno se rešitve izražajo 
v obliki kombinacij Besslovih funkcij. Izkaže se, da je osnovni rod, ki se  širi po
cilindričnem vlaknu, po obliki zelo podoben osnovnemu Gaussovemu snopu, zato je sklopitev
laserskih snopov v optična vlakna zelo učinkovita. Tudi v cilindričnih vlaknih 
obstaja končno število lastnih rodov, ki je odvisno od polmera sredice in
lomnih količnikov sredice in plašča. Če je polmer zadosti majhen (razlika lomnih
količnikov navadno je), obstaja le en lastni rod in optično vlakno je 
enorodovno.\index{Optično vlakno!enorodovno} Sicer je vlakno večrodovno\index{Optično 
vlakno!večrodovno}.

\subsection*{Valovna enačba v cilindričnem vlaknu}
\index{Valovna enačba}
Izračun rodov v cilindričnem vlaknu presega okvir te knjige, zato
si oglejmo le izhodiščne enačbe in rešitve.\footnote{Za račun glej npr. A. Yariv in 
P. Yeh, {\it Photonics}, šesta izdaja, Oxford
University Press (2007).} Za jakost električnega polja velja 
Helmholtzeva enačba~(enačba~\ref{eq:Helmholtz})\index{Helmholtzeva enačba}
\begin{equation}
\nabla^2 \mathbf{E} + n^2(r)\, k_0^2\, \mathbf{E} = 0,
\end{equation}
pri čemer je $n(r<a)=n_1$ lomni količnik sredice in $n(r>a)=n_2$ 
lomni količnik plašča, ki je dovolj debel, da njegova debelina ne 
vpliva na potovanje svetlobe. $\mathbf{E}$ in $\mathbf{H}$ sta vektorja in imata po 
tri komponente, ki pa so med seboj odvisne. Računamo v cilindričnem koordinatnem sistemu,
v katerem je os $z$ vzdolž osi vlakna. Izračunajmo naprej $E_z$, saj so za to
komponento robni pogoji nesklopljeni. Uporabimo nastavek
\begin{equation}
E_z = R(r)e^{i \nu \varphi}e^{i \beta z},
\end{equation}
pri čemer je $r$ razdalja od osi, $\varphi$ polarni kot in $\nu$ celo število zaradi 
zahteve po enoličnosti rešitve pri spremembi
kota za $2\pi$. Za $R(r)$ v sredici vlakna tako dobimo enačbo
\begin{equation}
r^2 R(r)'' + r R(r)' + (k_s^2r^2 - \nu^2)R(r) = 0,
\label{10BS}
\end{equation}
kjer za prečno komponento valovnega vektorja velja
\begin{equation}
k_s^2=k_0^2n_1^2- \beta^2,
\label{eq:ks}
\end{equation}
in v plašču
\begin{equation}
r^2 R(r)'' + r R(r)' + (-\kappa^2r^2 - \nu^2)R(r) = 0,
\label{10BP}
\end{equation}
kjer je 
\begin{equation}
\kappa^2=\beta^2-k_0^2n_2^2.
\end{equation}
V enačbah (\ref{10BS}) in (\ref{10BP}) prepoznamo Besslovo diferencialno enačbo. 
Upoštevajoč le funkcije, ki na izbranem območju ne divergirajo, zapišemo rešitve v sredici kot
\begin{equation}
E_z (r, \varphi, z) = A J_\nu(k_sr)\sin(\nu \varphi)e^{i \beta z} \quad  \mathrm{in} \quad 
H_z (r, \varphi, z) = B J_\nu(k_sr)\cos(\nu \varphi)e^{i \beta z}.
\end{equation}
Podobno zapišemo tudi rešitve v plašču
\begin{equation}
E_z (r, \varphi, z)= C K_\nu(\kappa r)\sin(\nu \varphi)e^{i \beta z} \quad \mathrm{in} \quad 
H_z (r, \varphi, z)= D K_\nu(\kappa r)\cos(\nu \varphi)e^{i \beta z}.
\end{equation}
Pri tem so $A,B,C$ in $D$ konstante, $J_\nu(x)$ je Besslova funkcija prve vrste reda 
$\nu$ in $K_\nu(x)$ modificirana Besslova funkcija druge vrste reda $\nu$ 
(slika~\ref{fig:J01}). 
\begin{figure}[ht]
\centering
\def\svgwidth{128truemm} 
\input{slike/10_Bessel1.pdf_tex} 
\caption{Besslove funkcije: (a) Besslove funkcije prve vrste 
$J_0(x)$ (črna), $J_1(x)$ (rdeča) in $J_2(x)$ (modra), 
ki predstavljajo oblike rešitev v sredici vlakna, in (b)
modificirane Besslove funkcije druge vrste $K_0(x)$ (črna), $K_1(x)$ (rdeča) in $K_2(x)$ (modra), 
ki prestavljajo rešitev v plašču vlakna.}
\label{fig:J01}
\end{figure}

Ko enkrat poznamo komponenti $E_z$ in $H_z$, lahko z uporabo Maxwellovih enačb izračunamo še 
preostale komponente $E_r$, $E_\varphi$, $H_r$ in $H_\varphi$. 
Nato z upoštevanjem robnih pogojev zapišemo štiri enačbe za 
pet neznank ($A,B,C,D$ in $\beta$),
tako da ostane ena spremenljivka (amplituda polja) prosta. Na ta način izračunamo celotni 
jakosti električnega in magnetnega polja v vlaknu in podobno kot pri valovnem vodniku 
tudi tukaj zapišemo sekularno enačbo, ki jo moramo rešiti numerično. Pri vsakem $\nu$ obstaja 
več rešitev, zato lastne rodove označujemo s parom indeksov $\nu$ in $m$, npr. TE$_{01}$.

\subsection*{TE in TM rodovi}
Najprej si oglejmo rešitve, pri katerih je $\nu=0$ in so zato neodvisne od kota $\varphi$. 
V klasični sliki so to žarki, ki potujejo po osi vlakna. Iz robnih pogojev sledi, da 
gre za transverzalne TE rodove, za katere velja $E_z=0$, $E_r=0$ in $E_\varphi \propto J_1(k_sr)$.
Jakost električnega polja za TE zapišemo kot \index{Optično vlakno!TE rodovi}
\begin{equation}
\mathbf{E} \propto \mathbf{e}_\varphi \, J_1(k_s r),
\end{equation}
kjer je $\mathbf{e}_\varphi$ enotski vektor.
Podobno lahko prepoznamo tudi TM rodove, pri katerih je $H_z=0$, $H_r=0$ in $H_\varphi \propto J_1(k_sr)$.
Jakost električnega polja za TM rodove je\index{Optično vlakno!TM rodovi}
\begin{equation}
\mathbf{E} \propto \mathbf{e}_r \, J_1(k_s r)
\end{equation}
kjer je $\mathbf{e}_r$ enotski vektor v radialni smeri. 
Amplitudi jakosti električnega polja sta za TE in TM rodove enaki, zato
sta enaki tudi sliki gostote svetlobnega toka (slika~\ref{fig:TE01}). Opazimo, da je v osi
vlakna gostota svetlobnega toka enaka nič, zato sklepamo, da to niso osnovni načini 
širjenja svetlobe po cilindričnem vlaknu. 
\begin{figure}[ht]
\centering
\def\svgwidth{90truemm} 
\input{slike/10_TE01.pdf_tex}
\caption{Intenziteta in smer električnega polja v vlaknu za rodova TE$_{01}$ in TM$_{01}$
}
\label{fig:TE01}
\end{figure}
\vglue-5truemm
Podobno kot smo zapisali sekularno enačbo v valovnem vodniku (enačbi~\ref{sekular1}
in~\ref{sekular2}), tudi tukaj zapišemo enačbo za dovoljene vrednosti $k_s$. 
Ob približku, da se lomna količnika\index{Sekularna enačba!za cilindrično vlakno}
sredice in plašča le malo razlikujeta, je poenostavljena enačba za TE 
valovanje\footnote{Glej npr. C. R. Pollock, {\it Fundamentals of Optoelectronics}, Irwin (1995).}
\boxeq{sekFiber}{
-k_sa\,\frac{J_0(k_sa)}{J_1(k_sa)}=\kappa a\,\frac{K_0(\kappa a)}{K_1(\kappa a)},
}
pri čemer velja zveza $\kappa^2+k_s^2=(NA)^2k_0^2$. Zaporedne rešitve enačbe ustrezajo rodovom TE$_{0m}$. 
Za izračun valovnih vektorjev za rodove TM$_{0m}$, moramo levo stran enačbe~(\ref{sekFiber}) pomnožiti z $(n_1/n_2)^2$. Ker je ta faktor približno ena, se rešitve enačb med seboj le malo razlikujejo.

\begin{figure}[ht]
\centering
\def\svgwidth{65truemm} 
\input{slike/10_TE_cilinder.pdf_tex}
\caption{K izračunu prečnih komponent valovnega vektorja $k_s$ v 
cilindričnem vlaknu za TE polarizacijo.
Leva stran sekularne enačbe je narisana s črno, desna stran pa za tri različne vrednosti parametra 
$V=k_0aNA$.}
\label{fig:TEsecFib}
\end{figure}
\vglue-3truemm
Zapisano sekularno enačbo rešujemo ali numerično ali se je lotimo grafično in 
na sliki~\ref{fig:TEsecFib} poiščemo presečišča krivulj. Na sliki je leva stran enačbe prikazana
s črno barvo in desna za tri različne vrednosti $V= k_0 a\,NA$. Pri velikem $V$ (modra črta) ima sistem tri 
rešitve, pri srednjem $V$ (turkizna črta) eno rešitev, medtem ko se pri majhnem $V$ (zelena črta) 
krivulji ne sekata. To pomeni, da pri dovolj majhnem polmeru vlakna TE rod ne obstaja. 

Zapišimo to ugotovitev še matematično. Desna stran enačbe da realne rešitve za 
$k_s a \le  V$. Po drugi strani leva stran enačbe postane pozitivna šele pri 
$J_0 (k_s a)  = 0$, to je pri $k_s a= 2,405$. Sistem ima vsaj eno rešitev, če velja
$V>2,405$. Polmer, pri katerem se TE$_{01}$ (ali TM$_{01}$) valovanje 
z dano valovno dolžino sploh širi po vlaknu, je torej navzdol omejen z 
\boxeq{10_cutoff}{
a \geq \frac{2,405}{k_0 NA}.
}

\subsection*{Hibridni HE in EH rodovi}
\index{Optično vlakno!HE rodovi}
\index{Optično vlakno!EH rodovi}
Poglejmo zdaj rešitve, pri katerih je $\nu \neq 0$. V tem primeru je 
vseh šest komponent električnega in magnetnega polja valovanja različnih od nič in vsi rodovi
imajo tudi komponento polja v smeri $z$. Take rodove imenujemo hibridni rodovi in jih 
označimo s HE, če je $E_z$ razmeroma velik ali vsaj primerljiv z $E_r$ in $E_\varphi$, 
oziroma z EH, če je $H_z$ po velikosti primerljiv s $H_r$ in $H_\varphi$ ali večji od njiju. 

Privzamemo, da se lomna količnika sredice in plašča le malo razlikujeta. Tudi v tem 
približku je sekularna enačba za hibridne rodove precej zapletena in je ne bomo 
zapisali.\footnote{Glej npr. C. R. Pollock, {\it Fundamentals of Optoelectronics}, Irwin (1995).}
Oglejmo si le njihovo obliko (slika~\ref{fig:HE11}). Najpomembnejši hibridni rod je HE$_{11}$, 
ki je sorazmeren z $J_0(k_sr)$ in zato v središču različen od nič. 
To je osnovni rod, za katerega rešitev sekularne enačbe vedno obstaja in se
zato širi po še tako tankem vlaknu. Zaradi rotacijske simetrije si lahko smer 
polarizacije lastnega roda poljubno izberemo, mi izberimo smeri $x$ in $y$.
\begin{figure}[ht]
\centering
\def\svgwidth{80truemm} 
\input{slike/10_HE11.pdf_tex}\\
\def\svgwidth{80truemm} 
\input{slike/10_HE21.pdf_tex} \\
\def\svgwidth{80truemm} 
\input{slike/10_HE31.pdf_tex} \\
\def\svgwidth{80truemm} 
\input{slike/10_EH11.pdf_tex}
\caption{Intenziteta in smer električnega polja v vlaknu za rodove
HE$_{11}$, HE$_{21}$, HE$_{31}$ in EH$_{11}$}
\label{fig:HE11}
\end{figure}
\vglue-5truemm
Po obliki je osnovni HE$_{11}$ rod zelo podoben Gaussovemu profilu $\exp(-r^2/w^2)$,
zato ga lahko razmeroma dobro opišemo z Gaussovim približkom. 
Pri tem efektivni polmer snopa
$w$ izračunamo po Marcusejevi\index{Marcusejeva formula}
\index{Gaussov snop!efektivni polmer} formuli\footnote{D. Marcuse, Bell Syst. Tech. J. $\mathbf{56}$, 
703 (1977).}
\begin{equation} 
w = \left(0,65 + \frac{1,619}{V^{3/2}}+\frac{2,879}{V^{6}}\right)\,a,
\label{Marcuse}
\end{equation}
pri čemer je $V = k_0 a\,NA $. Podobnost profila osnovnega
HE$_{11}$ roda z Gaussovo funkcijo omogoča zelo dobro sklopitev Gaussovih
snopov, ki izhajajo iz laserja, v cilindrična vlakna.

Na sliki~\ref{fig:HE11} je poleg osnovnega HE$_{11}$ roda še nekaj primerov višjih rodov. Opazimo, 
da imajo vsi rodovi, razen osnovnega, v osi vlakna minimum. Poleg tega opazimo tudi podobnost med 
oblikami posameznih rodov, ki je posledica majhne razlike med lomnima količnikoma sredice in plašča
($n_1 \approx n_2$). V takem primeru se sekularne enačbe poenostavijo, nekateri rodovi so 
med seboj degenerirani in dajo približno enako rešitev. 
Poleg rodov z enako obliko in različno polarizacijo so tako 
med seboj degenerirani še HE$_{\nu+1,m}$ in EH$_{\nu-1,m}$ rodovi. Degenerirane 
rodove lahko združimo v linearne kombinacije in nastanejo pretežno 
linearno polarizirani LP rodovi. 

\subsection*{LP rodovi}
\index{Optično vlakno!LP rodovi}
Za praktično uporabo so najpomembnejši linearno polarizirani (LP) rodovi.\footnote{Glej 
npr. A. Yariv in P. Yeh, {\it Photonics}, šesta izdaja, Oxford
University Press (2007).}  Taki rodovi niso
točne rešitve valovne enačbe v cilindrični geometriji, ampak jih zapišemo kot linearno 
kombinacijo lastnih rodov, ki so zaradi majhne razlike med lomnima količnikoma sredice
in plašča degenerirani. Tudi te rodove označimo z dvema indeksoma: prvi določa število azimutalnih
vozlov in drugi število radialnih vrhov. Poglejmo nekaj primerov (slika~\ref{fig:LP}).

Osnovni HE$_{11}$ rod je pri majhnih razlikah lomnih količnikov praktično linearno polariziran in ustreza polariziranemu rodu LP$_{01}$. Jakost električnega polja v njem je 
\begin{equation}
\mathbf{E}_\mathrm{LP01} \propto 
\left \{
  \begin{matrix}
  \mathbf{e}_x \\ \mathbf{e}_y 
  \end{matrix}
\right \} \, J_0(k_s r),
\end{equation}
saj ima dve možni smeri polarizacije. Na splošno HE$_{1m}$ rodovi ustrezajo rodovom LP$_{0m}$. 

Višje rodove, na primer LP$_{11}$ sestavimo kot linearno kombinacijo 
TE$_{01}$ ali TM$_{01}$ in HE$_{21}$.
Jakost električnega polja v LP$_{11}$ je tako v obliki štirih možnih kombinacij
\begin{equation}
\mathbf{E}_\mathrm{LP11} \propto \left \{
  \begin{matrix}
  \mathbf{e}_x \\ \mathbf{e}_y 
  \end{matrix}
\right \} \, J_1(k_s r)
\left \{
  \begin{matrix}
  \cos\varphi  \\ \sin\varphi 
  \end{matrix}
\right \}
\end{equation}
kar opisuje štiri različne oblike rodov LP$_{11}$.

Rodove LP$_{21}$, ki nastanejo kot kombinacija HE$_{31}$
in EH$_{11}$ rodov, zapišemo kot eno od štirih kombinacij
\begin{equation}
\mathbf{E}_\mathrm{LP21} \propto \left \{
  \begin{matrix}
  \mathbf{e}_x \\ \mathbf{e}_y 
  \end{matrix}
\right \} \, J_2(k_s r)
\left \{
  \begin{matrix}
  \cos 2 \varphi  \\ \sin 2 \varphi 
  \end{matrix}
\right \}.
\end{equation}
Linearno polarizirani LP rodovi imajo precejšnjo uporabno vrednost. To so 
namreč rodovi, ki jih v vlaknu vzbudimo, ko nanj posvetimo s polarizirano 
lasersko svetlobo. Zavedati  se moramo, da to niso lastni rodovi vlakna, 
ampak njihove linearne kombinacije, ki po vlaknu potujejo z malenkost različnimi
hitrostmi. Polarizacija svetlobe se zato vzdolž vlakna ne ohranja povsem.
\begin{remark}
Število rodov v večrodovnem cilindričnem vlaknu pri izbrani normirani 
frekvenci $V = k_0a\, NA$ lahko približno ocenimo z uporabo asimptotičnega 
razvoja Besslovih funkcij za velike argumente. Približna ocena vključuje 
vse dovoljene rešitve pri vsaki vrednosti $\nu$ in dve polarizaciji. 
da\index{Optično vlakno!število rodov}\footnote{Glej npr. C. R. Pollock, {\it Fundamentals of Optoelectronics}, Irwin (1995).}
\begin{equation} 
N = \frac{4 V^2}{\pi^2}.
\end{equation}
V vlaknu s polmerom sredice $20~\si{\micro\meter}$ in numerično aperturo 0,2
se  pri valovni dolžini $1~\si{\micro\meter}$ po tej oceni lahko širi 256 rodov. 
\end{remark}
\begin{figure}
\centering
\def\svgwidth{63truemm} 
\input{slike/10_LP01.pdf_tex} \\
\def\svgwidth{63truemm} 
\input{slike/10_LP02.pdf_tex} \\
\def\svgwidth{63truemm} 
\input{slike/10_LP11a.pdf_tex} \\
\def\svgwidth{63truemm} 
\input{slike/10_LP11b.pdf_tex} \\
\def\svgwidth{63truemm} 
\input{slike/10_LP21a.pdf_tex} \\
\def\svgwidth{63truemm} 
\input{slike/10_LP21b.pdf_tex} \\
\caption{Intenziteta in smer električnega polja v vlaknu za približno linearne rodove
LP$_{01}$, LP$_{02}$, LP$_{11}$ in LP$_{21}$}
\label{fig:LP}
\end{figure}

\subsection*{Cilindrično vlakno s paraboličnim profilom lomnega količnika}
Čeprav je izračun lastnih načinov v cilindričnem vlaknu zapleten, lahko 
razmeroma enostavno poiščemo rešitve za vlakno, v katerem je dielektrična 
konstanta kvadratna funkcija radialne koordinate $r$\index{Optično vlakno!parabolični profil lomnega količnika}. 
Zapišemo enačbo z lomnimi količniki
in vpeljemo brezdimenzijski parameter $\Delta$
\begin{equation}
n^2\left(r<a\right)=n_{1}^{2}- \Delta^2 \frac{r^2}{a^2},
\label{9.15}
\end{equation}
pri čemer $a$ označuje polmer sredice.
Enačbo lahko razvijemo za majhno razliko $\Delta$ in za vse smiselne vrednosti $r$
ima tudi lomni količnik parabolični profil. Parabolična
sredica je seveda omejena, okoli nje je plašč s konstantnim
lomnim količnikom $n_2 \approx n_1-\Delta^2/2n_1$ (slika~\ref{fig:GRIN}). 
Tipičen polmer sredice $a$ je nekaj deset mikrometrov in plašča približno petkrat toliko.
\begin{figure}[ht]
\centering
\def\svgwidth{90truemm} 
\input{slike/10_GRIN.pdf_tex} 
\caption{Parabolični profil lomnega količnika sredice zmanjša disperzijo v vlaknu. Plašč
 je praviloma bistveno debelejši od sredice vlakna.}
\label{fig:GRIN}
\end{figure}

Jakost električnega polja za izbrano polarizacijo zapišemo v obliki 
\begin{equation}
E=E_{0}\psi(x,y)\, e^{i\beta z-i\omega t}.
\label{9.16}
\end{equation}
Pri tem smo zanemarili, da zaradi odvisnosti od prečnih koordinat in pogoja $\nabla\cdot{\bf D}=0$
polje ne more imeti povsod iste smeri; za bolj natančen račun bi morali zapisati enačbo za
vektorsko polje. Vstavimo približni
nastavek~(enačba~\ref{9.16}) in krajevno odvisnost lomnega količnika~(enačba~\ref{9.15})
v valovno enačbo (enačba~\ref{eq:valovna-skalarna}) 
\begin{equation}
\nabla_{\perp}^{2}\psi+\left(k_{0}^{2}\left(n_{1}^{2}-\Delta^{2}\frac{r^{2}}{a^2}\right)-
\beta^{2}\right)\,\psi=0.
\label{9.17}
\end{equation}
Rešitve lahko zapišemo v obliki
\begin{equation}
\psi(x,y) = X(x)Y(y),
\end{equation}
od koder sledita dve neodvisni enačbi
\begin{equation}
X'' - \frac{k_0^2 \Delta^2}{a^2}\,X\,x^2 - \lambda_1 X = 0 \qquad \mathrm{in} \qquad
Y'' - \frac{k_0^2 \Delta^2}{a^2}\,Y\,y^2 - \lambda_2 Y = 0,
\label{eq:XY}
\end{equation}
pri čemer sta $\lambda_1$ in $\lambda_2$ konstanti. 
Opazimo, da sta enačbi popolnoma enaki enačbama za krajevni del lastnih funkcij 
harmonskega oscilatorja v kvantni mehaniki.
Rešitev posamezne enačbe je tako 
produkt Gaussove in Hermitove funkcije
\begin{equation}
X_n(x) = e^{-\xi^2 x^2/2} H_n(\xi x),
\label{eq:GH}
\end{equation}
pri čemer je $\xi = \sqrt{k_0 \Delta/a}$.
\begin{naloga}
Uporabi nastavek (enačba~\ref{eq:GH}) in pokaži, da reši enačbo~(\ref{eq:XY}). Pri tem si pomagaj z 
diferencialno enačbo za Hermitove polinome
\begin{equation}
\left( \frac{d^2}{dx^2}-2x\frac{d}{dx}+2n \right) H_n(x) = 0.
\end{equation}
\end{naloga}
Lastne vrednosti enačbe (\ref{9.17}) so oblike
\begin{equation}
\beta_{mn}^{2}=n_{1}^{2}k_{0}^{2}\left(1-\frac{2\Delta}{k_{0}n_{1}^2a}\left(m+n+1\right)\right).
\label{9.19}
\end{equation}
Drugi člen v oklepaju je navadno zelo majhen, zato lahko izraz razvijemo in 
\begin{equation}
\beta_{mn}=n_{1}k_{0}\left(1-\frac{\Delta}{k_{0}n_{1}^2 a}\left(m+n+1\right)\right)
= n_{1}k_{0} - \frac{\Delta \left(m+n+1\right)}{n_{1} a}.
\end{equation}
Ob predpostavki, da je parameter $\Delta$ neodvisen od krožne frekvence, je grupna 
hitrost\index{Hitrost valovanja!grupna}
\begin{equation}
v_{g}=\left(\frac{d\beta_{mn}}{d\omega}\right)^{-1}=\frac{c_{0}}{n_{1}}
\label{9.21}
\end{equation}
in torej enaka za vse rodove. To je pomembna značilnost vlakna s paraboličnim profilom
lomnega količnika. V dejanskem vlaknu je seveda taka odvisnost mogoča
le v omejenem območju sredice, zato je tudi opisani pristop le približen
in velja dobro za tiste rodove, ki se ne raztezajo dosti izven sredice.

Neodvisnost grupne hitrosti od roda je praktično zelo pomembna. 
Grupna hitrost namreč določa čas potovanja svetlobnega sunka, ki
lahko predstavlja en bit informacije. Če se po vlaknu širi več
rodov z različnimi grupnimi hitrostmi, se sunek po prehodu skozi
vlakno podaljša, kar -- kot bomo podrobneje videli v naslednjem razdelku -- omejuje 
uporabno dolžino vlakna. Temu se sicer lahko izognemo z uporabo enorodovnih vlaken,
ki pa so dražja, poleg tega morata divergenca in polmer svetlobnega snopa 
natančno ustrezati značilnostim enorodovnega vlakna, da se izognemo izgubam. 
Zato se za krajše povezave (do nekaj $100~\si{m}$) uporabljajo večrodovna vlakna, ki imajo sredico s 
približno paraboličnim profilom lomnega količnika.

\section{Disperzija}
\label{chap:Disperzija}
Pri prenosu velike količine podatkov na daljavo je zelo pomembno, da
se oblika svetlobnih sunkov, ki prenašajo informacijo, čim bolj ohranja.
Na obliko sunka močno vpliva disperzija, to je odvisnost fazne in grupne hitrosti
valovanja od krožne frekvence. Zaradi disperzije se kratki sunki, ki potujejo po vlaknu, podaljšajo in 
tako omejujejo količino informacije, ki jo lahko prenašamo po vlaknu dane dolžine\index{Disperzija}
(slika~\ref{fig:disp}).
Največja količina vhodnih podatkov na časovno enoto je kar 
obratno sorazmerna z dolžino izhodnih sunkov svetlobe. 
\begin{figure}[ht]
\centering
\def\svgwidth{120truemm} 
\input{slike/10_disperzija.pdf_tex} 
\caption{Zaradi disperzije se sunki svetlobe, ki potujejo skozi vlakno, 
podaljšajo. Na izhodu iz vlakna jih zato ne zaznamo več ločeno.}
\label{fig:disp}
\end{figure}

Pri potovanju svetlobe po optičnih vlaknih poznamo tri 
vrste disperzije: rodovno, materialno in valovodno. V večrodovnih vlaknih
je povsem prevladujoča rodovna disperzija, ki je posledica dejstva, da se 
različni rodovi po vlaknu širijo z različnimi hitrostmi. 
V enorodovnih vlaknih rodovne disperzije ni, zato prideta do izraza 
materialna disperzija, ki se pojavi zaradi odvisnosti lomnega količnika 
vlakna od valovne dolžine svetlobe, in valovodna disperzija, ki se pojavi zaradi 
nelinearne zveze med valovnim številom $\beta$ in krožno frekvenco valovanja. 

\subsection*{Rodovna disperzija}
\index{Disperzija!rodovna}
\index{Optično vlakno!večrodovno}
Na primeru planparalelnega vodnika smo pokazali, da vsaki rešitvi sekularnih 
enačb~(enačbi~\ref{sekular1} in \ref{sekular2}) ustreza en lastni rod v vlaknu.
Ker se vrednosti $k_x$ za različne rodove med seboj razlikujejo in so posledično vrednosti $\beta$ za 
vsak rod drugačne, se posamezni rodovi po vlaknu širijo z različnimi hitrostmi. Kratek
sunek svetlobe, sestavljen iz več različnih rodov, se tako po prehodu skozi vlakno
razdeli na posamezne delne sunke oziroma se efektivno podaljša. 
Izračunajmo razliko med časom, ki ga za širjenje po vlaknu dane dolžine 
potrebuje osnovni rod, in časom, ki ga za isto razdaljo potrebuje zadnji še dovoljeni rod.
 
Osnovnemu rodu ustreza prva rešitev sekularne enačbe (enačba~\ref{sekular1}) in je zato pripadajoča 
vrednost $k_x$ zelo majhna. V prvem približku je $k_x \approx 0$. Ustrezno valovno 
število je po enačbi
\begin{equation}
\beta = \sqrt{\left( \frac{\omega}{c_0}\right)^2n_1^2 - k_x(\omega)^2}
\label{nelinfib}
\end{equation}
kar približno enako $\beta_0 \approx k_0 n_1$. 
Za zadnji še dovoljeni rod velja $k_xa \approx V$ in $\beta_N \approx k_0 n_2$, pri čemer je
$n_2$ lomni količnik plašča. Zapišemo še grupno hitrost, s katero potujejo sunki 
svetlobe po vlaknu
\begin{equation}
v_{g}=\frac{d\omega}{d\beta}=\left(\frac{d\beta}{d\omega}\right)^{-1}.
\label{9.51}
\end{equation}
Za prehod vlakna z dolžino $L$ potrebuje osnovni rod čas
\beq
t_0 = \frac{L}{v_{g0}} =  L \,\frac{d\beta_0}{d\omega} = L\,\frac{n_1}{c_0},
\eeq
zadnji rod pa čas
\beq
t_N = \frac{L}{v_{gN}} = L\, \frac{d\beta_N}{d\omega} = L\,\frac{n_2}{c_0}.
\eeq
Podaljšanje sunka zaradi rodovne disperzije je potem 
\boxeq{DispRod}{
\tau\approx 
\frac{L}{c_0} (n_1-n_2).
}
Največja frekvenca modulacije vhodnega signala, pri kateri izhodne sunke še zaznamo ločeno,
je približno obratna vrednost dolžine izhodnih sunkov. Za $1~\si{\kilo\meter}$ dolgo vlakno z 
$\Delta n = 0,05$  je tako modulacijska frekvenca oziroma količina podatkov v časovni
enoti manj od $10~\si{\mega\hertz}$. Čeprav lahko disperzijo zmanjšamo 
s paraboličnim profilom lomnega količnika, so večrodovna vlakna za prenos podatkov na dolge 
razdalje praktično neuporabna.
\begin{remark}
 Med rodovno disperzijo uvrščamo tudi polarizacijsko disperzijo.\index{Disperzija!polarizacijska} 
 V idealnem cilindričnem vlaknu potujeta obe polarizaciji
 z enako hitrostjo. V realnem vlaknu pa imata valovanji z različnima polarizacijama zaradi
 nečistoč in asimetrij v vlaknu različni hitrosti. 
 Ker so nečistoče slučajno in neodvisno razporejene, tako disperzijo zelo težko odpravimo.
\end{remark}

\subsection*{Materialna disperzija}
\index{Disperzija!materialna}
Optična vlakna so navadno iz stekla, katerega lomni količnik je odvisen 
od valovne dolžine svetlobe. Zaradi tako imenovane materialne ali snovne disperzije različne 
spektralne komponente svetlobnega sunka po vlaknu potujejo z različnimi hitrostmi 
in sunek se po prehodu skozi vlakno podaljša. Pri obravnavi\index{Disperzija!snovna|see{Disperzija, materialna}}
se omejimo na enorodovna vlakna, v katerih rodovne disperzije ni. 

Ker je sunek svetlobe končno dolg, je končna tudi njegova spektralna širina 
$\Delta \omega$. 
Dolžino sunka $\tau$ po prehodu skozi vlakno dolžine $L$ približno zapišemo kot
\begin{equation}
\tau = \frac{dt}{d\omega}\Delta \omega = 
\Delta \omega \frac{d}{d\omega}\left(\frac{L}{v_g}\right)
= \Delta \omega \frac{d}{d\omega}\left(L \frac{d\beta}{d\omega}\right) =
\Delta \omega\, L\, \frac{d^2 \beta}{d \omega^2}.
\label{dispračun}
\end{equation}
Za enorodovno vlakno velja $k_x \approx 0$ in $\beta \approx n_1 \omega/c_0$ in zapišemo  
\begin{equation}
\tau = \Delta \omega\, \frac{L}{c_0}\,\frac{d^2 (n_1 \omega)}{d \omega^2}.
\end{equation}
Na področju telekomunikacij se podaljšanje sunka navadno zapiše v obliki
\boxeq{eq:dmat1}{
\tau= |D_m| L\, \Delta \lambda.
}
Pri tem smo vpeljali $D_m$ kot koeficient materialne disperzije
\begin{equation}
D_m = - \frac{2\pi}{\lambda^2}\frac{d^2(n_1 \omega)}{d\omega^2},
\end{equation}
ki ga navadno izrazimo v enotah $\si{\pico\second/\nano\meter\, \kilo\meter}$.
Njegova vrednost je lahko pozitivna ali negativna, zato smo v končnem izrazu za dolžino 
sunka dodali absolutno vrednost. V snoveh, ki jih uporabljamo za izdelavo optičnih vlaken, 
je $D_m \sim 10~\si{\pico\second/\nano\meter\, \kilo\meter}$, lahko pa seže
tudi do več $100~\si{\pico\second/\nano\meter\, \kilo\meter}$, odvisno seveda od valovne dolžine
in izbrane snovi.\footnote{~Glej npr. C. R. Pollock, {\it Fundamentals of Optoelectronics}, 
Irwin (1995).}

Materialno disperzijo lahko zmanjšamo na več načinov. Lahko uporabimo čim bolj enobarven
vir svetlobe, da zmanjšamo $\Delta \omega$. Za izbrano snov lahko celo izberemo valovno 
dolžino, pri kateri je materialna disperzija čim manjša in praktično enak nič. Za 
SiO$_2$\index{SiO$_2$} je to pri okoli $1300$--$1500~\si{\nano\meter}$, odvisno 
od primesi. Še najbolj uporabna je rešitev, pri kateri 
z materialno disperzijo izničimo vpliv valovodne disperzije in na ta način zmanjšamo skupno 
disperzijo v vlaknu.

\subsection*{Valovodna disperzija}
\index{Disperzija!valovodna}
V optičnem vodniku velja nelinearna zveza med prečno ($k_x$) in vzdolžno ($\beta$)
komponento valovnega vektorja (enačba~\ref{nelinfib}). Vrednost $k_x$ izračunamo numerično, 
pri čemer je rešitev odvisna od frekvence svetlobe. Tudi v cilindričnih vlaknih je valovno število 
$\beta$ nelinearna funkcija $\omega$, zato se pojavi disperzija. Izhajamo iz zveze 
(enačba~\ref{dispračun})
\begin{equation}
\tau = \Delta \omega\, L\, \frac{d^2 \beta}{d \omega^2}.
\end{equation}
Pogosto vpeljemo koeficient valovodne disperzije $D_v$
\begin{equation}
D_v = -\frac{2\pi c_0}{\lambda^2}\frac{d^2 \beta}{d\omega^2}
\end{equation}
in lahko zapišemo
\boxeq{dispVal}{
\tau = |D_v|\,L\, \Delta \lambda.
}
Koeficient valovodne disperzije je praviloma najmanjši, 
$D_v \sim 1$--$10~\si{\pico\second/\nano\meter\,\kilo\meter}$. 
Znaten postane v enorodovnih vlaknih v območju, kjer je materialna disperzija 
zelo majhna ali celo enaka nič. 
V vlaknih s homogeno sredico se valovodni disperziji ne moremo
izogniti, vendar jo lahko pri dani valovni dolžini približno izničimo z materialno~(slika~\ref{fig:MatVal}).\footnote{Slika povzeta po C. R. Pollock, {\it 
Fundamentals of Optoelectronics}, Irwin (1995).}

Ker ima sunek svetlobe vedno neko končno spektralno širino, disperzije 
v optičnem vlaknu nikoli ne moremo povsem odpraviti. Pri celotni disperziji 
$5~\si{\pico\second/\nano\meter\,\kilo\meter}$ in spektralni širini 
$1~\si{\nano\meter}$ znaša v $100~\si{\kilo\meter}$ dolgem vlaknu najvišja 
frekvenca modulacije vhodnega signala, ki ga na izhodu še lahko razločimo, 
okoli $2~\si{\giga\hertz}$. V nadaljevanju bomo videli, da je pri prenosu podatkov
v vlaknih poglavitni omejujoči faktor ravno disperzija in ne absorpcija. 

\begin{figure}[ht]
\centering
\def\svgwidth{90truemm} 
\input{slike/10_Zero.pdf_tex} 
\caption{Odvisnost koeficientov disperzije od valovne dolžine v SiO$_2$ vlaknu.\index{SiO$_2$} $D_m$ 
je koeficient materialne disperzije, $D_v$ valovodne, $D$ pa je vsota obeh. Pri valovni dolžini
okoli $1450~\si{\nano\meter}$ se materialna in valovodna disperzija odštejeta in skupna disperzija
je praktično enaka nič.}
\label{fig:MatVal}
\end{figure}
% \begin{remark}
% Na valovodno disperzijo je mogoče vplivati tudi s konstrukcijo vlakna. Pokazali smo že, da
% v idealnem primeru v vlaknu s paraboličnim profilom lomnega količnika disperzije ni. 
% V praksi je sredica sestavljena iz več plasti z različnimi lomnimi količniki in različnimi
% debelinami, s čimer se prispevek valovodne disperzije spremeni. Na ta način lahko 
% položaj ničle celotne disperzije premaknemo k valovni dolžini izvora oziroma k 
% valovni dolžini, pri kateri je v vlaknu najmanj absorpcije in izgub.
% \end{remark}

\section{*Potovanje kratkega sunka po enorodovnem vlaknu}
\label{chap:sunvl}
\subsection*{Podaljšanje sunka zaradi disperzije}
\index{Optično vlakno!enorodovno}
Poglejmo si podrobneje, kako po enorodovnem vlaknu ali drugem
sredstvu z disperzijo potuje kratek sunek valovanja z dano začetno obliko.
Sunek zapišemo kot  
\begin{equation}
E\left(x, y, z, t\right)=\psi\left(x,y\right)\, a\left(z,t\right),
\label{9.61}
\end{equation}
kjer je $\psi\left(x,y\right)$ lastna rešitev prečnega dela valovne
enačbe, ki določa zvezo $\beta\left(\omega\right)$. 
Funkcija $a\left(z,t\right)$ opisuje obliko in potovanje sunka v smeri $z$. 
Pri $z=0$ jo razvijemo po krožnih frekvencah z ustreznimi amplitudnimi faktorji 
\begin{equation}
a\left(0,t\right)=\int \tilde{A}(\omega)\, e^{- i\omega t}d\omega.
\label{9.62}
\end{equation}
Ko sunek potuje vzdolž osi $z$, vsaki komponenti pripišemo
ustrezen fazni faktor $i \beta (\omega) z$. Tako velja
\begin{equation}
a\left(z,t\right)=\int \tilde{A}(\omega)\, e^{i \beta (\omega) z - i\omega t}d\omega.
\label{9.62f}
\end{equation}
Osnovni sunek naj bo približno monokromatičen s krožno frekvenco $\omega_{0}$. Ta naj bo dovolj 
velika, da lahko privzamemo, da je sunek mnogo daljši od optične periode.

Razvijmo $\beta(\omega)$
okoli $\omega_{0}$, pri čemer vpeljemo razliko krožnih frekvenc $\Omega = \omega - \omega_0$
\begin{equation}
\beta(\omega_0 + \Omega) \approx \beta(\omega_{0})
+\frac{d\beta}{d\omega}\,\Omega+\frac{1}{2}\,\frac{d^{2}\beta}{d\omega^{2}}\,\Omega^{2} = 
\beta(\omega_{0}) +\beta '\,\Omega+\frac{1}{2}\,\beta ''\,\Omega^{2}.
\label{9.62c}
\end{equation}
S črtico smo označili odvod po $\omega$. Enačbo~(\ref{9.62f}) prepišemo v 
\begin{equation}
a\left(z,t\right)=\int \tilde{A}(\Omega)\, e^{i \beta (\omega_0 + \Omega)z - 
i(\omega_0 + \Omega) t}d\Omega  =  e^{i \beta (\omega_0)z - i\omega_0 t} A(z,t).
\label{9.62b}
\end{equation}
Funkcija $A(z, t)$ predstavlja prostorsko in časovno odvisnost ovojnice sunka. Z upoštevanjem
razvoja (enačba~\ref{9.62c}) jo zapišemo kot 
\begin{equation}
 A(z,t) = \int \tilde{A}(\Omega)\, \exp \left(i \beta'\, \Omega\,z + 
 \frac{i}{2}\beta''\,\Omega^2\, z - i\, \Omega\, t\right) d\Omega.
 \label{eq:ovojnica967}
\end{equation}
Odvajajmo ovojnico najprej parcialno po $z$
\begin{equation}
 \frac{\partial A(z,t)}{\partial z} = \int \tilde{A}(\Omega)\, 
 \left(i \beta'\, \Omega + \frac{i}{2}\beta''\Omega^2 \right) 
 \exp \left(i \beta'\, \Omega\,z + 
 \frac{i}{2}\beta''\,\Omega^2\, z - i\, \Omega\, t\right) d\Omega,
\label{9.67a}
\end{equation}
nato pa še enkrat 
\begin{equation}
 \frac{\partial A(z,t)}{\partial t} = \int \tilde{A}(\Omega)\, 
 \left(-i\Omega\right) 
 \exp \left(i \beta'\, \Omega\,z + 
 \frac{i}{2}\beta''\,\Omega^2\, z - i\, \Omega\, t\right) d\Omega
\label{9.67b}
\end{equation}
in dvakrat parcialno po $t$
\begin{equation}
 \frac{\partial^2 A(z,t)}{\partial t^2} = \int \tilde{A}(\Omega)\, 
 \left(-\Omega^2\right) 
 \exp \left(i \beta'\, \Omega\,z + 
 \frac{i}{2}\beta''\,\Omega^2\, z - i\, \Omega\, t\right) d\Omega.
\label{9.67c}
\end{equation}
Primerjamo izračunane odvode in dobimo enačbo
\begin{equation}
 \frac{\partial A(z,t)}{\partial z} = -\beta'\frac{\partial A}{\partial t} - \frac{i}{2} \beta''\frac{\partial^2 A}{\partial t^2}.
 \label{9.68b}
\end{equation}
Enačbo lahko nekoliko poenostavimo z vpeljavo novega para neodvisnih spremenljivk
\begin{equation}
\tau  =  t-\beta'z\nonumber \qquad \mathrm{in} \qquad \zeta = z.
\label{9.70}
\end{equation}
Uporabimo verižno pravilo odvajanja
\begin{equation}
 \frac{\partial}{\partial z}= \frac{\partial}{\partial\tau}\frac{\partial\tau}{\partial z}+ 
 \frac{\partial}{\partial\zeta}\frac{\partial\zeta}{\partial z} = 
 \frac{\partial}{\partial\tau}\left(-\beta'\right)+ \frac{\partial}{\partial \zeta}
\end{equation}
in 
\begin{equation}
 \frac{\partial}{\partial t}= \frac{\partial}{\partial\tau}\frac{\partial\tau}{\partial t}+ 
 \frac{\partial}{\partial\zeta}\frac{\partial\zeta}{\partial t} = 
 \frac{\partial}{\partial\tau}.
\end{equation}
Z novima spremenljivkama se enačba~(\ref{9.68b}) prepiše v 
\begin{equation}
\beta''\frac{\partial^{2}A}{\partial\tau^{2}}-
2\, i\,\frac{\partial A}{\partial\zeta}=0.
\label{9.71}
\end{equation}
Poglejmo enačbo podrobneje. Če ni disperzije in je $\beta''=0$, se $A$
vzdolž koordinate $\zeta$ ne spreminja. To pomeni, da se oblika sunka
ob odsotnosti disperzije ohranja in sunek poljubne začetne oblike nepopačen 
potuje po vlaknu z grupno hitrostjo $1/\beta'$.\index{Hitrost valovanja!grupna}

Če je disperzija različna od nič, ostaneta v enačbi oba člena. Opazimo, da
ima enačba enako obliko kot obosna valovna enačba, ki smo jo v
tretjem poglavju uporabili za obravnavo koherentnih 
snopov (enačba~\ref{eq:obosna-valovna-enacba}). 
Razlika med obosno valovno enačbo in enačbo~(\ref{9.71}) je v tem, da vlogo
prečne koordinate prevzame čas $\tau$. To, kar je bila prej širina snopa, 
je zdaj dolžina sunka. Spomnimo se, da 
obosno valovno enačbo rešijo Gaussovi snopi (enačba~\ref{eq:gaussov-snop}). 

\begin{remark}
Podobnost med pojavoma seže dlje od formalne oblike. Pri snopih, ki so omejeni 
v prečni smeri, disperzija fazne in grupne hitrosti po prečnih komponentah valovnega
vektorja povzroča spreminjanje prečnega preseka snopa. Pri časovno
omejenih sunkih v sredstvu s frekvenčno disperzijo se namesto preseka sunka
spreminja njegova vzdolžna oblika oziroma njegova dolžina.
\end{remark}

Celotnega računa
ni treba ponavljati, namesto tega kar v izrazu za Gaussove snope 
(enačba~\ref{eq:gaussov-snop}) napravimo ustrezno zamenjavo spremenljivk. 
Iz enačbe~(\ref{9.71}) razberemo, da valovnemu številu $k$ pri snopih  
ustreza parameter $\mu=(d^{2}\beta/d\omega^{2})^{-1}$. Poleg tega vpeljemo
dolžino sunka $2\sigma$, ki ustreza premeru Gaussovega snopa $2w$, in parameter
$b$, ki ustreza krivinskemu radiju $R$. Oba parametra sta seveda odvisna od $\zeta$, 
tako kot sta parametra $w$ in $R$ odvisna od $z$ (slika~\ref{fig:Gausstau}). 

Na podlagi analogije zapišemo obliko podaljšanega Gaussovega sunka\index{Gaussov sunek}
\begin{equation}
A\left(\zeta,\tau\right)=\frac{A_{0}}{\sqrt{1+\frac{\zeta^{2}
}{\zeta_{0}^{2}}}}\exp\left(-\frac{\tau^{2}}{\sigma^{2}}\right)\exp
\left(-i\frac{\mu\tau^{2}}{2b}\right)e^{i\phi\left(\zeta\right)}.
\label{9.72}
\end{equation}
Za $\sigma$ velja enaka zveza kot za polmer 
Gaussovega snopa (enačba~\ref{eq:w})
\boxeq{9.73}{
\sigma^{2}=\sigma_{0}^{2}\left(1+\left(\frac{\zeta}{\zeta_{0}}\right)^{2}\right).
}
Pri tem je $2\sigma_{0}$ trajanje sunka pri $\zeta=0$, to je na začetku,
kjer je sunek najkrajši. Krivinskemu radiju valovnih front (enačba~\ref{eq:R}) v tem primeru
ustreza količina $b=\zeta\left(1+\zeta_{0}^{2}/\zeta^{2}\right)$.
Po analogiji s snopi lahko sklepamo, da se najmanj 
podaljšuje ravno sunek z Gaussovo časovno odvisnostjo. 
\begin{figure}[ht]
\centering
\def\svgwidth{120truemm} 
\input{slike/10_Gausstau.pdf_tex}
\caption{Primerjava širitve Gaussovega snopa (a) in podaljšanja Gaussovega sunka (b).
Dolžina sunka $2\sigma$ med potovanjem po vlaknu narašča z enako odvisnostjo kot 
narašča premer Gaussovega snopa $2w$ z oddaljenostjo od grla. Znotraj $\zeta_0$ se sunek
še ne podaljša znatno.}
\label{fig:Gausstau}
\end{figure}

Zanimivo je pogledati odvod faze po $\tau$, ki predstavlja spremembo krožne
frekvence glede na centralno krožno frekvenco sunka $\omega_{0}$
\begin{equation}
\omega-\omega_{0}=\frac{\mu\tau}{b}.
\label{9.74}
\end{equation}
Za pozitivne vrednosti $\mu$ je krožna frekvenca na začetku sunka,
to je pri $\tau<0$, manjša od $\omega_0$, z naraščajočim časom pa se 
linearno povečuje proti koncu sunka. Obnašanje je zelo podobno čirikanju, 
ki ga bomo spoznali pri obravnavi nelinearnih optičnih pojavov (slika~\ref{fig:chirp}\,a). \index{Čirikanje}
\vglue-5truemm
\begin{remark}
Pri $\zeta=0$ je sunek najkrajši možen pri dani spektralni
širini. Lahko si mislimo, da je sunek najkrajši,
to je omejen s Fourierevo transformacijo spektra, kadar se
vse frekvenčne komponente seštejejo z isto fazo, to je pri $\zeta=0$.
Da nastanejo najkrajši sunki, kadar je faza vseh delnih valov enaka,
smo spoznali že pri fazno uklenjenih sunkih iz večfrekvenčnih laserjev
(razdelek~\ref{chap:Uklepanje}).
Pri potovanju sunka se zaradi disperzije faze frekvenčnih komponent
različno spreminjajo in sunek se podaljša. Pri tem je pomemben  
drugi odvod valovnega števila po krožni frekvenci. Linearno spreminjanje faze 
namreč ne povzroči razširitve, temveč le spremembo v hitrosti.
\end{remark}

\begin{naloga}
\label{naloga:pulzdisperzija}
Naj bo vpadni sunek svetlobe Gaussove oblike $E(x,y, z=0, t) = 
\psi(x,y) e^{-at^2-i \omega_0 t}$. Pokaži, da je ustrezna funkcija
$\tilde{A}(\Omega)$ oblike
\begin{equation}
\tilde{A}(\Omega) = \frac{1}{\sqrt{4 \pi a}}e^{-\Omega^2/4a}.
\end{equation}
Nato vpelji novi spremenljivki $\tau$ in $\zeta$ in z neposredno 
integracijo (enačba~\ref{eq:ovojnica967}) pokaži, 
da je podaljšan sunek pri $z\neq 0$ enak ovojnici, 
zapisani z enačbo~(\ref{9.72}), pri čemer je $\zeta_0 = \mu/2 a$.
\end{naloga}

\begin{naloga}
Uporabi enačbo~(\ref{9.73}) in pokaži, da je podaljšanje Gaussovega 
sunka svetlobe oblike $I \propto \exp(-2\tau^2/\sigma^2)$
pri dani dolžini vlakna enako
\begin{equation}
\sigma (L) = \sigma_0\sqrt{1 + \left(\frac{2 L }{\sigma_0^2 \mu}\right)^2}
\end{equation}
in za velike dolžine enako izrazu, ki smo ga izračunali pri 
valovodni disperziji (enačba~\ref{dispVal}).
\end{naloga}

\subsection*{Kompenzacija disperzije}
\index{Disperzija!kompenzacija}
\label{kompdisp}
Razširitev sunka zaradi pozitivne disperzije je pri $\mu > 0$ mogoče kompenzirati
s parom vzporednih uklonskih mrežic.\index{Uklonska 
mrežica}\footnote{~E. B. Treacy, IEEE J. Quantum. Electron. $\mathbf{5}$, 454 (1969).}
Prva mrežica različne frekvenčne komponente razkloni in druga ponovno
zbere, vendar so pri tem optične poti za različne komponente različno 
dolge (slika~\ref{fig:comp}).
Vzporednost uklonskih mrežic zagotavlja vzporednost izhodnih žarkov,
vendar so različne komponente vpadne svetlobe med seboj razmaknjene (slika~\ref{fig:comp}\,a).
V praksi zato uporabimo ali dva para uklonskih mrežic ali 
zrcalo, ki svetlobo usmeri po isti poti 
nazaj.
\begin{figure}[ht]
\centering
\def\svgwidth{115truemm} 
\input{slike/10_comp.pdf_tex}
\caption{Kompenzacija disperzije z uklonskima mrežicama (a) in shema z
oznakami za izpeljavo faznega premika (b)}
\label{fig:comp}
\end{figure}

Naj na par vzporednih uklonskih mrežic vpada ravni val pod kotom $\alpha$, ki se odbija
pod kotom $\beta = \beta(\omega)$ (slika~\ref{fig:comp}\,b). 
Pot, ki jo prepotuje žarek od vpada na mrežico 
do izhoda iz sistema (med točkama $A$ in $B$), je enaka 
\begin{equation}
P = \frac{L}{\cos\beta} \left(1+\cos(\alpha + \beta)\right).
\end{equation}
Zaradi uklona velja zveza $\sin\,\alpha - \sin\,\beta = \lambda/\Lambda$,
pri čemer je $\lambda$ valovna dolžina svetlobe in $\Lambda$ perioda uklonske mrežice. 
Pri fazi moramo upoštevati še fazni premik na drugi mrežici
\begin{equation}
\Phi_m=\frac{2\pi}{\Lambda} \, L \, \tan\,\beta = q L \,\tan\,\beta.
\end{equation}
Celotna sprememba faze je tako $\Phi = \omega P/c + \Phi_m$.

\begin{naloga}
\label{nal:dispk}
Pokaži, da je drugi odvod faze po krožni frekvenci enak
\begin{equation}
\frac{d^2 \Phi}{d \omega^2} = - \frac{L\, c\, q^2}
{\left(\omega^2 - (\omega\, \sin\alpha - cq)^2\right)^{3/2}}.
\label{eq:10faza}
\end{equation}
\end{naloga}
Račun v nalogi (\ref{nal:dispk})
pokaže, da je disperzija, ki je določena z drugim odvodom faze po krožni frekvenci
(enačba~\ref{eq:10faza}), vedno negativna. Par vzporednih uklonskih mrežic
torej deluje kot sredstvo
z negativno disperzijo in sunek, ki se je razširil zaradi potovanja
po sredstvu s pozitivno disperzijo, lahko ponovno skrajša do meje,
določene s širino spektra. 

\begin{remark}
Postopek kompenzacije disperzije se uporablja za pridobivanje zelo močnih zelo
kratkih sunkov svetlobe.\footnote{~D. Strickland in G. Mourou, Opt. Comm. $\mathbf{56}$, 219 (1985).} 
Sunku iz fazno uklenjenega barvilnega\index{Laser!organska barvila} 
ali Ti:safirnega\index{Laser!Ti:safir}\index{Uklepanje faz}
laserja se najprej v nelinearnem sredstvu razširi spekter, hkrati
se sunek tudi časovno podaljša. Podaljšan sunek lahko ojačimo, česar 
s prvotnim kratkim in razmeroma močnim sunkom ne bi mogli narediti. Razširjen
in ojačen sunek nato s parom uklonskih mrežic skrajšamo za 
faktor $10$--$100$ glede na prvotno dolžino sunka. Tako nastanejo zelo močni sunki
svetlobe, dolgi le okoli $10~\si{\femto\second}$, kar je le še nekaj optičnih 
period.\footnote{~Za to odkritje sta leta 2018 Donna Strickland in
G\'erard Mourou prejela Nobelovo nagrado.}
\end{remark}

\section{Izgube in ojačenje v optičnih vlaknih}
\index{Izgube v optičnih vlaknih}
Pri prenosu informacij z optičnimi vlakni je poleg disperzije, ki signal popači,
treba upoštevati tudi izgube, ki signal oslabijo. 
Izgube so posledica predvsem absorpcije svetlobe v vlaknu,\index{Absorpcija}
Rayleighovega\index{Rayleighovo sipanje} sipanja na fluktuacijah gostote, 
sipanja na nečistočah in upognjenosti vlakna. Do izgub prihaja tudi na stikih 
med vlakni. Za prenos na dolge
razdalje je tako potreben razmeroma močen signal, a ne premočen,
saj lahko v vlaknu pride do nelinearnih optičnih pojavov. V praksi zato 
optični signal, ki potuje po čezoceanskih vlaknih, ojačujemo in s tem nadomestimo
izgube.

Za merilo izgub v vlaknu vpeljemo atenuacijski 
koeficient\index{Atenuacijski koeficient}, merjen v decibelih ali dB/km
\boxeq{dB}{
A [dB] = -10 \log_{10}\frac{j(z)}{j(0)},
}
pri čemer je $j(z)$ gostota svetlobnega toka po prepotovani razdalji $z$ in $j(0)$
vpadna gostota svetlobnega toka. Če se po prepotovanem
kilometru signal zmanjša na polovico, so izgube $3~\si{\decibel/\kilo\meter}$.

Pri izdelavi optičnih vlaken se najpogosteje uporablja kremenovo steklo, ki 
ima pri valovni dolžini $1,55~\si{\micro\meter}$ izgube okoli 
$0,2~\si{\decibel/\kilo\meter}$.  
Za primerjavo: navadno steklo ima pri vidni svetlobi atenuacijski koeficient okoli 
$1000~\si{\decibel/\kilo\meter}$.

\begin{figure}[ht]
\centering
\def\svgwidth{90truemm} 
\input{slike/10_FibAbs.pdf_tex} 
\caption{Izgube v vlaknu v odvisnosti od valovne dolžine: vijolična črta -- UV absorpcija, 
črna črta -- IR absorpcija, zelena črta -- absorpcija na OH ionih in modra črta --
izgube zaradi Rayleighovega sipanja. Z rdečo črto so označene skupne izgube.}
\label{FibAbs}
\end{figure}
Slika~\ref{FibAbs} prikazuje odvisnost izgub od valovne dolžine 
za dobro enorodovno vlakno.\footnote{~Slika povzeta po
A. Yariv in P. Yeh, {\it Photonics}, šesta izdaja, Oxford
University Press (2007).}
\index{Optično vlakno!enorodovno}
Celotne izgube (rdeča črta)
so sestavljene iz vrste različnih prispevkov. 
Pri kratkih valovnih dolžinah je absorpcija velika zaradi elektronskih prehodov
v steklu (vijolična črta).\index{Ultravijolično valovanje} 
Širina reže za SiO$_2$\index{SiO$_2$} je namreč okoli $8,9$~eV, 
kar ustreza valovni dolžini\index{Infrardeče valovanje}
okoli $140~\si{\nano\meter}$. Pri velikih valovnih dolžinah je absorpcija posledica
vibracijskih prehodov (črna črta). Čeprav so ti prehodi pri nižjih frekvencah, 
so vrhovi zelo široki in sežejo do okoli $1500~\si{\nano\meter}$. 
Absorpcija na nečistočah lahko ob pazljivi izdelavi postane skoraj v celotnem 
območju praktično zanemarljiva. 
Najbolj problematična nečistoča je voda oziroma OH$^{-}$ ioni, ki imajo velik dipolni
moment in izrazito absorpcijo pri $1380~\si{\nano\meter}$ (zelena črta). Zelo pomemben prispevek k 
izgubam, posebej pri krajših valovnih dolžinah, je Rayleighovo sipanje na fluktuacijah 
gostote,\index{Izgube v optičnih vlaknih!Rayleighovo sipanje} 
saj je sorazmerno z $\lambda^{-4}$ (modra črta). 

\begin{remark}
Sipanje na fluktuacijah gostote predstavlja poglavitni del izgub v vlaknu. Na splošno
so gostotne fluktuacije v steklu zaradi amorfne zgradbe neizogibne, vendar so v vlaknih
navadno še precej večje. Med izdelavo steklo namreč močno segrejejo
(na okoli $2000~\si{\celsius}$), da lahko iz njega vlečejo vlakno in termične 
fluktuacije gostote pri hitrem ohlajanju ostanejo zamrznjene v vlaknu. 
\end{remark}

S slike~\ref{FibAbs} je razvidno, da so skupne izgube najmanjše pri 
okoli $1,55~\si{\micro\meter}$, zato se to območje največ uporablja za prenos signalov
na velike razdalje. Izgube so tako majhne, da omogočajo prenos signala 
več sto kilometrov brez vmesnega ojačevanja. Teh izgub se  
ne bo dalo več kaj dosti izboljšati, saj so že zdaj na meji,
določeni s termičnimi fluktuacijami. Pri dolžini optičnih zvez tako izgube niso več glavna
omejitev, ampak je to popačitev signala zaradi disperzije.

\begin{remark}
% Pri prenosu signalov z optičnimi vlakni vpeljemo različne pasove, ki ustrezajo 
% različnim valovnim dolžinam. Pri valovnih dolžinah $1260$--$1360~\si{\nano\meter}$ je tako imenovani
% pas O ({\it original}), ki so ga sprva uporabljali zaradi razpoložljivih virov svetlobe
% in nizke disperzije. Sledita pas E ({\it extended}, $1360$--$1460~\si{\nano\meter}$) in pas S 
% ({\it short}, $1460$--$1530~\si{\nano\meter}$). 
% Najširše uporabljan je pas C ({\it conventional}) pri valovnih dolžinah $1530$--$1565~\si{\nano\meter}$,
% sledita mu še pas L ({\it long}, $1565$--$1625~\si{\nano\meter}$) in pas U 
% ({\it ultralong}, $1625$--$1675~\si{\nano\meter}$).
\index{Razvrščanje po valovni dolžini|see Multipleksiranje}
Po optičnem vlaknu lahko prenašamo več signalov hkrati, če za vsakega posebej uporabimo
drugo valovno dolžino. Temu procesu pravimo razvrščanje po valovni dolžini
(WDM -- {\it Wavelength-Division Multiplexing})\index{Multipleksiranje}
 in z njim dosežemo vzporeden prenos podatkov in hitrosti prenosa do 100~Tb/s.
 
Shematsko je tak način prenosa podatkov prikazan na sliki~\ref{WDM}.
Oddajniki (O) oddajo sunke svetlobe, ki se rahlo razlikujejo v valovni dolžini. 
Z multiplekserjem (M) signale iz različnih kanalov zberemo in jih usmerimo v 
enorodovno vlakno. Vlakno prenaša signal, vmes ga po potrebi ojačimo (A), 
nato z demultiplekserjem (DM) signal razstavimo na posamezne kanale, ki jih 
zaznamo z ločenimi detektorji (D). Razlika v valovnih dolžinah med posameznimi signali 
je tipično $0,8~\si{nm}$. Zanimivo je tudi, da so (de)multiplekserji pasivni in za 
svoje delovanje ne potrebujejo elektrike.\footnote{~Glej npr. B. E. A. Saleh in M. C. Teich, 
{\it Fundamentals of Photonics}, druga izdaja, John Wiley \& Sons, Inc. (2007).}
\begin{figure}[ht]
\centering
\def\svgwidth{120truemm} 
\input{slike/10_WDM.pdf_tex} 
\caption{Shematski prikaz prenosa več signalov hkrati po enorodovnem vlaknu}
\label{WDM}
\end{figure}
\end{remark}
\subsection*{*Izgube v ukrivljenem vlaknu}
\index{Izgube v optičnih vlaknih!ukrivljeno vlakno}
V vseh primerih do zdaj smo privzeli, da je vlakno povsem ravno 
oziroma da so mejne ploskve valovnega vodnika vzporedne. 
Kadar je vlakno ukrivljeno, del valovanja uhaja v plašč in 
izgube pri prenosu se povečajo. Te izgube postanejo znatne, 
kadar je krivinski radij ukrivljenega vlakna tipično centimeter ali manj. 
Poglejmo si pojav podrobneje na planparalelnem vodniku.
\index{Optični vodnik!planparalelni}

Naj bo vodnik dvodimenzionalna plast debeline $a$ z lomnim količnikom
$n_{1}$, ki je obdana s snovjo z lomnim količnikom $n_{2}$. Vodnik naj
zdaj ne bo raven, temveč ukrivljen s krivinskim radijem $R$, tako da tvori 
del kolobarja z notranjim polmerom $R-a/2$ in zunanjim polmerom $R+a/2$.
Privzamemo, da je $R\gg a$ (slika~\ref{fig:bend}). 
\begin{figure}[ht]
\centering
\def\svgwidth{50truemm} 
\input{slike/10_Krivina.pdf_tex} 
\caption{K izračunu izgub v ukrivljenem vodniku}
\label{fig:bend}
\end{figure}
\vglue-5truemm
Zapišemo Helmholtzevo enačbo (enačba~\ref{eq:Helmholtz}) v cilindrični \index{Helmholtzeva enačba}
geometriji
\begin{equation}
\frac{1}{r}\,\frac{\partial}{\partial r}\, r\,\frac{\partial E}{\partial r}
+\frac{1}{r^{2}}\,\frac{\partial^{2}E}{\partial\varphi^{2}}+k_{0}^{2}n^{2}\left(r\right)\, E=0,
\label{9.31}
\end{equation}
kjer ima $n\left(r\right)$ vrednost $n_{1}$ v sredici in $n_{2}$ v plašču. 
Pri tem ne pozabimo, da $r$ ni več radialna koordinata vlakna, ampak
označuje oddaljenost od središča krivine. Zanimajo nas rešitve oblike 
\begin{equation}
E(r, \varphi) =\psi\left(r\right)\, e^{im\varphi},
\label{9.32}
\end{equation}
kjer bomo privzeli, da je $\psi\left(r\right)$ znatna le v sredici. 

Naj bo $z=R\varphi$ dolžina loka vzdolž sredine sredice. Tedaj je faza nastavka
(enačba~\ref{9.32}) enaka $m\varphi = m z/R$ in valovno število vzdolž 
sredine vlakna $\beta = m/R$.
Ker je valovna dolžina svetlobe dosti manjša od $R$, je $m$ zelo veliko število. 
Funkcija $\psi$ zadošča enačbi 
\begin{equation}
\frac{d^{2}\psi}{dr^{2}}+\frac{1}{r}\,\frac{d\psi}{dr}+\left(k_{0}^{2}\, 
n^{2}\left(r\right)-\frac{m^{2}}{r^{2}}\right)\psi=0.
\label{9.33}
\end{equation}
Rešitve za $\psi$ so kombinacije Besslovih funkcij reda $m$, kar
zaradi velikosti $m$ ni posebno zanimivo. 

Dosti več bomo izvedeli, če se problema lotimo drugače. Namesto $r$
in $\varphi$ vpeljemo koordinati $x=r-R$ in $z=R\varphi$.
S tem preidemo nazaj na koordinate planparalelne plasti in iščemo popravke valovne
enačbe v sredici (\ref{9.3a}), ki so reda $1/R.$ Zapišemo
\begin{equation}
\frac{m^{2}}{r^{2}}=\frac{m^{2}}{\left(R+x\right)^{2}}\approx\frac{m^{2}}
{R^{2}}\,\left(1-2\,\frac{x}{R}\right)=\beta^{2}\left(1-2\,\frac{x}{R}\right).
\label{9.34}
\end{equation}
Enačbo (\ref{9.33}) zdaj nadomestimo s približno enačbo za prečno obliko
polja 
\begin{equation}
\frac{d^{2}\psi}{dx^{2}}+\left(k_{0}^{2}\, n^{2}\left(r\right)-\beta^{2}\right)\,\psi+\frac{1}{R}\,
\left(\frac{d\psi}{dx}+2\,\beta^{2}x\,\psi\right)=0.
\label{9.35}
\end{equation}
Člen, ki vsebuje prvi odvod $d \psi/d x$, lahko odpravimo z nastavkom 
\begin{equation}
\psi(x) = e^{-x/2R} \zeta(x)
\end{equation}
in dobimo enačbo
\begin{equation}
\frac{d^{2}\zeta}{dx^{2}}+\left(k_{0}^{2}\, n^{2}\left(r\right)-\beta^{2}-\frac{1}{4R^2}\right)\,\zeta
+ \frac{2\beta^{2}}{R}\,x\,\zeta=0.
\label{9.35a}
\end{equation}
Enačba je podobna enačbi za izračun lastnih rodov v planparalelnem
vodniku (\ref{9.3a}), pri čemer se $\beta^2$ poveča za $1/4R^2$. Poleg tega je prisoten
člen, ki je linearen v $x$. Če ponovno naredimo primerjavo
med lastnimi načini v valovnem vodniku in stanji delca, ujetega v končno potencialno jamo, 
ta člen ustreza potencialni energiji delca v konstantnem zunanjem električnem 
polju (slika~\ref{fig:tunel}). Podobno kot ujeti delci uhajajo iz potencialne jame
v prisotnosti zunanjega polja (tunelirajo), uhaja tudi svetloba iz ukrivljenega vlakna.
\begin{figure}[ht]
\centering
\def\svgwidth{100truemm} 
\input{slike/10_Tunel.pdf_tex} 
\caption{Lastni načini širjenja svetlobe po ravnem vodniku so analogni stanjem 
delca v končni potencialni jami (a). Načini širjenja po ukrivljenem vodniku
so podobni stanjem delca v konstantnem zunanjem električnem polju (b). Podobno
kot delci zaradi spremenjenega potenciala tunelirajo, uhaja svetloba iz ukrivljenega vlakna.}
\label{fig:tunel}
\end{figure}

Po analogiji s kvantno mehaniko, kjer ukrivljenost vodnika ustreza jakosti električnega polja,
lahko izgube iz vodnika (oziroma delež prepuščene svetlobe) zapišemo kot 
\begin{equation}
A \propto e^{-CR},
\end{equation}
pri čemer je $C$ konstanta, odvisna od lomnih količnikov sredice in plašča, od polmera vlakna 
ter od valovne dolžine potujoče svetlobe.\footnote{~Glej npr. A. Yariv in 
P. Yeh, {\it Photonics}, šesta izdaja, Oxford
University Press (2007).} 


\subsection*{Ojačevanje v vlaknih}
\index{Optično ojačevanje!v vlaknih}
Zaradi izgub pri prenosu signalov na več tisoč kilometrov dolge razdalje 
je treba signal med prenosom ojačevati. To lahko naredimo elektronsko, kjer optični signal 
pretvorimo v električnega, tega ojačimo in ga nato pretvorimo nazaj v optičnega. 
Precej bolj priročna rešitev je optično ojačevanje v vlaknu 
samem.\footnote{~Glej npr. A. Ghatak in K. Thyagarajan, {\it Introduction to Fiber Optics}, 
Cambridge University Press (1997).} 

V ta namen se najpogosteje uporablja vlakno, dopirano z erbijevimi 
ioni\footnote{~EDFA - {\it Erbium-Doped Fiber Amplifier}, \index{Optično vlakno!dopirano z erbijem}
\index{Erbij}
ojačevalnik na vlakno, dopirano z erbijem}. 
Na določenih razdaljah (na okoli $100~\si{\kilo\meter}$) svetlobo iz navadnega vlakna 
sklopimo v vlakno, v katerem so erbijevi ioni.
S črpalnim laserjem erbijeve ione vzbudimo, da dosežemo obrnjeno zasedenost. \index{Obrnjena zasedenost}
Ko na dopirani del vlakna vpade svetlobni sunek z valovno dolžino okoli 
$1550~\si{\nano\meter}$, pride do stimulirane emisije in \index{Stimulirano sevanje}
sunek se ojači. To je povsem enak postopek ojačevanja svetlobe, kot ga poznamo iz 
delovanja laserja, le da tukaj svetloba ni ujeta v resonator, ampak se postopoma 
ojačuje vzdolž vlakna. Pri tem se intenziteta črpalnega laserja postopoma zmanjšuje,
kar omejuje dolžino, na kateri se signal ojačuje.
Spektralna širina ojačenja je zaradi sklopitev z ioni v steklu 
razmeroma široka, tudi $40~\si{\nano\meter}$. To pomeni, da se hkrati ojačujejo signali različnih 
valovnih dolžin, kar je še posebej uporabno pri prenosu več signalov naenkrat.

V praksi se uporabljajo vlakna, v katerih je delež erbijevih ionov okoli $\sim 10^{-4}$. 
Črpalni laser je polprevodniški laser, ki 
deluje pri valovni dolžini $900~\si{\nano\meter}$ ali $1480~\si{\nano\meter}$ 
z močjo okoli $20$--$100~\si{\milli\watt}$. Na ta način lahko v $10$--$30~\si{\meter}$
dolgih odsekih vlaken dosežemo več 1000-kratno ojačenje ($30$--$40~\si{\decibel}$), kar 
je dovolj za kompenzacijo izgub.

\section{Sklopitev svetlobe v optične vodnike}
\index{Sklopitev v optično vlakno}
Do zdaj smo govorili o svetlobi, ki potuje po valovnem vodniku ali optičnem vlaknu. Kako pa 
svetlobo sploh sklopimo v vodnik? Poznamo več načinov sklopitve, obravnavali bomo 
čelno sklopitev, bočno sklopitev s prizmo in bočno sklopitev s periodično strukturo.
Prvi način navadno uporabljamo pri 
cilindričnih vlaknih, medtem ko druga dva načina najpogosteje pri planarnih valovodnih strukturah.

\subsection*{Čelna sklopitev}
\index{Sklopitev v optično vlakno!čelna sklopitev}
Sklopitev svetlobe v večrodovno vlakno lahko obravnavamo geometrijsko, kot smo to naredili
na začetku poglavja (slika~\ref{fig:vodnik}). Izračunali smo, da je največji 
vpadni kot, pod katerim se svetloba še sklopi v vlakno, določen z numerično odprtino 
vlakna $\sin \alpha_{\mathrm{max}}= NA$  (enačba~\ref{10NAa}).

Za bolj natančen izračun izkoristka sklopitve svetlobe v optično vlakno vpeljemo 
tako imenovani prekrivalni integral, ki pove, kolikšen delež vpadne svetlobe 
z jakostjo električnega polja $E(r, \varphi)$ se sklopi z izbranim rodom 
vlakna.\index{Prekrivalni integral}
To naredimo tako, da vpadni val razvijemo po lastnih rodovih vlakna in iz ortogonalnosti
sledi prekrivalni integral, ki ga moramo seveda ustrezno normirati. 

Za sklopitev v rod, 
označen z indeksoma $n$ in $m$, zapišemo prekrivalni integral 
kot\footnote{~Glej npr. C. R. Pollock, {\it Fundamentals of Optoelectronics}, Irwin (1995).}
\boxeq{10:prekint}{
\eta = \frac{|\int E(x,y) E^*_{n,m}(x,y) dx\, dy|^2}
{\left(\int |E(x, y)|^2 dx\,dy \right) \left(\int |E_{n,m}(x, y)|^2 
dx\, dy \right)}.
}
Točen račun prekrivalnega integrala je na splošno precej zapleten, saj vsebuje integrale
Besslovih funkcij. V primeru osnovnega roda račun močno poenostavimo, tako da 
prečno odvisnost polja nadomestimo z Gaussovim profilom z ustreznim 
efektivnim polmerom (po enačbi~\ref{Marcuse}). 

\subsection*{Bočna sklopitev}
\index{Sklopitev v optično vlakno!bočna sklopitev}
Neposredna sklopitev svetlobe v optični vodnik preko plašča ni mogoča. V vodniku je namreč
lomni količnik sredice vedno večji od lomnega količnika plašča, zato dovolj velikega vstopnega 
kota, pod katerim bi se svetloba ujela v sredico, ni mogoče doseči. Za sklopitev preko stranice 
zato uporabimo drugačen pristop, navadno s prizmo ali s periodično strukturo na 
vlaknu.\footnote{~Glej npr. B. E. A. Saleh in M. C. Teich, 
{\it Fundamentals of Photonics}, druga izdaja, John Wiley \& Sons, Inc. (2007).}

V prvem primeru uporabimo prizmo (slika~\ref{fig:coupler}\,a). Lomni količnik prizme
je pri tem večji od lomnega količnika plašča $n_p > n_2$.
Vhodni žarek vpada na prizmo, se ob prehodu vanjo lomi, nato  se na stranici, ki je vzporedna
z vodnikom, totalno odbije.\index{Totalni odboj} 
V tankem vmesnem območju med prizmo in sredico vodnika se pojavi evanescentni\index{Evanescentno polje}
val s komponento valovnega vektorja $\beta_v  = k_0 n_p \sin \alpha$ v 
smeri vzporedno z vodnikom. 

Pogoj za uspešno 
sklopitev v vlakno je ujemanje vzdolžne komponente valovnega vektorja vpadne svetlobe $\beta_v$ 
z vzdolžno komponento valovnega vektorja $\beta_n$ tistega rodu, ki ga želimo vzbuditi. 
S spreminjanjem vpadnega kota $\alpha$ spreminjamo $\beta_v$ in v vodniku vzbujamo različne rodove. 
Pri tem mora biti razdalja med prizmo in vodnikom dovolj majhna (tipično reda valovne dolžine svetlobe), 
da se v valovod ob izpolnjenem pogoju ujemanja faze sklopi znaten delež vpadne svetlobe.

\begin{figure}[ht]
\centering
\def\svgwidth{128truemm} 
\input{slike/10_coupler.pdf_tex} 
\caption{Bočna sklopitev svetlobe v optični vodnik s prizmo (a) in s periodično strukturo (b)}
\label{fig:coupler}
\end{figure}
Tudi sklopitev s periodično strukturo na valovnem vodniku (slika~\ref{fig:coupler}\,b) deluje na 
ujemanju vzdolžnih komponent valovnega vektorja vpadnega vala in valovnega vektorja ustreznega rodu.
Ko vpade val pod kotom $\alpha$ glede na valovni vodnik, periodična struktura na vodniku 
spremeni njegovo fazo za večkratnik $2 \pi z/\Lambda$, pri čemer je $\Lambda$ perioda strukture.
Če dosežemo, da se komponenta novega valovnega vektorja $\beta = k_0 n_2 \sin \alpha+ 
2 \pi/\Lambda$ izenači s  komponento valovnega vektorja za izbrani rod v vlaknu, 
se vpadna svetloba sklopi v vlakno.

\begin{remark}
 Oba opisana načina za sklopitev svetlobe v vlakno uporabljamo tudi za sklopitev svetlobe 
 iz vlakna, pri čemer mora biti ravno tako izpolnjen pogoj ujemanja faz. 
 Sklapljanje svetlobe skozi prizmo je uporabno tudi za raziskave tankih plasti snovi. 
 Iz pogoja za ujemanje faz lahko določimo lastnosti tanke plasti, na primer njen lomni količnik. 
\end{remark}

\section{Sklopitev med optičnimi vodniki}
\index{Sklopitev med valovodi}
\subsection*{Čelna sklopitev dveh vlaken}
\index{Izgube v optičnih vlaknih!spoj dveh vlaken}
Pri telekomunikacijah z optičnimi vodniki so spoji med posameznimi vodniki neizogibni.
V idealnem primeru sta vodnika povsem enaka in se natančno stikata, tako da na spoju
ne prihaja do dodatnih izgub ali popačenja signala. Čim se pojavijo odstopanja 
v velikosti polmera sredice, razlike v vrednostih lomnih količnikov ali nenatančna poravnava 
sredice, na spoju pride do izgub. Tipično znašajo izgube na spoju vlaken do okoli 
$0,2$--$0,5~\si{\decibel}$.

Omejimo se na spoj dveh enorodovnih vlaken, v katerih krajevni del jakosti 
električnega polja osnovnega rodu zapišemo kot
\begin{equation}
E(r, \varphi, z)=\psi(r, \varphi) e^{i\beta z}.
\end{equation} 
Podobno kot smo zapisali prekrivalni integral pri sklopitvi svetlobe v vlakno (enačba~\ref{10:prekint}),
vpeljemo prekrivalni integral za izračun sklopitve med dvema vlaknoma,\index{Prekrivalni integral}
ki pove, kolikšen delež svetlobe moči iz prvega vlakna 
se sklopi v osnovni rod v drugem vlaknu. Sklopitveni faktor je
\boxeq{10:overlap}{
\eta = \frac{|\int \psi_1(r, \varphi) \psi_2^*(r, \varphi) r\, dr\, d\varphi|^2}
{\left(\int |\psi_1|^2 r\, dr\, d\varphi \right) \left(\int |\psi_2|^2 r\, dr\, d\varphi \right)},
}
pri čemer z indeksom $1$ označimo osnovni rod v prvem vlaknu in z indeksom $2$ v drugem. 
Tudi tukaj račun pogosto poenostavimo in namesto Besslovega profila uporabimo 
Gaussov profil z ustreznim efektivnim polmerom snopa (enačba~\ref{Marcuse}).\index{Gaussov 
snop!efektivni polmer}

Izračunajmo za primer sklopitveni faktor in izgube na spoju dveh vlaken z rahlo 
različnima polmeroma. Po Marcusejevi formuli najprej določimo efektivna polmera Gaussovih snopov
v obeh vlaknih $w_1$ in $w_2$. Prečni profil v vlaknih je potem\index{Marcusejeva formula}
\begin{equation}
\label{eq:pripsi1}
\psi_{1} = A_{1} e^{-r^2/w_{1}^2}
\end{equation}
in 
\begin{equation}
\label{eq:pripsi2}
\psi_{2} = A_{2} e^{-r^2/w_{2}^2}.
\end{equation}
Vstavimo nastavka (enačbi~\ref{eq:pripsi1} in \ref{eq:pripsi2}) 
v prekrivalni integral~(enačba~\ref{10:overlap}) in dobimo
\begin{equation}
\eta = \frac{|\int A_1 \, e^{-r^2/w_1^2}\, A_2\, e^{-r^2/w_2^2}\, 2 \pi\, r\, dr|^2}
{\left(\int A_1^2 \,e^{-2r^2/w_1^2} \, 2 \pi \, r\, dr \right) \left(\int A_2^2\, 
e^{-2r^2/w_2^2}\, 2 \pi \, r\, dr \right)},
\label{10:prekintw}
\end{equation}
od koder sledi
\begin{equation}
\eta = \frac{4 w_1^2 w_2^2}{(w_1^2+w_2^2)^2}.
\label{10:w1w2}
\end{equation}
Kadar sta polmera vlaken enaka, je prekrivanje popolno in $\eta = 1$. Z naraščajočo razliko
med polmeroma vrednost $\eta$ pojema. Pri tem
ni pomembno, ali ima večji polmer prvo ali drugo vlakno, v obeh primerih se signal oslabi. 
Intuitivno razumemo, da se signal izgubi pri prehodu iz večjega v manjše vlakno, 
vendar je obratno tudi res, saj se v širšem končnem vlaknu poleg osnovnega vzbudijo
tudi višji rodovi. 

Pri prehodu iz vlakna z $w = 10~\si{\micro\meter}$ v vlakno
s polmerom $w = 8~\si{\micro\meter}$ (ali obratno), je sklopitveni faktor (oziroma
razmerje med prepuščeno in vpadno intenziteto svetlobe) enak $0,95$. Po enačbi~(\ref{dB})
so izgube za izračunano sklopitev enake $0,21~\si{\decibel}$. 

\begin{naloga}
Izračunaj prekrivalni integral (enačba~\ref{10:prekintw}) in izpelji enačbo~(\ref{10:w1w2}).
Pokaži tudi, da je sklopitveni faktor za dve enaki vzporedni vlakni, ki sta iz osi izmaknjeni
za $\Delta$, enak
\begin{equation}
\eta = \exp \left( - \frac{\Delta^2}{w^2}\right).
\end{equation}

\end{naloga}
\subsection*{Vzdolžna sklopitev}
Ob prenosu signala po optičnem valovodu večina energijskega toka potuje po sredici,
vendar energijski tok seže tudi izven nje, v plašč (enačba~\ref{confinement}). 
Če sta dva vzporedna valovoda dovolj blizu, da\index{Evanescentno polje}
se evanescentni električni polji enega in drugega vodnika v plašču prekrivata, se vodnika 
sklopita in energijski tok se prenese iz enega vodnika v drugega 
(slika~\ref{fig:fcoupler}). 
\begin{figure}[ht]
\centering
\def\svgwidth{90truemm} 
\input{slike/10_fcoupler.pdf_tex} 
\caption{Sklopitev med dvema vzporednima vodnikoma.}
\label{fig:fcoupler}
\end{figure}

Za podrobnejšo obravnavo bi morali zapisati Maxwellove enačbe z ustreznimi robnimi pogoji 
in jih rešiti za sklopljen primer dveh vzporednih vodnikov. Tak račun je izredno zapleten, 
zato bomo uporabili približek šibke sklopitve in predpostavili, da so rodovi v vodnikih taki, 
kot če bi vodniki ne bili sklopljeni.
Sklopitev torej ne bo spremenila oblike lastnih rodov, bo pa spremenila njihove amplitude, ki 
bodo tako postale odvisne od vzdolžne koordinate $z$.\footnote{~Glej npr. B. E. A. Saleh in M. C. Teich, 
{\it Fundamentals of Photonics}, druga izdaja, John Wiley \& Sons, Inc. (2007).}

Imejmo dva enorodovna vodnika z lomnima količnikoma sredice $n_1$ in $n_2$ in enako 
debelino $a$, med njima in okoli njiju pa naj bo snov z lomnim količnikom $n_0$. Širina 
reže med vodnikoma naj bo $2d$.
Potem zapišemo jakosti električnega polja v prvem in drugem vodniku kot
\begin{equation}
E_1(x,z) = A(z) \psi_1(x) e^{i \beta_1 z}
\end{equation}
in
\begin{equation}
E_2(x,z) = B(z) \psi_2(x) e^{i \beta_2 z},
\end{equation}
pri čemer se $A(z)$ in $B(z)$ le počasi spreminjata s koordinato $z$.

Skupna jakost električnega
polja, ki je v našem približku kar vsota prispevkov $E_1$ in $E_2$, 
mora zadoščati Helmholtzevi enačbi
(enačba~\ref{eq:Helmholtz})
\begin{equation}
\nabla^{2}E(x,z)+k_0^{2}n(x)^2 E(x,z) =0.
\end{equation}
Z $n(x)$ smo označili prečno odvisnost lomnega količnika.
Vstavimo nastavek za jakost električnega polja v Helmholtzevo enačbo in zapišemo
\begin{align}
A e^{i \beta_1 z}\left(\frac{\partial^2\psi_1(x)}{\partial x^2} - \beta_1^2\psi_1 + k_0^2
n(x)^2 \psi_1 \right)
&+ 
B e^{i \beta_2 z}\left(\frac{\partial^2\psi_2(x)}{\partial x^2} - \beta_2^2\psi_2 + k_0^2
n(x)^2 \psi_2 \right)+ \nonumber \\ 
2 i \beta_1 \frac{\partial A}{\partial z}\psi_1 e^{i \beta_1 z}
+
2 i \beta_2 \frac{\partial B}{\partial z}\psi_2 e^{i \beta_2 z} &= 0.
\end{align}
Zaradi počasnega spreminjanja  smo člena z drugim odvodom $\partial^2 A/\partial z^2$ in $\partial^2 B/\partial z^2$ zanemarili. 

Zapišemo enačbi za nemoteni funkciji $\psi$
\begin{equation}
\frac{\partial^2\psi_1(x)}{\partial x^2} + \left(k_0^2n_1(x)^2-\beta_1^2\right) \psi_1 =0
\end{equation}
in
\begin{equation}
 \frac{\partial^2\psi_2(x)}{\partial x^2} + \left(k_0^2n_2(x)^2-\beta_2^2\right) \psi_2 =0.
\end{equation}
Lomna količnika $n_1(x)$ in $n_2(x)$ sta tukaj tudi funkciji prečne koordinate. Naj bo $n_1(x)$  
povsod enak $n_0$, razen v sredici prvega vodnika, kjer je $n_1$, in naj bo 
$n_2(x)$ povsod enak $n_0$, razen v sredici drugega vlakna, kjer je enak $n_2$. Sledi
\begin{align}
A e^{i \beta_1 z}k_0^2\left(n(x)^2 -n_1(x)^2\right)\psi_1 
+ 
B e^{i \beta_2 z}k_0^2\left(n(x)^2 -n_2(x)^2\right)\psi_2 &+ 
2 i \beta_1 \frac{\partial A}{\partial z} \psi_1 e^{i \beta_1 z}
+\nonumber \\ 
2 i \beta_2 \frac{\partial B}{\partial z} \psi_2 e^{i \beta_2 z} &= 0.
\end{align}
Enačbo pomnožimo s kompleksno konjugirano vrednostjo $\psi_1^*$ in integriramo po $x$.
Upoštevamo, da se funkciji $\psi_1$ in $\psi_2$ le malo prekrivata, in zapišemo 
\begin{equation}
\frac{\partial A}{\partial z} = i A K_{11}+i B e^{i(\beta_2-\beta_1)z} K_{12},
\end{equation}
pri čemer sta
\begin{equation}
K_{11}= \frac{k_0^2}{2 \beta_1}\int\psi_1^*\psi_1 (n^2-n_1^2)dx \qquad \mathrm{in} \qquad 
K_{12}= \frac{k_0^2}{2 \beta_1}\int\psi_1^*\psi_2 (n^2-n_2^2)dx.
\end{equation}
Koeficient $K_{11}$ določa spremembo faze v vlaknu zaradi prisotnosti 
drugega vlakna in ga lahko zanemarimo. Tako ostane samo 
\begin{equation}
\frac{\partial A}{\partial z} = i B e^{i(\beta_2-\beta_1)z} K_{12} \qquad \mathrm{in}
\qquad \frac{\partial B}{\partial z} = i A e^{i(-\beta_2+\beta_1)z} K_{21}.
\label{10_B}
\end{equation}
Prvo enačbo odvajamo po $z$, upoštevamo drugo in zapišemo
\begin{equation}
\frac{\partial^2 A}{\partial z^2}-i \Delta \beta \frac{\partial A}{\partial z} + K_{12}K_{21}A = 0.
\label{10_part}
\end{equation}
Enačbo~\ref{10_part} rešujemo z nastavkom 
\begin{equation}
A = e^{i \Delta \beta z/2}\left( a_1 e^{i \gamma z} + a_2 e^{-i \gamma z}\right).
\label{10_nastavek}
\end{equation}
\begin{naloga}
 Pokaži, da  nastavek (enačba~\ref{10_nastavek}) reši enačbo~(\ref{10_part}) in pokaži,
 da med parametri enačb velja sledeča zveza, pri čemer je $K = \sqrt{K_{12}K_{21}}$ in 
 $\Delta \beta = \beta_2 - \beta_1$,
 \begin{equation}
 \gamma^2 = K^2 + \frac{\Delta \beta ^2}{4}.
 \label{10kgamma}
 \end{equation}
 Uporabi enačbi~(\ref{10_B}) in pokaži, da je rešitev za amplitudo $B$
 enaka izrazu v enačbi~(\ref{10_BB}).
\end{naloga}
Ko poznamo $A$, lahko z uporabo enačb~(\ref{10_B}) izračunamo še $B$
\begin{equation}
B = \frac{1}{K_{21}}
e^{-i \Delta \beta z/2}\left(\left(\frac{\Delta \beta}{2} +\gamma \right) a_1 e^{i \gamma z} + 
\left(\frac{\Delta \beta}{2} -\gamma \right)a_2 e^{-i \gamma z}\right).
\label{10_BB}
\end{equation}
Naj bo $A(z=0) = A_0$ in $B(z=0)=0$. To pomeni, da potuje svetloba na začetku
le po prvem vlaknu, nato se sklopi v drugega. S tema začetnima pogojema zapišemo izraza 
za $A$ in $B$
\begin{equation}
 A = A_0 e^{i \Delta \beta z/2} \left( \cos(\gamma z) - 
\frac{i \Delta \beta}{2 \gamma}\sin(\gamma z) \right)
\quad \mathrm{in} \quad 
B =  A_0 e^{-i \Delta \beta z/2} \frac{i K_{21}}{\gamma}\sin(\gamma z).
\end{equation}
Moč, ki se pretaka po posameznem vlaknu, je tako z upoštevanjem zveze~(\ref{10kgamma})
\begin{equation}
P_1 = P_0 \left( \cos^2(\gamma z) + \frac{\Delta \beta^2}{4 \gamma^2}\sin^2(\gamma z) \right)
= P_0 \left( 1 - \frac{K^2}{\gamma^2}\sin^2(\gamma z) \right)
\end{equation}
in
\begin{equation}
P_2 =  P_0\, \frac{K^2}{\gamma^2}\sin^2(\gamma z).
\end{equation}
Privzeli smo, da velja $|K_{12}|=|K_{21}|= K$. 
Obe odvisnosti sta oscilirajoči in svetloba se periodično pretaka med vlaknoma
s periodo $\pi/\gamma$ (slika~\ref{fig:foscil}). Amplituda prenosa je 
odvisna od sklopitvenega faktorja $K$ in ujemanja valovnih 
števil v obeh vlaknih. Večji koeficient $K$
in manjše odstopanje $\Delta \beta$ vodita v večji prenos svetlobnega toka v drugo vlakno. 
\begin{figure}[ht]
\centering
\def\svgwidth{128truemm} 
\input{slike/10_Coupler.pdf_tex} 
\caption{Prenos svetlobnega toka med dvema sklopljenima vodnikoma. V prvem primeru (a) sta
vodnika različna, v drugem primeru (b) sta vodnika enaka in prenos je popoln.}
\label{fig:foscil}
\end{figure}

Če sta vlakni 
enaki, je $\Delta \beta = 0$ in $\gamma = K$, tako da pride do popolnega prenosa
svetlobnega toka iz enega vlakna v drugo in seveda tudi obratno.
Takrat veljata enačbi
\boxeq{10_couplfib}{
P_1 &= P_0 \cos^2 (\gamma z) \quad \mathrm{in}\\
P_2 &= P_0 \sin^2 (\gamma z).
}
Na ta način lahko v drugo vlakno sklopimo poljuben delež vpadne svetlobe. 
Celotni prenosa svetlobnega toka v drugo vlakno nastopi pri dolžini sklopitve $L = \pi/2 \gamma$. 
Pri dolžini $L = \pi/4 \gamma$ sklopimo eno polovico gostote vpadnega svetlobnega toka in govorimo o 3-dB sklopitvi. \index{Sklopitev med valovodi!3-dB sklopitev}

\begin{remark}
 Pri izbrani dolžini sklopitve med vlaknoma je intenziteta svetlobe, ki preide v drugo vlakno,
 močno odvisna od parametra $\gamma$, torej od prekrivalnega integrala in od razlike $\Delta \beta$. 
 Z rahlim spreminjanjem parametrov, na primer lomnega 
 količnika enega od vlaken, lahko spreminjamo delež svetlobe v drugem vlaknu. V ta 
 namen pogosto uporabimo elektro-optični pojav in s spreminjanjem priključene napetosti na
 enem od vlaken natančno določimo delež svetlobe, ki preide v drugo vlakno. \index{Elektro-optični pojav}
\end{remark}

\section{*Vpliv spremembe lomnega količnika vlakna na širjenje svetlobe}
Sprememba lomnega količnika sredice ali plašča vlakna povzroči spremembo
valovnega števila $\beta$ za izbran rod. V enorodovnih vlaknih je to
mogoče izkoristiti za izdelavo senzorjev, na primer temperature ali
tlaka. Spremembo valovnega števila, ki je posledica zunanjih vplivov,
izmerimo preko spremembe faze valovanja na izhodu iz vlakna
z ustrezno sestavljenim interferometrom. Ker je dolžina vlakna lahko
velika (v nekaj centimetrov velik tulec lahko brez težav navijemo
kilometre vlakna), je celotna sprememba faze velika že pri majhnih
spremembah merjene količine. Po drugi strani že majhna sprememba 
valovnega števila povzroča neželene spremembe faze in odboje pri prenosu 
informacij.
V tem razdelku zato poglejmo, kako se spremeni valovno število pri
dani spremembi lomnega količnika in koliko svetlobe se odbije.

Obravnavajmo rod z vzdolžno komponento valovnega vektorja $\beta_{lm}$ in prečnim
profilom $\psi_{lm}\left(r,\varphi\right).$ Ta mora zadoščati Helmholtzevi enačbi 
(enačba~\ref{eq:Helmholtz})
\begin{equation}
\nabla_{\bot}^{2}\psi_{lm}+\left(\epsilon(r)k_{0}^{2}-\beta_{lm}^{2}\right)\psi_{lm}=0.
\label{9.22}
\end{equation}
Naj se dielektrična konstanta na delu vlakna spremeni za $\delta\epsilon.$
Posledično se na tem mestu spremenita tudi valovno število $\beta=\beta_{lm}+\delta\beta$
in prečna oblika $\psi=\psi_{lm}+\delta\psi.$ 

Tudi popravljena funkcija
$\psi$ mora zadoščati enačbi (\ref{9.22}), zato za perturbacijo velja
\begin{equation}
\nabla_{\bot}^{2}\delta\psi+\left(\epsilon(r)k_{0}^{2}-\beta_{lm}^{2}\right)\delta\psi+
\delta\epsilon\, k_{0}^{2}\psi_{lm}=2\beta_{lm}\delta\beta\,\psi_{lm},
\label{9.23}
\end{equation}
pri čemer smo zanemarili produkte majhnih količin. Množimo obe strani
enačbe s $\psi_{lm}^{*}$, integriramo po preseku vlakna in dobimo
\begin{eqnarray}
 &  & \int\psi_{lm}^{*}\nabla_{\bot}^{2}\delta\psi\,
 dS+\int\left(\epsilon(r)k_{0}^{2}-\beta_{lm}^{2}\right)
 \delta\psi\,\psi_{lm}^{*}dS+k_{0}^{2}\int\delta\epsilon\,\left|\psi_{lm}\right|^{2}dS = \nonumber\\
 & & 2\beta_{lm}\,\delta\beta\int\left|\psi_{lm}\right|^{2}dS.
 \label{9.24}
\end{eqnarray}
Prvi člen na levi preoblikujmo z uporabo zvez 
\beq
\int(u\,\nabla_{\bot}^{2}v-v\nabla_{\bot}^{2}u)\,
dS=\int\nabla_{\bot}\cdot(u\nabla_{\bot}v-v\nabla_{\bot}u)\, 
dS=\oint (u\,\nabla_{\bot}v-v\,\nabla_{\bot}u)\cdot d\mathbf{s}.
\eeq
Funkciji $\psi_{lm}$ in $\delta\psi$ opisujeta vodene valove, zato
morata iti njune vrednosti za velike $r$ proti nič. Posledično gre
proti nič tudi integral po krivulji $d\mathbf{s}$ in velja 
\beq
\int\psi_{lm}^{*}\nabla_{\bot}^{2}\delta\psi\,
dS=\int\delta\psi\nabla_{\bot}^{2}\psi_{lm}^{*}\, dS.
\eeq
Funkcija $\psi_{lm}^{*}$ zadošča enačbi (\ref{9.22}), zato se v enačbi (\ref{9.24})
prvi in drugi člen odštejeta. Iskan popravek k valovnemu številu je tako 
\begin{equation}
\delta\beta=\frac{k_{0}^{2}\int\delta\epsilon\,\left|\psi_{lm}\right|^{2}dS}{2\,
\beta_{lm}\int\left|\psi_{lm}\right|^{2}dS}.
\label{9.25}
\end{equation}
\begin{remark}
Ta rezultat je seveda analogen kvantnomehanskemu rezultatu, ki sledi iz 
teorije motenj za spremembo energije lastnega stanja delca pri majhni 
spremembi Hamiltonovega operatorja. Rezultat je tudi intuitivno razumljiv: v
najnižjem redu je $\delta\beta$ sorazmerna s uteženim povprečjem
$\delta\epsilon$, pri čemer je utež $\psi_{lm}$.
\end{remark}

Sprememba valovnega števila $\delta \beta$ v delu vlakna ne povzroči le spremembe faze, 
ampak tudi delni odboj.
To je le nekoliko druga oblika odboja na (zvezni ali ostri) meji 
dveh dielektrikov ali, splošneje, odboja valovanja na območju,
kjer se spremeni fazna hitrost valovanja.
Amplitudo odbitega valovanja, ki se odbije na območju spreminjajočega $\beta$, 
najpreprosteje dobimo z uporabo
enačbe za odboj na meji dveh dielektrikov pri pravokotnem
vpadu. Odbita amplituda je tedaj (enačba~\ref{eq:Fresnel1})
\begin{equation}
E_{r}=\frac{n_{2}-n_{1}}{n_{2}+n_{1}}E_{0},
\label{9.26}
\end{equation}
pri čemer $n_1$ in $n_2$ označujeta nespremenjen in rahlo spremenjen lomni
količnik sredice vlakna. 

Mislimo si, da je sprememba $\beta$ na delu vlakna sestavljena iz
majhnih stopničastih sprememb $\Delta\beta_{i}$ na intervalih $\Delta z$.
Za ravno valovanje je sprememba fazne hitrosti sorazmerna s spremembo
lomnega količnika, zato iz enačbe~(\ref{9.26}) sledi, da je odbito valovanje
na stopničasti spremembi $\Delta\beta_{i}$ enako
\begin{equation}
\Delta E_{i}=\frac{\Delta\beta_{i}}{2\,\beta}\, E_{0}.
\label{9.27}
\end{equation}
Privzeli smo, da je delež odbitega valovanja tako majhen, da ni treba upoštevati 
spremembe amplitude vpadnega vala $E_{0}$. Celotno odbito valovanje je vsota 
prispevkov na posameznih stopnicah $\Delta\beta_{i}$,
pri čemer moramo upoštevati še različne faze delno odbitih valovanj
\begin{equation}
E_{r}=\sum\frac{\Delta\beta_{i}}{2\,\beta}\, e^{2i\beta z_{i}}\, 
E_{0}=\frac{1}{2\,\beta}\sum\frac{d\beta}{dz}\, e^{2i\beta z_{i}}\Delta z\, E_{0}.
\label{9.28}
\end{equation}
Preidemo z vsote na integral in zapišemo amplitudo odbitega valovanja
\begin{equation}
E_{r}=\frac{E_{0}}{2\,\beta}\,\int\frac{d\beta}{dz}\, e^{2i\beta z}dz.
\label{9.29}
\end{equation}
Za primer poglejmo linearno spremembo lomnega količnika in linearno spremembo 
valovnega števila za $\Delta\beta$ na razdalji $L$. Krajši račun pokaže, da je 
delež intenzitete odbitega valovanja 
\begin{equation}
\frac{I_{r}}{I_{0}}=\left( \frac{\Delta\beta}{2 \beta}\,\frac{\sin\beta L}{\beta L}\right)^2.
\label{9.30}
\end{equation}
Odbojnost je največja, kadar je $L \ll 1/\beta$,
torej kadar je sprememba $\beta$ ostra stopnica. Čim počasnejša je
sprememba, tem manj je odboja. Kadar je $\sin\beta L=0$, vsi delni 
odboji destruktivno interferirajo in odbojnost je enaka nič.

\begin{naloga}
Naj se valovno število ob prehodu spreminja kot funkcija erf
\begin{equation}
\beta (z)= \beta_0 + \frac{2\Delta \beta}{\sqrt{\pi}} \int_0^{z/a} e^{-t^2}dt.
\end{equation}
Pokaži, da je amplituda 
odbitega valovanja za majhne spremembe $\Delta \beta$ enaka
\begin{equation}
\frac{E_r}{E_0} = \frac{\Delta \beta}{\beta_0}e^{-a^2\beta_0^2}.
\end{equation}
Po pričakovanju je odbojnost največja pri $a\to 0$, to je v primeru ostre stopnice.
\end{naloga}

