\chapterimage{Mavrica.jpg} % Chapter heading image

\chapter{Optična vlakna}
Moderna komunikacijska tehnologija zahteva vedno hitrejši prenos
vedno večje količine informacij. Navadne kovinske vodnike
so zato v računalniških in telefonskih povezavah nadomestila optična 
vlakna, ki jih odlikujejo majhne izgube, neobčutljivost na elektromagnetne
in medsebojne motnje ter zmožnost prenosa izjemno velike količine podatkov. 

\section{Planparalelni vodnik}
\subsection{Klasična razlaga}
Klasično lahko razložimo delovanje optičnih vlaken s totalnim odbojem
na meji med dvema plastema. Kadar prehaja svetloba iz snovi 
z večjim lomnim količnikom v sredstvo z manjšim lomnim količnikom,
se pri kotih, ki so večji od kritičnega kota, totalno odbije. 
\begin{figure}[h]
\centering
\def\svgwidth{120truemm} 
\input{slike/10_Vodnik.pdf_tex}
\caption{Klasična razlaga valovnega vodnika}
\label{fig:vodnik}
\end{figure}

Najpreprostejši model optičnega vodnika je planparalelna plast prozornega
dielektrika z lomnim količnikom $n_{1}$, ki je večji od lomnega količnika
okolice $n_2$ (slika~\ref{fig:vodnik}). 
Plasti z večjim lomnim količnikom rečemo sredica, okolici
pa plašč vodnika. Žarek je ujet v sredici, če je vpadni kot 
na mejno plast $\vartheta$ večji od kota totalnega odboja, 
za katerega velja 
\boxeq{10totalni}{
\sin\vartheta_{c}=\frac{n_{2}}{n_{1}}.
}
Količini, ki določa največji kot divergence svetlobnega snopa, 
ki vpada na vodnik in ostane v njem ujet, pravimo
numerična odprtina vlakna. Izračunamo jo kot 
\beq
NA = \sin \alpha = n_1 \sin \beta = 
n_1 \sin(\pi/2-\vartheta_c) =
n_1 \cos\vartheta_c = n_1 \sqrt{1-\sin^2\vartheta_c}.
\eeq
Upoštevajoč enačbo~(\ref{10totalni}) sledi 
\boxeq{10NA}{
NA = \sqrt{n_1^2-n_2^2}.
}
Ker je razlika lomnih količnikov v vodnikih razmeroma majhna,
tipično le nekaj stotink, je tudi numerična apertura optičnih 
vodnikov navadno
$NA \lesssim 0,1 $. Kot, pod katerim lahko vpada svetloba
v vlakno, da se v njega ujame, je zato zelo majhen. 

\subsection{Valovni opis}
Za podroben opis širjenja svetlobe po vodnikih ali vlaknih, 
ki imajo navadno polmer
sredice od nekaj do nekaj deset mikrometrov, geometrijska optika ne
zadošča. Rešiti moramo Maxwellove enačbe (enačbe~\ref{eq:Maxwell1}--\ref{eq:Maxwell4}) 
z ustreznimi robnimi pogoji (enačbe~\ref{eq:robni-pogoji}--\ref{eq:robni-pogoji5}),
kar je za praktična vlakna dokaj dolg račun. Zato ugotovimo najprej, kakšne
so osnovne značilnosti valovanja, ki se širi po vodniku.

Glede na smer polarizacije električne poljske jakosti 
ločimo dva različna primera. Če leži električna poljska
jakost vzporedno z mejnima ploskvama (smer $y$), 
govorimo o transverzalnem električnem (TE) valovanju. 
V nasprotnem primeru, ko je 
z mejnima ploskvama vzporedna magnetna poljska jakost in 
leži električna poljska jakost v ravnini $xz$, 
govorimo o transverzalnem magnetnem (TM) valovanju.
\begin{figure}[h]
\centering
\def\svgwidth{120truemm} 
\input{slike/10_TETM.pdf_tex}
\caption{TE in TM polarizaciji v valovnem vodniku}
\label{fig:TETM}
\end{figure}

Geometrijskemu žarku, ki pod kotom potuje po sredici in se na njeni meji
odbija, ustreza v valovni sliki val, ki ima prečno komponento valovnega
vektorja $k_{x}$ različno od nič. Ker je valovanje v prečni smeri 
omejeno na sredico končne dimenzije (naj bo to debelina plasti $a$), ima lahko
$k_{x}$ le diskretne vrendosti, ki so približno enake $N\pi/a$. Pri 
tem je $N$ celo število in je enako številu vozlov, ki jih ima valovanje v prečni smeri.
Pravimo tudi, da vsak $N$ določa en rod valovanj v vlaknu. Po drugi strani 
pa obstaja v vodniku največji $k_x$, ki je določen 
s kotom totalnega odboja 
\beq
k_{x \mathrm{max}} \approx k_0 n_1 \cos\vartheta_c = k_0 \sqrt{n_1^2 -n_2^2}.
\eeq
Številom možnih rešitev za $k_x$ je torej omejeno in točno določeno, odvisno
pa je od razlike lomnih količnikov in od dimenzije vodnika oziroma vlakna. 
V nadaljevanju bomo spoznali, da v optičnih vlaknih en rod vselej obstaja,
za razliko od dielektričnih in kovinskih vodnikov, kakršne
poznamo iz mikrovalovne tehnike, po katerih se pod določeno frekvenco
valovanje ne more širiti. Optični vodniki, po katerih se širi
en sam rod, imajo posebej lepe lastnosti za uporabo v komunikacijskih
sistemih.

Povejmo še nekaj o hitrosti valovanja v vlaknu.
Naj bo $\beta$ komponenta valovnega vektorja vzdolž
vlakna, recimo ji tudi valovno število, tako da je odvisnost polja
od koordinate vzdolž vlakna $\exp (i\beta z)$. Po drugi strani pa velja
zveza
\begin{equation}
n_{1}\frac{\omega}{c_0}=\sqrt{\beta^{2}+k_{x}^{2}}
\label{9.0}.
\end{equation}
Za dano vrednost $k_{x}$ torej zveza med valovnim številom $\beta$
in frekvenco $\omega$ ni linearna, zato je fazna hitrost $v_{f}=\omega/\beta$
odvisna od frekvence in pride do disperzije. Grupna hitrost $v_{g}=d\omega/d\beta$ 
je zaradi nelinearne odvisnosti različna od fazne hitrosti in tudi odvisna od 
frekvence, kar ima za uporabo vlaken pomembne posledice. Več o tem bomo spoznali 
proti koncu poglavja. 

\section{Račun lastnih rodov v planparalelnem vodniku}
Poiščimo zdaj rešitve valovne enačbe v planparalelnem vodniku. 
To je preprost dvodimenzionalen model optičnega vlakna, ki je sestavljen iz 
plasti prozornega dielektrika in plašča, ki naj bo zaradi enostavnosti na obeh 
straneh sredice enak. 
\begin{figure}[h]
\centering
\def\svgwidth{120truemm} 
\input{slike/10_VodnikRacun.pdf_tex}
\caption{K izračunu lastnih rodov v vodniku}
\label{fig:vodnikracun}
\end{figure}

Krajevni del valovne enačbe, ki jo rešujemo,
je
\beq
\nabla^{2}\mathbf{E}+n\left(x\right)^{2}k_{0}^{2}\mathbf{E}=0,
\label{9.1}
\eeq
kjer je $k_{0}=\omega/c$, $n$ pa nezvezno spremeni vrednost, ko preidemo iz sredice v plašč. 
Rešitev iščemo v obliki 
\begin{equation}
{\mathbf E}(x,z)=\mathbf{e}\psi\left(x\right)\, e^{i\beta z}.
\label{9.2}
\end{equation}
Omejimo se le na primer TE polarizacije (za izračun lastnih rodov TM polariziranega
valovanja glej nalogo~\ref{naloga:TM}). Vstavimo nastavek (enačba~\ref{9.2}) v valovno enačbo
(\ref{9.1}) in dobimo
\begin{equation}
\frac{d^{2}{\bf \psi}}{dx^{2}}+\left(k_{0}^{2}n_1^{2}-\beta^{2}\right){\bf \psi}=0
\qquad \textrm{v sredici oziroma obmo\v cju II} 
\label{9.3}
\end{equation}
in 
\begin{equation}
\frac{d^{2}{\bf \psi}}{dx^{2}}+\left(k_{0}^{2}n_2^{2}-\beta^{2}\right){\bf \psi}=0
\qquad \textrm{v plašču oziroma obmo\v cjih I in III.} 
\label{9.3}
\end{equation}
Ker je po enačbi~(\ref{9.0}) $k_0^2n_1^2-\beta^2=k_x^2$, lahko rešitve prve enačbe
zapišemo v obliki
\beq
\psi_{\mathrm{II}}(x) = C \cos(k_x x)+D \sin(k_x x).
\eeq
Rešitve v plašču pa so oblike
\beq
\psi_{\mathrm{I}}(x) = A \exp(-\kappa x)+B \exp(\kappa x),\\
\psi_{\mathrm{III}}(x) = E \exp(-\kappa x)+F \exp(\kappa x),
\eeq
pri čemer je $\kappa^2= \beta^2-n_2^2k_0^2$.

Če želimo, da je valovanje ujeto v vlakno, mora biti $\kappa$ realno število.
Le tako namreč dosežemo eksponentno pojemanje z oddaljenostjo od sredice,
sicer je valovanje v vseh treh območjih oscilatorno in ni ujeto v vlakno. 
Tako dobimo pogoj za valovno število $\beta$
\boxeq{vlaknobeta}{
k_0n_2 < \beta < k_0 n_1.
}

Zahteva po končnost rešitve da pogoj, da je v območju I (pri $x>a/2$) $B=0$, 
v območju III (pri $x<-a/2$) pa $E=0$. Dodatne omejitve se pojavijo zaradi 
simetrije problema, saj so rešitve lahko le sode ali lihe funkcije. Tako 
dobimo dve vrsti rešitev, sode in lihe:
\begin{align}
\psi_{\mathrm{I}}(x) =&~ A \exp(-\kappa x), \qquad \qquad &\psi_{\mathrm{I}}(x) =&~ A \exp(-\kappa x),\\
\psi_{\mathrm{II}}(x) =&~ C \cos(k_x x), \qquad \qquad &\psi_{\mathrm{II}}(x) =&~ D \sin(k_x x)\\
\psi_{\mathrm{III}}(x) =&~ A \exp(\kappa x), \qquad \qquad &\psi_{\mathrm{III}}(x) =&~ -A \exp(\kappa x).
\end{align}
Zvezo med koeficienti določimo z upoštevanjem robnih pogojev. Na meji
med sredico in plaščem morata biti tangencialni komponenti 
električne in magnetne poljske jakosti zvezni. Iz tega takoj 
izluščimo pogoj, da se za TE valovanje na meji ohranja amplituda električne poljske jakosti.
Drugi pogoj dobimo iz zveze $\nabla\times{\bf E}=i\omega\mu_{0}{\bf H}$, ki izhaja
neposredno iz Maxwellove enačbe~(\ref{eq:Maxwell2}). Ta pogoj zahteva, da se na meji ohranja
odvod električne poljske jakosti $dE/dx$. Tako pogoje za sode in lihe rešitve zapišemo kot
\beq
A \exp(-\kappa a/2) = C \cos(k_x a/2), \qquad \qquad A \exp(-\kappa a/2) = D \sin(k_x a/2).
\eeq
in 
\beq
-A \kappa \exp(-\kappa a/2) = -C k_x \sin(k_x a/2), \qquad
-\kappa A \exp(-\kappa a/2) = D k_x \cos(k_x a/2).
\eeq
Enačbo, ki določa rešitev $k_x$, dobimo iz zahteve, da sta gornja robna pogoja hkrati izpoljnjena. Za sode 
načine tako velja
\boxeq{sekular1}{
\frac{\kappa}{k_x} = \tan \frac{k_x a}{2},
}
za lihe pa 
\boxeq{sekular2}{
-\frac{k_x}{\kappa} = \tan \frac{k_x a}{2}.
}
Pri tem zapišimo še zvezo med $k_x$ in $\kappa$
\boxeq{kappak}{
k_x^{2}+\kappa^{2}=k_{0}^{2}\left(n_{1}^{2}-n_{0}^{2}\right).
}
Sekularnih enačb za lastne načine nihanj ne moremo rešiti analitično. Zato jih rešujemo numerično,
zelo nazorna pa je grafična predstavitev. S slike~(\ref{fig:TEsec}) lahko namreč hitro razberemo
število rešitev in njihove vrednosti. Najprej narišemo desno stran enačb~(\ref{sekular1}) 
in~(\ref{sekular2}), to je $\tan (k_x a/2)$ (črna črta). 
Nato narišemo še levi strani enačb, pri čemer upoštevamo zvezo~(\ref{kappak}), rdeča krivulja naj 
bo za sode rešitve in modra za lihe rešitve. Število presečišč rdeče in modre krivulje s črno
 da število rodov, ki se lahko razširjajo po takem vlaknu. V našem primeru je takih rodov pet:
 trije sodi in dva liha. Z grafa razberemo  še eno pomembno lastnost. Ne glede na to, kako tanek
 je vodnik, vedno bo obstajala vsaj ena rešitev za $k_x$, saj rdeča krivulja vedno nekje seka črno. 
 Vlaknu, v katerem se širi samo eno valovanje, pravimo enorodnovno vlakno, sicer so vlakna večrodovna.
 Za tipično enorodovno vlakno velja $a\lesssim 5~\mu$m, za večrodovno z okoli 20 rodovi pa 
 $a\sim 50~\mu$m.
\begin{figure}[h]
\centering
\def\svgwidth{90truemm} 
\input{slike/10_TEsekularna.pdf_tex}
\caption{K izračunu prečnih komponent valovnega vektorja v planparalelnem valovnem vodniku
za TE polarizacijo. V skiciranem primeru je vodnik petrodoven.}
\label{fig:TEsec}
\end{figure}

Ocenimo število možnih rodov še z izračunom. S slike~(\ref{fig:TEsec}) vidimo, da je največja možna 
vrednost valovnega vektorja $k_x$, pri kateri valovanje še potuje po vlaknu, omejena z vrednostjo, 
pri kateri $\kappa$ pade na nič. Do te vrednosti pa je po ena rešitev na vsakih $\pi/a$. Celotno 
število rodov je tako
\beq
N \approx \frac{k_{x\mathrm{max}}}{\pi/a}  = \frac{k_0 a NA }{\pi},
\eeq
pri čemer smo uporabili zvezo~(\ref{kappak}) ob pogoju, da je $\kappa =0$. 

Ko enkrat izračunamo dovoljene vrednosti $k_x$, končno poznamo celotno električno poljsko
jakost v vodniku in izven njega. Za primer s slike~(\ref{fig:TEsec}) so osnovni načini 
narisani na sliki~(\ref{fig:TESodi}).
\begin{figure}[h]
\centering
\def\svgwidth{70truemm} 
\input{slike/10_TESodi.pdf_tex} 
\quad
\def\svgwidth{70truemm} 
\input{slike/10_TELihi.pdf_tex} 
\caption{Osnovni načini za širjenje svetlobe po valovnem vodniku, levo so sode rešitve, desno pa lihe.}
\label{fig:TESodi}
\end{figure}

\begin{definition}
\label{naloga:TM}
Ponovi izračun za TM valovanje in pokaži, da se sekularni enačbi v primeru TM polarizacije
zapišeta kot 
\beq
\frac{\kappa}{k_x} \left(\frac{n_1}{n_2}\right)^2= 
\tan \frac{k_x a}{2} \qquad \mathrm{in} \qquad -\frac{k_x}{\kappa} \left(\frac{n_2}{n_1}\right)^2= 
\tan \frac{k_x a}{2}.
\eeq
Namig: Zapiši enačbe za magnetno poljsko jakost $H$ in poišči ustrezne robne pogoje.
\end{definition}

\begin{remark}
Če ne prej, je bralec ob slikah~(\ref{fig:TESodi}) zagotovo opazil podobnost s kvantnim delcem, ujetim
v končni enodimenzionalni potencialni jami. Svetloba, ujeta v vlakno, ustreza vezanim stanjem delca,
numerična apertura pa je tisti parameter, ki določa globino potencialne jame. Pri majhnih vrednosti bomo
dobili samo eno rešitev za vezano stanje, pri globlji jami bo rešitev več. Podobno kot v kvantni mehaniki
tudi v tem primeru ena rešitev za vezano stanje vedno obstaja. 
\end{remark}

\section{Cilindrično vlakno}
Do zdaj smo obravnavali ravninski valovni vodnik. V praksi svetlobo
navadno usmerjamo po optičnih vlaknih, ki imajo cilindrično geometrijo.
Najpreprostejša struktura, ki je analogna gornjemu primeru planparalene
plasti, je cilindrično vlakno, pri katerem je lomni količnik cilindričnega
jedra konstanten in nekoliko večji od lomnega količnika plašča. Pogosto se uporablja
tudi bolj zapletene konstrukcije, pri katerih je sredica sestavljena iz več 
kolobarjev z različnimi lomnimi količniki. Zapletenejšo geometrijo izberemo zato, 
da zmanjšamo disperzijo v vlaknu. 

Račun za širjenje svetlobe po cilindričnem vlaknu s homogeno sredico
je podoben kot za planparalelni vodnik, vendar je precej bolj
zapleten. V cilindrični geomteriji namreč ni delitve na čiste električne in 
magnente transverzalne valove in robni pogoji so sklopljeni. Rešitve se izražajo 
v obliki kombinacij Besslovih funkcij, kot bomo obravnavali v razdelku~(\ref{chap:Cilinder}).
Izkaže se, da je osnovni rod, ki se širi po cilindričnem vlaknu, po obliki zelo podoben
Gaussovem snovu, zato je sklopitev laserskih snopov v optična vlakna zelo učinkovita.
Tudi v cilindričnih vlaknih obstaja končno število vodenih valov, odvisno od premera sredice in
razlike lomnih količnikov sredice in plašča. Če sta ti količini majhni,
obstaja le eno vodeno valovanje in imamo enorodovno vlakno. Za njegovo
valovno število velja $n_{0}k_{0}<\beta<n_{1}k_{0}$.

\subsection{Cilindrično vlakno s paraboličnim profilom lomnega količnika}
Čeprav je račun lastnih načinov v cilindričnem vlaknu zapleten, lahko 
razmeroma enostavno poiščemo rešitve za vlakno, v katerem je dielektrična 
konstanta kvadratna funkcija radialne koordinate $r$
\begin{equation}
n^2\left(r\right)=n_{0}^{2}-n_{2}^{2}\frac{r^{2}}{a^2}.
\label{9.15}
\end{equation}
Parameter $n_{2}$ je v praksi vselej majhen, zato ima za vse smiselne
vrednosti $r$ tudi lomni količnik paraboličen profil. Parabolična
sredica je seveda biti omejena, okoli nje je plašč s konstantnim
lomnim količnikom $n_1 \approx n_0-n_2^2/2n_0$ (slika~\ref{fig:GRIN}). Tipičen polmer sredice $a$ je nekaj
deset mikrometrov.
\begin{figure}[h]
\centering
\def\svgwidth{90truemm} 
\input{slike/10_GRIN.pdf_tex} 
\caption{Parabolični profil lomnega količnika sredice zmanjša disperzijo v vlaknu.}
\label{fig:GRIN}
\end{figure}

Komponento polja za izbrano polarizacijo napišimo v obliki 
\begin{equation}
E=E_{0}\psi(x,y)\, e^{i\beta z} e^{-i\omega t}
\label{9.16}
\end{equation}
Zanemarili smo, da zaradi odvisnosti od prečnih koordinat in $\nabla\cdot{\bf E}=0$
polje ne more imeti povsod iste smeri; če hočemo biti natančni, moramo
v gornji obliki zapisati vektorski potencial. Vstavimo približni
nastavek~(enačba~\ref{9.16}) in krajevno odvistnost lomnega količnika~(enačba~\ref{9.15})
v valovno enačbo~(\ref{eq:valovna-skalarna}) in dobimo 
\begin{equation}
\nabla_{\perp}^{2}\psi+\left[k_{0}^{2}\left(n_{0}^{2}-n_{2}^{2}\frac{r^{2}}{a^2}\right)-
\beta^{2}\right]\,\psi=0,
\label{9.17}
\end{equation}
Rešitve lahko zapišemo v obliki
\beq
\psi(x,y) = X(x)Y(y)
\eeq
in dobimo dve neodvisni enačbi
\beq
X'' - \frac{k_0^2 n_2^2}{a^2}\,X\,x^2 - \lambda_1 X = 0 \qquad \mathrm{in} \qquad
Y'' - \frac{k_0^2 n_2^2}{a^2}\,Y\,y^2 - \lambda_2 Y = 0,
\label{eq:XY}
\eeq
pri čemer sta $\lambda_1$ in $\lambda_2$ konstanti. 
Opazimo, da sta enačbi popolnoma enaki enačbama za krajevni del lastnih funkcij 
harmonskega oscilatorja v kvantni mehaniki. Rešitev posamezne enačbe je tako 
produkt Gaussove in Hermitove funkcije
\beq
X_n(x) = e^{-\xi^2 x^2/2} H_n(\xi x),
\label{eq:GH}
\eeq
pri čemer je $\xi = \sqrt{k_0n_2/a}$.
\begin{definition}
Uporabi nastavek~(\ref{eq:GH}) in pokaži, da reši enačbo~(\ref{eq:XY}). Pri tem si pomagaj z 
diferencialno enačbo za Hermitove polinome
\beq
\left( \frac{d^2}{dx^2}-2x\frac{d}{dx}+2n \right) H_n(x) = 0.
\eeq
\end{definition}
Lastne vrednosti enačbe so oblike
\begin{equation}
\beta_{mn}^{2}=n_{0}^{2}k_{0}^{2}\left(1-\frac{2n_{2}}{k_{0}n_{0}^2a}\left(m+n+1\right)\right).
\label{9.19}
\end{equation}
Drugi člen v oklepaju je navadno zelo majhen, zato lahko izraz razvijemo in 
\beq
\beta_{mn}=n_{0}k_{0}\left(1-\frac{n_{2}}{k_{0}n_{0}^2 a}\left(m+n+1\right)\right)
= n_{0}k_{0} - \frac{n_{2} \left(m+n+1\right)}{n_{0} a}.
\eeq
Ob privzetku, da je $n_{2}$ neodvisen od frekvence, je grupna hitrost 
\begin{equation}
v_{g}=\left(\frac{d\beta_{mn}}{d\omega}\right)^{-1}=\frac{c_{0}}{n_{0}}
\label{9.21}
\end{equation}
enaka za vse rodove. To je pomembna značilnost vlakna s kvadratnim profilom
lomnega količnika. V dejanskem vlaknu je seveda taka odvisnost možna
le v omejenem območju sredice, zato je tudi gornja analiza le približna
in velja dobro za tiste rodove, ki se ne raztezajo dosti izven sredice.

Neodvisnost grupne hitrosti od roda je praktično zelo pomembna. 
Grupna hitrost namreč določa čas potovanja svetlobnega sunka, ki
lahko predstavlja en bit informacije. Če se po vlaknu širi več
rodov z različno grupno hitrostjo, se sunek po prehodu skozi
vlakno razširi, kar -- kot bomo podrobneje videli pozneje -- omejuje 
uporabno dolžino vlakna. Temu se sicer lahko izognemo z uporabo enorodovnih vlaken,
ki pa so dražja, poleg tega morata divergenca in polmer svetlobnega snopa 
natančno ustrezati značilnostim enorodovnega vlakna, da se izognemo izgubam. 
Zato se za krajše zveze uporabljajo mnogorodovna vlakna, ki imajo sredico s 
približno paraboličnim profilom lomnega količnika.

\subsection{Rodovi v cilindričnem vlaknu}
\label{chap:Cilinder}
Točen izračun za rodove v cilindričnem vlaknu presega okvire tega učbenika, zato
si oglejmo le izhodiščne enačbe in rešitve\footnote{Točen izračun lahko bralec poišče npr. v Davis, 
{\it Lasers and Electro-optics}.} Za jakost električnega in magnetnega polja velja 
Helmholtzeva enačba~(\ref{eq:Helmholtz})
\beq
\nabla^2 \mathbf{E} + k_0^2 n(r)^2 \mathbf{E} = 0,
\eeq
pri čemer je $n(r<a)=n_1$ lomni količnik sredice in $n(r>a)=n_2$ 
lomni količnik plašča, ki je dovolj debel, da njegova debelina ne 
vpliva na potovanje svetlobe. $\mathbf{E}$ (in $\mathbf{H}$) je v splošnem vektor in ima
tri komponente, ki pa so med seboj odvisne. Izračunajmo naprej $E_z$ z nastavkom
\beq
E_z = R(r)e^{i \nu \phi}e^{i \beta z},
\eeq
pri čemer je $\nu$ celo število zaradi zahteve po enoličnosti rešitve pri spremembi
kota za $2\pi$. Za $R(r)$ dobimo v sredici vlakna enačbo
\beq
r^2 R(r)'' + r R(r)' + (k_s^2r^2 - \nu^2)R(r) = 0,
\eeq
kjer je $k_s^2=k_0^2n_1^2- \beta^2$,
in v plašču
\beq
r^2 R(r)'' + r R(r)' + (-\kappa^2r^2 - \nu^2)R(r) = 0,
\eeq
kjer je $\kappa^2=\beta^2-k_0^2n_2^2$. 
V gornjih enačbah prepoznamo Besslovo differencialno enačbo. 
Upoštevajoč le končne funkcije, dobimo v sredici rešitev
\beq
E_z = J_\nu(k_sr)e^{i \nu \phi}e^{i \beta z}
\eeq
in v plašču
\beq
E_z = K_\nu(\kappa r)e^{i \nu \phi}e^{i \beta z},
\eeq
kjer je $J_\nu(x)$ Besslova funkcija prve vrste reda $\nu$, $K_\nu(x)$ pa modificirana Besslova
funkcija druge vrste reda $\nu$.
\begin{figure}[h]
\centering
\def\svgwidth{70truemm} 
%\input{slike/10_J0.pdf_tex} 
\quad
\def\svgwidth{70truemm} 
%\input{slike/10_J1.pdf_tex} 
\caption{Osnovna rodova širjenja svetlobe po optičnem vlaknu}
\label{fig:J01}
\end{figure}

Tudi tukaj velja zveza $\kappa^2+k_s^2=(NA)^2k_0^2$. Vpeljemo lahko še normirano frekvenco 
$V = NA\, k_0 a$, ki se bo pozneje izkazala za uporabno, saj lahko z njo preprosto izrazimo število
rodov, ki se širijo po vlaknu.

Ko enkrat poznamo komponente $E_z$ in $H_z$, lahko z uporabo Maxwellovih enačb izrazimo še ostale
komponente in z upoštevanjem robnih pogojev izračunamo celotni jakosti električnega in magnetnega
polja. Zapis je zelo zapleten, navadno pa ga poenostavimo s približkom $n_1 \approx n_2$. Podobno
kot pri valovnem vodniku tudi tukaj dobimo sekularno enačbo, ki jo moramo rešiti numerično.

Rešitve so omejene z normiramo frekvenco in ničlami Besslove funkcije. Vlakno je enorodovno, kadar
je $V<2,405$, pri čemer je to numerična vrednost prve ničle Besslove funkcije $J_0$. 

Osnovni rod je linearno polariziran $LP_{01x}$ ali $LP_{01y}$. Je skoraj Gaussov, ga približamo 
po Marcusejevi formuli $exp(-r^2/w^2)$,
kjer je efektivni polmer polja
\beq 
w^2 = (0,65 + 1,619 V^{-3/2}+2,87V^{-6})a.
\eeq

Načini so TE, TM in hibridni;
Hibdrini so HE, kjer j e $E_z$ primerljiv z $E_r, E_\phi$, ali pa EH, kjer je 
$H_z$ primerljiv z $H_r, H_\phi$. Označimo jih z $HE_{\nu l}$ -KAJ JE l? Ali je isti?

$TE_{0l}$ za $\nu=0$ število radialnih oscilacij, je $E_z=E_r=0$, Ephi je J1 = TE val.
AMpak ima ničlo v sredini, torej to ni osnovni val. Osnovni način je hibridni HE11, kjer 
nu ni enak nič. 

TE, Tm in hibridne lahko združimo  v linearno polarizirane valove, LP01 = HE11.
To je edini, ki je J0, in različen od nič za vsako vlakno.  Vsi ostali porebujejo neko
najmanjšo debelino. 

HE so v približamoku degenerirani, zato jih združimo v linearno polariziran
LP0m = HE1m
LP1m = vsota TE0m TM0m, HE2m...
LP11 = TE01+HE21.
LP01 = HE11 vedno dovoljen. 

HE21+Te01 = LP11
He21+TM01 = LP11






% \section{Sprememba lomnega količnika vlakna}
% 
% Sprememba lomnega količnika sredice ali plašča vlakna povzroči spremembo
% valovnega števila $\beta$ danega roda. V enorodovnih vlaknih je to
% mogoče izkoristiti za izdelavo senzorjev, na primer temperature ali
% tlaka. Zaradi zunanjih vplivov, ki jih želimo zaznati, se spremeni
% lomni količnik vlakna in s tem propagacijska konstanta, kar lahko
% izmerimo preko spremembe faze valovanja na izhodu iz vlakna, to je,
% z ustrezno sestavljenim interferometrom. Ker je dolžina vlakna lahko
% velika, v nekaj centimetrov velik tulec lahko brez težav navijemo
% kilometre vlakna, je celotna sprememba faze velika že pri majhnih
% spremembah merjene količine. Sprememba valovnega števila povzroča
% tudi nezaželjene spremembe faze in odboje pri prenosu informacij.
% V tem razdelku zato poglejmo, kako se spremeni valovno število pri
% dani spremembi lomenga količnika.
% 
% Vzemimo rod vlakna s propagacijsko konstanto $\beta_{lm}$ in prečnim
% profilom $\psi_{lm}\left(r,\phi\right).$ Ta mora zadoščati valovni
% enačbi 
% \begin{equation}
% \nabla_{\bot}^{2}\psi_{lm}+\left(\epsilon(r)k_{0}^{2}-\beta_{lm}^{2}\right)\psi_{lm}=0\label{9.22}
% \end{equation}
%  Naj se dielektrična konstanta vlakna spremeni za $\delta\epsilon.$
% Zato se spremenita tudi propagacijska konstanta $\beta=\beta_{lm}+\delta\beta$
% in prečna oblika $\psi=\psi_{lm}+\delta\psi.$ Tudi motena funckija
% $\psi$ mora zadoščati enačbi \ref{9.22}, zato za perturbacijo velja
% \begin{equation}
% \nabla_{\bot}^{2}\delta\psi+\left(\epsilon(r)k_{0}^{2}-\beta_{lm}^{2}\right)\delta\psi+\delta\epsilon\, k_{0}^{2}\psi_{lm}=2\beta_{lm}\delta\beta\,\psi_{lm}\label{9.23}
% \end{equation}
%  kjer smo zanemarili produkte majhnih količin. Množimo obe strani
% enačbe s $\psi_{lm}^{*}$ in integrirajmo po preseku vlakna, pri čemer
% upoštevajmo, da je $\delta\psi$ ortogonalna na $\psi_{lm}$: 
% \begin{eqnarray}
%  &  & \int\psi_{lm}^{*}\nabla_{\bot}^{2}\delta\psi\, dS+\int\left(\epsilon(r)k_{0}^{2}-\beta_{lm}^{2}\right)\delta\psi\,\psi_{lm}^{*}+k_{0}^{2}\int\delta\epsilon\,\left|\psi_{lm}\right|^{2}dS\label{9.24}\\
%  & = & 2\beta_{lm}\,\delta\beta\int\left|\psi_{lm}\right|^{2}dS
% \end{eqnarray}
%  Prvi člen na levi preoblikujmo z uporabo zvez $\int(u\,\nabla_{\bot}^{2}v-v\nabla_{\bot}^{2}u)\, dS=\int\nabla_{\bot}\cdot(u\nabla_{\bot}v-v\nabla_{\bot}u)\, dS=\oint{\bf ds\times(}u\,\nabla_{\bot}v-v\,\nabla_{\bot}u)$.
% Ker fuknciji $\psi_{lm}$ in $\delta\psi$ opisujeta vodene valove,
% morata iti za velike $r$ proti nič, zato je integral po krivulji
% v gornji zvezi nič in velja $\int\psi_{lm}^{*}\nabla_{\bot}^{2}\delta\psi\, dS=\int\delta\psi\nabla_{\bot}^{2}\psi_{lm}^{*}\, dS$.
% Ker $\psi_{lm}^{*}$ zadošča enačbi \ref{9.22}, se v enačbi \ref{9.24}
% prvi in drugi člen uničita in dobimo željeno zvezo 
% \begin{equation}
% \delta\beta=\frac{k_{0}^{2}\int\delta\epsilon\,\left|\psi_{lm}\right|^{2}dS}{2\,\beta\int\left|\psi_{lm}\right|^{2}dS}\label{9.25}
% \end{equation}
%  ki je seveda analogna kvantnomehanskemu rezultatu s teorijo motenj
% prvega reda za spremembo energije lastnega stanja pri majhni sprmembi
% Hamiltonovega operatorja. Rezultat je tudi intuitivno razumljiv: v
% najnižjem redu je $\delta\beta$ sorazmerna s uteženim povprečjem
% $\delta\epsilon$, kjer je utež $\psi_{lm}$.
% 
% Sprememba valovnega števila $\beta$ v delu vlakna, po katerem potuje
% svetloba, ne povzroči le spremembe faze, ampak tudi odboj dela valovanja.
% Ta pojav je le nekoliko druga oblika odboja na meji (zvezni ali ostri)
% dveh dielektrikov, ali splošneje, odboja vsakega valovanja na območju,
% kjer se spremeni fazna hitrost valovanja.
% 
% Odboj na območju vlakna, kjer se spreminja $\beta$, bomo najlažje
% dobili preko formule za odboj na meji dveh dielektrikov pri pravokotnem
% vpadu. Odbita amplituda je tedaj 
% \begin{equation}
% E_{r}=\frac{n_{2}-n_{1}}{n_{2}+n_{1}}E_{0}\label{9.26}
% \end{equation}
%  Mislimo si, da je sprememba $\beta$ na delu vlakna sestavljena iz
% majhnih stopničastih sprememb $\Delta\beta_{i}$ na intervalih $\Delta\dot{z}$.
% Za ravno valovanje, za katerega velja enačba \ref{9.26}, je sprememba
% lomnega količnika sorazmerna s spremembo fazne hitrosti. Ker tudi
% $\beta$ določa fazno hitrost valovanja po vlaknu, je delež odbitega
% valovanja na stopničasti spremembi $\Delta\beta_{i}$
% \begin{equation}
% \Delta E_{i}=\frac{\Delta\beta_{i}}{2\,\beta}\, E_{0}\label{9.27}
% \end{equation}
%  Predpostavili smo, da je celotni del odbitega valovanja tako majhen,
% da ni treba upoštevati spremembe amplitude vpadnega vala $E_{0}$.
% Vse odbito valovanje je vsota prispevkov na posameznih stopnicah $\Delta\beta_{i}$,
% pri čemer moramo upoštevati še različne faze delnih odbitih valovanj:
% \begin{equation}
% E_{r}=\sum\frac{\Delta\beta_{i}}{2\,\beta}\, e^{2i\beta z_{i}}\, E_{0}=\frac{1}{2\,\beta}\sum\frac{d\beta}{dz}\, e^{2i\beta z_{i}}\Delta z\, E_{0}\label{9.28}
% \end{equation}
%  Preidimo z vsote na integral, pa dobimo, da je odbita amplituda 
% \begin{equation}
% E_{r}=\frac{E_{0}}{2\,\beta}\,\int\frac{d\beta}{dz}\, e^{2i\beta z}dz\label{9.29}
% \end{equation}
% 
% 
% \textit{Primer:} Naj se valovno število linearno spermeni za $\Delta\beta_{0}$
% na razdalji $L$. Tedaj je po gornji formuli 
% \begin{equation}
% \frac{E_{r}}{E_{0}}=2\,\frac{\Delta\beta_{0}}{L}\,\frac{\sin\beta L/2}{\beta}\label{9.30}
% \end{equation}
%  Odbojnost je največja, kadar je $L$ majhen v primeri z $1/\beta$,
% torej kadar je sprememba $\beta$ ostra stopnica. Čim počasnejša je
% sprememba, tem manj je odboja, poleg tega pa odboja ni vsakič, ko
% je $\sin\beta L/2=0$, to je, pride do destruktivne interference vseh
% delnih odbojev.(Naloga: Odboj na erf stopnici).


\section{Izgube v optičnih vlaknih}
Ena najpomembnejših lastnosti optičnih valken je majhno slabljenje
svetlobnega vala, posebej še v vlaknih iz kremenovega stekla. Najboljša
vlakna imajo danes izgube okoli 0,2 db/km pri valovni dolžini 1,55
$\mu$m. (1 decibel je$\,0,1\,$log$(j_{0}/j).$). Glavni vzroki izgub
so absorpcija in sipanje na nečistočah in sipanje na termičnih fluktuacijah
gostote (Rayleighovo sipanje). Slika \ref{sl9.5} prikazuje tipično
odvisnost izgub od valovne dolžine za dobro enorodovno vlakno iz kremenovega
stekla s primesjo GeO$_{2}.$ Sipanje na fluktuacijah gostote je sorazmerno
z $\lambda^{-4}$, zato dominira pri majhnih valovnih dolžinah, sipanje
na defektih in nepravilnostih je skoraj zanemarljivo, pri valovnih
dolžinah nad 1,6~$\mu$m pa prevlada absorpcija. Vrh med 1,3~$\mu$m
in 1,5~$\mu$m je posledica absorpcije na OH ionih, ki se jih je
v steklu zelo težko znebiti. Iz slike je razvidno, da so izgube najmanjše
okoli 1,55~$\mu$m, zato se to območje največ uporablja za zveze
na velike razdalje. Najboljših modernih vlaken tudi ni več mogoče
kaj dosti izboljšati glede izgub, saj so že dosegla spodnjo mejo,
določeno s termičnimi fluktuacijami.

Pri optičnih zvezah nastanejo še izgube na spojih vlaken, ki so okoli
0,2 db na spoj. Skupne izgube so tako dosti manjše kot v koaksialnem
kablu in je možen prenos signala do nekaj sto kilometrov brez vmesnega
ojačevanja. S tem pri dolžini optičnih zvez izgube niso več glavna
omejitev, ampak je to popačitev signala zaradi disperzije.

V optičnem vlaknu nastanejo izgube tudi, kadar je vlakno ukrivljeno.
Te izgube postanejo znantne, kadar je krivinski radij vlakna centimeter
ali manj. Ta pojav je dovolj zanimiv, da si ga je vredno nekoliko
podrobneje ogledati.

Vzemimo spet dvodimenzionalno plast debeline $2a$ z lomnim količnikom
$n_{1}$ v sredstvu z lomnim količnikom $n_{0},$ ki naj bo ukrivljena
s krivinskim radijem $R$. Taka plast tvori del kolobarja z notranjim
radijem $R-a$ in zunanjim radijem $R+a$, pri čemer je $\dot{R}>>a$
(Slika \ref{sl9.6}). Zapišimo valovno enačbo za dve dimenziji v cilindrični
geometriji: 
\begin{equation}
\frac{1}{r}\,\frac{\partial}{\partial r}\, r\,\frac{\partial E}{\partial r}+\frac{1}{r^{2}}\,\frac{\partial^{2}E}{\partial\phi^{2}}+k_{0}^{2}n^{2}\left(r\right)\, E=0\label{9.31}
\end{equation}
 kjer ima $n\left(r\right)$ vrednost $n_{1}$ v sredici in $n_{0}$
drugje. Zanimajo nas rešitve oblike 
\begin{equation}
E=\psi\left(r\right)\, e^{im\phi}\label{9.32}
\end{equation}
 kjer je $\psi\left(r\right)$ znatna le v sredici. Ker je valovna
dolžina svetlobe dosti manjša od $R$, je $m$ veliko število, ki
je povezano z valovnim številom $\beta$: naj bo $z=R\phi$ dolžina
loka vzdolž sredine sredice. Tedaj je $m\phi=(m/R)\, z$ in je torej
$\beta=m/R$. $\psi$ zadošča enačbi 
\begin{equation}
\frac{d^{2}\psi}{dr^{2}}+\frac{1}{r}\,\frac{d\psi}{dr}+\left[k_{0}^{2}\, n^{2}\left(r\right)-\frac{m^{2}}{r^{2}}\right]\psi=0\label{9.33}
\end{equation}
 Rešitve za $\psi$ so kombinacije Besslovih funkcij reda $m$, kar
pa zaradi velikosti $m$ ni posebno zanimivo. Dosti več bomo izvedeli,
če primerno preoblikujemo valovno enačbo \ref{9.31}. Namesto $r$
in $\phi$ vpeljimo koordinati $x=r-R$ in $z=R\phi$. S tem smo prešli
nazaj na koordinate planparelelne plasti in iščemo popravke valovne
enacbe \ref{9.3} reda $1/R.$ Tako je $1/r\simeq1/R$ in 
\begin{equation}
\frac{m^{2}}{r^{2}}=\frac{m^{2}}{\left(R+x\right)^{2}}\simeq\frac{m^{2}}{R^{2}}\,\left(1-2\,\frac{x}{R}\right)=\beta^{2}\left(1-2\,\frac{x}{R}\right)\label{9.34}
\end{equation}
 S tem dobimo iz enačbe \ref{9.33} približno enačbo za prečno obliko
polja 
\begin{equation}
\frac{d^{2}\psi}{dx^{2}}+\left[k_{0}^{2}\, n^{2}\left(r\right)-\beta^{2}\right]\,\psi+\frac{1}{R}\,\left(\frac{d\psi}{dr}-2\,\beta^{2}x\,\psi\right)=0\label{9.35}
\end{equation}

\section{Disperzija}
\label{chap:Disperzija}
Zaradi disperzije, to je odvisnosti fazne in grupne hitrosti od frekvence,
se sunek svetlobe, ki potuje po vlaknu, podaljšuje. Ta pojav omejuje
količino informacije, ki jo je mogoče prenašati po vlaknu dane dolžine.
Zato je čim manjša disperzija v vlaknih vsaj toliko pomembna kot majhne
izgube.

Vzemimo najprej enorodovno vlakno in naj bo svetloba v vlaknu modulirana
v obliki kratkih sunkov, ki nosijo informacijo. Sunki potujejo z grupno
hitrostjo 
\begin{equation}
v_{g}=\frac{d\omega}{d\beta}=\left(\frac{d\beta}{d\omega}\right)^{-1}\label{9.51}
\end{equation}
 Po enačbi \ref{9.0} je grupna hitrost odvisna od frekvence tako
zaradi eksplicitne korenske zveze med $\omega$ in $\beta$ kot zaradi
odvisnosti lomnih količnikov sredice in plašča od frekvence. Prvemu
prispevku recimo valovodna disperzija, ker je posledica omejitve valovanja
v sredico vlakna, drugemu pa materialna disperzija. Oba prispevka
sta pomembna.

Naj bo dolžina vlakna $L$ in trajanje posameznega sunka $\tau$.
Sunek potuje po vlaknu čas 
\begin{equation}
T=\frac{L}{v_{g}}=L\frac{d\beta}{d\omega}\label{9.52}
\end{equation}
 Svetloba ima končno spektralno širino $\Delta\omega$. Ta mora biti
vsaj 1/$\tau,$ lahko pa je tudi večja, če svetlobni izvor, običajno
polvodniški laser, ni povsem monokromatski. Ker vse spektralne komponente
ne potujejo z isto hitrostjo, se sunek na koncu vlakna razleze za
\begin{equation}
\Delta\tau=\left|\frac{dT}{d\omega}\right|\Delta\omega=L\,\frac{d^{2}\beta}{d\omega^{2}}\,\Delta\omega\label{9.53}
\end{equation}
 Naj bo tudi razmik me zaporednimi sunki $\tau.$ Da se sunki ne bodo
prekrivali, mora biti $\Delta\tau<\tau$ in sme biti najvišja frekvenca
modulacije kvečjemu 
\begin{equation}
v_{\max}=\frac{1}{2\Delta\tau}\label{9.54}
\end{equation}

Ločiti moramo dva mejna primera. Kadar je $\Delta\omega>>1/\tau$,
to je, kadar je spektralna širina izvora mnogo večja od širine zaradi
modulacije, je 
\begin{equation}
v_{\max}=\frac{1}{L\,\Delta\omega}\left(\frac{d^{2}\beta}{d\omega^{2}}\right)^{-1}\label{9.55}
\end{equation}
 Največja gostota informacije, ki jo lahko prenašamo po vlaknu, je
v tem primeru obratno sorazmerna z dolžino vlakna in spektralno širino
laserja.

V nasprotnem primeru, ko je laser dosti bolj monokromatski od razširitve
zaradi same modulacije, imamo $\Delta\omega=1/\tau$. Pri najvišji
možni frekvenci modulacije $v_{\max}$ je $\tau\simeq\Delta\tau$
in je po enačbi \ref{9.53} 
\begin{equation}
(\Delta\tau)^{2}=L\,\frac{d^{2}\beta}{d\omega^{2}}\,\label{9.56}
\end{equation}
 in je 
\begin{equation}
v_{\max}=\sqrt{\frac{1}{L\,}\left(\frac{d^{2}\beta}{d\omega^{2}}\right)^{-1}}\label{9.57}
\end{equation}
 V tem primeru je najvišja frekvenca modulacije obratno sorazmerna
s korenom dolžine vlakna.

Večji del dicperzije grupne hitrosti prinese materialna disperzija,
to je, odvisnost lomnega količnika od frekvence. Primer meritve razširitve
sunka z znano spektralno širino v izbrani dolžini vlakna kaže slika
\ref{sl9.7}. Pri valovni dolžini okoli 1,3 $\mu$m ima materialna
disperzija, to je $d^{2}n/d\omega^{2},$ ničlo, ki je v celotni disperziji
nekoliko premaknjena zaradi prispevka valovodne disperzije. Na valovodno
disperzijo je mogoče vplivati s konstrucijo vlakna. Sredica je lahko
sestavljena iz več plasti z različnimi lomnimi količniki in različnimi
debelinami, s čimer se spremeni prispevek valovodne disperzije in
se položaj ničle celotne disperzije premakne k valovni dolžini izvora.

Iz slike \ref{sl9.7} razberemo, da je tipična disperzija enorodovnega
vlakna okoli 10~ps/km pri spektralni širini 1~nm. Od tod je 
$d^{2}\beta/d\omega^{2}=\Delta\tau\lambda/\left(\Delta\lambda L\omega\right)\simeq5\cdot10^{-23}$
s$^{2}/$m. Po enačbi \ref{9.57} je v 100 km dolgem vlaknu tedaj
najvišja frekvenca modulacije okoli $10^{9}$ Hz. Pri zmogljivih zvezah
na velike razdalje je torej za največjo dolžino vlakna brez obnovitve
signala disperzija hujaša omejitev kot izgube. Največja možna razdalja
in najvišje frekvenca modulacije sta danes nekaj sto kilometrov in
nekaj GHz.

V mnogorodvnih vlaknih se sunek širi zaradi razli\textquotedbl{}nih
grupnih hitrosti posameznih rodov. Različne grupne hitrosti v valovni
sliki ustrezajo različni dolžini optične poti za žarke, ki potujejo
pod različnimi koti glede na os vlakna. Te razlike so mnogo večje
od disperzije v enorodnih vlaknih, zato je pri dani dolžini vlakna
popačitev signala dosti večja in se mnogorodovna vlakna ne uporabljajo
za dolge zveze.

\section{Potovanje sunka po enorodovnem vlaknu}
Poglejmo si nekoliko podrobneje, kako po enorodovnem vlaknu ali drugem
sredstvu z disperzijo potuje sunek valovanja z dano začetno obliko.
Denimo, da smo poiskali lastna valovanja in da torej poznamo zvezo
med valovnim številom $\beta$ in frekvenco $\omega$. Zapišimo sunek
v obliki 
\begin{equation}
E\left(z,t\right)=a\left(z,t\right)\,\psi\left(x,y\right)\label{9.61}
\end{equation}
 kjer je $\psi\left(x,y\right)$ lastna rešitev prečenega dela valovne
enačbe, ki določa tudi $\beta\left(\omega\right)$. Funkcijo $a\left(z,t\right)$,
ki opisuje širjenje sunka in njegovo obliko v $z$ smeri, lahko zapišemo
s Fourierovim integralom po frekvencah 
\begin{equation}
a\left(z,t\right)=\int a(\omega,z)\, e^{-i\omega t}d\omega\label{9.62}
\end{equation}
 Sunek naj bo približno monokromatičen s frekvenco $\omega_{0}$,
to pomeni, da je mnogo daljši od optične periode. Pri določeni $\omega$
ima Fourierova amplituda krajevno odvisnost $\exp[i\,\beta\left(\omega\right)\, z]$,
zato zadošča enačbi 
\begin{equation}
\frac{\partial a\left(z,\omega\right)}{\partial z}=i\,\beta\left(\omega\right)\, a\left(z,\omega\right)\label{9.63}
\end{equation}
 Privzeli smo, da je spekter sunka ozek, zato lahko $\beta\left(\omega\right)$
razvijemo okoli $\omega_{0}$: 
\begin{equation}
\frac{\partial a\left(z,\omega\right)}{\partial z}=i\,\left[\beta\left(\omega_{0}\right)+\frac{d\beta}{d\omega}\,\left(\omega-\omega_{0}\right)+\frac{1}{2}\,\frac{d^{2}\beta}{d\omega^{2}}\,\left(\omega-\omega_{0}\right)^{2}\right]\, a\label{9.64}
\end{equation}
 Vpeljimo novo amplitudno funkcijo, ki ne bo vsebovala osnovne odvisnosti
$\exp[i\beta\left(\omega_{0}\right)z]$ 
\begin{equation}
a\left(z,\omega\right)=A\left(z,\omega-\omega_{0}\right)\, e^{i\,\beta\left(\omega_{0}\right)\, z}\label{9.65}
\end{equation}
 Ker je spkter različen od nič le okoli $\omega_{0}$, je prikladno
$A$ pisati kot funkcijo $\omega-\omega_{0}$. Napravimo obratno Fourierovo
transformacijo zadnjega izraza: 
\begin{eqnarray}
a\left(z,t\right) & = & \int_{-\infty}^{\infty}A\left(z,\omega-\omega_{0}\right)\, e^{i\,[\beta\left(\omega_{0}\right)\, z-\omega\,\, t]}\, d\omega\label{9.66}\\
 & = & \int_{-\infty}^{\infty}A\left(z,\omega-\omega_{0}\right)\, e^{-i\left(\omega-\omega_{0}\right)\, t}d\left(\omega-\omega_{0}\right)\,\, e^{i\,[\beta\left(\omega_{0}\right)\, z-\omega\,_{0}\, t]}\nonumber \\
 & = & A\left(z,t\right)\,\, e^{i\,[\beta\left(\omega_{0}\right)\, z-\omega\,_{0}\, t]}\nonumber 
\end{eqnarray}
 Funkcija $A\left(z,t\right)$, katere Fourierova transformacija je
$A\left(z,\omega\right)$, očitno predstavlja prostorsko in časovno
odvisnost ovojnice sunka. Postavimo definicijo \ref{9.65} v enačbo
\ref{9.64}: 
\begin{equation}
\frac{\partial A\left(z,\omega-\omega_{0}\right)}{\partial z}=i\,\left[\frac{d\beta}{d\omega}\,\left(\omega-\omega_{0}\right)+\frac{1}{2}\,\frac{d^{2}\beta}{d\omega^{2}}\,\left(\omega-\omega_{0}\right)^{2}\right]\, A\left(z,\omega-\omega_{0}\right)\label{9.67}
\end{equation}
 Z obratno Fourierovo transformacijo dobimo 
\begin{equation}
\left(\frac{\partial}{\partial z}+\frac{1}{v_{g}}\,\frac{\partial}{\partial t}\right)\, A\left(z,t\right)=-\frac{i}{2}\,\frac{d^{2}\beta}{d\omega^{2}}\,\frac{\partial^{2}A\left(z,t\right)}{\partial t^{2}}\label{9.68}
\end{equation}
 Upoštevali smo, da je 
\begin{equation}
\int\left(i\omega\right)^{n}A\left(z,\omega\right)\, e^{-i\omega t}\, d\omega=\frac{\partial^{n}}{\partial t^{n}}A\left(z,t\right)\label{9.69}
\end{equation}
 in da je $d\beta/d\omega=1/v_{g}$.

Enačba \ref{9.68} opisuje razvoj oblike sunka pri širjenju po vlaknu.
Če ni disperzije grupne hitrosti, to je, če je desna stran enačbe
nič, je rešitev poljubna funkcija $f\left(z-v_{g}t\right)$. Sunek
poljubne začetne oblike potuje po vlaknu nepopačen z grupno hitrostjo.
Disperzija pa povzroči, da se spreminja tudi oblika. Enačbo lahko
še nekoliko poenostavimo z vpeljavo novih neodvisnih spremenljivk
\begin{eqnarray}
\tau & = & t-\frac{z}{v_{g}}\nonumber \\
\zeta & = & z\label{9.70}
\end{eqnarray}
 Za vrh sunka, ki naj ima pri $t=0$ koordinato $z=0$ in se giblje
z grupno hitrostjo, je vselej $\tau=0$. Spremenljivka predstavlja
$\tau$ torej čas v točki $z=\zeta$ , merjen od trenutka, ko tja
prispe center sunka. Z novima spremenljivkama se enačba \ref{9.68}
zapiše 
\begin{equation}
\frac{d^{2}\beta}{d\omega^{2}}\,\frac{\partial^{2}A}{\partial\tau^{2}}+2\, i\,\frac{\partial A}{\partial\zeta}=0\label{9.71}
\end{equation}
 Ta enačba ima isto obliko kot obosna valovna enačba, ki smo jo v
drugem poglavju uporabili za obravnavo koherentnih snopov. Podobnost
seže dlje od formalne oblike. Pri snopih, ki so omejeni v prečni smeri,
disperzija fazne in grupne hitrosti po prečnih komponentah valovnega
vektorja povzroča spreminjanje prečnega preseka snopa, pri časovno
omejenih sunkih v sredstvu s frekvenčno disperzijo pa se spreminja
vzdolžna oblika sunka. Kot se morda bralec spominja, je tudi v praznem
prostoru pri širjenju snopa v okolici grla fazna hitrost funkcija
frekvence. Zato se kratek sunek, ki je omejen v prečni smeri, tudi
v praznem prostoru razširi tako v prečni kot v vzdolžni smeri. (Naloga)

Obosno valovno enačbo rešijo Gaussovi snopi. V en. \ref{9.71} ima
vlogo prečne koordinate $\tau.$ Po analogiji s snopi se bo zaradi
disperzije najmanj širil sunek z Gaussovo časovno odvisnostjo. Računa
nam ni treba ponavljati, kar v izrazu za Gaussove snope napravimo
ustrezno zamenjavo črk. Valovnemu številu $k$ pri snopih na primer
ustreza parameter $\mu=(d^{2}\beta/d\omega^{2})^{-1}$. Tako dobimo
\begin{equation}
A\left(\tau,\zeta\right)=\frac{A_{0}}{\sigma_{0}\sqrt{1+\frac{\zeta^{2}}{\zeta_{0}^{2}}}}\exp\left(-\frac{\tau^{2}}{\sigma^{2}}\right)\exp\left(-i\frac{\mu\tau^{2}}{2b}\right)e^{i\phi\left(\zeta\right)}\label{9.72}
\end{equation}
 kjer je $\sigma$ trajanje sunka, za katerega velja enaka zveza kot
za polmer Gaussovega snopa: 
\begin{equation}
\sigma^{2}=\sigma_{0}^{2}\left(1+\frac{\zeta^{2}}{\zeta_{0}^{2}}\right)\label{9.73}
\end{equation}
 Tu je $\sigma_{0}$ trajanje sunka pri $\zeta=0$, to je na mestu,
kjer je sunek najkrajši. Dodatna skupna faza $\phi\left(\zeta\right)$
ni posebno pomembna, pač pa je zanimiv drugi eksponentni faktor v
enačbi \ref{9.72}. V njem smo z $b=\zeta\left(1+\zeta_{0}^{2}/\zeta^{2}\right)$
označili količino, ki je analogna krivinskemu radiju valovnih front
v primeru Gaussovih snopov. Odvod faze po $\tau$ predstavlja spremembo
frekvence glede na centralno frekvenco sunka $\omega_{0}$: 
\begin{equation}
\omega-\omega_{0}=\frac{\mu\tau}{b}\label{9.74}
\end{equation}
 Za pozitivno disperzijo $\mu$ je frekvenca na prednji strani sunka,
to je pri $\tau<0$, večja in se linearno zmanjšuje proti koncu sunka.
Pri $\zeta=0$ je sunek toliko kratek, kolikor je možno pri dani spektralni
širini. Pri potovanju po vlaknu se zaradi disperzije sunek razširi,
spektralna širina pa ostaja enaka, zato se je del pojavi kot spreminjanje
frekvence znotraj sunka. Lahko si mislimo tudi, da je sunek najkrajši,
to je omejen z Fourierovo transformacijo spektra, tedaj, kadar se
vse frekvenčne komponente seštejejo z isto fazo, to je pri $\zeta=0$.
Da dobimo najkrajše sunke, kadar je faza vseh delnih valov enaka,
smo srečali že pri fazno uklenjenih sunkih iz mnogofrekvenčnih laserjev.
Pri potovanju sunka se zaradi disperzije faze frekvenčnih komponent
različno spreminjajo in sunek se podaljša. Zanimivo je, da je pri
tem pomemben šele drugi odvod fazne hitrosti po frekvenci, ki je sorazmeren
z $\mu$, linearno spreminjanje faze pa ne povzroči razširitve.

Naloga: Pokaži, da je spekter sunka nespremenjen.

Naloga: Pokaži, da je za sunek poljubne začetne oblike razširitev
mogoče zapisati z uklonskim integralom.

Naloga: Pokaži, da iz en. \ref{9.73} sledi podobna ocena za maksimalno
frekvenco modulacije (minimalno razširitev sunka) pri dani dolžini
vlakna, kot jo da en. \ref{9.57}.

Razširitev sunka zaradi disperzije je pri $\dot{\mu}>0$ mogoče kompenzirati
s parom paralelnih uklonskih mrežic, kot kaže slika \ref{sl9.8}.
Prva mrežica različne frekvenčna komponente razkloni, druga pa zopet
zbere, vendar dolžine optičnih poti za različne komponente niso enake.
celoten učinek je enak kot pri razširjanju sunka po sredstvu z negativno
disperzijo. Račun je nekoliko preglednejši, rezultat pa povsem enak,
če namesto refleksijskih mrežic vzamemo transmisijski, kot kaže slika
\ref{sl9.9}. Naj na par vpada raven val pod kotom $\alpha.$ Pred
prvo mre\textquotedbl{}ico je fazni faktor $\exp\left(ik_{1}x\right),$
kjer je $k_{1}=\omega/c\,\sin\alpha$. Pri prehodu skozi mrežico se
polje pomnoži s kompleksno prepustnostjo mrežice, ki povzroči razcep
vala na uklonjene valove. Pri tem se faza za prvi uklonski red poveča
za $qx$, kje je $q=2\pi/\Lambda$ in je $\Lambda\,$perioda mrežice.
Premik do druge mrežice poveča fazo za $k_{3}L$. Za komponento valovnega
vektorja v smeri $z$ velja seveda $k_{3}=\sqrt{\left(\omega/c\right)^{2}-\left(k_{1}+q\right)^{2}}$.
Po prehodu skozi drugo mrežico nas zanima prvi negativni uklonski
red, ki da val v smeri prvotnega vala. Za ta red se faza spremeni
za $-qx$, tako da je celotna sprememba faze 
\begin{equation}
\Phi=L\sqrt{\left(\omega/c\right)^{2}-\left(k_{1}+q\right)^{2}}=
\frac{L}{c}\sqrt{\omega^{2}-\left(\omega\,\sin\alpha+qc\right)^{2}}\label{9.75}
\end{equation}
 Disperzija, ki jo povzroči par mrežic, je določena zdrugim odvodom
faze po frekvenci: 
\begin{equation}
\frac{d^{2}\Phi}{d\omega^{2}}=-\frac{L\, q}{\left[\omega^{2}-\left(\omega\,\sin\alpha+q\, c\right)^{2}\right]^{3/2}}\label{9.76}
\end{equation}
 Drugi odvod je vselej negativen. Par mrežic torej deluje kot sredstvo
z negativno disperzijo. Sunek, ki se je razširil zaradi potovanja
po sredstvu s pozitivno disperzijo, lahko ponovno skrajšamo do meje,
določene s širino spektra. Postopek se uporablja za pridobivanje zelo
kratkih sunkov. Sunku iz fazno uklenjenega barvilnega ali Ti:safirnega
laserja najprej v nelinearnem sredstvu razširijo spekter, pri čemer
se sunek tudi časovno podaljša. O tem najde bralec nekaj več v poglavju
o nelinearni optiki. Razširjen sunek nato s parom mrežic skrajšajo
za faktor 10-100 glede na prvotno dolžino sunka. Tako dobijo sunke
dolge le okoli 10 fs, kar je le še nekaj optičnih period.

\section{Sklopitev v optična vlakna}
