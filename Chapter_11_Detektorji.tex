%\chapterimage{CCD.jpg} % Chapter heading image

\chapter{Detektorji svetlobe}

Spoznali bomo detektorje, ki so nujni za kvantitativno obravnavo
optičnih pojavov in sprejemanje optičnih signalov. Podrobneje bomo spoznali
načine delovanja in specifikacije posameznih vrst detektorjev.\index{Detektor}
V grobem delimo detektorje v dve skupini, na termične\index{Detektor!termični} in 
kvantne\index{Detektor!kvantni}. Največ pozornosti bomo posvetili posebni vrsti kvantnih 
detektorjev, to je polprevodniškim detektorjem, ki so danes najbolj razširjena vrsta detektorjev.
Na koncu bomo spoznali še šum pri detekciji. Šum namreč omejuje najmanjši 
signal, ki ga z detektorjem še lahko izmerimo.

\section{Osnovne karakteristike detektorjev}
Osnovna naloga optičnih detektorjev je spremeniti vpadni svetlobni signal 
v nek drug signal, ki ga lahko natančno merimo. Navadno sta to električni tok 
ali električna napetost, ki sta sorazmerna z močjo vpadne svetlobe 
in ne z amplitudo jakosti električnega polja. Pri navadni detekciji se tako
podatek o fazi valovanja izgubi. 

Termični detektorji delujejo na osnovi povišanja temperature senzorja 
zaradi absorbirane svetlobe. Taki detektorji zaznavajo energijo 
vpadne svetlobe. Njihov odziv je razmeroma počasen, zato jih uporabljamo
predvsem za merjenje optične moči, ki je lahko tudi zelo velika. 
Po drugi strani je odziv termičnih detektorjev neodvisen
od valovne dolžine vpadne svetlobe, zaradi česar so termični detektorji uporabni na 
širokem območju od globokega ultravijoličnega do daljnega infrardečega valovanja
(slika~\ref{fig:shemaTermKv}). 
\index{Ultravijolično valovanje}\index{Infrardeče valovanje}Uporaba
prevlada predvsem v infrardečem, teraherčnem ali celo mikrovalovnem delu spektra, kjer so 
drugi detektorji bistveno manj občutljivi. \index{Teraherčno valovanje}
Primeri termičnih detektorjev so bolometer, termočlen in piroelektrični detektor.
\index{Bolometer}\index{Termočlen}\index{Piroelektrični detektor}

Druga skupina so kvantni detektorji, ki temeljijo na fotoefektu\index{Fotoefekt}.
V njih se vpadni fotoni absorbirajo in povzročijo pojav prostih nosilcev naboja, 
tako da iz snovi izbijejo elektron ali v snovi ustvarijo par elektron--vrzel. Kvantni 
detektorji zaznavajo število vpadnih fotonov. Odlikuje jih zelo hiter odziv 
(tipično pod $\si{\micro\second}$)
in velika občutljivost. Njihova poglavitna slabost je omejen obseg valovnih dolžin,
pri katerih zaznavajo svetlobo, poleg tega jih je za optimalno delovanje treba 
hladiti. Primeri so vakuumske, polprevodniške in plazovne fotodiode.
\index{Fotodioda!vakuumska}\index{Fotodioda!polprevodniška}\index{Fotodioda!plazovna}
\newpage
\begin{figure}[ht]
\centering
\def\svgwidth{48truemm} 
\input{slike/11_ShemaTermKv.pdf_tex}
\caption{Primerjava spektralnega odziva termičnega in kvantnega detektorja}
\label{fig:shemaTermKv}
\vglue-3truemm
\end{figure}

Osnovne karakteristike, ki omogočajo primerjavo detektorjev in določajo njihovo uporabnost,
so občutljivost, spektralni odziv, odzivni čas in najmanjši merljivi signal:
\index{Detektor!občutljivost}
\index{Detektor!spektralni odziv}
\index{Detektor!odzivni čas}
\index{Detektor!najmanjši merljivi signal}

\begin{enumerate}
\item Občutljivost $\mathcal{R}$ pove, koliko je izhodnega signala 
na enoto vpadnega svetlobnega toka. Enota za občutljivost je $\si{A/W}$, če merimo tok, ali 
$\si{V/W}$, če na izhodu zaznavamo napetost. 
\item Spektralni odziv pove, kako se občutljivost spreminja z valovno dolžino $\mathcal{R}(\lambda)$.
Pri termičnih detektorjih je $\mathcal{R}(\lambda)$ konstanta\footnote{Občutljivost je konstantna
le v območju konstantnega albeda, ki pa je lahko zelo široko.}, medtem ko kvantni detektorji 
delujejo le v določenem območju valovnih dolžin, ki je odvisno od snovi, 
iz katere je detektor narejen. 
\item Odzivni čas pove, kako hitro se detektor odzove na spremembo optičnega signala. Predvsem 
optične telekomunikacije zahtevajo izredno hiter odziv.
\item Najmanjši merljivi signal je določen s svetlobno močjo, pri kateri postane razmerje med 
signalom in šumom ({\it Signal to Noise Ratio}) $SNR = 1$. 
\index{Razmerje signal proti šumu}
\end{enumerate}

\section{Termični detektorji}
\index{Detektor!termični}
Termične detektorje se zaradi njihovega razmeroma počasnega odziva uporablja predvsem 
za merjenje vpadne moči in za detekcijo svetlobe s tistimi valovnimi dolžinami, za katero 
ni drugih preprostih ali učinkovitih detektorjev. Zelo so uporabni v daljnem
infrardečem območju ali za detekcijo velikih vpadnih moči.\footnote{Glej npr. G. H. Rieke, {\it Detection 
of light}, druga izdaja, Cambridge University Press (2002).}

Delovanje termičnih detektorjev temelji na spremembi temperature zaradi absorpcije svetlobe. 
Detektorji se med seboj razlikujejo predvsem v načinu pretvorbe spremembe 
temperature v električni signal.\index{Absorpcija}
Tipalo termičnih detektorjev mora biti pri vseh vrstah dobro počrnjeno, da absorbira
svetlobo v čim širšem spektralnem območju. Čeprav je njihova občutljivost načeloma 
neodvisna od valovne dolžine vpadne svetlobe, se v praksi pojavijo omejitve zaradi
prepustnosti okna in absorpcijskega spektra črnega nanosa. Tipala so majhna, zato 
da dosežemo čim hitrejši odziv, ki je kljub temu navadno počasnejši od $1~\si{ms}$. 
Sodobnejši detektorji se po odzivnem času že približujejo 
kvantnim, saj dosegajo odzivne čase do $\sim 10~\si{\micro\second}$. 
Termične detektorje navadno uporabljamo pri sobni temperaturi,  
za zahtevne meritve pa jih hladimo na nekaj $\si{K}$. 

Obravnavajmo termični detektor, katerega tipalo naj ima toplotno kapaciteto $C$. Toplota
se s tipala odvaja v toplotni zalogovnik s temperaturo $T_0$, 
toplotne izgube označimo z $\Lambda$. Ko na tipalo vpada svetloba z močjo $P$, 
začne temperatura tipala $T$ zaradi absorpcije svetlobe naraščati, hkrati se tipalo 
ohlaja zaradi odvajanja toplote. Zapišemo
\begin{equation}
\frac{dW}{dt} = C \frac{dT}{dt} = P - \Lambda (T-T_0).
\label{TD1}
\end{equation}
V stacionarnem stanju (ob konstantnem vpadnem svetlobnem toku) se
temperatura tipala ne spreminja in razlika temperature tipala in zalogovnika je 
\begin{equation}
T - T_0 = \frac{P}{\Lambda}.
\label{temp_sens}
\end{equation}
Občutljivost detektorja je sorazmerna z razliko temperatur in
obratno sorazmerna s toplotnimi izgubami. Za večjo občutljivost moramo
toplotne izgube kar se da zmanjšati.
Po enačbi~(\ref{TD1}) se temperatura približuje stacionarni vrednosti s časovno konstanto 
\begin{equation}
\tau = \frac{C}{\Lambda}.
\label{TermD_t}
\end{equation}
Odzivni čas je torej sorazmeren s kapaciteto senzorja, zato so tipala praviloma 
zelo majhna.\index{Detektor!odzivni čas}

Iz enačbe~(\ref{TermD_t}) sledi, da moramo za dosego čim krajšega odzivnega časa toplotne izgube 
kar se da povečati. Velike izgube sicer skrajšajo odzivni čas, 
vendar tudi zmanjšajo občutljivost (enačba~\ref{temp_sens}), 
zato termični detektorji ne morejo imeti hkrati velikega in hitrega odziva. 
Če želimo toplotne izgube povečati in s tem skrajšati odzivni čas, detektorje hladimo z zrakom 
ali celo z vodo, navzdol so toplotne izgube omejene s sevanjem.  

Poglejmo odziv termičnega detektorja na vpadno svetlobo. Naj se  moč vpadne svetlobe
spreminja s časom, temperatura na detektorju pa temu sledi z določeno zakasnitvijo. Odziv
najlepše izračunamo v Fourierevem prostoru. Vpadno moč in temperaturo izrazimo kot
\begin{equation}
P(t) = \int_{-\infty}^{\infty} P_\omega e^{i\omega t}d\omega \qquad \mathrm{in} \qquad
T(t) = T_0 + \int_{-\infty}^{\infty} T_\omega e^{i\omega t}d\omega.
\label{TermTF}
\end{equation}
To vstavimo v enačbo~(\ref{TD1}) in zapišemo
\begin{equation}
\int_{-\infty}^{\infty} i \omega T_\omega e^{i\omega t}d\omega = \frac{1}{C}
\int_{-\infty}^{\infty} (P_\omega - \Lambda T_\omega) e^{i\omega t}d\omega.
\end{equation}
Enačbi zadostimo, tako da izenačimo člene pred vsako spektralno komponento posebej
\begin{equation}
i \omega T_\omega = \frac{1}{C}\left(P_\omega - \Lambda T_\omega\right).
\end{equation}
Če vstavimo odzivni čas $\tau$ (enačba~\ref{TermD_t}), \index{Detektor!odzivni čas}dobimo
\begin{equation}
T_\omega = \frac{1}{\Lambda}\left(\frac{1}{1+i \omega \tau}\right)P_\omega = 
\frac{P_\omega}{\Lambda}\left(\frac{1-i \omega \tau}{1+(\omega \tau)^2}\right).
\label{TermOdziv}
\end{equation}
Imaginarni del predstavlja fazni zamik, amplituda odziva, ki nas zanima, je 
podana kot realni del. Odziv je velik le za počasne spremembe signala, ko
je $\omega$ majhen in je $T_\omega \approx P_\omega/\Lambda$. Z naraščajočo frekvenco
$\omega$ se odziv zmanjšuje in hitrim spremembam detektor ne more slediti. 
\begin{naloga}
Pokaži, da je odziv termičnega detektorja na sunek oblike $P(t) = P_0\delta(t-t_0)$
\begin{equation}
T(t)-T_0=\frac{P_0}{\tau\Lambda}e^{-(t-t_0)/\tau}.
\end{equation}
\end{naloga}

\subsection*{Bolometer}
Bolometer je termični detektor\index{Bolometer}, pri katerem zaznavamo spremembo 
električne upornosti
zaradi spremembe temperature tipala.\footnote{Prvi bolometer je leta 1881 naredil
ameriški fizik, astronom in letalski inženir Samuel Pierpont Langley, 1834--1906.}
Tipalo je praviloma počrnjena tanka ploščica, 
navadno je narejena iz termistorja\footnote{Termistor je upornik, katerega
upornost se spreminja s temperaturo.}, \index{Termistor} polprevodnika ali superprevodnika 
(slika~\ref{fig:Bolometer-shema}). Tipalo prek
referenčnega upora priključimo na napetost, prek kondenzatorja pa merimo napetost na njem.
Za meritve konstantnega svetlobnega toka tipalo navadno vežemo v Wheatstonov mostiček.\footnote{Glej
npr. J. Strnad, {\it Fizika, 2. del}, osmi natis, DMFA--založništvo (2018).} V obeh
primerih za referenčni upor vzamemo enako tipalo, ki ga zaščitimo pred vpadno svetlobo, 
tako da postane sistem neobčutljiv na morebitne spremembe temperature okolice.
\begin{figure}[ht]
\centering
\def\svgwidth{120truemm} 
\input{slike/11_bolometer.pdf_tex}
\caption{Shema bolometra (a). Počrnjena plast se zaradi vpadne svetlobe segreje, pri čemer
je sprememba temperature sorazmerna z močjo vpadne svetlobe. S hlajenjem skrajšamo odzivni čas
detektorja. Fotografija bolometra za merjenje prasevanja (b). 
Premer kovanca za primerjavo je 18~mm. 
Vir: NASA/JPL-Caltech.}
\label{fig:Bolometer-shema}
\end{figure}

Termistorji imajo približno eksponentno odvisnost upornosti od temperature, zato imajo bolometri
s termistorjem razmeroma veliko občutljivost. Občutljivost pri neki tipični
velikosti upora $R$ znaša $\mathcal{R}\sim 100~\si{\volt/\watt}$. Poleg tega so 
robustni, stabilni in delujejo pri sobni temperaturi. 

Odzivni časi bolometrov so okoli 
$\tau \sim 1$--$20~\si{\milli\second}$. 
Pri polprevodniških bolometrih upornost pojema eksponentno s temperaturo. 
Primerni so za detekcijo teraherčnih valovanj,\index{Teraherčno valovanje} 
vendar mora biti za ta namen bolometer (npr. germanijev) hlajen s tekočim helijem.\footnote{F. J. 
Low, J. Opt. Soc. Am. $\mathbf{51}$, 1300 (1961).}  
Tako lahko dosežemo občutljivosti večje od $\mathcal{R} \sim 10^8~\si{\volt/\watt}$. Zelo 
občutljivi so tudi detektorji s superprevodnimi tipali, saj je odvisnost upornosti 
od temperature v bližini prehoda v superprevodno stanje zelo velika 
($\mathcal{R} \sim 10^3~\si{\volt/\watt}$).

\subsection*{Termočlen}
Termočlen je sestavljen iz dveh različnih kovinskih vodnikov\index{Termočlen}. 
En spoj vodnikov počrnimo, drugega, 
referenčnega, zaščitimo pred svetlobo. Zaradi vpadne svetlobe se počrnjeni spoj 
segreje, med obema spojema se pojavi temperaturna razlika in zaradi termoelektričnega 
pojava električna napetost, ki jo lahko merimo. Električna prevodnost
vodnikov mora biti čim večja, njihova toplotna prevodnost pa čim manjša. Odzivni čas termočlenov je 
$\tau \sim 10$--$20~\si{\milli\second}$ in občutljivost $\mathcal{R} \sim 10~\si{\volt/\watt}$.
Ker so napetosti, ki se pojavijo med stikoma, razmeroma majhne 
($\sim 100~\si{\micro\volt/K}$) pogosto vežemo več (nekaj deset) termočlenov zaporedno v
termobaterijo. Občutljivost s tem naraste na $\mathcal{R} \sim 200~\si{\volt/\watt}$, medtem ko se 
podaljša časovna konstanta na $\tau \sim 10$--$2000~\si{\milli\second}$. Prednost termočlenov je,
da za svoje delovanje ne potrebujejo zunanjega napajanja. 

\subsection*{Piroelektrični detektor}
Piroelektriki so snovi brez centra inverzije\index{Piroelektrični detektor}, 
v katerih je lastna električna 
polarizacija odvisna od temperature (npr. LiTaO$_3$,\index{LiTaO$_3$} 
triglicin sulfat TGS\index{TGS} in vsi feroelektriki). Piroelektrični detektor je narejen iz 
ploščice piroelektrične snovi med dvema elektrodama oziroma ploščama kondenzatorja.
Ko se ploščica zaradi absorbirane svetlobe segreje, se ji spremeni polarizacija. Med 
elektrodama se pojavi premikalni tok, ki ga merimo na merilnem uporniku.

Zveza med spremembo temperature in spremembo električne polarizacije $P_\mathrm{el}$ je
\begin{equation}
dP_\mathrm{el} = a\, dT,
\end{equation}
kjer smo vpeljali $a$ kot piroelektrični koeficient. Njegova vrednost za npr. kristal 
LiTaO$_3$\index{LiTaO$_3$} je $a = 2,3 \times 10^{-4}~\si{\ampere \second /\metre^2 \kelvin}$.

Med dvema elektrodama s površino $S$ zaradi spremembe temperature steče naboj
\begin{equation}
de = I\, dt = S\, dP_\mathrm{el} = S a\, dT.
\end{equation}
Tok, ki steče skozi tipalo, je tako
\begin{equation}
I = S a \frac{dT}{dt}.
\label{piro}
\end{equation}
Piroelektrični detektor je torej občutljiv na časovni odvod temperature detektorja in s tem 
tudi na spreminjanje vpadne svetlobne moči. V stacionarnem stanju 
detektor ne proizvaja električnega toka, zato moramo za merjenje 
konstantnega svetlobnega toka vpadno svetlobo periodično modulirati.
Navadno to naredimo kar z mehanskim zaklopom. Piroelektrični detektorji
se večinoma uporabljajo kot preprosti infrardeči detektorji.\index{Infrardeče valovanje}
Njihova občutljivost je $\mathcal{R} \sim 1~\si{\micro\ampere/\watt}$, odzivni čas pa je odvisen od 
upornika v vezju, ampak lahko doseže vrednosti $\tau \sim 10~\si{\micro\second}$.

Poglejmo temperaturni odziv na tipalu. Izhajamo iz enačb~(\ref{TermTF}) in (\ref{piro})
in zapišemo
\begin{equation}
I = Sa \frac{dT}{dt} = Sa \frac{d}{dt} \int_{-\infty}^{\infty} T_\omega e^{i\omega t}d\omega.
\end{equation}
Upoštevamo še enačbo (\ref{TermOdziv}) in dobimo
\begin{equation}
I=Sa\int_{-\infty}^{\infty}\frac{1}{\Lambda}\left(\frac{P_\omega}{1+i \omega \tau}\right) \,i \omega\,
e^{i\omega t}d\omega.
\end{equation}
Od tu razberemo tok $I$ v odvisnosti od krožne frekvence modulacije
\begin{equation}
I_\omega = \frac{i \omega\, SaP_\omega/\Lambda}{1 + i \omega \tau} = 
\frac{SaP_\omega}{\Lambda}\frac{(\omega^2\tau + i \omega)}{1+(\omega\tau)^2}.
\end{equation}
Pri majhnih frekvencah tok narašča parabolično z naraščajočo frekvenci,
pri velikih frekvencah pa postane neodvisen od
frekvence modulacije vpadne svetlobe. Vendar to še ne pomeni, da lahko moduliramo s poljubno 
veliko frekvenco. Poleg relaksacijskega časa detektorja ima namreč karakteristični čas tudi
elektronsko vezje, ki določa zgornjo mejo za frekvenco modulacije. 
Ta je enak $\tau_e = RC_e$, pri čemer
sta $R$ upornost sistema in $C_e$ električna kapaciteta detektorja. Dejanska občutljivost detektorja, 
ki je sorazmerna z realnim delom električnega toka, je prikazana na sliki~\ref{fig:Piro}. 

\begin{figure}[ht]
\centering
\def\svgwidth{70truemm} 
\input{slike/11_Piro.pdf_tex}
\caption{Spektralni odziv piroelektričnega detektorja navzdol omejujejo toplotne izgube 
$\Lambda$ in toplotna kapaciteta detektorja $C$, navzgor pa odzivni čas elektronskega vezja $\tau_e$.}
\label{fig:Piro}
\vglue-3truemm
\end{figure}

\section{Fotoefekt}
Delovanje kvantnih detektorjev temelji na fotoefektu\index{Fotoefekt}. 
To je pojav, pri katerem vpadni\index{Detektor!kvantni}\index{Energijska reža}
fotoni iz snovi izbijajo elektrone. Izbiti elektroni lahko ubežijo kot prosti elektroni
(t.\ i.\ zunanji fotoefekt)\index{Fotoefekt!zunanji},
ali ostanejo ujeti v snovi -- a mobilni -- in tako povečajo 
prevodnost snovi (notranji fotoefekt)\index{Fotoefekt!notranji}. 
V obeh primerih pride do fotoefekta le, 
če je energija vpadnih fotonov večja od neke določene energije.
Pod to vrednostjo fotoefekta ni, ne glede na moč vpadne svetlobe.
Fotoefekt je prvič opazil Hertz\footnote{Nemški fizik Heinrich Rudolf Hertz, 1857--1894.} 
leta 1887, za njegovo razlago leta
1905 pa je Einstein\footnote{Nemški fizik in nobelovec Albert Einstein, 1879--1955.} 
dobil Nobelovo nagrado. 

Poglejmo najprej zunanji fotoefekt, pri katerem elektron postane povsem prost. 
Da se to sploh lahko zgodi, mora biti energija vpadnega fotona dovolj velika, da 
elektron premaga potencialno bariero in izstopi iz prevodnega pasu (slika~\ref{fig:Nivoji}\,a). 
Najmanjšo energijo, ki je za to potrebna, imenujemo v kovinah izstopno delo $\Phi$. 
Če je energija fotona večja, gre preostanek energije v kinetično energijo izbitega
elektrona.\index{Izstopno delo}

\begin{figure}[ht]
\centering
\def\svgwidth{128truemm} 
\input{slike/11_Nivoji.pdf_tex}
\caption{Shema energijskih pasov in zunanjega fotoefekta v kovini (a) in polprevodniku (b) ter
notranjega fotoefekta v polprevodniku (c). $\Phi$ označuje izstopno delo, $E_g$ širino reže 
med valenčnim in prevodnim pasom polprevodnika in $E_a$ elektronsko afiniteto. }
\label{fig:Nivoji}
\vglue-3truemm
\end{figure}

Zunanji fotoefekt poteka tudi v polprevodnikih (slika~\ref{fig:Nivoji}\,b),
v katerih foton izbije elektron iz valenčnega pasu. Da lahko elektron zapusti snov,
mora biti njegova energija večja od vsote energije reže in elektronske afinitete.
Z uporabo ustreznih materialov lahko dosežemo negativno elektronsko afiniteto
in je zato potrebna energija fotona kar enaka širini energijske 
reže.\footnote{Glej npr. C. R. Pollock, {\it Fundamentals of Optoelectronics}, Irwin (1995).}

Izstopno delo za kovine $\Phi$ je od okoli $2~\si{eV}$ za cezij in do okoli 
$6~\si{eV}$ za platino. 
Ustrezna valovna dolžina svetlobe, ki še povzroči fotoefekt, je 
\boxeq{eq:fefekt}{
\lambda \leq \frac{hc}{\Phi}.
}
V ceziju tako povzroči fotoefekt svetloba z valovno dolžino, krajšo od $580~\si{nm}$, in
v platini svetloba z $\lambda < 200~\si{nm}$. 
Da lahko fotoefekt izkoristimo za detektorje vidne svetlobe, 
uporabimo druge snovi, na primer Cs-Te, Cs-Sb, Na-K-Sb-Cs ali GaAs:Cs. 
Tako lahko zaznavamo fotone z valovnimi dolžinami od ultravijoličnega
vse do bližnjega infrardečega valovanja.\index{Ultravijolično valovanje}
\index{Infrardeče valovanje}

Pri notranjem fotoefektu (slika~\ref{fig:Nivoji}\,c) elektron 
snovi ne zapusti, ampak zgolj preide iz enega 
energijskega pasu v drugega. Tipično to poteka v polprevodnikih, kjer absorpcija fotona 
povzroči nastanek para elektron--vrzel, prag za nastanek para pa določa širina reže med 
energijskima pasovoma. 

Primeri detektorjev, ki temeljijo na zunanjem fotoefektu, so 
fotocelice in foto\-pomnoževalke, medtem ko na notranjem fotoefektu temeljijo na primer
fotoprevodniki, polprevodniške in plazovne fotodiode.\index{Fotodioda!vakuumska}
\index{Fotopomnoževalka}\index{Fotoprevodnik}\index{Fotodioda!polprevodniška}
\index{Fotodioda!plazovna}

Za zdaj smo napisali, da fotoefekt poteče, ko foton izbije elektron. Vendar pri tem ni 
uspešen prav vsak foton, zato vpeljemo še parameter, ki ga imenujemo 
\index{Kvantni izkoristek}kvantni izkoristek $\eta$.
Ta parameter pove verjetnost, da vpadni foton z valovno dolžino $\lambda$ 
oziroma frekvenco $\nu$ iz snovi izbije elektron. 
Električni tok, ki steče pri vpadni svetlobni moči $P$, je tako:
\boxeq{11:eta}{
I = \eta e_0 \frac{n_F}{t} = \eta \frac{e_0 P}{h \nu},
}
pri čemer $n_F$ označuje število vpadnih fotonov v času $t$.

Kvantni izkoristek je močno odvisen od valovne dolžine vpadne svetlobe in seveda
od snovi, na katero svetloba vpada. Za fotone z energijo, ki je manjša od izstopnega 
dela oziroma od širine energijske reže, je kvantni izkoristek praktično enak nič, 
nato strmo naraste in lahko doseže vrednosti, večje od $90~\%$. Podrobneje ga bomo 
obravnavali pri posameznih primerih detektorjev.
\vglue-3truemm
\begin{remark}
V praksi ločimo dve vrsti kvantnega izkoristka: zunanjega in notranjega. Zunanjega vpeljemo kot 
razmerje med številom izbitih elektronov in fotonov, ki vpadejo na detektor. Ker se 
ob vpadu na detektor vedno nekaj fotonov odbije ali siplje, vpeljemo še notranji kvantni 
izkoristek kot razmerje števila elektronov in fotonov, ki se dejansko absorbirajo v detektorju.
Zunanji izkoristek je vedno manjši od notranjega in je neke vrste efektivni 
izkoristek.\index{Kvantni izkoristek!notranji}\index{Kvantni izkoristek!zunanji}
\end{remark}
\vglue-3truemm
Iz enačbe~(\ref{11:eta}) hitro izračunamo še občutljivost detektorja 
\index{Detektor!občutljivost}
\boxeq{11:R}{
\mathcal{R} = \frac{I}{P} = \frac{\eta e_0 }{h \nu}.
}

\section{Vakuumska fotodioda (fotocelica) in fotopomnoževalka}
\subsection*{Fotocelica}
\index{Fotocelica|see {Fotodioda, vakuumska}}\index{Fotodioda!vakuumska}
Najpreprostejši kvantni detektor na zunanji fotoefekt je fotocelica ali vakuumska fotodioda
(slika~\ref{fig:Fotoefekt}).\footnote{~Glej npr. G. H. Rieke, {\it Detection 
of light}, druga izdaja, Cambridge University Press (2002).} 
Fotocelica deluje tako, da svetloba vpada na katodo, 
zaprto v vakuumirani stekleni bučki, in tam povzroči fotoefekt. Izbiti elektroni 
se pod zunanjo napetostjo $V$ pospešijo do anode in z ampermetrom ($A$) 
merimo električni tok, ki steče med katodo in anodo. 
Ker je tok sorazmeren s številom vpadnih fotonov, lahko na ta 
način izmerimo moč vpadne svetlobe.\index{Fotoefekt!zunanji}
\vglue-3truemm
\begin{figure}[ht]
\centering
\def\svgwidth{65truemm} 
\input{slike/11_Fotoefekt.pdf_tex}
\caption{Shema fotocelice, v kateri poteka fotoefekt. 
Vpadna svetloba iz kovinske katode izbije elektrone, zaradi česar med katodo 
in anodo steče tok.}
\label{fig:Fotoefekt}
\end{figure}

Območje detekcije fotocelice je določeno z izstopnim delom kovine, iz 
katere fotoni izbijajo elektrone. Potrebno energijo fotona lahko precej zmanjšamo, 
če namesto čistih kovin uporabimo bi- ali večalkalne katode 
(npr. Na$_2$KSbCs), ali polprevodnike, na katere nanesemo tanko plast 
Cs ali Cs$_2$O. To omogoča zaznavanje svetlobe do valovnih dolžin okoli $1600~\si{\nano\metre}$. 
Na ultravijoličnem območju je delovanje
omejeno na okoli $160~\si{\nano\metre}$ zaradi neprepustnosti stekla, 
iz katerega je narejena bučka 
(slika~\ref{fig:Fotodioda}).\index{Infrardeče valovanje}\index{Ultravijolično valovanje}
\footnote{~Slika povzeta po {\it Photomultiplier Tubes, Basics and Applications}, tretja izdaja, Hamamatsu
Photonics (2007).}
\begin{figure}[ht]
\centering
\def\svgwidth{100truemm} 
\input{slike/11_SpekterKatode.pdf_tex}
\caption{Kvantni izkoristek fotocelic za različne snovi\index{InGaAs}\index{GaAsP}}
\label{fig:Fotodioda}
\vglue-3truemm
\end{figure}

Odzivni čas vakuumske fotodiode je odvisen od časa preleta elektronov od katode do anode. 
Da je ta čas čim krajši, je napetost na fotocelici velika, pogosto več $\si{kV}$,
kar omogoča zelo kratke odzivne čase, tudi do $0,1~\si{ns}$. 
Enostavnost in hitrost sta torej prednosti fotocelice, \index{Detektor!odzivni čas}
njena glavna pomanjkljivost je razmeroma nizek kvantni izkoristek. 
Izkoristek je seveda močno odvisen od valovne dolžine vpadnega valovanja in snovi, iz 
katere je narejena katoda. Največje vrednosti, ki jih dosega, so okoli $40~\%$, vendar
pogosto bistveno manj . 
Vrednosti so razmeroma nizke, saj se izbiti elektroni gibljejo v vse
smeri in se pogosto sipljejo, preden sploh dosežejo površino katode. 

Dodaten problem fotocelic je, da pri končnih temperaturah prihaja do spontane oddaje elektronov.
Nekaj električnega toka zato teče tudi v popolni temi. To je tako imenovani temni 
tok\index{Temni tok} 
in tipično dosega vrednosti okoli $10^{-15}~\si{\ampere}$, 
lahko pa tudi do več $\si{nA}$. 
Za občutljive meritve je treba zato vakuumsko fotodiodo hladiti. 

\begin{naloga}
Izračunaj občutljivost fotocelice na osnovi GaAs\index{GaAs} 
za valovanje z valovno dolžino $\lambda=620$~nm.
Kvantni izkoristek odčitaj s slike~\ref{fig:Fotodioda}.
\end{naloga}

\subsection*{Fotopomnoževalka}
\index{Fotopomnoževalka}
Fotopomnoževalke so fotocelice z vgrajenim ojačenjem. Ojačenje dosežemo tako, da 
izbit elektron najprej pospešimo z napetostjo $100$--$150~\si{\volt}$ na vmesno elektrodo, 
tako imenovano dinodo, iz katere izbije več ($\sim 5$--$10$, redkeje tudi do 40) 
sekundarnih elektronov (slika~\ref{fig:PMT}). 
Izbiti elektroni \index{Fotoefekt!zunanji}\index{Notranje ojačenje}
potujejo do naslednje dinode, ki je pod višjo pozitivno napetostjo (tipično okoli $100~\si{\volt}$
višjo), kjer ponovno izbijejo elektrone, ki vpadejo na naslednjo dinodo, 
ki je pod še višjo napetostjo ... 
Pomnoževanje se večkrat ponovi (navadno okoli desetkrat),
število elektronov eksponentno narašča in na en vpadni foton lahko na anodo vpade $10^9$ elektronov. 
Občutljivost fotopomnoževalk je  precej večja od občutljivosti vakuumske fotodiode in
dosega odzivnost na anodi do $\mathcal{R}\sim 10^6~\si{\ampere/\watt}$.
Fotopomnoževalka tako omogoča štetje posameznih fotonov, po drugi strani moramo pri 
navadnih osvetlitvah paziti, da fotopomnoževalke ne osvetlimo preveč. 
\vglue-2truemm
\begin{figure}[ht]
\centering
\def\svgwidth{70truemm} 
\input{slike/11_PMT.pdf_tex}
\caption{Shema fotopomnoževalke. Vpadna svetloba iz katode izbije elektrone, ti 
iz dinod izbijajo dodatne elektrone. Signal na izhodu je zato močno ojačen. }
\label{fig:PMT}
\vglue-1truemm
\end{figure}
Fotopomnoževalke imajo zelo kratek odzivni čas, ki je odvisen od postavitve dinod. Posamezni 
elektroni do anode potujejo različno dolgo, zato je sunek na izhodu 
razširjen, tipično okoli $\sim 0,1$--$20~\si{\nano\second}$.  
Za manj zahtevne aplikacije pogosto merimo kar povprečni tok z anode. Kadar opazujemo
posamezne fotone, zaznamo na izhodu zaporedje sunkov. Takrat lahko 
amplituda izhodnega signala močno niha, saj je koeficient ojačenja 
odvisen od števila izbitih elektronov, kar je statistični proces. 

\section{Fotoprevodni detektorji}
Fotoprevodni detektorji\footnote{~Fotoprevodne detektorje včasih imenujemo tudi fotouporniki.} 
so detektorji, ki temeljijo na notranjem fotoefektu.\index{Fotoprevodnik}
\index{Fotoupornik|see {Fotoprevodnik}}\index{Fotoefekt!notranji}
Vpadni foton z dovolj veliko energijo se absorbira, vendar ne izbije elektrona v prostor, 
ampak ga iz valenčnega pasu dvigne v prevodnega. Pri tem nastane par elektron--vrzel. 
Ob priključeni napetosti se nosilci naboja začnejo premikati in steče tok, 
ki ga merimo. Z naraščajočim številom fotonov se prevodnost fotoprevodnika veča, 
zato lahko z merjenjem upornosti določimo 
intenziteto vpadne svetlobe. Tipično so fotoprevodniki iz polprevodnikov, 
lahko pa so tudi iz izolatorjev.\footnote{~Glej npr. 
B. E. A. Saleh in M. C. Teich, 
{\it Fundamentals of Photonics}, druga izdaja, John Wiley \& Sons, Inc. (2007).}

Da foton vzbudi elektron iz valenčnega v prevodni pas, mora imeti dovolj veliko energijo. 
V nedopiranih polprevodnikih to pomeni, da mora biti energija fotona večja od 
širine reže. Za silicij\index{Silicij}, na primer, je širina reže $1,1$~eV in največja
valovna dolžina, ki jo še lahko zaznavamo, je okoli 
$1,1~\si{\micro\meter}$. V germaniju\index{Germanij} je reža $0,67$~eV (do $1,8~\si{\micro\meter}$) in v 
PbS $0,37$~eV \index{PbS}(do $3,4~\si{\micro\meter}$). 

Za detekcijo daljših valovnih dolžin uporabljamo dopirane 
polprevodnike (glej razdelek~\ref{chap:SCL} 
in sliko~\ref{fig:FPrevodnik}). 
Z dodatnim energijskim nivojem med valenčnim in prevodnim pasom občutno zmanjšamo 
potrebno energijo vpadnih fotonov. Vendar je pri  nizkih energijah prispevek termično 
vzbujenih elektronov že tako velik, da je treba detektorje hladiti, navadno s tekočim
dušikom ali celo tekočim helijem. Primer fotoprevodnih detektorjev je germanij, dopiran s cinkom, 
s katerim lahko zaznavamo svetlobo valovnih dolžin do okoli $40~\si{\micro\meter}$. Pri tem ga hladimo
na $4~\si{\kelvin}$, da zmanjšamo pojav termično vzbujenih nosilcev naboja. \index{Infrardeče
valovanje}
\begin{figure}[ht]
\centering
\def\svgwidth{128truemm} 
\input{slike/11_FPrevodnik.pdf_tex}
\caption{Shema prehoda elektrona v fotoprevodniku: prehod v čistem polprevodniku (a), 
polprevodniku tipa $n$ (b) in polprevodniku tipa $p$ (c). 
Z dopiranjem razširimo območje delovanja detektorja v infrardeč del spektra. }
\label{fig:FPrevodnik}
\end{figure}

Izračunajmo električni tok, ki steče skozi fotoprevodnik, ko nanj posvetimo. 
Gostota električnega toka $j$ je enaka vsoti prispevkov elektronov  in 
vrzeli
\begin{equation}
j = e_0 n_v v_v + e_0 n_e v_e,
\end{equation}
pri čemer $n_v$ in $n_e$ pomenita gostoto vrzeli in elektronov v snovi, $v_v$ in $v_e$ pa 
njihove hitrosti. Hitrost premikanja 
je sorazmerna z jakostjo električnega polja $E$, ki je priključeno
 na vzorec, sorazmernostni faktor je gibljivost $\beta$. Ko posvetimo na vzorec, 
 se $n_v$ in $n_e$ povečata za $\Delta n_v$ in $\Delta n_e$, zato
gostota električnega toka naraste za
\begin{equation}
\Delta j = e_0 \Delta n_v v_v + e_0 \Delta n_e v_e.
\label{FP_j}
\end{equation}
V stacionarnem stanju se število nosilcev naboja ohranja. Za gostoto vrzeli velja 
(enačba~\ref{11:eta})
\begin{equation}
0 = \frac{dn_v}{dt} = \frac{\eta_v P}{h \nu Sl} - \frac{\Delta n_v}{\tau_v}.
\end{equation}
Pri tem je $\eta_v$ kvantni izkoristek, $P$ moč vpadne svetlobe, $h\nu$ energija fotona,
$Sl$ prostornina detektorja in $\tau_v$ življenjski čas vrzeli. Enako enačbo 
zapišemo tudi za gostoto elektronovo $n_e$, le oznake $v$ zamenjamo za $e$. 

Ko stacionarno vrednost $\Delta n_v$ in $\Delta n_e$ vstavimo v enačbo~(\ref{FP_j}), dobimo
\begin{equation}
\Delta j = e_0 \frac{\eta_v P \tau_v}{h \nu Sl} \beta_v  E + 
e_0 \frac{\eta_e P \tau_e}{h \nu Sl} \beta_e  E.
\end{equation}
Pri tem smo hitrost nosilcev naboja izrazili kot produkt gibljivosti $\beta$ in jakosti
električnega polja $E$. Če vpeljemo še napetost $U = E\,l$, zapišemo povečanje toka $\Delta I$
skozi fotoprevodnik zaradi vpadle svetlobe kot
\begin{equation}
\Delta I = S\Delta j = \frac{e_0 U P }{h \nu l^2} \left(\eta_v \tau_v \beta_v + 
\eta_e \tau_e \beta_e \right).
\end{equation}
Pogosto je gibljivost elektronov znatno večja od gibljivosti vrzeli (npr.
$0,135~\si{\meter}^2/\si{\volt\second}$ proti $0,048~\si{\meter}^2/\si{\volt\second}$ za
silicij),\footnote{~Glej npr. G. H. Rieke, {\it Detection 
of light}, druga izdaja, Cambridge University Press (2002).} 
\index{Silicij} zato prvi člen v oklepaju zanemarimo in zapišemo
\boxeq{eq:fup}{
\Delta I = G \frac{e_0 \eta_e}{h\nu} P,
}
pri čemer je koeficient ojačenja 
\begin{equation}
G = \frac{\beta_e \tau_e U}{l^2} = \frac{\tau_e}{\tau}.
\end{equation}
Vpeljali smo še čas preleta $\tau = l/v_e = l/\beta_e E = l^2/\beta_e U$.

Koeficient $G$, ki opiše ojačenje signala, je torej enak razmerju med 
življenjskim časom elektronov in časom preleta.\index{Notranje ojačenje} 
Njegova vrednost je odvisna od 
vrste snovi in gibljivosti nosilcev naboja v njej, velikosti
detektorja in tudi priključene napetosti, zato lahko $G$ zavzame vrednosti od manj kot ena pa
vse do $10^6$. 

% \begin{naloga}
% Izračunali smo spremembo toka, če fotoprevodnik osvetlimo s konstantno vpadno močjo. Pokaži, da
%  je v primeru periodično spremenljive moči odziv enak
%  \begin{equation}
% \Delta I_\omega = G \left( \frac{e_0 \eta_e}{h\nu}\right) \frac{P_\omega}{1+i \omega \tau_e}.
% %  \end{equation}
%  
% \end{naloga}

Fotoprevodniki so uporabni na širokem spektralnem območju, od ultra\-vijoličnega
do daljnega infra\-rdečega valovanja.\index{Ultravijolično valovanje}\index{Infrardeče valovanje} 
V vidnem in bližnjem infrardečem delu spektra se 
uporablja pretežno silicijeve fotoprevodnike, germanijeve\index{Silicij}\index{Germanij}
pa za valovne dolžine do $1,8~\si{\micro\meter}$. Za zaznavanje valovnih dolžin med okoli 
$2~\si{\micro\meter}$ in $7~\si{\micro\meter}$ so najprimernejši InAs, InSb in PbS detektorji, \index{InAs}\index{InSb}\index{PbS}
pri še daljših valovnih dolžinah se uporablja germanij, dopiran z zlatom, bakrom, cinkom, borom ...
Kvantni izkoristek takih detektorjev je razmeroma velik ($\eta = 0,5$ za Ge:Cu), vendar
je lahko faktor ojačenja $G \ll 1$ (npr. $G = 0,03$ za Ge:Hg). 

Hitrost odziva fotoprevodnika je odvisna od časa preleta nosilcev naboja $\tau$,
ki je določen z geometrijo detektorja in priključene napetosti, 
in od karakterističnega časa elektronskega vezja. 
Tipični odzivni časi so okoli mikrosekunde, vendar lahko sežejo
tudi do desetin milisekund, ali v izjemnih primerih do nanosekund za zelo majhne detektorje.
S skrajšanjem rekombinacijskega časa lahko sicer skrajšamo odzivni čas detektorja, 
vendar hkrati zmanjšamo njegovo občutljivost.

\begin{remark}
Fotoprevodni detektorji so narejeni iz zelo tankih plasti fotoprevodnika, saj fotoprevodnik močno absorbira
svetlobo. Tako za absorpcijo $70$--$90\%$ vpadne svetlobe zadošča le $1$--$2~\si{\micro\meter}$ debela plast.
Elektrode se pogosto prepletajo, da se zmanjša dolžina preleta $l$ in s tem poveča faktor ojačenja signala $G$. 
\end{remark}

\section{Polprevodniške fotodiode}
Drugi primer detektorjev, ki temeljijo na notranjem fotoefektu,\index{Fotoefekt!notranji}
so polprevodniške fotodiode.\index{Polprevodniška fotodioda|see {Fotodioda, 
polprevodniška}}\index{Fotodioda!polprevodniška}\footnote{~Glej npr. 
G. H. Rieke, {\it Detection of light}, druga izdaja, Cambridge University Press (2002) ali
B. E. A. Saleh in M. C. Teich, {\it Fundamentals of Photonics}, druga izdaja, John Wiley \& Sons, Inc. (2007).}
Te so danes najpogostejša in najbolj razširjena vrsta detektorjev svetlobe, uporabljamo jih med
drugim tudi v fotoaparatih in sončnih celicah. Fotodiode so sestavljene iz $p$- in $n$-dopiranega 
polprevodnika ($p$-$n$ fotodiode) ali pa je med njima še plast nedopiranega (intrinzičnega) 
polprevodnika ($p$-$i$-$n$ fotodioda). Ko svetloba vpade na spoj različno dopiranih polprevodnikov,
se fotoni absorbirajo in nastanejo \index{Fotodioda!$p$-$n$}\index{Fotodioda!$p$-$i$-$n$}
pari elektron--vrzel. Nosilci naboja potujejo v različnih smereh, elektroni stečejo v eno smer in
vrzeli v nasprotno. Odvisno od načina delovanja merimo tok, ki steče skozi 
spoj, ali napetost, ki se pojavi na spoju. 

Spektralni odziv\index{Detektor!spektralni odziv} 
fotodiod je odvisen od energijske reže polprevodnika, 
iz katerega je fotodioda narejena.
Silicijeve \index{Silicij} fotodiode so  uporabne za zaznavanje valovnih dolžin do največ okoli
$1,1~\si{\micro\meter}$, za večje valovne dolžine (do $1,6~\si{\micro\meter}$) 
uporabljamo InGaAs.\index{InGaAs} Izkoristek fotodiod
je navadno zelo velik in presega $50~\%$, pri energiji fotonov blizu energijske reže 
je vrednost izkoristka kar blizu 1.
Za razliko od fotoprevodnikov fotodiode signala
ne ojačujejo, imajo pa praviloma hitrejši odziv, tipično okoli nanosekunde.

\subsection*{Spoj $\textbf{\textit{p}}$-$\textbf{\textit{n}}$}
\index{Spoj $p$-$n$}
Obnovimo, kaj se zgodi ob spoju $p$- in $n$- tipa polprevodnika. Pri tem tip $p$ označuje
polprevodnik, dopiran s trivalentnimi akceptorskimi primesmi, ki v snovi ustvarijo vrzeli.
Energijski nivo primesi je malo nad vrhom valenčnega pasu, zato je Fermijeva energija
polprevodnika premaknjena navzdol proti valenčnemu pasu (slika~\ref{11_PN1}\,a). 
Po drugi strani tip $n$ označuje polprevodnike s petvalentnimi \index{Polprevodnik!tip $p$}
\index{Polprevodnik!tip $n$}\index{Fermijeva energija}
donorskimi primesmi, ki v snov prinesejo dodatne elektrone. Njihov energijski nivo je malo 
pod prevodnim pasom, zaradi česar je Fermijeva energija pomaknjena navzgor proti prevodnemu pasu
(slika~\ref{11_PN1}\,b).
\begin{figure}[ht]
\centering
\def\svgwidth{128truemm} 
\input{slike/11_PN1.pdf_tex}
\caption{Shema energijskih nivojev v polprevodniku tipa $p$ (a) in $n$ (b). 
$E_a$ in $E_d$ označujeta dodatna energijska pasova zaradi primesi, 
$E_F$ pa Fermijevo energijo. Na spoju $p$-$n$ (c) se Fermijevi energiji izenačita.
Pojavi se električno polje $\mathbf{E}$, ki kaže od $n$ proti $p$.}
\label{11_PN1}
\end{figure}
\newpage
Ko staknemo polprevodnik tipa $p$ s polprevodnikom tipa $n$, elektroni 
z območja z višjo koncentracijo (tip $n$) difundirajo v območje z nižjo koncentracijo
(tip $p$), kjer se rekombinirajo z vrzelmi. \index{Spoj $p$-$n$!izpraznjeni sloj}
Ob spoju nastane ozek pas, imenujemo ga izpraznjeni sloj, kjer ni  
prostih nosilcev naboja. Tipično je širok od $10~\si{nm}$ do $1~\si{\micro\metre}$.
Po rekombinaciji ostanejo na strani $n$ pozitivno nabiti donorski atomi in 
na strani $p$ negativno nabiti akceptorski atomi, ki povzročijo nastanek  
električnega polja $\mathbf{E}$. Nastalo polje, ki kaže od $n$ proti $p$, zaustavi rekombinacijo, saj odbija
elektrone in vrzeli od spoja. V ravnovesju se Fermijeva energija na obeh straneh spoja 
izenači (slika~\ref{11_PN1}\,c). 
Potencialni skok je približno enak $\Delta E \approx E_d-E_a$, kar je le malo manj od 
širine reže $E_g$. Tipična jakost električnega polja na spoju je $10^5$--$10^7~\si{V/m}$.

Priključimo na spoj $p$-$n$ napetost $U$, tako da je pozitivna na strani $p$ (slika~\ref{11_PNU}\,a). 
Takrat pravimo, da smo napetost priključili v prevodni smeri.\index{Fotodioda!prevodna smer}
Ker lahko energijske pasove razumemo kot potencialno energijo elektronov, 
s priključeno pozitivno napetostjo zmanjšamo razliko potencialnih energij 
in elektroni lažje prehajajo iz dela $n$ v del $p$ . 
Zaradi zmanjšanja potencialne razlike med stranjo $p$ in $n$ za $e_0U$ se tok 
večinskih elektronov iz $n$ v $p$ poveča za faktor $\exp(e_0 U/kT)$, tok manjšinskih elektronov
iz $p$ v $n$ pa ostaja enak, saj ni odvisen od globine potencialnega skoka. 
\begin{figure}[ht]
\centering
\def\svgwidth{128truemm} 
\input{slike/11_PNU.pdf_tex}
\caption{Shema energijskih nivojev na spoju $p$-$n$, ko na spoj priključimo napetost
v prevodni smeri (a) in v zaporni smeri (b). Če se v izpraznjenem sloju foton absorbira 
in nastane par elektron--vrzel, elektron ``zdrsi'' proti strani $n$ in vrzel proti
strani $p$ (c).}
\label{11_PNU}
\end{figure}

Povsem enak razmislek lahko naredimo, če priključimo \index{Fotodioda!zaporna smer}
na stran $n$ pozitivni pol in na stran $p$ negativnega, če torej priključimo
napetost v zaporni smeri. V tem primeru potencialna razlika naraste in tok 
večinskih elektronov se zmanjša za faktor $\exp(-e_0 |U|/kT)$, medtem ko tok 
manjšinskih elektronov ostane nespremenjen (slika~\ref{11_PNU}\,b).

Celotni tok skozi spoj $p$-$n$ oziroma diodo je sestavljen iz prispevkov elektronov in vrzeli. Opiše ga
karakteristična enačba diode (slika~\ref{11_IU})\index{Karakteristika diode}
\boxeq{11:dioda}{
I = I_0 (e^{e_0 U/kT}-1).
}
Pri tem $I_0$ označuje tok manjšinskih nosilcev naboja\footnote{~Pravimo
mu tudi zaporni tok, tok nasičenja ali temni tok. Slednje ime izhaja iz tega, da
ta tok teče skozi fotodiodo tudi v odsotnosti vpadne svetlobe.}\index{Temni tok}
\index{Zaporni tok|see {Temni tok}}
in je navadno zelo majhen. Njegova vrednost je odvisna od snovi, površine
detektorja, poleg tega je eksponentno odvisna od temperature. Znaša 
tipično okoli $10^{-15}$--$10^{-5}~\si{\ampere}$, pri čemer najmanjše
vrednosti dosežemo le ob močnem hlajenju.\footnote{~Glej npr. N. W. Ashcroft in 
N. D. Mermin, {\it Solid State Physics}, Harcourt College
Publishers (1976).}

\begin{figure}[ht]
\centering
\def\svgwidth{85truemm} 
\input{slike/11_IU.pdf_tex}
\caption{Karakteristika $I(U)$  neosvetljene fotodiode (modra črta)
in osvetljene fotodiode (rdeče črte). Naraščajoča intenziteta vpadne svetlobe
krivuljo premika navzdol. S simboli so označene točke delovanja za različne načine vezave.}
\index{Fotodioda!zaporna smer}
\index{Fotodioda!kratko sklenjena}
\index{Fotodioda!fotovoltaik}
\label{11_IU}
\end{figure}
 
% \begin{naloga}
% Izpelji karakteristično enačbo diode (enačba~\ref{11:dioda}) in pokaži, da
% je zaporni tok \index{Temni tok}
% \beq
% I_0 = e_0 S \left(\frac{D_p}{L_p}p_{0n}+ \frac{D_n}{L_n}n_{0p}\right),
% \eeq
% kjer je $S$ presek spoja, $D$ sta difuzijska koeficienta, $L$ je difuzijska
% dolžina posameznih nosilcev naboja, $p$ in $n$ pa sta koncentraciji manjšinskih
% nosilcev naboja v ravnovesju. Ugotovi, zakaj je zaporni tok močno odvisen od
% temperature. 
% \end{naloga}
 
\subsection*{Delovanje fotodiode}
Ko na polprevodnik vpade foton, ki ima energijo večjo od širine reže, 
lahko vzbudi elektron iz valenčnega v prevodni pas in nastane par elektron--vrzel. 
Če se to zgodi v izpraznjenem sloju spoja $p$-$n$, steče elektron pod vplivom 
električnega polja na stran $n$ in vrzel na stran $p$ (slika~\ref{11_PNU}\,c). 
Premik nosilcev naboja zaradi absorpcije fotona torej vedno
povzroči električni tok v zaporni smeri. \index{Karakteristika fotodiode}
Njegova velikost je odvisna od moči vpadne svetlobe in jo zapišemo kot 
(enačba~\ref{11:eta})
\boxeq{11:if}{
I_f = e_0 \frac{\eta P}{h \nu},
}
pri čemer je $\eta$ kvantni izkoristek, $P$ označuje moč vpadne svetlobe in $\nu$ njeno
frekvenco. Celoten tok skozi fotodiodo je vsota diodnega toka (enačba~\ref{11:dioda}) 
in toka zaradi vpadne svetlobe (enačba~\ref{11:if}), zato karakteristiko fotodiode zapišemo kot 
\boxeq{11:fotodioda}{
I = I_0 (e^{e_0 U/kT}-1) - I_f.
}
Vpadna svetloba povzroči zmanjšanje električnega toka skozi diodo, 
kar na sliki~\ref{11_IU} predstavlja premik karakteristične krivulje diode v vertikalni 
smeri navzdol (rdeče črte). Naraščajoča intenziteta svetlobe premika krivuljo proti 
bolj negativnim vrednostim tokov. 

Fotodioda lahko deluje v različnih načinih (slika~\ref{11_PD}). 
Lahko jo priključimo v prevodni smeri,
najpogosteje jo priključimo v zaporni smeri, saj je v\index{Fotodioda!prevodna smer}\index{Fotodioda!zaporna 
smer}\index{Fotodioda!kratko sklenjena}\index{Fotodioda!fotovoltaik}
tem primeru tok skozi diodo linearno sorazmeren z intenziteto vpadne svetlobe, lahko 
je dioda kratko sklenjena, lahko pa je dioda v odprtem električnem krogu, v t.\ i.\ fotovoltaičnem 
načinu. 
\begin{figure}[ht]
\centering
\def\svgwidth{125truemm} 
\input{slike/11_diode.pdf_tex}
\caption{Različne vezave fotodiode: v prevodni smeri (a), v zaporni smeri (b), kratko sklenjena (c) in 
v fotovoltaičnem načinu (d)}
\label{11_PD}
\end{figure}

Prvi način delovanja fotodiode, ki ga bomo obravnavali, je fotovoltaični način.
\index{Fotodioda!fotovoltaik}
To je način, pri katerem električni tokokrog ni sklenjen (slika~\ref{11_PD}\,d), 
zato ob absorpciji fotona in nastanku para elektron--vrzel tok ne more steči. 
Še vedno se izbiti elektron pod vplivom električnega polja
na spoju premakne proti območju $n$ in vrzel proti območju $p$.
Na diodi se tako pojavi napetost, 
katere vrednost lahko izračunamo iz karakteristične enačbe diode (enačba~\ref{11:fotodioda}), 
če upoštevamo, da je v nesklenjenem tokokrogu $I=0$. Sledi
\begin{equation}
U = \frac{kT}{e_0}\ln \left(1+ \frac{I_f}{I_0}\right).
\label{eq:fvd}
\end{equation}
Pri večji intenziteti vpadne svetlobe se karakteristična krivulja 
pomika navzdol (slika~\ref{11_IU}), rešitev enačbe~(\ref{eq:fvd}) pa se 
po abscisi premika proti desni.
Večja intenziteta vpadne svetlobe pomeni večjo 
pozitivno napetost na diodi, zato tudi odzivnost v tem primeru merimo v enotah $\si{\volt}/\si{\watt}$.
Pri dovolj velikih vpadnih močeh -- oziroma dovolj velikem $I_f$ -- je
zveza med vpadno močjo in fotonapetostjo
logaritemska. Fotovoltaična oziroma odprta vezava fotodiode zato omogoča 
zaznavanje vpadne moči v zelo širokem intervalu. 

Drugi način delovanja je kratko sklenjena fotodioda (slika~\ref{11_PD}\,c).
V tem primeru je \index{Fotodioda!kratko sklenjena} napetost na diodi 
vedno enaka nič, prav tako ni toka skozi diodo v odsotnosti svetlobe. 
Ko posvetimo na diodo, nastanejo pari elektron--vrzel in električni tok steče
v zaporni smeri. Tudi iz enačbe~(\ref{11:fotodioda}) vidimo, da je tok 
skozi tokokrog v primeru kratko sklenjene diode ($U=0$) kar enak toku 
zaradi vpadne svetlobe $I_f$ (slika~\ref{11_IU}). 

Najbolj splošno uporaben način za detekcijo svetlobe je način, v katerem 
napetost na diodo priključimo v zaporni smeri (slika~\ref{11_PD}\,b).\index{Fotodioda!zaporna smer}
Takrat se tok skozi diodo spreminja linearno z močjo vpadne svetlobe
(enačba~\ref{11:fotodioda}), odziv pa je hitrejši
kot pri kratko sklenjeni diodi. Če dodamo v tokokrog zaporedno vezan še nek upornik, se odziv
spremeni. Zvezo med napetostjo in tokom zapišemo kar z Ohmovim zakonom $U = -|U_0|-RI$. 
Na sliki to predstavlja premico, ki seka karakteristične krivulje (slika~\ref{11_IU}). Ker upornost
upornika ni enaka nič, se po grafu ob naraščajoči moči vpadne svetlobe ne premikamo več navpično navzdol, 
ampak pod kotom proti desni. Način, ko v tokokrog s fotodiodo vežemo še Ohmski upor,
se uporablja tudi v sončnih celicah. 

\begin{naloga}
\label{naloga:optR}
V sončnih celicah želimo iz vpadne svetlobne moči pridobiti kar največ električne moči.
Pokaži, da električna moč na uporniku, prek katerega sklenemo fotodiodo, pri nekem $R$
zavzame največjo vrednost. Poišči $R$ za primer $I_0 = 10^{-9}~\si{A}$, $I_f = 10^{-3}~\si{A}$ 
in $T=300~\si{K}$.
\end{naloga}

Prednosti merjenja ob napetosti, priključeni v zaporni smeri, je več. 
Zaradi priključene napetosti se zmanjša čas preleta
nosilcev naboja in posledično se zmanjša odzivni čas detektorja.\index{Detektor!odzivni čas}
Dodatno se poveča širina izpraznjenega pasu (naloga~\ref{naloga:depl}), kar zmanjša kapaciteto spoja 
(spoj $p$-$n$ namreč deluje kot  
kondenzator in časovni odziv je odvisen od njegove kapacitete) in s tem odzivni čas. Povečana
izpraznjena plast vodi tudi do večjega območja, v katerem se fotoni absorbirajo. 
\vglue2truemm

\begin{naloga}
\label{naloga:depl}
Pokaži, da je debelina izpraznjene plasti enaka $d = d_p+d_n$, kjer sta $d_p$ (debelina izpraznjene
plasti na strani $p$) in $d_n$ (debelina izpraznjene plasti na strani $n$) podana z enačbama 
\beq
d_{p}= \sqrt{\frac{2\epsilon \varepsilon_0(\Delta E_0-e_0U)}{e_0}\frac{N_d/N_a}{N_a+N_d}}
\quad\mathrm{in}\quad 
d_{n}= \sqrt{\frac{2\epsilon \varepsilon_0(\Delta E_0-e_0U)}{e_0}\frac{N_a/N_d}{N_a+N_d}}.
\eeq
Pri tem $N_a$ in $N_d$ označujeta gostoti akceptorskih in donorskih atomov, $\Delta E_0$ je ravnovesni 
potencialni skok med stranjo $n$ in $p$ in $U$ priključena napetost.

Namig: zapiši Gaussov zakon in 
upoštevaj zvezo $d_n N_d = d_p N_a$.
\end{naloga}
\vglue2truemm

Povejmo še nekaj o zgradbi fotodiode. Shema preproste fotodiode je prikazana na 
sliki~\ref{11_shema}\,a.
Na dnu je elektroda, sledi plast $n$, nad njo je tanka plast $p$, na katero vpada svetloba.
Bistveno je, da je osvetljena plast tanka, da svetloba lahko prodre v bližino spoja. Zato so 
debeline zgornje plasti tipično submikronske. Na fotodiode pogosto nanesemo
še dodatno antirefleksijsko plast (SiO$_2$)\index{SiO$_2$}. 
Fotoobčutljivi del komercialnih fotodiod
meri tipično od nekaj $100~\si{\micro\meter}^2$  do več $100~\si{\milli\metre}^2$. Pri 
tem imajo večje diode seveda počasnejši odziv. 
\vglue3truemm
\begin{figure}[ht]
\centering
\def\svgwidth{128truemm} 
\input{slike/11_shema.pdf_tex}
\caption{Sheme fotodiod: $p$-$n$ fotodioda (a), $p$-$i$-$n$ fotodioda (b),  ki se od 
navadne $p$-$n$ razlikuje po vmesni plasti intrinzičnega
polprevodnika, in Schottkyjeva fotodioda (c). 
\index{Fotodioda!$p$-$n$}
\index{Fotodioda!$p$-$i$-$n$}
Temno siva barva označuje elektrode, svetlo modra območje $n$, 
temnejša modra območje $p$ in svetlo siva območje intrinzičnega polprevodnika. }
\label{11_shema}
\end{figure}

\begin{remark}
Poleg do zdaj obravnavanih fotodiod poznamo tudi heterostrukturne 
\index{Fotodioda!heterostrukturna}fotodiode, kjer sta $p$ in $n$ del
narejena iz druge snovi. Poseben primer so Schottkyjeve fotodiode\footnote{~Nemški fizik
Walter Hans Schottky, 1886--1976.}, kjer eno plast polprevodnika
nadomestimo z zelo tanko plastjo kovine (slika~\ref{11_shema}\,c). Te so uporabne predvsem pri 
visokih energijah (v UV območju, slika~\ref{11_odziv}), \index{Fotodioda!Schottkyjeva}\index{Ultravijolično valovanje}
saj je v navadnih fotodiodah absorpcija za te valovne dolžine prevelika, na površini pride do 
rekombinacije in zmanjšanja kvantnega izkoristka.\footnote{~Glej npr. G. H. Rieke, {\it Detection 
of light}, druga izdaja, Cambridge University Press (2002).}  
Odziv Schottkyjevih fotodiod je zelo hiter, 
saj nizka upornost kovine občutno zmanjša $RC$ konstanto spoja. Odzivni časi 
dosegajo pikosekundne vrednosti.
\end{remark}

\subsection*{Fotodioda $\textbf{\textit{p}}$-$\textbf{\textit{i}}$-$\textbf{\textit{n}}$}
Fotodiode $p$-$i$-$n$ se od navadnih $p$-$n$ razlikujejo po tem, da med $p$- in $n$-plast 
vključimo še plast nedopiranega polprevodnika (slika~\ref{11_shema}\,b). 
S tem se bistveno poveča debelina\index{Fotodioda!$p$-$i$-$n$}
izpraznjene plasti, ki postane praktično neodvisna od priključene napetosti.
Povečanje izpraznjene plasti omogoča zaznavanje bistveno večjega deleža vpadne svetlobe, 
poleg tega zmanjša kapaciteto spoja in s tem njegovo $RC$ konstanto. 

Slabost dodatnega
sloja je povečanje časa preleta čez izpraznjeno plast, vendar lahko
z ustrezno optimizacijo konstrukcije dosežemo odzivne čase nekaj deset $\si{ps}$.
Kvantni izkoristki $p$-$i$-$n$ so visoki, a seveda odvisni od valovne dolžine svetlobe
(slika~\ref{11_odziv}).\footnote{~Slika povzeta po S. M. Sze in M. K. Lee, {\it Semiconductor Devices: Physics and Technology}, tretja izdaja, John Wiley \& Sons, Inc. (2012).}

\begin{figure}[ht]
\centering
\def\svgwidth{110truemm} 
\input{slike/11_SpekterFD.pdf_tex}
\caption{Kvantni izkoristek nekaterih $p$-$i$-$n$ (Si in Ge) in Schottkyjevih (Au-Si in Ag-ZnS)
fotodiod}
\label{11_odziv}
\end{figure}

\section{Plazovne fotodiode}
Ko smo risali karakteristiko fotodiode (slika~\ref{11_IU}), 
nismo narisali popolne slike.\index{Fotodioda!plazovna}
Pri velikih negativnih napetostih se namreč karakteristika znatno spremeni (slika~\ref{11_plaz}), 
česar ne moremo popisati s preprosto enačbo. Pri zapornih napetostih, ki za nekajkrat presegajo 
širino energijske reže (tipično $100$--$500~\si{\volt}$), \index{Karakteristika fotodiode}
se električni tok naglo poveča. Podobno velja ob vpadu in absorpciji fotona. 
Mobilni nosilci naboja se namreč v električnem polju tako pospešijo, da s trki ustvarjajo nove pare 
elektron--vrzel. Novonastali pari  ustvarjajo nove pare in pojavi se ``plaz'', podobno kot v 
fotopomnoževalki. En vpadni foton sproži cel plaz elektronov, zato pravimo, da je plazovna dioda
fotodioda z notranjim ojačenjem. \index{Notranje ojačenje}Pri tem je faktor ojačenja tipično $30$--$300$ 
in plazovne fotodiode lahko 
uporabimo za detekcijo posameznih fotonov. Slabost je, da je faktor ojačenja odvisen od
temperature in je zato za natančne meritve potrebna temperaturna stabilizacija.
\begin{figure}[ht]
\centering
\def\svgwidth{50truemm} 
\input{slike/11_plaz.pdf_tex}
\caption{Karakteristika plazovne fotodiode}
\label{11_plaz}
\end{figure}

Napetost, pri kateri deluje plazovna fotodioda, je priključena v zaporni smeri\index{Fotodioda!zaporna smer} 
in je tik pod prebojno napetostjo. Ker že  majhna odstopanja v napetosti povzročijo veliko
spremembo v toku, moramo napetost držati kar se da stabilno. Le tako dosežemo 
linearen odziv fotodiode od moči vpadne svetlobe. Plazovne fotodiode so praviloma zelo hitre 
($\sim 50~\si{\pico\second})$ in zelo občutljive. Z ojačenjem signala se ojači tudi šum, a je 
povečanje pogosto manjše kot bi bil prispevek k šumu na zunanjih elektronskih ojačevalcih. 

\section{CCD in CMOS detektorji}
Do zdaj smo obravnavali detektorje, \index{Detektor!CCD}
\index{Detektor!CMOS}ki zaznavajo pretok vpadnih fotonov in spreminjanje
pretoka v času. Dodatno informacijo dobimo, če več fotodetektorjev sestavimo v 
dvodimenzionalno matriko, saj lahko detektorji hkrati zaznavajo količino vpadne svetlobe 
iz različnih delov prostora. Podatke iz posameznih detektorjev sestavimo v sliko, pri čemer 
en detektor podaja informacijo o številu vpadnih fotonov 
v dani časovni enoti za en slikovni element -- piksel\index{Piksel}. Času zajemanja, ki
predstavlja integracijski čas, pravimo tudi čas osvetlitve.  
Slikovni detektorji z veliko ločljivostjo so sestavljeni iz 
več milijonov ali celo milijard posameznih polprevodniških detektorjev in so 
nepogrešljivi v fotoaparatih, kamerah, mikroskopiji in astronomiji.

Podrobneje bomo obravnavali dve vrsti matričnih detektorjev, to sta CCD 
({\it Charge-Coupled-Device})\footnote{~Za izum CCD detektorjev sta Willard 
S. Boyle in George E. Smith  leta 2009 prejela Nobelovo nagrado.} 
in CMOS ({\it Complementary Metal-Oxide-Semiconductor})\footnote{~Teh 
dveh oznak za detektorje praviloma ne prevajamo. Opisujeta 
strukturo in delovanje naprave in nista vezani zgolj na detekcijo svetlobe.}. 
Omenjeni vrsti
detektorjev sta si po načinu zaznavanja svetlobe zelo podobni, razlika
je predvsem v postopku, kako iz posameznega detektorja pridobimo podatek o številu 
vpadnih fotonov oziroma številu vzbujenih elektronov.
\vglue-2truemm
\begin{remark}
Slikovni detektorji so seveda lahko sestavljeni tudi iz drugih vrst svetlobnih detektorjev, 
ki smo jih obravnavali v prejšnjih razdelkih. Lahko so iz mikrobolometrov\index{Bolometer}
ali fotoprevodnikov\index{Fotoprevodnik} (za infrardeče 
valovanje),\index{Infrardeče valovanje}
Schottkyjevih fotodiod\index{Fotodioda!Schottkyjeva} (npr. PtSi, ki seže od UV do
okoli $6~\si{\micro\meter}$) ali plazovnih fotodiod.\index{Fotodioda!plazovna}\index{Ultravijolično valovanje}
\vglue-5truemm
\end{remark}

\subsection*{CCD}
Detektorji CCD\index{Detektor!CCD} so sestavljeni iz posameznih tako imenovanih MOS 
({\it Metal-Oxide-Semi\-conductor}
-- kovina-oksid-polprevodnik) kondenzatorjev. \footnote{~Glej npr. 
S. M. Sze in M. K. Lee, {\it Semiconductor Devices: Physics and Technology}, 
tretja izdaja, John Wiley \& Sons, Inc. (2012).}
Njihova osnova je dopiran silicij\index{Silicij}, vmesna 
plast med polprevodnikom in prevodno elektrodo pa je navadno zelo tanka plast (pod $100~\si{nm}$)
SiO$_2$ (slika~\ref{11_MOS}).\index{SiO$_2$}
Prevodna elektroda je bila prvotno iz kovine (npr. aluminija) in je detektorju 
dala tudi ime.
Danes je kovino večinoma nadomestil polikristalni silicij (polisilicij), ime pa je ostalo.
Tipična dolžina stranice posameznega elementa je $5$--$40~\si{\micro\meter}$.

\begin{figure}[ht]
\centering
\def\svgwidth{120truemm} 
\input{slike/11_MOS.pdf_tex}
\caption{Shema MOS strukture (a), na kateri temeljijo CCD in CMOS slikovni detektorji. Osnova je 
polprevodnik ($p$), na katerem je plast dielektrika (SiO$_2$) in na njej elektroda (siva). 
Ob absorpciji svetlobe se pojavijo fotoelektroni, katere pozitivna napetost
na elektrodi drži ujete v potencialno jamo (vijolična). S spreminjanjem napetosti
na sosednjih elektrodah lahko elektrone premikamo (b).}
\label{11_MOS}
\end{figure}
Ko foton skozi tanko prozorno elektrodo vpade na polprevodnik, v njem ustvari
par elektron--vrzel. Pozitivna napetost na elektrodi elektrone privlači, vendar jih 
vmesna plast izolatorja tik pod površino ustavi in elektroni tako ostanejo ujeti v potencialni jami
pod elektrodo. Število ujetih elektronov je sorazmerno številu vpadnih fotonov v času zajemanja slike, 
pomnoženih s kvantnim izkoristkom pri dani valovni dolžini. 

S spreminjanjem napetosti na posameznih elektrodah lahko naboj, ki se lokalno nabere 
v plasti pod izolatorjem v danem času, postopoma prenesemo od posameznega 
piksla do izhodne stopnje. Najprej poteka prenos iz enega elementa na sosednjega znotraj vrstic, 
nato izpraznimo zadnji stolpec in postopek ponovimo, dokler ne zajamemo celotne slike 
(slika~\ref{11_CCD}\,a). 

Na izhodu signal sproti ojačujemo, 
pretvorimo v napetost in nato  v digitalni zapis. Vsakemu slikovnemu elementu priredimo
digitalno vrednost barvne globine glede na število vpadnih fotonov oziroma elektronov. Pri tem
8-bitni zapis slike, na primer, vsakemu elementu priredi vrednost od 0 do 255, 16-bitni pa od 0 do 65535.

Delovanje detektorjev CCD temelji na zaporednem odčitavanju števila fotoelektronov v posameznem 
slikovnem elementu. Ta način je razmeroma počasen in omejuje hitrost zajemanja slike. Med 
prenašanjem nabojev do izhoda namreč slike ne moremo zajemati, saj bi prišlo do popačenja signala. 
Pomanjkljivost se večinoma rešuje tako, da le del celotnega zaslona zajema svetlobo, drugi del
pa je namenjen hkratnemu pretakanju elektronov in omogoča nemoteno praktično neprestano zajemanje slike.
Ker se s tem količina zajete svetlobe zmanjša, se na vsak element doda lečo, ki svetlobo zbere
na detektor. S tem postanejo slikovni detektorji CCD hitrejši in bolj občutljivi. Poleg
tega jih odlikuje tudi razmeroma nizek šum, ki se ga da s hlajenjem še dodatno 
zmanjšati. 
\begin{figure}[ht]
\centering
\def\svgwidth{100truemm} 
\input{slike/11_CCD.pdf_tex}
\caption{Shema zajemanja slike s slikovnima detektorjema CCD (a) in  CMOS (b). Puščice označujejo premikanje
fotoelektronov ob odčitavanju slike.}
\label{11_CCD}
\vglue-4truemm
\end{figure}
\begin{remark}
Pri zajemanju slike pogosto ne potrebujemo največje ločljivosti, ki jo omogoča detektor. 
Takrat se poslužujemo združevanja sosednjih elementov, t.\ i.\ bininga ({\it binning}), 
na primer $2\times2$ ali $4\times4$. Z združevanjem pikslov sicer zmanjšamo ločljivost slike, 
vendar hkrati skrajšamo čas njenega zajemanja in zmanjšamo razmerje signal proti šumu. 
\index{Bining|see {Združevanje pikslov}}\index{Združevanje pikslov}
\end{remark}

\subsection*{CMOS}
Osnovni element detektorjev CMOS\index{Detektor!CMOS} je enak kot za detektorje CCD (slika~\ref{11_MOS}\,a). 
Bistvena razlika je v načinu zajemanja fotoelektronov. Pri detektorjih CCD je branje 
fotoelektronov zaporedno, pri detektorjih CMOS pa poteka branje vseh slikovnih elementov 
hkrati, pri čemer ima vsak piksel tudi svoj ojačevalnik (slika~\ref{11_CCD}\,b).\index{Piksel}

Zaradi sprotnega odčitavanja vseh pikslov naenkrat so detektorji CMOS navadno bistveno hitrejši 
od CCD. Odlikuje jih tudi nizka poraba energije in nizka cena. Njihova slabost
je večji šum in manjša občutljivost, saj del zaslona, kjer so ojačevalniki, slike ne more
zajemati. 

\subsection*{Barvno zajemanje slik}
Detektorji zaznavajo število vpadnih fotonov oziroma število fotoelektronov.
Za nastanek barvne slike moramo vpadne fotone ločiti še po valovni dolžini, kar naredimo
z barvnimi filtri. Namesto enega elementa, ki bi podal informacijo o intenziteti vpadne 
svetlobe, uporabimo štiri senzorje v kvadratni mreži: enega za zaznavanje rdeče svetlobe,
enega za modro in dva za zeleno svetlobo.\index{Detektor!barvni}
Večji delež zelenih elementov je zaradi večje občutljivosti človeškega očesa na zeleno barvo. 
Intenziteto svetlobe na posameznem slikovnem elementu dane barve nato odčitamo, kot je
opisano zgoraj.
 
\section{Šum pri optični detekciji}
\label{chap:sum}
Pri vsakršni detekciji svetlobe je vedno prisoten tudi šum. \index{Šum}Beseda šum označuje naključne 
fluktuacije na izhodu iz detektorja, ki jih ne moremo ločiti od signala. Z različnimi 
pristopi lahko šum zmanjšamo, vendar ga povsem ne moremo nikoli odpraviti. Obravnava 
šuma je zato najbolj pomembna pri zaznavanju šibkih signalov svetlobe. Ključen
parameter je najmanjša moč vpadne svetlobe, ki jo še lahko ločimo od šuma. Pod to vrednostjo 
se signal v šumu izgubi (slika~\ref{11_sum}).
\begin{figure}[ht]
\centering
\def\svgwidth{120truemm} 
\input{slike/11_sum.pdf_tex}
\caption{Če je signal (rdeča črta) velik v primerjavi s šumom (modra črta), 
ga na detektorju lahko zaznamo (zgoraj). 
Pod določeno vrednostjo postane velikost signala primerljiva s šumom in signala ne zaznamo več
(spodaj).}
\label{11_sum}
\end{figure}

Na podlagi fizikalnega izvora ločimo več vrst šuma:
\begin{enumerate}
\item šum štetja, ki je posledica diskretne (kvantne) narave fotonov,
\item termični šum, ki je posledica termičnih fluktuacij,
\item šum temnega toka, ki predstavlja spontani nastanek para elektron--vrzel oziroma spontano
emisijo elektronov in
\item šum sevanja ozadja.\index{Šum!štetja}\index{Šum!termični}\index{Šum!temnega toka}\index{Šum!sevanja ozadja}
\end{enumerate}

\subsection*{Šum štetja} 
Naj na detektor vpada svetloba s konstantno vpadno 
močjo $P$. Ker je svetloba sestavljena iz diskretnih fotonov, 
fotoni na detektor vpadajo posamično in enkrat jih vpade več, \index{Šum!štetja}
drugič manj. Vpadna moč je zato dejansko povprečna moč $\overline{P}$ in število 
vpadnih fotonov na časovno enoto $\Phi$ je povprečna vrednost števila vpadnih fotonov na časovno enoto
\begin{equation}
\overline{\Phi} = \frac{\overline{P}}{h\nu}.
\end{equation}
Vpad fotonov predstavlja diskretne in neodvisne procese, zato za njihov vpad skoraj vedno velja
Poissonova porazdelitev (slika~\ref{11_Poiss}). Verjetnost, da v času $\tau$, ki predstavlja 
čas merjenja, na detektor vpade $n$ fotonov, je tako \index{Poissonova porazdelitev}
\begin{equation}
p(n) = \frac{\overline{n}^n e^{-\overline{n}}}{n!},
\label{Poisson}
\end{equation}
pri čemer je povprečno število vpadnih fotonov v tem časovnem intervalu 
enako $\overline{n} = \overline{\Phi}\tau$.
\begin{figure}[ht]
\centering
\def\svgwidth{80truemm} 
\input{slike/11_poisson.pdf_tex}
\caption{Poissonova porazdelitev verjetnosti za $\overline{n}=2$ (modra), 
$\overline{n}=5$ (rdeča) in $\overline{n}=10$ (zelena). Porazdelitev je 
diskretna, črta je zgolj vodilo.}
\label{11_Poiss}
\end{figure}

Fluktuacije števila fotonov, ki vpadejo na detektor v danem časovnem 
intervalu, označimo z $\Delta n = n-\overline{n}$. V povprečju je ta vrednost seveda enaka nič, 
zato sta bolj merodajni količini varianca, ki je enaka (glej nalogo~\ref{nal:Poiss})
\begin{equation}
\sigma^2 = \overline{(\Delta n)^2}= \overline{(n-\overline{n})^2} = \overline{n},
\label{varianca}
\end{equation}
in standardni odklon
\begin{equation}
\sigma = \sqrt{\overline{(\Delta n)^2}} = \sqrt{\overline{n}}.
\label{sigma}
\end{equation}
Standardni odklon, ki je merilo za velikost šuma, torej narašča korensko 
z naraščajočim povprečnim številom vpadnih fotonov $\overline{n}$. 
\begin{naloga}
Pokaži, da je povprečje Poissonove porazdelitve (enačba~\ref{Poisson}) vedno pri $n = \overline{n}$
in da je standardni odklon $\sigma = \sqrt{\overline{n}}$.
\label{nal:Poiss}
\end{naloga}

Absolutni šum nas večinoma ne zanima, 
saj je pri detekciji ključno razmerje med merjenim 
signalom in šumom.\index{Razmerje signal proti šumu} 
Označimo ga s $SNR$ ({\it Signal to Noise Ratio} --
razmerje \index{SNR@$SNR$|see{Razmerje signal proti šumu}}
signal proti šumu)\footnote{~Pogosto se uporablja tudi oznako $S/N$. Tukaj smo jo 
zaradi jasnosti zamenjali, saj $N$ označuje število elektronov.}.
V primeru Poissonove porazdelitve in šuma štetja velja
\boxeq{11:SNR}{
SNR = \frac{\overline{n}}{\sigma} = \sqrt{\overline{n}}.
}
Razmerje signal proti šumu z naraščajočim številom vpadnih fotonov narašča, 
relativni šum pa ob večji vpadni moči svetlobe pojema. 
Poglejmo dva primera. V prvem je povprečno število vpadnih 
fotonov v danem časovnem intervalu $10^6$ in v drugem $100$. 
Pri vpadu močnejšega signala 
na detektorju zaznavamo $10^6 \pm 1000$ fotonov in pri vpadu šibkejšega
$100 \pm 10$. Čeprav je absolutni šum v prvem primeru stokrat večji, 
je relativni šum stokrat manjši. Za zmanjšanje vpliva šuma štetja na meritev mora biti 
torej signal kar se da velik. 

\begin{remark}
Razmerje signal proti šumu $SNR$ lahko vpeljemo na več načinov. Prvi je ta, ki smo ga 
uporabili mi, pri katerem velja $SNR = \overline{n}/\sigma = \overline{n}$. To je
$SNR$ optične moči oziroma števila fotonov in nastalega električnega toka. Lahko
pa vpeljemo tudi $SNR_e$ električne moči na detektorju.
Zaradi kvadratne zveze med električnim tokom in električno močjo velja 
$SNR_e=SNR^2={\overline{n}}$.
\end{remark}

Pri šumu štetja je ključna diskretna narava fotonov, zato je ta vrsta šuma
prisotna pri prav vseh načinih detekcije. Podrobneje si oglejmo, 
kako se šum štetja izraža pri detekciji s fotodiodami. \index{Fotodioda}

Naj svetloba s povprečno močjo $\overline{P}$ vpada na fotodiodo.
Povprečno število fotoelektronov $\overline{N}$, ki se pojavijo v časovnem intervalu 
$\tau$, je kar enako številu vpadnih fotonov, pomnoženim 
s kvantnim izkoristkom. 
\begin{equation}
\overline{N} = \overline{n}\eta = \frac{\overline{P}\tau}{h \nu}\eta.
\end{equation}
Povprečni električni tok, ki steče skozi detektor, je  
\begin{equation}
\overline{I} = \frac{\overline{N} e_0}{\tau},
\end{equation}
in varianca izhodnega električnega toka 
\begin{equation}
\overline{(\Delta I)^2}=\overline{(I-\overline{I})^2} = \overline{(N-\overline{N})^2}\,
\frac{e_0^2}{\tau^2} = \overline{N}\,\frac{e_0^2}{\tau^2}= \overline{I}\,\frac{e_0}{\tau},
\end{equation}
pri čemer smo upoštevali enačbo~(\ref{varianca}). 

Vpeljemo še pasovno širino 
detekcije\footnote{~Pri detekciji signala navadno uporabljamo
čas osvetlitve $\tau$ in pri telekomunikacijah pasovno širino $\Delta\nu_B$.}
\footnote{~Glej npr. R. W. Boyd, 
{\it Radiometry and the Detection of Optical Radiation}, John Wiley \& Sons, Inc. (1983).}, ki je 
enaka $\Delta\nu_B = 1/(2\tau)$,
in zapišemo\index{Pasovna širina detekcije}
\boxeq{11:sum}{
\overline{(\Delta I)^2} = 2 \overline{I}\,e_0\, \Delta\nu_B.
}
Šum na izhodu je tako sorazmeren s povprečno intenziteto signala in 
s pasovno širino detekcije oziroma obratno sorazmeren z dolžino 
merjenja. Zapišemo še razmerje signal proti šumu 
\begin{equation}
SNR = \frac{\overline{I}}{\sqrt{\overline{(\Delta I)^2}}}= \sqrt{\frac{\overline{I}}
{2 e_0\, \Delta\nu_B}}.
\label{SNRs}
\end{equation}
Po pričakovanjih je to razmerje večje pri večjem povprečnem signalu in pri daljši meritvi.

\subsection*{Termični šum} 
Termični šum 
\index{Šum!termični}imenujemo tudi Johnsonov\footnote{~Švedsko-ameriški elektroinženir in fizik 
John Bertrand Johnson, 1887--1970.} ali Nyquistov\footnote{~Švedsko-ameriški elektroinženir
Harry Nyquist, 1889--1976.} ali Johnson-Nyquistov\footnote{~Johnson je bil prvi, 
ki je pojav opazoval, \index{Johnsonov šum|see{Šum, termični}}
\index{Nyquistov šum|see{Šum, termični}}
\index{Johnson-Nyquistov šum|see{Šum, termični}}
Nyquist pa je kmalu za eksperimentom podal teoretično razlago.} šum. 
Je posledica termično vzbujenega naključnega gibanja elektronov. Premiki elektronov
na danem uporniku povzročijo majhne kratkotrajne 
fluktuacije v napetosti. V povprečju napetost sicer ostaja enaka nič, njena varianca pa je od 
nič različna. 
Termični šum nastaja samo v uporniških elementih sistema, saj le ti lahko
sprejemajo in oddajajo energijo, v kapacitivnih in induktivnih elementih pa ne.

Načinov izpeljave termičnega šuma na uporniku je več. Mi ga obravnavajmo na preprostem vezju, 
v katerem sta povezana upornik z upornostjo $R$ in kondenzator s kapaciteto $C$. Izvorov 
napetosti ni, vendar se zaradi naključnega gibanja elektronov pojavljajo fluktuacije v 
napetosti $\Delta U$, ki predstavljajo prostostno stopnjo v sistemu.  
Iz termodinamike vemo, da je povprečna energija sistema, ki je v 
ravnovesju, $kT/2$ na vsako prostostno stopnjo.\footnote{~Glej
npr. I. Kuščer in S. Žumer, {\it Toplota}, tretji natis, DMFA--založništvo (2017).}
Povprečna energija na kondenzatorju 
je pri napetosti $\Delta U$ tako enaka
\begin{equation}
\overline{W} = \frac{C\overline{(\Delta U)^2}}{2} = \frac{kT}{2}.
\end{equation}
Ko se na kondenzatorju pojavi napetost, steče skozi upornik električni tok. 
Največja povprečna moč, ki jo lahko dobimo iz takega vezja zaradi šuma, je enaka 
\begin{equation}
\overline{P} = \frac{R \overline{(\Delta I)^2}}{2},
\end{equation}
pri čemer $\Delta I$ označuje fluktuacije električnega toka skozi upornik.

Velja tudi zveza
\begin{equation}
\overline{W} = \overline{P}\tau_e,
\end{equation}
pri čemer je $\tau_e$ odzivni čas vezja in je enak $\tau_e = RC$. Sledi 
\begin{equation}
\overline{(\Delta I)^2} = \frac{kT}{R^2C}.
\end{equation}
Vpeljemo še pasovno širino detekcije, ki je v primeru eksponentnega odziva vezja enaka
$\Delta \nu_B = 1/4\tau_e = 1/4RC$.\footnote{~Glej npr. R. W. Boyd, 
{\it Radiometry and the Detection of Optical Radiation}, John Wiley \& Sons, Inc. (1983).} 
Iz tega sledi zveza
\boxeq{11:termicnii}{
\overline{(\Delta I)^2}  = \frac{4 kT\Delta \nu_B}{R}.
}
Podobno lahko zapišemo še za fluktuacije napetosti 
\boxeq{11:termicni}{
\overline{(\Delta U)^2}  = 4 kTR \Delta \nu_B.
}
Termični šum je odvisen od temperature in od upornosti detektorja oziroma
vezja, preko katerega zaznavamo signal. En način za zmanjšanje termičnega šuma
je povečanje upornosti detektorja, vendar na ta način zmanjšamo hitrost
odziva. Tipične upornosti, ki se uporabljajo pri hitrih detektorjih, so 
tako $R \sim 50~\si{\ohm}$. Drugi način za zmanjševanje termičnega šuma je
hlajenje. Na ta način lahko termični šum zelo zmanjšamo, nikoli pa ga ne moremo povsem
odpraviti.  

\subsection*{Šum temnega toka} 
Natančna opazovanja pokažejo, da na večini kvantnih detektorjev zaznamo 
nek majhen izhodni signal tudi v odsotnosti vpadne svetlobe. 
\index{Šum!temnega toka}
\index{Temni tok}To je tako imenovani temni tok, ki je posledica
spontanih nastankov para elektron--vrzel ali spontane emisije elektronov. 
Izraza za temni tok tukaj ne bomo 
izpeljevali\footnote{~Glej npr. N. W. Ashcroft in N. D. Mermin, {\it Solid State Physics}, Harcourt College
Publishers (1976).},
povejmo le, da je sorazmeren s površino diode $S$ in 
eksponentno odvisen od temperature $T$ in energijske reže polprevodnika $E_g$
\begin{equation}
I_0 = j_0\, S\, e^{-E_g/kT}.
\end{equation}
Zaradi diskretne narave elektronov se -- podobno
kot v primeru diskretnih vpadnih fotonov -- tudi tukaj pojavi šum štetja, le da 
namesto povprečne vrednosti signala nastopa temni tok. 
Enačbo (\ref{11:sum}) zato zapišemo kot 
\boxeq{11:dark}{
\overline{(\Delta I)^2} = 2 I_0\,e_0\, \Delta\nu_B.
}
Manjši šum je pri detektorjih, ki imajo manjši temni tok, na primer pri siliciju.\index{Silicij}
Germanij \index{Germanij} ima na splošno večji temni tok in zato tudi več šuma temnega toka. Pomembno
vlogo ima tudi temperatura, saj v temnem toku nastopa v eksponentu. S hlajenjem lahko 
šum temnega toka znatno zmanjšamo. 

\subsection*{Šum zaradi sevanja ozadja}
Kot že ime pove, je ta vrsta šuma posledica sevanja ozadja pri končni temperaturi.
\index{Šum!sevanja ozadja}
Okolico obravnavamo kot črna telesa\index{Sevanje črnega telesa} 
in spekter njihovega sevanja opisuje Planckov \index{Planckov zakon}
zakon (enačba~\ref{eq:Planck} in slika~\ref{fig:Planck}). Z naraščajočo temperaturo telesa se 
spektralni vrh pomika h krajšim valovnim dolžinam in s tem 
v infrardeč del spektra\index{Infrardeče valovanje} ali celo v vidno svetlobo. 

Sevanje ozadja predstavlja največji problem pri meritvah v
območju valovnih dolžin okoli $10$--$30~\si{\micro\meter}$, v katerem znatno sevajo še telesa 
pri sobni temperaturi. Detektorjem za infrardeče valovanje zato pogosto zmanjšamo aperturo 
na najmanjšo možno, poleg tega jih izoliramo od okolice in hladimo. 

Sevanje ozadja je neodvisno od vpadnega signala. Ker detektor ne loči fotonov, ki 
vpadejo nanj kot signal, od tistih, ki vpadejo nanj iz ozadja, se prispevek ozadja 
kar prišteje signalu. Šum štetja (enačba~\ref{11:sum}) se tako poveča na
\begin{equation}
\overline{(\Delta I)^2} = \frac{2 \eta e_0^2\, \Delta\nu_B}{h\nu}\,\,
\overline{\left( P + P_o \right)},
\label{11:ozadje}
\end{equation}
pri čemer $P_o$ označuje moč vpadne svetlobe iz ozadja.

\begin{remark}
 V detektorjih, v katerih z notranjim ojačenjem
 \index{Notranje ojačenje}(npr. v fotopomnoževalki
 \index{Fotopomnoževalka} ali plazovni fotodiodi),
 \index{Fotodioda!plazovna}
 se skupaj s signalom ojači tudi šum. Če se signal ojači za faktor $G$, se za isti faktor
 povečajo tudi šum štetja, šum ozadja in šum temnega toka. Poleg tega se šum ojači
 zaradi naključnega večanja števila elektronov med pomnoževanjem signala. V tem primeru nastopi
 še dodaten faktor, večji od ena, ki je odvisen od snovi, strukture in ojačenja fotodetektorja. 
 Tipična vrednost je okoli $1,5$--$2$, vendar lahko doseže vrednosti tudi nad 
 $10$.\footnote{~Glej npr. B. E. A. Saleh in M. C. Teich, 
{\it Fundamentals of Photonics}, druga izdaja, John Wiley \& Sons, Inc. (2007).}
\end{remark}

\subsection*{Seštevanje šumov}
Spoznali smo, da je več vrst šuma, ki so pri različnih pogojih različno pomembni.
\index{Šum!seštevanje}
Na splošno lahko vse prispevke združimo v skupni šum, pri čemer seštevamo kvadrate
odstopanj
\begin{equation}
\overline{(\Delta I)^2} = \overline{(\Delta I)^2}_{\mathrm{\check{s}tetja}} + 
\overline{(\Delta I)^2}_{\mathrm{termi\check{c}ni}} + \overline{(\Delta I)^2}_{\mathrm{temni}} + 
\overline{(\Delta I)^2}_{\mathrm{ozadja}}.
\end{equation}
Če vstavimo izraze za tokove (enačbe~\ref{11:sum}, \ref{11:termicnii}, \ref{11:dark}
in \ref{11:ozadje}), sledi
\boxeq{skupensum}{
\overline{(\Delta I)^2} = \left( 2 \overline{I}\,e_0 + \frac{4 kT}{R} + 2 I_0\,e_0
+ 2 I_o\,e_0 \right) \Delta\nu_B.
}
Posamezne prispevke lahko pogosto zanemarimo, odvisno seveda od intenzitete signala,
načina detekcije, sevanja ozadja ... 
\newpage

Izraz za razmerje signal proti šumu na splošno 
zapišemo kot
\begin{equation}
SNR = \frac{\overline{I}}{\sqrt{\left( 2 \overline{I}\,e_0 + 2 I_0\,e_0
+ 2 I_o\,e_0 + \frac{4 kT}{R} \right) \Delta\nu_B}}.
\end{equation}
Iz zapisanega razberemo, da so razen šuma štetja vsi prispevki v 
imenovalcu neodvisni od intenzitete vpadne svetlobe. Le-ta je pri 
majhnih intenzitetah majhen in celoten šum 
zato praktično konstanten. V tem primeru $SNR$ narašča linearno z intenziteto
vpadne svetlobe. Pri velikih intenzitetah šum štetja prevlada nad ostalimi prispevki
in odvisnost $SNR$ od intenzitete postane korenska. 

\begin{naloga}
Oceni šum štetja, termični šum in šum temnega toka na silicijevi fotodiodi, če 
nanjo vpada svetloba z valovno dolžino $\lambda=850~\si{\nano\meter}$\index{Silicij}
\index{Fotodioda!polprevodniška}
in vpadno močjo $P=0,1~\si{\milli\watt}$. Kvantni izkoristek diode je $85~\%$,
spektralna širina $\Delta\nu_B=150~\si{\mega\hertz}$, temni tok $10~\si{\nano\ampere}$,
skupna upornost $50~\si{\ohm}$ in temperatura $300~\si{\kelvin}$. Pokaži, 
da je razmerje signal proti šumu $SNR\sim250$. 
\end{naloga}

Pomemben parameter, ki ga pogosto vpeljemo, je $NEP$ ({\it Noise Equivalent Power} -- 
moč, ki ustreza šumu).\index{NEP@$NEP$} To je vpadna moč svetlobe, ki je po velikosti primerljiva 
s šumom, in zato predstavlja spodnjo mejo še možne detekcije. To se navadno zgodi 
pri zelo nizkih močeh vpadne svetlobe, pri katerih je šum štetja zanemarljiv.
Pogoj, pri katerem je $SNR=1$, zapišemo kot
\begin{equation}
NEP\, \frac{e_0}{h \nu} \eta \approx \sqrt{\left(\frac{4 kT}{R} + 2 I_0\,e_0 \right) \Delta\nu_B}.
\end{equation}
Sledi
\begin{equation}
NEP = \frac{h \nu}{\eta e_0}\sqrt{\left(\frac{4 kT}{R} + 2 I_0\,e_0 \right) \Delta\nu_B}.
\label{NEP}
\end{equation}
\begin{naloga}
Izračunaj $NEP$ za primer germanijeve\index{Germanij} diode pri vpadni svetlobi z valovno dolžino
$\lambda = 1,5~\si{\micro\meter}$ in kvantnim izkoristkom $\eta=0,5$. Temperatura detektorja
je $T=300~\si{\kelvin}$ in temni tok $I_0=15~\si{\micro\ampere}$. Skupna upornost
je $R=2~\si{\kilo\ohm}$ in pasovna širina zajemanja svetlobe
$\Delta\nu_B=150~\si{\mega\hertz}$.
\end{naloga}

\begin{remark}
Zaradi priročnosti je pogosto podan $NEP$ na koren spektralne širine, saj ta ni 
karakteristična za detektor, ampak je odvisna od časa zajemanja. Podatek, ki 
ga navedejo proizvajalci detektorjev, je $NEP$ v enotah 
$\si{\watt}/\sqrt{\si{\hertz}}$. Tipične vrednosti so 
$10^{-11}$--$10^{-15}~\si{\watt}/\sqrt{\si{\hertz}}$, pri čemer najmanjše vrednosti
dosegajo silicijeve fotodiode. \index{Silicij}\footnote{~Glej npr. C. R. Pollock, {\it Fundamentals of Optoelectronics}, Irwin (1995).}
\end{remark}

\section{Heterodinska detekcija}
Heterodinska detekcija (pogosto imenovana tudi koherentna detekcija) je poseben način
\index{Heterodinska detekcija}
detekcije svetlobe, ki omogoča zaznavanje zelo šibkih signalov. Za razliko od direktne detekcije,
ki smo jo obravnavali do zdaj in pri kateri neposredno zaznavamo vpadne fotone, 
gre pri heterodinski detekciji za zaznavanje valovanja z amplitudo in fazo. Pri takem 
pristopu detektor svetlobe osvetlimo hkrati s signalom in z močno referenčno svetlobo, 
katere frekvenca se le malo razlikuje od frekvence signala (slika~\ref{11_Hetero}). 
Privzamemo, da sta oba snopa 
koherentna v času trajanja signala.
\begin{figure}[ht]
\centering
\def\svgwidth{70truemm} 
\input{slike/11_Hetero.pdf_tex}
\caption{Shema heterodinske detekcije}
\label{11_Hetero}
\end{figure}

Vpadni signal zapišemo z
\begin{equation}
E_s = E_{s0} \cos(\omega_st+\phi)
\end{equation}
in referenčnega z
\begin{equation}
E_r = E_{r0} \cos(\omega_rt),
\end{equation}
pri čemer je $E_{r0}$ konstanta. Če so valovne fronte obeh vpadnih snopov 
poravnane\footnote{~Dodaten pogoj je, da imata vpadna snopa isto polarizacijo in čim bolj 
podoben polmer.}, je intenziteta svetlobe, 
ki vpada na detektor, enaka
\begin{equation}
I = |E_s+E_r|^2 = E_{s0}^2 \, \cos^2(\omega_st+\phi)+
E_{r0}^2 \, \cos^2(\omega_rt) + 2E_{s0}E_{r0}\, \cos(\omega_st+\phi)\, \cos(\omega_rt).
\end{equation}
Prva dva člena v izrazu se zelo hitro spreminjata in zato predstavljata zgolj 
povprečen konstanten prispevek. 
Zanimiv je tretji člen, ki ga lahko zapišemo
kot
\begin{equation}
E_{s0}E_{r0}\left( \cos(\omega_st+\omega_rt+\phi)+\cos(\omega_st-\omega_rt+\phi)\right).
\end{equation}
Člen z vsoto obeh krožnih frekvenc se zelo hitro spreminja in se zato izpovpreči, 
drugi člen pa ne in ga lahko zaznavamo. 
Pri tem smo privzeli, da je razlika krožnih frekvenc dovolj majhna, da seže v odzivno območje
detektorja. Poseben primer, ko sta frekvenci povsem enaki, imenujemo homodinski režim 
detekcije. \index{Homodinska detekcija}

Ker je referenčna svetloba navadno bistveno močnejša od signalne, je celotna intenziteta
na detektorju enaka
\begin{equation}
I = |E|^2 \approx \frac{1}{2}E_{r0}^2 + E_{s0}E_{r0}\,\cos(\omega_st-\omega_rt+\phi).
\end{equation}
S tem znatno pridobimo na občutljivosti, saj na detektorju ne zaznavamo  
kvadrata majhnega signala $|E_{s0}|^2$, ampak majhen signal, pomnožen z velikim referenčnim. 

Poglejmo še razmerje $SNR$ za primer heterodinske detekcije. Največji prispevek k šumu je 
zaradi šuma štetja referenčne svetlobe, saj je ta praviloma bistveno močnejša od signala
\begin{equation}
\overline{(\Delta I)^2} = 2I_re_0 \Delta\nu_B= \varepsilon_0 
e_0^2\Delta\nu_B\frac{\eta c S E_{r0}^2}{h\nu},
\end{equation}
pri čemer smo z $I_r$ označili tok, ki steče zaradi referenčne svetlobe, in s $S$  
površino detektorja. 

Signal v tem primeru ni celotni vpadni signal, ampak kombinirani izhod iz 
detektorja, ki ga zaznavamo le pri razliki krožnih frekvenc $\omega_s-\omega_r$. Sledi
\begin{equation}
SNR = \frac{\frac{\varepsilon_0 e_0 c \eta S}{2h \nu} E_{s0}E_{r0}}{\sqrt{\frac{\varepsilon_0}{2}
e_0^2 \Delta\nu_B\frac{\eta c S E_{r0}^2}{h\nu}}} = 
\sqrt{\frac{\varepsilon_0\eta c S}{2 h \nu \Delta\nu_B}}E_{s0} = 
\sqrt{\frac{\eta P_s}{h \nu \Delta \nu_B}}.
\label{SNRhd}
\end{equation}
Če to primerjamo z vrednostjo $SNR$ pri navadni detekciji (enačba~\ref{SNRs}), 
vidimo,\index{Šum!{heterodinska detekcija}} 
da se razmerje signal proti šumu pri isti pasovni širini izboljša za faktor $\sqrt{2}$
(oziroma še več, če je prisoten še kakšen drug šum).\index{Razmerje signal proti šumu}
Ker je pri navadni detekciji težko meriti pri tako majhni pasovni širini, je razmerje
signal proti šumu v primeru heterodinske detekcije posledično praviloma znatno večje
od razmerja, ki ga dosežemo pri navadni detekciji. Prednost heterodinskega 
načina detekcije je tudi, da je neobčutljiv na svetlobo iz ozadja in se 
zato pogosto uporablja za detekcijo infrardečega valovanja.\index{Infrardeče valovanje}

Izračunajmo še najmanjšo moč svetlobe, ki jo lahko merimo s to metodo. Iz enačbe~(\ref{SNRhd}) 
sledi, da je $SNR=1$ pri $P_{\mathrm{min}} = h\nu \Delta \nu_B/\eta$. Če poenostavimo, da
je izkoristek enak $\eta = 1$, potem najmanjša merljiva vpadna moč svetlobe ustreza
enemu fotonu na $1/(\Delta\nu_B)$ časa. Ker detektor s pasovno širino $\Delta\nu_B$
ne zaznava procesov, ki so hitrejši od $\sim 1/(\Delta\nu_B)$, je to hkrati časovna ločljivost
detektorja. Najmanjša merljiva moč je torej en foton na enoto časovne ločljivosti 
detektorja.\footnote{~Glej npr. A. Yariv in P. Yeh, {\it Photonics}, šesta izdaja, Oxford
University Press (2007).}
