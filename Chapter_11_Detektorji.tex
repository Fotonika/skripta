\chapterimage{Mavrica.jpg} % Chapter heading image

\chapter{Detektorji svetlobe}

Detektorje svetlobe lahko razdelimo na dva tipa: kalorimetrične in kvantne. Pri prvih
zanzamo svetlobo tako, da merimo povečanje temperature senzorja zaradi absorbirane svetlobe.
Pri drugih pa na nek način direktno zaznavamo elektrone, ki so absorbirali svetlobne kvante.

\section{Termični detektorji}
Oglejmo si le  nekaj takih, ki se danes uporabljajo. Razlikujejo se predvsem po načinu
detekcije spremembe temperature. Tipalo mora biti pri vseh dobro počrnjeno, da absorbira
svetlobo v širokem spektralnem območju. Navadno tudi želimo, da je odziv hiter, zato mora 
biti tipalni element majhen.

Termočlen, katerega en spoj je počrnjen, drugi, referenčni, pa je zaščiten pred svetlobo.
Zaradi svetlobe se počrnjeni spoj segreje, med obema spojema nastane temperaturna razlika in 
zaradi termoelektričnega pojava tudi električna napetost. PAziti je treba, da je žica med 
obema spojema čim tanjša, da je prevajanje toplote po njej čim manjše. Napetosti niso velike,
pri temperaturni razliki ene stoopinje le nekaj deset mikrovoltov, zato včasih vežejo 
več termočlenov zaporedno.

Bolometer je počrnjen uporovni termometer, zvezan v Wheatstonov mostiček. Referenčni upor v
mostičku je enak kot merilni, le da je zaščiten pred svetlobo. Napterost na mostički je tako 
sorazmerna z razliko temperatur obeh uporov, torej z vpadlim svetlobnim tokom. Sama
sprememba temperature okolice ne moti, ker se zaradi nje temperatura obeh uporov enako spremeni. 
Kot uporovni termometer se pogosto uporablja termistor zaradi velikega temperaturnega koeficienta
upornosti.

Piroelektrični detektor je narejen iz ploščice snovi z lastno električno polarizacijo, ki 
je odvisna od temperature. Piroelektričnost  je dovoljena v kristalih, ki nimajo centra 
inverzije. Postavimo ploščico piroelektrika med dve elektrodi. Zaradi spremembe temperature se 
bo spremenila polarizacija, kar bomo zaznali kot premikalni tok. Spremembo polarizacije lahko zapišemo
kot 
\beq
dP = a dT,
\eeq
kjer je $a$ piroelektrični koeficient. Zaradi tega mora med obema elektrodama preteči naboj
\beq
de = I dt = S dP.
\eeq
Tok je torej
\beq
I = S a \frac{dT}{dt},
\eeq
kjer je $S$ površina detektorja. Piroelektrični detektor je torej občutljiv na časovni
odvod temperature detektorja, s tem pa tudi na spreminjanje vpadne sveltobne moči. 

Sprememba temperature bo sorazmerna razliki med vpadno svetlobno močjo in močjo, s katero
se detektor hladi v okolico. Hlajenje je deloma s prevanjanjem na podlago in po žicah, deloma pa s
sevanjem. Hlajenje s prevajanjem lahko zmanjšamo, s sevanjem pa ne, tako da je najmanjše
hlajenje podano kar s Stefanovim zakonom, v katerem lahko upoštevamo, da je razlika
v temperaturi med detektorjem in okolico majhna. Tedaj je sprememba temperature detektorja
\beq
c_r \varrho S b \frac{dT}{dt} = P_S - 4 \sigma S T_0^3 (T-T_0),
\eeq
kjer je $c_r$ specifična toplota piroelektrika, $b$ debelina detektorja, $T_0$ temperatura
okolice, $P_S$ pa vpadli svetlobni tok. Naj bo ta moduliran s frekvenco $\omega$. Tedaj se tudi
temperature spreminja s frekvenco $\omega$ in lahko pišemo
\beq
T - T_0 = T(\omega)e^{i \omega t}.
\eeq
Gornja enačba postane
\beq
c_r \varrho S b i \omega T(\omega) = P_S(\omega) - 4 \sigma S T_0^3 T(\omega),
\eeq
ali
\beq
i \omega T(\omega) = \frac{i \omega P_S}{G + i \omega C},
\eeq
kjer meri $G = 4 \sigma S T_0^3$ prevajanje toplote v okolico, $C$ pa toplotno
kapaciteto. Če to združimo s prvo enačbo, dobimo, da je tok piroelektričnega detektorja
pri frekvenci $\omega$ 
\beq
I(\omega) = \frac{Sai\omega P_S(\omega)}{G = i \omega C}.
\eeq
Tok je sorazmeren z $i\omega P_S$, torej s časovnim odvodom svetlobnega toka. Občutljivost
raste dofrekvence $\omega_T = G/C$, nato pa je konstantna do električne karakteristične frekvence,
določene z $RC$ konstanto merilnega kroga. Iz enačbe je tudi razvidno, da morata biti toplotna
kapaciteta in prevodnost čim manjši, da bo občutljivost čim večj.a

Če želimo s piroelektričnim senzorjem zaznavati konstanten svetlobni tok, da moramo najprej
periodično moduliratim, navadno to naredimo kar z mehanskim zaklopom. Piroelektrični detektorji
se uporabljajo kot dokaj ceneni in enostavni infrardeči detektorji, tudi v IR kamerah. 

Za vse kalorimetrične detektorje je značilno, da so neobčutljivi na valovno dolžino sveltobe. So
pa manj občutljivi od kvantnih detektorjev.

\section{Kvatni detektorji}

Kvantni detektorji temeljijo na fotoefektu ali na tvorbi parov elektron-vrzel
v polprevodnikih, čemur pravimo tudi notranji fotoefekt. V obeh primerih je potrebna neka
minimalna energija fotona, da ga detektor lahko zazna. Občutljivost je tako odvisna od 
valovne dolžine svetlobe. Pomemeben podatek za kvantne detektorje, ki je v tesni zvezi 
z občutljivostjo, je tudi kvantni izkotiste, to je delež vpadlih fotonov, ki povzročijo fotoefekt.

\section{Fotoefekt}

Najenostavnejši detektor na fotoefekt je fotocelica. Svetloba vpada na fotokatodo, tam povzroči
fotoefekt, elektrone z zunanjo napetostjo potegnemo na anodo in merimo tok. Da je čas preleta
elektronov od katode do anode čim krajši, je napetost na fotocelici pogosto nekaj kilovoltov.
Tedaj je lahko odzivni čas okoli 0,1~ns. Enostavnost in hitrost so torej prednosti fotocelice, slaba
stran pa majhna občutljivost.

\section{Fotopomnoževalka}
Fotopomnoževalke so fotocelice z vgrajenim ojačevanjem. Fotoelektron pospešimo z napetostjo
okoli 100-150~V na vmesno elektrodo - dinodo, kjer izbije nekaj (3-5) sekundarnih elektronov. 
Pomnoževanje se ponovi 8-14 krat, tako da dobimo na en fotoelektron na anodi $10^6 - 10^8$ 
elektronov, česar ni težko zaznati. Fotopomnoževalkanasm tako omogoča štetje posameznih fotonov.
Za manj zahtevne aplikacije pa pogosto merimo kar povprečni tok z anode. Zaravi delikega ojačenje
je treba tudi paziti, da fotopomnoževalke ne osvetlimo preveč. Spektralna občutljivost
tako fotocelice kot fotopomnoževalke je odvisna od sestave površine fotokatode. Večinoma
je največja v modrem delu spektra. Spodnja meja delovanja fotopomnoževalk je okoli $1~\si{\micro\metre}$.
Kvantni izkoristek je večinoma okoli $10~\%$, razen v bližnjem infrardečem območju, kjer
je znantno manjši. 

\section{Fotoprevodni detektorji}
V fotoprevodnikih svetloba z dovolj veliko energijo tvori pare elektron-vrzel, zaradi  česar
se poveča število nosilcev toka in s tem prevodnost. Hitrost odziva je odvisna od hitrosti
rekombinacije elektronov in vrzeli, to je okoli mikrosekunde po do desetin milisekund, kar
je dokaj počasi. 

\section{PN in PiN fotodiode}
Polvodniški detektroji so dveh vrst: fotoprevodniki in diodni. Spektralna občutljivost
obojih je odvisna od velikosti energijske reže emd valenčnim in prevodnim pasom. Silicij
z energijsko režo 1~eV je uporaben do valovne dolžine 1,2 mikron, germanij pa do 1,6 mikron.
V vidnem in bližnjem infrardečem področju deluje še GaAs, CdS in PbS, zadnja dva le kot
fotoprevodnika. Kot fotoprevodniki pri večjih valovnih dolžinah se uporabljajo  še PbTe
in HgCdTe. Posebej zadni je zelo pomemben, ker ima pri pravem razmerju Hg in Cd vrh občutljivosti
pri okoli 10 mikronov, to je ravno pri vrhu sevanja črnega telesa s temperaturo okoli 300~K. 
Z ustreznim doziranjem pa lahko polvodniški fotoprevodniki delujejo do valovnih dolžin nekaj 100 mikronov.
Detektorje  z majhno energijso režo je treba hladiti, navadno s tekočim dušikom, da zmanjšamo
termično vzbujanje elektronov v prevodni pas. Pri energiji fotonov blizu energijske reže 
imajo polvodniški detektorji zelo velik kvantni izkoriste, kar blizu 1.

Fotodiode zaznajo le pare elektron-vrzel v območju izpraznjenega sloja p-n spoja. Električno 
polje v tem sloju povzroči, da elektron odteče na eno stran, vrzel pa na drugo. Fotodopda tako
generira tok, ki je sorazmeren z vpadlo svetlobno močjo. Zaradi nelinearne zveze med tokom in napetostjo
na diodi je treba skrbeti, da je napetost na diodi vselej 0 ali celo v zaporni smeri, kar lahko
dosežemo z zunanjo napetostjo v zaporni smeri. S tem tudi dosežemo hitrejši odziv, ker 
se zmajša kapacitivnost p-n spoja. 

Da se bo svetloba absorbirala ravno v p-n spoju, mora biti seveda spoj pri površini dopde, obenem
pa mora biti spoj čim debelejši, kar dosežejo s tem, da je med  p in n stranjo še plast
čistega polvodnika. Takim diodam pravimo pin diode. 

V pn spoju je mooče doseči tudi pomnoževanje nastalih elektronov in vrzeli, če je zaporna napetost dovolj 
velika, da lahko fotoelektron dobi toliko energije, preden zapusti izpraznjeni sloj, da tvori nove pare.
Taka pomnoževalna dioda je zelo občutljiva, vendar ima tudi večji šum.

\section{Plazovne fotodiode}
 
\section{CCD in CMOS detektorji}
 
\section{Šum pri optični detekciji}
Sama občutljivost nekega detektorja, to je količina A/W za neko fotodiodo, nam ne pove, kolikšen 
je najmanjši signal, ki ga je mogoče meriti. Ta je določen s šumom detektorja in pa s časom 
meritve, to je s frekvenčno širino modulacije svetlobe, ki jo želimo meriti. Največja 
možna frekvencčna širina je seveda določena s hitrostjo odziva detektorja.

Ocenimo fluktuacije, torej šum, toka nekega detektorja, na primer fotodiode. V času
meritve $\tau$ steče $n$ elektronov. Tok je torej
\beq
I = ne/\tau.
\eeq
Povprečni tok je seveda
\beq
\overline{I} = \overline{n}e/\tau.
\eeq
Napravimo mnogo meritev, vse dolge $\tau$. Ker so posamezni foto dogodki med seboj neodvisni, 
lahko predpostavimo, da je porazdelitev pretečenih elektronov Poissonova: pa lahko izračunamo
povprečni kvadrat fluktuacij toka
\beq
\sigma^2 = \overline{(I-\overline{I})^2} = \frac{e^2}{\tau^2} \overline{(n-\overline{n})^2} = 
\frac{e^2}{\tau^2}\overline{n} = \frac{e\overline{I}}{\tau}.
\eeq
Tokovni šum je obratno sorazmeren korenu iz časa merjenja, oziroma sorazmeren s korenom
iz širine merjenjega frekvenčnega intervala. Najmanjši povprečni tok diode je tok zaradi
termičnega vzbujanja, s tem je seveda po enačbi določen tudi osnovni termični šum diode.
Temni tok diode je sorazmeren s površino diode in eksponentno odvisen od temperature in 
energijske reže polvodnika;
\beq
I_0 = Sj_0 e^{-E_g/kT}
\eeq
Tipičen temni tok silicijeve fotodiode s površino 1~mm$^2$ je pri sobni temperaturi 1~nA. 
Tokovni šum je torej po gornji enačbi
\beq
I_N = \sqrt{eI_0\Delta \nu} \approx 10^{-15}~A/\sqrt{Hz} \sqrt{\Delta \nu}.
\eeq

Svetlobni tok, označen pogosto z NEP (noise equivalent power), ki da enak signal, je $10^4 $
fotonov na sekundo ali $10^{-14}$~W. Še beseda o enoti za NEP. Navadno je podan na enoto 
frekvenčnega intervala, to je po enačbi .. na $\sqrt{Hz}$. 