\chapterimage{slike/Trinivojski.png} 
% Chapter heading image
\chapter{Interakcija svetlobe s snovjo}

V prejšnjih poglavjih smo obravnavali svetlobo v praznem prostoru. Oglejmo si
zdaj osnovne procese interakcije svetlobe s snovjo. To je seveda zelo
obširna tema in jo bomo obdelali le v obsegu, potrebnem za
razumevanje ojačevanja svetlobe s stimulirano emisijo, kar je osnova za
delovanje laserjev. Najprej bomo na kratko pogledali termodinamsko ravnovesje 
svetlobe v stiku s toplotnim zalogovnikom, torej sevanje črnega telesa, ki 
zahteva kvantno obravnavo elektromagnetnega polja. Nato bomo vpeljali fenomenološki
Einsteinov opis mikroskopskih procesov absorpcije, spontane in stimulirane
emisije in pokazali, da ti procesi niso neodvisni. Izpeljali bomo
izraze za absorpcijski koeficient in koeficient ojačenja. Na koncu poglavja
bomo nakazali še kvantnomehansko izpeljavo verjetnosti za prehod
atoma iz višjega energijskega stanja v nižje s sevanjem.

\section{Kvantizacija elektromagnetnega polja}
\index{Kvantizacija polja}
Ravni valovi\index{Ravni val} so enostavne in zelo prikladne rešitve valovne 
enačbe~(enačba~\ref{eq:valovna-skalarna}), po katerih navadno razvijemo elektromagnetno polje. Razvoj
lahko naredimo po celotnem prostoru, vendar je tedaj nekoliko nerodna normalizacija. 
Če se omejimo na le del prostora, se temu problemu izognemo. Izbrani del prostora 
mora biti dovolj velik, da končni rezultat ni odvisen od izbire 
njegove velikosti in oblike.

Najpreprosteje je vzeti votlino v obliki velike kocke s stranico
$L$ in idealno prevodnimi stenami. Rešitve Maxwellovih enačb~(enačbe~\ref{eq:Maxwell1}--\ref{eq:Maxwell4}) 
znotraj take votline so ob upoštevanju robnih pogojev 
(enačbe~\ref{eq:robni-pogoji}--\ref{eq:robni-pogoji5}) 
\index{Stoječe valovanje}stoječa valovanja. Zapišemo jih v obliki
\begin{align}
E_{x} & =  E_{x0}\cos\left(\frac{\pi lx}{L}\right)\sin\left(\frac{\pi my}{L}\right)\sin\left(\frac{\pi nz}{L}\right)e^{-i\omega t}\nonumber \\
E_{y} & =  E_{y0}\sin\left(\frac{\pi lx}{L}\right)\cos\left(\frac{\pi my}{L}\right)\sin\left(\frac{\pi nz}{L}\right)e^{-i\omega t}\nonumber \\
E_{z} & =  E_{z0}\sin\left(\frac{\pi lx}{L}\right)\sin\left(\frac{\pi my}{L}\right)\cos\left(\frac{\pi nz}{L}\right)e^{-i\omega t},
\label{eq:stojece_votlina}
\end{align}
kjer so $l,m$ in $n$ cela števila. Vsako stoječe valovanje je določeno z valovnim 
vektorjem\index{Valovni vektor}\index{Valovno število}
\begin{equation}
\mathbf{k}=\left(\frac{\pi l}{L},\frac{\pi m}{L},\frac{\pi n}{L}\right),
\end{equation} 
katerega velikost je povezana s krožno frekvenco $|\mathbf{k}|= k = \omega/c$.
Iz Maxwellove enačbe za prazen prostor $\nabla\cdot\mathbf{E}=0$ (enačba~\ref{eq:Maxwell3})
sledi $\mathbf{k}\cdot\mathbf{E}=0$. 
Za vsako trojico števil $l$, $m$ in $n$ obstajata tako dve
neodvisni polarizaciji.

\begin{definition}
 Pokaži, da stoječe valovanje, zapisano z enačbami~(\ref{eq:stojece_votlina}), reši 
 valovno enačbo (enačba~\ref{eq:valovna-skalarna}) v 
 kocki s stranico $L$ in zadosti robnim pogojem idealno prevodnih sten votline.
\end{definition}

Preštejmo, koliko je stoječih valovanj oziroma lastnih nihanj 
v intervalu velikosti valovnega
vektorja med $k$ in $k+dk$ -- to smo na hitro naredili že pri obravnavni
resonatorjev (enačba~\ref{eq:N-stevilo-stanj}). Mogoči valovni vektorji tvorijo tridimenzionalno
mrežo v prvem oktantu prostora vseh valovnih vektorjev. Razmik med
dvema zaporednima mrežnima točkama v smeri ene od osi je $\pi/L$.
Število točk v osmini krogelne lupine med $k$ in $k+dk$ je za dovolj
velike $l$, $m$ in $n$ enako prostornini lupine, deljeni
s prostornino, ki pripada posamezni mrežni točki, to je $(\pi/L)^{3}$.
Upoštevati moramo še, da sta pri vsakem $\mathbf{k}$ dovoljeni dve polarizaciji, zato
\begin{equation}
dN=\left(\frac{L}{\pi}\right)^{3}\pi k^{2}\, dk.
\label{4.2}
\end{equation}
Zapišemo število lastnih nihanj na enoto volumna
\begin{equation}
\frac{dN}{V}=\frac{ k^{2}}{\pi^{2}} dk
\label{4.3}
\end{equation}
in ga prevedemo na frekvenčno odvisnost
\begin{equation}
\frac{dN}{V}=\frac{8 \pi \nu^{2} }{c^{3}}d\nu = \frac{\omega^2}{\pi^2c^3}d\omega.
\end{equation}
Vpeljemo gostoto stanj  $\varrho (\omega)$, to je število lastnih nihanj na 
frekvenčni interval\footnote{V tem poglavju bomo ohlapno uporabljali besedo
frekvenca tudi za krožno frekvenco. Iz zapisa bo vedno jasno, za katero frekvenco gre.}
na enoto volumna votline\index{Gostota stanj}
\boxeq{4.4}{
\rho(\omega)=\frac{dN}{V d\omega}=\frac{\omega^{2}}{\pi^{2}c^{3}}.
}

Vsote po lastnih nihanjih, to je po dovoljenih vrednostih valovnega števila $k$,
z uporabo gostote stanj spremenimo v integrale po $k$ ali po $\omega$
\begin{equation}
\sum_{k}\ldots \quad \Rightarrow \quad V\int\rho(k)\ldots dk=V\int\rho(\omega)\ldots d\omega.
\label{4.5}
\end{equation}

Označimo brezdimenzijski krajevni del rešitve~(enačbe~\ref{eq:stojece_votlina}) z 
$\mathbf{E}_{\alpha}$, kjer $\alpha$\index{Električno polje!jakost}
\index{Magnetno polje!gostota}
označuje trojico števil $l$, $m$ in $n$ in še obe polarizaciji. 
Pripadajoče magnetno polje izračunamo z Maxwellovo enačbo (enačba~\ref{eq:Maxwell2}) 
\begin{equation}
\nabla\times\mathbf{E}_{\alpha}=i\omega_\alpha\mathbf{B}_{\alpha}.
\label{Maxalfa}
\end{equation}
Polja $\mathbf{E}_{\alpha}$ in $\mathbf{B}_{\alpha}$ tvorijo kompleten ortogonalen
sistem, zato jih lahko uporabimo za razvoj poljubnega elektromagnetnega polja v votlini
\begin{align}
\mathbf{E}(\mathbf{r},t) & =  -\frac{1}{\sqrt{V\epsilon_{0}}}
\sum_{\alpha}p_{\alpha}(t)\mathbf{E}_{\alpha}(\mathbf{r}) \quad \mathrm{in} \label{eq:pqrazvoj1}\\
\mathbf{B}(\mathbf{r},t) & =  i\sqrt{\frac{\mu_{0}}{V}}c_0\sum_{\alpha}
\omega_{\alpha}q_{\alpha}(t)\mathbf{B}_{\alpha}(\mathbf{r}).
\label{eq:pqrazvoj}
\end{align}
Vstavimo splošen razvoj~(enačbi~\ref{eq:pqrazvoj1} in \ref{eq:pqrazvoj}) v 
Maxwellovi enačbi~(enačbi~\ref{eq:Maxwell1} in \ref{eq:Maxwell2}), 
upoštevamo zvezo~(enačba~\ref{Maxalfa}) in njej analogno za rotor magnetnega polja. Sledi
\begin{equation}
p_{\alpha}=\dot{q}_{\alpha} \qquad \mathrm{in} \qquad 
\omega_{\alpha}^{2}q_{\alpha}=-\dot{p}_{\alpha},
\label{4.7}
\end{equation}
od koder sledi še 
\begin{equation}
\ddot{p}_{\alpha}+\omega_{\alpha}^{2}p_{\alpha}=0.
\label{4.8}
\end{equation}
Ta enačba da seveda pričakovano časovno odvisnost oblike $e^{-i \omega_\alpha t}$.

\begin{definition}
 Uporabi razvoj polja (enačbi~\ref{eq:pqrazvoj1} in \ref{eq:pqrazvoj}) 
 in iz Maxwellovih enačb izpelji
 enačbo~(\ref{4.8}).
\end{definition}

Z upoštevanjem razvoja (enačbi~\ref{eq:pqrazvoj1} in \ref{eq:pqrazvoj}) in ob ustrezni
normalizaciji zapišemo energijo 
polja ali \index{Hamiltonova funkcija}Hamiltonovo 
funkcijo
\begin{equation}
{\cal H}=\frac{1}{2}\int(\epsilon_{0}E^{2}+\frac{B^{2}}{\mu_0})\, 
dV=\frac{1}{2}\sum_{\alpha}(p_{\alpha}^{2}+\omega_{\alpha}^{2}q_{\alpha}^{2}).
\label{4.9}
\end{equation}
Gornji zapis (enačbi \ref{4.8} in \ref{4.9}) kaže, 
da lahko elektromagnetno polje v votlini
obravnavamo kot sistem neodvisnih enodimenzionalnih harmonskih oscilatorjev\index{Harmonski oscilator}. 
Pri tem se koeficienti razvoja $p_{\alpha}$ in $q_{\alpha}$ obnašajo kot
gibalne količine in koordinate. 

Prehod v kvantno mehaniko dosežemo tako, da klasičnim spremenljivkam gibalne količine
in koordinate priredimo operatorje $\hat{p}_{\alpha}$ in $\hat{q}_{\alpha}$,
ki morajo zadoščati komutacijskim pravilom 
\begin{equation}
[\hat{q}_{\alpha},\hat{p}_{\beta}]=i\hslash \delta_{\alpha, \beta}.
\label{4.10}
\end{equation}

Iz kvantne mehanike vemo, da so lastne vrednosti energije posameznega harmonskega oscilatorja, 
opisanega s Hamiltonovo funkcijo (enačba~\ref{4.9}), diskretne. Njihove vrednosti so enake
\boxeq{4.11}{
W_{n}=\hslash\omega(n+\frac{1}{2}), \quad n= 0, 1, 2 \ldots
}
{\bf Razliki energije harmonskega oscilatorja, če se \textit{\textbf{n}}
spremeni za 1, pravimo foton.}\index{Foton} Energija
fotona je enaka $\hslash \omega$, $n$ pa predstavlja število fotonov. Celotna
energija kvantiziranega elektro\-magnetnega polja v votlini je vsota prispevkov
posameznih oscilatorjev, pri čemer ničelno energijo zanemarimo
\begin{equation}
W = \sum_\alpha \hslash \omega_\alpha n_\alpha.
\end{equation}
Tudi v nadaljevanju bomo ničelno energijo izpuščali, 
saj je to energija osnovnega stanja, ki se ne more sprostiti. 

\begin{remark}
Vidna svetloba z valovno dolžino $500~\si{\nano\metre}$ ima frekvenco
$\nu = 6 \cdot 10^{14}~\si{\hertz}$. Ustrezna energija fotona je
$W = 4 \cdot 10^{-19}~\si{\joule}$ oziroma $W = 2,5~\mathrm{e}\si{\volt}$.
To vrednost si velja zapomniti.
\end{remark}

\section{Sevanje črnega telesa}
Obravnavajmo sevanje v votlini, ki je v toplotnem ravnovesju s stenami s temperaturo
$T$. Izberemo lastno nihanje votline s krožno frekvenco $\omega$. Iz statistične fizike 
vemo, da verjetnost $P$, da je v izbranem  nihanju število fotonov enako $n$, 
zapišemo z Boltzmannovo porazdelitvijo\index{Boltzmannova porazdelitev}
\begin{equation}
P(n)=\frac{e^{-W_{n}/k_BT}}{\sum_{n}e^{-W_{n}/k_BT}} = 
\frac{e^{-\beta\hslash\omega n}}
{\sum_{n}e^{-\beta\hslash\omega n}}=
e^{-\beta\hslash\omega n}(1-e^{-\beta\hslash\omega}),
\label{4.12}
\end{equation}
pri čemer je $\beta = 1/k_BT$ in $k_B$ Boltzmannova konstanta. 

Povprečno število fotonov v tem nihanju je potem \index{Sevanje črnega telesa}
\begin{equation}
\langle n\rangle =\sum_{n}n P(n)=\frac{1}{e^{\beta\hslash\omega}-1}.
\label{4.13}
\end{equation}

Povprečno energijo izbranega lastnega nihanja zapišemo kot produkt energije fotona in 
povprečnega števila fotonov
\begin{equation}
\langle W\rangle = \hslash \omega \langle n \rangle
= \frac{\hslash \omega}{e^{\beta\hslash\omega}-1}.
\end{equation}
Ravnovesno gostoto energije elektromagnetnega polja\index{Gostota energije} v votlini na
frekvenčni interval izračunamo tako, da povprečno energijo posameznega
nihanja $\langle W \rangle$ pomnožimo z gostoto stanj $\varrho (\omega)$ 
(enačba~\ref{4.4}). Dobimo znan izraz za energijo na enoto volumna na enoto frekvence, 
to je \index{Planckov zakon}Planckov zakon\footnote{Nemški fizik in nobelovec 
Max Karl Ernst Ludwig Planck, 1858--1947.} (slika~\ref{fig:Planck}).
Planckov zakon opiše spektralno gostoto energije svetlobe $u$ 
\index{Spektralna gostota energije}, izsevane iz \index{Spekter!Planckov}
\index{Sevanje črnega telesa}črnega telesa, ki je v toplotnem ravnovesju z 
okolico s temperaturo $T$
\boxeq{eq:Planck}{
u(\omega)=\hslash\omega\langle n\rangle \rho(\omega)
=\frac{\hslash}{\pi^{2}c^{3}}\frac{\omega^{3}}{e^{\beta\hslash\omega}-1}.
}

\begin{figure}[h]
\centering
\def\svgwidth{100truemm} 
\input{slike/05_PlanckOmega.pdf_tex}
\caption{Planckov spekter za sevanje črnega telesa pri različnih temperaturah}
\label{fig:Planck}
\end{figure}

\section{Absorpcija, spontano in stimulirano sevanje}
\label{chap:ASSS}
Oglejmo si osnovne procese interakcije svetlobe s snovjo. Naj
bo v votlini poleg elektro\-mag\-net\-nega polja še $N$ atomov, ki se med
seboj ne motijo. Za začetek naj bodo atomi prav enostavni\index{Dvonivojski sistem}:
imajo naj le dva energijska nivoja z energijama $E_{1}$ in $E_{2}$ (slika~\ref{sl4.1}\,a),
pri čemer naj bo $E_2$ višja energija od $E_1$. Razliko med energijama nivojev zapišemo kot 
\begin{equation}
 E_2 - E_1 = \hslash \omega_0.
\end{equation}

Zaradi interakcije s poljem
atomi prehajajo iz nižjega nivoja v višji in obratno. Prehajanje 
med nivojema opisujejo trije procesi: 
absorpcija, spontano sevanje in stimulirano sevanje.

\begin{figure}[h]
\centering
\def\svgwidth{145truemm} 
\input{slike/05_Dvonivojski.pdf_tex}
\caption{Shema energijskih nivojev dvonivojskega atoma (a) in treh vrst prehodov med njima:
absorpcija (b), spontano sevanje (c) in stimulirano sevanje (d). Črna črta označuje atomski
prehod, rdeča pa foton.}
\label{sl4.1}
\end{figure}

\subsection*{Absorpcija fotona}
Absorpcija fotona\index{Absorpcija fotona} je prehod, pri katerem se foton 
z ustrezno energijo absorbira, atom pa preide iz nižjega energijskega nivoja 
v višje (slika~\ref{sl4.1}\,b). 
Verjetnost za prehod na časovno enoto\index{Verjetnost za prehod}, ki jo označimo z $r_{12}$, 
je sorazmerna spektralni gostoti energije polja \index{Spektralna gostota energije}$u(\omega)$, 
to je energiji na enoto volumna in frekvenčni interval, pri frekvenci prehoda $\omega_{0}$.
Sorazmernostni koeficient označimo z $B_{12}$ in 
zapišemo
\begin{equation}
r_{12}=B_{12}u(\omega_{0}).
\label{4.16}
\end{equation}
To je enostavno razumeti. Več kot je fotonov v votlini pri frekvenci prehoda, 
več fotonov se lahko absorbira in večja je verjetnost za prehod. Pri absorpciji se
seveda število fotonov v enem od lastnih nihanj polja pri frekvenci
$\omega_{0}$ zmanjša za ena.

\subsection*{Spontano sevanje}
Atom v vzbujenem stanju ni stabilen, temveč se prej ali slej spontano vrne 
v osnovni nivo, pri čemer izseva foton. Temu pojavu pravimo spontano sevanje\index{Spontano sevanje} 
ali spontana emisija (slika~\ref{sl4.1}\,c).

Pri spontanem sevanju je foton izsevan v katerokoli stanje polja v bližini 
frekvence prehoda. Smer izsevane svetlobe je poljubna, v odsotnosti zunanjega polja
je poljubna tudi polarizacija izsevane svetlobe.
Verjetnost za prehod na časovno enoto\index{Verjetnost za prehod} označimo z $A_{21}$.
Za dovoljene prehode je vrednost $A_{21} \sim 10^6$--$10^8/\si{\second}$ in 
za prepovedane okoli $\sim 10^4/\si{\second}$. Karakteristični (naravni) 
razpadni čas\index{Razpadni čas} vzbujenega stanja vpeljemo kot
$\tau = 1/A_{21}$. 

Zaradi končnega življenjskega časa ima vzbujeno stanje končno spektralno 
širino. Če ni Dop\-pler\-je\-ve razširitve, je atomska spektralna 
črta $g$ \index{Spektralna črta} najpogosteje kar 
Lorentzove oblike\index{Spekter!Lorentzov} z vrhom pri $\omega_0$
(enačba~\ref{eq:spekter-primer})
\boxeq{4.21}{
g(\omega-\omega_0)=\frac{1}{\pi}\frac{\gamma}{(\omega-\omega_{0})^{2}+\gamma^{2}}.
}
Funkcija $g(\omega)$ je normirana, tako da velja
\begin{equation}
\int_{-\infty}^\infty g(\omega)\, d\omega=1.
\label{4.20}
\end{equation}
Za grobe ocene se funkcijo $g$ pogosto nadomesti s pravokotnikom širine
$2\gamma$ in višine $1/2\gamma$.


\subsection*{Stimulirano sevanje}
Tretji pojav je prehod atoma iz višjega nivoja v nižjega zaradi interakcije
s poljem. Ko na vzbujen atom vpade foton, se atom vrne v osnovni nivo in pri 
tem izseva foton, ki je povsem enak vpadnemu (slika~\ref{sl4.1}\,d). 
Temu procesu pravimo stimulirano sevanje\index{Stimulirano sevanje} ali 
stimulirana emisija. Tudi verjetnost za stimuliran prehod na časovno enoto $r_{21}$ 
je sorazmerna s spektralno gostoto energije polja pri frekvenci prehoda $\omega_{0}$
\begin{equation}
r_{21}=B_{21}u(\omega_{0}).
\label{4.17}
\end{equation}
V tem primeru smo sorazmernostni koeficient označili z $B_{21}$. Kadar pride do
stimuliranega sevanja, se število atomov v vzbujenem stanju zmanjša, 
število fotonov v stanju, ki je prehod povzročilo, pa se poveča. Pri tem je 
ključnega pomena, da je foton, ki nastane pri stimuliranjem sevanju, enak vpadnemu fotonu.
Izsevana svetloba ima tako enako fazo, frekvenco, polarizacijo in smer potovanja kot 
vpadna. Tipične vrednosti parametra so $B_{21} \sim 10^{16}$--$10^{20}~\si{\metre^3/\joule\second^2}$.

Poglejmo bolj natančno izraza za absorpcijo
(enačba~\ref{4.16}) in stimulirano emisijo (enačba~\ref{4.17}).
Zapisani enačbi veljata le, kadar je spektralna gostota \index{Spektralna gostota energije}
elektromagnetnega polja $u(\omega)$
znotraj celotne spektralne širine prehoda $g(\omega - \omega_0)$ približno konstantna 
(slika~\ref{fig:spektri}\,a). To je gotovo res, če
obravnavamo sevanje v votlini, ki je v termičnem ravnovesju z okolico (črno telo).

\begin{figure}[h]
\centering
\def\svgwidth{140truemm} 
\input{slike/05_Spektri.pdf_tex}
\caption{Pri izračunu verjetnosti za absorpcijo in stimulirano emisijo je 
pomembna oblika spektralne gostote vpadnega elektromagnetnega polja $u(\omega)$. V prvem primeru (a) je 
bistveno širša, v drugem (b) pa bistveno ožja od širine atomske spektralne črte $g(\omega-\omega_0)$.}
\label{fig:spektri}
\end{figure}

V splošnem primeru, ko se spekter vpadne svetlobe spreminja znotraj 
atomske spektralne črte, moramo sešteti prispevke po ozkih frekvenčnih intervalih
\begin{equation}
r_{12}=B_{12}\int g(\omega-\omega_0)\, u(\omega)\, d\omega.
\label{4.19}
\end{equation}
Zapis preverimo na primeru spektra črnega telesa, ki se ne spreminja 
dosti v območju prehoda. Takrat $u(\omega_0)$ postavimo pred integral in po pričakovanju
dobimo znano zvezo~(enačba~\ref{4.16}). 

Če na atome svetimo s svetlobo s spektrom, ki je ozek v primerjavi s spektralno 
širino prehoda (na primer iz laserskega resonatorja), je verjetnost za prehod 
odvisna od tega, kako blizu osrednje frekvence prehoda je frekvenca vpadne 
svetlobe (slika~\ref{fig:spektri}\,b). Naj bo  $w_{R}$ gostota energije skoraj
monokromatske vpadne svetlobe s frekvenco $\omega_R$. Verjetnost za absorpcijo na časovno 
enoto je potem  \index{Gostota energije}
\boxeq{4.18}{
r_{12}=B_{12}g(\omega_R-\omega_0)\, w_{R}.
}

Koeficiente $A_{21}$, $B_{12}$ in $B_{21}$, s katerimi smo opisali spontano sevanje,
absorpcijo in stimulirano emisijo, je prvi vpeljal Einstein\footnote{Nemški fizik
in nobelovec Albert Einstein, 1879--1955.}, zato jih imenujemo 
Einsteinovi koeficienti\index{Einsteinovi koeficienti}. 

\subsection*{Einsteinovi koeficienti}
\label{AB}\index{Einsteinovi koeficienti}
Število atomov v določenem atomskem nivoju imenujemo zasedenost stanj.\index{Zasedenost stanj} 
Ker zaenkrat obravnavamo preproste modele atomov z zgolj 
dvema nivojema, zapišemo samo dve zasedenosti\index{Dvonivojski sistem}. Naj bo $N_1$ zasedenost 
nižjega nivoja, $N_{2}$ zasedenost višjega nivoja in skupno število atomov 
$N_1+N_2=N$. V prisotnosti svetlobe se število atomov v spodnjem in zgornjem 
nivoju lahko spreminja, skupno število pa se ohranja.

Obravnavajmo termično ravnovesje, ko je spekter svetlobe bistveno širši
od širine atomskega prehoda (slika~\ref{fig:spektri}\,a). Verjetnosti za prehoda potem
zapišemo z enačbama~(\ref{4.16}) in (\ref{4.17}). Zasedenost višjega nivoja $N_2$
se zmanjšuje zaradi spontanih in stimuliranih prehodov v nižji nivo in
povečuje zaradi absorpcije. To zapišemo z enačbo
\begin{equation}
\frac{dN_{2}}{dt}=-A_{21}N_2 - r_{21}N_2 + r_{12}N_1 = 
-A_{21}N_{2}-B_{21}u(\omega_{0})N_{2}+B_{12}u(\omega_{0})N_{1}.
\label{4.22}
\end{equation}
Zaradi ohranitve skupnega števila atomov velja 
\begin{equation}
\frac{dN_{1}}{dt}=-\frac{dN_{2}}{dt}.
\end{equation}
V termičnem ravnovesju sta zasedenosti konstantni, tako da lahko zapišemo 
\begin{equation}
\frac{dN_{1}}{dt}=A_{21}N_{2}+B_{21}u(\omega_{0})N_{2}-B_{12}u(\omega_{0})N_{1}=0.
\label{4.23}
\end{equation}
Vemo tudi, da v termičnem ravnovesju za zasedenosti $N_{1}$ in $N_{2}$ velja
Boltzmannova porazdelitev\index{Boltzmannova porazdelitev}
\begin{equation}
\frac{N_{2}}{N_{1}}=e^{-\beta(E_{2}-E_{1})} = e^{-\beta \hslash \omega_0},
\label{4.25}
\end{equation}
kjer je $\beta=1/k_BT$. Izrazimo spektralno gostoto\index{Spektralna gostota energije} $u(\omega_0)$ 
iz enačbe~(\ref{4.23})
\begin{equation}
u(\omega_{0})=\frac{A_{21}}{B_{12}\frac{N_{1}}{N_{2}}-B_{21}}
\label{4.24}
\end{equation}
in z uporabo enačbe~(\ref{4.25}) dobimo
\begin{equation}
u(\omega_{0})=\frac{A_{21}/B_{12}}{e^{\beta\hslash\omega_{0}}-B_{21}/B_{12}}.
\label{4.26}
\end{equation}
Po drugi strani vemo, da v termičnem ravnovesju spektralno gostoto energije sevanja
$u(\omega_0)$ opišemo s Planckovim zakonom~(enačba~\ref{eq:Planck}).
Iz primerjave obeh zapisov ugotovimo, da morata biti koeficienta $B_{21}$ in $B_{12}$ enaka in
določimo zvezo med koeficientoma $A_{21}$ in $B_{12}$  
\boxeq{4.27}{
A_{21}=\frac{\hslash\omega^{3}}{\pi^{2}c^{3}}\, B_{12} \qquad \mathrm{in} \qquad B_{12}=B_{21}.
}
Koeficient pred $B_{12}$ v prvi enačbi je ravno enak gostoti stanj elektromagnetnega polja 
$\rho(\omega)$ (enačba~\ref{4.4}), pomnoženi z energijo fotona $\hslash\omega$. 
Videli bomo, da to ni slučaj, saj to izhaja iz verjetnosti za prehod v kvantni 
elektrodinamiki (razdelek~\ref{chap:verjetnost}).
Pozoren bralec je lahko tudi opazil, da je z enačbo~(\ref{4.26}),
ki smo jo dobili le z uporabo Boltzmannove porazdelitve za atome, že
določena oblika Planckove formule, ne da bi kar koli rekli o fotonih.

\begin{remark}
 Zveza $B_{12}=B_{21}$ velja le v primeru nedegeneriranih stanj. V realnih sistemih
 so stanja pogosto degenerirana in je treba zvezo $B_{12}=B_{21}$ ustrezno popraviti v
\begin{equation}
\frac{B_{21}}{B_{12}} = \frac{g_1}{g_2},
\label{eq:ABdeg}
\end{equation}
pri čemer $g_{1}$ in $g_2$ označujeta degeneriranost stanj. 
\end{remark}

\section{Absorpcijski koeficient}
\index{Absorpcija}
V plinu dvonivojskih atomov naj bo zasedenost osnovnega nivoja $N_1$ in zasedenost vzbujenega
$N_2$. Izbran volumen takega plina osvetlimo s snopom svetlobe s frekvenco
$\omega$, ki je blizu frekvence atomskega prehoda $\omega_{0}$. Gostota
vpadnega energijskega toka je $j=w_{\omega}c$ (enačba~\ref{eq:jcw}), 
pri čemer je $w_{\omega}$ gostota energije. Obravnavajmo primer, ko je 
spekter vpadnega snopa ozek v primerjavi s širino atomskega prehoda
(slika~\ref{fig:spektri}\,b). V tej obliki je zapis enačb sicer bolj zapleten,
a bo bolj priročen pri obravnavi laserja. Privzemimo še, da
je stanje stacionarno. 

Ko svetlobni snop vpade na plast plina debeline $dz$, se gostota
energijskega toka zmanjša zaradi absorpcije in hkrati poveča zaradi 
stimulirane emisije (slika~\ref{fig:abs}). 
Spontano sevanje, ki je tudi prisotno, lahko zanemarimo, saj
je svetloba izsevana na vse strani enakomerno in le majhen del je izsevan v smeri snopa.
Sprememba energije snopa v časovni enoti je enaka razliki med 
številom absorpcij in stimuliranih prehodov v tem času, pomnoženih z 
energijo fotona\footnote{Zaradi preglednosti tukaj pišemo obliko atomske spektralne črte kot $g$, 
pri čemer je to vrednost Lorentzove krivulje z osrednjo frekvenco $\omega_0$ pri $\omega$, 
torej $g(\omega-\omega_0)$.}
\begin{equation}
dP=r_{12}\,\frac{(N_{2}-N_{1})}{V}\,\,S dz\, \, \hslash\omega = 
\frac{(N_{2}-N_{1})}{V}\,B_{21}g w_{\omega} \, \hslash\omega \,S dz.
\label{4.28}
\end{equation}
Verjetnost za prehod smo izrazili iz enačbe~(\ref{4.18}),
$S$ označuje presek snopa in $V$ volumen plina. 
\begin{figure}[h]
\centering
\def\svgwidth{70truemm} 
\input{slike/05_Absorpcija.pdf_tex}
\caption{Ob prehodu skozi plin se vpadna svetloba absorbira in $dj < 0$.}
\label{fig:abs}
\end{figure}

Sledi
\begin{equation}
dj=\frac{(N_{2}-N_{1})}{V}\, B_{21}g\, \hslash\omega\,\frac{j}{c}\, dz.
\label{4.29}
\end{equation}
Priročno je vpeljati presek za absorpcijo\index{Presek za absorpcijo} 
\boxeq{sigmaabs}{
\sigma(\omega)=\frac{B_{21}\, g\, \hslash\omega}{c}.
}
Z njim se izraz (\ref{4.29}) poenostavi v 
\boxeq{4.30}{
\frac{dj}{dz}=\frac{\Delta N}{V}\sigma(\omega)j,
}
kjer je $\Delta N = N_{2}-N_{1}$.
Navadno obravnavamo pline, ki so blizu termičnega ravnovesja. V tem primeru 
je $N_{2}<N_{1}$ in $dj$ negativen in svetloba se absorbira.
Zapišemo 
\begin{equation}
\frac{dj}{j} = -\mu dz.
\label{eq:jabs}
\end{equation}
Vpeljali smo absorpcijski koeficient $\mu$ \index{Absorpcijski koeficient}
\begin{equation}
\mu=\frac{\Delta N}{V}\sigma(\omega)=
\frac{\Delta N}{V}\, B_{21}\, g\frac{\hslash\omega}{c}.
\label{eq:muabs1}
\end{equation}
 
Makroskopski koeficient absorpcije svetlobe v plinu atomov smo tako povezali
z Einsteinovim koefici\-entom $B_{21}$. Povejmo še, da so 
tipične velikosti presekov za absorpcijo $\sigma \sim 10^{-24}$--$10^{-16}~\si{\metre^2}$.

\begin{remark}
Energija se pri absorpciji v plinu dvonivojskih atomov seveda
ne izgublja. Atom, ki je prešel v vzbujeno stanje, se s spontano 
emisijo vrne nazaj v osnovno. Pri tem se svetloba izseva na vse strani -- se siplje. 
\end{remark}

\section{Nasičenje absorpcije}
\label{chap:NasAbs}
Čeprav je videti izraz za zmanjševanje 
gostote svetlobnega toka pri prehodu skozi absorbirajoči plin (enačba~\ref{eq:jabs}) 
preprost, ga ni mogoče enostavno integrirati, saj je absorpcijski koeficient 
$\mu$ odvisen od 
gostote energijskega toka $j$. Pri dovolj velikem svetlobnem toku namreč z 
absorpcijo znaten delež atomov preide v višji nivo, zato se zmanjša razlika $\Delta N$
in posledično se zmanjša tudi absorpcijski koeficient $\mu$. Takrat se absorpcija
v plinu nasiti in pojavu pravimo nasičenje absorpcije\index{Nasičena absorpcija}.

Naj na dvonivojski plin vpada snop monokromatske svetlobe. 
Atomi v plinu prehajajo med nivojema zaradi absorpcije, spontane in stimulirane emisije. 
Podobno kot smo zapisali termično ravnovesje v primeru
širokega spektra (enačba~\ref{4.23}), zapišemo stacionarno enačbo 
\begin{equation}
\frac{dN_{1}}{dt}=A_{21}N_{2}+B_{21}\,g\,\Delta N\,\frac{j}{c}=0,
\label{4.32}
\end{equation}
pri čemer smo za verjetnost za prehod uporabili
enačbo~(\ref{4.18}) in upoštevali $w=j/c$. Zasedenost višjega nivoja $N_{2}$ izrazimo s 
celotnim številom atomov $N$ in razliko zasedenosti $\Delta N$
\begin{equation}
N_{2}=\frac{1}{2}(N_1+N_2) + \frac{1}{2}(N_2-N_1) = \frac{1}{2}N+\frac{1}{2}\Delta N.
\label{4.321}
\end{equation}
Izračunamo razliko zasedenosti 
\begin{equation}
\Delta N=-\frac{N}{1+2\frac{B_{21}g}{cA_{21}}j}.
\label{4.33}
\end{equation}
Pri majhni gostoti toka $j$ so praktično vsi atomi v osnovnem stanju in prispevajo
k absorpciji. Pri velikih gostotah toka  imenovalec v izrazu za $\Delta N$
močno naraste, razlika zasedenosti gre proti nič in absorpcija se zmanjšuje.
Ko drugi člen v imenovalcu (enačba~\ref{4.33}) doseže vrednost 1, pravimo, da
gostota energijskega toka doseže vrednost saturacijske gostote\index{Saturacijska gostota toka}.
Zapišemo jo kot 
\begin{equation}
j_{s}(\omega)=\frac{cA_{21}}{2B_{21}g}=
\frac{\hslash\omega^{3}}{2\pi^{2}c^{2}g},
\label{4.34}
\end{equation}
pri čemer smo upoštevali zvezo med koeficientoma $A_{21}$ in $B_{21}$
(enačba~\ref{4.27}).
Kot vidimo, je saturacijska gostota odvisna le od krožne frekvence vpadnega valovanja 
in vrednosti $g(\omega - \omega_0)$, ki je približno obratna vrednost širine atomskega 
prehoda. Za črto z valovno dolžino $600~\si{\nano\metre}$ 
in širino $10^{8}~\si{\second}^{-1}$ znaša saturacijska gostota svetlobnega toka okoli 
$20~\si{\milli\watt/\centi\metre^2}$. Tako veliko gostoto svetlobnega toka je v tako ozkem
frekvenčnem intervalu z navadnimi svetili praktično nemogoče doseči, 
medtem ko jo z laserji z lahkoto.

Izraz za razliko zasedenosti stanj zapišemo v preglednejši obliki
\begin{equation}
\Delta N=-\frac{N}{1+j/j_{s}(\omega)}.
\label{4.35}
\end{equation}
Vstavimo ga v enačbo za zmanjševanje gostote toka (enačba~\ref{4.30}) in dobimo
\boxeq{4.36}{
dj=-\frac{\mu_{0}}{1+j/j_{s}}\, j\, dz,
}
kjer je 
\boxeq{eq:mu0abs}{
\mu_{0}=\frac{N}{V}\sigma = \frac{N\,B_{21}\,g\,\hslash\omega}{Vc}
}
absorpcijski koeficient\index{Absorpcijski koeficient} pri majhnih gostotah vpadnega toka.

Enačbo (\ref{4.36}) brez težav integriramo
\begin{equation}
\ln\frac{j}{j_{0}}+\frac{j-j_{0}}{j_{s}}=-\mu_{0}\, z.
\label{4.37}
\end{equation}
Z $j_{0}$ smo označili začetno gostoto svetlobnega toka. Kadar je ta dosti
manjša od $j_{s}$, lahko drugi člen v enačbi~(\ref{4.37}) zanemarimo in gostota svetlobnega
toka eksponentno pojema (slika~\ref{fig:abs2})
\begin{equation}
j = j_0 e^{-\mu_0 z}.
\end{equation}
Pri zelo velikih vpadnih gostotah, ko je absorpcija nasičena, lahko prvi člen v izrazu 
zanemarimo in gostota svetlobnega toka pojema linearno
\begin{equation}
j=j_{0}-\mu_{0}j_{s}z.
\label{4.38}
\end{equation}
V primeru močnega vpadnega toka sta zasedenosti osnovnega in vzbujenega nivoja skoraj
enaki in absorpcija je omejena s tem, kako hitro se atomi vračajo
v osnovno stanje s spontanim sevanjem. 

\begin{figure}[h]
\centering
\def\svgwidth{90truemm} 
\input{slike/05_jabs.pdf_tex}
\caption{Pojemanje gostote svetlobnega toka $j$ v absorbirajočem plinu (enačba~\ref{4.37}). 
Gostota vpadnega svetlobnega toka je $j_0$, saturacijsko
gostoto smo izbrali $j_s = 0,3 j_0$. Pri $j>j_s$ je absorbcija nasičena in pojemanje linearno, 
pri $j<j_s$ je pojemanje eksponentno.}
\label{fig:abs2}
\end{figure}

\section{Optično ojačevanje}
\index{Optično ojačevanje}
Do zdaj smo obravnavali prehod svetlobe skozi dvonivojski plin. V 
termičnem ravnovesju je zasedenost zgornjega nivoja manjša od zasedenosti spodnjega in 
svetloba, ki vpada na plin, se v njem absorbira. 
Če uspemo doseči stanje obrnjene zasedenosti\index{Obrnjena zasedenost},
za katerega velja $N_{2}>N_{1}$, se bo snop svetlobe pri prehodu skozi plin ojačeval. Ta pojav
je osnova za delovanje laserjev. 

Stanje obrnjene zasedenosti seveda ni v termičnem ravnovesju in ga je treba vzdrževati z dovajanjem 
energije plinu -- črpanjem\index{Črpanje}.
Načinov črpanja za dosego obrnjene zasedenosti je veliko. Zaenkrat poglejmo
le nekaj osnovnih mehanizmov, podrobneje jih bomo spoznali na konkretnih primerih 
laserjev (poglavje~\ref{chap:Primeri}).

V plinih je najpogostejši način črpanja vzbujanje z električnim tokom. Elektroni,
ki so glavni nosilci toka, se zaletavajo v atome ali ione plina in jih vzbujajo
v višje nivoje. To povzroči obrnjeno zasedenost med
nekim parom nivojev. Tako črpanje uporabljamo na primer v argonovem laserju\index{Laser!argonov}. 

Pogost proces v plinih je tudi prenos energije med atomi s trki. V
mešanici dveh plinov, pri katerih se nek nivo enih atomov ujema po energiji z
nekim nivojem drugih atomov, lahko vzbujen atom prve vrste pri trku preda 
energijo brez sevanja atomu druge vrste, ta pa iz osnovnega stanja preide v 
ustrezen višji nivo. Če je pod tem nivojem še drugo vzbujeno stanje, katerega
življenjski čas je krajši od življenjskega časa zgornjega nivoja, pride
do obrnjene zasedenosti. Primer uporabe takega črpanja je He-Ne laser\index{Laser!He-Ne}.

V trdnih neprevodnih kristalih sta v optičnem področju absorpcija
in sevanje navadno posledica primesi.
Obrnjeno zasedenost para nivojev primesi navadno dosežemo tako, da
kristal obsevamo s svetlobo s frekvenco, ki ustreza prehodu na nek
nivo, ki leži nad izbranim parom nivojev. Tako črpanje uporabljamo na primer v Nd:YAG in Ti:safir 
laserjih. \index{Laser!Nd:YAG} \index{Laser!Ti:safir}

V polprevodnikih \index{Laser!polprevodniški}dosežemo obrnjeno zasedenost med 
prevodnim in valenčnim pasom z vbrizgavanjem elektronov in vrzeli v območje spoja $p$-$n$ 
z električnim tokom v prevodni smeri. 

\section{Optično črpanje trinivojskega sistema}
Kot primer optičnega ojačevanja si oglejmo najpreprostejši model optičnega črpanja.
To je plin atomov s tremi nivoji, tako imenovani trinivojski sistem (slika~\ref{fig:3nivojski}\,a).
Osnovno stanje, ki 
ga označimo z $|0\rangle$,  naj ima energijo $E_0$. Poleg tega naj imajo atomi še 
dve vzbujeni stanji z energijo $E_1$ (stanje $|1\rangle$) in energijo $E_2>E_1$
(stanje $|2\rangle$)\index{Trinivojski sistem}, tako da je energijska razlika med vzbujenima 
nivojema $E_2-E_1 = \hslash \omega_0$.

Na tak trinivojski plin svetimo s črpalno svetlobo, ki vzbuja atome iz osnovnega stanja 
$|0\rangle$ v stanje $|2\rangle$, pri čemer je lahko spektralna gostota $u_{p}$ črpalne 
svetlobe široka. Po plinu naj se širi še monokromatska svetloba z gostoto 
energije $w$ in frekvenco $\omega$, ki je blizu frekvence prehoda $\omega_{0}$. 
Ugotoviti želimo, pri katerih pogojih  dosežemo obrnjeno zasedenost med 
stanjema $|1\rangle$ in $|2\rangle$ in s tem ojačevanje svetlobe s frekvenco blizu
$\omega_{0}$ (slika~\ref{fig:3nivojski}\,b).

\begin{figure}[h]
\centering
\def\svgwidth{130truemm} 
\input{slike/05_Trinivojski.pdf_tex}
\caption{Shema energijskih nivojev trinivojskega sistema in oznake koeficientov za prehode
med njimi (a). V plinskih laserjih je stanje obrnjene zasedenosti navadno med vzbujenima 
stanjema (b). Pogosto so laserji štiri- ali večnivojski (c).}
\label{fig:3nivojski}
\end{figure}
\begin{remark}
Trinivojski laserski sistem na sliki~\ref{fig:3nivojski}\,b je pravzaprav 
poseben primer bolj realističnega štiri\-nivojskega sistema\index{Štirinivojski sistem}, 
pri katerem zgornji črpalni nivo sovpada z zgornjim laserskim nivojem. Sicer se tretji vzbujeni nivo, 
v katerega črpamo, praviloma zelo hitro prazni v drugega vzbujenega in od tam počasi v prvega vzbujenega, 
kot kaže slika~\ref{fig:3nivojski}\,c.
Obravnava štirinivojskih sistemov je bolj zapletena od obravnave trinivojskih sistemov, 
ki za opis delovanja laserjev povsem zadošča. Podrobneje bomo večnivojske sisteme 
obravnavali na konkretnih laserskih primerih (poglavje~\ref{chap:Primeri}).
\end{remark}

Zapišimo enačbe za spreminjanje zasedenosti posameznih stanj. Osnovno stanje
$|0\rangle$ se prazni zaradi absorpcije črpalne svetlobe in polni zaradi
spontanih prehodov\index{Spontano sevanje} iz stanj $|1\rangle$ in $|2\rangle$. Stimulirane
prehode iz stanja $|2\rangle$ v osnovno stanje bomo zanemarili. Zasedenost stanja $|2\rangle$ se
povečuje zaradi absorpcije s spodnjih nivojev in zmanjšuje
zaradi spontanega in stimuliranega sevanja\index{Stimulirano sevanje}. Srednje stanje se polni
s stimuliranimi in spontanimi prehodi iz stanja $|2\rangle$ in prazni
zaradi absorpcije in spontanih prehodov.
Pri tem velja, da je vsota vseh treh zasedenosti enaka številu vseh atomov in $N_{0}+N_{1}+N_{2}=N$. 
Zasedbene enačbe so tako\index{Zasedenost stanj}
\begin{align}
\frac{dN_{0}}{dt} & =  -rN_0+A_{20}N_{2}+A_{10}N_{1}, \label{4.39.1}\\
\frac{dN_{1}}{dt} & =  -A_{10}N_{1}+B_{21}\,g\,w\, (N_{2}-N_{1})+A_{21}N_{2} \label{4.39.2} \quad \mathrm{in}\\
\frac{dN_{2}}{dt} & =  rN_0-A_{20}N_{2}-A_{21}N_{2}-B_{21}\,g\,w\, (N_2-N_1).
\label{4.39}
\end{align}
Pri zapisu smo predpostavili, da je $N_0 \approx N \gg N_1, N_2$ in črpanje $B_{20}\, 
u_{p} (N_0-N_2)$, ki je praktično konstantno, zapisali s koeficientom $r$. Mehanizem črpanja 
smo s tem skrili v $r$ in prav nič ni pomembno, na kakšen način poteka.\index{Optično črpanje}
S tem smo obravnavo posplošili z optičnega na druge sisteme črpanja. 

Zanima nas stacionarno stanje, ko so vsi trije časovni odvodi enaki nič. 
Tako iz druge enačbe sistema~(enačba~\ref{4.39.2}) sledi
\begin{eqnarray}
B_{21}\,g\,w\, N_{2}+A_{21}N_{2} = B_{21}\,g\,w\, N_{1} + A_{10}N_{1} 
\end{eqnarray}
in
\begin{eqnarray}
N_2 = \frac{B_{21}\,g\,w + A_{10}}{B_{21}\,g\,w+A_{21}}N_1.  
\end{eqnarray}
Brez škode lahko zanemarimo spontano sevanje iz stanja
$|2\rangle$ v osnovno stanje. Tako iz prve enačbe sistema~(enačba~\ref{4.39.1}) dobimo
\begin{equation}
N_1= \frac{rN}{A_{10}}
\end{equation}
in razliko zasedenosti zapišemo kot 
\begin{equation}
N_{2}-N_{1}=\left(\frac{N_2}{N_1}-1\right)N_1=\left(\frac{A_{10}-A_{21}}{A_{21}+
B_{21}g\,w}\right)\,\frac{rN}{A_{10}}.
\label{4.42}
\end{equation}
Sledi, da je zasedenost obrnjena\index{Obrnjena zasedenost}, 
kadar je $A_{10}>A_{21}$, torej kadar je
razpadni čas stanja $|1\rangle$ krajši od razpadnega časa stanja $|2\rangle$.
Tak rezultat smo seveda pričakovali.

V praktičnih primerih navadno velja $A_{10}\gg A_{21}$. Ob upoštevanju zveze $j=wc$ povežemo
razliko zasedenosti z gostoto vpadnega svetlobnega toka
\begin{equation}
N_{2}-N_{1}=\frac{rN}{A_{21}} \, \frac{1}{1+\frac{B_{21}gj}{c A_{21}}} = 
\frac{rN}{A_{21}} \, \frac{1}{1+j/j_s}.
\label{eq:3n_N}
\end{equation}
Konstante smo pospravili v saturacijsko gostoto svetlobnega toka\index{Saturacijska gostota toka} 
\begin{equation}
j_s = \frac{c A_{21}}{B_{21}g}.
\label{eq:jsatg}
\end{equation}
\vglue-5truemm
\begin{remark}
 Vidimo, da je  izraz za saturacijsko gostoto toka v trinivojskem sistemu 
 (enačba~\ref{eq:jsatg}) zelo podoben izrazu za saturacijsko gostoto v dvonivojskem
 sistemu (enačba~\ref{4.34}), razlikujeta se le
v faktorju 2. Ta razlika je posledica različnega števila nivojev, saj pogoj $N_{1}+N_{2}=N$
v trinivojskem sistemu ne velja. 
\end{remark}
Poglejmo, kaj se zgodi s svetlobo ob vpadu na plast trinivojskega plina. Naj ima vpadna
svetloba krožno frekvenco $\omega$ in gostoto svetlobnega toka $j=wc$. Račun je zelo podoben 
računu za absorpcijo (enačba~\ref{4.29}). Sprememba gostote toka na debelini $dz$ je enaka
\begin{equation}
dj=\frac{(N_{2}-N_{1})}{V}\, B_{21}g\, \frac{\hslash\omega}{c}j\, dz,
\label{eq:dj}
\end{equation}
pri čemer gostota toka $j$ nastopa tudi v izrazu za razliko
zasedenosti (enačba~\ref{eq:3n_N}). Če to upoštevamo, 
dobimo diferencialno enačbo za gostoto toka
\begin{equation}
\frac{1}{j}\left(1+\frac{j}{j_{s}}\right)\, dj=G\, dz
\label{4.43}
\end{equation}
oziroma
\boxeq{eq:djG}{
dj=\frac{G}{1+j/j_{s}}\, j\, dz,
}
ki je spet zelo podobna enačbi za absorpcijo (enačba~\ref{4.36}).
Z $G$ smo označili t.\,i.\, koeficient ojačenja pri majhni gostoti vpadnega
toka\index{Koeficient ojačenja}. Podan je z 
\begin{equation}
G=\frac{N}{V}\frac{r}{A_{21}}\sigma=\frac{rNB_{21}\hslash\omega g}{VcA_{21}},
\label{4.44}
\end{equation}
Rešitev diferencialne enačbe (enačba~\ref{eq:djG}) je prikazana na sliki~\ref{fig:ojacanje}. 
\begin{figure}[h]
\centering
\def\svgwidth{80truemm} 
\input{slike/05_joja.pdf_tex}
\caption{Naraščanje gostote svetlobnega toka $j$ pri optičnem ojačevanju. 
Gostota vpadnega svetlobnega toka je $j_0$, saturacijsko
gostoto smo izbrali $j_s = 20 j_0$. Pri $j<j_s$ je naraščanje eksponentno, 
pri $j>j_s$ je zaradi nasičenja linearno.
}
\label{fig:ojacanje}
\end{figure}

Obnašanje gostote svetlobnega toka ima, tako kot pri absorpciji, dva režima. 
Pri majhnih gostotah toka $j\ll j_{s}$ je naraščanje eksponentno 
\begin{equation}
j(z)=j_{0}e^{Gz}.
\label{4.45}
\end{equation}
Pri velikih gostotah toka zaradi nasičenja gostota svetlobnega
toka narašča linearno
\begin{equation}
j(z)=j_{0}+j_{s}Gz.
\label{4.46}
\end{equation}
V tem primeru je gostota toka tako velika, da vsi atomi, ki jih
s črpanjem spravimo v najvišje stanje, preidejo v stanje $|1\rangle$
s stimuliranim sevanjem. Pri konstantnem črpanju je tedaj 
linearno naraščanje gostote toka razumljivo. 

Vrnimo se k preseku za stimulirano sevanje $\sigma$
 (enačba~\ref{4.44})\index{Presek
za stimulirano sevanje}. Opazimo, da je enak preseku za
absorpcijo (enačba~\ref{sigmaabs}) dvonivojskega sistema\index{Presek za absorpcijo}.
Odvisen je od frekvence svetlobe in sorazmeren vrednosti atomske spektralne 
črte pri frekvenci prehoda. Za He-Ne laser\index{Laser!He-Ne} 
($\lambda=633~\si{nm}$ in $\Delta \nu \sim 1,5~\si{\giga\hertz}$) znaša   
$\sigma \sim 10^{-16}~\si{\metre}^2$ in
za Nd:YAG\index{Laser!Nd:YAG} ($1064~\si{nm}$ in $\Delta \nu \sim 150~\si{\giga\hertz}$)
$\sigma \sim 10^{-22}~\si{\metre}^2$.
Zaradi različnih presekov, različnih gostot atomov in različnih načinov črpanja se 
koeficienti ojačenja v večnivojskih sistemih med seboj precej razlikujejo. Tipično
ojačenje v He-Ne laserju z dolžino $L = 0,5~\si{\metre}$ je 
$GL \sim 1,015$ in v Nd:YAG laserju z dolžino ojačevalnega sredstva 
$L = 10~\si{\centi\metre}$ $GL \sim 50$. Pri prvem laserju je sicer 
velik presek za stimulirano sevanje, vendar je gostota atomov v obrnjeni zasedenosti 
razmeroma majhna. V drugem primeru močno črpanje prevlada nad majhnim presekom 
in ojačenje je veliko.\index{Optično ojačevanje}

\section{Homogena in nehomogena razširitev spektralne črte}
\label{Razsiritev}
Doslej smo privzeli, da svetijo vsi atomi obravnavane snovi
pri isti krožni frekvenci $\omega_{0}$ in z isto spektralno širino, ki smo
jo popisali s funkcijo $g(\omega-\omega_0)$ z vrhom pri $\omega_0$. Če to velja, 
je razširitev spektralne črte homogena\index{Spektralna črta!homogena razširitev}. 
Funkcija $g(\omega-\omega_0)$ je v tem primeru Lorentzove oblike\index{Spekter!Lorentzov} 
\boxeq{eq:homogenasirina}{
g_L(\omega-\omega_0)=\frac{1}{\pi}\frac{\gamma}{(\omega-\omega_{0})^{2}+\gamma^{2}}
}
s širino črte $\Delta \omega_L = 2\gamma$ (glej sliko~\ref{fig:SpekterAc}). 
Primera homogene razširitve sta naravna širina in razširitev zaradi trkov med atomi.
Homogena razširitev je pogosto večja od obratne vrednosti razpadnega časa nivoja. 
V plinu namreč prihaja do trkov, ki lahko zmotijo le fazo sevanja, ne da bi povzročili 
prehod, vendar razširijo spektralno črto. V trdni snovi homogeno razširitev 
brez prehoda povzročajo termična nihanja lokalnega polja. 

Spektralna črta je lahko razširjena tudi zato, ker svetloba, izhajajoča iz različnih
atomov, nima povsem iste frekvence. Tedaj govorimo o nehomogeni 
razširitvi\index{Spektralna črta!nehomogena razširitev}.
Najpomembnejši primer nehomogene razširitve je Dopplerjeva 
\index{Dopplerjeva razširitev} razširitev v plinu. 
Atomi plina vedno sevajo pri praktično isti frekvenci $\omega_0$, vendar jih zaradi gibanja
opazovalec v mirujočem (laboratorijskem) sistemu v skladu z Dopplerjevim pojavom 
zazna pri različnih frekvencah. 

Naj se atom giblje s hitrostjo $v$ glede na smer opazovanja. Potem opazovane krožne 
frekvence posameznih atomov zapišemo kot  
\begin{equation}
\omega=\omega_{0}-\frac{v}{c}\omega_{0}=\omega_{0}-k_{0}v.
\label{4.81}
\end{equation}
Označimo z ${\cal N}(v)$ porazdelitev gostote atomov po hitrostih, pri čemer se omejimo 
le na premikanje v smeri opazovanja. V termičnem ravnovesju je ${\cal N}(v)$
Maxwellova porazdelitev\index{Maxwellova porazdelitev}
\begin{equation}
{\cal N}(v)=\frac{N}{V}\left(\frac{m}{2\pi k_{B}T}\right)^{1/2}e^{-\frac{mv^{2}}{2k_{B}T}},
\label{4.82}
\end{equation}
kjer je $m$ masa posameznega atoma.
Porazdelitev atomov po frekvencah izračunamo tako, da hitrost izrazimo
iz enačbe~(\ref{4.81}), poleg tega funkcijo $g_{D}(\omega-\omega_0)$
normiramo. Sledi
\boxeq{4.821}{
g_{D}(\omega-\omega_0)=\frac{c}{\omega_{0}}\left(\frac{m}{2\pi 
k_{B}T}\right)^{1/2}\exp \left(-\frac{mc^{2}}{2k_{B}T}\frac{(\omega-\omega_{0})^{2}}{\omega_{0}^2}\right).
}
Dopplerjeva razširitev v plinu je torej Gaussove oblike\index{Spekter!Gaussov}.
Njena širina pri polovični 
višini\footnote{Celotno širino na polovični višini imenujemo FWHM -- \it{Full Width at Half Maximum}.} je
\begin{equation} 
\Delta\omega_{D}=2 \sqrt{\frac{2k_{B}T \ln 2}{mc^{2}}}\omega_{0}.
\label{4.83}
\end{equation}
\begin{definition}
Izpelji obliko nehomogeno razširjene črte za Dopplerjevo razširitev (enačba~\ref{4.821})
in pokaži, da je njena širina podana z enačbo~(\ref{4.83}).
\end{definition}

Izračunajmo Dopplerjevo razširitev na primeru He-Ne laserja. Za prehod
atoma neona pri $633~\si{nm}$ in temperaturi $300~\si{K}$ je izračunana vrednost
$\Delta\omega_{D}=8\cdot10^{9}~\si{\second}^{-1}$ oziroma $\Delta \nu = 1,4~\si{\giga\hertz}$. 
Dejanske izmerjene vrednosti širine črte za He-Ne \index{Laser!He-Ne}laser 
znašajo okoli $1,5~\si{\giga\hertz}$, kar je znatno več od naravne širine
črte ($1,2~\si{\mega\hertz}$). Še bolj izrazite so razširitve zaradi nehomogenosti
v trdninskih laserjih, na primer v Nd:YAG laserju, v katerem \index{Laser!Nd:YAG}
je širina črte $\Delta \nu= 150~\si{\giga\hertz}$. Nehomogena razširitev zaradi Dopplerjevega pojava v 
redkem plinu ali zaradi nehomogenosti v trdnih snoveh je tako kar nekaj redov velikosti 
večja od homogene naravne širine in razširitve zaradi trkov.

\begin{remark}
Pri nehomogenih razširitvah bi za bolj natančen izračun morali upoštevati 
tudi naravno širino posameznega atoma. To bi zapisali s konvolucijo Lorentzove
in Gaussove funkcije in dobili tako imenovan Voigtov 
profil, ki ga ne moremo preprosto analitično zapisati\index{Spekter!Voigtov}.
\end{remark}

\section{*Nasičenje nehomogeno razširjene absorpcijske črte}
\label{NasabsNehom}
\index{Nasičena absorpcija!nehomogeno razširjene črte}
V razdelku~(\ref{chap:NasAbs}) smo obravnavali nasičenje absorpcije pri homogeno 
razširjenem prehodu. Pri nasičenju absorpcije, kadar prevladuje nehomogena razširitev,
nastopijo pomembni novi pojavi.

Začnimo z dvonivojskim plinom\index{Dvonivojski sistem}, na katerega 
vpada močan snop monokromatske svetlobe s frekvenco $\omega_S$,
ki je blizu osrednje frekvence $\omega_{0}$ Dopplerjevo razširjene 
črte\index{Dopplerjeva razširitev}, in gostoto toka $j$. S svetlobo
lahko sodeluje le skupina atomov, pri kateri se Dopplerjevo premaknjena
frekvenca od $\omega_S$ ne razlikuje več kot za homogeno širino, ki
jo opisuje funkcija $g(\omega-\omega_S)$. Zato ne moremo zapisati zasedbenih
enačb za vse atome hkrati, ampak le za tiste, ki imajo hitrost med
$v$ in $v+dv$ in ki absorbirajo svetlobo pri frekvenci $\omega_{0}-kv$.

Naj bosta ${\cal N}_{1}(v)$ in ${\cal N}_{2}(v)$ hitrostni porazdelitvi
atomov v osnovnem in vzbujenem stanju. Gostota
${\cal N}_{2}(v)$ se spreminja podobno kot celotna
zasedenost v homogenem primeru (enačba~\ref{4.22})
\begin{equation}
\frac{d{\cal N}_{2}(v)}{dt}=-A{\cal N}_{2}(v) -B\, g(\omega_S-\omega_{0}+kv)
\frac{j}{c}\,
\left({\cal N}_{2}(v)-{\cal N}_{1}(v)\right).
\label{4.85}
\end{equation}
Upoštevali smo, da je zaradi Dopplerjevega pojava prehod premaknjen k frekvenci
$\omega_{0}-kv$. Velja
\begin{equation}
 \frac{d{\cal N}_{2}(v)}{dt}=-\frac{d{\cal N}_{1}(v)}{dt}.
\label{4.86}
\end{equation}
Vpeljemo še ${\cal Z}(v)={\cal N}_{1}(v)-{\cal N}_{2}(v)$. Podobno kot 
v enačbi~(\ref{4.321}) zapišemo
\begin{equation}
{\cal N}_{2}(v)=\frac{1}{2}{\cal N}(v)-\frac{1}{2}{\cal Z}(v)
\end{equation}
in dobimo 
\begin{equation}
\frac{d{\cal Z}(v)}{dt}=-A{\cal Z}(v)+A{\cal N}(v)
-2B\,g(\omega_S-\omega_{0}+kv)\frac{j}{c}
{\cal Z}(v).
\label{4.87}
\end{equation}
V stacionarnem stanju je odvod enak nič in dobimo
\begin{equation}
{\cal Z}(v)=\frac{{\cal N}(v)}{1+\frac{2B}{Ac}g(\omega_S-\omega_{0}+kv)j}.
\label{4.88}
\end{equation}
 Če je gostota vpadnega svetlobnega toka majhna, lahko imenovalec razvijemo
\begin{equation}
{\cal Z}(v)\approx{\cal N}(v)\left(1-\frac{2B}{Ac}g(\omega_S-\omega_{0}+kv)j\right).
\label{4.89}
\end{equation}
Porazdelitev ${\cal Z}(v)$ je podobna nemoteni porazdelitvi atomov
po hitrosti ${\cal N}(v)$, le da je pri hitrosti $v=(\omega_{0}-\omega_S)/k$
zmanjšana zaradi vpliva vpadne svetlobe. Atomi s to hitrostjo namreč svetlobo
absorbirajo in s tem prehajajo v zgornje stanje. V porazdelitvi
atomov tako nastane vdolbina\index{Bennettova vdolbina}, ki jo imenujemo
Bennettova vdolbina\footnote{Ameriški fizik William Ralph Bennett Jr., 1930--2008.} 
(slika \ref{fig:Bennet}). Širina vdolbine je določena
s homogeno širino prehoda, to je s funkcijo $g(\omega_S-\omega_{0}+kv)$, in 
globina  z gostoto vpadnega toka~$j$.\index{Spektralna črta!homogena razširitev}
\begin{figure}[h]
\centering
\def\svgwidth{90truemm} 
\input{slike/05_Hole.pdf_tex}
\caption{Porazdelitev atomov po hitrosti v osnovnem stanju, v kateri zaradi
absorbcije svetlobe nastane Bennettova vdolbina. Podobno obliko ima 
tudi absorpcijski koeficient za testno svetlobo.}
\label{fig:Bennet}
\end{figure}

Naj na snov poleg močne vpadne svetlobe pri $\omega_S$ vpada še šibko testno valovanje pri 
frekvenci $\omega^{\prime}$. Izračunajmo absorpcijski koeficient za valovanje pri $\omega^\prime$. 
Upoštevati moramo, da k absorpciji testne svetlobe prispevajo vsi atomi, katerih
hitrost je taka, da je prehod dovolj blizu $\omega^{\prime}$. Absorpcijski 
koeficient\index{Absorpcijski koeficient} potem izračunamo s seštevanjem 
po porazdelitvi ${\cal Z}(v)$
(enačba~\ref{eq:muabs1})
\begin{equation}
\mu(\omega^{\prime})=\frac{\hslash\omega^{\prime}}{c}\int{\cal Z}(v)Bg(\omega^{\prime}-\omega_{0}+k'v)\, dv.
\label{4.90}
\end{equation}
Homogena razširitev je dosti manjša od Dopplerjeve širine, zato
v prvem približku Lorentzovo funkcijo $g$ v enačbi (\ref{4.90}) nadomestimo kar z
$\delta(\omega)$. V izrazu za $\cal Z$ (enačba~\ref{4.88}) jo pustimo. 
Tako je absorpcijski koeficient za šibko testno svetlobo 
\begin{align}
\label{eq:mumumu}
\mu(\omega^{\prime}) & =  \frac{\hslash\omega^{\prime}}{k'c}B\frac{{\cal N}
(\frac{\omega_0-\omega'}{k'})}{1+\frac{2Bj}{Ac}g(\omega_S-\omega')} \\ \nonumber 
 & \approx  \hslash B{\cal N}\left(\frac{\omega_0-\omega'}{k'}\right)\left(1-\frac{2Bj}{Ac}g(\omega_S-\omega')\right).
\end{align}
V drugi vrstici smo uporabili približek (enačba~\ref{4.89}). Vidimo, da je 
odvisnost $\mu(\omega^{\prime})$ Gaussove oblike z vdolbino pri $\omega_S$ in je tako
podobna porazdelitvi, kot jo kaže slika~\ref{fig:Bennet}. Odvisnost 
$\mu(\omega^{\prime})$ lahko tudi izmerimo, tako da spreminjamo 
frekvenco testnega snopa $\omega^{\prime}$.

\begin{remark}
 Merjenje absorpcije s testnim
žarkom omogoča opazovanje oblike homogene črte kljub mnogo večji
nehomogeni Dopplerjevi razširitvi. V moderni spektroskopiji ima zato ta metoda
velik pomen.
\end{remark}

Izračunajmo še absorpcijski koeficient za prvi, močan vpadni snop, tako da v
enačbi~(\ref{eq:mumumu}) vstavimo $\omega^{\prime}=\omega_S$. Vodilni člen ${\cal N}((\omega_0-
\omega_S)/k)$ opisuje običajno Gaussovo obliko Dopplerjevo
razširjene črte, pri čemer izraz v oklepaju da zmanjšanje absorpcije
zaradi nasičenja, ki je odvisno le od vrednosti $g(0)$ in zato enako za vse $\omega_S$. 
Z enim samim vpadnim snopom svetlobe torej vdolbine v absorpciji ne moremo zaznati, saj 
je izmerjena črta kljub nasičenju Gaussove oblike. 

Namesto z dvema snopoma, od katerih šibkemu testnemu snopu spreminjamo
frekvenco, lahko vdolbino v porazdelitvi zaznamo tudi z enim samim snopom
spremenljive frekvence, ki se po prvem prehodu skozi plin odbije od
zrcala in vrne v nasprotni smeri. S tem se v porazdelitvi atomov
v spodnjem stanju simetrično pri hitrostih $\pm(\omega_{0}-\omega_S)/k$
pojavita dve Bennettovi vdolbini (slika \ref{fig:Lamb}\,a).
Kadar je $\omega_S$ blizu $\omega_{0}$, se vdolbini vsaj delno prekrivata, 
stopnja nasičenja se poveča in v krivulji za absorpcijo svetlobe se pojavi 
vdolbina (slika \ref{fig:Lamb}\,b).
Imenujemo jo Lambova vdolbina\footnote{Ameriški fizik in nobelovec 
Willis Eugene Lamb Jr., 1913--2008.}\index{Lambova vdolbina}. 
\begin{figure}[h]
\centering
\def\svgwidth{140truemm} 
\input{slike/05_Lamb.pdf_tex}
\caption{Porazdelitev atomov po hitrosti v osnovnem stanju, kadar svetloba prehaja 
skozi plin v dveh smereh (a). Če frekvenca vpadne svetlobe približno sovpada z osrednjo
frekvenco prehoda, se vdolbini prekrivata in absorpcija se zmanjša (b).}
\label{fig:Lamb}
\end{figure}

Zapišimo enačbe še za ta primer. Vpadni snop svetlobe povzroči spremembo zasedenosti
pri prehodu skozi plin v obeh smereh, zato je zdaj 
\begin{equation}
{\cal Z}(v)\approx{\cal N}(v)\left(1-\frac{2Bj}{Ac}\left(g(\omega_S-\omega_{0}+kv)+
g(\omega_S-\omega_{0}-kv)\right)\right).
\label{4.92}
\end{equation}
Podobno kot prej izračunamo absorpcijski koeficient za širjenje svetlobe v
pozitivni smeri 
\begin{align}
\mu_{+}(\omega_S) & =  \frac{\hslash\omega}{c}B\int{\cal Z}(v)g(\omega_S-\omega_{0}+kv)\, dv\nonumber \\
 & \approx  \hslash B{\cal N}\left(\frac{\omega_S-\omega_0}{k}\right)\left(1-\frac{2Bj}
 {Ac}\left(g(0)+g(2(\omega_S-\omega_{0}))\right)\right)\;.
 \label{eq:lamb}
\end{align}
Izmerjeni absorpcijski profil je odvisen od frekvence vpadne svetlobe $\omega_S$ in ima pri $\omega_0$ vdolbino,
ki je podobna homogeno razširjeni črti. Faktor 2 v argumentu
funkcije $g(2(\omega_S-\omega_{0}))$ je posledica  grobega približka,
ko smo v integraciji $g(\omega_S-\omega_{0}+kv)$ nadomestili kar z
$\delta$ funkcijo. Natančnejši račun pokaže, da je vrh pri $\omega_{0}$
kar oblike $g(\omega_S-\omega_{0})$.

\begin{definition}
Pokaži, da je rezultat natančnejše izpeljave absorpcijskega koeficienta  
\begin{equation}
 \mu_{+}(\omega_S) = \hslash B{\cal N}\left(\frac{\omega_S-\omega_0}{k}\right)\left(1-\frac{Bj}
 {Ac}\left(g(0)+g(\omega_S-\omega_{0})\right)\right)\;.
\end{equation}
Pri računu privzemi, da je širina Dopplerjeve porazdelitve bistveno večja od širine homogene
razširitve (enačba~\ref{eq:homogenasirina}) in Maxwellovo porazdelitev postavi pred integral. 
\end{definition}

\section{*Izpeljava verjetnosti za prehod}
\label{chap:verjetnost}
\index{Verjetnost za prehod}
Verjetnosti za prehod atoma iz enega stanja v drugo s sevanjem, ki
smo jih opisali s fenomenološkimi Einsteinovimi koeficienti $A_{21}$
in $B_{21}$ (razdelek~\ref{AB}), \index{Einsteinovi koeficienti}
je mogoče izpeljati tudi drugače.
Pri tem se poslužimo kvantne\index{Kvantizacija polja} elektrodinamike, 
kar pomeni kvantno obravnavo 
tako atoma kot elektromagnetnega polja. Povsem strog račun je zahteven in presega
okvir te knjige, zato si na kratko oglejmo le, kako pridemo do rezultata z uporabo
Fermijevega zlatega pravila.\index{Fermijevo zlato pravilo}

Postavimo dvonivojski atom v votlino z elektromagnetnim poljem.\index{Dvonivojski sistem}
Izračunajmo verjetnost, da zaradi interakcije s poljem atom
preide iz stanja $|2\rangle$ v stanje $|1\rangle$, pri čemer se
število fotonov v izbranem stanju elektromagnetnega polja $\alpha$
poveča z $n_{\alpha}$ na $n_{\alpha}+1$. V vseh ostalih stanjih
polja naj bo število fotonov enako nič.

Med atomom in poljem privzamemo električno dipolno interakcijo 
\begin{equation}
\hat{H}_{i}=-e\hat{E}(\mathbf{r},t)\hat{x},
\label{4.47}
\end{equation}
kjer je $\hat{x}$ operator koordinate elektrona v atomu. 
Privzeli
smo, da je nihajoče polje polarizirano v smeri osi $x$. Stanja celotnega sistema, 
to je atoma in polja, zapišemo v obliki produkta atomskih stanj in
stanja elektromagnetnega polja, pri čemer navedemo število fotonov
v posameznih lastnih nihanjih votline $\alpha$. Zapišemo okrajšano
\begin{equation}
|i,n_{\alpha}\rangle\equiv|i\rangle|\{n_{\alpha}\}\rangle.
\label{4.48}
\end{equation}
Začetno stanje celotnega sistema je torej $|2,n_{\alpha}\rangle$, kar pomeni, da je
atom v vzbujenem stanju (stanju 2), polje pa ima $n_{\alpha}$ fotonov v stanju $\alpha$.
Ustrezno končno stanje po prehodu je $|1,n_{\alpha}+1\rangle$.

V prvem redu teorije motenj je verjetnost za prehod na časovno enoto enaka
\begin{equation}
w_{21}=\frac{2\pi}{\hslash}|\langle1,n_{\alpha}+
1|\,\hat{H}_{i}\,|2,n_{\alpha}\rangle|^{2}\,
\delta(E_{2}-E_{1}-\hslash\omega_{\alpha}).
\label{4.49}
\end{equation}
S funkcijo $\delta$ izberemo prehod, pri katerem se energija ohranja.

Operator elektromagnetnega polja razvijemo po lastnih nihanjih votline (enačba~\ref{eq:pqrazvoj}) 
\begin{equation}
\hat{E}(\mathbf{r},t)=-\frac{1}{\sqrt{V\epsilon_{0}}}\sum_{\alpha}
\hat{p}_{\alpha}(t)E_{\alpha}(\mathbf{r}),
\label{4.50}
\end{equation}
kjer je $\hat{p}_{\alpha}$ operator gibalne količine stanja $\alpha$, $E_{\alpha}$
pa funkcija, ki opisuje krajevno odvisnost polja. Vemo, da se vsako lastno
elektromagnetno nihanje votline obnaša kot harmonski oscilator (enačba~\ref{4.9}).\index{Harmonski oscilator}
Vpeljemo kreacijske in anihilacijske operatorje
\begin{align}
\hat{a}_{\alpha}^{\dagger} & =  \frac{1}{\sqrt{2\hslash\omega_{\alpha}}}\,
(\omega_{\alpha}\hat{q}_{\alpha}-i\hat{p}_{\alpha}) \qquad \mathrm{in} \\
\hat{a}_{\alpha} & =  \frac{1}{\sqrt{2\hslash\omega_{\alpha}}}\,(\omega_{\alpha}\hat{q}_{\alpha}+i\hat{p}_{\alpha}).
\end{align}
Kreacijski operatorji povečujejo, anihilacijski pa zmanjšujejo število
fotonov v danem stanju. Tako velja
\begin{align}
\hat{a}_{\alpha}^{\dagger}|n_{\alpha}\rangle & =  \sqrt{n_{\alpha}+1}
|n_{\alpha}+1\rangle\qquad \mathrm{in} \\
\hat{a}_{\alpha}|n_{\alpha}\rangle & =  \sqrt{n_{\alpha}}|n_{\alpha}-1\rangle.
\end{align}
Edini od nič različni matrični elementi so tako oblike
\begin{align}
\langle n_\alpha +1|\, \hat{a}_{\alpha}^{\dagger}\,|n_{\alpha}\rangle & = 
\sqrt{n_{\alpha}+1} \qquad \mathrm{in} \label{eq:ankr.1}\\
\langle n_\alpha-1|\,\hat{a}_{\alpha}\,|n_{\alpha}\rangle & =  \sqrt{n_{\alpha}}.
\label{eq:ankr}
\end{align}
Operatorje $\hat{p}_{\alpha}$ izrazimo s kreacijskimi in anihilacijskimi
operatorji in jih vstavimo v razvoj električnega polja (enačba~\ref{4.50})
\begin{equation}
\hat{E}(\mathbf{r},t)=-i\sum_{\alpha}\sqrt{\frac{\hslash\omega_{\alpha}}{2V\epsilon_{0}}}\,
\left(\hat{a}_{\alpha}^{\dagger}-\hat{a}_{\alpha}\right)E_{\alpha}(\mathbf{r}).
\label{4.53}
\end{equation}
Nadaljujemo z izračunom matričnega elementa, pri čemer upoševamo, da operator koordinate
$\hat{x}$ deluje na atomski del stanja in $\hat{E}$ na elektromagnetno
polje. Dobimo
\begin{align}
\langle1,n_{\alpha}+1|\,\hat{H}_{i}\,|2,n_{\alpha}\rangle & =  -e\,
\langle1,n_{\alpha}+1|\,\hat{E}\,\hat{x}\,|2,n_{\alpha}\rangle \\
 & =  -e\,\langle1|\,\hat{x}\,|2\rangle\langle n_{\alpha}+1|\,\hat{E}\,|n_{\alpha}\rangle.
\end{align}
Vstavimo polje, ki smo ga izrazili s kreacijskimi in anihilacijskimi operatorji (enačba~\ref{4.53}),
upoštevamo zvezi~(\ref{eq:ankr.1}) in (\ref{eq:ankr}) in zapišemo
\begin{align}
\langle n_{\alpha}+1|\, \hat{E}\,|n_{\alpha}\rangle & = 
 -i\sum_{\beta}\sqrt{\frac{\hslash\omega_{\beta}}{2V\epsilon_{0}}}
\langle n_{\alpha}+1|\,\hat{a}_{\beta}^{\dagger}-\hat{a}_{\beta}\,|n_{\alpha}\rangle\, 
E_{\beta}(\mathbf{r})\nonumber \\
 & =  -i\sqrt{\frac{\hslash\omega_{\alpha}}{2V\epsilon_{0}}}
 \sqrt{n_{\alpha}+1}\, E_{\alpha}(\mathbf{r}).
\end{align}
Od vseh operatorjev v razvoju polja je namreč od nič različen matrični
element le za kreacijski operator za stanje $\alpha$.

Vpeljemo $\mathcal{V} = -e \langle1|\hat{x}|2\rangle$.
 Iskana verjetnost za prehod iz 
začetnega stanja, v katerem je v votlini vzbujen atom in $n_{\alpha}$ fotonov, v končno
stanje, v katerem je atom v osnovnem stanju in $n_{\alpha}+1$ fotonov v stanju $\alpha$, je tako
\begin{equation}
w_{21}=\frac{\pi \omega_{\alpha}\mathcal{V}^{2}}{V\epsilon_{0}}
(n_{\alpha}+1)\,E_{\alpha}^{2}(\mathbf{r})\,\delta(E_{2}-E_{1}-\hslash\omega_{\alpha}).
\label{4.56}
\end{equation}
Verjetnost za prehod je sorazmerna z $n_{\alpha}+1$ in je od nič
različna, tudi če je število kvantov polja enako nič. To opisuje 
spontano sevanje\index{Spontano sevanje}. Prispevek, ki je 
sorazmeren s številom že prisotnih fotonov, predstavlja stimulirano 
sevanje\index{Stimulirano sevanje}. Verjetnost za prehod vsebuje
še kvadrat prostorske odvisnosti polja $E_{\alpha}^{2}(\mathbf{r})$.
Če ne poznamo natančnega položaja atoma ali če je plin atomov enakomerno
porazdeljen po votlini, ta člen nadomestimo s povprečno vrednostjo.
Za stoječe valovanje je to 1/2.

Kolikšna pa je verjetnost za spontano emisijo?
Spontana emisija je mogoča v vsa elektromagnetna nihanja votline s
pravo frekvenco. Celotno verjetnost za prehod atoma iz vzbujenega stanja
v osnovno izračunamo tako, da seštejemo verjetnosti za prehod z izsevanim fotonom 
v določenem stanju. Vemo, da je ta verjetnost ravno enaka 
Einsteinovemu koeficientu $A_{21}$\index{Einsteinovi koeficienti} (enačba~\ref{4.27})
\begin{equation}
A_{21}=\sum_{\alpha}w_{21}=\sum_{\alpha}\frac{\pi \omega_{\alpha}\mathcal{V}^{2}}
{2V\epsilon_{0}}\,\delta(E_{2}-E_{1}-\hslash\omega_{\alpha}).
\label{4.57}
\end{equation}
Za prostorsko odvisnost polja $E^{2}(\mathbf{r})$ smo vzeli povprečje
1/2. Vsoto po nihanjih z uporabo enačbe~(\ref{4.5}) spremenimo v integral
in upoštevamo enačbo~(\ref{4.4}). Dobimo
\begin{equation}
A_{21}=\frac{\pi \mathcal{V}^{2}}{2\hslash\epsilon_{0}}\int\rho(\omega_{\alpha})\omega_\alpha\, 
\delta(\omega_{0}-\omega_{\alpha})\, d\omega_{\alpha}=\frac{\omega_{0}^{3}\mathcal{V}^{2}}{\epsilon_{0} h c^{3}},
\label{4.58}
\end{equation}
 pri čemer smo z $\omega_{0}=(E_{2}-E_{1})/\hslash$ označili frekvenco prehoda. Tako smo 
 izpeljali vrednost Einsteinovega koeficienta $A_{21}$. 
\begin{remark}
Pri izračunu Einsteinovega koeficienta $A_{21}$ smo privzeli, da so vsi dipoli urejeni  
 v smeri jakosti električnega polja svetlobe. Če želimo rezultat izenačiti s koeficientom, ki smo ga vpeljali
 za izotropno sevanje črnega telesa, ga moramo pomnožiti s faktorjem $\langle \cos^2\vartheta
 \rangle = 1/3$.
\end{remark}

Zaradi spontanega sevanja vzbujeno atomsko stanje nikoli ni popolnoma
stacionarno. Poleg tega energija stanja s končnim razpadnim časom ni natančno
določena, zato moramo verjetnost za stimulirano sevanje (enačba~\ref{4.56}) malo 
popraviti. Delta funkcijo energije nadomestimo s končno široko  
funkcijo $g(\omega-\omega_0)$, ki ima vrh pri $\omega_{0}$. Zaradi 
spremembe integracijske spremenljivke dobimo še dodaten faktor $1/\hslash$ in zapišemo
\begin{equation}
w_{21}=\frac{\pi \omega_{\alpha}\mathcal{V}^{2}}{2V\epsilon_{0}\hslash}
(n_{\alpha}+1)g(\omega_{\alpha}-\omega_0).
\label{4.59}
\end{equation}

Poglejmo še Einsteinov koeficient za stimulirano sevanje $B_{21}$. Lahko ga 
izrazimo iz enačbe~(\ref{4.18}), če upoštevamo, da je gostota energije 
polja $n_{\alpha}\hslash\omega_{\alpha}/V$
\begin{equation}
B_{21}=\frac{V\,w_{21}}{n_{\alpha}\,\hslash\omega_{\alpha}\, g(\omega_{\alpha}-\omega_0)}
=\frac{\pi \mathcal{V}^{2}}{2\epsilon_{0}\hslash^{2}}.
\label{4.60}
\end{equation}
Razmerje Einsteinovih koeficientov izračunamo z uporabo enačb~(\ref{4.58}) in 
(\ref{4.60}) in dobimo
\begin{equation}
 \frac{A_{21}}{B_{21}}=\frac{\hslash \omega_0^3}{\pi^2 c^3},
\end{equation}
ki se ujema z razmerjem, ki smo ga izpeljali z uporabo
Planckove formule (enačba~\ref{4.27}). Prehojena pot jasno kaže zvezo med spontanim in
stimuliranim sevanjem ter gostoto stanj elektromagnetnega polja. 

\section{*Rabijeve oscilacije}
Če je svetloba, ki vpada na dvonivojski sistem, zelo močna, se lahko v primeru, da je frekvenca 
vpadne svetlobe $\omega$ blizu frekvence prehoda $\omega_0$, energija med 
svetlobnim poljem in dvonivojskim sistemom periodično izmenjuje.\index{Dvonivojski sistem} 
Oscilacije števila fotonov oziroma pričakovane 
vrednosti zasedenosti nivojev\index{Rabijeve oscilacije} imenujemo Rabijeve 
oscilacije\footnote{Ameriški fizik in nobelovec Isidor Isaac Rabi, 1898--1988.}. 

Obravnavajmo sklopitev dvonivojskega sistema z elektromagnetnim valovanjem 
v semiklasičnem modelu.\index{Semiklasični model} 
To pomeni, da dvonivojski sistem obravnavamo kvantno in 
svetlobo, ki vpada nanj, kot klasično skalarno polje. 
V odsotnosti električnega polja zapišemo Hamiltonian\index{Hamiltonova
funkcija} za elektron kot
\begin{equation}
H_0 = \hslash \omega_1 |1\rangle \langle1| + \hslash \omega_2 |2\rangle \langle2|,
\end{equation}
pri čemer je $\omega_2- \omega_1 = \omega_0$ frekvenca prehoda. V prisotnosti 
svetlobnega polja moramo dodati še člen, ki opisuje dipolno interakcijo. Celoten
Hamiltonian postane časovno odvisen in ga zapišemo kot
\begin{equation}
H = \hslash \omega_1 |1\rangle \langle1| + \hslash \omega_2 |2\rangle \langle2|
-e\hat{x}E_0 \cos (\omega t).
\label{eq:sk-H}
\end{equation}
Schr\"odingerjevo enačbo\index{Schr\"odingerjeva enačba}
\begin{equation}
i \hslash \frac{\partial}{\partial t}|\psi\rangle = H|\psi\rangle
\label{eq:sk-S}
\end{equation}
rešujemo z nastavkom
\begin{equation}
|\psi\rangle = c_1(t)e^{-i \omega_1t}|1\rangle + c_2(t)e^{-i \omega_2t}|2\rangle,
\label{eq:sk-n}
\end{equation}
saj je valovna funkcija, ki popisuje stanje sistema, na splošno
kombinacija obeh stanj. Nastavek (enačba~\ref{eq:sk-n}) in Hamiltonian 
(enačba~\ref{eq:sk-H}) vstavimo v enačbo (\ref{eq:sk-S}), ki jo enkrat pomnožimo 
z $\langle1|$ in drugič z $\langle2|$. Izpeljemo sistem dveh sklopljenih enačb
\begin{equation}
\frac{d c_1}{dt}=-\frac{i}{\hslash} \mathcal{V} E_0\cos (\omega t) e^{-i\omega_0 t}\, c_2 
\qquad \mathrm{in} \qquad
\frac{d c_2}{dt}=-\frac{i}{\hslash} \mathcal{V} E_0\cos (\omega t) e^{i\omega_0 t}\, c_1,
\label{eq:c1c2}
\end{equation}
pri čemer je $\mathcal{V} = -e\langle1|\hat{x}|2\rangle$. Zapišemo $\cos(\omega t)$ kot
kompleksno število in zanemarimo hitro spreminjajočo se komponento pri $\omega_0 + \omega$,
tako da enačbi prepišemo v 
\begin{equation}
\frac{d c_1}{dt}=-\frac{i}{2\hslash} \mathcal{V} E_0 e^{-it\Delta}\, c_2 
\qquad \mathrm{in} \qquad
\frac{d c_2}{dt}=-\frac{i}{2\hslash} \mathcal{V} E_0 e^{it\Delta}\, c_1,
\label{eq:rabi2}
\end{equation}
kjer je $\Delta = \omega_0-\omega$. Dodamo začetni pogoj, ki pravi, da je na začetku
sistem v osnovnem stanju in torej $c_1(0)=1$ in $c_2(0)=0$. Rešitvi enačb
(\ref{eq:rabi2}) sta tako
\begin{align}
c_1(t)&=e^{-it\Delta/2}\, \left(\cos\left(\frac{\Omega t}{2}\right) + 
i \frac{\Delta}{\Omega} \sin\left(\frac{\Omega t}{2}\right) \right)\qquad \mathrm{in} 
\label{eq:rabi3} \\
c_2(t)&=\frac{\mathcal{V}E_0}{i\hslash \Omega} e^{it\Delta/2}\, \sin\left(\frac{\Omega t}{2}\right).
\label{eq:rabi4}
\end{align}
Pri tem smo vpeljali krožno frekvenco
\begin{equation}
\Omega = \sqrt{\Delta^2 + \left(\frac{\mathcal{V}E_0}{\hslash}\right)^2} = \sqrt{(\omega_0-\omega)^2 
+ \left(\frac{\langle1|\hat{x}|2\rangle\, eE_0}{\hslash}\right)^2} = 
\sqrt{(\omega_0-\omega)^2 + \Omega_R^2}.
\end{equation}
Krožno frekvenco $\Omega_R = \mathcal{V} E_0/\hslash$ imenujemo Rabijeva krožna frekvenca.\index{Rabijeva frekvenca}

\begin{definition}
Pokaži, da enačbi~(\ref{eq:rabi3}) in (\ref{eq:rabi4}) rešita
sistem enačb (\ref{eq:rabi2}) ob izbranih začetnih pogojih.
\end{definition}
Poglejmo rezultat podrobneje. Verjetnost, da najdemo atom v stanju $|2\rangle$, je enaka
\begin{equation}
P_2(t) = |c_2(t)|^2 = \frac{\mathcal{V}^2E_0^2}{\hslash^2 \Omega^2}\sin^2(\Omega t/2).
\end{equation}
Če je frekvenca vpadne svetlobe točno enaka frekvenci prehoda, je $\Delta = 0$ in 
$\Omega = \Omega_R = \mathcal{V}E_0/\hslash$. Takrat je amplituda nihanja zasedenosti vzbujenega stanja kar enaka 1
in sistem v celoti periodično prehaja iz osnovnega stanja v vzbujeno in nazaj (slika~\ref{fig:Rabi}). To pomeni,
da prihaja izmenično do popolne absorpcije svetlobe in do popolne stimulirane emisije.
Pri odstopajoči vpadni frekvenci se amplituda nihanja zmanjša, hkrati se poveča
frekvenca oscilacij. Frekvenca oscilacij ni odvisna zgolj od frekvence vpadnega valovanja, 
ampak tudi od jakosti električnega polja. Groba ocena
Rabijeve frekvence je $\Omega_R \sim~\si{MHz}$.
\begin{figure}[h]
\centering
\def\svgwidth{100truemm} 
\input{slike/05_rabi.pdf_tex}
\caption{Rabijeve oscilacije za tri različne vrednosti odstopanja frekvence vpadne
svetlobe od frekvence prehoda $\Delta=\omega_0-\omega$. 
Z naraščajočim odstopanjem se amplituda oscilacij
zmanjšuje, njihova frekvenca pa povečuje.}
\label{fig:Rabi}
\end{figure}

Omenjeno velja, kadar je vpadna svetloba zelo močna, povsem koherentna in v sistemu ni 
motenj. V realnih sistemih so prisotni relaksacijski pojavi, kot na primer trki med atomi
ali spontana emisija, zato so Rabijeve oscilacije dušene. Zaznamo jih lahko le v času, ki 
je krajši od obratne vrednosti širine spektralne črte ($\sim 10^{-10}~\si{s}$).

\begin{remark}
Rabijeve oscilacije niso omejene samo na optične prehode, ampak se pojavijo pri 
vrsti dvonivojskih sistemov, ki interagirajo z močnim spreminjajočim se zunanjim poljem. 
Poznamo jih na primer pri jedrski magnetni resonanci (NMR) ali kvantnih logičnih vezjih.
\end{remark}
