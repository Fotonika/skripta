\chapterimage{slike/Trinivojski.png} 
% Chapter heading image
\chapter{Interakcija svetlobe s snovjo}

V prejšnjih poglavjih smo obravnavali svetlobo v praznem prostoru. Oglejmo si
zdaj osnovne procese interakcije svetlobe s snovjo. To je seveda zelo
obširna tema in je bomo obdelali le toliko, kolikor jo potrebujemo za
razumevanje ojačevanja svetlobe s stimulirano emisijo, ki je osnova za
delovanje laserjev. Najprej bomo na kratko pogledali termodinamsko ravnovesje 
svetlobe v stiku s toplotnim zalogovnikom, torej sevanje črnega telesa, ki 
zahteva kvantno obravnavo elektromagnetnega polja. Nato bomo vpeljali fenomenološki
Einsteinov opis mikroskopskih procesov absorpcije, spontane in stimulirane
emisije in pokazali, da ti procesi niso neodvisni. Izpeljali bomo
izraze za absorpcijski koeficient in koeficient ojačenja, na koncu poglavja
pa bomo nakazali še kvantnomehansko izpeljavo verjetnosti za prehod
atoma iz višjega energijskega stanja v nižje s sevanjem.

\section{Kvantizacija elektromagnetnega polja}
\index{Kvantizacija polja}
Ravni valovi\index{Ravni val} so enostavne in zelo prikladne rešitve valovne 
enačbe~(enačba~\ref{eq:valovna-skalarna}), zato jih pogosto uporabimo kot 
bazo, po kateri razvijemo elektromagnetno polje\index{Elektromagnetno valovanje}. Možen je razvoj
po celotnem prostoru, vendar je tedaj nekoliko nerodna normalizacija baznih
funkcij. Če se omejimo na le del prostora, se temu problemu izognemo. Mora pa biti 
izbrani del prostora dovolj velik, da končni rezultat ni odvisen od izbire 
njegove velikosti in oblike.

Najpreprosteje je vzeti votlino v obliki velike kocke s stranico
$L$ in idealno prevodnimi stenami. Rešitve Maxwellovih enačb~(enačbe~\ref{eq:Maxwell1}--\ref{eq:Maxwell4}) 
znotraj take votline so ob upoštevanju robnih pogojev 
(enačbe~\ref{eq:robni-pogoji}--\ref{eq:robni-pogoji5}) 
\index{Stoječe valovanje}stoječa valovanja. Zapišemo jih v obliki
\begin{eqnarray}
E_{x} & = & E_{x0}\cos\frac{\pi lx}{L}\sin\frac{\pi my}{L}\sin\frac{\pi nz}{L}e^{-i\omega t},\nonumber \\
E_{y} & = & E_{y0}\sin\frac{\pi lx}{L}\cos\frac{\pi my}{L}\sin\frac{\pi nz}{L}e^{-i\omega t},\nonumber \\
E_{z} & = & E_{z0}\sin\frac{\pi lx}{L}\sin\frac{\pi my}{L}\cos\frac{\pi nz}{L}e^{-i\omega t},
\label{eq:stojece_votlina}
\end{eqnarray}
kjer so $l,m$ in $n$ cela števila. Vsako stoječe valovanje je določeno z valovnim 
vektorjem\index{Valovni vektor}
\begin{equation}
\mathbf{k}=\left(\frac{\pi l}{L},\frac{\pi m}{L},\frac{\pi n}{L}\right),
\end{equation} 
katerega velikost je povezana s frekvenco $k = \omega/c$.
Iz Maxwellove enačbe za prazen prostor (enačba~\ref{eq:Maxwell3}) 
$\nabla\cdot\mathbf{E}=0$ sledi $\mathbf{k}\cdot\mathbf{E}=0$. 
Za vsako trojico števil $l$, $m$ in $n$ obstajata tako dve
neodvisni polarizaciji.

\begin{definition}
 Pokaži, da stoječe valovanje, zapisano z enačbami~(\ref{eq:stojece_votlina}), reši 
 valovno enačbo (enačba~\ref{eq:valovna-skalarna}) v 
 kocki s stranico $L$ in zadosti robnim pogojem idealno prevodnih sten votline.
\end{definition}

Preštejmo, koliko je lastnih valovanj v intervalu velikosti valovnega
vektorja med $k$ in $k+dk$ -- to smo na hitro naredili že pri obravnavni
resonatorjev (enačba~\ref{eq:N-stevilo-stanj}). Možni valovni vektorji tvorijo tridimenzionalno
mrežo v prvem oktantu prostora vseh valovnih vektorjev. Razmik med
dvema zaporednima mrežnima točkama v smeri ene od osi je $\pi/L$.
Število točk v osmini krogelne plasti med $k$ in $k+dk$ je za dovolj
velike $l$, $m$ in $n$ enako prostornini plasti, deljeni
s prostornino, ki pripada posamezni mrežni točki, to je $(\pi/L)^{3}$.
Upoštevati moramo še, da sta pri vsakem $\mathbf{k}$ dovoljeni dve polarizaciji, zato
\begin{equation}
dN=\left(\frac{L}{\pi}\right)^{3}\pi k^{2}\, dk.
\label{4.2}
\end{equation}
Zapišemo število stanj na enoto volumna
\begin{equation}
\frac{dN}{V}=\frac{ k^{2}}{\pi^{2}} dk
\label{4.3}
\end{equation}
in ga prevedemo na frekvenčno odvisnost
\begin{equation}
\frac{dN}{V}=\frac{8 \pi \nu^{2} }{c^{3}}d\nu = \frac{\omega^2}{\pi^2c^3}d\omega.
\end{equation}
Vpeljemo gostoto stanj  $\varrho (\omega)$, to je število valovanj na frekvenčni interval na enoto volumna 
votline\index{Gostota stanj}
\boxeq{4.4}{
\rho(\omega)=\frac{dN}{V d\omega}=\frac{\omega^{2}}{\pi^{2}c^{3}}.
}

Vsote po lastnih valovanjih, to je po dovoljenih vrednostih valovnega vektorja $k$,
lahko z uporabo gostote stanj spremenimo v integrale po $k$ ali po $\omega$
\begin{equation}
\sum_{k}\ldots \quad \rightarrow \quad V\int\rho(k)\ldots dk=V\int\rho(\omega)\ldots d\omega.
\label{4.5}
\end{equation}

Označimo zdaj brezdimenzijski krajevni del rešitve~(enačba~\ref{eq:stojece_votlina}) z 
$E_{\alpha}$, kjer $\alpha$
označuje trojico števil $l$, $m$ in $n$ in še obe možni polarizaciji. 
Pripadajoče magnetno polje izračunamo z Maxwellovo enačbo (enačba~\ref{eq:Maxwell2}) 
\begin{equation}
\nabla\times\mathbf{E}_{\alpha}=i\omega_\alpha\mathbf{B}_{\alpha}.
\label{Maxalfa}
\end{equation}
Polja $\mathbf{E}_{\alpha}$ in $\mathbf{B}_{\alpha}$ tvorijo poln ortogonalen
sistem, zato jih lahko uporabimo za razvoj poljubnega elektromagnetnega polja v votlini
\begin{eqnarray}
\mathbf{E}(\mathbf{r},t) & = & -\frac{1}{\sqrt{V\epsilon_{0}}}
\sum_{\alpha}p_{\alpha}(t)\mathbf{E}_{\alpha}(\mathbf{r}) \quad \mathrm{in}\nonumber \\
\mathbf{B}(\mathbf{r},t) & = & i\sqrt{\frac{\mu_{0}}{V}}c_0\sum_{\alpha}
\omega_{\alpha}q_{\alpha}(t)\mathbf{B}_{\alpha}(\mathbf{r}).
\label{eq:pqrazvoj}
\end{eqnarray}
Vstavimo splošen razvoj~(enačbi~\ref{eq:pqrazvoj}) v Maxwellovi enačbi~(enačbi~\ref{eq:Maxwell1}
in \ref{eq:Maxwell2}), upoštevamo zvezo~(enačba~\ref{Maxalfa}) in njej analogno za rotor magnetnega polja
in dobimo 
\begin{equation}
p_{\alpha}=\dot{q}_{\alpha} \quad \mathrm{in} \quad 
\omega_{\alpha}^{2}q_{\alpha}=-\dot{p}_{\alpha},
\label{4.7}
\end{equation}
od koder sledi še 
\begin{equation}
\ddot{p}_{\alpha}+\omega_{\alpha}^{2}p_{\alpha}=0.
\label{4.8}
\end{equation}
Ta enačba da seveda pričakovano časovno odvisnost oblike $e^{-i \omega_\alpha t}$.

\begin{definition}
 Uporabi razvoj polja (enačbi~\ref{eq:pqrazvoj}) in iz Maxwellovih enačb izpelji
 enačbo~(\ref{4.8}).
\end{definition}

Z upoštevanjem razvoja (enačbi~\ref{eq:pqrazvoj}) lahko zapišemo še energijo 
polja -- \index{Hamiltonova funkcija}Hamiltonovo 
funkcijo\footnote{Irski matematik Sir William Rowan Hamilton, 1805--1865.}
\begin{equation}
{\cal H}=\frac{1}{2}\int(\epsilon_{0}E^{2}+\frac{B^{2}}{\mu_0})\, 
dV=\frac{1}{2}\sum_{\alpha}(p_{\alpha}^{2}+\omega_{\alpha}^{2}q_{\alpha}^{2}).
\label{4.9}
\end{equation}
Gornji zapis (enačbi \ref{4.8} in \ref{4.9}) kaže, 
da lahko elektromagnetno polje v votlini
obravnavamo kot sistem neodvisnih harmonskih oscilatorjev\index{Harmonski oscilator}. 
Pri tem se koeficienti razvoja $p_{\alpha}$ in $q_{\alpha}$ obnašajo kot
gibalne količine in koordinate. 

Prehod v kvantno mehaniko dosežemo tako, da klasičnim spremenljivkam gibalne količine
in koordinate priredimo operatorje $\hat{p}_{\alpha}$ in $\hat{q}_{\alpha}$,
ki morajo zadoščati komutacijskim pravilom 
\begin{equation}
[\hat{q}_{\alpha},\hat{p}_{\beta}]=i\hbar \delta_{\alpha, \beta}.
\label{4.10}
\end{equation}

Iz kvantne mehanike vemo, da so lastne vrednosti energije harmonskega oscilatorja, 
opisanega s Hamiltonovo funkcijo (enačba~\ref{4.9}), diskretne. Njihove vrednosti so enake
\boxeq{4.11}{
W_{n,\alpha}=\hbar\omega_{\alpha}(n_{\alpha}+\frac{1}{2}), \quad n_{\alpha}= 0, 1, 2 \ldots
}
{\bf Razliki energije harmonskega oscilatorja, če se $n_{\alpha}$
spremeni za 1, pravimo foton.}\index{Foton} Energija
fotona je tako enaka $\hbar \omega$, $n$ pa predstavlja število fotonov z dano energijo.

\begin{remark}
Vzemimo svetlobo z valovno dolžino $500~\si{\nano\metre}$. Temu ustreza frekvenca
$\omega = 3,8 \cdot 10^{15}~\si{\hertz}$ in energija fotona $W = 4 \cdot 10^{-19}~\si{\joule}$
oziroma $W = 2,5~\mathrm{e}\si{\volt}$.
\end{remark}

Celotno energijo kvantiziranega elektromagnetnega polja \index{Energija polja}v votlini
izračunamo tako, da seštejemo prispevke vseh možnih stanj
\begin{equation}
W=\sum_{\alpha}\hbar\omega_{\alpha}n_{\alpha},
\end{equation}
pri čemer smo izpustili ničelno energijo\index{Ničelna energija}. 
Izpuščali jo bomo tudi v nadaljevanju, saj
je to energija osnovnega stanja, ki se ne more sprostiti. 

\section{Sevanje črnega telesa}
Obravnavajmo sevanje v votlini, ki je v toplotnem ravnovesju s stenami s temperaturo
$T$. Iz statistične fizike vemo, da verjetnost $P$, da je v izbranem stanju 
votline $\alpha$ število fotonov enako $n_{\alpha}$, zapišemo z Boltzmannovo porazdelitvijo
\index{Boltzmannova porazdelitev}
\begin{equation}
P(n_{\alpha})=\frac{e^{-W_{n,\alpha}/k_BT}}{\sum_{n_{\alpha}}e^{-W_{n,\alpha}/k_BT}} = 
\frac{e^{-\beta\hbar\omega_{\alpha}n_{\alpha}}}
{\sum_{n_{\alpha}}e^{-\beta\hbar\omega_{\alpha}n_{\alpha}}}=
e^{-\beta\hbar\omega_{\alpha}n_{\alpha}}(1-e^{-\beta\hbar\omega_{\alpha}}),
\label{4.12}
\end{equation}
pri čemer $\beta = 1/k_BT$ in $k_B$ Boltzmannova konstanta. Povprečno število fotonov 
v stanju $\alpha$ je potem\index{Sevanje črnega telesa}
\begin{equation}
\langle n_{\alpha}\rangle =\sum_{n_{\alpha}}n_{\alpha}P(n_{\alpha})=\frac{1}{e^{\beta\hbar\omega_{\alpha}}-1}.
\label{4.13}
\end{equation}

Povprečno energijo posameznega stanja zapišemo kot produkt energije tega stanja in 
povprečnega števila fotonov v tem stanju
\begin{equation}
\langle W_{\alpha}\rangle = \hbar \omega_\alpha \langle n_\alpha \rangle
= \frac{\hbar \omega_\alpha}{e^{\beta\hbar\omega_{\alpha}}-1}.
\end{equation}
Ravnovesno gostoto energije elektromagnetnega polja\index{Gostota energije} v votlini na
frekvenčni interval $u$ izračunamo tako, da povprečno energijo posameznega
stanja pomnožimo z gostoto stanj $\varrho (\omega)$ 
(enačba~\ref{4.4}). Dobimo znano formulo za energijo na enoto volumna na enoto frekvence, 
to je \index{Planckov zakon}Planckov 
zakon\footnote{Nemški fizik in nobelovec Max Karl Ernst Ludwig Planck, 1858--1947.}.
Planckov zakon opiše spektralno gostoto energije svetlobe\index{Spektralna gostota energije}, izsevane iz 
\index{Sevanje črnega telesa}črnega telesa, ki je v toplotnem ravnovesju z 
okolico s temperaturo $T$
\boxeq{eq:Planck}{
u(\omega)=\hbar\omega\langle n\rangle \rho(\omega)
=\frac{\hbar}{\pi^{2}c^{3}}\frac{\omega^{3}}{e^{\beta\hbar\omega}-1}.
}
Planckov zakon lahko zapišemo tudi z valovno dolžino in dobimo energijo na enoto volumna
na interval valovne dolžine\index{Spekter!Planckov}
\begin{equation}
u(\lambda)=\frac{8 \pi h c}{\lambda^5}\frac{1}{e^{\beta h c/\lambda}-1}.
\end{equation}

\begin{figure}[h]
\centering
\def\svgwidth{110truemm} 
\input{slike/05_Planck.pdf_tex}
\caption{Planckov spekter za sevanje črnega telesa pri različnih temperaturah}
\label{fig:Planck}
\end{figure}

\section{Absorpcija, spontano in stimulirano sevanje}
Oglejmo si zdaj osnovne procese interakcije svetlobe s snovjo. Naj
bo v votlini poleg elektromagnetnega polja še $N$ atomov, ki se med
seboj ne motijo. Za začetek naj bodo atomi prav enostavni\index{Dvonivojski sistem}:
imajo naj le dve energijski stanji z energijama $E_{1}$ in $E_{2}$ (slika~\ref{sl4.1}\,a). 
Stanje $E_2$ naj ima večjo energijo od $E_1$, razlika med njima pa naj bo
\begin{equation}
 E_2 - E_1 = \hbar \omega_0.
\end{equation}

Zaradi interakcije s poljem pri frekvenci prehoda $\omega_{0}$
atomi prehajajo iz nižjega stanja v višje in obratno. Prehajanje 
med obema stanjema opisujejo trije procesi: 
spontano sevanje, absorpcija in stimulirano sevanje.

\begin{figure}[h]
\centering
\def\svgwidth{140truemm} 
\input{slike/05_Dvonivojski.pdf_tex}
\caption{Shema energijskih nivojev dvonivojskega atoma (a) in prehodov med njima:
spontano sevanje (b), absorpcija (c) in stimulirano sevanje (d).}
\label{sl4.1}
\end{figure}

\subsection*{Spontano sevanje}
Vemo, da atom v vzbujenem stanju tudi brez vpliva zunanjega polja
ni stabilen, temveč prej ali slej preide v nižje stanje. Temu pojavu
pravimo spontano sevanje\index{Spontano sevanje} ali spontana emisija (slika~\ref{sl4.1}\,b). 
Pri spontanem sevanju je foton izsevan v katerokoli stanje polja v bližini 
frekvence prehoda, pri tem sta smer in polarizacija izsevane svetlobe poljubni.
Verjetnost za prehod na časovno enoto\index{Verjetnost za prehod} označimo z $A_{21}$.
Za dovoljene prehode je vrednost $A_{21} \sim 10^6$--$10^8/\si{\second}$, 
za prepovedane pa okoli $\sim 10^4/\si{\second}$. 
Karakteristični (naravni) razpadni čas\index{Razpadni čas} gornjega stanja vpeljemo kot
$\tau = 1/A_{21}$. 

Zaradi končnega življenjskega časa ima vzbujeno stanje končno spektralno 
širino. Najpogosteje je atomska spektralna črta\index{Spektralna črta} kar 
Lorentzove oblike\index{Spekter!Lorentzov} z vrhom pri $\omega_0$
(enačba~\ref{eq:spekter-primer})
\boxeq{4.21}{
g(\omega-\omega_0)=\frac{1}{\pi}\frac{\gamma}{(\omega-\omega_{0})^{2}+\gamma^{2}}.
}
Funkcija $g(\omega)$ je normirana
\begin{equation}
\int_{-\infty}^\infty g(\omega)\, d\omega=1,
\label{4.20}
\end{equation}
za grobe ocene pa funkcijo $g$ aproksimiramo tudi s pravokotnikom širine
$2\gamma$ in višine $1/2\gamma$.

\subsection*{Absorpcija fotona}
Absorpcija fotona\index{Absorpcija fotona} je prehod, pri katerem se foton 
z ustrezno energijo absorbira, atom pa preide iz nižjega energijskega stanja v višje (slika~\ref{sl4.1}\,c). 
Verjetnost za prehod na časovno enoto\index{Verjetnost za prehod} $r_{12}$ je sorazmerna 
spektralni gostoti energije polja \index{Spektralna gostota energije}$u(\omega)$, to je
energiji na enoto volumna in frekvenčni interval, 
pri frekvenci prehoda $\omega_{0}$. Sorazmernostni koeficient označimo z $B_{12}$ in 
zapišemo
\begin{equation}
r_{12}=B_{12}u(\omega_{0}).
\label{4.16}
\end{equation}
To je enostavno razumeti. Več kot je fotonov v votlini pri frekvenci, ki je
v bližini frekvence prehoda, več fotonov se bo absorbiralo in večja je 
verjetnost za prehod atoma v višje stanje. Pri absorpciji se
seveda število fotonov v enem od stanj polja pri frekvenci
$\omega_{0}$ zmanjša za ena.

\subsection*{Stimulirano sevanje}
Tretji pojav je prehod atoma iz višjega stanja v nižje zaradi interakcije
s poljem. Temu procesu pravimo stimulirano sevanje\index{Stimulirano sevanje} ali 
stimulirana emisija. Tudi verjetnost za stimuliran prehod na časovno enoto $r_{21}$ 
je sorazmerna s spektralno gostoto energije polja pri frekvenci prehoda $\omega_{0}$
\begin{equation}
r_{21}=B_{21}u(\omega_{0}).
\label{4.17}
\end{equation}
V tem primeru smo sorazmernostni koeficient označili z $B_{21}$. Kadar pride do
stimuliranega sevanja, se število atomov v vzbujenem stanju zmanjša, 
število fotonov v stanju, ki je prehod povzročilo, pa se poveča za ena. 
Izsevana svetloba ima zato enako fazo, frekvenco, polarizacijo in smer potovanja kot 
vpadla. Tipične vrednosti parametra so $B_{21} \sim 10^{16}$--$10^{20}~\si{\metre^3/\joule\second^2}$.

Preden nadaljujemo, se še nekoliko pomudimo pri izrazih za absorpcijo
(enačba~\ref{4.16}) in stimulirano emisijo (enačba~\ref{4.17}).
Zapisani enačbi veljata le, kadar je spektralna gostota \index{Spektralna gostota energije}
elektromagnetnega polja $u(\omega)$
preko celotne spektralne širine prehoda približno konstantna 
(slika~\ref{fig:spektri}\,a). To je gotovo res, če
obravnavamo sevanje v votlini, ki je v termičnem ravnovesju (črno telo). 

\begin{figure}[h]
\centering
\def\svgwidth{140truemm} 
\input{slike/05_Spektri.pdf_tex}
\caption{Pri izračunu verjetnosti za absorpcijo in stimulirano emisijo je 
pomembno, ali je spektralna gostota elektromagnetnega polja $u(\omega)$ bistveno širša
(a) ali bistveno ožja (b) od širine atomske spektralne črte $g(\omega-\omega_0)$.}
\label{fig:spektri}
\end{figure}

V splošnem primeru, ko se spekter vpadne svetlobe spreminja v območju 
frekvence prehoda, moramo sešteti prispevke po ozkih frekvenčnih intervalih
\begin{equation}
r_{12}=B_{12}\int g(\omega-\omega_0)\, u(\omega)\, d\omega.
\label{4.19}
\end{equation}
Gornji zapis preverimo na primeru spektra črnega telesa, ki se ne spreminja 
dosti v območju prehoda. Takrat $u(\omega_0)$ postavimo pred integral in po pričakovanju
dobimo znano zvezo~(enačba~\ref{4.16}). 

Če pa na atome svetimo s svetlobo s spektrom, ki je ozek v primerjavi s spektralno 
širino prehoda (na primer iz laserskega resonatorja), je verjetnost za prehod 
odvisna tudi od tega, kako blizu centralne frekvence prehoda je frekvenca vpadne 
svetlobe (slika~\ref{fig:spektri}\,b). Naj bo  $w_{R}$ gostota energije skoraj
monokromatske svetlobe s frekvenco $\omega_R$. Verjetnost za absorpcijo na časovno 
enoto je potem  
\boxeq{4.18}{
r_{12}=B_{12}g(\omega_R-\omega_0)\, w_{\omega R}.
}

Koeficiente $A_{21}$, $B_{12}$ in $B_{21}$, s katerimi smo opisali spontano sevanje,
absorpcijo in stimulirano emisijo, je prvi vpeljal Einstein\footnote{Nemški fizik
in nobelovec Albert Einstein, 1879--1955.}, zato jih imenujemo 
tudi Einsteinovi koeficienti\index{Einsteinovi koeficienti}. Poglejmo si jih podrobneje.

\subsection*{Einsteinovi koeficienti}
\label{AB}
Zasedenost stanj\index{Zasedenost stanj} pove število atomov v določenem stanju. 
Ker zaenkrat obravnavamo preproste modele atomov z zgolj 
dvema stanjema, zapišemo samo dve zasedenosti\index{Dvonivojski sistem}. Naj bo $N_1$ zasedenost 
nižjega stanja, $N_{2}$ zasedenost višjega stanja, skupno število atomov pa
$N=N_1+N_2$. V prisotnosti svetlobe se število atomov v spodnjem in zgornjem 
stanju v splošnem spreminja, skupno število pa se ohranja.

Obravnavajmo termično ravnovesje, ko je spekter svetlobe bistveno širši
od širine atomskega prehoda (slika~\ref{fig:spektri}\,a), tako da lahko za zapis
verjetnosti za prehod uporabimo enačbi~(\ref{4.16}) in (\ref{4.17}). Zasedenost višjega nivoja
se zmanjšuje zaradi spontanih in stimuliranih prehodov v nižje
stanje, povečuje pa se zaradi absorpcije. To zapišemo z enačbo
\begin{equation}
\frac{dN_{2}}{dt}=-A_{21}N_2 - r_{21}N_2 + r_{12}N_1 = 
-A_{21}N_{2}-B_{21}u(\omega_{0})N_{2}+B_{12}u(\omega_{0})N_{1}.
\label{4.22}
\end{equation}
Zaradi ohranitve skupnega števila atomov velja 
\begin{equation}
\frac{dN_{1}}{dt}=\frac{-dN_{2}}{dt}.
\end{equation}
V termičnem ravnovesju sta zasedenosti konstantni, tako da lahko zapišemo 
\begin{equation}
\frac{dN_{1}}{dt}=A_{21}N_{2}+B_{21}u(\omega_{0})N_{2}-B_{12}u(\omega_{0})N_{1}=0.
\label{4.23}
\end{equation}
Vemo tudi, da v termičnem ravnovesju za zasedenosti $N_{1}$ in $N_{2}$ velja
Boltzmannova porazdelitev\index{Boltzmannova porazdelitev}
\begin{equation}
\frac{N_{2}}{N_{1}}=e^{-\beta(E_{2}-E_{1})} = e^{-\beta \hbar \omega_0},
\label{4.25}
\end{equation}
kjer je $\beta=1/k_BT$. Izrazimo spektralno gostoto\index{Spektralna gostota energije} $u(\omega_0)$ 
iz enačbe~(\ref{4.23})
\begin{equation}
u(\omega_{0})=\frac{A_{21}}{B_{12}\frac{N_{1}}{N_{2}}-B_{21}}
\label{4.24}
\end{equation}
in z uporabo enačbe~(\ref{4.25}) dobimo
\begin{equation}
u(\omega_{0})=\frac{A_{21}/B_{12}}{e^{\beta\hbar\omega_{0}}-B_{21}/B_{12}}.
\label{4.26}
\end{equation}
Po drugi strani pa vemo, da je v termičnem ravnovesju spektralna gostota energije sevanja
$u(\omega_0)$ kar enaka termični Planckovi gostoti $u(\omega_{0})$~(enačba~\ref{eq:Planck}).
Iz primerjave obeh zapisov ugotovimo, da morata biti koeficienta $B_{21}$ in $B_{12}$ enaka,
med $A_{21}$ in $B_{12}$ pa velja zveza 
\boxeq{4.27}{
A_{21}=\frac{\hbar\omega^{3}}{\pi^{2}c^{3}}\, B_{12} \qquad \mathrm{in} \qquad B_{12}=B_{21}.
}
Koeficient pred $B_{12}$ je ravno enak gostoti stanj\index{Gostota stanj} elektromagnetnega polja 
$\rho(\omega)$ (enačba~\ref{4.4}), pomnoženi z energijo fotona $\hbar\omega$. 
Videli bomo, da to ni slučaj, saj to izhaja iz verjetnosti za prehod v kvantni 
elektrodinamiki (poglavje~\ref{chap:verjetnost}).
Pozoren bralec je lahko tudi opazil, da je z enačbo~(\ref{4.26}),
ki smo jo dobili le z uporabo Boltzmannove porazdelitve za atome, že
določena oblika Planckove formule, ne da bi kar koli rekli o fotonih.

\begin{remark}
 Zveza $B_{12}=B_{21}$ velja le v primeru nedegeneriranih stanj. V realnih sistemih
 so stanja pogosto degenerirana in je treba gornje enačbe ustrezno popraviti
\begin{equation}
\frac{B_{21}}{B_{12}} = \frac{g_1}{g_2},
\label{eq:ABdeg}
\end{equation}
pri čemer $g_{1}$ in $g_2$ označujeta degeneriranost stanj. 
\end{remark}

\section{Absorpcijski koeficient}
\index{Absorpcija}
Naj na izbran volumen plina vpada snop svetlobe s frekvenco
$\omega$, ki je blizu frekvence atomskega prehoda $\omega_{0}$. Gostota
vpadnega energijskega toka je $j_{\omega}=w_{\omega}c$ (enačba~\ref{eq:jcw}), 
pri čemer je $w_{\omega}$ gostota energije. Obravnavajmo primer, ko je 
spekter vpadnega snopa ozek v primerjavi s širino atomskega prehoda
(slika~\ref{fig:spektri}\,b). V tej obliki je zapis enačb sicer bolj zapleten,
a hkrati bolj priročen pozneje pri obravnavi laserja. Privzemimo še, da
je stanje stacionarno. 

Ko svetlobni snop vpade na plast plina debeline $dz$, se gostota
energijskega toka zmanjša zaradi absorpcije in hkrati poveča zaradi 
stimulirane emisije (slika~\ref{fig:abs}). 
Spontano sevanje, ki je seveda tudi prisotno, lahko zanemarimo, saj
je svetloba izsevana na vse strani enakomerno in le majhen del je izsevan v smeri snopa.
Sprememba energije snopa na enoto časa je enaka razliki med 
številom absorpcij in stimuliranih prehodov na enoto časa, pomnoženih z 
energijo fotona\footnote{Za poenostavitev tukaj pišemo obliko atomske spektralne črte kot $g$, 
pri čemer je to Lorentzova krivulja okoli osrednje frekvence $\omega_0$.}
\begin{equation}
dP=r_{12}\,\frac{(N_{2}-N_{1})}{V}\,\,S dz\, \, \hbar\omega = 
\frac{(N_{2}-N_{1})}{V}\,B_{21}g w_{\omega} \, \hbar\omega \,S dz,
\label{4.28}
\end{equation}
pri čemer smo verjetnost za prehod izrazili iz 
enačbe~(\ref{4.18}).
\begin{figure}[h]
\centering
\def\svgwidth{70truemm} 
\input{slike/05_Absorpcija.pdf_tex}
\caption{K absorpciji svetlobe v plasti atomov}
\label{fig:abs}
\end{figure}

S $S$ smo označili presek snopa, z $V$ pa volumen plina. Sledi
\begin{equation}
dj=\frac{(N_{2}-N_{1})}{V}\, B_{21}g\, \hbar\omega\,\frac{j_{\omega}}{c}\, dz.
\label{4.29}
\end{equation}
Priročno je vpeljati presek za absorpcijo\index{Presek za absorpcijo} 
\boxeq{sigmaabs}{
\sigma(\omega)=\frac{B_{21}\, g\, \hbar\omega}{c}.
}
Z njim se izraz (\ref{4.29}) poenostavi v 
\boxeq{4.30}{
\frac{dj}{dz}=\frac{\Delta N}{V}\sigma(\omega)j,
}
kjer $\Delta N$ označuje $N_{2}-N_{1}$.
Navadno obravnavamo pline, ki so blizu termičnega ravnovesja. V tem primeru 
je $N_{2}<N_{1}$ in $dj$ negativen, zato pride do
absorpcije z absorpcijskim koeficientom\index{Absorpcijski koeficient}
$\mu$. Zapišemo 
\begin{equation}
\frac{dj}{j} = -\mu dz
\label{eq:jabs}
\end{equation}
in
\begin{equation}
\mu(\omega)=\frac{\Delta N}{V}\sigma(\omega)=
\frac{\Delta N}{V}\, B_{21}\, g\frac{\hbar\omega}{c}.
\label{eq:muabs1}
\end{equation}
Tako smo makroskopski koeficient absorpcije v plinu atomov povezali
z Einsteinovim koeficientom $B_{21}$. Povejmo še, da so 
tipične velikosti presekov $\sigma \sim 10^{-22}$--$10^{-16}~\si{\metre^2}$. 

\begin{remark}
Energija se pri absorpciji na plinu dvonivojskih atomov seveda
ne izgublja. Atom, ki je prešel v vzbujeno stanje, se s spontano 
emisijo vrne nazaj v osnovno, svetloba pa se izseva na vse strani -- se siplje. 
\end{remark}

\section{Nasičenje absorpcije}
\label{chap:NasAbs}
Čeprav je videti izraz za zmanjševanje 
gostote svetlobnega toka pri prehodu skozi absorbirajoči plin (enačba~\ref{eq:jabs}) 
preprost, ga ni mogoče enostavno integrirati, saj je absorpcijski koeficient 
$\mu$ odvisen od 
gostote energijskega toka. Pri dovolj velikem svetlobnem toku namreč z 
absorpcijo znaten delež atomov preide v višje stanje, zato se zmanjša razlika $\Delta N$
in posledično se zmanjša tudi absorpcijski koeficient $\mu$. Takrat se absorpcija
v plinu nasiti in pojavu pravimo nasičenje absorpcije\index{Nasičena absorpcija}.

Naj na plin vpada snop monokromatske svetlobe. 
Atomi v plinu prehajajo med nivoji zaradi absorpcije, spontane in stimulirane emisije. 
Podobno kot smo zapisali termično ravnovesje v primeru
širokega spektra (enačba~\ref{4.23}), zapišemo stacionarno enačbo 
\begin{equation}
\frac{dN_{1}}{dt}=A_{21}N_{2}+B_{21}\,g\,\Delta N\,\frac{j}{c}=0,
\label{4.32}
\end{equation}
pri čemer smo za verjetnost za prehod vzeli
enačbo~(\ref{4.18}) in upoštevali $w=j/c$. Zasedenost višjega stanja $N_{2}$ lahko izrazimo s 
celotnim številom atomov $N$ in razliko zasedenosti 
\begin{equation}
N_{2}=\frac{1}{2}(N_1+N_2) + \frac{1}{2}(N_2-N_1) = \frac{1}{2}N+\frac{1}{2}\Delta N.
\label{4.321}
\end{equation}
S tem lahko izračunamo razliko zasedenosti 
\begin{equation}
\Delta N=-\frac{N}{1+2\frac{Bg}{cA}j}.
\label{4.33}
\end{equation}
Pri majhni gostoti toka $j$ so praktično vsi atomi v osnovnem stanju in prispevajo
k absorpciji. Pri velikih gostotah toka pa imenovalec gornjega
izraza močno naraste, razlika zasedenosti gre proti nič in absorpcija se zmanjšuje.
Ko drugi člen v imenovalcu (enačba~\ref{4.33}) doseže vrednost 1, pravimo, da
gostota energijskega toka doseže vrednost saturacijske gostote\index{Saturacijska gostota toka}.
Zapišemo jo kot 
\begin{equation}
j_{s}(\omega)=\frac{cA_{21}}{2B_{21}g}=
\frac{\hbar\omega^{3}}{2\pi^{2}c^{2}g},
\label{4.34}
\end{equation}
pri čemer smo upoštevali zvezo med koeficientoma $A_{21}$ in $B_{21}$
(enačba~\ref{4.27}).
Kot vidimo, je saturacijska gostota odvisna le od frekvence vpadnega valovanja 
in širine atomskega prehoda. Za črto z valovno dolžino $600~\si{\nano\metre}$ 
in širino $10^{8}~\si{\hertz}$ znaša saturacijska gostota svetlobnega toka okoli 
$20~\si{\milli\watt/\centi\metre^2}$. Tako veliko gostoto svetlobnega toka je v tako ozkem
frekvenčnem intervalu z običajnimi svetili praktično nemogoče doseči, 
medtem ko jo z laserji z lahkoto.

Izraz za razliko zasedenosti stanj lahko zdaj zapišemo v preglednejši obliki
\begin{equation}
\Delta N=-\frac{N}{1+j/j_{s}(\omega)}.
\label{4.35}
\end{equation}
Vstavimo gornji izraz v enačbo za zmanjševanje gostote toka (enačba~\ref{4.30}). Sledi
\boxeq{4.36}{
dj=-\frac{\mu_{0}}{1+j/j_{s}}\, j\, dz,
}
kjer je 
\boxeq{eq:mu0abs}{
\mu_{0}=\frac{N}{V}\sigma = \frac{N\,B_{21}\,g\,\hbar\omega}{Vc}
}
absorpcijski koeficient\index{Absorpcijski koeficient} pri majhnih gostotah vpadnega toka.

Enačbo (\ref{4.35}) brez težav integriramo in dobimo 
\begin{equation}
\ln\frac{j}{j_{0}}+\frac{j-j_{0}}{j_{s}}=-\mu_{0}\, z,
\label{4.37}
\end{equation}
kjer smo z $j_{0}$ označili začetno gostoto toka. Kadar je ta dosti
manjša od $j_{s}$, lahko drugi člen v gornji enačbi zanemarimo in dobimo 
navadno eksponentno pojemanje
\begin{equation}
j = j_0 e^{-\mu_0 z}.
\end{equation}
Pri zelo velikih vpadnih gostotah pa lahko zanemarimo prvi člen in pride do 
linearnega zmanjševanja gostote svetlobnega toka
\begin{equation}
j=j_{0}-\mu_{0}j_{s}z = j_0 - \frac{N}{2V}\, A\, \hbar \omega\, z.
\label{4.38}
\end{equation}
V primeru močnega vpadnega toka je zasedenost osnovnega in vzbujenega nivoja skoraj
enaka in absorpcija je omejena s tem, kako hitro se lahko atomi vračajo
v osnovno stanje preko spontanega sevanja. To je razvidno tudi iz
zadnje oblike zapisa~(enačba~\ref{4.38}).

\begin{figure}[h]
\centering
\def\svgwidth{100truemm} 
\input{slike/05_jabs.pdf_tex}
\caption{Pojemanje gostote svetlobnega toka v absorbirajočem plinu (enačba~\ref{4.37})}
\label{fig:abs2}
\end{figure}

\section{Optično ojačevanje}
V prejšnjih razdelkih smo obravnavali prehod svetlobe skozi dvonivojski plin. V primeru
termičnega ravnovesja je zgornji nivo manj zaseden od spodnjega in v plinu pride
do absorpcije svetlobe. Če pa nekako dosežemo stanje obrnjene zasedenosti\index{Obrnjena zasedenost},
za katerega velja $N_{2}>N_{1}$, se snop svetlobe pri prehodu skozi plin ojačuje. 
Tako stanje seveda ni v termičnem ravnovesju in ga je treba vzdrževati z dovajanjem 
energije plinu oziroma črpanjem\index{Črpanje}. 
Načinov, kako dosežemo obrnjeno zasedenost s črpanjem je veliko.  

V plinih je najpogostejši način vzbujanja z električnim tokom. Elektroni,
ki so glavni nosilci toka, se zaletavajo v atome ali ione in jih vzbujajo
na višje nivoje, pri čemer lahko pride do obrnjene zasedenosti med
nekim parom nivojev. Primer takega laserja je argonski laser\index{Laser!argonski}. 

Pogost proces v plinih je tudi prenos energije med atomi s trki. V
mešanici dveh plinov, pri katerih se nek nivo enih atomov ujema po energiji z
nekim nivojem drugih atomov, lahko vzbujen atom prve vrste pri trku preda 
energijo brez sevanja atomu druge vrste, ta pa iz osnovnega stanja preide v 
ustrezen višji nivo. Če je pod tem nivojem še drugo vzbujeno stanje, katerega
življenjski čas je krajši od življenjskega časa zgornjega nivoja, pride
do obrnjene zasedenosti. Primer takega laserja je He-Ne laser\index{Laser!He-Ne}.

V trdnih neprevodnih kristalih sta v optičnem področju absorpcija
in sevanje pri določeni valovni dolžini navadno posledica primesi.
Obrnjeno zasedenost para nivojev primesi največkrat dobimo tako, da
kristal obsevamo s svetlobo s frekvenco, ki ustreza prehodu na nek
nivo nad izbranim parom. Primera takega laserja sta Nd:YAG in Ti:safir 
laser. \index{Laser!Nd:YAG} \index{Laser!Ti:safir}

V polprevodnikih \index{Laser!polprevodniški}dosežemo obrnjeno zasedenost med 
prevodnim in valenčnim pasom z vbrizgavanjem elektronov in vrzeli v območje $p$-$n$ 
stika z električnim tokom v prevodni smeri. Bolj podrobno bomo nekaj teh mehanizmov spoznali 
na konkretnih primerih laserjev, ki jih bomo obravnavali v poglavju~\ref{chap:Primeri}. 

\section{Optično črpanje trinivojskega sistema}
Kot primer optičnega ojačevanja si oglejmo najpreprostejši model optičnega črpanja.
Gre za plin atomov s tremi nivoji, tako imenovani trinivojski sistem. Osnovno stanje, ki 
ga označimo z $|0\rangle$,  naj ima energijo $E_0$. Poleg tega imajo atomi še 
dve vzbujeni stanji z energijo $E_1$ (stanje $|1\rangle$) in energijo $E_2>E_1$
(stanje $|2\rangle$)\index{Trinivojski sistem}, tako da je energijska razlika med vzbujenima 
nivojema je $E_2-E_1 = \hbar \omega_0$.

Na tak trinivojski plin svetimo s črpalno svetlobo, ki vzbuja atome iz osnovnega stanja 
$|0\rangle$ v stanje $|2\rangle$, pri čemer je lahko spektralna gostota $u_{p}$ črpalne 
svetlobe široka. Poleg tega naj se po plinu širi še monokromatska svetloba z gostoto 
energije $w$ in frekvenco $\omega$, ki je blizu frekvence prehoda $\omega_{0}$ med 
stanjema $|1\rangle$ in $|2\rangle$. 

Ugotoviti želimo, pri katerih pogojih lahko dosežemo obrnjeno zasedenost med 
stanjema $|1\rangle$ in $|2\rangle$ in s tem ojačevanje svetlobe okoli 
frekvence $\omega_{0}$ (slika~\ref{fig:3nivojski}\,b).

\begin{figure}[h]
\centering
\def\svgwidth{100truemm} 
\input{slike/05_Trinivojski.pdf_tex}
\caption{Shema energijskih nivojev trinivojskega sistema in oznake koeficientov za prehode
med njimi (a). 
V plinskih laserjih imamo navadno stanje obrnjene zasedenosti med drugim in prvim
vzbujenim stanjem (b), v navadnih trdninskih laserjih (npr. rubinskem) pa med 
prvim vzbujenim in osnovnim stanjem (c). Pogosto so laserji štiri- ali večnivojski (d).}
\label{fig:3nivojski}
\end{figure}
\begin{remark}
Trinivojski laserski sistem na sliki (\ref{fig:3nivojski}\,b) je pravzaprav 
poseben primer bolj realističnega štirinivojskega sistema\index{Štirinivojski sistem}, 
pri katerem gornji črpalni nivo sovpada z gornjim laserskim nivojem. Sicer se tretji vzbujeni nivo, 
v katerega črpamo, praviloma zelo hitro prazni v drugega vzbujenega, od tam pa počasi v prvega vzbujenega, 
kot kaže slika~(\ref{fig:3nivojski}\,d).
Obravnava štirinivojskih sistemov je bolj zapletena kot obravnava trinivojskih sistemov, 
ki za opis delovanja laserjev povsem zadošča. Podrobneje bomo štirinivojske sisteme 
predstavili pri obravnavi konkretnih laserskih primerov (poglavje~\ref{chap:Primeri}).
\end{remark}

Zapišimo enačbe za spreminjanje zasedenosti posameznih stanj. Osnovno stanje
$|0\rangle$ se prazni zaradi absorpcije\index{Absorpcija fotona} črpalne svetlobe in polni zaradi
spontanih prehodov\index{Spontano sevanje} iz stanj $|1\rangle$ in $|2\rangle$, stimulirane
prehode iz stanja $|2\rangle$ pa bomo zanemarili. Zasedenost stanja $|2\rangle$ se
povečuje zaradi absorpcije s spodnjih nivojev in zmanjšuje
zaradi spontanega in stimuliranega sevanja\index{Stimulirano sevanje}. Srednje stanje se polni
s stimuliranimi in spontanimi prehodi iz stanja $|2\rangle$ in prazni
zaradi absorpcije v $|2\rangle$ in spontanih prehodov v $|0\rangle$.
Pri tem velja, da je vsota vseh treh zasedenosti enaka številu vseh atomov: $N_{0}+N_{1}+N_{2}=N$. 
Zasedbene enačbe so tako
\begin{eqnarray}
\frac{dN_{0}}{dt} & = & -rN_0+A_{20}N_{2}+A_{10}N_{1} \label{4.39.1}\\
\frac{dN_{1}}{dt} & = & -A_{10}N_{1}+B_{21}\,g\,w\, (N_{2}-N_{1})+A_{21}N_{2} \label{4.39.2}\\
\frac{dN_{2}}{dt} & = & rN_0-A_{20}N_{2}-A_{21}N_{2}-B_{21}\,g\,w\, (N_2-N_1).
\label{4.39}
\end{eqnarray}
Pri zapisu smo predpostavili, da je $N_0 \approx N \gg N_1, N_2$ in zato lahko črpanje $B_{20}\, 
u_{p} (N_0-N_2)$, ki je praktično konstantno, zapisali s koeficientom $r$. Mehanizem črpanja 
smo tako skrili v $r$ in prav nič ni pomembno, na kakšen način poteka.\index{Optično črpanje}
S tem smo obravnavo posplošili z optičnega črpanja na druge sisteme. 

Zanima nas stacionarno stanje, ko so vsi trije časovni odvodi enaki nič. 
Tako iz druge enačbe sistema~(enačba~\ref{4.39.2}) sledi
\begin{eqnarray}
B_{21}\,g\,w\, N_{2}+A_{21}N_{2} = B_{21}\,g\,w\, N_{1} + A_{10}N_{1} 
\end{eqnarray}
in
\begin{eqnarray}
N_2 = \frac{B_{21}\,g\,w + A_{10}}{B_{21}\,g\,w+A_{21}}N_1.  
\end{eqnarray}
Brez škode lahko zanemarimo tudi spontano sevanje iz stanja
$|2\rangle$ v osnovno stanje. Tako iz prve enačbe sistema~(\ref{4.39.1}) dobimo
\begin{equation}
N_1= \frac{rN}{A_{10}}
\end{equation}
in zapišemo razliko zasedenosti kot 
\begin{equation}
N_{2}-N_{1}=\left(\frac{N_2}{N_1}-1\right)N_1=\frac{A_{10}-A_{21}}{A_{21}+
B_{21}g\,w} \,\frac{rN}{A_{10}}.
\label{4.42}
\end{equation}
Iz gornje enačbe sledi, da pride do obrnjene \index{Obrnjena zasedenost}
zasedenosti, če je $A_{10}>A_{21}$, torej kadar je
razpadni čas stanja $|1\rangle$ krajši kot razpadni čas stanja $|2\rangle$.
Tak rezultat smo seveda lahko pričakovali.

V praktični primerih navadno velja $A_{10}\gg A_{21}$. Ob upoštevanju zveze $j=wc$ povežemo
razliko zasedenosti z gostoto vpadnega svetlobnega toka
\begin{equation}
N_{2}-N_{1}=\frac{rN}{A_{21}} \, \frac{1}{1+\frac{B_{21}gj}{c A_{21}}} = 
\frac{rN}{A_{21}} \, \frac{1}{1+j/j_s}.
\label{eq:3n_N}
\end{equation}
Konstante pospravimo v saturacijsko gostota svetlobnega toka\index{Saturacijska gostota toka} 
\begin{equation}
j_s = \frac{c A_{21}}{B_{21}g}.
\label{eq:jsatg}
\end{equation}
\begin{remark}
 Vidimo, da je dobljen izraz za saturacijsko gostoto toka v trinivojskem sistemu zelo 
 podoben saturacijski gostoti za dvonivojski sistem (enačba~\ref{4.34}), razlikujeta se le
v faktorju 2. Do te razlike pride zaradi različnega števila stanj, saj pogoj $N_{1}+N_{2}=N$
v trinivojskem sistemu ne velja. 
\end{remark}
Poglejmo zdaj, kaj se zgodi s svetlobo ob vpadu na plast trinivojskega plina. Naj ima vpadna
svetloba frekvenco $\omega$ in gostoto svetlobnega toka $j=wc$. Račun je zelo podoben 
računu pri absorpciji (enačba~\ref{4.29}). Zapišemo spremembo gostote toka na debelini $dz$ 
\begin{equation}
dj=\frac{(N_{2}-N_{1})}{V}\, B_{21}g\, \frac{\hbar\omega}{c}j\, dz,
\label{eq:dj}
\end{equation}
pri čemer gostota toka $j = wc$ nastopa tudi v izrazu za razliko zasedenosti (enačba~\ref{eq:3n_N}). 
Če to upoštevamo, dobimo diferencialno enačbo za gostoto toka
\begin{equation}
\frac{1}{j}\left(1+\frac{j}{j_{s}}\right)\, dj=G\, dz
\label{4.43}
\end{equation}
oziroma
\boxeq{eq:djG}{
dj=\frac{G}{1+j/j_{s}}\, j\, dz,
}
ki je spet zelo podobna enačbi za absorpcijo (enačba~\ref{4.36}).
Z $G$ smo označili t.i. koeficient ojačenja pri majhnih vpadnih gostotah
toka\index{Koeficient ojačenja}. Podan je z 
\begin{equation}
G=\frac{N}{V}\frac{r}{A_{21}}\sigma=\frac{rNB_{21}\hbar\omega g}{VcA_{21}},
\label{4.44}
\end{equation}
pri čemer smo koeficient ojačenja izrazili s presekom za stimulirano 
\index{Presek za stimulirano sevanje}sevanje $\sigma$. 
Rešitev diferencialne enačbe je prikazana na sliki~(\ref{fig:ojacanje}). 
\begin{figure}[h]
\centering
\def\svgwidth{100truemm} 
\input{slike/05_joja.pdf_tex}
\caption{Naraščanje gostote svetlobnega toka pri optičnem ojačenju}
\label{fig:ojacanje}
\end{figure}

Obnašanje gostote svetlobnega toka ima, tako kot pri absorpciji, dva režima. 
Pri majhnih gostotah toka $j\ll j_{s}$ je naraščanje eksponentno 
\begin{equation}
j(z)=j_{0}e^{Gz}.
\label{4.45}
\end{equation}
Pri velikih gostotah toka pride do nasičenja in gostota svetlobnega
toka narašča linearno
\begin{equation}
j(z)=j_{0}+j_{s}Gz.
\label{4.46}
\end{equation}
 V tem primeru je gostota toka dovolj velika, da vsi atomi, ki jih
s črpanjem spravimo v najvišje stanje, preidejo v stanje $|1\rangle$
s stimuliranim sevanjem. Pri konstantnem črpanju je tedaj 
linearno naraščanje gostote toka razumljivo. 

Pomudimo se še malo pri preseku za stimulirano sevanje $\sigma$
 (enačba~\ref{4.44})\index{Presek
za stimulirano sevanje}. Opazimo, da je ta presek enak preseku za
absorpcijo (enačba~\ref{sigmaabs}) dvonivojskega sistema\index{Presek za absorpcijo}. 
Presek za stimulirano
sevanje je tako odvisen od frekvence in je sorazmeren vrednosti atomske spektralne 
črte pri frekvenci prehoda. Za He-Ne laser\index{Laser!He-Ne}, 
ki deluje pri valovni dolžini 633~nm in ima 
širino prehoda $\Delta \omega \sim 10~\si{\giga\hertz}$, znaša tako  $\sigma \sim 10^{-16}~\si{\metre}^2$, 
za Nd:YAG\index{Laser!Nd:YAG} pri 1064~nm in širini prehoda 
$\Delta \omega \sim 900~\si{\giga\hertz}$ pa  $\sigma \sim 10^{-22}~\si{\metre}^2$.
Zaradi različnih presekov, različnih gostot atomov in različnih načinov črpanja se 
koeficienti ojačenja v večnivojskih sistemih med seboj precej razlikujejo. Primere 
laserjev bomo sicer podrobneje obravnavali v nadaljevanju, zaenkrat povejmo 
le, da je tipičen koeficient ojačenja v He-Ne laserju z dolžino $L = 0,5~\si{\metre}$
enak $GL \sim 1,015$, v Nd:YAG laserju z dolžino ojačevalnega sredstva 
$L = 10~\si{\centi\metre}$ pa $GL \sim 50$. Pri prvem laserju je sicer velik presek za stimulirano sevanje, vendar 
je gostota atomov v obrnjeni zasedenosti razmeroma majhna. V drugem primeru pa močno 
črpanje prevlada nad majhnim presekom in pride do močnega ojačenja.

\section{Homogena in nehomogena razširitev spektralne črte}
\label{Razsiritev}
Doslej smo predpostavili, da svetijo vsi atomi obravnavane snovi
pri isti frekvenci $\omega_{0}$ in z isto spektralno širino, ki smo
jo popisali s funkcijo $g(\omega-\omega_0)$, z vrhom pri $\omega_0$. Če to drži, 
pravimo, da je razširitev spektralne črte homogena\index{Spektralna črta!homogena razširitev}. 
Primera homogene razširitve sta naravna širina in razširitev zaradi trkov med atomi. 
Funkcija $g(\omega-\omega_0)$ je v tem primeru Lorentzove oblike\index{Spekter!Lorentzov} 
\boxeq{eq:homogenasirina}{
g_L(\omega-\omega_0)=\frac{1}{\pi}\frac{\gamma}{(\omega-\omega_{0})^{2}+\gamma^{2}}
}
s širino črte $\Delta \omega_L = 2\gamma$ (glej sliko~\ref{fig:SpekterAc}). 

Spektralna črta je lahko razširjena tudi zato, ker vsi atomi ne svetijo
pri povsem isti frekvenci. Tedaj govorimo o nehomogeni 
razširitvi\index{Spektralna črta!nehomogena razširitev}.
Najpomembnejši primer nehomogene razširitve je Dopplerjeva 
\index{Dopplerjeva razširitev} razširitev v plinu. 
Atomi plina vedno sevajo pri praktično isti frekvenci, vendar jih zaradi gibanja
opazovalec v mirujočem (laboratorijskem) sistemu v skladu z Dopplerjevim pojavom 
zazna pri različnih frekvencah. Tako so opazovane frekvence posameznih atomov $\omega$
odvisne od hitrosti $v$ atoma glede na smer opazovanja. Zapišemo jih kot  
\begin{equation}
\omega=\omega_{0}-\frac{v}{c}\omega_{0}=\omega_{0}-k_{0}v.
\label{4.81}
\end{equation}
Označimo z ${\cal N}(v)$ porazdelitev gostote atomov po hitrostih, pri čemer se omejimo 
le na premikanje v smeri opazovanja. V termičnem ravnovesju je ${\cal N}(v)$
Maxwellova porazdelitev\index{Maxwellova porazdelitev}
\begin{equation}
{\cal N}(v)=\frac{N}{V}\left(\frac{m}{2\pi k_{B}T}\right)^{1/2}e^{-\frac{mv^{2}}{2k_{B}T}},
\label{4.82}
\end{equation}
kjer je $m$ masa posameznega atoma.
Porazdelitev atomov po frekvencah izračunamo tako, da hitrost izrazimo
iz enačbe~(\ref{4.81}), poleg tega dobljeno funkcijo $g_{D}(\omega-\omega_0)$
normiramo. Sledi
\boxeq{4.821}{
g_{D}(\omega-\omega_0)=\frac{c}{\omega_{0}}\left(\frac{m}{2\pi 
k_{B}T}\right)^{1/2}e^{-\frac{mc^{2}}{2k_{B}T}\frac{(\omega-\omega_{0})^{2}}{\omega_{0}^2}}.
}
Dopplerjeva razširitev v plinu je torej Gaussove oblike\index{Spekter!Gaussov}.
Njena širina pri polovični 
višini\footnote{Celotno širino na polovični višini imenujemo FWHM -- \it{Full width at half maximum}.} je
\begin{equation} 
\Delta\omega_{D}=2 \sqrt{\frac{2k_{B}T \ln 2}{mc^{2}}}\omega_{0}.
\label{4.83}
\end{equation}
\begin{definition}
Izpelji obliko nehomogeno razširjene črte za Dopplerjevo razširitev (enačba~\ref{4.821})
in pokaži, da je njena širina podana z enačbo~(\ref{4.83}).
\end{definition}

Izračunajmo Dopplerjevo razširitev na primeru He-Ne laserja. Za prehod
atoma neona pri 632,8~nm in temperaturi 300~K dobimo 
$\Delta\omega_{D}=8\cdot10^{9}~\si{\hertz}$. Dejanske izmerjene vrednosti širine 
črte za He-Ne \index{Laser!He-Ne}laser 
znašajo okoli $10~\si{\giga\hertz}$, kar je znatno več od naravne širine
črte ($7,5~\si{\mega\hertz}$). Še bolj izrazite so nehomogene razširitve v trdninskih laserjih.
V Nd:YAG laserju \index{Laser!Nd:YAG}je naravna širina le okoli $1~\si{\kilo\hertz}$, celotna širina črte
pa $900~\si{\giga\hertz}$. Nehomogena razširitev zaradi Dopplerjevega pojava v 
redkem plinu ali zaradi nehomogenosti v trdnih snoveh je tako kar nekaj redov velikosti 
večja od homogene naravne širine in razširitve zaradi trkov.

\begin{remark}
Pri nehomogenih razširitvah bi za bolj natančen izračun morali upoštevati 
tudi naravno širino posameznega atoma. To bi zapisali s konvolucijo Lorentzove
in Gaussove funkcije in dobili tako imenovan Voigtov 
profil\footnote{Nemški fizik Woldemar Voigt, 1850--1919.}, ki pa ga ne moremo 
preprosto analitično zapisati\index{Spekter!Voigtov}.
\end{remark}

\section{*Nasičenje nehomogeno razširjene absorpcijske črte}
\index{Nasičena absorpcija!nehomogeno razširjene črte}
V razdelku~(\ref{chap:NasAbs}) smo obravnavali nasičenje absorpcije pri homogeno 
razširjenem prehodu. Pri nasičenju absorpcije, kadar prevladuje nehomogena razširitev,
nastopijo pomembni novi pojavi, zato si to podrobneje oglejmo.

Naj na dvonivojski plin \index{Dvonivojski sistem}
vpada močan snop monokromatske svetlobe s frekvenco $\omega_S$,
ki je blizu osrednje frekvence $\omega_{0}$ Dopplerjevo razširjene 
črte\index{Dopplerjeva razširitev}. S svetlobo
lahko sodeluje le skupina atomov, pri kateri se Dopplerjevo premaknjena
frekvenca od $\omega_S$ ne razlikuje več kot za homogeno širino, ki
jo opisuje funkcija $g(\omega-\omega_S)$. Zato ne moremo zapisati zasedbenih
enačb za vse atome hkrati, ampak le za tiste, ki imajo hitrost med
$v$ in $v+dv$ in ki absorbirajo svetlobo pri frekvenci $\omega_{0}-kv$.

Naj bosta ${\cal N}_{1}(v)$ in ${\cal N}_{2}(v)$ hitrostni porazdelitvi
atomov v osnovnem in vzbujenem stanju. Gostota
${\cal N}_{2}(v)$ se spreminja podobno kot celotna
zasedenost v homogenem primeru (enačba~\ref{4.22})
\begin{equation}
\frac{d{\cal N}_{2}(v)}{dt}=-A{\cal N}_{2}(v) -B\, g(\omega_S-\omega_{0}+kv)
\frac{j}{c}\,
\left({\cal N}_{2}(v)-{\cal N}_{1}(v)\right),
\label{4.85}
\end{equation}
 kjer je $j$ gostota vpadnega svetlobnega toka. Upoštevali
smo, da je zaradi Dopplerjevega pojava prehod premaknjen k frekvenci
$\omega_{0}-kv$. Velja tudi
\begin{equation}
{\cal N}_{1}(v)+{\cal N}_{2}(v)={\cal N}(v)
\end{equation}
in 
\begin{equation}
 \frac{d{\cal N}_{2}(v)}{dt}=-\frac{d{\cal N}_{1}(v)}{dt}.
\label{4.86}
\end{equation}

Vpeljimo ${\cal Z}(v)={\cal N}_{1}(v)-{\cal N}_{2}(v)$. Podobno kot 
v enačbi~(\ref{4.321}) zapišemo
\begin{equation}
{\cal N}_{2}(v)=\frac{1}{2}{\cal N}(v)-\frac{1}{2}{\cal Z}(v)
\end{equation}
in dobimo 
\begin{equation}
\dot{{\cal Z}}(v)=-A{\cal Z}(v)+A{\cal N}(v)
-2B\,g(\omega_S-\omega_{0}+kv)\frac{j}{c}
{\cal Z}(v).
\label{4.87}
\end{equation}
V stacionarnem stanju je 
\begin{equation}
{\cal Z}(v)=\frac{{\cal N}(v)}{1+\frac{2B}{Ac}g(\omega_S-\omega_{0}+kv)j}.
\label{4.88}
\end{equation}
 Če je nasičenje majhno, lahko imenovalec razvijemo
\begin{equation}
{\cal Z}(v)\approx{\cal N}(v)\left(1-\frac{2B}{Ac}g(\omega_S-\omega_{0}+kv)j\right).
\label{4.89}
\end{equation}
Porazdelitev ${\cal Z}(v)$ je podobna nemoteni porazdelitvi atomov
po hitrosti ${\cal N}(v)$, le da je pri hitrosti $v=(\omega_{0}-\omega_S)/k$
zmanjšana zaradi vpliva vpadne svetlobe. Atomi s to hitrostjo namreč svetlobo
absorbirajo in s tem prehajajo v gornje stanje. V porazdelitvi
atomov tako nastane vdolbina\index{Bennetova vdolbina}, 
pravimo ji tudi Bennettova
vdolbina\footnote{Ameriški fizik William Ralph Bennett Jr., 1930--2008.}, 
(slika \ref{fig:Bennet}). Širina vdolbine je določena
s homogeno širino prehoda, to je s funkcijo $g(\omega_S-\omega_{0}+kv)$,
globina pa z gostoto vpadnega toka~$j$.
\begin{figure}[h]
\centering
\def\svgwidth{80truemm} 
\input{slike/05_Hole.pdf_tex}
\caption{Porazdelitev atomov po hitrosti v osnovnem stanju, kjer zaradi
absorbirane svetlobe nastane Bennettova vdolbina. Podobno obliko ima 
tudi absorpcijski koeficient.}
\label{fig:Bennet}
\end{figure}

Zapišimo še absorpcijski koeficient za šibko vpadno valovanje pri 
frekvenci $\omega^{\prime}$. Upoštevati moramo, da k absorpciji prispevajo vsi atomi, katerih
hitrost je taka, da je prehod dovolj blizu $\omega^{\prime}$. Absorpcijski 
koeficient\index{Absorpcijski koeficient} potem izračunamo s seštevanjem 
po porazdelitvi ${\cal Z}(v)$
(enačba~\ref{eq:muabs1})
\begin{equation}
\mu(\omega^{\prime})=\frac{\hbar\omega^{\prime}}{c}\int{\cal Z}(v)Bg'
(\omega^{\prime}-\omega_{0}+k'v)\, dv.
\label{4.90}
\end{equation}
V splošnem se funkcija $g'$ razlikuje od funkcije $g$, ki nastopa v izrazu za
${\cal Z}$, saj sta njuni širini lahko različni. Vsekakor pa velja, da je 
homogena razširitev dosti manjša od Dopplerjeve širine.
V prvem približku vzemimo, da lahko Lorentzovo $g'(\omega)$ nadomestimo kar z
$\delta(\omega)$. Tako je absorpcijski koeficient za šibko testno svetlobo 
\begin{eqnarray}
\mu(\omega^{\prime}) & = & \frac{\hbar\omega^{\prime}}{k'c}B\frac{{\cal N}
(\frac{\omega_0-\omega'}{k'})}{1+\frac{2Bj}{Ac}g(\omega_S-\omega')} \\ \nonumber 
 & \approx & \hbar B{\cal N}\left(\frac{\omega_0-\omega'}{k'}\right)\left(1-\frac{2Bj}{Ac}g(\omega_S-\omega')\right).
\end{eqnarray}
 V drugi vrstici smo uporabili približek (enačba~\ref{4.89}). Odvisnost $\mu(\omega^{\prime})$,
ki jo izmerimo tako, da spreminjamo frekvenco testnega snopa $\omega^{\prime}$,
je Gaussove oblike z vdolbino pri $\omega_S$ in je podobna porazdelitvi, 
kot jo kaže slika (\ref{fig:Bennet}). 

\begin{remark}
 Merjenje nasičenja absorpcije s testnim
žarkom torej omogoča opazovanje oblike homogene črte kljub mnogo večji
nehomogeni Dopplerjevi razširitvi in je zato v moderni spektroskopiji
velikega pomena.
\end{remark}

Izračunajmo še absorpcijski koeficient za prvi, močan vpadni snop, tako da v
gornjem izrazu vstavimo $\omega^{\prime}=\omega_S$. Vodilni člen ${\cal N}((\omega_0-
\omega_S)/k)$ opisuje običajno Gaussovo obliko Dopplerjevo
razširjene črte, izraz v oklepaju pa da zmanjšanje absorpcije
zaradi nasičenja, ki je odvisno le od vrednosti $g(0)$ in zato enako za vse $\omega_S$. 
Z enim samim vpadnim snopom svetlobe torej vdolbine v absorpciji ne moremo zaznati, saj 
je izmerjena črta kljub nasičenju Gaussove oblike. 

Namesto z dvema različnima snopoma, od katerih lahko šibkemu testnemu snopu spreminjamo
frekvenco, lahko vdolbino v porazdelitvi zaznamo tudi z enim samim snopom
spremenljive frekvence, ki se po prvem prehodu skozi plin odbije od
ogledala in vrne v nasprotni smeri. S tem se v porazdelitvi atomov
v spodnjem stanju simetrično pri hitrostih $\pm(\omega_{0}-\omega_S)/k$
pojavita dve Bennettovi vdolbini (slika \ref{fig:Lamb}\,a).
Kadar je $\omega_S$ blizu $\omega_{0}$, se vdolbini vsaj delno prekrivata, 
stopnja nasičenja se poveča in v krivulji za absorpcijo svetlobe se pojavi 
vdolbina (slika \ref{fig:Lamb}\,b).
Imenujemo jo Lambova vdolbina\footnote{Ameriški fizik in nobelovec 
Willis Eugene Lamb Jr., 1913--2008.}\index{Lambova vdolbina}. 
\begin{figure}[h]
\centering
\def\svgwidth{140truemm} 
\input{slike/05_Lamb.pdf_tex}
\caption{Porazdelitev atomov po hitrosti v osnovnem stanju, če svetloba prehaja 
skozi plin v dveh smereh (a). Če frekvenca vpadne svetlobe približno sovpada s centralno 
frekvenco prehoda, se obe vdolbini prekrivata in absorpcija se zmanjša (b).}
\label{fig:Lamb}
\end{figure}

Zapišimo še enačbe za ta primer. Snop povzroči spremembo zasedenosti
pri prehodu skozi plin v obeh smereh, zato je sedaj 
\begin{equation}
{\cal Z}(v)\approx{\cal N}(v)\left(1-\frac{2Bj}{Ac}\left(g(\omega_S-\omega_{0}+kv)+
g(\omega_S-\omega_{0}-kv)\right)\right).
\label{4.92}
\end{equation}
Podobno kot prej izračunamo absorpcijski koeficient za širjenje svetlobe v
pozitivni smeri 
\begin{eqnarray}
\mu_{+}(\omega_S) & = & \frac{\hbar\omega}{c}B\int{\cal Z}(v)g(\omega_S-\omega_{0}+kv)\, dv\nonumber \\
 & \approx & \hbar B{\cal N}\left(\frac{\omega_S-\omega_0}{k}\right)\left(1-\frac{2Bj}
 {Ac}\left(g(0)+g(2(\omega_S-\omega_{0}))\right)\right)\;.
 \label{eq:lamb}
\end{eqnarray}
Izmerjeni absorpcijski profil je odvisen od frekvence vpadne svetlobe in ima na sredini vdolbino,
ki je zopet podobna homogeno razširjeni črti. Faktor 2 v argumentu
funkcije $g(2(\omega_S-\omega_{0}))$ je posledica našega grobega približka,
ko smo v integraciji $g(\omega_S-\omega_{0}+kv)$ nadomestili kar z
$\delta$ funkcijo. Natančnejši račun pokaže, da je vrh pri $\omega_{0}$
kar oblike $g(\omega_S-\omega_{0})$.

\begin{definition}
Pokaži, da je rezultat natančnejše izpeljave absorpcijskega koeficienta  
\begin{equation}
 \mu_{+}(\omega_S) = \hbar B{\cal N}\left(\frac{\omega_S-\omega_0}{k}\right)\left(1-\frac{Bj}
 {Ac}\left(g(0)+g(\omega_S-\omega_{0})\right)\right)\;.
\end{equation}
Pri računu privzemi, da je Dopplerjeva razširitev bistveno večja in Maxwellovo porazdelitev
prestavi pred integral. 
\end{definition}

\section{*Izpeljava verjetnosti za prehod}
\label{chap:verjetnost}
\index{Verjetnost za prehod}
Verjetnosti za prehod atoma iz enega stanja v drugo s sevanjem, ki
smo jih opisali s fenomenološkimi Einsteinovimi koeficienti $A_{21}$
in $B_{21}$ (razdelek~\ref{AB}), \index{Einsteinovi koeficienti}
je mogoče izpeljati tudi drugače.
Pri tem se poslužimo kvantne\index{Kvantizacija polja} elektrodinamike, 
kar pomeni kvantno obravnavo 
tako atoma kot elektromagnetnega polja. Povsem strog račun je zahteven in presega
okvir te knjige, zato si na kratko oglejmo le, kako pridemo do rezultata s
perturbacijsko metodo.

Postavimo dvonivojski atom v votlino z elektromagnetnim poljem.\index{Dvonivojski sistem}
Izračunajmo verjetnost, da zaradi interakcije s poljem atom
preide iz stanja $|2\rangle$ v stanje $|1\rangle$, pri čemer se
število fotonov v izbranem stanju elektromagnetnega polja $\alpha$
poveča z $n_{\alpha}$ na $n_{\alpha}+1$. V vseh ostalih stanjih
polja naj bo število fotonov enako nič.

Med atomom in poljem privzemimo električno dipolno interakcijo 
\begin{equation}
\hat{H}_{i}=-e\hat{E}(\mathbf{r},t)\hat{x},
\label{4.47}
\end{equation}
kjer je $\hat{x}$ operator koordinate elektrona v atomu. 
Privzeli
smo, da je nihajoče polje polarizirano v smeri osi $x$. Stanja celotnega sistema, 
to je atoma in polja, zapišemo v obliki produkta atomskih stanj in
stanja elektromagnetnega polja, pri čemer moramo navesti število fotonov
v vsakem nihanju votline $\alpha$. Zapišemo okrajšano
\begin{equation}
|i,n_{\alpha}\rangle\equiv|i\rangle|\{n_{\alpha}\}\rangle.
\label{4.48}
\end{equation}
Začetno stanje celotnega sistem naj bo tako $|2,n_{\alpha}\rangle$, kar pomeni, da je
atom v gornjem stanju, polje pa ima $n_{\alpha}$ fotonov v enem samem stanju $\alpha$.
Ustrezno končno stanje po prehodu je $|1,n_{\alpha}+1\rangle$.

V prvem redu teorije motenj je verjetnost za prehod iz začetnega v končno stanje
na časovno enoto enaka
\begin{equation}
w_{21}=\frac{2\pi}{\hbar}|\langle1,n_{\alpha}+
1|\,\hat{H}_{i}\,|2,n_{\alpha}\rangle|^{2}\,
\delta(E_{2}-E_{1}-\hbar\omega_{\alpha}).
\label{4.49}
\end{equation}
Z delta funkcijo izberemo le prehod, pri katerem se ohranja
energija.

Operator elektromagnetnega polja lahko po enačbi~(\ref{eq:pqrazvoj}) 
razvijemo po lastnih nihanjih votline 
\begin{equation}
\hat{E}(\mathbf{r},t)=-\frac{1}{\sqrt{V\epsilon_{0}}}\sum_{\alpha}
\hat{p}_{\alpha}(t)E_{\alpha}(\mathbf{r}),
\label{4.50}
\end{equation}
kjer je $\hat{p}_{\alpha}$ operator gibalne količine stanja $\alpha$, $E_{\alpha}$
pa funkcija, ki popisuje krajevno odvisnost polja. Vemo, da se vsako 
elektromagnetno nihanje votline obnaša kot harmonski oscilator.
Zato lahko vpeljemo kreacijske in anihilacijske operatorje
\begin{eqnarray}
\hat{a}_{\alpha}^{\dagger} & = & \frac{1}{\sqrt{2\hbar\omega_{\alpha}}}\,
(\omega_{\alpha}\hat{q}_{\alpha}-i\hat{p}_{\alpha}) \\
\hat{a}_{\alpha} & = & \frac{1}{\sqrt{2\hbar\omega_{\alpha}}}\,(\omega_{\alpha}\hat{q}_{\alpha}+i\hat{p}_{\alpha}).
\end{eqnarray}
 Kreacijski operatorji povečujejo, anihilacijski pa zmanjšujejo število
fotonov v danem stanju
\begin{eqnarray}
\hat{a}_{\alpha}^{\dagger}|n_{\alpha}\rangle & = & \sqrt{n_{\alpha}+1}
|n_{\alpha}+1\rangle\qquad \mathrm{in} \\
\hat{a}_{\alpha}|n_{\alpha}\rangle & = & \sqrt{n_{\alpha}}|n_{\alpha}-1\rangle.
\end{eqnarray}
Edini od nič različni matrični elementi so tako oblike
\begin{eqnarray}
\langle n_\alpha +1|\, \hat{a}_{\alpha}^{\dagger}\,|n_{\alpha}\rangle & = 
& \sqrt{n_{\alpha}+1} \qquad \mathrm{in} \nonumber\\
\langle n_\alpha-1|\,\hat{a}_{\alpha}\,|n_{\alpha}\rangle & = & \sqrt{n_{\alpha}}.
\label{eq:ankr}
\end{eqnarray}
Operatorje $\hat{p}_{\alpha}$ zdaj lahko izrazimo s kreacijskimi in anihilacijskimi
operatorji in jih vstavimo v razvoj električnega polja (enačba~\ref{4.50}). Dobimo
\begin{equation}
\hat{E}(\mathbf{r},t)=-i\sum_{\alpha}\sqrt{\frac{\hbar\omega_{\alpha}}{2V\epsilon_{0}}}\,
\left(\hat{a}_{\alpha}^{\dagger}-\hat{a}_{\alpha}\right)E_{\alpha}(\mathbf{r}).
\label{4.53}
\end{equation}
Nadaljujemo z izračunom potrebnega matričnega elementa. Operator koordinate
$\hat{x}$ deluje le na atomski del stanja, $\hat{E}$ pa le na elektromagnetno
polje, zato velja 
\begin{eqnarray}
\langle1,n_{\alpha}+1|\,\hat{H}_{i}\,|2,n_{\alpha}\rangle & = & -e\,
\langle1,n_{\alpha}+1|\,\hat{E}\,\hat{x}\,|2,n_{\alpha}\rangle \\
 & = & -e\,\langle1|\,\hat{x}\,|2\rangle\langle n_{\alpha}+1|\,\hat{E}\,|n_{\alpha}\rangle.
\end{eqnarray}
Vstavimo polje, ki smo ga izrazili s kreacijskimi in anihilacijskimi operatorji (enačba~\ref{4.53}),
upoštevamo zvezi~(\ref{eq:ankr}) in zapišemo
\begin{eqnarray}
\langle n_{\alpha}+1|\, \hat{E}\,|n_{\alpha}\rangle & = 
& -i\sum_{\beta}\sqrt{\frac{\hbar\omega_{\beta}}{2V\epsilon_{0}}}
\langle n_{\alpha}+1|\,\hat{a}_{\beta}^{\dagger}-\hat{a}_{\beta}\,|n_{\alpha}\rangle\, 
E_{\beta}(\mathbf{r})\nonumber \\
 & = & -i\sqrt{\frac{\hbar\omega_{\alpha}}{2V\epsilon_{0}}}
 \sqrt{n_{\alpha}+1}\, E_{\alpha}(\mathbf{r}).
\end{eqnarray}
Od vseh operatorjev v razvoju polja je namreč od nič različen matrični
element le za kreacijski operator za stanje $\alpha$.
Vpeljimo še simbol za matrični element koordinate med 
atomskimi stanji 
$\langle1|\hat{x}|2\rangle=x_{12}$.
 Iskana verjetnost za prehod iz 
začetnega stanja, v katerem smo imeli vzbujen atom in $n_{\alpha}$ fotonov, v končno
stanje z atomom v osnovnem stanju in $n_{\alpha}+1$ fotonov v stanju $\alpha$ je tako
\begin{equation}
w_{21}=\frac{\pi e^{2}\omega_{\alpha}x_{12}^{2}}{V\epsilon_{0}}
(n_{\alpha}+1)\,E_{\alpha}^{2}(\mathbf{r})\,\delta(E_{2}-E_{1}-\hbar\omega_{\alpha}).
\label{4.56}
\end{equation}
Verjetnost za prehod je sorazmerna z $n_{\alpha}+1$ in je od nič
različna, tudi če je število kvantov polja enako nič. To opisuje seveda
spontano sevanje\index{Spontano sevanje}. Prispevek, ki je 
sorazmeren s številom že prisotnih fotonov, pa predstavlja stimulirano 
sevanje\index{Stimulirano sevanje}. Verjetnost za prehod vsebuje
še kvadrat prostorske odvisnosti polja $E_{\alpha}^{2}(\mathbf{r})$.
Če ne poznamo natančnega položaja atoma ali če je plin atomov enakomerno
porazdeljen po votlini, lahko ta člen nadomestimo kar s povprečno vrednostjo.
Za stoječe valovanje je to kar 1/2.

Najprej poglejmo verjetnost za spontano emisijo. 
Spontana emisija je možna v vsa elektromagnetna nihanja votline s
pravo frekvenco. Celotno verjetnost za prehod atoma iz vzbujenega
v osnovno stanje izračunamo tako, da seštejemo verjetnosti za prehod z izsevanim fotonom 
v določenem stanju. Spomnimo se, da je ta verjetnost ravno enaka 
Einsteinovem koeficientu $A_{21}$\index{Einsteinovi koeficienti} (enačba~\ref{4.27})
\begin{equation}
A_{21}=\sum_{\alpha}w_{21}=\sum_{\alpha}\frac{\pi 
e^{2}\omega_{\alpha}x_{12}^{2}}{2V\epsilon_{0}}\,\delta(E_{2}-E_{1}-\hbar\omega_{\alpha}).
\label{4.57}
\end{equation}
Za prostorsko odvisnost polja $E^{2}(\mathbf{r})$ smo vzeli povprečje
1/2. Vsoto po nihanjih lahko z uporabo enačbe~(\ref{4.5}) spremenimo v integral
in upoštevamo enačbo~(\ref{4.4}). Dobimo
\begin{equation}
A_{21}=\frac{\pi e^{2}x_{12}^{2}}{2\hbar\epsilon_{0}}\int\rho(\omega_{\alpha})\omega_\alpha\, 
\delta(\omega_{0}-\omega_{\alpha})\, d\omega_{\alpha}=\frac{e^{2}\omega_{0}^{3}x_{12}^{2}}{2\pi\epsilon_{0}\hbar c^{3}}.
\label{4.58}
\end{equation}
 Z $\omega_{0}=(E_{2}-E_{1})/\hbar$ smo označili frekvenco prehoda. S tem smo 
 izpeljali vrednost Einsteinovega koeficienta $A_{21}$. 
\begin{remark}
Pri gornjem izračunu Einsteinovega koeficienta $A_{21}$ smo privzeli, da so vsi dipoli urejeni  
 v smeri širjenja svetlobe. Če želimo rezultat izenačiti s koeficientom, ki smo ga vpeljali
 za izotropno sevanje črnega telesa, ga moramo pomnožiti s faktorjem $\langle \cos^2\vartheta
 \rangle = 1/3$.
\end{remark}

Zaradi spontanega sevanja vzbujeno atomsko stanje nikoli ni popolnoma
stacionarno. Poleg tega energija stanja s končnim razpadnim časom ni natančno
določena, zato moramo verjetnost za stimulirano sevanje (enačba~\ref{4.56}) malo 
popraviti. Delta funkcijo energije nadomestimo s končno široko  
funkcijo $g(\omega-\omega_0)$, ki ima vrh pri $\omega_{0}$. Zaradi 
spremembe integracijske spremenljivke nastopi še dodaten faktor $1/\hbar$ in zapišemo
\begin{equation}
w_{21}=\frac{\pi e^{2}\omega_{\alpha}x_{12}^{2}}{2V\epsilon_{0}\hbar}
(n_{\alpha}+1)g(\omega_{\alpha}-\omega_0).
\label{4.59}
\end{equation}

Poglejmo še Einsteinov koeficient za stimulirano sevanje $B_{21}$. Lahko ga 
izrazimo iz enačbe~(\ref{4.18}), če upoštevamo, da je gostota energije 
polja $n_{\alpha}\hbar\omega_{\alpha}/V$
\begin{equation}
B_{21}=\frac{V\,w_{21}}{n_{\alpha}\,\hbar\omega_{\alpha}\, g(\omega_{\alpha}-\omega_0)}
=\frac{\pi e^{2}x_{12}^{2}}{2\epsilon_{0}\hbar^{2}}.
\label{4.60}
\end{equation}
Poglejmo  še razmerje izračunanih Einsteinovih koeficientov iz enačb~(\ref{4.58}) in 
(\ref{4.60})
\begin{equation}
 \frac{A_{21}}{B_{21}}=\frac{\hbar \omega_\alpha^3}{\pi^2 c^3},
\end{equation}
ki se ujema z razmerjem, ki smo ga izpeljali z uporabo
Planckove formule (enačba~\ref{4.27}). Prehojena pot jasno kaže zvezo med spontanim in
stimuliranim sevanjem ter gostoto stanj elektromagnetnega polja. 

\section{*Rabijeve oscilacije}
Ko močna svetloba vpada na dvonivojski sistem, lahko v primeru, da je frekvenca 
svetlobe $\omega$ blizu frekvence prehoda $\omega_0$, pride do periodične 
izmenjave energije med 
svetlobnim poljem in dvonivojskim sistemom.\index{Dvonivojski sistem} 
Oscilacije števila fotonov oziroma pričakovane 
vrednosti zasedenosti nivojev\index{Rabijeve oscilacije} imenujemo Rabijeve 
oscilacije\footnote{Ameriški fizik in nobelovec Isidor Isaac Rabi, 1898--1988.}. 

Obravnavajmo sklopitev dvonivojskega sistema z elektromagnetnim valovanjem 
v semiklasičnem modelu. To pomeni, da dvonivojski sistem obravnavamo kvantno, 
svetlobo, ki vpada, pa kot klasično skalarno polje. 
V odsotnosti električnega polja zapišemo Hamiltonian\index{Hamiltonova
funkcija} za elektron kot
\begin{equation}
H_0 = \hbar \omega_1 |1\rangle \langle1| + \hbar \omega_2 |2\rangle \langle2|,
\end{equation}
pri čemer je $\omega_2- \omega_1 = \omega_0$ frekvenca prehoda. V prisotnosti 
svetlobnega polja moramo dodati še člen, ki opisuje dipolno interakcijo in dobimo
časovno odvisen Hamiltonian
\begin{equation}
H = \hbar \omega_1 |1\rangle \langle1| + \hbar \omega_2 |2\rangle \langle2|
-e\hat{x}E_0 \cos (\omega t).
\end{equation}
Schr\"odingerjevo enačbo\index{Schr\"odingerjeva enačba}
\begin{equation}
i \hbar \frac{\partial}{\partial t}|\psi\rangle = H|\psi\rangle
\end{equation}
rešujemo z nastavkom 
\begin{equation}
|\psi\rangle = c_1(t)e^{-i \omega_1t}|1\rangle + c_2(t)e^{-i \omega_2t}|2\rangle,
\end{equation}
saj je valovna funkcija, ki popisuje stanje sistema, v splošnem 
kombinacija obeh stanj. Nastavek vstavimo v enačbo, ki jo enkrat pomnožimo 
z $\langle1|$, drugič pa z $\langle2|$ in izpeljemo sistem dveh sklopljenih enačb
\begin{equation}
\frac{d c_1}{dt}=-\frac{i}{\hbar} V \cos (\omega t) e^{-i\omega_0 t}\, c_2 
\qquad \mathrm{in} \qquad
\frac{d c_2}{dt}=-\frac{i}{\hbar} V \cos (\omega t) e^{i\omega_0 t}\, c_1,
\end{equation}
pri čemer je $V = -\langle1|e\hat{x}E_0|2\rangle$. Zapišemo še $\cos(\omega t)$ kot
kompleksno število in zanemarimo hitro spreminjajočo se komponento (pri $\omega_0 + \omega)$,
tako da enačbi prepišemo v 
\begin{equation}
\frac{d c_1}{dt}=-\frac{i}{2\hbar} V e^{-i\Delta t}\, c_2 
\qquad \mathrm{in} \qquad
\frac{d c_2}{dt}=-\frac{i}{2\hbar} V e^{i\Delta t}\, c_1,
\label{eq:rabi2}
\end{equation}
kjer je $\Delta = \omega_0-\omega$. Dodajmo še začetni pogoj, da je 
sistem v osnovnem stanju in torej $c_1(0)=1$ in $c_2(0)=0$ in rešitvi enačb
(\ref{eq:rabi2}) sta 
\begin{eqnarray}
c_1(t)&=&e^{-i\Delta t/2}\, \left(\cos\left(\frac{\Omega t}{2}\right) + 
i \frac{\Delta}{\Omega} \sin\left(\frac{\Omega t}{2}\right) \right)\qquad \mathrm{in} 
\label{eq:rabi3} \\
c_2(t)&=&\frac{V}{i\hbar \Omega} e^{i\Delta t/2}\, \sin\left(\frac{\Omega t}{2}\right).
\label{eq:rabi4}
\end{eqnarray}
Pri tem smo vpeljali tako imenovano Rabijevo frekvenco
\begin{equation}
\Omega = \sqrt{\Delta^2 + \left(\frac{V}{\hbar}\right)^2} = \sqrt{(\omega_0-\omega)^2 
+ \left(\frac{\langle1|\hat{x}|2\rangle\, eE_0}{\hbar}\right)^2}
\end{equation}
\begin{definition}
Pokaži, da sta gornja izraza (enačbi~\ref{eq:rabi3} in \ref{eq:rabi4}) res rešita
sistem enačb (\ref{eq:rabi2}) ob izbranih začetnih pogojih.
\end{definition}
Poglejmo rezultat podrobneje. Verjetnost, da najdemo atom v stanju $|2\rangle$, je enaka
\begin{equation}
P_2(t) = |c_2(t)|^2 = \frac{V^2}{\hbar^2 \Omega^2}\sin^2(\Omega t/2).
\end{equation}
Če je frekvenca vpadne svetlobe točno enaka frekvenci prehoda, je $\Delta = 0$ in 
$\Omega = V/\hbar$. Takrat je amplituda nihanja zasedenosti vzbujenega stanja kar enaka 1
in sistem v celoti periodično prehaja iz osnovnega stanja v vzbujeno in nazaj. To pomeni,
da pride izmenično do popolne absorpcije svetlobe in do popolne stimulirane emisije.
Pri odstopajoči vpadni frekvenci se amplituda nihanja zmanjša, hkrati pa se poveča
frekvenca oscilacij. Frekvenca oscilacij pa ni odvisna zgolj od frekvence vpadnega valovanja, 
ampak tudi od amplitude električne poljske jakosti vpadnega valovanja. Zelo groba ocena
frekvence je zato $\Omega \sim~\si{MHz}$.
\begin{figure}[h]
\centering
\def\svgwidth{100truemm} 
\input{slike/05_rabi.pdf_tex}
\caption{Rabijeve oscilacije za tri različne vrednosti odstopanja frekvence vpadne
svetlobe od frekvence prehoda $\Delta=\omega_0-\omega$. 
Z naraščajočim odstopanjem se amplituda oscilacij
zmanjšuje, njihova frekvenca pa povečuje.}
\label{fig:Rabi}
\end{figure}
\begin{remark}
Rabijeve oscilacije niso omejene samo na optične prehode, ampak se pojavijo pri 
vrsti dvonivojskih sistemov, ki interagirajo z spreminjajočim se zunanjim poljem, na primer
pri jedrski magnetni resonanci (NMR) ali kvantnih logičnih vezjih oziroma kvantnih 
računalnikih.
\end{remark}
