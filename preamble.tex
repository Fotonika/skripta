\usepackage[slovene]{babel}
\usepackage[utf8]{inputenc}
\usepackage[T1]{fontenc}
\usepackage{lmodern}

\usepackage{graphicx}
\usepackage{framed}

% FORMAT STRANI ===================
\usepackage[driver=none,papersize={165mm,235mm},totalheight=190mm,textwidth=128mm,centering,marginpar=15mm]{geometry}
\usepackage{crop}

% DEFINICIJA BARVE ===================
\usepackage{xcolor}
\definecolor{MojaModra}{RGB}{0,56,102} 

% DEFINICIJA FONTA ===================
%\usepackage{avant} % Use the Avantgarde font for headings
\usepackage{mathptmx} % Use the Adobe Times Roman as the default text font together with math symbols from the Sym­bol, Chancery and Com­puter Modern fonts

\usepackage{amssymb}
\usepackage{microtype} % Slightly tweak font spacing for aesthetics
\usepackage[unicode]{hyperref}
\usepackage{empheq}
\usepackage{color}
\usepackage{caption}
\usepackage{afterpage}

% OKVIRČEK ZA ENAČBE ===================
\newcommand{\boxeq}[2]{\begin{empheq}[box=\colorbox{MojaModra!25}]{align}\label{#1}#2\end{empheq}}

% HYPERREF ===================
\usepackage{hyperref}
\hypersetup{hidelinks,backref=true,pagebackref=true,hyperindex=true,colorlinks=true,breaklinks=true,urlcolor= MojaModra,linkcolor=MojaModra,bookmarks=true,bookmarksopen=false,pdftitle={Title},pdfauthor={Author}}

% MATHS ===================
\usepackage{amsmath,tikz}
\usetikzlibrary{matrix}
\newcommand*{\horzbar}{\rule[0.05ex]{2.5ex}{0.5pt}}
\usepackage{calc}
\usepackage{siunitx}

% ZA SLIKE (wrap & sub) ===================
\usepackage{wrapfig}
\usepackage{subfig}
\usepackage{parskip}

% RAZNO ===================
\newcommand{\beq}{\begin{equation}}
\newcommand{\eeq}{\end{equation}}
\renewcommand{\arraystretch}{1.5}

\newcommand{\usk}{\mbox()}

\makeatletter
\setlength{\@fptop}{0pt}
\makeatother

% POT ZA SLIKE ===================
\graphicspath{{./slike/}}

% HEADER ===================
\usepackage{fancyhdr} 
\fancyfoot{}
\fancyhead{}
\fancyhead[LE]{\thepage}
\fancyhead[RO]{\thepage}
\fancyhead[CE]{\itshape\nouppercase{\leftmark}}
\fancyhead[CO]{\itshape\nouppercase{\leftmark}} %dala left na obe strani.
\pagestyle{fancy}
\fancyhfoffset[L]{0cm}
\renewcommand{\chaptermark}[1]{\markboth{\thechapter~~#1}{\thechapter~~#1}}
\renewcommand{\sectionmark}[1]{\markright{\thesection~~#1}}

% ŠTEVILČENJE NASLOVOV ===================
\renewcommand{\thepart}{\Roman{part}.} 
\renewcommand{\thechapter}{\arabic{chapter}} %odstranila sem pike, sicer imajo enačbe po dve piki, npr. 4..2
\renewcommand{\thesection}{\arabic{chapter}.\arabic{section}} %odstranila sem pike
\renewcommand{\thesubsection}{}

% NASLOVI ===================
\usepackage{titlesec}
\titleformat{\part}[hang]{\normalfont\Huge\bfseries}{\thepart}{15pt}{\thispagestyle{empty}\filcenter}{}
\titleformat{\chapter}[hang]{\normalfont\huge\bfseries}{\vspace*{7mm}\thechapter}{12pt}{}{}
\titleformat{\section}[hang]{\normalfont\Large\bfseries}{\thesection}{10pt}{}{}
\titleformat{\subsection}[hang]{\normalfont\large\bfseries}{}{0pt}{}{}

% OPISI SLIK IN TABEL ===================
\usepackage[font=small,format=plain,labelfont=sc,justification=centerlast]{caption} %odstranila italic
%,figurewithin=none,tablewithin=none

% OPOMBE POD ČRTO ===================
\usepackage[bottom]{footmisc}
\setlength{\footnotesep}{4mm}
\setlength{\skip\footins}{5mm}
\renewcommand{\footnoterule}{\kern -2.4pt\hrule width 2cm height 0.4pt\kern 2pt}

% STVARNO KAZALO ===================
\usepackage{makeidx}
\makeindex 

% RAZNO ===================
\newcommand{\procent}[1]{\SI{#1}{\%}}
\newcommand{\narekovaji}[1]{\guillemotright#1\guillemotleft}

\renewcommand{\tan}{\mathop{\mathrm{tg}}\nolimits}
\renewcommand{\arctan}{\mathop{\mathrm{arctg}}\nolimits}

\newcommand{\UL}{\textsc{Univerza v Ljubljani}}
\newcommand{\FMF}{\textsc{Fakulteta za matematiko in fiziko}}
\newcommand{\LJ}[1]{\textsc{Ljubljana #1}}
\newcommand{\avtor}[1]{{\centering\Large\textsc{#1}\par}}
\newcommand{\naslov}[1]{{\centering\Huge\textbf{#1}\par}}
\newcommand{\logotip}[1][1]{{\centering\includegraphics[scale=#1]{fmf.pdf}\par}}
\newcommand{\zalozba}[1]{{\centering\UL\\\FMF\\[1mm]\LJ{#1}\par}}
\newcommand{\sodastran}{\clearpage\ifodd\value{page}~\thispagestyle{empty}\clearpage\fi}
\newcommand{\lihastran}{\clearpage\ifodd\value{page}\else~\thispagestyle{empty}\clearpage\fi}

% OKVIRČEK ZA CIP ===================
\newenvironment{CIP}{
   \parskip=10pt
   \parindent=0pt
   \setlength{\FrameRule}{0.6pt}
   \setlength{\FrameSep}{10pt}
   \OuterFrameSep=0pt
   \begin{framed}\raggedright\small
   CIP -- Kataložni zapis o publikaciji \\
   Narodna in univerzitetna knjižnica, Ljubljana\par}
   {\end{framed}}
   

% KAZALO ===================
\setcounter{tocdepth}{2}
\usepackage{titletoc} % Required for manipulating the table of contents
%\contentsmargin{0cm} % Removes the default margin
% Chapter text styling
\titlecontents{chapter}[1.0cm] % Indentation
{\addvspace{12pt}\normalsize\bfseries} % Spacing and font options for chapters
{\color{MojaModra}\contentslabel[\bfseries\thecontentslabel]{1.00cm}\color{MojaModra}} % Chapter number
{}  
{\color{MojaModra}\bfseries\;\titlerule*[.5pc]{.}\;\thecontentspage} % Page number
% Section text styling
\titlecontents{section}[1.75cm] % Indentation
{\addvspace{2pt}\color{black}} % Spacing and font options for sections
{\contentslabel[\thecontentslabel]{1.00cm}} % Section number
{}
{\hfill\color{black}\thecontentspage} % Page number
[]
% Subsection text styling
\titlecontents{subsection}[2.25cm] % Indentation
{\addvspace{1pt}} % Spacing and font options for subsections
{\contentslabel[\thecontentslabel]{1.00cm}} % Subsection number
{}
{\hfill\color{black}\thecontentspage} % Page number
[] 


% OKOLJE: ZANIMIVOSTI Z ZVEZDICO (environment remark) ===================
\newenvironment{remark}{\par\vspace{10pt}\small % Vertical white space above the remark and smaller font size
\begin{list}{}{
\leftmargin=25pt % Indentation on the left
\rightmargin=0pt}\item\ignorespaces % Indentation on the right
\makebox[-2.5pt]{\begin{tikzpicture}[overlay]
\node[draw=MojaModra!60,line width=1pt,circle,fill=MojaModra!25,inner sep=2pt,outer sep=0pt] at (-15pt,0pt){\textcolor{MojaModra}{$\bigstar$}};\end{tikzpicture}} % Star in a circle
\advance\baselineskip -1pt}{\end{list}\vskip5pt} % Tighter line spacing and white space after remark


% OKOLJE: NALOGA ===================
\usepackage{amsmath,amsfonts,amssymb,amsthm}
\makeatletter
\newtheoremstyle{nalogabox} % Theorem style name
{0pt}% Space above
{0pt}% Space below
{\normalfont}% Body font
{}% Indent amount
{\bf}% Theorem head font
{\;}% Punctuation after theorem head
{0.25em}% Space after theorem head
{\normalsize\bfseries\thmname{#1}\nobreakspace\thmnumber{\@ifnotempty{#1}{}\@upn{#2}}% Theorem text (e.g. Theorem 2.1)
\thmnote{\nobreakspace\the\thm@notefont\sffamily\bfseries---\nobreakspace#3.}}% Optional theorem note
\makeatother

\newcounter{dummy} 
\numberwithin{dummy}{section}
\theoremstyle{nalogabox}
\newtheorem{nalogaT}{Naloga}[section]

\RequirePackage[framemethod=default]{mdframed}

% OKOLJE: NALOGA, OKVIR ===================
\newmdenv[
skipabove=7pt,
skipbelow=7pt,
rightline=false,
leftline=false,
topline=true,
bottomline=true,
linecolor=MojaModra,
innerleftmargin=0pt,
innerrightmargin=0pt,
innertopmargin=7pt,
leftmargin=0cm,
rightmargin=0cm,
linewidth=1pt,
innerbottommargin=7pt]{dBox}

\newenvironment{naloga}{\begin{dBox}\begin{nalogaT}}{\end{nalogaT}\end{dBox}}	
