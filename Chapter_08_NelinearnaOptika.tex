\chapterimage{slike/NLO.jpg} 
% Chapter heading image

\chapter{Nelinearna optika}
\label{chap:NLO}
Pri obravnavi svetlobnega valovanja v snovi smo doslej vedno privzeli linearno 
zvezo med polarizacijo in jakostjo električnega polja. To 
je seveda približek, ki je dovolj dober le pri razmeroma majhnih jakostih
polja. Kadar doseže električna poljska jakost velike vrednosti -- in v laserskih snopih
jih nedvomno lahko doseže -- je treba upoštevati tudi višje člene v razvoju. Takrat
govorimo o nelinearni optiki\index{Nelinearna optika}, saj zveza med polarizacijo
in električnim poljem ni linearna. V tem poglavju bomo spoznali zanimive pojave, ki jih 
povzroči nelinearni del polarizacije, med drugim optično 
frekvenčno podvajanje, optično parametrično ojačevanje, optično usmerjanje, 
samozbiranje laserskega snopa, optične solitone in optično fazno konjugacijo. 

\section{Nelinearna susceptibilnost}
\label{Chap:Chi}
V linearnem približku odziva snovi velja, da je polarizacija snovi\index{Električna polarizacija} 
$\mathbf{P}$ linearna funkcija električne poljske jakosti svetlobe 
$\mathbf{E}$\index{Električno polje!jakost}. Takrat zapišemo (enačba~\ref{eq:PM})
\begin{equation}
\mathbf{P} = \mathbf{D} - \varepsilon_0 \mathbf{E} = 
\varepsilon_0 \underline{\epsilon} \cdot\mathbf{E} - \varepsilon_0 \mathbf{E} = 
\varepsilon_0 (\underline{\epsilon} - 1)\cdot\mathbf{E}. 
\end{equation}
Če uvedemo tenzor linearne susceptibilnosti\index{Susceptibilnost!linearna}
\begin{equation}
\chi^{(1)} = \underline{\epsilon} - 1,
\end{equation}
lahko linearni odziv snovi zapišemo strnjeno kot
\begin{equation}
\mathbf{P}_{\mathrm{L}} =  \varepsilon_0 \chi^{(1)} \cdot \mathbf{E}.
\end{equation}
Ta približek je dober za majhne jakosti električnega polja. Pri večjih poljih
postanejo pomembni tudi členi višjega reda v razvoju polarizacije
po $\mathbf{E}$
\boxeq{8.1}{
\mathbf{P}=\mathbf{P}_{\mathrm{L}}+ \mathbf{P}_{\mathrm{NL}}=
\epsilon_{0} \chi^{(1)}\cdot \mathbf{E}+
\epsilon_{0}\chi^{(2)}:\mathbf{E}\, \mathbf{E}+
\epsilon_{0}\chi^{(3)}\vdots \mathbin \mathbf{E}\mathbin \mathbf{E}\mathbin\mathbf{E} + \dots
}
Vpeljali smo nelinearni susceptibilnosti\index{Susceptibilnost!nelinearna} 
$\chi^{(2)}$ in $\chi^{(3)}$, ki sta tenzorja tretjega in četrtega ranga. 
Za bolj nazorno predstavo izpišimo nelinearna dela še po komponentah
\begin{equation}
\left(\mathbf{P}_{\mathrm{NL,2}}\right)_i= \epsilon_{0}\chi^{(2)}_{ijk} \,E_j \,E_k
\label{eq:nlin2}
\end{equation}
in 
\begin{equation}
\left(\mathbf{P}_{\mathrm{NL,3}}\right)_i= \epsilon_{0}\chi^{(3)}_{ijkl} \,E_j \,E_k\, E_l,
\label{eq:nlin3}
\end{equation}
pri čemer smo uporabili Einsteinov zapis seštevanja po indeksih. Značilne vrednosti
susceptibilnosti v trdnih snoveh so $\chi^{(1)} \sim 1$, 
$\chi^{(2)} \sim 10^{-11}~\si{\metre/\volt}$ 
in $\chi^{(3)} \sim 10^{-22}~\si{\metre^2/\volt^2}$. Obravnavali bomo samo snovi, v katerih
ni izgub in so susceptibilnosti realne.

\begin{definition}
Pokaži, da so gostote svetlobnega toka, pri katerih dosežemo znaten nelinearen 
prispevek k polarizaciji in velja
 $$\frac{P_{NL}}{P_L} \sim 10^{-5},$$
velikostnega reda $1~\si{\mega\watt/\centi\metre^2}$. 
Ker so take vrednosti z navadnimi svetili povsem nedosegljive, je bilo mogoče nelinearne
optične pojave opazovati šele po iznajdbi laserjev.
\end{definition}
 
Tenzor $\chi^{(2)}$ je od nič različen le v snoveh, ki nimajo centra inverzije. 
Ker lahko v produktu (enačba~\ref{eq:nlin2}) vrstni red $E_j E_k$ zamenjamo, mora biti
tenzor invarianten na zamenjavo
\begin{equation}
\chi_{ijk} = \chi_{ikj}.
\label{eq:chijk}
\end{equation}
Zaradi te simetrije lahko vpeljemo poenostavljen zapis, pri katerem prvi indeks 
prepišemo $(x = 1, y = 2, z = 3)$,
zadnja dva indeksa pa združimo. Dogovorjene oznake so $xx = 1, yy = 2, zz = 3, yz = zy =4, 
xz = zx =5, xy = yx = 6$. Tako na primer $\chi_{xxz}$ zapišemo kot $\chi_{15}$. Namesto
splošnega tenzorja tretjega ranga smo s tem uvedli matriko velikosti $3\times6$,
v kateri je zaradi simetrijskih lastnosti kristala navadno le nekaj komponent 
različnih od nič. 

Kadar je v snovi absorpcija dovolj majhna, lahko matriko poenostavimo
z dodatnim približkom, tako imenovano  
\index{Kleinmanova domneva} Kleinmanovo domnevo\footnote{D. A. Kleinman, Phys. Rev. 126, 1977 (1962).}.
Ta pravi, da je 
\begin{equation}
\chi_{ijk} = \chi_{ikj} = \chi_{kij} = \chi_{kji} = \chi_{jik} = \chi_{jki}.
\label{Klein}
\end{equation}
\begin{table}[h!]
 \centering
\begin{tabular}{|c|c|c|c|} \hline  
      Kristal & Grupa & Neničelne komponente tenzorja $\chi$ & Vrednosti ($10^{-12}~\si{\metre/\volt}$)\\ \hline
      BaTiO\index{BaTiO$_3$}$_3$ & 4mm & $\chi_{xxz} = \chi_{yyz} = \chi_{xzx} = \chi_{yzy} = 
      \chi_{15} = \chi_{24}$  &
	    $\chi_{15} = 42,6$ \\
	      & & $\chi_{zxx} = \chi_{zyy} = \chi_{31} = \chi_{32}$ &  $\chi_{31} = 45,2$ \\
	      & & $\chi_{zzz} = \chi_{33}$ & $\chi_{33} = 16,0$ \\ \hline
      KDP\index{KDP} & 
      $\overline{4}$2m & $\chi_{xyz} = \chi_{yxz} = \chi_{xzy} = \chi_{yzx} = \chi_{14} = \chi_{25}$  &
	    $\chi_{14} = 0,88$ \\
	    & & $\chi_{zxy} = \chi_{zyx} = \chi_{36}$ &  $\chi_{36} =1,12$ \\ \hline
      Telur\index{Telur} & 32 & $\chi_{xxx} = -\chi_{xyy} = -\chi_{yyx} = -\chi_{yxy} =$  & \\
      & &  = $\chi_{11} = -\chi_{12}=-\chi_{26}$  &
	    $\chi_{11} = 1300$ \\
	    & & $\chi_{xyz} = \chi_{xzy} = -\chi_{yxz}= - \chi_{yzx}= \chi_{14} = 
	    -\chi_{25}$ &  $\chi_{14} \approx 0$ 
	    \\ \hline
      LiNbO$_3$\index{LiNbO$_3$} & 3m & $\chi_{xxz} = \chi_{yyz} = \chi_{xzx} = \chi_{yzy} = \chi_{15} = \chi_{24}$  &
	     \\
	     & & $\chi_{zxx} = \chi_{zyy} = \chi_{31} = \chi_{32}$ &  $\chi_{31} = -11,9$ \\
	      & & $\chi_{zzz} = \chi_{33}$ & $\chi_{33} = 68,8$ \\
	    & &  $-\chi_{xxy} = - \chi_{xyx} = \chi_{yyy} = -\chi_{yxx}  = $ & \\
	    & & $=-\chi_{16} = \chi_{22}$ = $-\chi_{21}$  &
	    $\chi_{22}  = 5,52$ \\
\hline 
\end{tabular}
  \caption{Koeficienti nelinearne susceptibilnosti za nekaj izbranih 
  snovi}
  \index{Susceptibilnost!nelinearna} 
\label{table:chi}
\end{table}

Poglejmo primer. 
Vzemimo barijev titanat (BaTiO$_3$)\index{BaTiO$_3$} s točkovno grupo 4mm. To pomeni, da
ima 4-števno os simetrije in dve zrcalni ravnini, od katerih ena preslika $x \to -x$ ali $y \to -y$, 
druga pa $x\to y$ in $y\to x$. Od nič različni elementi susceptibilnosti so tako samo
\begin{equation}
\chi_{xxz} = \chi_{xzx} =   \chi_{yyz} = \chi_{yzy}, \quad  \chi_{zzz} \quad \mathrm{in} 
\quad \chi_{zxx} = \chi_{zyy}.   
\end{equation}
Z upoštevanjem Kleinmanove domneve se število različnih členov še zmanjša in ostaneta le dva
\begin{equation}
\chi_{xxz} = \chi_{xzx} = \chi_{yyz} = \chi_{yzy} =\chi_{zxx} = \chi_{zyy} \quad \mathrm{in} \quad \chi_{zzz}.   
\end{equation}
Primerjajmo zdaj gornjo trditev z vrednostmi, podanimi v tabeli~(\ref{table:chi}). V tabeli
so navedene izmerjene nelinearne susceptibilnosti\footnote{Izmerjene vrednosti, 
ki jih najdemo v literaturi, se od vira do vira pogosto znatno razlikujejo.} in vidimo, da Kleinmanova
domneva ni povsem točna, je pa razmeroma dober približek. 

\section{Nelinearni optični pojavi drugega reda}
\index{Nelinearna optika!drugega reda}
Vzemimo optično nelinearen kristal s $\chi^{(2)} \neq 0$. V smeri pravokotno 
glede na njegovo mejno ploskev naj vpadata dve valovanji s frekvencama
$\omega_{1}$ in $\omega_{2}$. Zaradi nelinearne sklopitve nastajajo v snovi nova 
valovanja z različnimi kombinacijami frekvenc (slika~\ref{fig:nl2}).
Tako poleg valovanj z osnovnima frekvencama izhajajo iz kristala tudi 
valovanja pri dvakratnikih obeh vstopnih frekvenc, pri njuni vsoti, 
razliki in celo pri frekvenci nič. Oglejmo si te pojave podrobneje.
\\ 

\begin{figure}[h]
\centering
\def\svgwidth{140truemm} 
\input{slike/08_nl3.pdf_tex}
\caption{Shematski prikaz nastanka valovanj pri nelinearnih optičnih pojavih drugega reda 
in spekter izhodne svetlobe, pri čemer intenzitete izhodnih žarkov niso risane v merilu}
\label{fig:nl2}
\end{figure}

\begin{remark}
Nastanku valovanja pri podvojeni frekvenci oziroma optičnemu frekvenčnemu podvajanju pravimo tudi
SHG\index{SHG|see {Optično frekvenčno podvajanje}} ({\it Second harmonic 
generation})\index{Optično frekvenčno podvajanje}, 
nastanku valovanja pri vsoti frekvenc SFG\index{SFG|see {Generacija vsote frekvenc}}
({\it Sum frequency generation})\index{Generacija vsote frekvenc}, 
nastanku valovanja pri razliki frekvenc DFG\index{DFG|see {Generacija razlike frekvenc}} 
({\it Difference frequency generation})\index{Generacija razlike frekvenc} in pojavu 
statičnega polja pri $\omega = 0$ optično usmerjanje\index{Optično usmerjanje}
({\it Optical rectification}).  
\end{remark}

Ugotovili smo že, da navadna valovna enačba ne velja za opis pojavov pri velikih 
intenzitetah vpadnih valovanj. V tem primeru pride namreč  do pojava
nelinearne polarizacije in valovanje opišemo z nelinearno valovno 
enačbo\index{Valovna enačba!nelinearna}
\boxeq{8.3}{
\nabla^{2}\mathbf{E}-\frac{\epsilon}{c_0^{2}}{\frac{\partial^2\mathbf{E}}{\partial t^2}}=
\mu_{0}{\frac{\partial^2\mathbf{P}_{\textrm{NL}}}{\partial t^2}}.
}

\begin{definition}
Iz Maxwellovih enačb (enačbe~\ref{eq:Maxwell1}--\ref{eq:Maxwell4}) izpelji 
nelinearno valovno enačbo (enačba~\ref{8.3}), pri čemer upoštevaj enačbo~(\ref{8.1}). 
Pomagaj si z identiteto
$$
\nabla \times (\nabla \times \mathbf{A}) = \nabla (\nabla \cdot \mathbf{A}) 
- \nabla^2 \mathbf{A}.
$$
\end{definition} 

Nehomogene valovne enačbe v splošnem ne znamo rešiti in se moramo zateči k približkom.
Prva poenostavitev, ki jo naredimo, je omejitev na vzporedna vpadna žarka,
ki se širita v smeri osi $z$. Poleg tega se omejimo na izračun samo enega
nastalega valovanja in privzamemo, da je neodvisno od drugih nastalih valovanj.
Ta omejitev ni huda. Dokler sta namreč amplitudi valovanj pri vsoti in razliki
frekvenc majhni, ju lahko obravnavamo vsako posebej. Ni sicer nujno,
da sta amplitudi obeh nastalih valov vedno majhni, vendar je lahko, kot bomo videli 
pozneje, le eno valovanje naenkrat po jakosti primerljivo z vpadnim. 

V snovi so tako prisotna tri valovanja:
dve vpadni in tretje, novo nastalo. Zapišemo jih z 
\begin{eqnarray}
\mathbf{E}_{1} & = & \frac{\mathbf{e}_{1}}{2}\left[A_{1}(z)\, 
e^{i(k_{1}z-\omega_{1}t)}+A_{1}^{*}(z)\, e^{-i(k_{1}z-\omega_{1}t)}\right],\nonumber \\
\mathbf{E}_{2} & = & \frac{\mathbf{e}_{2}}{2}\left[A_{2}(z)\, 
e^{i(k_{2}z-\omega_{2}t)}+A_{2}^{*}(z)\, e^{-i(k_{2}z-\omega_{2}t)}\right] \quad \mathrm{in} \nonumber \\
\mathbf{E}_{3} & = & \frac{\mathbf{e}_{3}}{2}\left[A_{3}(z)\, 
e^{i(k_{3}z-\omega_{3}t)}+A_{3}^{*}(z)\, e^{-i(k_{3}z-\omega_{3}t)}\right].
\end{eqnarray}
Polja smo zapisali v realni obliki, to je s kompleksno konjugiranimi
deli, saj valovna enačba (enačba~\ref{8.3}) ni linearna. Upoštevali smo tudi,
da so zaradi nelinearnih pojavov amplitude funkcije kraja, za
katere pa lahko privzamemo, da se le počasi spreminjajo. Njihova kompleksna vrednost
dopušča pojav dodatnega faznega zamika. Za valovna
števila velja $k_{n}^{2}=\epsilon_{n}\omega_n^{2}/c_0^{2}$,
pri čemer je $\epsilon_{n}$ dielektrična konstanta pri frekvenci
$\omega_{n}$ in polarizaciji $\mathbf{e}_{n}$, indeks $n = 1...3$ pa označuje
valovanje. S tem nastavkom vsako od treh valovanj
pri konstantni amplitudi reši linearni del valovne enačbe. 

Naša naloga
je ugotoviti, kako se zaradi nelinearnosti spreminjajo amplitude posameznih valovanj.
Nastavek za polje, ki bo rešil nelinearno valovno enačbo, je tako
\begin{equation}
\mathbf{E}(z,t) = \sum_{n=1}^3 \frac{\mathbf{e}_{n}}{2}\left[A_{n}(z)\, 
e^{i(k_{n}z-\omega_{n}t)}+A_{n}^{*}(z)\, e^{-i(k_{n}z-\omega_{n}t)}\right].
\label{eq:nlnastavek}
\end{equation}
Izračunajmo najprej 
\begin{equation}
\nabla^{2}\mathbf{E}=-\sum_{n=1}^3 \frac{\mathbf{e}_{n}}{2}\left[k_{n}^{2}A_{n}(z)-2ik_{n}
\frac{dA_{n}(z)}{dz}\right]\, e^{i(k_{n}z-\omega_{n}t)}+\mbox{ k. k.}
\label{8.5}
\end{equation}
S k. k. smo označili kompleksno konjugirani del. Upoštevali smo,
da se amplituda $A_{n}(z)$ le počasi spreminja s krajem in smo zato njen
drugi odvod po kraju zanemarili.
Izračunamo še drugi odvod po času 
\begin{equation}
\frac{\partial^2\mathbf{E}}{\partial t^2}=\sum_{n=1}^3 \frac{\mathbf{e}_{n}}{2}
\left(-\omega_n^2\right) \left[A_{n}(z)\, e^{i(k_{n}z-\omega_{n}t)}+\mbox{ k. k.}\right].
\label{8.5a}
\end{equation}
Nelinearna polarizacija vsebuje produkte polj, ki nihajo z vsemi možnimi
vsotami in razlikami parov frekvenc $\omega_{1}$, $\omega_{2}$ in
$\omega_{3}$
\begin{multline}
\mathbf{P}_{\mathrm{NL}}= \epsilon_{0}\chi^{(2)}:\mathbf{E}\, \mathbf{E} =
\varepsilon_0 \sum_{n=1}^3 \sum_{m=1}^3 
 \left( \frac{1}{4} \chi^{(2)}:\mathbf{e}_{n}\,\mathbf{e}_{m}\right) 
 A_{n}(z)\,A_{m}(z) e^{i(k_{n}+k_{m})z-i(\omega_{n}+\omega_{m})t}+  \\
\left( \frac{1}{4} \chi^{(2)}:\mathbf{e}_{n}\,\mathbf{e}_{m}\right)
A_{n}(z)\,A_{m}^*(z) e^{i(k_{n}-k_{m})z-i(\omega_{n}-\omega_{m})t}+ \mathrm{k. k.}
\label{8.5b}
\end{multline}
Da je valovna enačba (enačba~\ref{8.3}) izpolnjena ob vsakem času $t$, se morajo
ujemati izrazi pri istih časovnih odvisnostih, to je pri istih frekvencah. Najprej 
zberemo člene pri $\omega_n = \omega_3$ in $\omega_3 = \omega_1 + \omega_2$. Zapišemo
\begin{equation}
ik_{3}\mathbf{e}_{3}\frac{dA_{3}}{dz}e^{ik_{3}z}=-\frac{\mu_{0} 
\varepsilon_0 \omega_{3}^{2}}{4}\chi:\mathbf{e}_{1}\mathbf{e}_{2}\,A_{1}\,A_{2}e^{i(k_{1}+k_{2})z}.
\label{8.7}
\end{equation}
Množimo še obe strani skalarno z $\mathbf{e}_{3}$, upoštevajmo zvezo med $k_{3}$ in $\omega_{3}$
in ravnajmo podobno še za drugi dve valovanji. Tako dobimo sistem sklopljenih
enačb za amplitude valovanj v optično nelinearnem sredstvu
\boxeq{eq:nlAz}{
\frac{dA_{3}}{dz} &= \frac{i\omega_{3}\chi_{ef}}{4c_0 n_3} A_{1}\, A_{2}\, e^{-i\Delta kz}\\
\frac{dA_{2}}{dz} &= \frac{i\omega_{2}\chi_{ef}}{4c_0 n_2} A_{1}^*\, A_{3}\, e^{i\Delta kz}\\
\frac{dA_{1}}{dz} &= \frac{i\omega_{1}\chi_{ef}}{4c_0 n_1} A_{2}^*\, A_{3}\, e^{i\Delta kz}\label{eq:nlA3}.
}
Pri tem je efektivna susceptibilnost\index{Susceptibilnost!efektivna}
\begin{equation}
\chi_{ef}=\mathbf{e}_{3}\cdot\chi:\,\mathbf{e}_{1}\,\mathbf{e}_{2} = 
\sum_{ijk} \chi_{ijk} e_{3i} e_{1j} e_{2k}.
\label{eq:chicomp}
\end{equation}
Ker polarizacijski vektorji niso nujno vzporedni s koordinatnimi osmi, tudi $\chi_{ef}$ 
niso čiste kartezične komponente tenzorja nelinearne susceptibilnosti.
\begin{definition}
Pokaži, da iz Kleinmanove domneve (enačba~\ref{Klein}) sledi, da so 
efektivne susceptibilnosti $\chi_{ef}$ v vseh treh enačbah~(\ref{eq:nlAz}--\ref{eq:nlA3}) enake.
\end{definition}
\begin{definition}
Pokaži, da nastavek za polje v nelinearni snovi (enačba~\ref{eq:nlnastavek}) reši nelinearno
valovno enačbo (enačba~\ref{8.3}), in pokaži, da spreminjanje amplitude posameznih valovanj 
ustreza enačbam~(\ref{eq:nlAz}--\ref{eq:nlA3}).
\end{definition}
Z $\Delta k$ smo označili razliko valovnih vektorjev
\begin{equation}
\Delta k = k_{3}-k_{1}-k_{2}.
\end{equation}
Čeprav je $\omega_{3}-\omega_{2}-\omega_{1}=0$, je $\Delta k$ navadno različen od nič zaradi 
frekvenčne disperzije lomnega količnika. Videli bomo, da je to ključnega pomena 
za vrsto nelinearnih optičnih pojavov. 

Zapisani sistem diferencialnih enačb (enačbe~\ref{eq:nlAz}--\ref{eq:nlA3}) opisuje več pojavov, 
odvisno od začetnih pogojev in relativnih intenzitet. Opisali bomo nekaj
najpomembnejših primerov.


\section{Optično frekvenčno podvajanje}

Obravnavajmo optično nelinearno sredstvo, na katerega vpadata valovanji ${\mathbf E}_1$ in
$\mathbf{E}_2$.
Naj bosta frekvenci vpadnih valovanj enaki $\omega_{1}=\omega_{2}=\omega$, valovanji
pa razlikujemo zaradi možnosti dveh različnih polarizacij. Takrat je $\omega_{3}=2\omega$
in govorimo o najpreprostejšem in tudi najpomembnejšem optičnem nelinearnem 
pojavu~\textendash~frekvenčnem podvajanju\index{Optično frekvenčno podvajanje}. 
Pogosto ga uporabljamo za pridobivanje laserskih snopov pri krajših valovnih dolžinah, na primer
pri Nd:YAG laserju\index{Laser!Nd:YAG}, ko infrardeče izhodno valovanje ($1064~\si{\nano\metre}$) 
pretvorimo v vidno svetlobo zelene barve ($532~\si{\nano\metre}$).\index{Infrardeče valovanje} 

Zanima nas, kako se $A_{3}(z) = A_{2\omega}(z)$ spreminja vzdolž nelinearnega kristala
pri začetnem pogoju $A_{2\omega}(0)=0$.
Privzemimo še, da se pretvori le manjši del vpadnega energijskega toka,
tako da sta amplitudi $A_{1}=A_{2}=A_0$ približno konstantni. Tedaj lahko
enačbo za $A_{3}(z)$ (enačba~\ref{eq:nlAz}) brez težav integriramo do dolžine kristala $L$ in 
zapišemo
\begin{equation}
A_{2\omega}(L)=\frac{i\omega \chi_{ef} A_0^2}{2c_0 n_{2\omega}}
\,e^{-i\Delta kL/2}\, \frac{\sin\left(\frac{\Delta k L}{2}\right)}{\frac{\Delta kL}{2}}L,
\label{8.9}
\end{equation}
kjer smo z $n_{2\omega}$ označili lomni količnik pri dvojni frekvenci.
Iz tega izraza izračunamo izhodno gostoto svetlobnega toka pri dvojni
frekvenci 
\begin{equation}
j_{2\omega}(L) =\frac{1}{2}\epsilon_{0}n_{2\omega}c_0|A_3|^2 = 
\frac{\omega^2 \chi_{ef}^2}{2 n_{2\omega} n_\omega^2c_0^3\varepsilon_0} j_\omega^2 L^2
\left(\frac{\sin\left(\frac{\Delta k L}{2}\right)}{\frac{\Delta kL}{2}}\right)^2.
\label{8.10}
\end{equation}
Gostota energijskega toka frekvenčno podvojene svetlobe torej narašča s kvadratom
intenzitete vpadne svetlobe. Naj bo $S$ presek snopa. Potem je razmerje med 
energijskim tokom pri podvojeni in osnovni frekvenci (izkoristek pretvorbe) enako
\boxeq{8.11}{
\frac{P_{2\omega}}{P_{\omega}}=
\frac{\omega^2 \chi_{ef}^2}{2 S n_{2\omega} n_\omega^2c_0^3\varepsilon_0} P_\omega L^2
\left(\frac{\sin\left(\frac{\Delta k L}{2}\right)}{\frac{\Delta kL}{2}}\right)^2.
}

Poglejmo si faktor $\sin^{2}(\Delta kL/2)/(\Delta kL/2)^{2}$, katerega odvisnost
od $\Delta kL/2$ je prikazana na sliki~(\ref{fig:shg2}). Vidimo, da je zaradi tega
faktorja na poti, daljši od $2\pi /\Delta k$, stopnja pretvorbe zelo majhna.
\begin{figure}[h]
\centering
\def\svgwidth{90truemm} 
\input{slike/08_shg2.pdf_tex}
\caption{Izkoristek pretvorbe v frekvenčno podvojeno valovanje je 
sorazmeren s funkcijo $(\sin(x)/x)^2$,
pri čemer je $x = \Delta k L/2$.}
\label{fig:shg2}
\end{figure}

Poglejmo primer. Faktor $\Delta k$ je različen od nič zaradi odvisnosti
lomnih količnikov od valovne dolžine. V KDP\index{KDP} je 
redni lomni količnik pri $1000~\si{\nano\metre}$ 1,496 in pri 
$500~\si{\nano\metre}$ 1,514. Vrednost, pri kateri
pade intenziteta frekvenčno podvojenega valovanja na nič
$L_{c}=2\pi /\Delta k$, je tako le okoli $30~\si{\micro\meter}$. Na večjih dolžinah
postane stopnja pretvorbe zanemarljivo majhna.

\begin{remark}
Pri izpeljavi frekvenčnega podvajanja iz enačb za nelinearne pojave drugega
reda (enačbe~\ref{eq:nlAz}--\ref{eq:nlA3}) moramo biti pazljivi. 
Uporabili smo splošne enačbe in tako privzeli, da je 
vpadno valovanje sestavljeno iz dveh ločenih valovanj s frekvenco $\omega$
z intenziteto $j_\omega$. Lahko pa frekvenčno podvajanje obravnavamo
z enim vpadnim valovanjem s frekvenco $\omega$ in intenziteto $2j_\omega$, 
ki nelinearno interagira samo s sabo. Takrat je zapis enačb za predfaktor
drugačen, končen rezultat pa seveda enak. 
\end{remark}

Za visok izkoristek pretvorbe v frekvenčno podvojeno valovanje je  
pomembno, da je $\Delta k$ čim manjši oziroma da se faze valovanj čim bolj ujemajo.
Če uspemo doseči, da je $\Delta k = 0$, je vrednost faktorja 
$\sin(\Delta kL/2)/(\Delta kL/2)$ največja in neodvisna od dolžine poti $L$.
V tem primeru izkoristek pretvorbe narašča sorazmerno s kvadratom poti
\begin{equation}
\frac{P_{2\omega}}{P_{\omega}}=
\frac{\omega^2 \chi_{ef}^2}{2 S n_{2\omega} n_\omega^2c_0^3\varepsilon_0} P_\omega L^2.
\end{equation}
Za uporabno pretvorbo v frekvenčno podvojeno valovanje je torej treba doseči 
fazno ujemanje valovnih vektorjev\index{Ujemanje faz} 
pri osnovni in podvojeni frekvenci. Kako to naredimo,
bomo spoznali v nadaljevanju.

\begin{definition}
\label{deplet}
Pokazali smo, da intenziteta frekvenčno podvojenega valovanja narašča sorazmerno s
kvadratom dolžine kristala (enačba~\ref{8.10}). Takšna odvisnost velja le, če je intenziteta valovanja
pri podvojeni frekvenci bistveno manjša od intenzitete vpadnega valovanja, 
oziroma $A_3 \ll A_1, A_2$. Pokaži, da v nasprotnem primeru intenziteta frekvenčno
podvojenega valovanja $j_{2\omega}(L)$ narašča kot
\begin{equation}
j_{2\omega} (L) = j_\omega \tanh^2 \left(\chi_{ef}\omega L \sqrt{\frac{j_\omega}
{2 n_{2\omega} n_\omega^2 c_0^3 \varepsilon_0}} \right) = j_\omega \tanh^2(\kappa L),
\end{equation}
pri čemer je $j_\omega$ vpadna intenziteta valovanja pri osnovni frekvenci. 

Namig: upoštevaj, 
da se celotna energija ohranja.
\end{definition}

\begin{figure}[h]
\centering
\def\svgwidth{90truemm} 
\input{slike/08_shg_depletion.pdf_tex}
\caption{Izkoristek pretvorbe v frekvenčno podvojeno valovanje. Če privzamemo, da se
intenziteta osnovnega žarka ne zmanjšuje, je odvisnost parabolična (rdeča krivulja), kar 
je dober približek le za majhne intenzitete. Bolj natančen izračun pokaže, da je izkoristek 
pretvorbe sorazmeren s $\tanh^2(\kappa L)$ (črna krivulja).}
\label{fig:shg2dep}
\end{figure}

Kaj pa se zgodi, kadar pogoj ujemanja faz ni izpolnjen in 
 $\Delta k \neq 0$? Takrat dolžino kristala $L$ v enačbi~(\ref{8.11})
okrajšamo in izkoristek pretvorbe z naraščajočim
$L$ sinusno niha med nič in neko največjo vrednostjo. Tak pojav lahko opazimo, če
uporabimo klinast vzorec, ki se mu debelina spreminja, ali pa če vzorec sučemo 
in na ta način spreminjamo razliko faz. Ta pojav, imenujemo ga Makerjeve 
oscilacije\footnote{P. D. Maker et al., Phys. Rev. Lett. 8, 21 (1962).}, 
uporabljamo za merjenje nelinearne susceptibilnosti kristalov.\index{Makerjeve oscilacije}

\subsection*{Ujemanje faz}
\index{Ujemanje faz}
Poglejmo, kako lahko dosežemo ujemanje faz, ki je nujno za učinkovito optično
frekvenčno podvajanje. Spomnimo se, da je pogoj za ujemanje faz 
\begin{equation}
\Delta k = k_3 - k_1 -k_2 = k_3^{2\omega} - k_1^{\omega} -k_2^\omega = 
\frac{2\omega}{c_0} n_3 - \frac{\omega}{c_0} n_1- \frac{\omega}{c_0} n_2 =0.
\end{equation}
Iz tega sledi pogoj
\boxeq{eq:dk0}{
n_1^\omega + n_2^\omega = 2n_3^{2\omega}.
}
Da lahko zadostimo temu pogoju, izkoristimo dvojni lom\index{Dvolomnost} v 
anizotropnih kristalih. Zaradi enostavnosti se omejimo le na optično 
enoosne kristale\index{Dvolomnost!enoosne snovi} brez absorpcije in z normalno disperzijo, 
pri katerih oba lomna količnika naraščata s frekvenco.  

Za razumevanje je najbolj nazoren grafični prikaz (slika~\ref{fig:dk}), pri katerem
rišemo presek ploskve valovnega vektorja z ravnino, določeno z optično osjo in valovnim
vektorjem (glej poglavje~\ref{chap:anizotropni}). 
Za vsako smer valovnega vektorja obstajata dve rešitvi:
rednemu žarku, katerega polarizacija je pravokotna na omenjeno ravnino,
ustreza krožnica s polmerom $n_o$, izrednemu, katerega polarizacija leži v ravnini, 
pa elipsa s polosema $n_o$ in $n_e$.\index{Lomni količnik} 
Rdeča barva nakazuje presek ploskve pri vpadni frekvenci, modra pa pri podvojeni. 
Ekscentričnost elipse in frekvenčna disperzija sta zaradi večje nazornosti močno 
pretirani. 

Podrobneje poglejmo primer s slike (a), za katerega velja $n_e>n_o$. 
Opazimo, da se v neki točki rdeča elipsa, ki ustreza vpadnemu valovanju, seka
z modro krožnico, ki ustreza valovanju s podvojeno frekvenco. Pri tem kotu 
$\sphericalangle (\mathbf{k}, z)$ je torej redni lomni 
količnik pri dvojni frekvenci enak izrednemu količniku pri osnovni
frekvenci. Če izberemo izredno polarizacijo vpadnega valovanja, je za podvojeno 
valovanje z redno polarizacijo pri kotu $\vartheta_m$ izpolnjen pogoj ujemanja 
faz~(enačba~\ref{eq:dk0}). Takrat leži polarizacija vpadnega valovanja v ravnini 
optične osi in smeri širjenja, polarizacija izhodnega frekvenčno podvojenega žarka pa 
je pravokotna na optično os. Zapišimo ta razmislek še z enačbo.

\begin{figure}[h]
\centering
\def\svgwidth{140truemm} 
\input{slike/08_phasematch.pdf_tex}
\caption{Štirje primeri, pri katerih je izpolnjen pogoj za ujemanje faz. 
Ujemanje faz prvega reda za pozitivno anizotropno snov (a), 
ujemanje faz prvega reda za negativno anizotropno snov (b) ter 
ujemanje faz drugega reda za pozitivno (c) in negativno (d) anizotropno snov.
Za razlago glej besedilo.}
\label{fig:dk}
\end{figure}

V obravnavanem primeru mora biti lomni količnik za redno polarizirano valovanje pri 
podvojeni frekvenci $n_o^{2\omega}$ enak lomnemu količniku za izredno 
polarizirano valovanje pri osnovni frekvenci $n^\omega$. Lomni količnik
za izredno valovanje je seveda odvisen od kota (enačba~\ref{eq:izreden})
\begin{equation}
\frac{1}{(n^{\omega}(\vartheta))^2}=
\frac{\cos^{2}\vartheta}{(n_{o}^{\omega})^2}+\frac{\sin^{2}\vartheta}{(n_{e}^{\omega})^2}
=\frac{1}{(n_o^{2\omega})^2}.
\label{8.12}
\end{equation}
Sledi izraz
\begin{equation}
\cos^{2}\vartheta_m=\frac{(n_o^{2\omega})^{-2}-(n_{e}^{\omega})^{-2}}
{(n_{o}^{\omega})^{-2}-(n_{e}^{\omega})^{-2}},
\label{8.13}
\end{equation}
iz katerega lahko izračunamo kot $\vartheta_m$, pri katerem pride do ujemanja faz.
Pri optično enoosnih kristalih je pogoj ujemanja faz določen s kotom širjenja
svetlobe glede na smer optične osi v kristalu in obstaja cel stožec dovoljenih smeri.
\begin{definition}
Pokaži, da v primeru negativne anizotropije pogoj za ujemanje faz zapišemo kot
\begin{equation}
\cos^{2}\vartheta_m=\frac{(n_o^{\omega})^{-2}-(n_{e}^{2\omega})^{-2}}
{(n_{o}^{2\omega})^{-2}-(n_{e}^{2\omega})^{-2}}.
\label{8.13a}
\end{equation}
\end{definition}

S slike~(\ref{fig:dk} c in d) lahko razberemo, da obstaja še en primer, pri 
katerem je izpolnjen pogoj za ujemanje faz. Kot zgled t.\ i.\ ujemanja faz drugega reda
obravnavajmo primer na sliki (c).
Vzemimo različno polarizirani vhodni valovanji z ustreznima različnima lomnima
količnikoma $n_o^\omega$ in $n^\omega(\vartheta)$ . 
Enačba~(\ref{eq:dk0}) je izpolnjena, kadar je povprečje lomnih količnikov
vhodnih valovanj  enako lomnemu količniku frekvenčno podvojenega žarka $n_o^{2\omega}$. 
To se zgodi pri tistem kotu $\vartheta_m$, pri katerem je modra krožnica ravno na 
sredini med rdečo krožnico in elipso. Za praktično uporabo je ta izbira, kadar obstaja,
celo ugodnejša, saj je pri njej kot ujemanja faz bliže $\pi/2$. 
Ujemanje faz je zato manj občutljivo na majhna odstopanja v kotu ali na temperaturne
spremembe lomnih količnikov. Račun kota $\vartheta_m$ za ta primer je
bolj zahteven, saj je treba rešiti enačbo četrte stopnje.

\subsection*{Efektivna susceptibilnost}
\index{Susceptibilnost!efektivna}
Na izhodno moč frekvenčno podvojenega snopa poleg faznega faktorja bistveno vpliva tudi efektivna 
susceptibilnost $\chi_{ef}$ (enačba~\ref{eq:chicomp}). Ta je odvisna 
od polarizacij vhodnega in izhodnega žarka ter seveda od simetrije kristala. Ugotovili smo že, 
da je v optično enoosnem kristalu kriterij ujemanja faz izpolnjen na stožcu okoli 
optične osi, pri čemer je stožec določen 
z izračunanim kotom $\vartheta_m$ (enačbi~\ref{8.13} in \ref{8.13a}). 
Drugi kot, ki določa smer širjenja v ravnini, pravokotni na optično os, 
izberemo tako, da izkoristimo največje komponente nelinearne 
susceptibilnosti. 

Oglejmo si kot primer KDP (KH$_{2}$PO$_{4}$)\index{KDP}, ki je negativno anizotropen 
z vrednostmi $n_o^{\omega} = 1,4942$, 
$n_e^{\omega} = 1,4603$, $n_o^{2\omega} = 1,5129$ in $n_e^{2\omega} = 1,4709$
(slika~\ref{fig:dk}\,b). Valovna dolžina osnovnega snopa naj bo 1064~nm. 
Zaradi negativne anizotropije za izračun kota ujemanja faz 
uporabimo enačbo~(\ref{8.13a}) in dobimo $\vartheta_m = 41,25^\circ$. 
Poleg tega iz tabele~(\ref{table:chi}) razberemo, da ima nelinearna susceptibilnost v tetragonalni
simetriji $\bar{4}2m$ od nič različne komponente $\chi_{xyz}$, $\chi_{xzy}$
$\chi_{zxy}$, $\chi_{zyx}$, $\chi_{yzx}$ in $\chi_{yxz}$.
Zaradi poenostavitve privzamemo, da so njihove vrednosti enake. 

Naj se osnovno in frekvenčno podvojeno valovanje širita 
v smeri $\mathbf{s}$. Pomagamo si s sliko (\ref{fig:chi}) in zapišemo 
vektor $\mathbf{s}$, pri čemer $\varphi$ označuje kot med osjo $x$ in projekcijo 
$\mathbf{s}$ na ravnino $xy$
\begin{equation}
\mathbf{s}=(\cos\varphi\sin\vartheta_m,\sin\varphi\sin\vartheta_m,\cos\vartheta_m).
\label{8.14}
\end{equation}

\begin{figure}[h]
\centering
\def\svgwidth{80truemm} 
\input{slike/08_chi.pdf_tex}
\caption{K izračunu efektivne susceptibilnosti. Črtkan krog opisuje osnovno ploskev
stožca, ki je določen s $\vartheta_m$, pikčast krog pa njegovo projekcijo
na ravnino $xy$. Rdeč vektor označuje polarizacijo redno polariziranega valovanja, 
moder pa polarizacijo izredno polariziranega valovanja.}
\label{fig:chi}
\end{figure}

Naša naloga je poiskati kot $\varphi$, pri katerem je 
$\chi_{ef}$ največji in s tem največja tudi moč frekvenčno podvojenega valovanja.
Iz pogoja za ujemanje faz smo določili, da mora biti vpadna svetloba redno polarizirana, 
izhodna frekvenčno podvojena pa izredno polarizirana. Pri tem je redna polarizacija pravokotna na 
os $z$ (optično os) in hkrati pravokotna na smer vektorja $\mathbf{s}$. Zapišemo jo kot
\begin{equation}
\mathbf{e}_o=(e_{ox}, e_{oy}, e_{oz}) = (\sin\varphi,-\cos\varphi,0).
\label{8.15}
\end{equation}
To najlažje preverimo, tako da postavimo vektor $\mathbf{s}$ enkrat v ravnino $xz$ in
drugič v ravnino $yz$. Izredna polarizacija leži v ravnini, ki jo tvori vektor $\mathbf{s}$ z osjo $z$,
hkrati pa je pravokotna na vektor $\mathbf{s}$, 
tako da jo zapišemo kot 
\begin{equation}
\mathbf{e}_e=(e_{ex}, e_{ey}, e_{ez}) 
=(-\cos \varphi \cos \vartheta_m,-\sin \varphi \cos \vartheta_m ,\sin \vartheta_m).
\label{8.15a}
\end{equation}
Zdaj lahko izračunamo efektivno susceptibilnost (enačba~\ref{eq:chicomp}), 
pri čemer upoštevamo, da sta žarka 1 in 2 pri osnovni frekvenci redno polarizirana, 
žarek z oznako 3 pa opisuje izredno polariziran žarek pri podvojeni frekvenci
\begin{equation}
\chi_{ef} = \sum_{ijk} \chi_{ijk} e_{3i} e_{1j} e_{2k} = \sum_{ijk} \chi_{ijk} e_{ei} e_{oj} e_{ok}.
\end{equation}
Krajši račun pokaže, da je zaradi oblike tenzorja nelinearne susceptibilnosti v izbranem 
primeru od nič različna le ena komponenta nelinearne polarizacije, komponenta $z$. Zapišemo
\begin{equation}
\chi_{ef} = \chi_{zxy} e_{ez} e_{ox} e_{oy} + \chi_{zyx} e_{ez} e_{oy} e_{ox}.
\end{equation}
Sledi
\begin{eqnarray}
P_{z}^{2\omega}=- 2\varepsilon_0\, \chi_{zxy}E_{0}^2\cos\varphi\sin\varphi
\sin\vartheta_m = - \varepsilon_0\, \chi_{zxy}E_{0}^2\sin(2\varphi) \sin\vartheta_m.
\label{8.151}
\end{eqnarray}
Nelinearna polarizacija je največja pri $\varphi=\pi/4$, največji $\chi_{ef}$  pa je 
\begin{equation}
\chi_{ef}= 
\sin\vartheta_m \chi_{zxy} \approx 0,66\, \chi_{zxy} \approx 
0,74~\si{\pico\metre/\volt}.
\label{8.16}
\end{equation}

\begin{definition}
Izračunaj največjo možno efektivno nelinearno susceptibilnost za
frekvenčno podvajanje svetlobe z valovno
dolžino $10~\si{\micro\metre}$ v kristalu telurja s simetrijsko grupo 32 (glej tabelo~\ref{table:chi}). 
Lomni količniki: $n_o^{\omega} = 4,7969$, 
$n_e^{\omega} = 6,2455$, $n_o^{2\omega} = 4,8657$ in $n_e^{2\omega} = 6,3152$.\index{Telur}
\end{definition}

\begin{remark}
Namesto zvezne svetlobe za optično podvajanje frekvenc pogosto uporabimo laserske sunke, saj je 
vršna moč zelo velika in je zato velika tudi pretvorba v frekvenčno podvojen signal.
Vendar je treba biti pri tem pazljiv, saj lahko zaradi disperzije grupne hitrosti osnovni in 
podvojeni signal ne potujeta z enakima hitrostma. Navadno podvojeni signal potuje 
počasneje in zaostaja za osnovnim, zato lahko iz kristala izhaja razmeroma sploščen in 
precej razvlečen frekvenčno podvojen signal. Pojav je izrazit predvsem v podvajanju v
ultravijolični del spektra.\index{Ultravijolično valovanje}
\end{remark}

\section{Frekvenčno podvajanje Gaussovih snopov}
\index{Optično frekvenčno podvajanje}
Doslej smo vpadni in frekvenčno podvojeni snop obravnavali kot ravni valovanji,
ki sta bili razsežni v prečni smeri. Izračunali smo, da v primeru \index{Ujemanje faz}
ujemanja faz ($\Delta k=0$)
moč frekvenčno podvojene svetlobe narašča s kvadratom dolžine poti po nelinearnem
sredstvu. Pretvorba v frekvenčno podvojeno svetlobo je po enačbi~(\ref{8.11}) tem
učinkovitejša, čim večja je gostota svetlobnega toka pri osnovni frekvenci.
Zato v praksi vpadno svetlobo vselej zberemo in tako povečamo gostoto toka. 
Pri tem moramo paziti, da je nelinearen kristal odporen proti poškodbam
zaradi velike gostote svetlobnega toka. Odpornost in možnost izpolnitve kriterija ujemanja 
faz sta poglavitna kriterija pri izbiri snovi za frekvenčno podvajanje. 

Poglejmo, kako se enačbe spremenijo, če je vpadni snop pri osnovni 
frekvenci Gaussove oblike\index{Gaussov snop!frekvenčno podvajanje}. 
Rezultat lahko ocenimo, če vzamemo, da je
efektivna dolžina za pretvorbo $L$ kar dolžina grla; izven grla je 
gostota toka znatno manjša kot v grlu, s tem pa tudi izkoristek pretvorbe v 
frekvenčno podvojeni snop.
Dolžina grla je (enačba~\ref{eq:z0})
\begin{equation}
L=2z_{0}=2\frac{\pi w_{0}^{2}}{\lambda/n} = \frac{n w_0^2 \omega}{c_0}  \quad \mathrm{in} \quad 
w_{0}^{2} = \frac{c_0 L}{n \omega}.
\label{SHGG}
\end{equation}
Pri zapisu preseka vpadnega snopa upoštevajmo še faktor ena polovica, do katerega 
pridemo, če integriramo intenziteto 
snopa po celotni površini (glej nalogo~\ref{naloga-širina-snopa}). Sledi
\begin{equation}
S=\frac{1}{2}\pi w_{0}^{2} = \frac{\pi c_0 L}{2 n \omega}.
\end{equation}
Daljše ko je grlo in večja dolžina $L$, na kateri pride do frekvenčnega podvajanja, 
večji je tudi presek snopa $S$ in zato intenziteta svetlobe manjša, kar zmanjša
učinek pretvorbe v frekvenčno podvojeno valovanje.
V enačbi~(\ref{8.10}) upoštevamo ujemanje faz in $S$ pri podvojeni frekvenci. Tako velja 
\boxeq{8.17}{
\frac{P_{2\omega}}{P_{\omega}}=
\frac{\omega^3 \chi_{ef}^2}{2\pi n_{2\omega} n_\omega c_0^4\varepsilon_0} P_\omega\, L.
}
Ob optimalnem fokusiranju je izkoristek pretvorbe torej sorazmeren z dolžino kristala.
\begin{definition}
Imamo $1~\si{\centi\metre}$ dolg kristal KH$_{2}$PO$_{4}$.\index{KDP} Valovna dolžina vpadne svetlobe 
je $1,06~\si{\micro\metre}$, vhodna moč $P_\omega = 5~\si{\kilo\watt}$, efektivna nelinearna susceptibilnost
$\chi_{ef}=7\cdot10^{-13}~\si{\metre/\volt}$, $\Delta k=0$ in $n=1,5$. Pokaži, da je
faktor pretvorbe v frekvenčno podvojeno svetlobo okoli $27~\%$.
Da je dolžina grla $2z_{0}=1~\si{\centi\metre}$, mora biti polmer
grla okoli $40~\si{\micro\metre}$. Gostota svetlobnega toka v kristalu je pri
tem $2\cdot10^{8}~\si{\watt/\centi\metre^{2}}$, kar je že blizu praga za poškodbe,
predvsem na vstopni ali izstopni površini. 
\end{definition}

\section{{*}Račun podvajanja Gaussovih snopov}
\index{Gaussov snop!frekvenčno podvajanje}
V prejšnjem razdelku smo na hitro grobo ocenili vpliv oblike Gaussovih snopov
na frekvenčno podvajanje. Naredimo zdaj še natančnejši izračun. Vrnimo se k valovni
enačbi (enačba~\ref{8.3}), vpadna snopa naj bosta pri frekvencah
$\omega_{1}$ in $\omega_{2}$, nastajajoč snop pa pri frekvenci
$\omega_{3}=\omega_{1}+\omega_{2}$.
Vsako od polj naj ima obliko 
\begin{equation}
\mathbf{E}_{i}  = \frac{\mathbf{e}_{i}}{2}\left[\tilde{A}_{i}(r,z)\, 
e^{i(k_{i}z-\omega_{i}t)}+\tilde{A}_{i}^{*}(r,z)\, e^{-i(k_{i}z-\omega_{i}t)}\right],
\end{equation}
pri čemer je $\tilde{A}(r,z)$ funkcija tako vzdolžne kot tudi prečne koordinate. Privzeli
bomo, da se vzdolž smeri  $z$ le počasi spreminja.
Zaradi poenostavljenega zapisa vpeljemo novo spremenljivko 
\begin{equation}
\psi_i = \sqrt{\frac{n_i}{\omega_i}}\tilde{A}_i.
\end{equation}
Tako je nastavek za električno poljsko jakost
\begin{equation}
\mathbf{E}_{i}=\frac{\mathbf{e}_{i}}{2}\sqrt{\frac{\omega_{i}}{n_{i}}}\psi_{i}(r,z)
e^{i(k_{i}z-\omega_{i}t)}+\mbox{ k. k.}
\label{8.18}
\end{equation}
Vstavimo nastavek (enačbe~\ref{8.18}) v valovno
enačbo (enačba~\ref{8.3}) in ločimo na levi in desni člene z enako časovno odvisnostjo.
Zaradi počasnega spreminjanja vzdolž smeri $z$ lahko zanemarimo tudi druge odvode 
$\psi$ po $z$. Od tod sledi sklopljen sistem obosnih enačb 
\begin{eqnarray}
\nabla_{\perp}^{2}\psi_{1}+2ik_{1}\psi_{1}^{\prime} & = & -
\frac{k_{1}}{2}\kappa\psi_{2}^{\ast}\psi_{3}e^{i\Delta kz}\\
\nabla_{\perp}^{2}\psi_{2}+2ik_{2}\psi_{2}^{\prime} & = & -
\frac{k_{2}}{2}\kappa\psi_{1}^{\ast}\psi_{3}e^{i\Delta kz}\\
\nabla_{\perp}^{2}\psi_{3}+2ik_{3}\psi_{3}^{\prime} & =
& - \frac{k_{3}}{2}\kappa\psi_{1}\psi_{2}e^{-i\Delta kz}
\label{SHGGauss_3}
\end{eqnarray}
s pripadajočim sistemom konjugiranih enačb. Pri tem je 
\begin{equation}
\kappa=\frac{\chi_{ef}}{c_0} \sqrt{\frac{\omega_{1}\omega_{2}\omega_{3}}{n_{1}n_{2}n_{3}}}.
\label{8.20}
\end{equation}
S črtico smo označili odvajanje po $z$. Gornji sistem enačb je očitno
posplošitev sistema enačb~(\ref{eq:nlAz}--\ref{eq:nlA3}) za primer, ko je valovanje odvisno
tudi od prečne koordinate. Reševanje tega nelinearnega sistema parcialnih
diferencialnih enačb je v splošnem zelo zapleteno.

Poglejmo le najenostavnejši primer frekvenčnega podvajanja, ko je 
$\omega_{3}=2\omega_{1}=2\omega$.
Vpadna snopa naj bosta enaka in Gaussove oblike~(enačba~\ref{eq:gaussov-snop}), 
njuna amplituda pa naj bo enaka $A_1$
\begin{equation}
\psi_{1} = \psi_2 = A_{1}\frac{1}{1+iz/z_1}
\exp\left(-\frac{r^{2}}{w_1^{2}(z)}+\frac{ik_1r^{2}}{2R_1(z)}\right).
\label{8.21}
\end{equation}
Privzemimo še, da je izpolnjen pogoj za ujemanje faz\index{Ujemanje faz} ($\Delta k=0$) in da je $\psi_{3}$ dovolj majhen, da zmanjševanja $\psi_{1}$
ni treba upoštevati. Tudi za podvojeni snop privzemimo Gaussovo 
obliko, njegova amplituda $A_3$ pa naj le počasi narašča. Zapišemo ga kot
\begin{equation}
\psi_{3}=A_{3}(z)\psi_{3H}(z,r)=A_{3}(z)\frac{1}{1+iz/z_{3}}
\exp\left(-\frac{r^{2}}{w_{3}^{2}(z)}+\frac{ik_{3}r^{2}}{2R_{3}(z)}\right),
\label{8.22}
\end{equation}
pri čemer $\psi_{3H}$ reši homogeno obosno valovno 
enačbo (enačba~\ref{eq:obosna-valovna-enacba})\index{Obosna valovna enačba}. Ko izraza za $\psi_{1}$
in $\psi_{3}$ vstavimo v tretjo enačbo sistema sklopljenih enačb (\ref{SHGGauss_3}),
ostane na levi le člen oblike $2ik_{3}A_{3}^{\prime}(z)\psi_{3H}$. Tako dobimo pogoj
\begin{multline}
A_{3}^{\prime}(z)\frac{1}{1+iz/z_3}\exp\left(-\frac{r^{2}}{w_{3}^{2}(z)}+\frac{ik_{3}r^{2}}
{2R_{3}(z)}\right)=\\
\frac{i\kappa}{4}A_{1}^{2}\frac{1}{(1+iz/z_{1})^{2}}\exp\left(-\frac{2r^{2}}
{w_{1}^{2}(z)}+\frac{ik_{1}r^{2}}{R_{1}(z)}\right).
\label{8.23}
\end{multline}
Poiščimo rešitev te enačbe v obliki, za katero velja $w_{30}^{2}=w_{10}^{2}/2$. Tedaj je 
\begin{equation}
z_{3}=\frac{k_{3}w_{30}^{2}}{2}=\frac{2k_{1}w_{10}^{2}}{4}=z_{1}
\end{equation}
in je tudi $w_{3}^{2}(z)=w_{1}^{2}(z)/2$. Poleg tega je $R_{3}(z)=R_{1}(z)$
in lahko na obeh straneh krajšamo eksponentna faktorja. Ostane 
\begin{equation}
A_{3}^{\prime}(z)=\frac{i\kappa}{4}A_{1}^{2}\frac{1}{1+iz/z_1}.
\label{8.24}
\end{equation}
Gornjo enačbo seveda brez težav integriramo. Naj bo grlo vpadnega
snopa ravno na sredini nelinearnega sredstva, tako da integriramo
od $-L/2$ do $L/2$
\begin{eqnarray}
A_{3}(L) & = & \frac{i\kappa}{4}A_{1}^{2}\int_{-L/2}^{L/2}\frac{dz}{1+iz/z_1} 
  = \frac{\kappa}{4}A_{1}^{2}z_{1}\ln\frac{1+i\frac{L}{2z_{1}}}{1-i\frac{L}{2z_{1}}}= \nonumber \\
 & = & \frac{\kappa}{2}A_{1}^{2}z_{1}\arctan\frac{L}{2z_{1}}\;.
\end{eqnarray}
Moč Gaussovega snopa je
\begin{equation}
P_{i}=\frac{1}{2}\pi w_{i0}^{2} \frac{1}{2}c_0 n_i \epsilon_{0}E_{i0}^{2}=
\frac{\pi}{4}w_{i0}^{2}\varepsilon_0 c_0 \omega_{i} A_{i}^{2},
\label{8.26}
\end{equation}
tako da je izkoristek pri frekvenčnem podvajanju Gaussovega snopa 
\begin{equation}
\begin{aligned}
\frac{P_{2\omega}}{P_{\omega}}=\frac{A_3^2}{A_1^2}  = &
\frac{\chi_{ef}^2 \omega^3 P_\omega z_1}{\pi c_0^4 \varepsilon_0 n_\omega n_{2\omega}} 
\arctan^2 \left( \frac{L}{2z_1}\right) \\
 = &\frac{\chi_{ef}^2 \omega^3 P_\omega}{\pi c_0^4 \varepsilon_0 n_\omega n_{2\omega}} \frac{L}{2}
\arctan^2 \left( \frac{L}{2z_1}\right) \frac{1}{L/2z_1}.
\label{8.27}
\end{aligned}
\end{equation}
Funkcija $(\arctan^{2}x)/x$ zavzame največjo vrednost 0,64 pri $x =L/2z_1=1,39$.
Pri dani dolžini nelinearnega sredstva $L$ je torej 
izkoristek največji, kadar je $z_{1}=0,36\,L$, kar je malo manj kot pri
preprosti oceni $z_{1}=0,5\,L$ (enačba~\ref{SHGG}). Največji izkoristek
frekvenčnega podvajanja Gaussovih snopov je tako
\begin{equation}
\frac{P_{2\omega}}{P_{\omega}}
= 0,64 \frac{\omega^3 \chi_{ef}^2}{2\pi n_{2\omega} n_{\omega} c_0^4 \varepsilon_0 } P_\omega L.
\label{8.28}
\end{equation}
S preprosto oceno, ki smo jo naredili v prejšnjem razdelku (enačba~\ref{8.17}), smo tako 
rezultat le malo zgrešili, v obeh primerih pa izkoristek narašča linearno z dolžino kristala.

\section{Optično parametrično ojačevanje}
\index{Optično parametrično ojačevanje}
\index{Parametrično ojačevanje|see{Optično parametrično ojačevanje}}

Oglejmo si še en zelo uporaben primer mešanja treh valovanj, 
ki ga opisujejo enačbe (\ref{eq:nlAz}--\ref{eq:nlA3}). Gre za
optično parametrično ojačevanje, pri katerem nelinearne optične pojave
izkoristimo za ojačenje optičnih signalov. Imejmo razmeroma šibek vhodni
signal pri frekvenci $\omega_{1}$, ki ga želimo ojačati, in močno črpalno valovanje
pri frekvenci $\omega_{3}>\omega_{1}$. Zaradi nelinearnosti v snovi  
intenziteta valovanja pri $\omega_{1}$ narašča, 
intenziteta valovanja pri $\omega_{3}$ se zmanjšuje, hkrati pa zaradi
ohranitve energije nastaja dodatno valovanje pri razliki frekvenc
$\omega_{2}=\omega_{3}-\omega_{1}$. Proces parametričnega ojačevanja 
si torej lahko predstavljamo kot pretvorbo enega fotona pri frekvenci 
$\omega_{3}$ v dva fotona pri $\omega_{1}$ in $\omega_{2}$.
Parametrično ojačevanje pogosto uporabljamo za ojačenje šibkih signalov 
v infrardečem območju.\index{Infrardeče valovanje}
\begin{figure}[h]
\centering
\def\svgwidth{80truemm} 
\input{slike/08_opa.pdf_tex}
\caption{Shematski prikaz nastanka valovanj pri optičnem parametričnem ojačevanju}
\label{fig:opa2}
\end{figure}

Izhajamo iz splošnih enačb za nelinearne optične pojave drugega reda 
(enačbe~\ref{eq:nlAz}--\ref{eq:nlA3}). 
\begin{eqnarray}
\frac{dA_{3}}{dz} &=& \frac{i\omega_{3}\chi_{ef}}{4c_0 n_3} A_{1}\, A_{2}\, e^{-i\Delta kz}, \\
\frac{dA_{2}}{dz} &=&\frac{i\omega_{2}\chi_{ef}}{4c_0 n_2} A_{1}^*\, A_{3}\, e^{i\Delta kz}\quad \textrm{in}\\
\frac{dA_{1}}{dz} &=&\frac{i\omega_{1}\chi_{ef}}{4c_0 n_1} A_{2}^*\, A_{3}\, e^{i\Delta kz}.
\label{eq:opaA}
\end{eqnarray}
Privzamemo, da je črpalno valovanje vselej dosti močnejše od drugih dveh
($A_{3}\gg A_{1}$, $A_{2}$) in njegova jakost približno konstantna $A_3 = A_{30}$.
Poskrbimo še, da je izpolnjen pogoj za ujemanje faz $\Delta k=0$, 
začetna pogoja pa zapišemo kot $A_{1}(z=0)=A_{10}$ in $A_{2}(z=0)=0$. Ko vse to upoštevamo,
dobimo dve sklopljeni enačbi
\begin{eqnarray}
\frac{dA_{1}}{dz} &=& \frac{i\omega_{1}\chi_{ef}}{4c_0 n_1} A_{2}^*\, A_{30}\label{eq:opaA1} 
\qquad \mathrm{in} \\
\frac{dA_{2}^*}{dz} &=& -\frac{i\omega_{2}\chi_{ef}}{4c_0 n_2} A_{1}\, A_{30}^*.
\label{eq:opaA2}
\end{eqnarray}
Enačbi lahko rešimo, tako da prvo odvajamo po $z$ in vanjo vstavimo drugo enačbo.
Sledi
\begin{equation}
\frac{d^2 A_1}{d z^2} = \frac{\omega_1 \omega_2 \chi_{ef}^2|A_{30}|^2}
{16 c_0^2 n_1 n_2} A_1 = \kappa^2 A_1
\end{equation}
in podobno za $A_2$
\begin{equation}
\frac{d^2 A_2}{d z^2} = \frac{\omega_1 \omega_2 \chi_{ef}^2|A_{30}|^2}
{16 c_0^2 n_1 n_2} A_2 = \kappa^2 A_2.
\end{equation}
Ob upoštevanju začetnih pogojev izračunamo rešitev za naraščanje amplitude signalnega žarka
z začetno amplitudo $A_{10}$
\boxeq{eq:opa}{
A_1 = A_{10} \cosh (\kappa L).
}
Hkrati z njim narašča tudi amplituda dodatnega nedejavnega ({\it idle}) žarka, ki nastane
med procesom ojačenja\index{Nedejavni žarek}
\boxeq{eq:opan}{
A_2 = A_{20} \sinh (\kappa L).
}
V gornjih enačbah je $L$ dolžina nelinearnega sredstva, 
\begin{equation}
\kappa^2 = \frac{\omega_1 \omega_2 \chi_{ef}^2|A_{30}|^2}
{16 c_0^2 n_1 n_2} 
\label{opakapa}
\end{equation}
in
\begin{equation}
A_{20} = i \sqrt{\frac{\omega_2 n_1}{\omega_1 n_2}} A_{10}.
\label{opakapaA}
\end{equation}
Na začetku intenziteti obeh valovanj naraščata približno eksponentno na račun črpalnega
valovanja (slika~\ref{fig:opagraf}). Ko postane njuna intenziteta znatna in se 
začne $A_3$ zmanjševati, je treba to seveda
upoštevati pri izračunu. Rešiti je treba bolj zahteven sistem treh 
sklopljenih enačb, podobno~\textendash~a še bolj zapleteno~\textendash~kot v nalogi~(\ref{deplet}).

\begin{figure}[h]
\centering
\def\svgwidth{90truemm} 
\input{slike/08_opagraf.pdf_tex}
\caption{Normirani intenziteti ojačanega žarka in dodatnega nedejavnega žarka, ki nastane zaradi ohranitve
energije. Naraščajoči funkciji sta seveda samo približek, ki velja, dokler je ojačenje majhno in 
se intenziteta črpalnega žarka ne zmanjšuje znatno.}
\label{fig:opagraf}
\end{figure}

\begin{definition}
Pokaži, da sta izraza za amplitudi polji $A_1$ in $A_2$ (enačbi~\ref{eq:opa} in~\ref{eq:opan})
rešitvi sklopljenih enačb~(\ref{eq:opaA1} in \ref{eq:opaA2}) ob parametrih $A_{20}$ in $\kappa$,
kot sta zapisana v enačbah~(\ref{opakapa}) in (\ref{opakapaA}).
\end{definition}

Do zdaj smo vedno privzeli, da je izpolnjen pogoj ujemanja faz \index{Ujemanje faz}
in $\Delta k=k_{3}-k_{1}-k_{2}=0$. 
Ta pogoj lahko izpolnimo na enak način kot pri podvajanju frekvence: v dvolomnem kristalu 
izberemo ustrezne polarizacije in smer širjenja svetlobe glede na optično os, 
tako da velja $\omega_{3}n_{3}=\omega_{1}n_{1}+\omega_{2}n_{2}$.

Lahko na primer vzamemo izredno polarizacijo za črpalno valovanje
in redni polarizaciji za obe ojačevani valovanji, podobno kot pri
podvajanju frekvence. Tedaj mora biti izpolnjen naslednji pogoj 
\begin{equation}
\left[\left(\frac{\cos\vartheta_{m}}{n_{o}^{\omega_{3}}}\right)^{2}
+\left(\frac{\sin\vartheta_{m}}{n_{e}^{\omega_{3}}}\right)^{2}\right]^{-1/2}=
\frac{\omega_{1}}{\omega_{3}}n_{o}^{\omega_{1}}+\frac{\omega_{2}}{\omega_{3}}n_{o}^{\omega_{2}}.
\label{8.34}
\end{equation}

\begin{definition}
Obravnavali smo optično parametrično ojačevanje, ko je bil izpolnjen kriterij za ujemanje faz. 
Pokaži, v primeru neujemanja faz $\Delta k \neq 0$ amplitudi ojačevanega in dodatnega 
žarka naraščata kot \index{Neujemanje faz}
\begin{equation}
A_1 = A_{10} \left( \cosh(\kappa z) - \frac{i \Delta kz}{2 \kappa} \sinh (\kappa z) 
\right) e^{\frac{i \Delta kz}{2}}
\end{equation}
in
\begin{equation}
A_2 = A_{20} \sinh(\kappa z) e^{\frac{i \Delta k}{2}},
\end{equation}
pri čemer sta
\begin{equation}
\kappa^2 = \frac{\omega_1 \omega_2 \chi_{ef}^2|A_{30}|^2}
{16 c_0^2 n_1 n_2} - \frac{\Delta k^2}{4}
\end{equation}
in
\begin{equation}
A_{20} = i \sqrt{\frac{\omega_2 n_1}{\omega_1 n_2}} \sqrt{1 + \frac{\Delta k^2}{4 \kappa^2}}
~A_{10}.
\end{equation}
Hitro uvidimo, da so gornje enačbe v limitnem primeru, ko se faze ujamejo in je $\Delta k = 0$,
enake  enačbam~(\ref{eq:opa}, \ref{eq:opan} in \ref{opakapa}).
\end{definition}

Za konec ocenimo koeficient ojačenja v kristalu 
LiNbO$_{3}$\index{LiNbO$_3$}, v katerem želimo
ojačati svetlobo z valovno dolžino $\lambda = 1~\si{\micro\metre}$. Črpamo z laserjem z 
valovno dolžino okoli $500~\si{\nano\metre}$ in gostoto svetlobnega 
toka $5~\si{\mega\watt}/\si{\centi\metre}^{2}$. Lomni količnik snovi je 
$n = 2,2$, efektivna nelinearna susceptibilnost pa  $\chi_{ef} = 5~\si{\pico\metre}/\si{\volt}$. 
Vstavimo podatke v enačbo~(\ref{opakapa}) in izračunamo vrednost 
$\kappa \sim 0,15~/\si{\centi\metre}$. Porast intenzitete vpadne svetlobe v $1~\si{cm}$ 
dolgem kristalu je tako le približno $2~\%$. 

\subsection*{Optični parametrični oscilator (OPO)}
\index{Optični parametrični oscilator}
Gornji izračun kaže, da optično parametrično ojačevanje pri prehodu skozi kristal ni prav veliko
kljub dokaj močnemu črpalnemu žarku. Zato je smiselno, da svetloba večkrat preleti
ojačevalno sredstvo in se postopoma ojačuje. To naredimo tako, 
da optično ojačevalno sredstvo zapremo v optični 
resonator\index{Resonator!parametrični oscilator}
in signal se ob vsakem obhodu ojači. Sestavili smo t.\ i.\ optični parametrični oscilator
(slika~\ref{fig:opo}). 
\begin{figure}[h]
\centering
\def\svgwidth{80truemm} 
\input{slike/08_opo.pdf_tex}
\caption{Shematski prikaz tipičnega optičnega parametričnega oscilatorja. Ojačevalno sredstvo
zapremo med resonatorja, da se signalni žarek ($\omega_1$) ob vsakem preletu ojači.}
\label{fig:opo}
\end{figure}

V optičnemu resonatorju je odbojnost zrcal za črpalni žarek ($\omega_3$) zelo majhna, 
odbojnost za ojačani žarek pa blizu ena. Valovanje pri $\omega_1$,
ki se v parametričnem oscilatorju ojačuje, nastane spontano, prav tako valovanje pri 
$\omega_2 = \omega_3 -\omega_1$. Njuni frekvenci sta dodatno določeni s pogojem za 
ujemanje faz $ k_3 - k_1 - k_2 = 0$, 
hkrati pa mora ojačevano nihanje sovpadati z lastnim nihanjem resonatorja. 
S sukanjem ojačevalnega kristala lahko na ta način spreminjamo
ojačano frekvenco in naredili smo nastavljiv izvor svetlobe, navadno infrardeče.\index{Infrardeče valovanje}

Za delovanje oscilatorja mora biti jakost črpalnega žarka tako velika, da je 
ojačenje na obhod večje od izgub. Izračunajmo za primer zgoraj narisanega 
oscilatorja. Signal z močjo 
$P_0$ se ob prehodu skozi ojačevalno sredstvo ojača (enačba~\ref{eq:opa})
\begin{equation}
P_1 = P_0 \cosh^2 (\kappa L),
\end{equation}
hkrati pa se zaradi izhodnega zrcala z odbojnostjo $\mathcal{R}$ in notranjih izgub $\Lambda_0$ 
intenziteta žarka zmanjšuje. Pogoj ujemanja faz je izpolnjen le v eni smeri in se svetloba
ojačuje le enkrat na celoten obhod. Ob preletu v drugo smer je namreč $\Delta k \neq 0$ in žarek se ne ojačuje. V stacionarnem stanju je moč 
žarka po obhodu $P_2$ enaka začetni moči $P_0$ in ojačenje je ravno enako izgubam 
\begin{equation}
P_2 = P_1\,(1-\Lambda_0)\mathcal{R} = P_0 \,(1-\Lambda_0) \mathcal{R} \,\cosh^2 (\kappa L) = P_0
\end{equation}
oziroma
\begin{equation}
\cosh^2 (\kappa L) =\frac{1}{(1-\Lambda_0)\, \mathcal{R}}.
\end{equation}
Iz gornjega pogoja določimo parameter $\kappa$, po enačbi~(\ref{opakapa}) pa mejno 
amplitudo in intenziteto črpalnega žarka. Nadaljujmo še prejšnji primer ojačenja 
svetlobe v $1~\si{cm}$ dolgem kristalu LiNbO$_{3}$.\index{LiNbO$_3$}
Če je odbojnost izhodnega zrcala $\mathcal{R}=0,85$, notranje izgube $\Lambda_0 = 0,05$ in prečni presek 
žarka $10~\si{\micro\metre^2}$, je moč praga $P_{\omega_3} = 5~\si{\watt}$.

\begin{remark}
Optični parametrični oscilator oddaja svetlobo, podobno kot laser. Tudi sicer
sta si do neke mere podobna: oba sistema potrebujeta močen črpalni mehanizem, oba sistema
sta sestavljena iz resonatorja, v katerem se žarek velikokrat odbije in postopoma ojačuje,
in oba oddajata koherentno svetlobo pri točno določeni valovni dolžini. Vendar
je med parametričnim oscilatorjem in laserjem velika razlika. Pri laserju pride do
ojačenja svetlobe zaradi obrnjene zasedenosti stanj, pri oscilatorju pa 
zaradi nelinearnega optičnega pojava. Ker pri oscilatorju energija ni shranjena v
snovi, ampak se signal ojačuje sproti, je zelo pomembno, da sunek črpalnega laserja vpade
na kristal istočasno kot ojačevan žarek. Velika prednost oscilatorjev pred laserji 
je zvezno nastavljiva frekvenca delovanja v zelo širokem frekvenčnem območju, saj ni določena
s prehodom med nivoji, ampak z izpolnjevanjem pogoja za ujemanje faz.
\end{remark}

\section{Optično usmerjanje in teraherčno valovanje}
\index{Optično usmerjanje}
\index{Teraherčno valovanje}
Ko smo obravnavali nelinearne optične pojave drugega reda, smo zapisali
različne frekvence, ki so vsebovane v izhodnem signalu (slika~\ref{fig:nl2}). Eno izmed
izhodnih valovanj ima tudi frekvenco enako nič, kar pomeni, da je to statično električno polje. Iz analogije
z elektronskimi vezji, kjer izmenično napetost z usmernikom spremenimo v enosmerno napetost, 
pojav imenujemo optično usmerjanje, saj iz svetlobnega valovanja nastane statično polje. Tako statično 
polje navadno ni veliko, saj sunek svetlobe z vršno močjo nekaj $\si{\mega\watt}$ tipično povzroči 
nekaj deset $\si{\milli\volt}$ napetosti v smeri prečno na smer potovanja svetlobe. 

\begin{definition}
Pokaži, da je napetost, ki se pojavi pri optičnem usmerjanju, približno enaka
\begin{equation}
U = \frac{\chi P_0}{n^3 \varepsilon_0 c_0 a},
\end{equation}
pri čemer je $P_0$ moč vpadne svetlobe, $n$ lomni količnik snovi in $a$ širina kristala.\\
Namig: nelinearen kristal obravnavaj kot ploščati kondenzator. 

Oceni še napetost, če je
$\chi = 3~\si{\pico\volt/\meter}, P_0 = 1~\si{\mega\watt}, n = 2,2$ in $a = 5~\si{\milli\metre}$. 
\end{definition}

Precej bolj uporaben je pojav, ko na nelinearen kristal posvetimo z ultrakratkimi 
sunki svetlobe, tipično okoli $\si{ps}$ ali krajšimi. Spomnimo se, da je povsem 
monokromatsko valovanje lahko samo tako, ki je časovno neomejeno in ima neskončen koherenčni čas
(enačba~\ref{eq:spektralna-sirina-zveza}). 
Čim je valovanje časovno omejeno, ima njegov spekter končno širino, pri čemer 
imajo krajši sunki svetlobe širši spekter valovanja. 

Ko z ultrakratkim sunkom osvetlimo optično 
nelinearen kristal, v kristal vstopajo vse frekvence z danega intervala $\omega \pm \Delta \omega/2$.
Optično usmerjanje ni več popolno, saj se frekvence ne odštejejo povsem, ampak se 
namesto statičnega polja pojavi sunek svetlobe s širokim spektrom, ki sega od ničelne
frekvence do neke največje vrednosti. Celotna spektralna širina tega signala je 
približno enaka spektralni širini vstopnega sunka, ta pa je obratno sorazmerna z njegovo dolžino.
Ocenimo te vrednosti še numerično. 

\begin{figure}[h]
\centering
\def\svgwidth{110truemm} 
\input{slike/08_THz.pdf_tex}
\caption{Shematski prikaz nastanka teraherčnega valovanja v optično nelinearnem sredstvu}
\label{fig:THz}
\end{figure}

Vzemimo kratek sunek svetlobe dolžine $\tau$ s spektralno širino
$\Delta \omega = 2/\tau$.
Če je sunek svetlobe dolg $1~\si{\pico\second}$, je razlika v frekvencah 
spektra 
\begin{equation}
\Delta \omega = \frac{2}{10^{-12}~\si{\hertz}} = 2~\si{\tera\hertz}.
\end{equation}
Valovanje, ki nastane pri takem optičnem kvazi-usmerjanju, ima torej frekvence v teraherčnem
področju in naredili smo izvor teraherčnega valovanja. 

Teraherčno valovanje, to je 
elektromagnetno valovanje s frekvencami v območju od 0,3 do $3~\si{\tera\hertz}$
oziroma z valovnimi dolžinami med 0,1 in $1~\si{\milli\metre}$, 
se uporablja za neinvazivno slikanje in preiskave tkiv in materialov. Kristali, ki 
se najpogosteje uporabljajo za nastanek teraherčnega valovanja, so ZnTe\index{ZnTe}, 
GaP, GaSe in GaAs.\index{GaP} \index{GaSe} \index{GaAs}

\section{Nelinearni pojavi tretjega reda}
\index{Nelinearna optika!tretjega reda}
Doslej smo obravnavali najnižji red nelinearnosti, katerega glavni
učinek je mešanje treh frekvenc, na primer optično frekvenčno podvajanje ali
optično parametrično ojačevanje. Ti pojavi so možni le v kristalih brez centra
inverzije. Naslednji člen razvoja nelinearne polarizacije po električnem
polju obstaja v vsaki snovi. V njem nastopa polje v tretji potenci
\boxeq{eq:nl3P}{
\mathbf{P}_{\mathrm{NL,3}}= \epsilon_{0}\chi^{(3)}\vdots \mathbin 
\mathbf{E}\mathbin \mathbf{E}\mathbin\mathbf{E}
}
oziroma izpisano po komponentah
\begin{equation}
\left(\mathbf{P}_{\mathrm{NL,3}}\right)_i= \epsilon_{0}\chi^{(3)}_{ijkl} \,E_j \,E_k\, E_l.
\end{equation}
Pri tem je $\chi^{(3)}$ tenzor četrtega ranga, njegova tipična velikost je okoli 
$10^{-22}~\si{\metre^2/\volt^2}$. V splošnem ima 81 različnih neodvisnih komponent, to
število pa se lahko zelo zmanjša zaradi simetrije snovi. V izotropni snovi je tako
21 neničelnih elementov, od katerih so le trije neodvisni. 

Če vsebuje vpadno polje le eno frekvenco, se zaradi nelinearnosti tretjega
reda pojavi polarizacija pri 3$\omega$ in $\omega$. Pri dveh vpadnih
frekvencah $\omega_{1}$ in $\omega_{2}$ so možne kombinacije $2\omega_{1}\pm\omega_{2}$
in $\omega_{1}\pm2\omega_{2}$, pri treh vpadnih frekvencah pa vse
možne vsote in razlike frekvenc, to so $\omega_1$, $\omega_2$, $\omega_3$, 
$3\omega_1$, $3 \omega_2$, $3\omega_3$, 
$\omega_1 + \omega_2 + \omega_3$, $\omega_1 + \omega_2 - \omega_3$, 
$\omega_1 - \omega_2 + \omega_3$, $- \omega_1 + \omega_2 + \omega_3$, 
$2 \omega_1\pm\omega_2$, $2 \omega_1\pm\omega_3$, $2 \omega_2\pm\omega_1$,
$2 \omega_2\pm\omega_3$, $2 \omega_3\pm\omega_1$, $2 \omega_3\pm\omega_2$.
Možnosti je torej precej več kot pri nelinearnosti drugega reda in računi so zato 
v splošnem precej bolj zapleteni.

Obravnava nastanka valovanja pri
kombinaciji frekvenc je zelo podobna obravnavi podvajanja frekvence in parametričnemu
ojačevanju. V enačbah za nastanek novega valovanja ali ojačevanje
katerega od vpadnih snopov spet nastopi fazni faktor, ki vsebuje razliko
vseh valovnih vektorjev $\Delta{\bf k}$. Da je intenziteta novega
valovanja znatna, mora biti $\Delta kL\simeq0$, spet mora biti torej
izpolnjen pogoj ujemanja faz. Ker v tem primeru nastopajo v splošnem štirje
valovni vektorji, je seveda tudi pri izbiri geometrije in polarizacij
za ujemanje faz precej več možnosti.\index{Ujemanje faz}

Omejimo se na najpreprostejši primer, pri katerem ima vpadno valovanje le eno 
frekvenco. Takrat se pojavi valovanje pri potrojeni frekvenci, pa tudi
pri frekvenci, ki je enaka vpadni. Pojavi se torej polarizacija pri 
vpadni frekvenci, ki spremeni obnašanje osnovnega žarka, in žarek vpliva sam nase.
Ti pojavi, ki jih poimenujemo s predpono {\it samo-}, kot na primer samozbiranje, so
značilni za nelinearne pojave tretjega reda\index{Samozbiranje}. 

\section{Optični Kerrov pojav}
\index{Optični Kerrov pojav|see Kerrov pojav!optični}
\index{Kerrov pojav!optični}
\label{OKP}
Naj valovanje vpada na nelinearno snov, za katero velja $\chi^{(2)} = 0$.
Polarizacija je potem enaka vsoti linearnega in nelinearnega dela tretjega reda 
(enačba~\ref{8.1})\index{Električna polarizacija}
\begin{equation}
\mathbf{P}=
\epsilon_{0} \chi^{(1)}\cdot \mathbf{E}+
\epsilon_{0}\chi^{(3)}\vdots \mathbin \mathbf{E}\mathbin \mathbf{E}\mathbin\mathbf{E}.
\end{equation}
Ker obravnavamo nelinearne pojave, moramo tudi v tem primeru zapisati realna
električna polja. To naredimo z vsoto dveh kompleksno konjugiranih členov
\begin{equation}
\mathbf{E}=\frac{\mathbf{e}}{2}(Ae^{i(kz-\omega t)}+A^{*}e^{-i(kz-\omega t)}).
\label{8.71}
\end{equation}
Podobno zapišemo tudi polarizacijo, pri čemer nas bodo zanimali samo členi,
ki nihajo s frekvenco $\omega$
\begin{equation}
\mathbf{P}=\frac{\mathbf{e}}{2}(P_\omega e^{i(kz-\omega t)}+P_\omega^{*}e^{-i(kz-\omega t)}).
\label{8.71a}
\end{equation}
Ti členi nastopijo v primeru, ko
v izrazu $\mathbf{E}\mathbin \mathbf{E}\mathbin\mathbf{E}$ 
vzamemo dvakrat nekonjugirani del, enkrat pa konjugiranega. To lahko naredimo na tri
možne načine, zato nastopajo trije enaki členi. Sledi
\begin{equation}
\frac{\mathbf{e}}{2}P_{\omega,\mathrm{NL}} = 3 \frac{1}{8} A A^* \left( 
\varepsilon_0 \chi^{(3)}\vdots \mathbin \mathbf{e}\mathbin \mathbf{e} \mathbin \mathbf{e} \right) A.
\label{pomega}
\end{equation}
Celotna polarizacija je
\begin{equation}
\mathbf{P}=
\epsilon_{0} \chi^{(1)}\cdot \mathbf{E}+\frac{3}{4} |A|^2 \left( 
\varepsilon_0 \chi^{(3)}\vdots \mathbin \mathbf{e}\mathbin \mathbf{e} \right) \mathbf{E}.
\label{eq:ptnl}
\end{equation}
Z upoštevanjem zveze med amplitudo polja in povprečno gostoto energijskega toka (enačba~\ref{eq:j})
zapišemo
\begin{equation}
\mathbf{P}=
\epsilon_{0} \left( \chi^{(1)} +\frac{3}{4} \frac{2  j }
{\varepsilon_0 \tilde{n} c_0} \chi^{(3)}\vdots \mathbin \mathbf{e}\mathbin 
\mathbf{e} \right) \mathbf{E}.
\end{equation}
Z $\tilde{n}$ smo označili lomni količnik pri frekvenci $\omega$. 
Faktor v oklepaju ni nič drugega kot efektivna susceptibilnost, ki je neposredno povezana
z lomnim količnikom snovi $\chi_{ef} = \varepsilon -1 =n^2 -1$. Gornja enačba torej opisuje pojav, 
pri katerem vpadna svetloba vpliva na lomni količnik snovi.\index{Susceptibilnost!nelinearna, efektivna}
Gre za podoben učinek kot pri navadnem Kerrovem pojavu, pri katerem se lomni količnik 
spremeni pod vplivom zunanjega električnega polja (enačba~\ref{7.1}).
Opisani optični ekvivalent zato imenujemo optični Kerrov pojav\footnote{Škotski fizik John Kerr, 1824\textendash1907.}. 

Poglejmo pojav podrobneje na primeru izotropne snovi. Na snov naj vpada valovanje, ki je polarizirano
v smeri $x$, tako da ima nelinearna polarizacija le komponento 
\begin{equation}
P_{\mathrm{NL},x}=
\epsilon_{0} \left(\chi_{xx} +\frac{3}{4} \chi_{xxxx}\frac{2 j }
{\varepsilon_0 \tilde{n} c_0}\right)E = \varepsilon_0 \chi_{ef}E = \varepsilon_0 (n^2-1) E.
\end{equation}
Izrazimo še lomni količnik
\begin{equation}
n \approx \tilde{n} + \frac{3 \chi_{xxxx}}{4 \varepsilon_0 c_0 \tilde{n}^2} j.
\end{equation}
Efektivni lomni količnik v snovi lahko zapišemo kot 
\boxeq{eq:nnl}{
n= \tilde{n} + n_2 j,}\index{Lomni količnik!efektivni}
pri čemer smo vpeljali nelinearni lomni količnik\index{Lomni količnik!nelinearni}
\boxeq{eq:n2}{
n_2 = \frac{3 \chi_{xxxx}}{4 \varepsilon_0 c_0 \tilde{n}^2}.
}
Efektivni lomni količnik snovi je torej odvisen od intenzitete svetlobe, ki vpada nanjo. 
Tipične vrednosti nelinearnega lomnega količnika za vidno svetlobo so $10^{-20}~\si{\metre^2/\watt}$.
V tekočini CS$_2$ je $n_2 = 3,2 \cdot 10^{-18}~\si{\metre^2/\watt}$, v nekaterih \index{CS$_2$}
drugih snoveh (npr. polprevodnikih) je lahko vrednost $n_2$ večja še za več 
velikostnih redov, $n_2$ pa je lahko tudi negativen. 

\begin{table}[h]
 \centering
\begin{tabular}{|c|c|c|} \hline  
      Snov & $\chi^{(3)}~(\si{\metre^2/\volt^2})$ & $n_2~(\si{\metre^2/\watt})$\\ \hline
     steklo BK7 & 2,8 $\times 10^{-22}$ & $3,2 \times 10^{-20}$ \\ \hline
     voda & $2,5 \times 10^{-22}$ & $4,1 \times 10^{-20}$ \\ \hline
     GaAs & $1,4 \times 10^{-18}$ & $3,3 \times 10^{-17}$ \\ \hline\index{GaAs}
     ZnSe & $6,2 \times 10^{-20}$ & $3,0 \times 10^{-18}$ \\ \hline\index{ZnSe}
     CS$_2$ & $3,1 \times 10^{-20}$ & $3,2 \times 10^{-18}$ \\ \hline \index{CS$_2$}
     polimer 4BCMU  & $-1,2 \times 10^{-19}$ & $-1,5 \times 10^{-17}$ \\ \hline      
\end{tabular}
  \caption{Nelinearna susceptibilnost tretjega reda in nelinearni lomni količnik za nekaj izbranih snovi}
\label{table:chi3}
\end{table}

Zanimivi posledici lomnega količnika, odvisnega od intenzitete svetlobe, 
sta samozbiranje svetlobnega snopa in širjenje solitonov po optičnih vodnikih, 
kar si bomo pogledali v naslednjih
razdelkih.

\begin{remark}
Ničesar nismo povedali o ujemanju faz, ki je sicer nujno potrebno za učinkovite nelinearne 
optične pojave. V tem primeru vpada na snov en sam laserski žarek in pogoj ujemanja faz
je vedno izpolnjen. 
\end{remark}

\section{Samozbiranje in krajevni solitoni}
\index{Samozbiranje}
\index{Soliton!krajevni}
Za začetek si oglejmo pojav samozbiranja svetlobe. Osnovni Gaussov snop 
(enačba~\ref{eq:gaussov-snop}) naj vpada na sredstvo,
v katerem je lomni količnik odvisen od intenzitete vpadne svetlobe po enačbi~(\ref{eq:nnl}).
Naj bo $n_{2}>0$, tako da je lomni količnik v sredini snopa večji 
od nemotenega lomnega količnika na robu. V osi snopa se optična pot 
zaradi optično gostejšega sredstva podaljša in valovna fronta 
v osi zaostaja glede na fronte na robu snopa. Če je zaostajanje dovolj veliko,
lahko krivinski radij valovne fronte postane negativen in snop se
ne širi, temveč oži (slika~\ref{fig:sf1}). Temu pojavu pravimo 
samozbiranje. Samozbiranje je pri dovolj
veliki moči snopa lahko tako veliko, da pride do katastrofične zožitve snopa
in s tem do tolikšnega povečanja gostote svetlobnega toka, da nastanejo
poškodbe v snovi.
\begin{figure}[h]
\centering
\def\svgwidth{100truemm} 
\input{slike/08_sf1.pdf_tex}
\caption{V Gaussovem snopu je intenziteta valovanja odvisna od prečne koordinate $r$, 
zato je tudi lomni količnik nelinearnega sredstva odvisen od nje. To vodi do 
 samozbiranja svetlobe. Na sliki so fronte vpadnega Gaussovega snopa narisane kot ravni valovi.}
\label{fig:sf1}
\end{figure}

\begin{definition}
Gaussov snop svetlobe z močjo $P$ in polmerom $w$ naj vpada pravokotno na ploščico
kristala debeline $d$. Pokaži, da ploščica deluje na snop kot leča z goriščno razdaljo 
\begin{equation}
f = \frac{\pi w^4}{8 n_2 d P},
\end{equation}
pri čemer je $n_2$ nelinearni lomni količnik.
\end{definition}

Zaradi uklona se Gaussov snop širi, pojav samozbiranja pa ima ravno nasprotni
učinek. Snop samemu sebi ustvarja valovni vodnik, v katerem je v sredi lomni količnik 
večji kot na robu. Pri določeni moči snopa je tako možno doseči, da se oba pojava po
velikosti ravno izenačita. Snop, ki se širi po snovi, ima tako konstanten polmer, 
valovne fronte pa so ravne -- nastane t.\,i. krajevni soliton. 
\index{Soliton!krajevni}

Ocenimo, pri kateri moči vpadne svetlobe pride do pojava krajevnih solitonov. 
Vzemimo, da je na izbranem mestu valovna fronta ravna. Lahko si mislimo,
da je tam grlo Gaussovega snopa. Brez samozbiranja bi bil na razdalji
dolžine grla $z_{0}$ krivinski radij valovne fronte (enačba~\ref{eq:R})
\begin{equation}
R(z_{0})=z_{0}\left( 1+\left(\frac{z_{0}}{z_{0}}\right)^{2}\right)=2z_{0}.
\label{8.75}
\end{equation}
V bližini osi lahko Gaussovo funkcijo, ki opisuje prečno odvisnost
amplitude polja v snopu, razvijemo po prečni koordinati $r$ do drugega
reda. Po enačbi~(\ref{eq:nnl}) je odvisnost lomnega količnika približno
\begin{equation}
n(r)=\tilde{n}+n_2 j_0 e^{-2r^2/w_0^2} \approx \tilde{n}+n_2 j_0 \left(1 - 2\frac{r^2}{w_0^2}\right).
\label{8.76}
\end{equation}
Razlika med lomnim količnikom na osi in pri $w_{0}$ od osi je $\Delta n= 2j_{0} n_{2}$.
Zaradi tega je na poti od grla do $z_0$ razlika optičnih poti med žarkoma na osi $(r=0)$ in 
pri $r= w_{0}$ enaka $\Delta nz_{0} = 2 n_2 j_0 z_0$ in valovna fronta se 
ukrivi na nek krivinski radij $-R$. Iz preproste geometrije velja zveza 
\begin{equation}
\Delta nz_{0}=R-R\sqrt{1-\frac{w_{0}^{2}}{R^{2}}}\approx \frac{w_{0}^{2}}{2R}
\label{8.77}
\end{equation}
Da valovna fronta ostane ravna, se morata krivinska radija zaradi uklona 
(enačba~\ref{8.75}) in samozbiranja (enačba~\ref{8.77}) ravno izenačiti. 
Od tod sledi 
\begin{equation}
\Delta n=\frac{w_{0}^{2}}{4z_{0}^{2}}.
\label{8.78}
\end{equation}
Moč snopa, pri katerem se polmer ne spreminja, je potem 
\begin{equation}
P_{s}= \frac{1}{2}\pi w_0^2 \,j_0 = \frac{1}{2}\pi w_0^2 \, \frac{\Delta n}{2 n_2} = 
\frac{1}{2}\pi w_0^2 \,\frac{w_{0}^{2}}{4z_{0}^{2}}\,\frac{1}{2 n_2} = \frac{\lambda^2}{16 \pi n_2},
\label{8.79}
\end{equation}
pri čemer smo upoštevali zvezo med $z_0$ in $w_0$ (enačba~\ref{eq:z0}).

Pri moči, ki je manjša od kritične $P_s$, se vpadli Gaussov snop širi, 
čeprav nekoliko počasneje kot v sredstvu s konstantnim lomnim količnikom. 
Če pa je moč znatno večja od kritične moči, lahko
pride do katastrofičnega samozbiranja in porušitve snovi.
Zanimivo je, da kritična moč, pri kateri se pojavijo solitoni, 
ni odvisna od začetnega polmera snopa.

\begin{definition}
Nariši skico k enačbi~(\ref{8.77}) in izpelji izraz za moč, pri kateri pride do pojava
solitonov (enačba~\ref{8.79}). 

Izračunaj še kritično moč za pojav solitonov v CS$_{2}$,\index{CS$_2$}
če je valovna dolžina vpadnega valovanja $1~\si{\micro\metre}$, 
nelinearni lomni količnik te tekočine pa je 
 $n_{2}=3,2 \cdot 10^{-18}~\si{\metre^2/\watt}$. 
\end{definition}

\begin{remark}
Eksperimentalna metoda, \index{Metoda vzdolžnega premika}
\index{Z-scan|see {Metoda vzdolžnega premika}}s katero merimo nelinearni 
lomni količnik, je tako imenovana
metoda vzdolžnega premika ({\it Z-scan}). Optično nelinearno sredstvo (naj ima $n_2>0$)
postavimo v zožan laserski snop (slika~\ref{fig:zscan}). 
Zaradi samozbiranja deluje vzorec kot leča, njena goriščna razdalja
pa je odvisna od intenzitete snopa in od nelinearnega lomnega količnika. Ko vzorec 
premikamo vzdolž snopa, se skupna efektivna goriščna razdalja leče in nelinearne snovi 
spreminja in žarek na detektorju je enkrat bolj zbran, drugič manj. 
Za lege vzorca desno od prvotnega gorišča ($z>0$), je skupna goriščna
razdalja daljša od goriščne razdalje leče, snop je bolj zbran (pikčasta črta) in signal 
na detektorju (D) naraste. Za lege vzorca levo
od prvotnega gorišča ($z<0$) je ravno obratno, snop se razširi (črtkana črta) in 
signal na detektorju se zmanjša. Za snovi z negativnim nelinearnim lomnim količnikom
je odziv ravno nasprotnega predznaka. Pri določanju nelinearnega lomnega količnika je
ključno uporabiti zaslonko (Z), s katero omejimo premer vpadnega snopa pred detektorjem. 
Če zaslonko odstranimo in merimo 
odvisnost celotne vpadne intenzitete od lege vzorca, nelinearnega lomnega količnika 
ne moremo meriti, lahko pa določimo nelinearni absorpcijski koeficient. 

\begin{figure}[h!]
\raggedleft 
\def\svgwidth{130truemm} 
\input{slike/08_zscan.pdf_tex}
\caption{Shema metode vzdolžnega premika}
\label{fig:zscan}
\end{figure}
\end{remark}

\section{*Izpeljava krajevnih solitonov}
\index{Soliton!krajevni}
\label{chap:ks}
Za podrobnejšo obravnavo krajevnih solitonov moramo rešiti valovno 
enačbo v obosnem približku. Začnimo s krajevnim delom valovne 
enačbe za monokromatsko valovanje v skalarni obliki (enačba~\ref{eq:Helmholtz})
\begin{equation}
\nabla^{2}E+n^{2}\frac{\omega^{2}}{c_0^{2}}E=0.
\label{8.80}
\end{equation}
Polje zapišemo v obliki počasi spreminjajoče se amplitude in faznega faktorja, podobno kot 
smo to naredili pri izpeljavi Gaussovega snopa (enačba~\ref{eq:ravni-val-nastavek})
\begin{equation}
E=\psi(\mathrm{r},z)e^{ik_{0}z},
\label{8.81}
\end{equation}
 kjer je $k_{0}=\tilde{n}\omega/c_0$ valovno število brez upoštevanja nelinearnosti.
Funkcija $\psi(\mathbf{r},z)$ naj se v smeri osi $z$ le počasi spreminja, tako da lahko
drugi odvod po $z$ zanemarimo in zapišemo \index{Obosna valovna enačba}
\begin{equation}
\nabla_{\bot}^{2}\psi+\frac{\omega^{2}}{c_0^{2}}(n^{2}-\tilde{n}^{2})\psi+2ik_{0}
\frac{\partial\psi}{\partial z}=0.
\label{8.82}
\end{equation}
Upoštevamo odvisnost lomnega količnika od intenzitete, pri čemer
zanemarimo člen z $n_{2}^{2}$, ker je gotovo majhen. Sledi
\begin{equation}
\nabla_{\bot}^{2}\psi+2k_{0}^{2}\frac{n_{2}}{\tilde{n}}j\psi+2ik_{0}\frac{\partial\psi}{\partial z}=0.
\label{8.83}
\end{equation}
Izrazimo še gostoto svetlobnega toka z amplitudo električne poljske jakosti 
\begin{equation}
\nabla_{\bot}^{2}\psi+
k_{0}^{2} n_2 \varepsilon_0 c_0 |\psi|^2 \psi+
2ik_{0}\frac{\partial\psi}{\partial z}=0.
\label{8.83a}
\end{equation}
Preden se lotimo reševanja gornje enačbe, vpeljimo še
\begin{equation}
\kappa=k_{0}^{2} n_2 \varepsilon_0 c_0
\end{equation}
 in novo spremenljivko vzdolž osi $z$
\begin{equation}
\zeta=\frac{z}{2k_{0}}.
\end{equation}
 S tem preide enačba (\ref{8.83a}) v standardno obliko nelinearne Schr\"odingerjeve
enačbe, le da namesto odvoda po času tukaj nastopa odvod po koordinati $\zeta$. Sledi
\index{Schr\"odingerjeva enačba!nelinearna}
\boxeq{8.84}{
i\frac{\partial\psi}{\partial\zeta}+\nabla_{\bot}^{2}\psi+\kappa\left|\psi\right|^{2}\psi=0.
}
V treh dimenzijah je reševanje enačbe (\ref{8.84}) težavno in analitične
rešitve niso znane. V dveh dimenzijah pa stacionarno rešitev znamo
poiskati. Stacionarni rešitvi se vzdolž $\zeta$ lahko spreminja le faza, zato
rešitev iščemo v obliki 
\begin{equation}
\psi=e^{i\eta^{2}\zeta}\, u(x),
\label{8.87}
\end{equation}
 kjer je $\eta$ konstanta, katere pomen bomo videli v nadaljevanju, 
 funkcija $u(x)$ pa naj bo realna. 
Uporabimo gornji nastavek v enačbi (\ref{8.84}) in dobimo
\begin{equation}
\frac{d^{2}u}{dx^{2}}=\eta^{2}u-\kappa u^{3}.
\end{equation}
 Z množenjem obeh strani z $u^{\prime}$ lahko enačbo enkrat integriramo
\begin{equation}
\left(\frac{du}{dx}\right)^{2}=\eta^{2}u^{2}-\frac{1}{2}\kappa u^{4}.
\end{equation}
Ločimo spremenljivki in zapišemo 
\begin{equation}
\int_{\eta\sqrt{2/\kappa}}^{u}\frac{du}{\sqrt{\eta^{2}u^2-\frac{1}{2}\kappa u^{4}}}=x-x_{0},
\label{8.85}
\end{equation}
pri čemer smo uvedli integracijsko konstanto $x_{0}$ in integracijsko mejo postavili 
tako, da so vrednosti pod korenom pozitivne.
Integral brez težav izračunamo
\begin{equation}
\frac{1}{\eta}\ln\left(\sqrt{\frac{\kappa}{2}}\frac{u}{\eta+
\sqrt{\eta^{2}-\kappa u^{2}/2}}\right)=x-x_{0}
\end{equation}
in izrazimo iskano funkcijo $u(x)$
\begin{equation}
u=\sqrt{\frac{2}{\kappa}}\frac{2 \eta }{e^{\eta(x-x_{0})}+e^{-\eta(x-x_{0})}}=
\sqrt{\frac{2}{\kappa}}\frac{\eta}{\cosh\eta(x-x_{0})}.
\label{8.86}
\end{equation}
Po enačbi (\ref{8.87}) je rešitev
\begin{equation}
\psi(x,z)=\sqrt{\frac{2}{\kappa}}\,\eta\,\frac{e^{i\eta^{2}\zeta}}{\cosh\eta(x-x_{0})}.
\label{8.88}
\end{equation}
Vidimo, da predstavlja spremenljivka $1/\eta$  karakteristično širino snopa, $x_{0}$ pa
je le njegov prečni premik, ki ga lahko brez škode postavimo na $x_0=0$. Tako
lahko zapišemo celotno polje stacionarnega snopa 
\begin{equation}
E_{s}(x,z)=\sqrt{\frac{2}{\kappa}}\,\frac{\eta}{\cosh(\eta x)}\,\exp\left(ik_{0}z\left(1+
\frac{\eta^{2}}{2k_{0}^{2}}\right)\right).
\label{8.89}
\end{equation}
Intenziteta svetlobe je neodvisna od koordinate $z$ in je sorazmerna kvadratu amplitude polja 
\boxeq{8.89a}{
j_{s}(x,z)= j_0 \frac{1}{\cosh^2(\eta x)}.
}\index{Gostota energijskega toka}
\begin{figure}[h]
\centering
\def\svgwidth{100truemm} 
\input{slike/08_soliton.pdf_tex}
\caption{Prečni profil krajevnega solitona v 2D. Vzdolž koordinate $z$ se profil ohranja.}
\label{fig:soliton}
\end{figure}

Če se vrnemo k izrazu za električno poljsko jakost (enačba~\ref{8.89}), vidimo, da
parameter $\eta$ nastopa tudi v faznem faktorju. To pomeni, da je od njega odvisna 
tudi konstanta širjenja in s tem fazna hitrost
\index{Hitrost valovanja!solitonov}
\begin{equation}
v_{f}=\frac{\omega}{k} = \frac{c}{\tilde{n}\left(1+\frac{\eta^{2}}{2k_{0}^{2}}\right)}.
\end{equation}
Fazna hitrost omejenih snopov oziroma solitonov je torej vedno manjša od fazne hitrosti ravnih valov. 
Bolj ko je snop omejen, manjša je fazna hitrost, za velike polmere snopa pa doseže 
limitno vrednost $c_0/\tilde{n}$.

Moč dvodimenzionalnega snopa je enaka integralu
gostote svetlobnega toka (enačba~\ref{8.89a}) po $x$. Integriramo in zapišemo 
\begin{equation}
P_s = \int j_s dx \propto \int |E_s|^2 dx  = 
\frac{2}{\kappa}\,\eta^{2}\int_{-\infty}^{\infty}\frac{dx}
{\cosh^{2}\eta x}=\frac{4\eta}{\kappa}.
\label{eq:solj}
\end{equation}
Moč stacionarnega snopa~\textendash~solitona~\textendash~v dveh dimenzijah je 
torej obratno sorazmerna s širino snopa $1/\eta$. Zato tudi pri poljubno veliki moči 
obstaja stacionarna širina. To je bistvena razlika med obravnavanim dvo- in 
tridimenzionalnim primerom, kjer se snop z nadkritično močjo skrči v singularnost.

\section{Optični solitoni}
\index{Soliton!optični}
V prejšnjem razdelku smo ugotovili, da pojav samozbiranja lahko izniči širjenje 
svetlobnega snopa zaradi uklona, tako da ima pri
ustrezni moči snop vzdolž smeri širjenja konstantno širino in obliko. Takim snopom 
smo rekli krajevni solitoni. Povsem podoben pojav poznamo tudi v časovni 
domeni, kjer se pojavijo časovni ali optični solitoni. 

Sunek svetlobe  naj se širi po valovnem vodniku. Zaradi disperzije je lomni količnik\index{Disperzija}
odvisen od frekvence valovanja in sunek svetlobe se med potovanjem po vodniku podaljšuje. 
Več o tem bomo spoznali pri 
obravnavi disperzije v optičnih vlaknih (poglavje~\ref{chap:Disperzija}). 
Ob primernih pogojih lahko nelinearna odvisnost lomnega količnika $n(j)$ 
ravno izniči disperzijo $n(\lambda)$  in sunek
ohranja obliko. Sunkom svetlobe, ki potujejo po sredstvu brez spremembe
oblike, pravimo optični solitoni. Posebej so pomembni v optičnih vlaknih, 
kjer želimo  vpliv disperzije zaradi učinkovitosti prenosa
informacije kar se da zmanjšati. 

Pojava optičnih solitonov ni težko pojasniti. Naj na optično nelinearno sredstvo
vpade sunek svetlobe, ki je Gaussove oblike v času
\begin{equation}
j(t) = j_0 e^{-2t^2/\tau^2}.
\label{08_pulz}
\end{equation}
Faza takega sunka je 
\begin{equation}
\phi (t) = k_o n z - \omega_0 t = k_0 (\tilde{n} + n_2 j)z - \omega_0 t = 
\phi_0 + k_0 n_2 z j - \omega_0 t,
\end{equation}
frekvenca pa 
\begin{equation}
\omega = -\frac{d\phi}{dt} = \omega_0 - k_0 n_2 z \frac{dj}{dt}.
\end{equation}
Če vstavimo časovno obliko sunka svetlobe (enačba~\ref{08_pulz}), vidimo, da se 
frekvenca takega sunka spreminja s časom
\begin{equation}
\omega = \omega_0 + \frac{4k_0 n_2 z j_0}{\tau^2} \, t \, e^{-2t^2/\tau^2}.
\end{equation}
Začetnemu delu sunka (pri $t<0$) se frekvenca zmanjša, zadnjemu delu sunka
(pri $t>0$) pa se poveča (slika~\ref{fig:optsoliton}). 
Ta pojav spreminjanja frekvence znotraj kratkega sunka imenujemo čričkanje 
({\it chirping}), \index{Čričkanje} po podobnosti z oglašanjem čričkov.

\begin{figure}[h]
\centering
\def\svgwidth{80truemm} 
\input{slike/08_OpticniSoliton.pdf_tex}
\caption{Zaradi nelinearnega lomnega količnika pride do frekvenčnega premika v sunku svetlobe.}
\label{fig:optsoliton}
\end{figure}
Pri prehodu optičnega sunka z osnovno frekvenco $\omega_0$ se različnim delom sunka
frekvenca različno spremeni (slika~\ref{fig:chirp}\,a), začetnemu delu se zmanjša, 
končnemu pa poveča. Po drugi strani pa v snoveh poznamo barvno disperzijo, 
kar pomeni, da se valovanja z različnimi frekvencami širijo z različnimi hitrostmi.
\index{Disperzija} Pojav disperzije je še bolj zapleten pri potovanju sunkov svetlobe,
kar bomo podrobneje obravnavali pri optičnih vlaknih (poglavje~\ref{chap:sunvl}).
Zaenkrat povejmo le, da je pomemben parameter disperzija grupne hitrosti, ki je sorazmerna
z drugim odvodom lomnega količnika po valovni dolžini (enačba~\ref{eq:dmat})
\begin{equation}
D = -\frac{\lambda}{c_0}\frac{d^2n}{d\lambda^2}.
\end{equation}
Pri določenih pogojih (izbrana snov in določeno frekvenčno območje) 
lahko dosežemo, da potuje del valovanja z daljšo valovno dolžino počasneje kot del valovanja
s krajšo valovno dolžino (slika~\ref{fig:chirp}\,b). V tem primeru končni del sunka 
dohiteva sprednjega in učinek disperzije ravno izniči učinek nelinearnosti. 
Nastane signal, ki ohranja svojo obliko~\textendash~soliton. 
\begin{figure}[h]
\centering
\def\svgwidth{145truemm} 
\input{slike/08_Chirp.pdf_tex}
\caption{Čričkanje sunkov svetlobe zaradi nelinearnega pojava (a). Z ustrezno disperzijo lahko
čričkanje izničimo (b) in nastane sunek svetlobe, ki oblike ne spreminja~\textendash~soliton.}
\label{fig:chirp}
\end{figure}

\section{*Izpeljava optičnih solitonov}
\index{Soliton!optični}
Za matematični opis optičnih solitonov izhajamo iz nelinearne \index{Valovna enačba!nelinearna}
valovne enačbe (enačba~\ref{8.3}), ki jo zapišemo v skalarni obliki
\begin{equation}
\nabla^{2}E-\frac{n^2}{c_0^{2}}{\frac{\partial^2 E}{\partial t^2}}=
\mu_{0}{\frac{\partial^2P_{\textrm{NL}}}{\partial t^2}},
\end{equation}
pri čemer je 
$P_\textrm{NL}$ nelinearna polarizacija tretjega reda (enačba~\ref{eq:nlin3}).
Namesto v časovni domeni je enačbo prikladnejše reševati v frekvenčni domeni, zato
namesto $E$ in $P_{\mathrm{NL}}$ vpeljemo Fourierevi transformiranki $\tilde{E}$ in $\tilde{P}$.

Sledi
\begin{equation}
\nabla^{2}\tilde{E}+\frac{n^2}{c_0^{2}}\omega^2 \tilde{E}=
- \mu_{0}\omega^2 \tilde{P}.
\end{equation}
Gornjo enačbo rešujemo z nastavkoma
\begin{equation}
\tilde{E} = \tilde{A} (z,\omega - \omega_0) e^{ik_0z}
\end{equation}
in 
\begin{equation}
 \tilde{P} = \tilde{B} (z,\omega - \omega_0) e^{ik_0z},
\end{equation}
pri čemer je $\omega_0$ osrednja frekvenca svetlobnega sunka in $k_0 = \omega_0 n/c_0$. Vpeljemo še
$\Omega =\omega - \omega_0$
\begin{equation}
\left(\frac{\partial^2}{\partial z^2}+k^2\right)\tilde{A}(z,\Omega) e^{ik_0z} =
- \mu_{0}\omega^2 \tilde{B} (z,\Omega) e^{ik_0z}.
\end{equation}
Da lahko rešimo to enačbo, naredimo nekaj približkov. Ker je $\omega \approx \omega_0$, na desni strani
enačbe nadomestimo frekvenco z osrednjo frekvenco. Poleg tega upoštevamo, da se amplituda 
glede na valovno dolžino le počasi spreminja, zato drugi odvod zanemarimo in 
\begin{equation}
2 i k_0 \frac{\partial \tilde{A}}{\partial z} + (k^2-k_0^2) \tilde{A} = - \mu_{0}\omega_0^2 \tilde{B}.
\end{equation}
Če je disperzija šibka, lahko zapišemo $k^2 - k_0^2$ kot razliko kvadratov, $k(\omega_0 + \Omega)$ pa 
razvijemo v Taylorjevo vrsto okoli osrednje frekvence $\omega_0$ do tretjega člena. Sledi
\begin{equation}
k^2 - k_0^2 \approx 2k_0 (k-k_0) \approx 2k_0 (k'\Omega + \frac{1}{2}k''\Omega^2),
\end{equation}
pri čemer $'$ označuje odvod po frekvenci, in prepišemo enačbo v 
\begin{equation}
2 i k_0 \frac{\partial \tilde{A}}{\partial z} + 2k_0(k'\Omega + \frac{1}{2}k''\Omega^2) \tilde{A} 
= - \mu_{0}\omega_0^2 \tilde{B}.
\end{equation}
Vrnimo se v časovno domeno, tako da naredimo inverzno Fourierevo transformacijo. Naj bo 
$A(z,t)$ kompleksna amplituda električne poljske jakosti in inverzna transformiranka 
funkcije $\tilde{A}(z,\Omega)$, funkcija $B(z,t)$ pa naj bo 
amplituda polarizacije in inverzna transformiranka 
funkcije $\tilde{B}(z,\Omega)$.
Sledi
\begin{equation}
i (\frac{\partial}{\partial z}+\frac{1}{v_{g}}\frac{\partial}{\partial t})A-
\frac{1}{2}\frac{d^{2}k}{d\omega^{2}}\,\frac{\partial^{2}A}{\partial t^{2}}=
-\frac{\mu_0\omega_0^2}{2 k_0}B,
\label{8.93}
\end{equation}
pri čemer smo z $v_g = d\omega/dk = 1/k'$ označili grupno hitrost.\index{Hitrost valovanja!grupna}
Vpeljimo novo spremenljivko 
\begin{equation}
\tau=t-\frac{z}{v_{g}},
\label{nelinver}
\end{equation}
s katero opišemo obliko sunka $A_S(z,\tau)$, kot ga vidi opazovalec, ki se giblje
z grupno hitrostjo skupaj s sunkom. Uporabimo pravilo verižnega odvajanja 
\begin{equation}
\frac{\partial A}{\partial z} = \frac{\partial A_S}{\partial z} + \frac{\partial A_S}{\partial \tau}
\frac{\partial \tau}{\partial z}
= \frac{\partial A_S}{\partial z} -\frac{1}{v_g} \frac{\partial A_S}{\partial \tau}.
\end{equation}
Podobno naredimo še za odvod po času $\tau$, ki pa se ne razlikuje od odvoda po času $t$
\begin{equation}
\frac{\partial A}{\partial t} = \frac{\partial A_S}{\partial z}\frac{\partial z}{\partial \tau}+
\frac{\partial A_S}{\partial \tau}\frac{\partial \tau}{\partial t} 
= \frac{\partial A_S}{\partial \tau} 
\end{equation}
in
\begin{equation}
\frac{\partial^2 A}{\partial t^2} = \frac{\partial^2 A_S}{\partial\tau^2}.
\end{equation}
Vstavimo še amplitudo nelinearne polarizacije (enačba~\ref{eq:ptnl}), pri čemer izraz popravimo
za faktor $4$, ker smo drugače vpeljali parameter $A$. Dobimo
\begin{equation}
B = 3\varepsilon_0\chi |A|^2 A
\end{equation}
in enačbo~(\ref{8.93}) zapišemo kot 
\begin{equation}
i\,\frac{\partial A_S}{\partial z}-\frac{1}{2}\frac{d^{2}k}{d\omega^{2}}\,\frac{\partial^{2}A_S}{\partial\tau^{2}}+\kappa\left|A_S\right|^{2}A_S=0.
\label{8.95}
\end{equation}
Pri tem je parameter
\begin{equation}
\kappa = \frac{3\omega_0\chi}{2c_0 \tilde{n}} = 2 \omega_0 \varepsilon_0 n_2 \tilde{n}
\end{equation}
sorazmeren nelinearnemu lomnemu količniku $n_2$ 
(enačba~\ref{eq:n2}). Enačba~(\ref{8.95}) ni nič drugega kot nelinearna Schr\"odingerjeva 
enačba\index{Schr\"odingerjeva enačba!nelinearna}, ki smo jo 
zapisali že pri izpeljavi krajevnih solitonov~(enačba~\ref{8.84}). Enačbi se razlikujeta v tem, da
ima vlogo prečne koordinate $x$ tukaj čas $\tau$ in rešitve nimajo več konstantnega premera,
ampak imajo konstantno dolžino sunka. Stacionarne rešitve obstajajo le v primeru, kadar je  $d^{2}k/d\omega^{2}<0$ oziroma kadar ima drugi odvod nasprotni predznak od nelinearnega lomnega količnika $n_2$. Kot pri krajevnih solitonih tudi tukaj vpeljemo parameter $\eta$, ki je sorazmeren 
z energijo solitona (enačba~\ref{eq:solj}). Sledi 
\begin{equation}
A_S\left(z,\tau\right)=\sqrt{\frac{2}{\kappa}}\eta\frac{e^{i\eta^{2}z}}{{\cosh}\left(\eta \tau 
\sqrt{2\left|\frac{d^{2}\beta}{d\omega^{2}}\right|^{-1}}\right)}
\end{equation}
oziroma
\begin{equation}
A\left(z,t\right)=\sqrt{\frac{2}{\kappa}}\eta\frac{e^{i\eta^{2}z}}{{\cosh}\left(\eta (t-\frac{z}{v_g}) 
\sqrt{2\left|\frac{d^{2}\beta}{d\omega^{2}}\right|^{-1}}\right)}.
\label{8.96}
\end{equation}
Zapisana je oblika solitona, ki potuje z grupno hitrostjo $v_g$ in pri tem ohranja obliko. Zaradi tega
so solitoni izredno zanimivi za prenos velike gostote informacij na velike razdalje, saj se izognemo
omejitvam zaradi disperzije. 

\begin{remark}
Ena izmed snovi, ki izpolnjuje pogoj, da je $k''$ nasprotnega predznaka kot $n_2$, so kvarčna 
optična vlakna. Pri valovnih dolžinah vidne svetlobe to sicer ne velja, velja pa za 
$\lambda \gtrsim 1,3~\si{\micro\metre}$.\index{SiO$_2$}
Pogoj je torej izpolnjen pri valovnih dolžinah okoli $1,5~\si{\micro\metre}$, ki se navadno uporabljajo 
pri prenosu signalov po optičnih vlaknih, in signal lahko potuje brez podaljševanja. 
\end{remark}

\section{Optična fazna konjugacija}
\index{Optična fazna konjugacija}
Optična fazna konjugacija je zanimiv in danes tudi praktično pomemben
pojav, pri katerem nastane iz danega valovanja novo valovanje, ki ima enake valovne
fronte, vendar potuje v nasprotni smeri od prvotnega valovanja. Novo valovanje je tako,
kot bi začetnemu valovanju obrnili predznak časa in ga ``zavrteli nazaj''.

Vzemimo optično nelinearno snov, na katero posvetimo z dvema močnima ravnima
snopoma v nasprotnih smereh. To sta črpalna snopa in njuna valovna vektorja 
naj bosta ${\bf k}_{1}$ in ${\bf k}_2 = -{\bf k}_{1}$. Poleg njiju naj na snov vpada
še tretji, signalni snop, ki ni nujno raven val (slika~\ref{08_OPC1}). 
Signalni snop interferira s prvim črpalnim valom in s tem zaradi nelinearnosti 
tretjega reda povzroči modulacijo lomnega količnika. Ta je skoraj periodična, če
je signalni val podoben ravnemu valu. Na modulaciji lomnega količnika se
drugo črpalno valovanje uklanja. Uklonjeno valovanje je enake oblike
kot signalno, le potuje v nasprotni smeri, saj ima drugo črpalno valovanje
nasprotno smer od prvega. Črpalni valovanji sta seveda enakovredni in ni
mogoče ločiti med valovanjem, s katerim signalno valovanje interferira, in valovanjem, 
ki se uklanja.
\begin{figure}[h]
\centering
\def\svgwidth{60truemm} 
\input{slike/08_opc1.pdf_tex}
\caption{Optična fazna konjugacija. Dva močna črpalna žarka (modra) vpadata 
na optično nelinearno snov v nasprotnih smereh, vpadni signal (rdeč) pa se odbije v 
smer, iz katere vpada.}
\label{08_OPC1}
\end{figure}

\begin{remark}Optična fazna konjugacija je zelo podobna holografiji, 
le da pri holografiji najprej zapišemo predmetni snop, ki ga kasneje reproduciramo, 
pri fazni konjugaciji pa zapis začetnega valovanja in njegova reprodukcija 
potekata sočasno. 
\end{remark}
Naj se signalno valovanje razširja v smeri $z$. Zapišemo ga  
\begin{equation}
E_{3}=\mathfrak{\Re}\left(A_3\left(z\right)\, e^{i\left(kz-\omega t\right)}\right).
\label{8.97}
\end{equation}
V nadaljevanju bomo pokazali, da je novonastalo valovanje sorazmerno
\begin{equation}
E_{4} \propto \mathfrak{\Re}\left(A_3^{*}\left(z\right)\, e^{i\left(-kz-\omega t\right)}\right).
\label{8.98}
\end{equation}
Zaradi nasprotnega predznaka $k$ potuje nastalo valovanje v obratni smeri od signalnega
valovanja. Poleg tega je kompleksno konjugirana tudi njegova amplituda. To seveda
ne vpliva na obliko valovnih front, saj so te popolnoma enake kot pri signalnem
valovanju. Zaradi lastnosti, da lahko novo valovanje iz signalnega nastane tako,
da krajevni del kompleksno konjugiramo, nastalemu valovanju pravimo fazno
konjugirano valovanje.
\begin{figure}[h!]
\centering
\def\svgwidth{80truemm} 
\input{slike/08_opc2.pdf_tex}
\caption{Primerjava odbojev na navadnem zrcalu (levo) in faznem konjugatorju (desno): odboj ravnega
vala (a), odboj krogelnega valovanja (b) in odboj popačenega vala (c). Valovne fronte 
vpadnega vala so označene s polno črto, odbitega pa s črtkano.}
\label{08_OPC2}
\end{figure}

Uporabna posledica fazne konjugacije je prikazana na sliki~(\ref{08_OPC2}).
Najpreprostejši primer je vpad ravnega vala (a), ki se ne odbije po 
odbojnem zakonu (slika levo), ampak se odbije v smer, iz katere 
je vpadel na snov (slika desno). Drugi primer je krogelni val 
ali v približku tudi Gaussov snop (b). Ko vpade na navadno zrcalo (levo), se njegova divergenca
ohranja in se žarek še naprej razširja. Na fazno konjugiranem zrcalu se krogelni val spet
zbere v izvoru (desno). 

Tretji primer je sredstvo, ki valovanju doda naključno
fazo, zato po prehodu valovne fronte niso več gladke (c). Od navadnega zrcala
se popačen snop odbije, pri ponovnem prehodu skozi sredstvo pa se popačenje
še poveča. Povsem drugačno je obnašanje pri odboju na faznem konjugatorju. 
Ko popačen snop vpade na fazni konjugator, v njem generira fazno konjugiran snop, 
ki potuje v nasprotni smeri in ima enako nepravilne valovne fronte kot vpadni val. Po prehodu
skozi nepravilno sredstvo se neravnosti valovne fronte izničijo
in nastanejo enake gladke valovne fronte ravnega vala kot na začetku. 
To lastnost popravljanja valovne fronte je mogoče 
koristno uporabiti, na primer namesto enega zrcala v laserskem resonatorju.

\begin{remark}
Omenili smo že, da se fazno konjugirana zrcala uporabljajo v laserjih za izničenje
popačenj Gaussovega snopa. Drug primer uporabe je pri optični astronomiji
ali optičnih komunikacijah skozi atmosfero. Naključne spremembe gostote v atmosferi
signalu dodajo naključni fazni premik, ki signal popači. Če se signal odbije od zrcala nazaj
proti izvoru, je torej dvakratno popačen. Če pa se odbije od fazno konjugiranega zrcala, 
se vpliv nehomogenosti atmosfere ravno izniči in na prenos signala ne vpliva, poleg
tega je šibek vpadni signal lahko še dodatno ojačan. 
\end{remark}

\section{*Izpeljava optične fazne konjugacije}
\index{Optična fazna konjugacija}
Poglejmo podrobneje, kako v nelinearnem sredstvu nastane fazno konjugiran
val. Kot kaže slika~(\ref{08_OPC1}), je celotno polje v nelinearnem
sredstvu vsota štirih valovanj, dveh močnih črpalnih (oznaki 1 in 2), signalnega 
(oznaka 3) in odbitega (oznaka 4)
\begin{equation}
E=\frac{1}{2}A_{1}e^{i{\bf k}_{1}\cdot{\bf r}-i\omega t}+\frac{1}{2}A_{2}e^{-i{\bf k}_{1}\cdot{\bf r}-
i\omega t}+\frac{1}{2}A_{3}\left(z\right)e^{ikz-i\omega t}+\frac{1}{2}A_{4}
\left(z\right)e^{-ikz-i\omega t}+{\rm k.k.}
\label{8.99}
\end{equation}
S k.k. smo spet označili kompleksno konjugirane člene.  Vsa valovanja naj imajo
enako frekvenco, zaradi enostavnosti še privzemimo, da so enake tudi vse polarizacije.
Račun poenostavimo še s privzetkom, da sta črpalna vala $E_{1}$
in $E_{2}$ dosti močnejša od $E_{3}$ in $E_{4}$, tako da sta njuni
amplitudi konstantni, $E_{3}\left(z\right)$ in $E_{4}\left(z\right)$
pa se le počasi spreminjata.

Vstavimo $E$ v valovno enačbo z nelinearno polarizacijo (enačba~\ref{8.3})
\index{Valovna enačba!nelinearna}
\begin{equation}
\nabla^{2}E+\epsilon\frac{\omega^{2}}{c_0^{2}}\, 
E=\mu_{0}\frac{\partial^2 P_{\mathrm{NL}}}{\partial t^2}.
\label{8.100}
\end{equation}
Pri tem je $\epsilon\,\omega^{2}/c_0^{2}=k^{2}$, $P_{\textrm{NL}}$ pa je po enačbi~(\ref{eq:nl3P})
enak $P_\mathrm{NL}= \epsilon_{0}\chi^{(3)}E^3$, kjer je $\chi^{(3)} = \chi$
efektivna nelinearna susceptibilnost
za izbrano polarizacijo vseh polj. 

Ker je $E$ zapisan kot vsota osmih členov
(enačba~\ref{8.99}), vsebuje produkt $E^3$ kar 512 členov. Vendar se njihovo število znatno zmanjša, 
če upoštevamo le tiste z enako časovno odvisnostjo oziroma enako frekvenco.
Poleg tega nas ne zanimajo različne kombinacije valovnih vektorjev, ampak k enačbi za $E_{3}$ 
prispevajo le tisti členi s krajevnim faznim faktorjem $\exp(ikz)$, 
k enačbi za $E_4$ pa tisti z $\exp(-ikz)$. Sledi
\begin{eqnarray}
P_{\mathrm{NL}\,3,4} &=& \frac{1}{8}\varepsilon_0\chi [
\left(6 A_1 A_2 A_4^*+ 6A_1 A_1^*A_3 + 6A_2A_2^*A_3 + 3 A_3A_3^*A_3 + 6 A_4 A_4^* A_3\right)
e^{i k z - i\omega t} \nonumber\\
&+& 
\left(6 A_1 A_2 A_3^*+6 A_1 A_1^*A_4 + 6A_2A_2^*A_4 + 6 A_3A_3^*A_4 + 3 A_4 A_4^* A_4\right)
e^{-i k z - i\omega t}].
\end{eqnarray}
Če zanemarimo še člene, v katerih nastopata $A_3$ in $A_4$ v višjih potencah, dobimo
\begin{eqnarray}
P_{\mathrm{NL}\,3,4} &=& \frac{3}{4}\varepsilon_0\chi [
\left( A_1 A_2 A_4^*+ |A_1|^2 A_3 + |A_2|^2 A_3 \right)
e^{i k z - i\omega t} \nonumber\\
&+& 
\left( A_1 A_2 A_3^*+|A_1|^2 A_4 + |A_2|^2A_4 \right)
e^{-i k z - i\omega t}].
\end{eqnarray}
Vstavimo gornji izraz v valovno enačbo~(enačba~\ref{8.100}) in upoštevamo, 
da se $A_i(z)$ le počasi 
spreminja (kar pomeni, da zanemarimo drugi odvod po $z$). Sledi 
\begin{equation}
i k \frac{dA_3}{dz} = - \frac{3}{4} \mu_0\varepsilon_0 \chi \omega^2 
\left( A_1 A_2 A_4^*+ (|A_1|^2 + |A_2|^2) A_3 \right)
\label{eq:opc1}
\end{equation}
in 
\begin{equation}
-i k \frac{dA_4}{dz} = - \frac{3}{4} \mu_0\varepsilon_0 \chi \omega^2 
\left( A_1 A_2 A_3^*+ (|A_1|^2 + |A_2|^2) A_4 \right).
\label{eq:opc2}
\end{equation}
Drugi člen na desni že poznamo: opisuje odvisnost lomnega količnika
od intenzitete črpalnih valov, torej optični Kerrov\index{Kerrov pojav!optični}
pojav, in je zato le dodaten prispevek
k fazi. Vpeljimo novi amplitudi, ki se od prejšnjih razlikujeta zgolj v faznem faktorju.
\begin{equation}
\tilde{A}_3 = A_3 \exp\left(-i\frac{ 3 \chi \omega}{4 c_0 n}(|A_1|^2 + |A_2|^2) z\right)
\end{equation}
in 
\begin{equation}
\tilde{A}_4 = A_4 \exp\left(i\frac{ 3 \chi \omega}{4 c_0 n}(|A_1|^2 + |A_2|^2)z\right).
\end{equation}
Ko novi amplitudi vstavimo v diferencialni enačbi~(enačbi~\ref{eq:opc1} in 
\ref{eq:opc2}), se Kerrov prispevek k fazi ravno odšteje
in enačbi se prepišeta v 
\begin{equation}
\frac{d\tilde{A}_{3}}{dz}=i\frac{ 3 \chi \omega}{4 c_0 n}\,
A_{1}A_{2}\tilde{A}_{4}^{*} \quad \textrm{in} \quad 
\frac{d\tilde{A}_{4}}{dz}=-i\frac{ 3 \chi \omega}{4 c_0 n}\,
A_{1}A_{2}\tilde{A}_{3}^*.
\label{8.105}
\end{equation}
Vpeljemo sklopitveno konstanto
\begin{equation}
\kappa=\frac{ 3 \chi \omega}{4 c_0 n}A_1 A_2.
\label{8.106}
\end{equation}
Enačbi se poenostavita v 
\begin{equation}
\frac{d\tilde{A}_{3}}{dz}=i\kappa \tilde{A}_{4}^{*} \quad
\textrm{oziroma} \quad \frac{d\tilde{A}^*_{3}}{dz}=-i\kappa^* \tilde{A}_{4} 
\quad \textrm{in} \quad
\frac{d\tilde{A}_{4}}{dz}=-i\kappa \tilde{A}_{3}^*.
\label{8.107}
\end{equation}
Zelo težaven problem nelinearne valovne enačbe smo prevedli na linearen
sistem dveh preprostih sklopljenih enačb za amplitudi signalnega in
odbitega vala. Rešitvi sistema enačb~(\ref{8.107}) 
sta 
\begin{eqnarray}
\tilde{A}_3^* \left(z\right) & = & C_{1}\cos(\left|\kappa\right|z)+
C_{2}\sin(\left|\kappa\right|z)
\label{8.108}\\
\tilde{A}_4 \left(z\right) & = & D_{1}\cos(\left|\kappa\right|z)+
D_{2}\sin(\left|\kappa\right|z).
\label{8.108a}
\end{eqnarray}
Z upoštevanjem zveze, ki izhaja  iz prve diferencialne enačbe 
(enačba~\ref{8.107}), zapišemo
\begin{equation}
C_1 = \frac{i \kappa^*}{|\kappa|}D_2 \qquad
\textrm{in} \qquad 
C_2 = -\frac{i \kappa^*}{|\kappa|}D_1. 
\end{equation}
Potrebujemo še robne pogoje za obe valovanji. Z leve, pri $z=0$,
poznamo $\tilde{A}_{3}^{*}\left(0\right)$, pri $z=L$ pa ne more biti odbitega
vala in je zato $\tilde{A}_{4}\left(L\right)=0$. S tem določimo konstanti $D_{1}$
in $D_{2}$
\begin{equation}
D_2 = -\frac{i|\kappa|}{\kappa^*} \tilde{A}_3^*(0) \qquad
\textrm{in} \qquad 
D_1 = -D_2 \tan(|\kappa|L). 
\end{equation}
Gornje enačbe združimo in zapišemo amplitudi znotraj nelinearne snovi
\begin{eqnarray}
\tilde{A}_{3}\left(z\right) & = & \tilde{A}_3(0)
\frac{\cos\left(|\kappa|(L-z)\right)}{\cos\left(|\kappa|L\right)}
\end{eqnarray}
in
\begin{eqnarray}
\tilde{A}_{4}\left(z\right) & = & \tilde{A}_3^*(0)\frac{i \kappa}{|\kappa|}
\frac{\sin\left(|\kappa|(L-z)\right)}{\cos\left(|\kappa|L\right)}.
\label{8.109}
\nonumber 
\end{eqnarray}
Izračunajmo še amplitudi odbitega in prepuščenega vala. Amplituda odbitega vala 
pri $z=0$ je 
\boxeq{8.110}{
\tilde{A}_{4}(0)  =  \tilde{A}_3^*(0)\frac{i \kappa}{|\kappa|}
\tan \left(|\kappa|L\right),
}
amplituda prepuščenega pri $z = L$ pa
\boxeq{8.110a}{
\tilde{A}_{3}(L)  =  \frac{\tilde{A}_3^*(0)}{
\cos \left(|\kappa|L\right)}.
}
Oglejmo si gornja rezultata. Vidimo, da je odbiti val sorazmeren 
kompleksno konjugirani amplitudi vpadnega vala, kar smo omenili že v prejšnjem
razdelku. Poleg konjugirane amplitude ima tudi natanko nasproten valovni vektor, 
zato tudi ime fazno konjugiran val. Zanimiva je tudi njegova velikost. Ker 
je lahko $\tan\left(|\kappa|L\right)>1$, je odbit val lahko močnejši od vpadnega.
To ojačenje odbitega vala gre seveda na račun moči črpalnih
valov. V našem računu bi lahko amplituda odbite svetlobe narasla proti neskončnosti, 
vendar zapisane enačbe takrat niso več veljavne, saj smo privzeli, 
da sta signalni in odbiti žarek precej šibkejša od črpalnih.

Poglejmo še prepuščeni žarek. Ker je $\cos(x)\leq1$, je amplituda prepuščenega
žarka vedno večja od amplitude vpadnega. To pomeni, da smo na račun črpalnih žarkov
dobili prepustnost, ki je vedno večja od $100~\%$, in odbojnost, ki je lahko 
večja od $100~\%$.

Pri računu smo predpostavili, da je vpadni signal ravni val. Če je njegova
amplituda odvisna še od prečne koordinate, ga lahko razvijemo po ravnih
valovih in zgoraj izpeljana enačba~(\ref{8.110}) velja za vsako komponento posebej. 
Odbite komponente so sorazmerne s konjugiranimi komponentami signalnega valovanja
z nasprotnim valovnim vektorjem in dajo skupaj valovno fronto enake
oblike kot pri signalnem valovanju, le giblje se v nasprotni smeri.

\section{Stimulirano Ramanovo in stimulirano Brillouinovo sipanje}
\index{Ramanovo sipanje!stimulirano}
\index{Brillouinovo sipanje!stimulirano}
Ko svetloba vpade na snov, se je del siplje. Poleg elastičnega 
Rayleighovega sipanja,\index{Rayleighovo sipanje}
pri katerem se energija vpadlih fotonov (in z njo frekvenca) svetlobe ohranja, pride
tudi do sipanja, pri katerem se energija izhodnih fotonov razlikuje od energije vpadnih. 

Če se energija fotonov spremeni zaradi prehajanja molekul snovi med različnimi
vibracijskimi ali rotacijskimi stanji, govorimo o Ramanovem\index{Ramanovo sipanje}
sipanju\footnote{Indijski fizik in nobelovec Sir Chandrasekhara Venkata Raman, 1888--1970.}. 
Do tega pojava lahko pride tako v plinih in tekočinah kot tudi v trdnih snoveh. Navadno 
ločimo dva primera: Stokesovo sipanje\footnote{Irski fizik in matematik Sir George Gabriel
Stokes, 1819--1903.}, pri katerem foton odda energijo molekuli, in anti-Stokesovo sipanje,
\index{Ramanovo sipanje!Stokesovo}\index{Ramanovo sipanje!anti-Stokesovo}
pri katerem foton prejme energijo od vzbujene molekule. V prvem primeru je frekvenca
sipane svetlobe $\omega_s=\omega_0-\omega_v$, kjer $\omega_0$ označuje frekvenco vpadne
svetlobe, $\omega_v$ pa vibracijsko frekvenco, v drugem primeru pa $\omega_s=\omega_0+\omega_v$.
Slednji procesi so razmeroma redki, zato je intenziteta anti-Stokesovega sipanja 
še znatno šibkejša od že tako šibkega Ramanovega sipanja. Tipični Ramanov premik 
$\omega_0-\omega_s$ znaša okoli $10^{12}$--$10^{13}~\si{\hertz}$.
Drug zanimiv primer je, kadar pride do spremembe energije 
fotonov zaradi vzbujanja akustičnih valov (fononov). Takrat govorimo o Brillouinovem 
sipanju\footnote{Francoski fizik L\'eon Nicolas Brillouin, 1889--1969.}. Tipičen
Brillouinov premik je $\sim 10^{10}\si{\hertz}$. \index{Brillouinovo sipanje}

\subsection*{Stimulirano Ramanovo sipanje}
Pri spontanem Ramanovem sipanju se svetloba siplje na termično vzbujenih fluktuacijah
v snovi. Če pa se svetloba siplje na fluktuacijah, ki jih je povzročilo vpadno
svetlobno polje, govorimo o stimuliranem Ramanovem sipanju. Dosežemo ga tako,
da na snov poleg osnovnega žarka s frekvenco $\omega_0$ usmerimo dodaten Stokesov 
žarek s frekvenco $\omega_s$, ki stimulira določen prehod. Fotoni 
osnovnega žarka se absorbirajo in izhajajo fotoni pri $\omega_s$. Povedano drugače: 
moč svetlobe pri $\omega_s$, ki je po fazi in smeri enaka vpadni, 
se povečuje na račun črpalnega žarka pri $\omega_0$. Tako pride do 
resonančnega ojačenja svetlobe pri $\omega_s$ oziroma do stimuliranega 
Ramanovega sipanja. 

\begin{figure}[h]
\centering
\def\svgwidth{140truemm} 
\input{slike/08_Raman.pdf_tex}
\caption{Prehodi med energijskimi nivoji za Ramanov sipanje in shema Brillouinovega sipanja. 
Pri Stokesovem sipanju snov prevzame energijo (a), pri anti-Stokesovem sipanju energijo odda (b).
Pri Brillouinovem sipanju svetloba vzbuja akustične valove (c) in se na njih odbije.}
\label{08_Raman}
\end{figure}

Obravnavajmo Ramanovo sipanje v klasičnem približku in snov opišemo 
z $N$ neodvisnimi enodimenzionalnimi harmonskimi oscilatorji. Nihanje
posameznega oscilatorja zadošča enačbi\index{Harmonski oscilator}
\begin{equation}
\frac{d^2X(z,t)}{dt^2}+ \gamma \frac{dX}{dt}+\omega_v^2X = \frac{F(z,t)}{m},
\label{srs:X}
\end{equation}
kjer je $X$ koordinata, $\gamma$ koeficient dušenja, $m$ masa in $F$ zunanja sila.
Glavna predpostavka modela je, da polarizabilnost molekul ni konstantna, ampak odvisna od 
``raztega'' molekule oziroma oscilatorja. Polarizabilnost $\alpha$ potem 
razvijemo v Taylorjevo vrsto do prvega člena 
\begin{equation}
\varepsilon = \varepsilon_0(1+N\alpha) = \varepsilon_0+\varepsilon_0N\left(\alpha_0 + 
\frac{d\alpha}{dX}\,X\right).
\label{srs:a}
\end{equation}
Silo na en oscilator izračunamo kot odvod energije po koordinati 
\begin{equation}
F = \frac{dW}{dX}= \frac{1}{2}\varepsilon_0 \frac{d\alpha}{dX}\,\overline{E^2}.
\label{srs:F}
\end{equation}
Električno poljsko jakost smo zapisali kot povprečje, saj so optične frekvence 
tako hitre, da jim molekule ne morejo slediti. 
Celotno električno polje zapišemo kot vsoto osnovnega in Stokesovega polja
\begin{equation}
E(z,t)= \frac{1}{2}\left( E_0(z)e^{-i\omega_0t}+ E_s(z)e^{-i\omega_st} + k.k.\right)
\end{equation}
in povprečje kvadrata
\begin{equation}
\overline{E^2} = \frac{1}{4}\left(E_0^2E_s^* e^{-i(\omega_0-\omega_s)t}+k.k.\right).
\end{equation}
Vstavimo zapisano električno poljsko jakost najprej v izraz za silo (enačba~\ref{srs:F}),
nato pa v enačbo oscilatorja (enačba~\ref{srs:X}). Sledi
\begin{equation}
\frac{1}{2}\left(\omega_v^2-\omega^2-i\omega\gamma\right)\tilde{X} = 
\frac{\varepsilon_0}{8m}\frac{d\alpha}{dX}E_0 E_s^*,
\end{equation}
pri čemer je 
\begin{equation}
X(z,t) = \frac{1}{2}\left(\tilde{X}e^{-i\omega t}+ k.k.\right)
\end{equation}
in $\omega = \omega_0-\omega_s$. Oscilatorji torej nihajo s kompleksno amplitudo
\begin{equation}
\tilde{X} = \frac{\varepsilon_0\frac{d\alpha}{dX}E_0 E_s^*}{4m\left(
\omega_v^2-(\omega_0-\omega_s)^2-i(\omega_0-\omega_s)\gamma\right)}.
\end{equation}
Zdaj lahko zapišemo polarizacijo $P = \varepsilon_0N\alpha E$ in nelinearen del
polarizacije je\index{Električna polarizacija}
\begin{equation}
P_{\mathrm{NL}} = \frac{1}{4}\varepsilon_0 N \frac{d\alpha}{dX} \, \left(\tilde{X}
e^{-i(\omega_0-\omega_s)t}
+ k.k.\right) \cdot \left( E_0(z)e^{-i\omega_0t}+ E_s(z)e^{-i\omega_st} + k.k.\right).
\end{equation}
Omejimo se le na polarizacijo pri frekvenci $\omega_s$ in zapišemo
\begin{equation}
P_{\mathrm{NL},\omega_s} = \varepsilon_0 \chi_{ef}E_s = 
\frac{\varepsilon_0^2 N }{8m}\left(\frac{d\alpha}{dX}\right)^2 \, 
\frac{|E_0|^2}{\omega_v^2-(\omega_0-\omega_s)^2+i(\omega_0-\omega_s)\gamma}\,E_s.
\label{srs:chi}
\end{equation}
Efektivna susceptibilnost za svetlobo pri frekvenci $\omega_s$ 
je \index{Susceptibilnost!efektivna} torej kompleksna in sorazmerna intenziteti 
vpadne laserske svetlobe pri osnovni frekvenci $\omega_0$. 
V resonanci, ko je $\omega_0-\omega_s = \omega_v$, je efektivna susceptibilnost
imaginarna in negativnega predznaka, kar ima, kot bomo videli, zelo pomembne 
fizikalne posledice. Iz gornjega izraza je tudi razvidno, zakaj gre pri 
stimuliranem Ramanovem sipanju za nelinearen optični pojav tretjega reda. 

Kompleksna susceptibilnost pomeni kompleksni lomni količnik. Če upoštevamo
le prve člene v razvoju, se vpadna svetloba po snovi širi kot\index{Susceptibilnost!kompleksna}
\begin{equation}
E_s(z) = E_s(0)\exp\left(i k z + ikz\frac{1}{2}\Re(\chi)- k z \frac{1}{2}\Im(\chi)\right).
\label{srs:Es}
\end{equation}
\begin{definition}
\label{nalogasrs}
 Izpelji enačbo~(\ref{srs:Es}), tako da iz efektivne susceptibilnosti izračunaš lomni količnik,
in pokaži, da je $\Im(\chi)$ vedno negativen. 
\end{definition}
Ker je imaginarni del efektivne susceptibilnosti vedno negativen (naloga~\ref{nalogasrs}), 
električna poljska jakost eksponentno narašča na račun črpalnega laserskega snopa. 
Največjo ojačenje je seveda v primeru, ko je sistem v resonanci
in razlika frekvenc vpadne svetlobe ravno enaka vibracijski frekvenci.
Če zanemarimo izgube, lahko zapišemo
\begin{equation}
|E_s(L)|^2 = |E_s(0)|^2 e^{G_RjL}.
\end{equation}
Vrednosti $G_R$ so $0,024~\si{\cm/\mega\watt}$ za CS$_2$, \index{CS$_2$}
$0,029~\si{\cm/\mega\watt}$ za LiNbO$_3$ \index{LiNbO$_3$}
in $0,0008~\si{\cm/\mega\watt}$ za SiO$_2$.\index{SiO$_2$} 
V meter dolgem odseku vlakna pride tako pri intenziteti svetlobe $10^{10}~\si{\watt/\meter^2}$ 
do faktorja ojačanja $1,08$, na vlaknu dolžine $20~\si{\metre}$ pa do faktorja $5$.

\begin{remark}
Pojav stimuliranega Ramanovega sevanja je posebej pomemben v ojačenje signala v 
optičnih vlaknih, predvsem zaradi velike intenzitete in velike dolžine, na kateri 
pride do medsebojne interakcije. \index{Optično ojačevanje!v vlaknih}
\end{remark}

\subsection*{Stimulirano Brillouinovo sipanje}
Stimulirano Brillouinovo sipanje je pojav, pri katerem vpadlo svetlobno valovanje
vzbudi akustični val (fonon), nato pa se na njem siplje. Poleg vpadne svetlobe pri $\omega_0$
se tako pojavi še Stokesova svetloba pri frekvenci $\omega_s = \omega_0-\omega_f$, pri čemer 
$\omega_f$ označuje frekvenco akustičnega vala  (slika~\ref{08_Raman}\,c). Interferenca
vpadnega in Stokesovega valovanja, ki ima komponento ravno pri $\omega_f$, povratno
povečuje intenziteto vzbujenega zvočnega valovanja, njegovo povečanje pa vodi do
večje intenzitete Stokesovega valovanja. Pride do pozitivne povratne zanke 
in eksponentnega ojačenja svetlobe. Najmočnejši pojav je 
ravno v nasproti smeri od vpadne svetlobe, v smeri naprej pa Brillouinovega sipanja ni.
Če bi namreč vstopna in izstopna svetloba bili vzporedni, bi bila razlika njunih
valovnih vektorjev enaka nič. 

Računa za Brillouinovo sipanje ne bomo naredili, ga pa lahko bralec poišče
v literaturi\footnote{Npr. R. W. Boyd, {\it Nonlinear Optics}, Academic Press.}. Poglavitno 
je, da tudi pri stimuliranem Brillouinovem sipanju pride do eksponentnega 
ojačenja signala pri zmanjšani frekvenci
\begin{equation}
|E_s(z)|^2 = |E_s(L)|^2 e^{G_Bj(L-z)}.
\end{equation}
Pri zapisu smo upoštevali, da se val širi in ojačuje v nasprotni smeri od vpadnega.
Vrednosti parametra $G_B$ so na primer $0,13~\si{\cm/\mega\watt}$ za CS$_2$ \index{CS$_2$}
in $0,0045~\si{\cm/\mega\watt}$ za SiO$_2$.\index{SiO$_2$} 
Ker je koeficient $G_B$ odvisen od zunanjih parametrov, na primer od 
temperature ali pritiska, lahko stimulirano Brillouinovo sipanje uporabimo tudi za
senzoriko. 