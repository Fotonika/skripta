%\chapterimage{slike/NLO.jpg} 

\chapter{Nelinearna optika}
\label{chap:NLO}
Pri obravnavi svetlobnega valovanja v snovi smo doslej vedno privzeli linearno 
zvezo med polarizacijo in jakostjo električnega polja. To 
je seveda približek, ki je dovolj dober le pri razmeroma majhnih jakostih
polja. Kadar doseže jakost električnega polja velike vrednosti -- in v laserskih snopih
jih nedvomno lahko doseže -- je treba upoštevati tudi višje člene v razvoju. Takrat
govorimo o nelinearni optiki\index{Nelinearna optika}, saj zveza med polarizacijo
in električnim poljem ni linearna. V tem poglavju bomo spoznali zanimive pojave, ki jih 
povzroči nelinearni del polarizacije, med drugim optično 
frekvenčno podvajanje, optično parametrično ojačevanje, optično usmerjanje, 
samozbiranje laserskega snopa, optične solitone in optično fazno konjugacijo. 

\section{Nelinearna susceptibilnost}
\label{Chap:Chi}
V linearnem približku odziva snovi velja, da je polarizacija snovi\index{Električna polarizacija} 
$\mathbf{P}$ linearna funkcija jakosti električnega polja 
$\mathbf{E}$\index{Električno polje!jakost}. Takrat zapišemo (enačba~\ref{eq:PM})
\begin{equation}
\mathbf{P} = \mathbf{D} - \varepsilon_0 \mathbf{E} = 
\varepsilon_0 \underline{\epsilon} \cdot\mathbf{E} - \varepsilon_0 \mathbf{E} = 
\varepsilon_0 (\underline{\epsilon} - 1)\cdot\mathbf{E}. 
\end{equation}
Pri tem smo uporabili splošen zapis dielektričnosti v obliki tenzorja. Če vpeljemo še tenzor linearne susceptibilnosti\index{Susceptibilnost!linearna}
\begin{equation}
\chi^{(1)} = \underline{\epsilon} - 1,
\end{equation}
linearni odziv snovi zapišemo strnjeno kot
\begin{equation}
\mathbf{P}_{\mathrm{L}} =  \varepsilon_0 \chi^{(1)} \cdot \mathbf{E}.
\end{equation}
Ta približek je dober za majhne jakosti električnega polja. Pri večjih jakostih polja
postanejo pomembni tudi členi višjega reda v razvoju polarizacije
po $\mathbf{E}$
\boxeq{8.1}{
\mathbf{P}=\mathbf{P}_{\mathrm{L}}+ \mathbf{P}_{\mathrm{NL}}=
\epsilon_{0} \chi^{(1)}\cdot \mathbf{E}+
\epsilon_{0}\chi^{(2)}:\mathbf{E}\, \mathbf{E}+
\epsilon_{0}\chi^{(3)}\vdots \mathbin \mathbf{E}\mathbin \mathbf{E}\mathbin\mathbf{E} + \dots
}
Vpeljali smo nelinearni susceptibilnosti\index{Susceptibilnost!nelinearna} 
$\chi^{(2)}$ in $\chi^{(3)}$, ki sta tenzorja tretjega in četrtega ranga. 
Za bolj nazorno predstavo izpišimo notranja produkta tenzorjev z vektorji še 
po komponentah
\begin{equation}
\left(\mathbf{P}_{\mathrm{NL,2}}\right)_i= 
\epsilon_{0}\sum_{j,k}\chi^{(2)}_{ijk} \,E_j \,E_k =
\epsilon_{0}\chi^{(2)}_{ijk} \,E_j \,E_k 
\label{eq:nlin2}
\end{equation}
in 
\begin{equation}
\left(\mathbf{P}_{\mathrm{NL,3}}\right)_i= 
\epsilon_{0}\sum_{j,k,l}\chi^{(3)}_{ijkl} \,E_j \,E_k\, E_l=
\epsilon_{0}\chi^{(3)}_{ijkl} \,E_j \,E_k\, E_l,
\label{eq:nlin3}
\end{equation}
pri čemer smo uporabili Einsteinov zapis seštevanja po indeksih. Značilne vrednosti
susceptibilnosti v trdnih snoveh so $\chi^{(1)} \sim 1$, 
$\chi^{(2)} \sim 10^{-11}~\si{\metre/\volt}$ 
in $\chi^{(3)} \sim 10^{-22}~\si{\metre^2/\volt^2}$. Obravnavali bomo samo snovi, v katerih
ni izgub in so susceptibilnosti realne.

\begin{naloga}
Oceni gostoto svetlobnega toka, pri kateri postane nelinearni 
prispevek k polarizaciji znaten in velja 
 $$P_{NL}/P_L \sim 10^{-5}.$$
Tipične vrednosti so $\sim~1~\si{\giga\watt/\metre^2}$. Ker je to z navadnimi
svetili povsem nedosegljivo, je bilo mogoče nelinearne
optične pojave opazovati šele po iznajdbi laserjev.
\end{naloga}
 
Tenzor $\chi^{(2)}$ je od nič različen le v snoveh, ki nimajo centra inverzije. 
Ker lahko v produktu (enačba~\ref{eq:nlin2}) vrstni red $E_j E_k$ zamenjamo, mora biti
tenzor invarianten na zamenjavo
\begin{equation}
\chi_{ijk} = \chi_{ikj}.
\label{eq:chijk}
\end{equation}
Zato lahko vpeljemo poenostavljeni zapis, pri katerem prvi indeks 
prepišemo ($x=1$, $y=2$, $z=3$) in zadnja dva indeksa združimo, 
podobno kot pri elektro-optičnem pojavu (razdelek~\ref{chap:EO}).
Dogovorjene oznake so enake: $xx=1$, $yy=2$, $zz=3$, $yz=zy=4$, 
$xz=zx=5$, $xy=yx=6$. Namesto
splošnega tenzorja tretjega ranga smo uvedli matriko velikosti $3\times6$,
v kateri je zaradi simetrije navadno le nekaj komponent 
različnih od nič. 

Kadar je v snovi absorpcija dovolj majhna, lahko matriko poenostavimo
z dodatnim približkom, tako imenovano  
\index{Kleinmanova domneva} Kleinmanovo domnevo\footnote{D. A. Kleinman, Phys. Rev. $\mathbf{126}$, 1977 (1962).}.
Ta pravi, da je 
\begin{equation}
\chi_{ijk} = \chi_{ikj} = \chi_{kij} = \chi_{kji} = \chi_{jik} = \chi_{jki}.
\label{Klein}
\end{equation}
\begin{table}[!h]
 \centering
\begin{tabular}{|c|c|c|c|} \hline  
      Kristal & Grupa & Neničelne komponente tenzorja $\chi$ & Vrednosti ($10^{-12}~\si{\metre/\volt}$)\\ \hline
      BaTiO\index{BaTiO$_3$}$_3$ & 4mm & $\chi_{xxz} = \chi_{yyz} = \chi_{xzx} = \chi_{yzy} = 
      \chi_{15} = \chi_{24}$  &
	    $\chi_{15} = 42,6$ \\
	      & & $\chi_{zxx} = \chi_{zyy} = \chi_{31} = \chi_{32}$ &  $\chi_{31} = 45,2$ \\
	      & & $\chi_{zzz} = \chi_{33}$ & $\chi_{33} = 16,0$ \\ \hline
      KDP\index{KDP} & 
      $\overline{4}$2m & $\chi_{xyz} = \chi_{yxz} = \chi_{xzy} = \chi_{yzx} = \chi_{14} = \chi_{25}$  &
	    $\chi_{14} = 0,88$ \\
	    & & $\chi_{zxy} = \chi_{zyx} = \chi_{36}$ &  $\chi_{36} =1,12$ \\ \hline
      Telur\index{Telur} & 32 & $\chi_{xxx} = -\chi_{xyy} = -\chi_{yyx} = -\chi_{yxy} =$  & \\
      & &  = $\chi_{11} = -\chi_{12}=-\chi_{26}$  &
	    $\chi_{11} = 1300$ \\
	    & & $\chi_{xyz} = \chi_{xzy} = -\chi_{yxz}= - \chi_{yzx}= \chi_{14} = 
	    -\chi_{25}$ &  $\chi_{14} \approx 0$ 
	    \\ \hline
      LiNbO$_3$\index{LiNbO$_3$} & 3m & $\chi_{xxz} = \chi_{yyz} = \chi_{xzx} = \chi_{yzy} = \chi_{15} = \chi_{24}$  &
	     \\
	     & & $\chi_{zxx} = \chi_{zyy} = \chi_{31} = \chi_{32}$ &  $\chi_{31} = -11,9$ \\
	      & & $\chi_{zzz} = \chi_{33}$ & $\chi_{33} = 68,8$ \\
	    & &  $-\chi_{xxy} = - \chi_{xyx} = \chi_{yyy} = -\chi_{yxx}  = $ & \\
	    & & $=-\chi_{16} = \chi_{22}$ = $-\chi_{21}$  &
	    $\chi_{22}  = 5,52$ \\
\hline 
\end{tabular}
  \caption{Koeficienti nelinearne susceptibilnosti za nekaj izbranih 
  snovi}
  \index{Susceptibilnost!nelinearna} 
\label{table:chi}
\end{table}

Poglejmo primer. 
Vzemimo barijev titanat (BaTiO$_3$)\index{BaTiO$_3$} s točkovno grupo 4mm. To pomeni, da
ima 4-števno os simetrije in dve zrcalni ravnini, od katerih ena preslika $x \to -x$ ali $y \to -y$, 
druga pa $x\to y$ in $y\to x$. Od nič različni elementi susceptibilnosti so tako samo
\begin{equation}
\chi_{xxz} = \chi_{xzx} =   \chi_{yyz} = \chi_{yzy}, \quad  \chi_{zzz} \qquad \mathrm{in} 
\qquad \chi_{zxx} = \chi_{zyy}.   
\end{equation}
Z upoštevanjem Kleinmanove domneve se število različnih členov še zmanjša in ostaneta le dva
\begin{equation}
\chi_{xxz} = \chi_{xzx} = \chi_{yyz} = \chi_{yzy} =\chi_{zxx} = \chi_{zyy} \qquad \mathrm{in} \qquad \chi_{zzz}.   
\end{equation}
Primerjajmo rezultat s tabelo~(\ref{table:chi}).\footnote{A. Yariv in 
P. Yeh, {\it Photonics}, šesta izdaja, Oxford University Press (2007) in
{\it CRC Handbook of Chemistry and Physics}, CRC Press (2002).} V tabeli
so navedene izmerjene nelinearne susceptibilnosti\footnote{Izmerjene vrednosti, 
ki jih najdemo v literaturi, se od vira do vira pogosto znatno razlikujejo.} in vidimo, da Kleinmanova
domneva ni povsem točna, je pa razmeroma dober približek. 

\section{Nelinearni optični pojavi drugega reda}
\index{Nelinearna optika!drugega reda}
Vzemimo optično nelinearni kristal s $\chi^{(2)} \neq 0$. V smeri pravokotno 
glede na njegovo mejno ploskev naj vpadata dve valovanji s frekvencama\footnote{Tudi 
v tem poglavju bomo $\omega$ namesto krožna frekvenca pogosto imenovali zgolj frekvenca.}
$\omega_{1}$ in $\omega_{2}$. Zaradi nelinearne sklopitve nastajajo v snovi nova 
valovanja z različnimi kombinacijami frekvenc (slika~\ref{fig:nl2}).
Tako poleg valovanj z osnovnima frekvencama izhajajo iz kristala tudi 
valovanja pri dvakratnikih obeh vstopnih frekvenc, pri njuni vsoti, 
razliki in celo pri frekvenci nič. Oglejmo si te pojave podrobneje.
\begin{figure}[ht]
\centering
\def\svgwidth{140truemm} 
\input{slike/08_nl3.pdf_tex}
\caption{Shematski prikaz nastanka valovanj pri nelinearnih optičnih pojavih drugega reda (a)
in spekter izhodne svetlobe (b). Intenzitete izhodnih žarkov niso v merilu.}
\label{fig:nl2}
\end{figure}

\begin{remark}
Nastanku valovanja pri podvojeni frekvenci oziroma optičnemu frekvenčnemu podvajanju pravimo tudi
SHG\index{SHG|see {Frekvenčno podvajanje}} ({\it Second Harmonic 
Generation})\index{Frekvenčno podvajanje}, 
nastanku valovanja pri vsoti frekvenc SFG\index{SFG|see {Generacija vsote frekvenc}}
({\it Sum Frequency Generation})\index{Generacija vsote frekvenc}, 
nastanku valovanja pri razliki frekvenc DFG\index{DFG|see {Generacija razlike frekvenc}} 
({\it Difference Frequency Generation})\index{Generacija razlike frekvenc} in pojavu 
statičnega polja pri $\omega = 0$ optično usmerjanje\index{Optično usmerjanje}
({\it optical rectification}).  
\end{remark}

Navadna valovna enačba za opis svetlobe v snovi  
ne velja za opis pojavov pri velikih 
intenzitetah vpadnih valovanj, saj se pojav nelinearna polarizacija. Valovanje 
v takem primeru opišemo z nelinearno valovno 
enačbo\index{Valovna enačba!nelinearna}
\boxeq{8.3}{
\nabla^{2}\mathbf{E}-\frac{\epsilon}{c_0^{2}}{\frac{\partial^2\mathbf{E}}{\partial t^2}}=
\mu_{0}{\frac{\partial^2\mathbf{P}_{\textrm{NL}}}{\partial t^2}}.
}

\begin{naloga}
Iz Maxwellovih enačb (enačbe~\ref{eq:Maxwell1}--\ref{eq:Maxwell4}) izpelji 
nelinearno valovno enačbo (enačba~\ref{8.3}), pri čemer upoštevaj enačbo~(\ref{8.1}). 
Pomagaj si z identiteto
$$
\nabla \times (\nabla \times \mathbf{A}) = \nabla (\nabla \cdot \mathbf{A}) 
- \nabla^2 \mathbf{A}.
$$
\end{naloga} 

Nelinearne valovne enačbe na splošno ne znamo rešiti, zato se zatečemo k približkom.
Prva poenostavitev je omejitev na vzporedna vpadna žarka,
ki se širita v smeri osi $z$. Poleg tega se omejimo na izračun samo enega
nastalega valovanja in privzamemo, da je neodvisno od drugih nastalih valovanj.
Ta omejitev ni huda. Dokler so namreč amplitude nastalih valovanj majhne, 
jih lahko obravnavamo ločeno. Ni sicer nujno,
da so amplitude nastalih valov vedno majhne, vendar je lahko, kot bomo videli 
pozneje, le eno nastalo valovanje naenkrat po jakosti primerljivo z vpadnima. 

V snovi so tako prisotna tri valovanja:
dve vpadni in tretje, novo nastalo. Zapišemo jih z 
\begin{align}
\mathbf{E}_{1} & =  \frac{\mathbf{e}_{1}}{2}\left(A_{1}(z)\, 
e^{i(k_{1}z-\omega_{1}t)}+A_{1}^{*}(z)\, e^{-i(k_{1}z-\omega_{1}t)}\right),\nonumber \\
\mathbf{E}_{2} & =  \frac{\mathbf{e}_{2}}{2}\left(A_{2}(z)\, 
e^{i(k_{2}z-\omega_{2}t)}+A_{2}^{*}(z)\, e^{-i(k_{2}z-\omega_{2}t)}\right) \quad \mathrm{in} \nonumber \\
\mathbf{E}_{3} & =  \frac{\mathbf{e}_{3}}{2}\left(A_{3}(z)\, 
e^{i(k_{3}z-\omega_{3}t)}+A_{3}^{*}(z)\, e^{-i(k_{3}z-\omega_{3}t)}\right).
\end{align}
Ker valovna enačba (enačba~\ref{8.3}) ni linearna, smo polja 
zapisali v realni obliki s kompleksno konjugiranimi
deli. Upoštevali smo tudi,
da so zaradi nelinearnih pojavov amplitude $A$ funkcije kraja, za
katere privzamemo, da se le počasi spreminjajo. Njihova kompleksna vrednost
dopušča pojav dodatnega faznega zamika. Za valovna
števila velja $k_{n}^{2}=\epsilon_{n}\omega_n^{2}/c_0^{2}$,
pri čemer je $\epsilon_{n}$ dielektrična konstanta pri frekvenci
$\omega_{n}$ in polarizaciji $\mathbf{e}_{n}$, indeks $n = 1...3$ pa označuje
valovanje. S tem nastavkom vsako od treh valovanj
pri konstantni amplitudi reši linearni del valovne enačbe. 

Naša naloga
je ugotoviti, kako se zaradi nelinearnih pojavov spreminjajo amplitude posameznih valovanj.
Nastavek za polje, ki bo približno rešil nelinearno valovno enačbo, je tako
\begin{equation}
\mathbf{E}(z,t) = \sum_{n=1}^3 \frac{\mathbf{e}_{n}}{2}\left(A_{n}(z)\, 
e^{i(k_{n}z-\omega_{n}t)}+A_{n}^{*}(z)\, e^{-i(k_{n}z-\omega_{n}t)}\right).
\label{eq:nlnastavek}
\end{equation}
Izračunajmo najprej 
\begin{equation}
\nabla^{2}\mathbf{E}=-\sum_{n=1}^3 \frac{\mathbf{e}_{n}}{2}\left(k_{n}^{2}A_{n}(z)-2ik_{n}
\frac{dA_{n}(z)}{dz}\right)\, e^{i(k_{n}z-\omega_{n}t)}+\mbox{k.\,k.}
\label{8.5}
\end{equation}
S k.\,k. smo označili kompleksno konjugirani del. Upoštevali smo,
da se amplituda $A_{n}(z)$ le počasi spreminja s krajem in smo zato njen
drugi odvod po kraju zanemarili.
Izračunamo še drugi odvod po času 
\begin{equation}
\frac{\partial^2\mathbf{E}}{\partial t^2}=\sum_{n=1}^3 \frac{\mathbf{e}_{n}}{2}
\left(-\omega_n^2\right) \left(A_{n}(z)\, e^{i(k_{n}z-\omega_{n}t)}+\mbox{ k. k.}\right).
\label{8.5a}
\end{equation}
Nelinearna polarizacija vsebuje produkte polj, ki nihajo z
vsotami in razlikami parov frekvenc $\omega_{1}$, $\omega_{2}$ in
$\omega_{3}$.

Vstavimo nastavek za polje (enačba~\ref{eq:nlnastavek}) 
in dobimo\footnote{Spomnimo, da je $\chi^{(2)}:\mathbf{e}_{n}\,\mathbf{e}_{m}$ 
notranji produkt tenzorja z enotskima vektorjema polarizacije, 
katerega $i$-ta komponenta se izračuna kot $\sum_{jk}\chi^{(2)}_{ijk}\mathbf{e}_{nj}\,\mathbf{e}_{mk}$.
Indeksa $n$ in $m$ označujeta valovanje, $i, j$ in $k$ pa kartezične koordinate.}
\begin{multline}
\mathbf{P}_{\mathrm{NL}}= \epsilon_{0}\chi^{(2)}:\mathbf{E}\, \mathbf{E} =
\varepsilon_0 \sum_{n=1}^3 \sum_{m=1}^3 
 \left( \frac{1}{4} \chi^{(2)}:\mathbf{e}_{n}\,\mathbf{e}_{m}\right) 
 A_{n}(z)\,A_{m}(z) e^{i(k_{n}+k_{m})z-i(\omega_{n}+\omega_{m})t}+  \\
\left( \frac{1}{4} \chi^{(2)}:\mathbf{e}_{n}\,\mathbf{e}_{m}\right)
A_{n}(z)\,A_{m}^*(z) e^{i(k_{n}-k_{m})z-i(\omega_{n}-\omega_{m})t}+ \mathrm{k. k.}
\label{8.5b}
\end{multline}
Valovna enačba (enačba~\ref{8.3}) je izpolnjena ob vsakem času $t$, če se izrazi 
pri istih časovnih odvisnostih, to je pri istih frekvencah, ujemajo. Najprej 
zberemo člene pri $\omega_3 = \omega_1 + \omega_2$. Dobimo
\begin{equation}
ik_{3}\mathbf{e}_{3}\frac{dA_{3}}{dz}e^{ik_{3}z}=-\frac{\mu_{0} 
\varepsilon_0 \omega_{3}^{2}}{4}\chi^{(2)}:\mathbf{e}_{1}\mathbf{e}_{2}\,A_{1}\,A_{2}e^{i(k_{1}+k_{2})z}.
\label{8.7}
\end{equation}
Množimo obe strani skalarno z $\mathbf{e}_{3}$, upoštevamo zvezo med $k_{3}$ in $\omega_{3}$
in ravnamo podobno še za drugi dve valovanji. Dobimo sistem sklopljenih
enačb za amplitude valovanj v optično nelinearnem sredstvu
\boxeq{eq:nlAz}{
\frac{dA_{3}}{dz} &= \frac{i\omega_{3}\chi_{ef}}{4c_0 n_3} A_{1}\, A_{2}\, e^{-i\Delta kz}\\
\frac{dA_{2}}{dz} &= \frac{i\omega_{2}\chi_{ef}}{4c_0 n_2} A_{1}^*\, A_{3}\, e^{i\Delta kz}\\
\frac{dA_{1}}{dz} &= \frac{i\omega_{1}\chi_{ef}}{4c_0 n_1} A_{2}^*\, A_{3}\, e^{i\Delta kz}\label{eq:nlA3}.
}
Pri tem je $\Delta k$ razlika valovnih vektorjev
\begin{equation}
\Delta k = k_{3}-k_{1}-k_{2},
\end{equation}
s $\chi_{ef}$ pa smo označili efektivno susceptibilnost.\footnote{Glej npr. F. Zernike in J. E. Midwinter, 
{\it Applied Nonlinear Optics}, John Wiley \& Sons, Inc. (1973).}
Izračunamo jo kot \index{Susceptibilnost!efektivna}
\begin{equation}
\chi_{ef}=\mathbf{e}_{3}\cdot\chi:\,\mathbf{e}_{1}\,\mathbf{e}_{2} = 
\sum_{ijk} \chi_{ijk}^{(2)} e_{3i} e_{1j} e_{2k}.
\label{eq:chicomp}
\end{equation}
Ker polarizacijski vektorji niso nujno vzporedni s koordinatnimi osmi, $\chi_{ef}$ 
niso čiste kartezične komponente tenzorja nelinearne susceptibilnosti.
\begin{naloga}
Pokaži, da iz Kleinmanove domneve (enačba~\ref{Klein}) sledi, da so \index{Kleinmanova domneva}
efektivne susceptibilnosti $\chi_{ef}$ v vseh treh enačbah~(\ref{eq:nlAz}--\ref{eq:nlA3}) enake.
\end{naloga}
\begin{naloga}
Pokaži, da nastavek za polje v nelinearni snovi (enačba~\ref{eq:nlnastavek}) reši nelinearno
valovno enačbo (enačba~\ref{8.3}), in pokaži, da spreminjanje amplitude posameznih valovanj 
ustreza enačbam~(\ref{eq:nlAz}--\ref{eq:nlA3}).
\end{naloga}
Čeprav je $\omega_{3}-\omega_{2}-\omega_{1}=0$, je $\Delta k$ navadno različen od nič zaradi 
frekvenčne disperzije lomnega količnika. Videli bomo, da je to ključnega pomena 
pri vrsti nelinearnih optičnih pojavov. 

Zapisani sistem diferencialnih enačb (enačbe~\ref{eq:nlAz}--\ref{eq:nlA3}) opisuje več pojavov, 
odvisno od začetnih pogojev in relativnih intenzitet. Opisali bomo nekaj
najpomembnejših primerov.

\section{Optično frekvenčno podvajanje}
\label{chap:SHG}
Obravnavajmo optično nelinearno sredstvo, na katerega vpadata valovanji ${\mathbf E}_1$ in
$\mathbf{E}_2$ z enakima frekvencama $\omega_{1}=\omega_{2}=\omega$. Vpadni valovanji
razlikujemo zaradi možnosti dveh različnih polarizacij. Izhodna svetloba vsebuje valovanje
s frekvenco $\omega_{3}=2\omega$, zato govorimo o frekvenčnem 
podvajanju\index{Frekvenčno podvajanje}. To je
najpreprostejši in tudi najpomembnejši nelinearni optični pojav.\footnote{P. A. Franken et al, Phys. Rev. Lett.
$\mathbf{7}$, 118 (1961).} 
Pogosto ga uporabljamo za pridobivanje laserske svetlobe pri krajših valovnih dolžinah, na primer
pri Nd:YAG laserju\index{Laser!Nd:YAG}, ko infrardeče izhodno valovanje ($1064~\si{\nano\metre}$) 
pretvorimo v vidno svetlobo zelene barve ($532~\si{\nano\metre}$).\index{Infrardeče valovanje} 

Zanima nas, kako se $A_{3}(z) = A_{2\omega}(z)$ spreminja vzdolž nelinearnega kristala
pri začetnem pogoju $A_{2\omega}(0)=0$.
Privzamemo, da se pretvori le manjši del vpadnega energijskega toka in da ostaneta 
amplitudi $A_{1}=A_{2}=A_0$ približno konstantni. Tedaj lahko
enačbo za $A_{3}(z)$ (enačba~\ref{eq:nlAz}) brez težav integriramo do dolžine kristala $L$ in 
dobimo
\begin{equation}
A_{2\omega}(L)=\frac{i\omega \chi_{ef} A_0^2}{2c_0 n_{2\omega}}
\,e^{-i\Delta kL/2}\, \frac{\sin\left(\frac{\Delta k L}{2}\right)}{\frac{\Delta kL}{2}}L,
\label{8.9}
\end{equation}
kjer smo z $n_{2\omega}$ označili lomni količnik pri dvojni frekvenci.
Iz tega izraza izračunamo izhodno gostoto svetlobnega toka pri dvojni
frekvenci (enačba~\ref{eq:jcw})
\begin{equation}
j_{2\omega}(L) =\frac{1}{2}\epsilon_{0}n_{2\omega}c_0|A_3|^2 = 
\frac{\omega^2 \chi_{ef}^2}{2 n_{2\omega} n_\omega^2c_0^3\varepsilon_0} j_\omega^2 L^2
\left(\frac{\sin\left(\frac{\Delta k L}{2}\right)}{\frac{\Delta kL}{2}}\right)^2.
\label{8.10}
\end{equation}
\vglue-3truemm
\begin{remark}
Pri izpeljavi frekvenčnega podvajanja iz enačb za nelinearne pojave drugega
reda (enačbe~\ref{eq:nlAz}--\ref{eq:nlA3}) moramo biti pazljivi. 
Uporabili smo splošne enačbe in predpostavili, da je 
vpadno valovanje se\-stav\-lje\-no iz dveh ločenih valovanj s frekvenco $\omega$
z gostoto svetlobnega toka $j_\omega$. Lahko pa frekvenčno podvajanje obravnavamo
z enim vpadnim valovanjem s frekvenco $\omega$ in gostoto svetlobnega toka $2j_\omega$, 
ki nelinearno interagira samo s sabo. Takrat je zapis enačb za predfaktor
drugačen, končni rezultat pa seveda enak. 
\end{remark}

Gostota energijskega toka frekvenčno podvojene svetlobe torej narašča s kvadratom
intenzitete vpadne svetlobe. Naj bo $S$ presek snopa. Potem je razmerje med 
energijskim tokom pri podvojeni in osnovni frekvenci (izkoristek pretvorbe) enako
\boxeq{8.11}{
\frac{P_{2\omega}}{P_{\omega}}=
\frac{\omega^2 \chi_{ef}^2}{2 S n_{2\omega} n_\omega^2c_0^3\varepsilon_0} P_\omega L^2
\left(\frac{\sin\left(\frac{\Delta k L}{2}\right)}{\frac{\Delta kL}{2}}\right)^2.
}

Odvisnost faktorja v oklepaju od $\Delta kL/2$ je prikazana 
na sliki~\ref{fig:shg2}. Vidimo, da je faktor največji pri $\Delta kL = 0$, potem pa zelo hitro
pade na zelo majhne vrednosti. 
\begin{figure}[ht]
\centering
\def\svgwidth{72truemm} 
\input{slike/08_shg2.pdf_tex}
\caption{Izkoristek pretvorbe v frekvenčno podvojeno valovanje $j_{2\omega}/j_\omega$ je 
sorazmeren s funkcijo $(\sin(x)/x)^2$,
kjer je $x = \Delta k L/2$ (enačba~\ref{8.11}). 
Pri tem $\Delta k$ označuje razliko valovnih vektorjev in
$L$ prepotovano pot v kristalu.}
\label{fig:shg2}
\end{figure}

Izkoristek pretvorbe v frekvenčno podvojeno valovanje je tako velik le pri  majhnih vred\-no\-stih 
$\Delta kL$. To lahko dosežemo z majhno debelino kristala, veliko bolj smiselno
pa je poiskati pogoje, pri katerih je $\Delta k = 0$. Če uspemo izpolniti pogoj, da
se faze valovanj ujemajo, je vrednost faktorja 
$\sin(\Delta kL/2)/(\Delta kL/2)=1$ in neodvisna od dolžine poti $L$.

V tem primeru izkoristek pretvorbe narašča sorazmerno s kvadratom poti
\begin{equation}
\frac{P_{2\omega}}{P_{\omega}}=
\frac{\omega^2 \chi_{ef}^2}{2 S n_{2\omega} n_\omega^2c_0^3\varepsilon_0} P_\omega L^2.
\label{eq:shgl2}
\end{equation}
Za čim večjo pretvorbo v frekvenčno podvojeno valovanje je torej treba 
izpolniti pogoj $\Delta k = 0$\index{Ujemanje faz}. Kako to naredimo,
bomo spoznali v nadaljevanju.

\begin{naloga}
\label{deplet}
Pokazali smo, da izkoristek pretvorbe v frekvenčno podvojeno valovanje 
pri $\Delta k = 0$  narašča sorazmerno s kvadratom 
dolžine kristala (enačba~\ref{eq:shgl2}). Takšna odvisnost velja, 
dokler je $j_{2\omega}$ bistveno manjša od $j_{\omega}$ oziroma $A_3 \ll A_1, A_2$ (slika~\ref{fig:shg2dep}). 
Pokaži, da v nasprotnem primeru gostota svetlobnega toka frekvenčno
podvojenega valovanja $j_{2\omega}(L)$ narašča kot
\begin{equation}
j_{2\omega} (L) = j_\omega \tanh^2 \left(\chi_{ef}\omega L \sqrt{\frac{j_\omega}
{2 n_{2\omega} n_\omega^2 c_0^3 \varepsilon_0}} \right) = j_\omega \tanh^2(\kappa L).
\label{eq:nondepl}
\end{equation}
Namig: upoštevaj, da se celotna energija ohranja.
\end{naloga}

\begin{figure}[ht]
\centering
\def\svgwidth{72truemm} 
\input{slike/08_shg_depletion.pdf_tex}
\caption{Izkoristek pretvorbe v frekvenčno podvojeno valovanje $j_{2\omega}/j_\omega$ 
pri $\Delta k = 0$.
Če privzamemo, da se gostota svetlobnega toka osnovnega žarka ne zmanjšuje, 
je odvisnost parabolična (rdeča krivulja), kar 
je dober približek le za majhne gostote toka. Bolj natančen izračun pokaže, da je izkoristek 
pretvorbe sorazmeren s $\tanh^2(\kappa L)$ (enačba~\ref{eq:nondepl}, črna krivulja).}
\label{fig:shg2dep}
\end{figure}

Kaj pa se zgodi, kadar pogoj ujemanja faz ni izpolnjen in 
 $\Delta k \neq 0$? Takrat lahko  $L^2$ v enačbi~(\ref{8.11})
okrajšamo in izkoristek pretvorbe z naraščajočim
$L$ sinusno niha med nič in neko največjo vrednostjo. Omenjen pojav lahko opazimo, če
uporabimo klinast vzorec, ki se mu debelina spreminja, ali če vzorec sučemo 
in tako spreminjamo razliko faz. Pojav, imenujemo ga Makerjeve 
oscilacije\footnote{P. D. Maker et al., Phys. Rev. Lett. $\mathbf{8}$, 21 (1962).}, 
uporabljamo za merjenje nelinearne susceptibilnosti kristalov.\index{Makerjeve oscilacije}

\subsection*{Ujemanje faz}
\index{Ujemanje faz}
Poglejmo, kako lahko dosežemo ujemanje faz, ki je nujno za učinkovito optično
frekvenčno podvajanje. Spomnimo se, da je pogoj za ujemanje faz 
\begin{equation}
\Delta k = k_3 - k_1 -k_2 = k_3(2\omega) - k_1(\omega) -k_2(\omega) = 
\frac{2\omega}{c_0} n_3(2\omega) - \frac{\omega}{c_0} n_1(\omega)- \frac{\omega}{c_0} n_2(\omega) =0.
\end{equation}
Pri tem smo lomnim količnikom pripisali ustrezno frekvenco svetlobe in jo navedli v oklepaju. Dobimo
pogoj za ujemanje faz 
\boxeq{eq:dk0}{
n_1(\omega) + n_2(\omega) = 2n_3(2\omega).
}
Da lahko zadostimo temu pogoju, izkoristimo dvojni lom\index{Dvolomnost} v 
anizotropnih kristalih. Omejimo se le na optično 
enoosne kristale\index{Dvolomnost!enoosne snovi} brez absorpcije in z 
normalno disperzijo, pri katerih oba lomna količnika naraščata s frekvenco.  

Za razumevanje je najbolj nazoren grafični prikaz (slika~\ref{fig:dk}), pri katerem
rišemo presek ploskve valovnega vektorja z ravnino, določeno z optično osjo in valovnim
vektorjem (glej razdelek~\ref{chap:anizotropni}). 
Za vsako smer valovnega vektorja obstajata dve rešitvi:
rednemu žarku, katerega polarizacija je pravokotna na omenjeno ravnino,
ustreza krožnica s polmerom $n_o$, izrednemu, katerega polarizacija leži v ravnini, 
pa elipsa s polosema $n_o$ in $n_e$.\index{Lomni količnik} 
Rdeča barva nakazuje presek ploskve pri vpadni frekvenci, modra pa pri podvojeni. 
Ekscentričnost elipse in frekvenčna disperzija sta zaradi večje nazornosti na sliki 
močno pretirani. 

Podrobneje poglejmo primer s slike~\ref{fig:dk}\,a, za katerega velja $n_e>n_o$. 
Opazimo, da se v neki točki rdeča elipsa, ki ustreza vpadnemu valovanju s frekvenco $\omega$, 
seka z modro krožnico, ki ustreza valovanju s podvojeno frekvenco $2\omega$. Pri kotu 
$\vartheta_m$ je torej redni lomni 
količnik pri dvojni frekvenci enak izrednemu pri osnovni
frekvenci. Če izberemo izredno polarizacijo vpadnega valovanja, je za podvojeno 
valovanje z redno polarizacijo pri kotu $\vartheta_m$ izpolnjen pogoj ujemanja 
faz~(enačba~\ref{eq:dk0}). Takrat leži polarizacija vpadnega valovanja v ravnini,
ki jo določata smer optične osi in smer valovnega vektorja, polarizacija 
izhodnega frekvenčno podvojenega žarka pa 
je pravokotna na optično os. Zapišimo ta razmislek še z enačbo.

\begin{figure}[ht]
\centering
\def\svgwidth{140truemm} 
\input{slike/08_phasematch.pdf_tex}
\caption{Štirje primeri, pri katerih je izpolnjen pogoj za ujemanje faz pri kotu $\vartheta_m$. 
Ujemanje faz prve vrste za pozitivno anizotropno snov (a), 
ujemanje faz prve vrste za negativno anizotropno snov (b) ter 
ujemanje faz druge vrste za pozitivno (c) in negativno (d) anizotropno snov.}
\label{fig:dk}
\end{figure}

V obravnavanem primeru mora biti lomni količnik za redno polarizirano valovanje pri 
podvojeni frekvenci $n_o(2\omega)$ enak lomnemu količniku za izredno 
polarizirano valovanje pri osnovni frekvenci $n(\omega)$. Lomni količnik
za izredno valovanje je seveda odvisen od kota (enačba~\ref{eq:izreden})
\begin{equation}
\frac{1}{n^2(\omega,\vartheta)}=
\frac{\cos^{2}\vartheta}{n_{o}^2(\omega)}+\frac{\sin^{2}\vartheta}{n_{e}^2(\omega)}.
\label{8.12}
\end{equation}
Ko izenačimo $n_o(2\omega) = n(\omega,\vartheta_m)$, dobimo
\begin{equation}
\cos^{2}\vartheta_m=\frac{(n_o(2\omega))^{-2}-(n_{e}(\omega))^{-2}}
{(n_{o}(\omega))^{-2}-(n_{e}(\omega))^{-2}},
\label{8.13}
\end{equation}
iz katerega lahko izračunamo kot $\vartheta_m$, pri katerem pride do ujemanja faz.
\vglue-2truemm
\begin{naloga}
Pokaži, da v primeru negativne anizotropije (slika~\ref{fig:dk}\,b) 
pogoj za ujemanje faz zapišemo kot
\begin{equation}
\cos^{2}\vartheta_m=\frac{(n_o(\omega))^{-2}-(n_{e}(2\omega))^{-2}}
{(n_{o}(2\omega))^{-2}-(n_{e}(2\omega))^{-2}}.
\label{8.13a}
\end{equation}
\end{naloga}

S slik~\ref{fig:dk} c in d razberemo, da obstaja še en primer, pri 
katerem je izpolnjen pogoj za ujemanje faz. Kot zgled t.\ i.\ ujemanja faz druge vrste
obravnavajmo primer na sliki (c).
Vzemimo različno polarizirani vhodni valovanji z ustreznima različnima lomnima
količnikoma $n_o(\omega)$ in $n(\omega,\vartheta)$, ki ju na sliki predstavljata
rdeča krog in elipsa.
Enačba~(\ref{eq:dk0}) je izpolnjena, kadar je povprečje lomnih količnikov
vhodnih valovanj  enako lomnemu količniku frekvenčno podvojenega žarka $n_o(2\omega)$. 
To se zgodi pri tistem kotu $\vartheta_m$, pri katerem je modra krožnica ravno na 
sredini med rdečo krožnico in rdečo elipso. Za praktično uporabo je ta izbira, kadar obstaja,
celo ugodnejša, saj je pri njej kot ujemanja faz bliže $\pi/2$. 
Ujemanje faz je zato manj občutljivo na majhna odstopanja v kotu ali na temperaturne
spremembe lomnih količnikov. 

\subsection*{Efektivna susceptibilnost}
\index{Susceptibilnost!efektivna}
Na izhodno moč frekvenčno podvojenega snopa (enačba~\ref{8.11}) 
poleg faznega faktorja bistveno vpliva tudi efektivna 
susceptibilnost $\chi_{ef}$ (enačba~\ref{eq:chicomp}). Ta je odvisna 
od polarizacij vhodnega in izhodnega žarka ter seveda od simetrije kristala. 
Ugotovili smo, da je v optično enoosnih kristalih pogoj ujemanja faz 
določen s smerjo valovnega vektorja glede na smer optične osi 
$\vartheta_m$ (enačbi~\ref{8.13} in \ref{8.13a}). Zaradi simetrije
to pomeni, da obstaja cel stožec smeri, v katerih se faze ujemajo.
Drugi kot, ki določa smer širjenja valovanja v ravnini, pravokotni na optično os, 
izberemo tako, da izkoristimo največje komponente nelinearne 
susceptibilnosti.\footnote{Glej npr. C. C. Davis, {\it Lasers and Electro-Optics}, 
Cambridge University Press (2006).}
\newpage

Poglejmo primer KDP (KH$_{2}$PO$_{4}$)\index{KDP}, ki je negativno anizotropen 
z vrednostmi $n_o(\omega) = 1,4942$, 
$n_e(\omega) = 1,4603$, $n_o(2\omega) = 1,5129$ in $n_e(2\omega) = 1,4709$.\footnote{A. Yariv in 
P. Yeh, {\it Photonics}, šesta izdaja, Oxford University Press (2007).}
Podatki so za valovno dolžino osnovnega snopa 1064~nm. 
Zaradi negativne anizotropije za izračun kota ujemanja faz 
uporabimo enačbo~(\ref{8.13a}) in dobimo $\vartheta_m = 41,25^\circ$. 
Poleg tega iz tabele~(\ref{table:chi}) razberemo, da ima nelinearna susceptibilnost v tetragonalni
simetriji $\bar{4}2m$ od nič različne komponente $\chi_{xyz}$, $\chi_{xzy}$, $\chi_{yxz}$,
$\chi_{yzx}$, $\chi_{zxy}$ in $\chi_{zyx}$.
Zaradi poenostavitve privzamemo, da so njihove vrednosti enake. 

Označimo enotski vektor v smeri valovnih vektorjev osnovnega in frekvenčno 
podvojenega valovanja s $\mathbf{s}$. Pomagamo si s sliko~\ref{fig:chi} in zapišemo 
vektor $\mathbf{s}$, pri čemer $\varphi$ označuje kot med osjo $x$ in projekcijo 
$\mathbf{s}$ na ravnino $xy$
\begin{equation}
\mathbf{s}=(\cos\varphi\sin\vartheta_m,\sin\varphi\sin\vartheta_m,\cos\vartheta_m).
\label{8.14}
\end{equation}

\begin{figure}[ht]
\centering
\def\svgwidth{80truemm} 
\input{slike/08_chi.pdf_tex}
\caption{K izračunu efektivne susceptibilnosti. Črtkan krog opisuje osnovno ploskev
stožca, ki je določen s $\vartheta_m$, in pikčast krog njegovo projekcijo
na ravnino $xy$. Rdeč vektor označuje polarizacijo redno polariziranega valovanja in
moder polarizacijo izredno polariziranega valovanja.}
\label{fig:chi}
\end{figure}

Naša naloga je poiskati kot $\varphi$, pri katerem je 
$\chi_{ef}$ največji in s tem največja tudi moč frekvenčno podvojenega valovanja.
Iz pogoja za ujemanje faz smo določili, da mora biti vpadna svetloba redno polarizirana in 
izhodna frekvenčno podvojena izredno polarizirana. Pri tem je redna polarizacija pravokotna na 
ravnino, ki jo tvorita os $z$ in vektor $\mathbf{s}$. Zapišemo jo kot
\begin{equation}
\mathbf{e}_o=(e_{ox}, e_{oy}, e_{oz}) = (\sin\varphi,-\cos\varphi,0).
\label{8.15}
\end{equation}
To najlažje preverimo, tako da postavimo vektor $\mathbf{s}$ enkrat v ravnino $xz$ in
drugič v ravnino $yz$. Izredna polarizacija leži v ravnini, ki jo tvori 
vektor $\mathbf{s}$ z osjo $z$, hkrati pa je približno pravokotna na vektor $\mathbf{s}$. 
Majhno odstopanje zaradi anizotropije kristala tukaj zanemarimo. Izredno polarizacijo 
tako zapišemo kot  
\begin{equation}
\mathbf{e}_e=(e_{ex}, e_{ey}, e_{ez}) 
=(-\cos \varphi \cos \vartheta_m,-\sin \varphi \cos \vartheta_m ,\sin \vartheta_m).
\label{8.15a}
\end{equation}
Zdaj lahko izračunamo efektivno susceptibilnost (enačba~\ref{eq:chicomp}), 
pri čemer upoštevamo, da sta žarka 1 in 2 pri osnovni frekvenci redno polarizirana, medtem ko
žarek z oznako 3 opisuje izredno polariziran žarek pri podvojeni frekvenci
\begin{equation}
\chi_{ef} = \sum_{ijk} \chi_{ijk} e_{3i} e_{1j} e_{2k} = \sum_{ijk} \chi_{ijk} e_{ei} e_{oj} e_{ok}.
\end{equation}
Krajši račun pokaže, da je zaradi oblike tenzorja nelinearne susceptibilnosti v izbranem 
primeru od nič različna le ena komponenta nelinearne polarizacije, komponenta $z$. Zapišemo
\begin{equation}
\chi_{ef} = \chi_{zxy} e_{ez} e_{ox} e_{oy} + \chi_{zyx} e_{ez} e_{oy} e_{ox}.
\end{equation}
Vstavimo komponente vektorjev $\mathbf{e}_o$ in $\mathbf{e}_e$ (enačbi~\ref{8.15} in \ref{8.15a}) in dobimo
\begin{align}
P_{z}^{2\omega}=- 2\varepsilon_0\, \chi_{zxy}E_{0}^2\cos\varphi\sin\varphi
\sin\vartheta_m = - \varepsilon_0\, \chi_{zxy}E_{0}^2\sin(2\varphi) \sin\vartheta_m.
\label{8.151}
\end{align}
Nelinearna polarizacija je največja pri $\varphi=\pi/4$, največji $\chi_{ef}$  pa je 
\begin{equation}
\chi_{ef}= 
\sin\vartheta_m \chi_{zxy} \approx 0,66\, \chi_{zxy} \approx 
0,74~\si{\pico\metre/\volt}.
\label{8.16}
\end{equation}

\begin{naloga}
Izračunaj največjo efektivno nelinearno susceptibilnost za
frekvenčno podvajanje svetlobe z valovno
dolžino $10~\si{\micro\metre}$ v kristalu telurja s simetrijsko grupo 32 (glej tabelo~\ref{table:chi}). 
Lomni količniki: $n_o(\omega) = 4,7969$, 
$n_e(\omega) = 6,2455$, $n_o(2\omega) = 4,8657$ in $n_e(2\omega) = 6,3152$.\index{Telur}
\end{naloga}

\begin{remark}
Namesto zvezne svetlobe za optično podvajanje frekvenc pogosto uporabimo kratke laserske sunke, saj je 
vršna moč v njih zelo velika in je zato velika tudi pretvorba v frekvenčno podvojen signal.
Vendar je treba biti pazljiv, saj lahko zaradi disperzije grupne hitrosti osnovni in 
podvojeni signal ne potujeta z enakima hitrostma. Navadno podvojeni signal potuje 
počasneje in zaostaja za osnovnim, zato lahko iz kristala izhaja razmeroma sploščen in 
precej razvlečen frekvenčno podvojen sunek svetlobe. Pojav je izrazit predvsem pri podvajanju v
ultravijolični del spektra.\index{Ultravijolično valovanje}
\end{remark}

\section{Frekvenčno podvajanje Gaussovih snopov}
\index{Frekvenčno podvajanje!Gaussovih snopov}
Doslej smo vpadno in frekvenčno podvojeno svetlobo obravnavali kot ravni valovanji,
ki sta bili razsežni v prečni smeri. Izračunali smo, da v primeru \index{Ujemanje faz}
ujemanja faz ($\Delta k=0$)
moč frekvenčno podvojene svetlobe narašča s kvadratom dolžine poti po nelinearnem
sredstvu. Pretvorba v frekvenčno podvojeno svetlobo je po enačbi~(\ref{8.11}) tem
učinkovitejša, čim večja je gostota svetlobnega toka pri osnovni frekvenci.
Zato v praksi vpadno svetlobo vselej zberemo in tako povečamo gostoto toka. 
Pri tem moramo paziti, da je nelinearni kristal odporen proti poškodbam
zaradi velike gostote svetlobnega toka. Odpornost in možnost izpolnitve kriterija ujemanja 
faz sta poglavitna kriterija pri izbiri snovi za frekvenčno podvajanje. 

Poglejmo, kako se enačbe spremenijo, če je vpadni snop pri osnovni 
frekvenci Gaussove oblike\index{Gaussov snop!frekvenčno podvajanje}. 
Rezultat lahko ocenimo, če vzamemo, da je
efektivna dolžina za pretvorbo $L$ kar enaka dolžini območja bližnjega polja; izven tega območja je 
gostota toka znatno manjša, s tem pa tudi izkoristek pretvorbe v 
frekvenčno podvojeni snop.
Celotna  dolžina $L$ je (enačba~\ref{eq:z0})
\begin{equation}
L=2z_{0}=2\frac{\pi w_{0}^{2}}{\lambda/n} = \frac{n w_0^2 \omega}{c_0}  \qquad \mathrm{in~~zato} \qquad 
w_{0}^{2} = \frac{c_0 L}{n \omega}.
\label{SHGG}
\end{equation}
Pri zapisu preseka vpadnega snopa upoštevajmo še faktor ena polovica, do katerega 
pridemo, če integriramo gostoto svetlobnega toka
snopa po celotni površini (glej nalogo~\ref{naloga-širina-snopa}). Sledi
\begin{equation}
S=\frac{1}{2}\pi w_{0}^{2} = \frac{\pi c_0 L}{2 n \omega}.
\end{equation}
Večja ko je dolžina $L$, na kateri pride do frekvenčnega podvajanja, 
večji je tudi presek snopa $S$ in zato intenziteta svetlobe manjša, kar zmanjša
učinek pretvorbe v frekvenčno podvojeno valovanje.
V enačbi~(\ref{8.10}) upoštevamo ujemanje faz in $S$ pri podvojeni frekvenci. Dobimo
\boxeq{8.17}{
\frac{P_{2\omega}}{P_{\omega}}=
\frac{\omega^3 \chi_{ef}^2}{2\pi n_{2\omega} n_\omega c_0^4\varepsilon_0} P_\omega\, L.
}
Ob optimalnem zbiranju je izkoristek pretvorbe torej sorazmeren z dolžino kristala.
\begin{naloga}
Naj na $1~\si{\centi\metre}$ dolg kristal KH$_{2}$PO$_{4}$\index{KDP} vpada svetloba
z valovno dolžino $1,06~\si{\micro\metre}$ in vhodno močjo $P_\omega = 5~\si{\kilo\watt}$.
Efektivna nelinearna susceptibilnost je $\chi_{ef}=7\cdot10^{-13}~\si{\metre/\volt}$, 
$\Delta k=0$ in $n=1,5$. Izračunaj
faktor pretvorbe v frekvenčno podvojeno svetlobo.
Da je Rayleighova dolžina $2z_{0}=1~\si{\centi\metre}$, mora biti polmer
grla okoli $40~\si{\micro\metre}$. Gostota svetlobnega toka v kristalu je pri
tem $2\cdot10^{8}~\si{\watt/\centi\metre^{2}}$, kar je že blizu praga za poškodbe,
predvsem na vstopni ali izstopni ploskvi. 
\end{naloga}

\section{{*}Račun podvajanja Gaussovih snopov}
\index{Gaussov snop!frekvenčno podvajanje}
V prejšnjem razdelku smo grobo ocenili vpliv oblike Gaussovih snopov
na frekvenčno podvajanje. Naredimo zdaj še natančnejši izračun. Vrnimo se k valovni
enačbi (enačba~\ref{8.3}), vpadna snopa naj bosta pri frekvencah
$\omega_{1}$ in $\omega_{2}$ in nastajajoč snop pri frekvenci
$\omega_{3}=\omega_{1}+\omega_{2}$.
Vsako od polj ($i=1,2,3$) naj ima obliko 
\begin{equation}
\mathbf{E}_{i}  = \frac{\mathbf{e}_{i}}{2}\left(\tilde{A}_{i}(r,z)\, 
e^{i(k_{i}z-\omega_{i}t)}+\tilde{A}_{i}^{*}(r,z)\, e^{-i(k_{i}z-\omega_{i}t)}\right),
\end{equation}
pri čemer je $\tilde{A}(r,z)$ funkcija tako vzdolžne kot tudi prečne koordinate. Privzamemo, 
da se vzdolž smeri  $z$ le počasi spreminja.
Zaradi poenostavljenega zapisa vpeljemo nove spremenljivke 
\begin{equation}
\psi_i = \sqrt{\frac{n_i}{\omega_i}}\tilde{A}_i.
\end{equation}
Tako je nastavek za jakost električnega polja
\begin{equation}
\mathbf{E}_{i}=\frac{\mathbf{e}_{i}}{2}\sqrt{\frac{\omega_{i}}{n_{i}}}\psi_{i}(r,z)
e^{i(k_{i}z-\omega_{i}t)}+\mbox{ k. k.}
\label{8.18}
\end{equation}
Vstavimo nastavek (enačbe~\ref{8.18}) v valovno
enačbo (enačba~\ref{8.3}) in ločimo na levi in desni strani člene z enako časovno odvisnostjo.
Zaradi počasnega spreminjanja vzdolž smeri $z$ smo zanemarili druge odvode 
$\psi$ po $z$. Od tod sledi sklopljen sistem obosnih enačb 
\begin{align}
\nabla_{\perp}^{2}\psi_{1}+2ik_{1}\psi_{1}^{\prime} & =  -
\frac{k_{1}}{2}\kappa\psi_{2}^{\ast}\psi_{3}e^{i\Delta kz},\label{SHGGAuss-1}\\
\nabla_{\perp}^{2}\psi_{2}+2ik_{2}\psi_{2}^{\prime} & =  -
\frac{k_{2}}{2}\kappa\psi_{1}^{\ast}\psi_{3}e^{i\Delta kz} \qquad \mathrm{in}\\
\nabla_{\perp}^{2}\psi_{3}+2ik_{3}\psi_{3}^{\prime} & =
 - \frac{k_{3}}{2}\kappa\psi_{1}\psi_{2}e^{-i\Delta kz}
\label{SHGGauss_3}
\end{align}
s pripadajočim sistemom kompleksno konjugiranih enačb. S črtico smo označili odvajanje po $z$. 
Pri tem je 
\begin{equation}
\kappa=\frac{\chi_{ef}}{c_0} \sqrt{\frac{\omega_{1}\omega_{2}\omega_{3}}{n_{1}n_{2}n_{3}}}.
\label{8.20}
\end{equation}
Sistem enačb~(\ref{SHGGAuss-1}--\ref{SHGGauss_3}) je očitno
posplošitev sistema enačb~(\ref{eq:nlAz}--\ref{eq:nlA3}) za primer, ko je valovanje odvisno
tudi od prečne koordinate. Reševanje tega nelinearnega sistema parcialnih
diferencialnih enačb je na splošno zelo zapleteno.

Poglejmo najenostavnejši primer frekvenčnega podvajanja, ko je 
$\omega_{3}=2\omega_{1}=2\omega$.
Vpadna snopa naj bosta enaka in Gaussove oblike~(enačba~\ref{eq:gaussov-snop}), 
njuna amplituda  naj bo enaka $A_1$
\begin{equation}
\psi_{1} = \psi_2 = A_{1}\frac{1}{1+iz/z_1}
\exp\left(-\frac{r^{2}}{w_1^{2}(z)}+\frac{ik_1r^{2}}{2R_1(z)}\right).
\label{8.21}
\end{equation}
Privzamemo, da je izpolnjen pogoj za ujemanje faz\index{Ujemanje faz} 
($\Delta k=0$) in da je pretvorba dovolj majhna, da zmanjševanja $\psi_{1}$
ni treba upoštevati. Tudi za podvojeni snop privzamemo Gaussovo 
obliko, vendar naj njegova amplituda $A_3$ le počasi narašča. Zapišemo ga kot
\begin{equation}
\psi_{3}=A_{3}(z)\psi_{3H}(z,r)=A_{3}(z)\frac{1}{1+iz/z_{3}}
\exp\left(-\frac{r^{2}}{w_{3}^{2}(z)}+\frac{ik_{3}r^{2}}{2R_{3}(z)}\right),
\label{8.22}
\end{equation}
pri čemer $\psi_{3H}$ reši homogeno obosno valovno 
enačbo (enačba~\ref{eq:obosna-valovna-enacba}). Ko izraza za $\psi_{1}$
in $\psi_{3}$ vstavimo v enačbo (\ref{SHGGauss_3}),
ostane na levi le člen oblike $2ik_{3}A_{3}^{\prime}(z)\psi_{3H}$. Tako dobimo pogoj
\begin{multline}
A_{3}^{\prime}(z)\frac{1}{1+iz/z_3}\exp\left(-\frac{r^{2}}{w_{3}^{2}(z)}+\frac{ik_{3}r^{2}}
{2R_{3}(z)}\right)=\\
\frac{i\kappa}{4}A_{1}^{2}\frac{1}{(1+iz/z_{1})^{2}}\exp\left(-\frac{2r^{2}}
{w_{1}^{2}(z)}+\frac{ik_{1}r^{2}}{R_{1}(z)}\right).
\label{8.23}
\end{multline}
Poiščimo rešitev te enačbe v obliki, za katero velja $w_{30}^{2}=w_{10}^{2}/2$. Tedaj je 
\begin{equation}
z_{3}=\frac{k_{3}w_{30}^{2}}{2}=\frac{2k_{1}w_{10}^{2}}{4}=z_{1}
\end{equation}
in je tudi $w_{3}^{2}(z)=w_{1}^{2}(z)/2$. Poleg tega je $R_{3}(z)=R_{1}(z)$
in lahko na obeh straneh krajšamo eksponentna faktorja. Ostane 
\begin{equation}
A_{3}^{\prime}(z)=\frac{i\kappa}{4}A_{1}^{2}\frac{1}{1+iz/z_1}.
\label{8.24}
\end{equation}
Enačbo seveda brez težav integriramo. Naj bo grlo vpadnega
snopa ravno na sredini nelinearnega sredstva, tako da integriramo
od $-L/2$ do $L/2$
\begin{align}
A_{3}(L) & =  \frac{i\kappa}{4}A_{1}^{2}\int_{-L/2}^{L/2}\frac{dz}{1+iz/z_1} 
  = \frac{\kappa}{4}A_{1}^{2}z_{1}\ln\frac{1+i\frac{L}{2z_{1}}}{1-i\frac{L}{2z_{1}}}= \nonumber \\
 & =  \frac{\kappa}{2}A_{1}^{2}z_{1}\arctan\frac{L}{2z_{1}}\;.
\end{align}
Moč Gaussovega snopa je
\begin{equation}
P_{i}=\frac{1}{2}\pi w_{i0}^{2} \frac{1}{2}c_0 n_i \epsilon_{0}E_{i0}^{2}=
\frac{\pi}{4}w_{i0}^{2}\varepsilon_0 c_0 \omega_{i} A_{i}^{2},
\label{8.26}
\end{equation}
tako da je izkoristek pri frekvenčnem podvajanju Gaussovega snopa 
\begin{equation}
\begin{aligned}
\frac{P_{2\omega}}{P_{\omega}}=\frac{A_3^2}{A_1^2}  = &
\frac{\chi_{ef}^2 \omega^3 P_\omega z_1}{\pi c_0^4 \varepsilon_0 n_\omega n_{2\omega}} 
\arctan^2 \left( \frac{L}{2z_1}\right) \\
 = &\frac{\chi_{ef}^2 \omega^3 P_\omega}{\pi c_0^4 \varepsilon_0 n_\omega n_{2\omega}} \frac{L}{2}
\left(\frac {\arctan^2 \left( L/2z_1\right)}{L/2z_1}\right).
\label{8.27}
\end{aligned}
\end{equation}
Funkcija $(\arctan^{2}x)/x$ zavzame največjo vrednost 0,64 pri $x =L/2z_1=1,39$.
Pri dani dolžini nelinearnega sredstva $L$ je torej 
izkoristek največji, kadar je $z_{1}=0,36\,L$, kar je malo manj kot pri
preprosti oceni $z_{1}=0,5\,L$ (enačba~\ref{SHGG}). Največji izkoristek
frekvenčnega podvajanja Gaussovih snopov je tako
\begin{equation}
\frac{P_{2\omega}}{P_{\omega}}
= 0,64 \frac{\omega^3 \chi_{ef}^2}{2\pi n_{2\omega} n_{\omega} c_0^4 \varepsilon_0 } P_\omega L.
\label{8.28}
\end{equation}
S preprosto oceno, ki smo jo naredili v prejšnjem razdelku (enačba~\ref{8.17}), smo tako 
rezultat le malo zgrešili, v obeh primerih pa izkoristek narašča linearno z dolžino kristala.

\section{Optično parametrično ojačevanje}
\index{Optično parametrično ojačevanje}
\index{Parametrično ojačevanje|see{Optično parametrično ojačevanje}}

Oglejmo si še en zelo uporaben primer mešanja treh valovanj, 
ki ga opisujejo enačbe (\ref{eq:nlAz}--\ref{eq:nlA3}). To je
optično parametrično ojačevanje, pri katerem nelinearne optične pojave
izkoristimo za ojačenje optičnih signalov.\footnote{N. M. Kroll, Phys. Rev. $\mathbf{127}$, 1207 (1962).}
Imejmo razmeroma šibek vhodni
signal pri frekvenci $\omega_{1}$, ki ga želimo ojačiti, in močno črpalno valovanje
pri frekvenci $\omega_{3}>\omega_{1}$. Zaradi nelinearnosti v snovi  
intenziteta valovanja pri $\omega_{1}$ narašča, 
intenziteta valovanja pri $\omega_{3}$ se zmanjšuje, hkrati pa zaradi
ohranitve energije nastaja dodatno valovanje pri razliki frekvenc
$\omega_{2}=\omega_{3}-\omega_{1}$ (slika~\ref{fig:opa2}). Proces parametričnega ojačevanja 
si torej lahko predstavljamo kot pretvorbo enega fotona pri frekvenci 
$\omega_{3}$ v dva fotona pri $\omega_{1}$ in $\omega_{2}$.
Parametrično ojačevanje pogosto uporabljamo za ojačenje šibkih signalov 
v infrardečem delu spektra.\index{Infrardeče valovanje}
\begin{figure}[ht]
\centering
\def\svgwidth{70truemm} 
\input{slike/08_opa.pdf_tex}
\caption{Shematski prikaz valovanj pri optičnem parametričnem ojačevanju. Valovanje
pri krožni frekvenci $\omega_1$ se ojačuje na račun črpalnega valovanja s krožno frekvenco
$\omega_3$. Pri tem nastane nedejaven žarek s krožno frekvenco $\omega_2 =  \omega_3-\omega_1$.}
\label{fig:opa2}
\end{figure}

Izhajamo iz splošnih enačb za nelinearne optične pojave drugega reda 
(enačbe~\ref{eq:nlAz}--\ref{eq:nlA3}). 
\begin{align}
\frac{dA_{3}}{dz} &=\frac{i\omega_{3}\chi_{ef}}{4c_0 n_3} A_{1}\, A_{2}\, e^{-i\Delta kz}, \\
\frac{dA_{2}}{dz} &=\frac{i\omega_{2}\chi_{ef}}{4c_0 n_2} A_{1}^*\, A_{3}\, e^{i\Delta kz}\quad \textrm{in}\\
\frac{dA_{1}}{dz} &=\frac{i\omega_{1}\chi_{ef}}{4c_0 n_1} A_{2}^*\, A_{3}\, e^{i\Delta kz}.
\label{eq:opaA}
\end{align}
Privzamemo, da je črpalno valovanje vselej dosti močnejše od drugih dveh
($A_{3}\gg A_{1}$, $A_{2}$) in njegova jakost približno konstantna $A_3 = A_{30}$.
Poskrbimo še, da je izpolnjen pogoj za ujemanje faz $\Delta k=0$ in
začetna pogoja zapišemo kot $A_{1}(z=0)=A_{10}$ in $A_{2}(z=0)=0$. Ko vse to upoštevamo,
dobimo sklopljeni enačbi
\begin{align}
\frac{dA_{1}}{dz} &= \frac{i\omega_{1}\chi_{ef}}{4c_0 n_1} A_{2}^*\, A_{30}\label{eq:opaA1} 
\qquad \mathrm{in} \\
\frac{dA_{2}^*}{dz} &= -\frac{i\omega_{2}\chi_{ef}}{4c_0 n_2} A_{1}\, A_{30}^*.
\label{eq:opaA2}
\end{align}
Enačbi lahko rešimo, tako da prvo odvajamo po $z$ in vanjo vstavimo drugo enačbo.
Dobimo
\begin{equation}
\frac{d^2 A_1}{d z^2} = \frac{\omega_1 \omega_2 \chi_{ef}^2|A_{30}|^2}
{16 c_0^2 n_1 n_2} A_1 = \kappa^2 A_1
\end{equation}
in podobno za $A_2$
\begin{equation}
\frac{d^2 A_2}{d z^2} = \frac{\omega_1 \omega_2 \chi_{ef}^2|A_{30}|^2}
{16 c_0^2 n_1 n_2} A_2 = \kappa^2 A_2.
\end{equation}
Ob upoštevanju začetnih pogojev izračunamo rešitev za naraščanje amplitude signalnega žarka
z začetno amplitudo $A_{10}$
\boxeq{eq:opa}{
A_1 = A_{10} \cosh (\kappa L),
}
pri čemer je $L$ dolžina nelinearnega sredstva in 
\begin{equation}
\kappa^2 = \frac{\omega_1 \omega_2 \chi_{ef}^2|A_{30}|^2}
{16 c_0^2 n_1 n_2}.
\label{opakapa}
\end{equation}
Hkrati s signalnim žarkom narašča tudi amplituda dodatnega nedejavnega 
({\it idle}) žarka, ki nastane med procesom ojačenja\index{Nedejavni žarek}
\boxeq{eq:opan}{
A_2 = A_{20} \sinh (\kappa L),
}
pri čemer je 
\begin{equation}
A_{20} = i \sqrt{\frac{\omega_2 n_1}{\omega_1 n_2}} A_{10}.
\label{opakapaA}
\end{equation}
Na začetku intenziteti obeh valovanj naraščata približno eksponentno na račun črpalnega
valovanja (slika~\ref{fig:opagraf}). Ko postane njuna intenziteta znatna in se 
začne $A_3$ zmanjševati, je treba to seveda
upoštevati v računu. Rešiti je treba bolj zahteven sistem treh 
sklopljenih enačb, podobno~\textendash~a še bolj zapleteno~\textendash~kot v nalogi~(\ref{deplet}).

\begin{figure}[ht]
\centering
\def\svgwidth{90truemm} 
\input{slike/08_opagraf.pdf_tex}
\caption{Normirani intenziteti ojačenega žarka ($|A_1/A_{10}|^2$) in dodatnega 
nedejavnega žarka ($|A_2/A_{20}|^2$), ki nastane zaradi zahteve po ohranitvi energije. 
Naraščajoči funkciji sta seveda samo približek, ki velja, dokler je ojačenje majhno in 
se intenziteta črpalnega žarka ne zmanjšuje znatno.}
\label{fig:opagraf}
\end{figure}

\begin{naloga}
Pokaži, da sta izraza za amplitudi polji $A_1$ in $A_2$ (enačbi~\ref{eq:opa} in~\ref{eq:opan})
rešitvi sklopljenih enačb~(\ref{eq:opaA1} in \ref{eq:opaA2}).
\end{naloga}

Do zdaj smo privzeli, da je izpolnjen pogoj ujemanja faz \index{Ujemanje faz}
in $\Delta k=k_{3}-k_{1}-k_{2}=0$. 
Ta pogoj lahko izpolnimo na enak način kot pri podvajanju frekvence: v dvolomnem kristalu 
izberemo ustrezne polarizacije in smer širjenja svetlobe glede na optično os, 
tako da velja $\omega_{3}n_{3}=\omega_{1}n_{1}+\omega_{2}n_{2}$.

Če na primer vzamemo izredno polarizacijo za črpalno valovanje
in redni polarizaciji za obe ojačevani valovanji, mora biti izpolnjen naslednji pogoj 
\begin{equation}
\left(\left(\frac{\cos\vartheta_{m}}{n_{o}(\omega_3)}\right)^{2}
+\left(\frac{\sin\vartheta_{m}}{n_{e}(\omega_3)}\right)^{2}\right)^{-1/2}=
\frac{\omega_{1}}{\omega_{3}}n_{o}(\omega_1)+\frac{\omega_{2}}{\omega_{3}}n_{o}(\omega_2).
\label{8.34}
\end{equation}

\begin{naloga}
Pokaži, da v primeru neujemanja faz $\Delta k \neq 0$ amplitudi ojačevanega in dodatnega 
nedejavnega žarka naraščata kot \index{Neujemanje faz}
\begin{equation}
A_1 = A_{10} \left( \cosh(\kappa z) - \frac{i \Delta kz}{2 \kappa} \sinh (\kappa z) 
\right) e^{\frac{i \Delta kz}{2}}
\label{phmisa1}
\end{equation}
in
\begin{equation}
A_2 = A_{20} \sinh(\kappa z) e^{\frac{i \Delta k}{2}},
\end{equation}
pri čemer sta
\begin{equation}
\kappa^2 = \frac{\omega_1 \omega_2 \chi_{ef}^2|A_{30}|^2}
{16 c_0^2 n_1 n_2} - \frac{\Delta k^2}{4}
\end{equation}
in
\begin{equation}
A_{20} = i \sqrt{\frac{\omega_2 n_1}{\omega_1 n_2}} \sqrt{1 + \frac{\Delta k^2}{4 \kappa^2}}
~A_{10}.
\label{phmisa20}
\end{equation}
Pokaži še, da so enačbe~(\ref{phmisa1}--\ref{phmisa20}) v limitnem primeru, 
ko se faze ujamejo in je $\Delta k = 0$,
enake enačbam~(\ref{eq:opa}--\ref{opakapaA}).
\end{naloga}

Za konec ocenimo koeficient ojačenja v kristalu 
LiNbO$_{3}$\index{LiNbO$_3$}, v katerem želimo
ojačiti svetlobo z valovno dolžino $\lambda = 1~\si{\micro\metre}$. Črpamo z laserjem z 
valovno dolžino okoli $500~\si{\nano\metre}$ in gostoto svetlobnega 
toka $5~\si{\mega\watt}/\si{\centi\metre}^{2}$. Lomni količnik snovi je 
$n = 2,2$ in efektivna nelinearna susceptibilnost  $\chi_{ef} = 5~\si{\pico\metre}/\si{\volt}$. 
Vstavimo podatke v enačbo~(\ref{opakapa}) in izračunamo vrednost 
$\kappa \sim 0,15~/\si{\centi\metre}$. Porast intenzitete vpadne svetlobe v $1~\si{cm}$ 
dolgem kristalu je tako le približno $2~\%$. 

\subsection*{Optični parametrični oscilator (OPO)}
\index{Optični parametrični oscilator}
Navedeni primer kaže, da optično parametrično ojačevanje svetlobe pri prehodu skozi kristal ni prav veliko
kljub dokaj močnemu črpalnemu žarku. Zato je smiselno, da svetloba večkrat preleti
ojačevalno sredstvo in se postopoma ojačuje. To naredimo tako, 
da optično ojačevalno sredstvo zapremo v optični 
resonator\index{Resonator!parametrični oscilator}
in signal se ob vsakem obhodu ojači. Sestavili smo t.\ i.\ optični parametrični oscilator
(slika~\ref{fig:opo}). 
\begin{figure}[ht]
\centering
\def\svgwidth{80truemm} 
\input{slike/08_opo.pdf_tex}
\caption{Shematski prikaz tipičnega optičnega parametričnega oscilatorja. Ojačevalno sredstvo
zapremo med resonatorja, da se signalni žarek ($\omega_1$) ob vsakem obhodu ojači.}
\label{fig:opo}
\end{figure}

V optičnem resonatorju je odbojnost zrcal za črpalni žarek ($\omega_3$) zelo majhna, 
odbojnost za ojačeni žarek pa blizu ena. Valovanje pri $\omega_1$,
ki se v parametričnem oscilatorju ojačuje, nastane spontano, prav tako valovanje pri 
$\omega_2 = \omega_3 -\omega_1$. Njuni frekvenci sta dodatno določeni s pogojem za 
ujemanje faz $ k_3 - k_1 - k_2 = 0$, 
hkrati mora ojačevano nihanje sovpadati z lastnim nihanjem resonatorja. 
S sukanjem ojačevalnega kristala lahko na ta način spreminjamo
ojačeno frekvenco in naredili smo nastavljiv izvor svetlobe, navadno v 
infrardečem delu spektra.\index{Infrardeče valovanje}

\pagebreak
Za delovanje oscilatorja mora biti jakost črpalnega žarka tako velika, da je 
ojačenje signala na obhod večje od izgub. Signal z močjo 
$P_0$ se ob prehodu skozi ojačevalno sredstvo ojači (enačba~\ref{eq:opa})
\begin{equation}
P_1 = P_0 \cosh^2 (\kappa L),
\end{equation}
hkrati pa se zaradi izhodnega zrcala z odbojnostjo $\mathcal{R}<1$ in notranjih izgub $\Lambda_0$ 
intenziteta žarka zmanjšuje. Ker je pogoj ujemanja faz izpolnjen le v eni smeri, se svetloba
ojačuje le enkrat na celoten obhod. Ob preletu v drugo smer je namreč $\Delta k \neq 0$ in 
žarek se ne ojačuje. V stacionarnem stanju je ojačenje ravno enako izgubam, moč 
signalnega žarka po obhodu $P_2$ pa je enaka začetni moči $P_0$. Velja
\begin{equation}
P_2 = P_1\,(1-\Lambda_0)\mathcal{R} = P_0 \,(1-\Lambda_0) \mathcal{R} \,\cosh^2 (\kappa L) = P_0
\end{equation}
oziroma
\begin{equation}
\cosh^2 (\kappa L) =\frac{1}{(1-\Lambda_0)\, \mathcal{R}}.
\end{equation}
Od tu izračunamo parameter $\kappa$, po enačbi~(\ref{opakapa}) pa mejno 
amplitudo oziroma intenziteto črpalnega žarka. Nadaljujmo še prejšnji primer ojačenja 
svetlobe v $1~\si{cm}$ dolgem kristalu LiNbO$_{3}$.\index{LiNbO$_3$}
Če je odbojnost izhodnega zrcala $\mathcal{R}=0,85$, notranje izgube $\Lambda_0 = 0,05$ in prečni presek 
žarka $10~\si{\micro\metre^2}$, je moč praga $P_{\omega_3} = 5~\si{\watt}$.
\vglue-5truemm
\begin{remark}
Optični parametrični oscilator oddaja svetlobo, podobno kot laser. Tudi sicer
sta si do neke mere podobna: oba sistema potrebujeta močen črpalni mehanizem, oba sistema
sta sestavljena iz resonatorja, v katerem se žarek velikokrat odbije in postopoma ojačuje,
in oba oddajata koherentno svetlobo pri točno določeni valovni dolžini. Vendar
je med parametričnim oscilatorjem in laserjem velika razlika. Pri laserju se svetloba
ojači zaradi obrnjene zasedenosti stanj, pri oscilatorju pa 
zaradi nelinearnega optičnega pojava. Pri oscilatorju energija torej ni shranjena v
snovi, ampak se signal ojačuje sproti. Velika prednost oscilatorjev pred laserji 
je zvezno nastavljiva frekvenca delovanja v zelo širokem frekvenčnem območju, saj ni določena
s prehodom med nivoji, ampak z izpolnjevanjem pogoja za ujemanje faz.
\end{remark}

\section{Optično usmerjanje in teraherčno valovanje}
\index{Optično usmerjanje}
\index{Teraherčno valovanje}
Ko smo obravnavali nelinearne optične pojave drugega reda, smo zapisali
različne frekvence, ki so vsebovane v izhodnem signalu (slika~\ref{fig:nl2}). Eno izmed
izhodnih valovanj ima tudi frekvenco enako nič, kar pomeni, da je to statično električno polje. Iz analogije
z elektronskimi vezji, kjer izmenično napetost z usmernikom spremenimo v enosmerno napetost, 
pojav imenujemo optično usmerjanje, saj iz svetlobnega valovanja nastane statično polje. Tako statično 
polje navadno ni veliko, saj sunek svetlobe z vršno močjo nekaj $\si{\mega\watt}$ tipično povzroči 
več $\si{\milli\volt}$ napetosti v smeri prečno na smer potovanja svetlobe.\footnote{M. Bass
et al., Phys. Rev. Lett. $\mathbf{9}$, 446 (1962).}

\begin{naloga}
Pokaži, da je napetost, ki se pojavi pri optičnem usmerjanju, približno enaka
\begin{equation}
U = \frac{\chi P_0}{n^3 \varepsilon_0 c_0 a},
\end{equation}
pri čemer je $P_0$ moč vpadne svetlobe, $n$ lomni količnik snovi in $a$ širina kristala.\\
Namig: nelinearni kristal obravnavaj kot ploščati kondenzator in zapiši polarizacijo.

Oceni še napetost, če je
$\chi = 3~\si{\pico\meter/\volt}, P_0 = 1~\si{\mega\watt}, n = 2,2$ in $a = 5~\si{\milli\metre}$. 
\end{naloga}

Precej bolj uporaben je pojav, ko na nelinearni kristal posvetimo z ultrakratkimi 
sunki svetlobe, tipično okoli $\si{ps}$ ali krajšimi. Spomnimo se, da je povsem 
monokromatsko valovanje lahko samo tako, ki je časovno neomejeno in ima neskončen koherenčni čas
(enačba~\ref{eq:spektralna-sirina-zveza}). 
Čim je valovanje časovno omejeno, ima njegov spekter končno širino, pri čemer 
imajo krajši sunki svetlobe širši spekter valovanja. 

Ko z ultrakratkim sunkom osvetlimo optično 
nelinearni kristal, v kristal vstopajo vse frekvence z danega intervala $\omega \pm \Delta \omega/2$.
Optično usmerjanje ni več popolno, saj se frekvence ne odštejejo povsem, ampak se 
namesto statičnega polja pojavi sunek svetlobe s širokim spektrom, ki sega od ničelne
frekvence do neke največje vrednosti. Celotna spektralna širina tega signala je 
približno enaka spektralni širini vstopnega sunka, ta pa je obratno sorazmerna z njegovo dolžino
(slika~\ref{fig:THz}).
Ocenimo te vrednosti še numerično. 

\begin{figure}[ht]
\centering
\def\svgwidth{110truemm} 
\input{slike/08_THz.pdf_tex}
\caption{Shematski prikaz nastanka teraherčnega valovanja v optično nelinearnem sredstvu}
\label{fig:THz}
\end{figure}

Vzemimo kratek sunek svetlobe dolžine $\tau$ s spektralno širino
$\Delta \nu = \Delta \omega/2 \pi = 1/\pi \tau$.
Če je sunek svetlobe dolg $1~\si{\pico\second}$, je razlika v frekvencah 
spektra 
\begin{equation}
\Delta \nu = \frac{1}{3 \times 10^{-12}~\si{s}} = 0,3~\si{\tera\hertz}.
\end{equation}
Valovanje, ki nastane pri takem optičnem kvazi-usmerjanju, ima torej frekvence v teraherčnem
področju in naredili smo izvor teraherčnega valovanja. 

Teraherčno valovanje, to je 
elektromagnetno valovanje s frekvencami v območju od 0,3 do $3~\si{\tera\hertz}$
oziroma z valovnimi dolžinami med 0,1 in $1~\si{\milli\metre}$, 
se uporablja za neinvazivno slikanje in preiskave tkiv in materialov. Kristali, ki 
se najpogosteje uporabljajo za nastanek teraherčnega valovanja, so ZnTe\index{ZnTe}, 
GaP, GaSe in GaAs.\index{GaP} \index{GaSe} \index{GaAs}

\section{Nelinearni pojavi tretjega reda}
\index{Nelinearna optika!tretjega reda}
Doslej smo obravnavali najnižji red nelinearnosti, katerega glavni
učinek je mešanje treh frekvenc, na primer optično frekvenčno podvajanje ali
optično parametrično ojačevanje. Ti pojavi so mogoči le v kristalih brez centra
inverzije. Nelinearna polarizacija tretjega reda pa je monžna v vsaki snovi. 
V njej nastopa jakost električnega polja v tretji potenci
\boxeq{eq:nl3P}{
\mathbf{P}_{\mathrm{NL,3}}= \epsilon_{0}\chi^{(3)}\vdots \mathbin 
\mathbf{E}\mathbin \mathbf{E}\mathbin\mathbf{E}
}
oziroma izpisano po komponentah
\begin{equation}
\left(\mathbf{P}_{\mathrm{NL,3}}\right)_i= \epsilon_{0}\chi^{(3)}_{ijkl} \,E_j \,E_k\, E_l.
\end{equation}
Pri tem je $\chi^{(3)}$ tenzor četrtega ranga, njegova tipična velikost je okoli 
$10^{-22}~\si{\metre^2/\volt^2}$. Na splošno ima 81 različnih neodvisnih komponent, vendar se to
število lahko zelo zmanjša zaradi simetrije snovi. V izotropni snovi je tako
21 neničelnih elementov, od katerih so le trije neodvisni. 

Če vsebuje vpadno polje le eno frekvenco, se zaradi nelinearnosti tretjega
reda pojavi polarizacija pri 3$\omega$ in $\omega$. Pri dveh vpadnih
frekvencah $\omega_{1}$ in $\omega_{2}$ so mogoče kombinacije $2\omega_{1}\pm\omega_{2}$
in $\omega_{1}\pm2\omega_{2}$, pri treh vpadnih frekvencah pa vse
mogoče vsote in razlike frekvenc, to so $\omega_1$, $\omega_2$, $\omega_3$, 
$3\omega_1$, $3 \omega_2$, $3\omega_3$, 
$\omega_1 + \omega_2 + \omega_3$, $\omega_1 + \omega_2 - \omega_3$, 
$\omega_1 - \omega_2 + \omega_3$, $- \omega_1 + \omega_2 + \omega_3$, 
$2 \omega_1\pm\omega_2$, $2 \omega_1\pm\omega_3$, $2 \omega_2\pm\omega_1$,
$2 \omega_2\pm\omega_3$, $2 \omega_3\pm\omega_1$, $2 \omega_3\pm\omega_2$.
Možnosti je torej precej več kot pri nelinearnosti drugega reda in računi so zato 
precej bolj zapleteni.

Obravnava nastanka valovanja pri
kombinaciji frekvenc je zelo podobna obravnavi frekvenčnega podvajanja ali  parametričnega
ojačevanja. V enačbah za nastanek novega valovanja ali ojačevanje
katerega od vpadnih snopov spet nastopi fazni faktor, ki vsebuje razliko
vseh valovnih vektorjev $\Delta{\bf k}$. Da je intenziteta novega
valovanja znatna, mora biti $\Delta kL\simeq0$, spet mora biti torej
izpolnjen pogoj ujemanja faz. Ker v tem primeru nastopajo na splošno štirje
valovni vektorji, je seveda tudi pri izbiri geometrije in polarizacij
za dosego ujemanja faz precej več možnosti.\index{Ujemanje faz}

Omejimo se na najpreprostejši primer, pri katerem ima vpadno valovanje le eno 
frekvenco. Takrat se pojavi valovanje pri potrojeni frekvenci, pa tudi
pri frekvenci, ki je enaka vpadni. Pojavi se torej polarizacija pri 
vpadni frekvenci, ki spremeni obnašanje osnovnega valovanja, in valovanje vpliva samo nase.
Ti pojavi, ki jih poimenujemo s predpono {\it samo-}, kot na primer samozbiranje, so
značilni za nelinearne pojave tretjega reda\index{Samozbiranje}.\footnote{Glej 
npr. R. W. Boyd, {\it Nonlinear Optics}, tretja izdaja, Academic Press (2008).}

\section{Optični Kerrov pojav}
\index{Optični Kerrov pojav|see Kerrov pojav!optični}
\index{Kerrov pojav!optični}
\label{OKP}
Naj valovanje vpada na nelinearno snov, za katero velja $\chi^{(2)} = 0$.
Polarizacija je potem enaka vsoti linearnega in nelinearnega dela tretjega reda 
(enačba~\ref{8.1})\index{Električna polarizacija}
\begin{equation}
\mathbf{P}=
\epsilon_{0} \chi^{(1)}\cdot \mathbf{E}+
\epsilon_{0}\chi^{(3)}\vdots \mathbin \mathbf{E}\mathbin \mathbf{E}\mathbin\mathbf{E}.
\end{equation}
Ker obravnavamo nelinearne pojave, moramo tudi v tem primeru zapisati realna
električna polja. To naredimo z vsoto dveh kompleksno konjugiranih členov
\begin{equation}
\mathbf{E}=\frac{\mathbf{e}}{2}\left(Ae^{i(kz-\omega t)}+A^{*}e^{-i(kz-\omega t)}\right).
\label{8.71}
\end{equation}
Podobno zapišemo tudi polarizacijo, pri čemer nas zanimajo samo členi,
ki nihajo s frekvenco $\omega$
\begin{equation}
\mathbf{P}=\frac{\mathbf{e}}{2}\left(P_\omega e^{i(kz-\omega t)}+P_\omega^{*}e^{-i(kz-\omega t)}\right).
\label{8.71a}
\end{equation}
Nelinearna polarizacija tretjega reda ima frekvenco $\omega$, kadar v produktu 
$\mathbf{E}\mathbin \mathbf{E}\mathbin\mathbf{E}$ 
dvakrat nastopa nekonjugirani del, enkrat pa konjugirani. To se lahko zgodi na tri
načine, zato dobimo
\begin{equation}
\frac{\mathbf{e}}{2}P_{\omega,\mathrm{NL}} = 3 \frac{1}{8} A A^* \left( 
\varepsilon_0 \chi^{(3)}\vdots \mathbf{e}\, \mathbf{e} \, \mathbf{e} \right) A.
\label{pomega}
\end{equation}
Celotna polarizacija je
\begin{equation}
\mathbf{P}=
\epsilon_{0} \chi^{(1)}\cdot \mathbf{E}+\frac{3}{4} |A|^2 \left( 
\varepsilon_0 \chi^{(3)}\vdots \mathbin \mathbf{e}\mathbin \mathbf{e} \right) \mathbf{E}.
\label{eq:ptnl}
\end{equation}
Z upoštevanjem zveze med amplitudo polja in povprečno gostoto energijskega toka (enačba~\ref{eq:jcw})
zapišemo
\begin{equation}
\mathbf{P}=
\epsilon_{0} \left( \chi^{(1)} +\frac{3}{4} \frac{2  j }
{\varepsilon_0 \tilde{n} c_0} \chi^{(3)}\vdots \mathbin \mathbf{e}\mathbin 
\mathbf{e} \right) \mathbf{E}.
\label{eq:pppeee}
\end{equation}
Z $\tilde{n}$ smo označili lomni količnik pri frekvenci $\omega$. 
Faktor v oklepaju ni nič drugega kot efektivna susceptibilnost, ki je neposredno povezana
z lomnim količnikom snovi $\chi_{ef} = \varepsilon -1 =n^2 -1$. Enačba~(\ref{eq:pppeee}) 
torej opisuje pojav, 
pri katerem vpadna svetloba vpliva na lomni količnik snovi, po kateri 
potuje.\index{Susceptibilnost!efektivna}
To je podoben učinek kot pri navadnem Kerrovem pojavu, pri katerem se lomni količnik 
spremeni pod vplivom zunanjega električnega polja (enačba~\ref{7.1}).
Opisani optični ekvivalent zato imenujemo optični 
Kerrov pojav\footnote{Škotski fizik John Kerr, 1824\textendash1907.}.

Poglejmo pojav podrobneje na primeru izotropne snovi. Na snov naj vpada valovanje, ki je polarizirano
v smeri $x$, tako da ima nelinearna polarizacija le komponento 
\begin{equation}
P_{\mathrm{NL},x}=
\epsilon_{0} \left(\chi_{xx} +\frac{3}{4} \chi_{xxxx}\frac{2 j }
{\varepsilon_0 \tilde{n} c_0}\right)E = \varepsilon_0 \chi_{ef}E = \varepsilon_0 (n^2-1) E.
\end{equation}
Izrazimo še efektivni lomni količnik
\begin{equation}
n \approx \tilde{n} + \frac{3 \chi_{xxxx}}{4 \varepsilon_0 c_0 \tilde{n}^2} j,
\end{equation}
ki ga lahko zapišemo v obliki 
\boxeq{eq:nnl}{
n= \tilde{n} + n_2 j,}\index{Lomni količnik!efektivni}
pri čemer smo vpeljali nelinearni lomni količnik\index{Lomni količnik!nelinearni}
\boxeq{eq:n2}{
n_2 = \frac{3 \chi_{xxxx}}{4 \varepsilon_0 c_0 \tilde{n}^2}.
}
Efektivni lomni količnik snovi je torej odvisen od gostote svetlobnega toka, ki vpada nanjo. 
Tipične vrednosti nelinearnega lomnega količnika za vidno svetlobo so $10^{-20}~\si{\metre^2/\watt}$.
V tekočini CS$_2$ je $n_2 = 3,2 \cdot 10^{-18}~\si{\metre^2/\watt}$, v nekaterih \index{CS$_2$}
drugih snoveh (npr. polprevodnikih) je lahko vrednost $n_2$ večja še za nekaj 
velikostnih redov, $n_2$ pa je lahko tudi negativen (tabela~\ref{table:chi3}).\footnote{R.
W. Boyd, {\it Nonlinear Optics}, tretja izdaja, Academic Press (2008).}

\begin{table}[ht]
 \centering
\begin{tabular}{|c|c|c|} \hline  
      Snov & $\chi^{(3)}~(\si{\metre^2/\volt^2})$ & $n_2~(\si{\metre^2/\watt})$\\ \hline
     steklo BK7 & 2,8 $\times 10^{-22}$ & $3,4 \times 10^{-20}$ \\ \hline
     voda & $2,5 \times 10^{-22}$ & $4,1 \times 10^{-20}$ \\ \hline
     GaAs & $1,4 \times 10^{-18}$ & $3,3 \times 10^{-17}$ \\ \hline\index{GaAs}
     ZnSe & $6,2 \times 10^{-20}$ & $3,0 \times 10^{-18}$ \\ \hline\index{ZnSe}
     CS$_2$ & $3,1 \times 10^{-20}$ & $3,2 \times 10^{-18}$ \\ \hline \index{CS$_2$}
     polimer 4BCMU  & $-1,3 \times 10^{-19}$ & $-1,5 \times 10^{-17}$ \\ \hline      
\end{tabular}
  \caption{Nelinearna susceptibilnost tretjega reda in nelinearni lomni količnik za nekaj snovi}
\label{table:chi3}
\end{table}

Zanimivi posledici lomnega količnika, odvisnega od gostote svetlobnega toka vpadne svetlobe, 
sta samozbiranje svetlobnega snopa in širjenje solitonov po optičnih vodnikih, 
kar si bomo pogledali v naslednjih razdelkih.

\begin{remark}
Ničesar nismo povedali o ujemanju faz, ki je sicer nujno potrebno za učinkovite nelinearne 
optične pojave. V tem primeru vpada na snov en sam laserski žarek in pogoj ujemanja faz
je vedno izpolnjen. 
\end{remark}

\section{Samozbiranje in krajevni solitoni}
\index{Samozbiranje}
\index{Soliton!krajevni}
Za začetek si oglejmo pojav samozbiranja svetlobe.\footnote{P. L. Kelley, Phys. Rev. Lett. $\mathbf{15}$,
1005 (1965).}
Osnovni Gaussov snop 
(enačba~\ref{eq:gaussov-snop}) naj vpada na sredstvo, v katerem je lomni 
količnik odvisen od gostote energijskega toka vpadne svetlobe po enačbi~(\ref{eq:nnl}).
Naj bo $n_{2}>0$, tako da je lomni količnik v sredini snopa večji 
od nemotenega lomnega količnika na robu. V osi snopa se optična pot 
zaradi optično gostejšega sredstva podaljša in valovna fronta 
v osi zaostaja glede na fronte na robu snopa. Če je zaostajanje dovolj veliko,
lahko krivinski radij valovne fronte postane negativen in snop se
ne širi, temveč oža (slika~\ref{fig:sf1}). Temu pojavu pravimo 
samozbiranje. Samozbiranje je pri dovolj
veliki moči snopa lahko tako veliko, da pride do katastrofične zožitve snopa
in s tem do tolikšnega povečanja gostote svetlobnega toka, da nastanejo
poškodbe v snovi.
\begin{figure}[ht]
\centering
\def\svgwidth{95truemm} 
\input{slike/08_sf1.pdf_tex}
\caption{V Gaussovem snopu je intenziteta valovanja $j$ odvisna od prečne koordinate $r$, 
zato je tudi lomni količnik nelinearnega sredstva  $n$ odvisen od nje. To vodi do 
 samozbiranja svetlobe. Na sliki so fronte vpadnega Gaussovega snopa narisane kot ravni valovi.}
\label{fig:sf1}
\end{figure}
\begin{naloga}
Gaussov snop svetlobe z močjo $P$ in polmerom $w$ naj vpada pravokotno na ploščico
kristala debeline $d$. Pokaži, da ploščica deluje na snop kot leča z goriščno razdaljo 
\begin{equation}
f = \frac{\pi w^4}{8 n_2 d P},
\end{equation}
pri čemer je $n_2$ nelinearni lomni količnik.
\end{naloga}

Zaradi uklona se Gaussov snop širi, medtem ko ima pojav samozbiranja ravno nasprotni
učinek. Snop samemu sebi ustvarja valovni vodnik, v katerem je v sredi lomni količnik 
večji kot na robu. Pri določeni moči snopa se oba pojava po
velikosti ravno izenačita. Snop, ki potuje po snovi, ima tako konstanten polmer, 
valovne fronte pa so ravne -- nastane t.\,i. krajevni soliton.\footnote{Glej npr. G. New, {\it Introduction
to Nonlinear Optics}, Cambridge University Press (2011).}
\index{Soliton!krajevni}

Ocenimo, pri kolikšni moči vpadne svetlobe se pojavijo krajevni solitoni. 
Vzemimo, da je na izbranem mestu valovna fronta ravna. Lahko si mislimo,
da je tam ravno grlo Gaussovega snopa. Brez samozbiranja bi bil na razdalji
dolžine bližnjega polja $z_{0}$ krivinski radij valovne fronte (enačba~\ref{eq:R})
\begin{equation}
R(z_{0})=z_{0}\left( 1+\left(\frac{z_{0}}{z_{0}}\right)^{2}\right)=2z_{0}.
\label{8.75}
\end{equation}
Po enačbi~(\ref{eq:nnl}) je odvisnost lomnega količnika približno
\begin{equation}
n(r)=\tilde{n}+n_2 j_0 e^{-2r^2/w_0^2}.
\label{8.76}
\end{equation}
Razlika med lomnim količnikom na osi (pri $r=0$) in pri $r = w_{0}$ 
od osi je kar približno $\Delta n= j_{0} n_{2}$.
Zaradi tega je na poti od grla do $z_0$ razlika optičnih poti med žarkoma na osi $(r=0)$ in 
pri $r= w_{0}$ enaka $\Delta nz_{0} = n_2 j_0 z_0$ in valovna fronta se 
ukrivi na nek krivinski radij $-R$. Iz preproste geometrije velja zveza 
\begin{equation}
\Delta nz_{0}=R-R\sqrt{1-\frac{w_{0}^{2}}{R^{2}}}\approx \frac{w_{0}^{2}}{2R}.
\label{8.77}
\end{equation}
Da valovna fronta ostane ravna, se morata krivinska radija zaradi uklona 
(enačba~\ref{8.75}) in samozbiranja (enačba~\ref{8.77}) ravno izenačiti. 
Od tod sledi 
\begin{equation}
\Delta n=\frac{w_{0}^{2}}{4z_{0}^{2}}.
\label{8.78}
\end{equation}
Moč snopa, pri katerem se polmer ne spreminja, je potem 
\begin{equation}
P_{s}= \frac{1}{2}\pi w_0^2 \,j_0 = \frac{1}{2}\pi w_0^2 \, \frac{\Delta n}{n_2} = 
\frac{1}{2}\pi w_0^2 \,\frac{w_{0}^{2}}{4z_{0}^{2}}\,\frac{1}{n_2} = \frac{\lambda^2}{8\pi n_2},
\label{8.79}
\end{equation}
pri čemer smo upoštevali zvezo med $z_0$ in $w_0$ (enačba~\ref{eq:z0}).

Pri moči, ki je manjša od kritične moči, se vpadli Gaussov snop širi, 
čeprav nekoliko počasneje kot v sredstvu s konstantnim lomnim količnikom. 
Če pa je moč znatno večja od kritične moči, lahko
nastopi katastrofično samozbiranje in porušitev snovi.
Zanimivo je, da kritična moč, pri kateri se pojavijo solitoni, 
ni odvisna od začetnega polmera snopa.

\begin{naloga}
Nariši skico k enačbi~(\ref{8.77}) in izpelji izraz za moč, pri kateri se pojavijo
solitoni (enačba~\ref{8.79}). 

Izračunaj še kritično moč za pojav solitonov v CS$_{2}$,\index{CS$_2$}
če je valovna dolžina vpadnega valovanja $1~\si{\micro\metre}$, 
nelinearni lomni količnik te tekočine pa je 
 $n_{2}=3,2 \cdot 10^{-18}~\si{\metre^2/\watt}$. 
\end{naloga}
\pagebreak

\begin{remark}
Eksperimentalna metoda, \index{Metoda vzdolžnega premika}
\index{Z-scan|see {Metoda vzdolžnega premika}}s katero merimo nelinearni 
lomni količnik, je tako imenovana
metoda vzdolžnega premika ({\it Z-scan}).\footnote{M. Sheik-bahae, A. A. Said in E. W. Van Stryland, 
Opt. Lett. $\mathbf{14}$, 955 (1989).} 
Optično nelinearno sredstvo (naj ima $n_2>0$)
postavimo v zbran laserski snop (slika~\ref{fig:zscan}). 
Zaradi samozbiranja deluje vzorec kot leča, njena goriščna razdalja
pa je odvisna od intenzitete snopa in od nelinearnega lomnega količnika. Ko vzorec 
premikamo vzdolž snopa, se skupna efektivna goriščna razdalja leče in nelinearne snovi 
spreminja in snop na detektorju je enkrat bolj zbran, drugič manj. 
Za lege vzorca desno od prvotnega gorišča ($z>0$), je skupna goriščna
razdalja daljša od goriščne razdalje leče, snop je bolj zbran (pikčasta črta) in signal 
na detektorju (D) naraste. Za lege vzorca levo
od prvotnega gorišča ($z<0$) je ravno obratno, snop se razširi (črtkana črta) in 
signal na detektorju se zmanjša. Za snovi z negativnim nelinearnim lomnim količnikom
je odziv ravno nasprotnega predznaka. Pri določanju nelinearnega lomnega količnika je
ključno uporabiti zaslonko (Z), s katero omejimo premer vpadnega snopa pred detektorjem. 
Če zaslonko odstranimo in merimo 
odvisnost celotne vpadne intenzitete od lege vzorca, nelinearnega lomnega količnika 
ne moremo meriti, lahko pa določimo nelinearni absorpcijski koeficient. 

\begin{figure}[h!]
\raggedleft 
\def\svgwidth{130truemm} 
\input{slike/08_zscan.pdf_tex}
\caption{Shema metode vzdolžnega premika (a). Z -- zaslonka, D -- detektor, $n_2$ - optično
nelinearno sredstvo in $f$ leča. Intenziteta svetlobe, ki vpade na detektor, je odvisna
od lege nelinearnega sredstva (b).}
\label{fig:zscan}
\end{figure}
\end{remark}

\section{*Izpeljava krajevnih solitonov}
\index{Soliton!krajevni}
\label{chap:ks}
Za podrobnejšo obravnavo krajevnih solitonov moramo rešiti valovno 
enačbo v obosnem približku. Začnemo s skalarno obliko Helmholtzeve enačbe (enačba~\ref{eq:Helmholtz})
\begin{equation}
\nabla^{2}E+n^{2}\frac{\omega^{2}}{c_0^{2}}E=0.
\label{8.80}
\end{equation}
Polje zapišemo v obliki počasi spreminjajoče se amplitude $\psi$ in faznega faktorja, podobno kot 
smo to naredili pri izpeljavi Gaussovega snopa (enačba~\ref{eq:ravni-val-nastavek})
\begin{equation}
E=\psi(\mathbf{r},z)e^{ik_{0}z},
\label{8.81}
\end{equation}
 kjer je $\mathbf{r}$ krajevni vektor v ravnini $xy$ in 
\begin{equation}
k_{0}=\frac{\tilde{n}\omega}{c_0} 
\end{equation}
valovno število brez upoštevanja nelinearnosti.
Funkcija $\psi(\mathbf{r},z)$ naj se v smeri osi $z$ le počasi spreminja, tako da lahko
drugi odvod po $z$ zanemarimo in zapišemo \index{Obosna valovna enačba}
\begin{equation}
\nabla_{\bot}^{2}\psi+\frac{\omega^{2}}{c_0^{2}}(n^{2}-\tilde{n}^{2})\psi+2ik_{0}
\frac{\partial\psi}{\partial z}=0.
\label{8.82}
\end{equation}
Upoštevamo odvisnost lomnega količnika od intenzitete, pri čemer
zanemarimo člen z $n_{2}^{2}$, ker je gotovo majhen. Sledi
\begin{equation}
\nabla_{\bot}^{2}\psi+2k_{0}^{2}\frac{n_{2}}{\tilde{n}}j\psi+2ik_{0}\frac{\partial\psi}{\partial z}=0.
\label{8.83}
\end{equation}
Izrazimo še gostoto svetlobnega toka z amplitudo jakosti električnega polja in dobimo
\begin{equation}
\nabla_{\bot}^{2}\psi+
k_{0}^{2} n_2 \varepsilon_0 c_0 |\psi|^2 \psi+
2ik_{0}\frac{\partial\psi}{\partial z}=0.
\label{8.83a}
\end{equation}
Preden se lotimo reševanja enačbe, vpeljemo še
\begin{equation}
\kappa=k_{0}^{2} n_2 \varepsilon_0 c_0
\end{equation}
 in novo spremenljivko vzdolž osi $z$
\begin{equation}
\zeta=\frac{z}{2k_{0}}.
\end{equation}
 S tem preide enačba (\ref{8.83a}) v standardno obliko nelinearne Schr\"odingerjeve
enačbe, le da namesto odvoda po času tukaj nastopa odvod po koordinati $\zeta$. Sledi
\index{Schr\"odingerjeva enačba!nelinearna}
\boxeq{8.84}{
i\frac{\partial\psi}{\partial\zeta}+\nabla_{\bot}^{2}\psi+\kappa\left|\psi\right|^{2}\psi=0.
}
V treh dimenzijah je reševanje enačbe (\ref{8.84}) težavno in analitične
rešitve niso znane. V dveh dimenzijah pa stacionarno rešitev znamo
poiskati. Stacionarni rešitvi se vzdolž $\zeta$ lahko spreminja le faza, zato
rešitev iščemo v obliki 
\begin{equation}
\psi=e^{i\eta^{2}\zeta}\, u(x),
\label{8.87}
\end{equation}
 kjer je $\eta$ konstanta, katere pomen bomo videli v nadaljevanju, 
 funkcija $u(x)$ pa naj bo realna. 
Uporabimo nastavek (enačba~\ref{8.87}) v enačbi (\ref{8.84}) in dobimo
\begin{equation}
\frac{d^{2}u}{dx^{2}}=\eta^{2}u-\kappa u^{3}.
\end{equation}
 Z množenjem obeh strani z $u^{\prime}$ lahko enačbo enkrat integriramo
\begin{equation}
\left(\frac{du}{dx}\right)^{2}=\eta^{2}u^{2}-\frac{1}{2}\kappa u^{4}.
\end{equation}
Ločimo spremenljivki in zapišemo 
\begin{equation}
\int_{\eta\sqrt{2/\kappa}}^{u}\frac{du}{\sqrt{\eta^{2}u^2-\frac{1}{2}\kappa u^{4}}}=x-x_{0},
\label{8.85}
\end{equation}
pri čemer smo uvedli integracijsko konstanto $x_{0}$ in integracijsko mejo postavili 
tako, da so vrednosti pod korenom pozitivne.
Integral brez težav izračunamo
\begin{equation}
\frac{1}{\eta}\ln\left(\sqrt{\frac{\kappa}{2}}\frac{u}{\eta+
\sqrt{\eta^{2}-\kappa u^{2}/2}}\right)=x-x_{0}
\end{equation}
in izrazimo iskano funkcijo $u(x)$
\begin{equation}
u=\sqrt{\frac{2}{\kappa}}\frac{2 \eta }{e^{\eta(x-x_{0})}+e^{-\eta(x-x_{0})}}=
\sqrt{\frac{2}{\kappa}}\frac{\eta}{\cosh \left(\eta(x-x_{0})\right)}.
\label{8.86}
\end{equation}
Po enačbi (\ref{8.87}) je rešitev
\begin{equation}
\psi(x,z)=\sqrt{\frac{2}{\kappa}}\,\eta\,\frac{e^{i\eta^{2}\zeta}}{\cosh \left(\eta(x-x_{0})\right)}.
\label{8.88}
\end{equation}
Vidimo, da predstavlja spremenljivka $1/\eta$  karakteristično širino snopa, $x_{0}$ pa
je le njegov prečni premik, ki ga lahko brez škode postavimo na $x_0=0$. Tako
zapišemo celotno polje stacionarnega snopa 
\begin{equation}
E_{s}(x,z)=\sqrt{\frac{2}{\kappa}}\,\frac{\eta}{\cosh(\eta x)}\,\exp\left(ik_{0}z\left(1+
\frac{\eta^{2}}{2k_{0}^{2}}\right)\right).
\label{8.89}
\end{equation}
Zapišemo gostoto svetlobnega toka,\index{Gostota energijskega toka} ki je sorazmerna 
kvadratu amplitude polja, kot
\boxeq{8.89a}{
j_{s}(x,z)= j_0 \frac{1}{\cosh^2(\eta x)}.
}
Vidimo, da je gostota svetlobnega toka neodvisna od $z$, kar pomeni, da sunek 
(slika~\ref{fig:soliton}) ohranja svojo obliko, ko potuje vzdolž osi $z$.

\begin{figure}[ht]
\centering
\def\svgwidth{100truemm} 
\input{slike/08_soliton.pdf_tex}
\caption{Prečna odvisnost relativne intenzitete krajevnega solitona $j/j_0$ v dveh dimenzijah. 
Vzdolž koordinate $z$ se profil ohranja.}
\label{fig:soliton}
\end{figure}

Če se vrnemo k izrazu za jakost električnega polja (enačba~\ref{8.89}), vidimo, da
parameter $\eta$ nastopa tudi v faznem faktorju. To pomeni, da je od njega odvisna 
tudi fazna hitrost
\index{Hitrost valovanja!solitonov}
\begin{equation}
v_{f}= \frac{c_0}{\tilde{n}\left(1+\frac{\eta^{2}}{2k_{0}^{2}}\right)}.
\end{equation}
Fazna hitrost omejenih snopov oziroma solitonov je torej vedno manjša od fazne hitrosti ravnih valov. 
Bolj ko je snop omejen, manjša je fazna hitrost, za velike polmere snopa pa doseže 
limitno vrednost $c_0/\tilde{n}$.

Celotna moč svetlobe v dvodimenzionalnem snopu je enaka integralu
gostote svetlobnega toka (enačba~\ref{8.89a}) po $x$. Integriramo in zapišemo 
\begin{equation}
P_s = \int j_s dx \propto \int |E_s|^2 dx  = 
\frac{2}{\kappa}\,\eta^{2}\int_{-\infty}^{\infty}\frac{dx}
{\cosh^{2}\eta x}=\frac{4\eta}{\kappa}.
\label{eq:solj}
\end{equation}
Moč stacionarnega snopa (solitona) je torej obratno sorazmerna 
z njegovo širino $1/\eta$. Manjši kot je polmer snopa, močnejši je uklon
in večja moč je potrebna, da nelinearni pojavi izničijo vpliv uklona. 
V dveh dimenzijah obstaja stacionarna rešitev, v treh dimenzijah pa 
se snop z nadkritično močjo skrči v singularnost.

\section{Optični solitoni}
\index{Soliton!optični}
\label{chap:soliton}
V prejšnjem razdelku smo ugotovili, da pojav samozbiranja lahko izniči širjenje 
svetlobnega snopa zaradi uklona, tako da ima pri
ustrezni moči snop vzdolž smeri širjenja konstantno širino in obliko. Takim snopom 
smo rekli krajevni solitoni. Povsem podoben pojav poznamo tudi v časovni 
domeni, kjer se pojavijo časovni ali optični solitoni.\footnote{Glej npr. R. W. 
Boyd, {\it Nonlinear Optics}, tretja izdaja, Academic Press (2008).}

Sunek svetlobe  naj se širi po valovnem vodniku. Zaradi disperzije je lomni količnik\index{Disperzija}
odvisen od frekvence valovanja in sunek svetlobe se med potovanjem po vodniku podaljšuje. 
Več o tem smo spoznali pri 
obravnavi disperzije v optičnih vlaknih (razdelek~\ref{chap:Disperzija}). 
Ob primernih pogojih lahko nelinearna odvisnost lomnega količnika $n(j)$ 
ravno izniči disperzijo $n(\omega)$ in sunek
ohranja obliko. Sunkom svetlobe, ki potujejo po sredstvu brez spremembe
oblike, pravimo optični solitoni. Posebej so pomembni v optičnih vlaknih, 
kjer želimo  vpliv disperzije zaradi učinkovitosti prenosa
informacije kar se da zmanjšati. 

Pojava optičnih solitonov ni težko pojasniti. Naj na optično nelinearno sredstvo
vpade sunek svetlobe, ki je Gaussove oblike v času
\begin{equation}
j(t) = j_0 e^{-2t^2/\tau^2}.
\label{08_pulz}
\end{equation}
Faza takega sunka je 
\begin{equation}
\phi (t) = k_0 n z - \omega_0 t = k_0 (\tilde{n} + n_2 j)z - \omega_0 t = 
\phi_0 + k_0 n_2 z j - \omega_0 t,
\end{equation}
krožna frekvenca pa 
\begin{equation}
\omega = -\frac{d\phi}{dt} = \omega_0 - k_0 n_2 z \frac{dj}{dt}.
\end{equation}
Če vstavimo časovno obliko sunka svetlobe (enačba~\ref{08_pulz}), vidimo, da se 
frekvenca takega sunka spreminja s časom
\begin{equation}
\omega = \omega_0 + \frac{4k_0 n_2 z j_0}{\tau^2} \, t \, e^{-2t^2/\tau^2}.
\label{eq:chirpi}
\end{equation}
Začetnemu delu sunka (pri $t<0$) se krožna frekvenca zmanjša, zadnjemu delu sunka
(pri $t>0$) pa se poveča (slika~\ref{fig:optsoliton}). 
Ta pojav spreminjanja frekvence znotraj kratkega sunka imenujemo čirikanje
({\it chirping}), \index{Čirikanje} po podobnosti z oglašanjem čričkov.

\begin{figure}[ht]
\centering
\def\svgwidth{70truemm} 
\input{slike/08_OpticniSoliton.pdf_tex}
\caption{Kratkemu sunku svetlobe, ki se širi po valovnem vodniku, se zaradi nelinearnega 
lomnega količnika snovi v začetnem delu frekvenca zmanjša, v končnem delu pa poveča.
Spreminjanje frekvence znotraj kratkega sunka imenujemo čirikanje.}
\label{fig:optsoliton}
\end{figure}
Pri prehodu optičnega sunka z osnovno krožno frekvenco $\omega_0$ se različnim delom sunka
frekvenca različno spremeni (slika~\ref{fig:chirp}\,a), začetnemu delu se zmanjša in 
končnemu poveča. Po drugi strani v snoveh poznamo barvno disperzijo, 
kar pomeni, da se valovanja z različnimi frekvencami širijo z različnimi hitrostmi.
\index{Disperzija} Pojav disperzije je še bolj zapleten pri potovanju sunkov svetlobe po vlaknih
(glej razdelka~\ref{chap:Disperzija} in \ref{chap:sunvl}). Ključni parameter je disperzija 
grupne hitrosti oziroma drugi odvod valovnega vektorja po krožni frekvenci (enačba~\ref{9.71}).

Pod določenimi pogoji (izbrana snov in določeno frekvenčno območje) 
lahko dosežemo, da potuje del valovanja z daljšo valovno dolžino počasneje kot del valovanja
s krajšo valovno dolžino (slika~\ref{fig:chirp}\,b). V tem primeru končni del sunka 
dohiteva sprednjega in učinek disperzije ravno izniči učinek nelinearnosti. 
Nastane signal, ki ohranja svojo obliko~\textendash~soliton. 
\begin{figure}[ht]
\centering
\def\svgwidth{145truemm} 
\input{slike/08_Chirp.pdf_tex}
\caption{Čirikanje sunkov svetlobe zaradi nelinearnega pojava (a). Z ustrezno disperzijo lahko
čirikanje izničimo (b) in nastane soliton.}
\label{fig:chirp}
\end{figure}

\section{*Izpeljava optičnih solitonov}
\index{Soliton!optični}
Za matematični opis optičnih solitonov izhajamo iz nelinearne \index{Valovna enačba!nelinearna}
valovne enačbe (enačba~\ref{8.3}), ki jo zapišemo v skalarni obliki
\begin{equation}
\nabla^{2}E-\frac{n^2}{c_0^{2}}{\frac{\partial^2 E}{\partial t^2}}=
\mu_{0}{\frac{\partial^2P_{\textrm{NL}}}{\partial t^2}},
\end{equation}
pri čemer je 
$P_\textrm{NL}$ nelinearna polarizacija tretjega reda (enačba~\ref{eq:nlin3}).
Namesto v časovni domeni je enačbo prikladnejše reševati v frekvenčni domeni, zato
namesto $E$ in $P_{\mathrm{NL}}$ vpeljemo Fourierevi transformiranki $\tilde{E}$ in $\tilde{P}$.

Dobimo
\begin{equation}
\nabla^{2}\tilde{E}+\frac{n^2}{c_0^{2}}\omega^2 \tilde{E}=
- \mu_{0}\omega^2 \tilde{P}.
\label{eq:soleq1}
\end{equation}
Enačbo rešujemo z nastavkoma
\begin{equation}
\tilde{E} = \tilde{A} (z,\omega - \omega_0) e^{ik_0z}
\end{equation}
in 
\begin{equation}
 \tilde{P} = \tilde{B} (z,\omega - \omega_0) e^{ik_0z},
\end{equation}
pri čemer je $\omega_0$ osrednja krožna frekvenca svetlobnega sunka in $k_0 = \omega_0 \tilde{n}/c_0$. 
Vpeljemo še 
$\Omega =\omega - \omega_0$ in valovna enačba~(\ref{eq:soleq1}) se prepiše v 
\begin{equation}
\left(\frac{\partial^2}{\partial z^2}+k^2\right)\tilde{A}(z,\Omega) e^{ik_0z} =
- \mu_{0}\omega^2 \tilde{B} (z,\Omega) e^{ik_0z}.
\end{equation}
Da lahko rešimo to enačbo, naredimo nekaj približkov. Ker je $\omega \approx \omega_0$, na desni strani
enačbe nadomestimo frekvenco z osrednjo frekvenco. Poleg tega upoštevamo, da se amplituda 
glede na valovno dolžino le počasi spreminja, zato drugi odvod zanemarimo in 
\begin{equation}
2 i k_0 \frac{\partial \tilde{A}}{\partial z} + (k^2-k_0^2) \tilde{A} = - \mu_{0}\omega_0^2 \tilde{B}.
\end{equation}
Člen $k^2 - k_0^2$ razstavimo kot razliko kvadratov, ob šibki disperziji pa $k(\omega_0 + \Omega)$
razvijemo v Taylorjevo vrsto za majhne $\Omega$ 
okoli osrednje krožne frekvence $\omega_0$ do tretjega člena. Dobimo
\begin{equation}
k^2 - k_0^2 \approx 2k_0 (k-k_0) \approx 2k_0 (k'\Omega + \frac{1}{2}k''\Omega^2),
\end{equation}
pri čemer $'$ označuje odvod po krožni frekvenci. Enačbo prepišemo v 
\begin{equation}
2 i k_0 \frac{\partial \tilde{A}}{\partial z} + 2k_0(k'\Omega + \frac{1}{2}k''\Omega^2) \tilde{A} 
= - \mu_{0}\omega_0^2 \tilde{B}.
\end{equation}
Vrnimo se v časovno domeno, tako da naredimo inverzno Fourierevo transformacijo. Naj bo 
$A(z,t)$ kompleksna amplituda jakosti električnega polja in inverzna transformiranka 
funkcije $\tilde{A}(z,\Omega)$, funkcija $B(z,t)$ pa naj bo 
amplituda polarizacije in inverzna transformiranka 
funkcije $\tilde{B}(z,\Omega)$.
Sledi
\begin{equation}
i \left(\frac{\partial}{\partial z}+\frac{1}{v_{g}}\frac{\partial}{\partial t}\right)A-
\frac{1}{2}\frac{d^{2}k}{d\omega^{2}}\,\frac{\partial^{2}A}{\partial t^{2}}=
-\frac{\mu_0\omega_0^2}{2 k_0}B,
\label{8.93}
\end{equation}
pri čemer smo z 
\begin{equation}
 v_g = \frac{d\omega}{dk} = \frac{1}{k'}
\end{equation}
označili grupno hitrost.\index{Hitrost valovanja!grupna}
Vpeljemo novo spremenljivko 
\begin{equation}
\tau=t-\frac{z}{v_{g}},
\label{nelinver}
\end{equation}
s katero opišemo obliko sunka $A_S(z,\tau)$, kot ga vidi opazovalec, ki se giblje
z grupno hitrostjo skupaj s sunkom. Uporabimo pravilo verižnega odvajanja 
\begin{equation}
\frac{\partial A}{\partial z} = \frac{\partial A_S}{\partial z} + \frac{\partial A_S}{\partial \tau}
\frac{\partial \tau}{\partial z}
= \frac{\partial A_S}{\partial z} -\frac{1}{v_g} \frac{\partial A_S}{\partial \tau}.
\end{equation}
Podobno naredimo še za odvod po času $\tau$, ki pa se ne razlikuje od odvoda po času $t$
\begin{equation}
\frac{\partial A}{\partial t} = \frac{\partial A_S}{\partial z}\frac{\partial z}{\partial \tau}+
\frac{\partial A_S}{\partial \tau}\frac{\partial \tau}{\partial t} 
= \frac{\partial A_S}{\partial \tau} 
\end{equation}
in
\begin{equation}
\frac{\partial^2 A}{\partial t^2} = \frac{\partial^2 A_S}{\partial\tau^2}.
\end{equation}
Vstavimo še amplitudo nelinearne polarizacije (enačba~\ref{eq:ptnl}), pri čemer izraz popravimo
za faktor $4$, ker smo drugače vpeljali parameter $A$. Dobimo
\begin{equation}
B = 3\varepsilon_0\chi^{(3)} |A|^2 A
\end{equation}
in enačbo~(\ref{8.93}) zapišemo kot 
\begin{equation}
i\,\frac{\partial A_S}{\partial z}-\frac{1}{2}\frac{d^{2}k}{d\omega^{2}}\,
\frac{\partial^{2}A_S}{\partial\tau^{2}}+\kappa\left|A_S\right|^{2}A_S=0.
\label{8.95}
\end{equation}
Pri tem je parameter
\begin{equation}
\kappa = \frac{3\omega_0\chi^{(3)}}{2c_0 \tilde{n}} = 2 \omega_0 \varepsilon_0 n_2 \tilde{n}
\end{equation}
sorazmeren nelinearnemu lomnemu količniku $n_2$ 
(enačba~\ref{eq:n2}). Enačba~(\ref{8.95}) ni nič drugega kot nelinearna Schr\"odingerjeva 
enačba\index{Schr\"odingerjeva enačba!nelinearna}, ki smo jo 
zapisali že pri izpeljavi krajevnih solitonov~(enačba~\ref{8.84}). Enačbi se razlikujeta v tem, da
ima vlogo prečne koordinate $x$ tukaj čas $\tau$ in rešitve nimajo več konstantnega premera,
ampak imajo konstantno dolžino sunka. Stacionarne rešitve obstajajo le v primeru, kadar je  $d^{2}k/d\omega^{2}<0$ oziroma kadar ima drugi odvod nasprotni predznak od nelinearnega lomnega količnika $n_2$. Kot pri krajevnih solitonih tudi tukaj vpeljemo parameter $\eta$, ki je sorazmeren 
z energijo solitona (enačba~\ref{eq:solj}). Sledi 
\begin{equation}
A_S\left(z,\tau\right)=\sqrt{\frac{2}{\kappa}}\eta\frac{e^{i\eta^{2}z}}{{\cosh}\left(\eta \tau 
\sqrt{2\left|\frac{d^{2}k}{d\omega^{2}}\right|^{-1}}\right)}
\end{equation}
oziroma
\begin{equation}
A\left(z,t\right)=\sqrt{\frac{2}{\kappa}}\eta\frac{e^{i\eta^{2}z}}{{\cosh}\left(\eta (t-\frac{z}{v_g}) 
\sqrt{2\left|\frac{d^{2}k}{d\omega^{2}}\right|^{-1}}\right)}.
\label{8.96}
\end{equation}
Zapisana je oblika solitona, ki potuje z grupno hitrostjo $v_g$ in pri tem ohranja obliko. Zaradi tega
so solitoni izredno zanimivi za prenos velike gostote informacij na velike razdalje, saj se izognemo
omejitvam disperzije. 

\begin{remark}
Ena izmed snovi, ki izpolnjuje pogoj, da je $k''$ nasprotnega predznaka kot $n_2$, so kremenova 
optična vlakna. Pri valovnih dolžinah vidne svetlobe to sicer ne velja, velja pa za 
$\lambda \gtrsim 1,3~\si{\micro\metre}$.\index{SiO$_2$}
Pogoj je torej izpolnjen pri valovnih dolžinah okoli $1,5~\si{\micro\metre}$, ki se navadno uporabljajo 
pri prenosu signalov po optičnih vlaknih, in signal lahko potuje brez podaljševanja. 
\end{remark}

\section{Optična fazna konjugacija}
\index{Optična fazna konjugacija}
Optična fazna konjugacija je zanimiv in danes tudi praktično pomemben
pojav, pri katerem nastane iz danega valovanja novo valovanje z enakimi valovnimi
frontami, vendar potuje novo valovanje v nasprotni smeri od prvotnega.\footnote{Glej npr. R.
W. Boyd, {\it Nonlinear Optics}, tretja izdaja, Academic Press (2008).} Novo valovanje je tako,
kot bi začetnemu valovanju obrnili predznak časa in ga ``zavrteli nazaj''.\footnote{R. W. Hellwarth, 
J. Opt. Soc. Am $\mathbf{67}$, 1 (1977).}

Vzemimo optično nelinearno snov, na katero posvetimo z dvema močnima 
snopoma, ki potujeta v nasprotnih smereh. To sta črpalna snopa z valovnima vektorjema
${\bf k}_{1}$ in ${\bf k}_2 = -{\bf k}_{1}$, ki naj bosta kar se da podobna ravnemu valu. 
Poleg njiju naj na snov vpada
še tretji, signalni snop, ki ni nujno ravni val (slika~\ref{08_OPC1}). 
Frekvence vseh snopov naj bodo enake.
Signalni snop interferira s prvim črpalnim valom in zaradi nelinearnosti 
tretjega reda povzroči modulacijo lomnega količnika, na kateri se
drugo črpalno valovanje uklanja. Uklonjeno valovanje je enake oblike
kot signalno, le potuje v nasprotni smeri, saj ima drugo črpalno valovanje
nasprotno smer od prvega. Črpalni valovanji sta seveda enakovredni in ni
mogoče ločiti med valovanjem, s katerim signalno valovanje interferira, in valovanjem, 
ki se uklanja.

\begin{figure}[ht]
\centering
\def\svgwidth{60truemm} 
\input{slike/08_opc1.pdf_tex}
\caption{Optična fazna konjugacija. Dva močna črpalna žarka (modra) z valovnima
vektorjema ${\bf k}_{1}$ in ${\bf k}_2$ vpadata na optično nelinearno snov v 
nasprotnih smereh, vpadni signal (rdeč) pa se odbije v 
smer, iz katere vpada.}
\label{08_OPC1}
\end{figure}

\begin{remark}Optična fazna konjugacija je zelo podobna holografiji, 
le da pri holografiji najprej zapišemo predmetni snop, ki ga kasneje reproduciramo, 
medtem ko pri fazni konjugaciji zapis valovanja in njegova reprodukcija \index{Hologram}
potekata sočasno. 
\end{remark}

Koordinatni sistem izberemo tako, da se signalno valovanje s frekvenco $\omega$ in amplitudo $A_3$
širi v smeri $z$. Zapišemo ga
\begin{equation}
E_{3}=\mathfrak{\Re}\left(A_3\left(z\right)\, e^{i\left(kz-\omega t\right)}\right),
\label{8.97}
\end{equation}
pri čemer $\mathfrak{\Re}$ označuje realni del. 
V naslednjem razdelku bomo z računom pokazali, da je novonastalo valovanje sorazmerno
\begin{equation}
E_{4} \propto \mathfrak{\Re}\left(A_3^{*}\left(z\right)\, e^{i\left(-kz-\omega t\right)}\right).
\label{8.98}
\end{equation}
Zaradi nasprotnega predznaka $k$ potuje nastalo valovanje v obratni smeri od signalnega
valovanja, poleg tega je kompleksno konjugirana tudi njegova amplituda. To seveda
ne vpliva na obliko valovnih front, saj so te popolnoma enake kot pri signalnem
valovanju. Ker novo valovanje iz signalnega nastane tako,
da krajevni del faze kompleksno konjugiramo, nastalemu valovanju pravimo fazno
konjugirano valovanje.
\begin{figure}[h!]
\centering
\def\svgwidth{100truemm} 
\input{slike/08_opc2.pdf_tex}
\caption{Primerjava odbojev na navadnem zrcalu (zgoraj) in faznem konjugatorju (spodaj): odboj ravnega
vala (a), odboj krogelnega valovanja (b) in odboj popačenega vala (c). Valovne fronte 
vpadnega vala so označene s polno črto in odbitega s črtkano.}
\label{08_OPC2}
\end{figure}

Uporabna posledica fazne konjugacije je prikazana na sliki~\ref{08_OPC2}.
Najpreprostejši primer je vpad ravnega vala (a), ki se ne odbije po 
odbojnem zakonu (slika zgoraj), ampak se odbije v smer, iz katere 
je vpadel na snov (slika spodaj). Drugi primer je krogelni val 
ali v približku tudi Gaussov snop (b). Po odboju od navadnega zrcala (zgoraj), se 
valovanje še naprej razširja. Na fazno konjugiranem zrcalu se krogelni val po odboju spet
zbere v izvoru (spodaj). 

V tretjem primeru vpada svetloba skozi sredstvo, ki valovanju doda naključno
fazo, zato po prehodu valovne fronte niso več gladke (c). Od navadnega zrcala
se popačen snop odbije, pri ponovnem prehodu skozi sredstvo pa se popačenje
še poveča. Povsem drugačno je obnašanje pri odboju na faznem konjugatorju. 
Ko popačen snop vpade na fazni konjugator, v njem ustvari fazno konjugiran snop, 
ki potuje v nasprotni smeri in ima enako nepravilne valovne fronte kot vpadni val. Po prehodu
skozi nepravilno sredstvo se neravnosti valovne fronte izničijo
in nastanejo enake gladke valovne fronte ravnega vala kot na začetku. 
To lastnost popravljanja valovne fronte je mogoče 
koristno uporabiti, na primer namesto enega zrcala v laserskem resonatorju.
\begin{remark}
Omenili smo, da se fazno konjugirana zrcala uporabljajo v laserjih za izničenje
popačenj Gaussovega snopa. Drugi primer uporabe je v optični astronomiji
ali optičnih komunikacijah skozi atmosfero. Naključne spremembe gostote v atmosferi
signalu dodajo naključni fazni premik, ki signal popači. Če se signal odbije od zrcala nazaj
proti izvoru, je torej dvakratno popačen. Če pa se odbije od fazno konjugiranega zrcala, 
se vpliv nehomogenosti atmosfere ravno izniči in na prenos signala ne vpliva, poleg
tega je šibek vpadni signal lahko še dodatno ojačen. 
\end{remark}

\section{*Izpeljava optične fazne konjugacije}
\index{Optična fazna konjugacija}
Poglejmo podrobneje, kako v nelinearnem sredstvu nastane fazno konjugirani
val. Kot kaže slika~\ref{08_OPC1}, je celotno polje v nelinearnem
sredstvu vsota štirih valovanj, dveh močnih črpalnih (oznaki 1 in 2), signalnega 
(oznaka 3) in novonastalega (oznaka 4)
\begin{equation}
E=\frac{1}{2}A_{1}e^{i{\bf k}_{1}\cdot{\bf r}-i\omega t}+\frac{1}{2}A_{2}e^{-i{\bf k}_{1}\cdot{\bf r}-
i\omega t}+\frac{1}{2}A_{3}\left(z\right)e^{ikz-i\omega t}+\frac{1}{2}A_{4}
\left(z\right)e^{-ikz-i\omega t}+{\rm k.k.}
\label{8.99}
\end{equation}
S k.k. smo spet označili kompleksno konjugirane člene.  Vsa valovanja naj imajo
enako frekvenco, zaradi enostavnosti privzamemo, da so enake tudi vse polarizacije.
Račun poenostavimo še s predpostavko, da sta črpalna vala dosti močnejša 
od signalnega in novonastalega, tako da sta njuni
amplitudi $E_{1}$ in $E_{2}$ konstantni, $E_{3}\left(z\right)$ in $E_{4}\left(z\right)$
pa se le počasi spreminjata.

Vstavimo nastavek za $E$ (enačba~\ref{8.99}) v valovno enačbo (enačba~\ref{8.3})
\index{Valovna enačba!nelinearna}
\begin{equation}
\nabla^{2}E+n^2\frac{\omega^{2}}{c_0^{2}}\, 
E=\mu_{0}\frac{\partial^2 P_{\mathrm{NL}}}{\partial t^2},
\label{8.100}
\end{equation}
pri čemer je $n\,\omega/c_0=k$. $P_{\textrm{NL}}$ je po enačbi~(\ref{eq:nl3P})
enak $P_\mathrm{NL}= \epsilon_{0}\chi^{(3)}E^3$, kjer je $\chi^{(3)} = \chi$
efektivna nelinearna susceptibilnost
za izbrano polarizacijo. 

Ker je $E$ zapisan kot vsota osmih členov
(enačba~\ref{8.99}), vsebuje produkt $E^3$ kar 512 členov. Vendar se njihovo število znatno zmanjša, 
če upoštevamo le tiste z enako časovno odvisnostjo oziroma enako frekvenco.
Poleg tega nas ne zanimajo različne kombinacije valovnih vektorjev, ampak k enačbi za $E_{3}$ 
prispevajo le tisti členi s krajevnim faznim faktorjem $\exp(ikz)$, 
k enačbi za $E_4$ pa tisti z $\exp(-ikz)$. Sledi
\begin{align}
P_{\mathrm{NL}\,3,4} &= \frac{\varepsilon_0\chi}{8} \big(
\left(6 A_1 A_2 A_4^*+ 6A_1 A_1^*A_3 + 6A_2A_2^*A_3 + 3 A_3A_3^*A_3 + 6 A_4 A_4^* A_3\right)
e^{i k z - i\omega t} \nonumber\\
&+
\left(6 A_1 A_2 A_3^*+6 A_1 A_1^*A_4 + 6A_2A_2^*A_4 + 6 A_3A_3^*A_4 + 3 A_4 A_4^* A_4\right)
e^{-i k z - i\omega t} \big).
\end{align}
Če zanemarimo še člene, v katerih nastopata $A_3$ in $A_4$ v višjih potencah, dobimo
\begin{align}
P_{\mathrm{NL}\,3,4} = \frac{3\varepsilon_0\chi}{4} \big(
\left( A_1 A_2 A_4^*+ |A_1|^2 A_3 + |A_2|^2 A_3 \right)
e^{i k z - i\omega t} \nonumber\\
+ 
\left( A_1 A_2 A_3^*+|A_1|^2 A_4 + |A_2|^2A_4 \right)
e^{-i k z - i\omega t} \big).
\end{align}
Vstavimo izračunani izraz v valovno enačbo~(enačba~\ref{8.100}) in upoštevamo, 
da se $A_i(z)$ le počasi 
spreminja (kar pomeni, da zanemarimo drugi odvod po $z$). Za $A_3$ so pomembni 
samo členi s potenco $ikz-i\omega t$, za $A_4$ pa členi s potenco $-ikz-i\omega t$. Dobimo 
\begin{equation}
i k \frac{dA_3}{dz} = - \frac{3}{4} \mu_0\varepsilon_0 \chi \omega^2 
\left( A_1 A_2 A_4^*+ (|A_1|^2 + |A_2|^2) A_3 \right)
\label{eq:opc1}
\end{equation}
in 
\begin{equation}
-i k \frac{dA_4}{dz} = - \frac{3}{4} \mu_0\varepsilon_0 \chi \omega^2 
\left( A_1 A_2 A_3^*+ (|A_1|^2 + |A_2|^2) A_4 \right).
\label{eq:opc2}
\end{equation}
Drugi člen na desni že poznamo: opisuje odvisnost lomnega količnika
od intenzitete črpalnih valov, torej optični Kerrov\index{Kerrov pojav!optični}
pojav, in je zato le dodaten prispevek
k fazi. Vpeljemo novi amplitudi, ki se od prejšnjih razlikujeta zgolj v faznem faktorju.
\begin{equation}
\tilde{A}_3 = A_3 \exp\left(-i\frac{ 3 \chi \omega}{4 c_0 n}(|A_1|^2 + |A_2|^2) z\right)
\end{equation}
in 
\begin{equation}
\tilde{A}_4 = A_4 \exp\left(i\frac{ 3 \chi \omega}{4 c_0 n}(|A_1|^2 + |A_2|^2)z\right).
\end{equation}
Ko novi amplitudi vstavimo v diferencialni enačbi~(enačbi~\ref{eq:opc1} in 
\ref{eq:opc2}), se Kerrov prispevek k fazi odšteje
in enačbi se prepišeta v 
\begin{equation}
\frac{d\tilde{A}_{3}}{dz}=i\frac{ 3 \chi \omega}{4 c_0 n}\,
A_{1}A_{2}\tilde{A}_{4}^{*} \qquad \textrm{in} \qquad 
\frac{d\tilde{A}_{4}}{dz}=-i\frac{ 3 \chi \omega}{4 c_0 n}\,
A_{1}A_{2}\tilde{A}_{3}^*.
\label{8.105}
\end{equation}
Vpeljemo še sklopitveno konstanto
\begin{equation}
\kappa=\frac{ 3 \chi \omega}{4 c_0 n}A_1 A_2.
\label{8.106}
\end{equation}
Enačbi se poenostavita v 
\begin{equation}
\frac{d\tilde{A}_{3}}{dz}=i\kappa \tilde{A}_{4}^{*} \quad
\textrm{oziroma} \quad \frac{d\tilde{A}^*_{3}}{dz}=-i\kappa^* \tilde{A}_{4} 
\qquad \textrm{in} \qquad
\frac{d\tilde{A}_{4}}{dz}=-i\kappa \tilde{A}_{3}^*.
\label{8.107}
\end{equation}
Zelo težaven problem nelinearne valovne enačbe smo prevedli na linearen
sistem dveh preprostih sklopljenih enačb za amplitudi signalnega in
odbitega vala. Rešitvi sistema enačb~(\ref{8.107}) 
sta 
\begin{align}
\tilde{A}_3^* \left(z\right) & =  C_{1}\cos(\left|\kappa\right|z)+
C_{2}\sin(\left|\kappa\right|z)
\label{8.108} \qquad \mathrm{in}\\
\tilde{A}_4 \left(z\right) & =  D_{1}\cos(\left|\kappa\right|z)+
D_{2}\sin(\left|\kappa\right|z).
\label{8.108a}
\end{align}
Z upoštevanjem zveze, ki izhaja  iz prve diferencialne enačbe 
(enačba~\ref{8.107}), zapišemo
\begin{equation}
C_1 = \frac{i \kappa^*}{|\kappa|}D_2 \qquad
\textrm{in} \qquad 
C_2 = -\frac{i \kappa^*}{|\kappa|}D_1. 
\end{equation}
Potrebujemo še robne pogoje za obe valovanji. Z leve, pri $z=0$,
poznamo $\tilde{A}_{3}^{*}\left(0\right)$, pri $z=L$ pa ne more biti odbitega
vala in je zato $\tilde{A}_{4}\left(L\right)=0$. S tem določimo konstanti $D_{1}$
in $D_{2}$
\begin{equation}
D_2 = -\frac{i|\kappa|}{\kappa^*} \tilde{A}_3^*(0) \qquad
\textrm{in} \qquad 
D_1 = -D_2 \tan(|\kappa|L). 
\end{equation}
Zdaj lahko zapišemo amplitudi znotraj nelinearne snovi
\begin{align}
\tilde{A}_{3}\left(z\right) & =  \tilde{A}_3(0)
\frac{\cos\left(|\kappa|(L-z)\right)}{\cos\left(|\kappa|L\right)}
\qquad \mathrm{in} \qquad
\tilde{A}_{4}\left(z\right) & =  \tilde{A}_3^*(0)\frac{i \kappa}{|\kappa|}
\frac{\sin\left(|\kappa|(L-z)\right)}{\cos\left(|\kappa|L\right)}.
\label{8.109}
\end{align}
Izračunajmo še amplitudi odbitega in prepuščenega vala. Amplituda odbitega vala 
pri $z=0$ je 
\boxeq{8.110}{
\tilde{A}_{4}(0)  =  \tilde{A}_3^*(0)\frac{i \kappa}{|\kappa|}
\tan \left(|\kappa|L\right),
}
amplituda prepuščenega pri $z = L$ pa
\boxeq{8.110a}{
\tilde{A}_{3}(L)  =  \frac{\tilde{A}_3^*(0)}{
\cos \left(|\kappa|L\right)}.
}
Oglejmo si rezultat podrobneje. Vidimo, da je odbiti val sorazmeren 
kompleksno konjugirani amplitudi vpadnega vala, kar je poglavitna značilnost
fazne konjugacije. 
Poleg konjugirane amplitude ima tudi natanko nasprotni valovni vektor. 
Zanimiva je tudi njegova velikost. Ker 
je lahko $\tan\left(|\kappa|L\right)>1$, je odbit val lahko močnejši od vpadnega.
Ojačenje odbitega vala gre seveda na račun moči črpalnih
valov. V našem računu bi lahko amplituda odbite svetlobe narasla proti neskončnosti, 
vendar zapisane enačbe takrat niso več veljavne, saj smo privzeli, 
da sta signalni in odbiti žarek precej šibkejša od črpalnih.

Poglejmo še prepuščeni žarek. Ker je $\cos(x)\leq1$, je amplituda prepuščenega
žarka vedno večja od amplitude vpadnega. To pomeni, da smo na račun črpalnih žarkov
dobili prepustnost, ki je vedno večja od $100~\%$, in odbojnost, ki je lahko 
večja od $100~\%$.

Pri računu smo predpostavili, da je vpadni signal ravni val. Če je njegova
amplituda odvisna še od prečne koordinate, ga lahko razvijemo po ravnih
valovih in zgoraj izpeljana enačba~(\ref{8.110}) velja za vsako komponento posebej. 
Odbite komponente so sorazmerne s konjugiranimi komponentami signalnega valovanja
z nasprotnim valovnim vektorjem in se seštejejo v valovno fronto enake
oblike kot pri signalnem valovanju, le da potujejo v nasprotni smeri.

\section{Stimulirano Ramanovo in stimulirano Brillouinovo sipanje}
\label{chap:SRS}
\index{Ramanovo sipanje!stimulirano}
\index{Brillouinovo sipanje!stimulirano}
Ko svetloba vpade na snov, se je del siplje. Poleg elastičnega Rayleighovega sipanja,
pri katerem se energija vpadnih fotonov (in z njo frekvenca svetlobe) ohranja, opazimo 
tudi sipanje, pri katerem se energija izhodnih fotonov razlikuje od energije 
vpadnih. 

Če se energija fotonov spremeni zaradi prehajanja molekul snovi med različnimi
vibracijskimi ali rotacijskimi stanji, govorimo o Ramanovem\index{Ramanovo sipanje}
sipanju\footnote{Indijski fizik in nobelovec Sir Chandrasekhara Venkata Raman, 1888--1970.}. 
Ta pojav je mogoč tako v plinih in tekočinah kot tudi v trdnih snoveh. Navadno 
ločimo dva primera: Stokesovo sipanje\footnote{Irski fizik in matematik Sir George Gabriel
Stokes, 1819--1903.}, pri katerem foton preko virtualnega vmesnega stanja 
odda energijo molekuli (slika~\ref{08_Raman}\,a),
in anti-Stokesovo sipanje,
\index{Ramanovo sipanje!Stokesovo}\index{Ramanovo sipanje!anti-Stokesovo}
pri katerem foton prejme energijo od vzbujene molekule (slika~\ref{08_Raman}\,b).
V prvem primeru je frekvenca
sipane svetlobe $\nu_s=\nu_0-\nu_v$, kjer $\nu_0$ označuje frekvenco vpadne
svetlobe in $\nu_v$ vibracijsko frekvenco, ter v drugem primeru $\nu_{as}=\nu_0+\nu_v$.
Ker je v termičnem ravnovesju razmeroma malo molekul v vzbujenem stanju, so slednji 
procesi redki in intenziteta anti-Stokesovega sipanja je zato 
še znatno šibkejša od že tako šibkega Stokesovega sipanja. Tipični Ramanov premik 
$\nu_0-\nu_s$ znaša okoli $10^{12}$--$10^{13}~\si{\hertz}$.
Drug zanimiv primer je, kadar se energija fotonov spremeni zaradi 
vzbujanja akustičnih valov (fononov). Takrat govorimo o Brillouinovem 
sipanju\footnote{Francoski fizik L\'eon Nicolas Brillouin, 1889--1969.} (slika~\ref{08_Raman}\,c).
Tipični Brillouinov premik je $\sim 10^{10}~\si{\hertz}$. \index{Brillouinovo sipanje}

\begin{figure}[ht]
\centering
\def\svgwidth{140truemm} 
\input{slike/08_Raman.pdf_tex}
\caption{Prehodi med energijskimi nivoji za Ramanovo sipanje in shema Brillouinovega sipanja. Vpadna 
svetloba ima frekvenco $\nu_0$, sipana pa $\nu_s$ oziroma $\nu_{as}$. Pri Stokesovem sipanju 
snov prevzame energijo (a), pri anti-Stokesovem sipanju energijo odda (b).
Pri Brillouinovem sipanju svetloba vzbuja akustične valove s frekvenco $\nu_f$ (c) in se na njih odbije.}
\label{08_Raman}
\end{figure}

\subsection*{Stimulirano Ramanovo sipanje}
Pri spontanem Ramanovem sipanju se svetloba siplje na termično vzbujenih fluktuacijah
v snovi. Če se svetloba siplje na fluktuacijah, ki jih je povzročilo vpadno
svetlobno polje, govorimo o stimuliranem Ramanovem sipanju.\footnote{G. Eckhardt et al., Phys. Rev. Lett.
$\mathbf{9}$, 455 (1962).} Dosežemo ga tako, da na snov svetimo z dvema žarkoma hkrati: 
s črpalnim žarkom s frekvenco $\nu_0$ in s Stokesovim žarkom s frekvenco $\nu_s$. 
Fotoni črpalnega žarka se absorbirajo in molekule vzbudijo v virtualno stanje, nato pa 
Stokesov žarek stimulira točno določen prehod. Če razlika frekvenc vpadnih 
žarkov ravno ustreza frekvenci določenega vzbujenega stanja, se  molekule vračajo
v izbrano vzbujeno stanje in pri tem izsevajo dodatne fotone s frekvenco $\nu_s$. Zaradi
stimuliranega sevanja\index{Stimulirano sevanje} je izsevana svetloba po fazi in smeri enaka vpadni in Stokesov
žarek se ojačuje. Tako resonančno ojačenje svetlobe na račun črpalne svetlobe imenujemo stimulirano  
Ramanovo sipanje. Moč svetlobe pri $\nu_s$  narašča eksponentno -- do določene meje, seveda.
Stimulirano Ramanovo sipanje je navadno zelo močno, saj se lahko v Stokesov žarek pretvori 
več kot $\sim 10\%$ energije črpalnega žarka. Za primerjavo, tipični delež sipane svetlobe v primeru 
spontanega Ramanovega sipanja je $10^{-6}/\si{\centi\meter}$.\footnote{Glej npr. 
G. New, {\it Introduction to Nonlinear Optics}, Cambridge University Press (2011).}

Obravnavajmo stimulirano Ramanovo sipanje v klasičnem približku, v katerem snov opišemo 
z $N$ neodvisnimi enodimenzionalnimi harmonskimi oscilatorji na enoto prostornine.\footnote{Glej npr. 
R. W. Boyd, {\it Nonlinear Optics}, tretja izdaja, Academic Press (2008).} Nihanje
posameznega oscilatorja zadošča enačbi\index{Harmonski oscilator}
\begin{equation}
\frac{d^2X(z,t)}{dt^2}+ \gamma \frac{dX}{dt}+\omega_v^2X = \frac{F(z,t)}{m}.
\label{srs:X}
\end{equation}
Pri tem je $X$ odmik od ravnovesne lege, $\gamma$ koeficient dušenja, $m$ masa in $F$ zunanja sila.
Glavna predpostavka modela je, da polarizirnost molekul ni konstantna, ampak odvisna od 
``raztega'' nihajoče molekule oziroma oscilatorja. Polarizirnost $\alpha$, ki jo vpeljemo
kot kvocient med induciranim dipolnim momentom in jakostjo električnega polja $\mathbf{p} = \alpha \mathbf{E}$,
razvijemo v Taylorjevo vrsto do prvega člena in zapišemo dielektričnost
\begin{equation}
\varepsilon = 1+N\alpha/\varepsilon_0 = 1+\frac{N}{\varepsilon_0}\left(\alpha_0 + \frac{d\alpha}{dX}\,X\right),
\label{srs:a}
\end{equation}
pri čemer je $\alpha_0$ polarizirnost pri ravnovesnem razmiku.
Silo na en oscilator izračunamo kot odvod energije po odmiku
\begin{equation}
F = \frac{dW}{dX}= \frac{1}{2}\frac{d\alpha}{dX}\,\overline{E^2}.
\label{srs:F}
\end{equation}
Jakost električnega polja smo zapisali kot povprečje, saj so optične frekvence 
tako hitre, da jim molekule ne morejo slediti. 
Celotno električno polje zapišemo kot vsoto črpalnega in Stokesovega polja
\begin{equation}
E(z,t)= \frac{1}{2}\left( E_0(z)e^{-i\omega_0t}+ E_s(z)e^{-i\omega_st} + \mathrm{k.k.}\right)
\label{eq:srsE}
\end{equation}
in povprečje kvadrata
\begin{equation}
\overline{E^2} = \frac{1}{4}\left(2E_0E_s^* e^{-i(\omega_0-\omega_s)t}+\mathrm{k.k.}\right).
\end{equation}
Vstavimo zapisano jakost električnega polja najprej v izraz za silo (enačba~\ref{srs:F}),
nato pa v enačbo oscilatorja (enačba~\ref{srs:X}). Dobimo
\begin{equation}
\frac{1}{2}\left(\omega_v^2-\omega^2-i\omega\gamma\right)\tilde{X} = 
\frac{1}{4m}\frac{d\alpha}{dX}E_0 E_s^*,
\end{equation}
pri čemer smo vpeljali $\omega = \omega_0-\omega_s$ in $\tilde{X}$, tako da velja
\begin{equation}
X(z,t) = \frac{1}{2}\left(\tilde{X}e^{-i\omega t}+ \mathrm{k.k.}\right).
\label{eq:srsX}
\end{equation}
Oscilatorji torej nihajo s kompleksno amplitudo
\begin{equation}
\tilde{X} = \frac{\frac{d\alpha}{dX}E_0 E_s^*}{2m\left(
\omega_v^2-(\omega_0-\omega_s)^2-i(\omega_0-\omega_s)\gamma\right)}.
\end{equation}
Zdaj lahko zapišemo polarizacijo $P = N\alpha E$. Nelinearni del
polarizacije dobimo z uporabo razvoja v enačbi~(\ref{srs:a}) \index{Električna polarizacija}
in upoštevanjem enačb~(\ref{eq:srsE}) in (\ref{eq:srsX}) 
\begin{equation}
P_{\mathrm{NL}} = \frac{1}{4} N \frac{d\alpha}{dX} \, \left(\tilde{X}
e^{-i(\omega_0-\omega_s)t}
+ \mathrm{k.k.}\right) \cdot \left( E_0(z)e^{-i\omega_0t}+ E_s(z)e^{-i\omega_st} + \mathrm{k.k.}\right).
\end{equation}
Omejimo se le na polarizacijo pri krožni frekvenci $\omega_s$ in zapišemo
\begin{equation}
P_{\mathrm{NL},\omega_s} = \varepsilon_0 \chi_s E_s = 
\frac{N }{8m}\left(\frac{d\alpha}{dX}\right)^2 \, 
\frac{|E_0|^2}{\omega_v^2-(\omega_0-\omega_s)^2+i(\omega_0-\omega_s)\gamma}\,E_s.
\label{srs:chi}
\end{equation}
Efektivna susceptibilnost za svetlobo pri krožni frekvenci $\omega_s$ 
je \index{Susceptibilnost!efektivna} torej kompleksna in sorazmerna intenziteti 
vpadne laserske svetlobe pri krožni frekvenci $\omega_0$. 
V resonanci, ko je $\omega_0-\omega_s = \omega_v$, je efektivna susceptibilnost
imaginarna in negativnega predznaka, kar ima, kot bomo videli, zelo pomembne 
fizikalne posledice. Iz zapisanega izraza je tudi razvidno, zakaj gre pri 
stimuliranem Ramanovem sipanju za nelinearen optični pojav tretjega reda. 

Kompleksna susceptibilnost pomeni kompleksni lomni količnik. Če upoštevamo
le prve člene v razvoju, se vpadna svetloba po snovi širi kot\index{Susceptibilnost!kompleksna}
\begin{equation}
E_s(z) = E_s(0)\exp\left(i k z + ikz\frac{1}{2}\Re(\chi_s)- k z \frac{1}{2}\Im(\chi_s)\right),
\label{srs:Es}
\end{equation}
pri čemer $\Re$ označuje realni in $\Im$ imaginarni del.
\begin{naloga}
\label{nalogasrs}
 Izpelji enačbo~(\ref{srs:Es}), tako da iz efektivne susceptibilnosti izračunaš lomni količnik,
in pokaži, da je imaginarni del susceptibilnosti $\Im(\chi_s)$ vedno negativen. 
\end{naloga}
Ker je imaginarni del efektivne susceptibilnosti vedno negativen (naloga~\ref{nalogasrs}), 
jakost e\-lek\-trič\-nega polja eksponentno narašča na račun črpalnega laserskega snopa. 
Največjo ojačenje je seveda v primeru, ko je sistem v resonanci
in razlika frekvenc vpadne svetlobe ravno enaka vibracijski frekvenci.
Če zanemarimo izgube, lahko zapišemo
\begin{equation}
|E_s(L)|^2 = |E_s(0)|^2 e^{G_RjL}.
\end{equation}
Vrednosti $G_R$ za najmočnejša nihanja so $0,024~\si{\cm/\mega\watt}$ za CS$_2$, \index{CS$_2$}
$0,029~\si{\cm/\mega\watt}$ za LiNbO$_3$ \index{LiNbO$_3$}
in $0,0008~\si{\cm/\mega\watt}$ za SiO$_2$.\index{SiO$_2$}\footnote{A. Yariv in 
P. Yeh, {\it Photonics}, šesta izdaja, Oxford University Press (2007).}
V meter dolgem odseku optičnega vlakna je tako pri gostoti svetlobnega toka 
$10^{10}~\si{\watt/\meter^2}$  faktorja ojačenja $1,08$, na vlaknu dolžine $20~\si{\metre}$
pa $5$.

\begin{remark}
Pojav stimuliranega Ramanovega sipanja se uporablja za izdelavo optičnih 
ojačevalnikov. Posebej pomemben je ta pojav za ojačenje signala v 
optičnih vlaknih, predvsem zaradi velike intenzitete na velikih dolžinah.
\index{Optično ojačevanje!v vlaknih}
\end{remark}

\subsection*{Stimulirano Brillouinovo sipanje}
\index{Brillouinovo sipanje!stimulirano}
Stimulirano Brillouinovo sipanje je pojav, pri katerem vpadno svetlobno valovanje
vzbudi akustični val (fonon), nato pa se na njem siplje. Poleg vpadne svetlobe pri $\nu_0$
se tako pojavi še Stokesova svetloba pri frekvenci $\nu_s = \nu_0-\nu_f$, pri čemer 
$\nu_f$ označuje frekvenco akustičnega vala  (slika~\ref{08_Raman}\,c). Interferenca
vpadnega in Stokesovega valovanja, ki ima komponento ravno pri $\nu_f$, povratno
povečuje intenziteto vzbujenega zvočnega valovanja, njegovo povečanje pa vodi do
večje intenzitete Stokesovega valovanja. Zaradi pozitivne povratne zanke se 
svetloba eksponentno ojačuje. Najmočnejši pojav je 
ravno v nasproti smeri od vpadne svetlobe, v smeri naprej pa Brillouinovega sipanja ni.
Če bi namreč vstopna in izstopna svetloba bili vzporedni, bi bila razlika njunih
valovnih vektorjev enaka nič. 

Računa za Brillouinovo sipanje ne bomo naredili, ga pa lahko bralec poišče
v literaturi.\footnote{Glej npr. R. W. Boyd, {\it Nonlinear Optics}, tretja izdaja, Academic Press (2008).}
Poglavitno 
je, da se tudi pri stimuliranem Brillouinovem sipanju signal pri zmanjšani frekvenci
eksponentno ojačuje
\begin{equation}
|E_s(z)|^2 = |E_s(L)|^2 e^{G_Bj(L-z)}.
\end{equation}
Pri zapisu smo upoštevali, da se val širi in ojačuje v nasprotni smeri od vpadnega.
Vrednosti parametra $G_B$ so na primer $0,13~\si{\cm/\mega\watt}$ za CS$_2$ \index{CS$_2$}
in $0,0045~\si{\cm/\mega\watt}$ za SiO$_2$.\index{SiO$_2$}\footnote{A. Yariv in 
P. Yeh, {\it Photonics}, šesta izdaja, Oxford University Press (2007).}
Ker je koeficient $G_B$ odvisen od zunanjih parametrov, na primer od 
temperature ali pritiska, lahko stimulirano Brillouinovo sipanje uporabimo tudi za
izdelavo natančnih senzorjev. 

\section{Nelinearni pojavi v optičnih vlaknih}
\label{NLOFIB}
V osmem poglavju smo podrobno obravnavali optična vlakna. Omenili smo, da pri 
velikih intenzitetah lahko nastopijo nelinearni pojavi. Oglejmo si nekaj najpomembnejših.\footnote{Glej npr.
G. P. Agrawal, {\it Applications of Nonlinear Fiber Optics}, Academic Press (2001).}

\subsection*{Pojavi drugega reda}\index{Nelinearna optika!v vlaknih}
Optična vlakna so praviloma narejena iz SiO$_2$, \index{SiO$_2$}
za katerega je zaradi simetrije molekul $\chi^{(2)}=0$. 
Nelinearnih pojavov drugega reda zato ne opazimo, razen izjemoma na morebitnih 
nepravilnostih v steklu.\index{Nelinearna optika!drugega reda}
Da bi izkoristili nelinearne optične pojave
drugega reda in v vodnikih dosegli na primer optično frekvenčno podvajanje,
\index{Frekvenčno podvajanje}
morajo biti vodniki zgrajeni iz snovi, ki imajo nelinearno susceptibilnost 
različno od nič. Prednost frekvenčnega podvajanja v vodnikih pred navadnimi 
kristali je v tem, 
da svetloba znotraj sredice vodnika potuje brez uklona. Posledično je
pretvorba iz osnovne v frekvenčno podvojeno svetlobo, ki je sorazmerna s kvadratom 
dolžine poti (enačba~\ref{eq:shgl2}), zelo učinkovita. Seveda 
mora biti izpolnjen tudi pogoj za ujemanje faz.\index{Ujemanje faz} To 
omogoča rodovna disperzija, zaradi \index{Disperzija!rodovna}
katere lahko rodova pri osnovni in podvojeni frekvenci potujeta po vodniku
z enako komponento valovnega vektorja $\beta$.

\subsection*{Pojavi tretjega reda}
V navadnih optičnih vlaknih  
prevladujejo nelinearni pojavi tretjega reda,\index{Nelinearna 
optika!tretjega reda} ki jih v grobem delimo v dve skupini. Prva vključuje
neelastično sipanje (Ramanovo in Brillouinovo), druga pa pojave, ki 
temeljijo na optičnem Kerrovem pojavu.\index{Kerrov pojav!optični}

Obravnavajmo najprej stimulirano Ramanovo sipanje 
(SRS)\index{Ramanovo sipanje!stimulirano}, pri katerem se intenziteta 
svetlobe pri vpadni frekvenci zmanjšuje, na njen račun pa se eksponentno povečuje 
intenziteta valovanja z malenkost nižjo frekvenco. Razlika frekvenc ustreza
vibracijskemu prehodu molekul v snovi. Zaradi SRS se signal v telekomunikacijskih vlaknih 
popači in poveča se njegova spektralna širina. Po drugi strani stimulirano Ramanovo sipanje 
izkoriščamo za ojačenje signala v vodnikih. Če v vlakno posvetimo z močno črpalno
svetlobo, katere frekvenca se od signalnega žarka razlikuje za vibracijsko frekvenco
($\sim 13~\si{THz}$), se signalni žarek ojači. Ker je spekter SiO$_2$ razmeroma 
širok ($\sim 5~\si{THz}$), tega pogoja ni težko izpolniti. 

Pri stimuliranem Brillouinovem sipanju (SBS) se vpadna svetloba
\index{Brillouinovo sipanje!stimulirano} odbije na 
optično vzbujenem akustičnem valu v snovi. Signal
v smeri naprej oslabi, pojavi pa se odbiti val, katerega intenziteta
narašča eksponentno z intenziteto vpadne svetlobe. 
Neželenemu pojavu se lahko izognemo z zmanjšanjem vpadne
moči (pod $\sim 100~\si{mW}$ na kanal) 
ali povečanjem spektralne širine vpadne svetlobe.

Druga skupina nelinearnih pojavov temelji na spreminjanju 
lomnega količnika z intenziteto vpadne svetlobe (optični Kerrov pojav).
Opišimo tri pojave:
\begin{enumerate}
\item 
Samomodulacija faze ({\it SPM -- Self-Phase Modulation}). Različni deli
sunka zaradi različne intenzitete občutijo različen lomni količnik in pride
do tako imenovanega čirikanja~(enačba~\ref{eq:chirpi} 
in slika~\ref{fig:optsoliton}). \index{Samomodulacija faze}
Pojav vodi do spektralne razširitve in zaradi\index{Čirikanje}
disperzije tudi do časovnega podaljšanja sunka. Z ustrezno disperzijo dosežemo 
krajšanje sunkov ali pojav optičnih solitonov (razdelek~\ref{chap:soliton}).
\index{Soliton!optični}\\

\item
Navzkrižna modulacija faze ({\it CPM -- Cross-Phase Modulation}).\index{Navzkrižna
modulacija faze} Ko po vlaknu
potuje več svetlobnih sunkov hkrati, prvi sunek povzroči spremembo lomnega količnika, 
drugi sunki pa to spremembo občutijo. Medsebojna motnja med sunki povzroči 
spektralno razširitev. Pojav lahko izkoristimo za krajšanje sunkov ali
za izdelavo optičnih stikal, 
saj lahko z zunanjim kontrolnim žarkom spreminjamo fazo izbranega sunka. \\

\item
Sklopitev štirih valov ({\it FWM -- Four-wave mixing}). Če po vlaknu potuje več
valovanj z različnimi frekvencami,\index{Sklopitev štirih valov} 
se zaradi nelinearne sklopitve pojavijo
valovanja pri dodatnih frekvencah (vsotah in razlikah frekvenc obstoječih valovanj). To velja
predvsem v vlaknih z zelo majhno disperzijo, v katerih pride do ujemanja
faz.  
\end{enumerate}

Čeprav so nelinerani pojavi v vlaknih pogosto nezaželeni, jih lahko s pridom
izkoriščamo pri izdelavi npr. novih vlakenskih laserjev, ojačevalnikov, preklopnikov ali 
optičnih logičnih vezij. 
