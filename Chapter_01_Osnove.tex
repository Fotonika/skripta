\chapterimage{slike/Nebo.jpg} % Chapter heading image

\chapter{Elektromagnetno valovanje}
Za začetek bomo osvežili osnove teorije elektromagnetnega polja in 
elektromagnetnega valovanja. Obnovili bomo zapis Maxwellovih enačb in 
valovne enačbe, opisali osnovne pojave valovanja (lom, odboj in uklon)
in si ogledali razširjanje svetlobe v anizotropnih snoveh. 

\section{Maxwellove enačbe}
Elektromagnetno polje v praznem prostoru opišemo z dvema vektorskima
poljema, električnim in magnetnim, ki sta v splošnem funkciji lege $\mathbf{r}$
in časa $t$. Vsaki točki v prostoru lahko priredimo \index{Električno polje!jakost}jakost
električnega polja $\mathbf{E}(\mathbf{r},t)$ in \index{Magnetno polje!jakost}jakost
magnetnega polja $\mathbf{H}(\mathbf{r},t)$. Za opis elektromagnetnega
polja v snovi vpeljemo dve dodatni količini. To sta \index{Električno polje!gostota}gostota
električnega polja $\mathbf{D}(\mathbf{r},t)$ in gostota magnetnega
polja\index{Magnetno polje!gostota} $\mathbf{B}(\mathbf{r},t)$.
Vse te količine povezujejo \index{Maxwellove enačbe}Maxwellove
enačbe\footnote{Škotski fizik James Clerk Maxwell, 1831--1879.}
\boxeq{eq:Maxwell1}{
\nabla\times\mathbf{H} & =\frac{\partial\mathbf{D}}{\partial t}+\mathbf{j}_e\\
\nabla\times\mathbf{E} & =-\frac{\partial\mathbf{B}}{\partial t}\label{eq:Maxwell2}\\
\nabla\cdot\mathbf{D} & =\mbox{\ensuremath{\rho}}_{e}\label{eq:Maxwell3}\\
\nabla\cdot\mathbf{B} & =0.\label{eq:Maxwell4}
}\\
Pri zapisu enačb smo upoštevali tudi izvore polj, to je gostoto
električnega toka $\mathbf{j}_e(\mathbf{r},t)$ in gostoto naboja $\rho_{e}(\mathbf{r},t)$, ki 
pa ju bomo v nadaljevanju izpuščali.

Poleg Maxwellovih enačb veljata zvezi
\begin{align}
\mathbf{D} & =\epsilon_{0}\mathbf{E}+\mathbf{P} \quad \mathrm{in}\\
\mathbf{B} & =\mu_{0}\mathbf{H}+\mu_{0}\mathbf{M},
\end{align}
kjer $\mathbf{P}$ označuje \index{Električna polarizacija}električno
polarizacijo, to je gostoto električnih dipolov, $\mathbf{M}$
pa \index{Magnetizacija}magnetizacijo, to je gostoto magnetnega momenta.
Polarizacija in magnetizacija sta odvisni od zunanjih polj $\mathbf{E}$
in $\mathbf{H}$. V splošnem sta njuni odvisnosti zelo zapleteni,
v izotropnih in linearnih snoveh\footnote{Med nelinearne snovi, za katere napisani zvezi ne veljata,
sodijo na primer feroelektriki in feromagneti.}
pa se zvezi poenostavita v 
\begin{equation}
\mathbf{P}=\epsilon_{0}\chi_e\mathbf{E} = \epsilon_{0}(\epsilon-1)\mathbf{E} \qquad \textrm{in} 
\qquad
\mathbf{M}=\chi_m \mathbf{H} = (\mu-1)\mathbf{H}
\label{eq:PM}.
\end{equation}
Vpeljali smo \index{Električna susceptibilnost} električno ($\chi_e$) in 
\index{Magnetna susceptibilnost}magnetno ($\chi_m$) susceptibilnost ter
\index{Dielektričnost}dielektričnost $\epsilon$ in
\index{Magnetna permeabilnost}magnetno permeabilnost $\mu$. Ko združimo gornje
enačbe, lahko zapišemo dve konstitutivni
relaciji
\begin{align}
\mathbf{D} & =\epsilon_{0}\epsilon\mathbf{E}\quad \mathrm{in}\\
\mathbf{B} & =\mu_{0}\mu\mathbf{H}.
\end{align}
V linearnih anizotropnih snoveh moramo namesto skalarnih vrednosti $\varepsilon$
in $\mu$ zapisati tenzorje. 

\subsection*{Robni pogoji}
Navedene Maxwellove enačbe zadoščajo za opis elektromagnetnega polja
v neomejeni snovi, kjer so vse komponente polj zvezne funkcije. Za
obravnavo v omejeni snovi moramo vedeti tudi, kaj se z elektromagnetnim
poljem zgodi na meji dveh sredstev\index{Maxwellove enačbe!robni pogoji}. Pri prehodu
iz enega dielektrika v drugega se ohranjata normalni komponenti gostote
električnega in magnetnega polja ter tangentni komponenti jakosti
električnega in magnetnega polja (slika \ref{fig:Robni-pogoji}) 
\boxeq{eq:robni-pogoji}{
D_{1n} &=  D_{2n}\\
B_{1n} &=  B_{2n}\\
E_{1t} &=  E_{2t}\label{eq:robni-pogoji4}\\
H_{1t} &=  H_{2t}\label{eq:robni-pogoji5}.
}
Privzeli smo, da na meji med dielektrikoma ni površinskih
tokov ali nabojev, sicer bi morali robna pogoja za $D_n$ in $H_t$
ustrezno popraviti. Na meji dielektrika z idealnim prevodnikom (kovino,
zrcalom) so robni pogoji drugačni. Za magnetno polje jih ne moremo
preprosto zapisati, za električno polje pa velja, da mora biti tangentna
komponenta jakosti električnega polja enaka nič. Posledica tega 
pogoja je, da se pri pravokotnem vpadu na zrcalo faza valovanja
spremeni za $\pi$. 

\begin{figure}[h]
\centering
  \def\svgwidth{75truemm} 
  \input{slike/01_robni_pogoji.pdf_tex}
\caption{Na meji med dvema dielektrikoma brez površinskih tokov
in nabojev se ohranjata tangentni komponenti $E_t$ in $H_t$ ter 
normalni komponenti $D_n$ in $B_n$.}
\label{fig:Robni-pogoji}
\end{figure}

\section{Valovna enačba in Poyntingov vektor}
Večinoma bomo obravnavali elektromagnetna valovanja v izotropnih, 
homogenih in linearnih snoveh brez zunanjih izvorov in
nosilcev naboja ($\mathbf{j}_e=0$ in $\rho_{e}=0$). 
Iz Maxwellovih enačb (enačbe~\ref{eq:Maxwell1}--\ref{eq:Maxwell4}) izpeljemo valovno 
enačbo\index{Valovna enačba} za jakost električnega ali magnetnega polja 
\boxeq{eq:valovna-skalarna}{
\nabla^{2}\mathbf{E}-\frac{1}{c^{2}}\frac{\partial^{2}\mathbf{E}}{\partial t^{2}} = 0
\quad \mathrm{in} \quad
\nabla^{2}\mathbf{H}-\frac{1}{c^{2}}\frac{\partial^{2}\mathbf{H}}{\partial t^{2}} = 0.
}
Pri tem je hitrost valovanja\index{Hitrost valovanja} v snovi enaka 
\boxeq{eq:c}{
c=\frac{1}{\sqrt{\epsilon\epsilon_{0}\mu\mu_{0}}}=\frac{c_{0}}{n}.
}
Magnetne in dielektrične lastnosti snovi smo pospravili
v lomni količnik $n$\index{Lomni količnik}, ki pove, kolikokrat je hitrost svetlobe v snovi manjša
od hitrosti svetlobe v praznem prostoru.
\begin{equation}
n=\frac{c_{0}}{c}=\sqrt{\epsilon\mu}.
\end{equation}
Za izotropno in nemagnetno snov ($\mu=1$) je lomni količnik $n=\sqrt{\epsilon}$.

Vpeljimo še vektor gostote energijskega toka, to je Poyntingov vektor\footnote{Angleški 
fizik John Henry Poynting, 1852--1914.} 
\index{Poyntingov vektor}{$\mathbf{S}$}
\boxeq{eq:Poyntingov-vektor}{
\mathbf{S} = \mathbf{E} \times \mathbf{H}.
}
Iz lastnosti vektorskega produkta sledi, da je smer energijskega toka vedno pravokotna na 
smeri $\mathbf{E}$ in $\mathbf{H}$. Gostoto energijskega toka $j$, to je količino
energije, ki v danem času preteče skozi dano ploskev
z normalo $\mathbf{\hat{n}}$, izračunamo kot časovno povprečje projekcije
Poyntingovega vektorja \index{Gostota energijskega toka}
\begin{equation}
j=\left\langle \mathbf{\mathbf{S}}\cdot\mathbf{\hat{n}}\right\rangle.
\end{equation}
Gostoti energijskega toka, predvsem gostoti svetlobnega toka, pravimo tudi 
intenziteta\index{Intenziteta|see {Gostota energijskega toka}}.

\begin{definition}[Poyntingov teorem]
Iz Maxwellovih enačb (enačbe~\ref{eq:Maxwell1}--\ref{eq:Maxwell4}) 
izpelji kontinuitetno enačbo v odsotnosti električnih tokov
\begin{equation}
-\nabla\cdot\mathbf{\mathbf{S}}=\frac{\partial}{\partial t}\left(\frac{1}{2}\epsilon_{0}\mathbf{E}^{2}+
\frac{1}{2}\mu_{0}\mathbf{H}^{2}\right)+\mathbf{E}\cdot\frac{\partial\mathbf{P}}{\partial t}+
\mu_{0}\mathbf{H}\cdot\frac{\partial\mathbf{M}}{\partial t}.
\label{eq:PoyntingTh}
\end{equation}
Pomagaj si z zvezo $\nabla\cdot(\mathbf{E}\times\mathbf{H})=(\nabla\times\mathbf{E})\cdot\mathbf{H}-
(\nabla\times\mathbf{H})\cdot\mathbf{E}$.
\end{definition}

\index{Poyntingov teorem}Poyntingov teorem (enačba~\ref{eq:PoyntingTh}) predstavlja
izrek o ohranitvi energije. Prvi in drugi člen na desni strani Poyntingovega 
teorema namreč opišeta gostoto energije, shranjene v električnem in magnetnem
polju, medtem ko tretji in četrti člen opišeta energijo,
shranjeno v snovi (električnih in magnetnih dipolih). Za valovanje
v homogeni izotropni snovi tako velja
\begin{equation}
-\nabla\cdot\mathbf{S}=\frac{\partial w}{\partial t},
\end{equation}
kjer je $w$\index{Gostota energije} celotna
gostota energije elektromagnetnega polja. Zapišemo jo kot 
\boxeq{eq:gostota-energije}{
w=\frac{1}{2}\epsilon\epsilon_{0}\mathbf{E}^{2}+\frac{1}{2}\mu\mu_{0}\mathbf{H}^{2}.
}

Valovno enačbo in ohranitvene zakone lahko zapišemo tudi za anizotropne,
nehomogene ali nelinearne snovi. Nekaj teh primerov bomo srečali v nadaljevanju
in jih bomo obravnavali sproti.

\section{Monokromatski elektromagnetni val}
Reševanje valovne enačbe si navadno poenostavimo s kompleksnim
zapisom jakosti električnega in magnetnega polja. Račun si
oglejmo na primeru monokromatskega elektromagnetnega vala. Nastavek
za monokromatski val s krožno frekvenco $\omega$ naj bo\index{Elektromagnetno valovanje}
\begin{equation}
\mathbf{E}(\mathbf{r},t)  =\mathfrak{\Re}(\mathbf{E}(\mathbf{r})e^{-i\omega t})\qquad \textrm{in} \qquad
\mathbf{H}(\mathbf{r},t)  =\mathfrak{\Re}(\mathbf{H}(\mathbf{r})e^{-i\omega t}),
\label{eq:rval}
\end{equation}
kjer sta $\mathbf E(\mathbf{r})$ in $\mathbf H(\mathbf{r})$ časovno
neodvisna vektorja jakosti električnega\index{Električno polje!jakost} in 
magnetnega polja\index{Magnetno polje!jakost} s kompleksno
amplitudo. Podobno lahko vpeljemo tudi kompleksne vektorje $\mathbf{P}$,
$\mathbf{M}$, $\mathbf{D}$ in $\mathbf{B},$ ki opisujejo realne količine (polarizacijo,
magnetizacijo, gostoto električnega in gostoto magnetnega polja).
V nadaljevanju bomo večinoma pisali polja v kompleksni obliki, pri
čemer se bomo držali gornje definicije. Zavedati pa se moramo, da
je uporaba kompleksnega zapisa zgolj računski pripomoček, na koncu
je treba rezultate vedno izraziti z realnimi količinami. 

Če vstavimo nastavek za monokromatski val (enačba~\ref{eq:rval}) v valovno enačbo
(enačba~\ref{eq:valovna-skalarna}), dobimo \index{Helmholtzeva enačba}Helmholtzevi
enačbi\footnote{Nemški zdravnik in fizik Hermann Ludwig Ferdinand von Helmholtz, 1821--1894.} 
za kompleksna vektorja jakosti električnega in magnetnega polja.
V homogenem in izotropnem sredstvu ju zapišemo kot
\boxeq{eq:Helmholtz}{
\nabla^{2}\mathbf{E}(\mathbf{r})+k^{2}\mathbf{E}(\mathbf{r}) &=0\\
\nabla^{2}\mathbf{H}(\mathbf{r})+k^{2}\mathbf{H}(\mathbf{r}) &=0,
}
kjer je $k=nk_{0}=n\omega/c_{0}$ valovno število\index{Valovno število}. Vpeljemo lahko
tudi kompleksni Poyntingov vektor\index{Poyntingov vektor}
\begin{equation}
\mathbf{S}(\mathbf{r})=\frac{1}{2}\mathbf{E}(\mathbf{r})\times\mathbf{H}^{*}(\mathbf{r}).
\label{eq:Poyntingov-vektor-c}
\end{equation}
\begin{definition}
Upoštevajoč izraz za Poyntingov vektor $\mathbf{S}$
(enačba \ref{eq:Poyntingov-vektor}) pokaži, da lahko gostoto svetlobnega toka $j$
(oziroma povprečje projekcije Poyntingovega vektorja)
\index{Gostota energijskega toka} izrazimo
s kompleksnim Poyntingovim vektorjem $\mathbf{S}(\mathbf{r})$
\begin{equation}
\label{eq:jReS}
j=\left\langle \mathbf{\mathbf{S}}\cdot\mathbf{\hat{n}}\right\rangle =
\frac{1}{4}\left(\mathbf{E}\times\mathbf{H}^{*}+\mathbf{E}^{*}\times\mathbf{H}\right)=\Re(\mathbf{S}
(\mathbf{r})).
\end{equation}
\end{definition}

\section{Ravni val}
Osnovna rešitev valovne enačbe (enačba~\ref{eq:valovna-skalarna}) je ravni 
val\index{Ravni val}. Nastavek, ki predstavlja
ravni val in hkrati reši Helmholtzevo enačbo (enačba~\ref{eq:Helmholtz}), je oblike
\begin{equation}
\mathbf{E}(\mathbf{r},t)  =\mathbf{E}(\mathbf{r})e^{-i\omega t}=
\mathbf{E}_{0}e^{i\mathbf{k}\cdot\mathbf{r}-i \omega t}
\end{equation}
in podobno za magnetno polje
\begin{equation}
 \mathbf{H}(\mathbf{r},t)  =\mathbf{H}(\mathbf{r})e^{-i\omega t}=
\mathbf{H}_{0}e^{i\mathbf{k}\cdot\mathbf{r}-i \omega t},
\end{equation}
pri čemer sta vektorja $\mathbf{E}_{0}$ ter $\mathbf{H}_{0}$ od kraja in časa neodvisna. 
Velikost valovnega vektorja\index{Valovni vektor} $\mathbf{k}$ je valovno število $k=nk_{0},$ pri
čemer je $n$ lomni količnik izotropne in homogene snovi. 

Vektorja jakosti električnega in magnetnega polja zadostujeta
Maxwellovim enačbam (enačbe~\ref{eq:Maxwell1}--\ref{eq:Maxwell4}), iz katerih sledi,
da sta polji vedno medsebojno pravokotni, 
hkrati pa pravokotni na valovni vektor $\mathbf{k}$. Elektromagnetno valovanje je torej
transverzalno valovanje. 
Zaradi enolične zveze med električnim in magnetnim poljem zadošča
za opis ravnega vala le eno polje, navadno se odločimo za električno.

\begin{definition}
\label{naloga-TEM-ortogonalnost}
Pokaži, da za ravni val vedno velja $\mathbf{D} \perp \mathbf{k}$ in 
$\mathbf{B} \perp \mathbf{k}$. Izpelji še zvezi
\begin{align}
\mathbf{k}\times\mathbf{H}_{0} & =-\omega\epsilon\epsilon_{0}\mathbf{E}_{0}\label{eq:TEM-pogoj1}\
\quad \mathrm{in}\\
\mathbf{k}\times\mathbf{E}_{0} & =\omega\mu\mu_{0}\mathbf{H}_{0}\label{eq:TEM-pogoj2},
\end{align}
iz katerih izhaja, da sta v izotropni snovi $\mathbf{E}_0\perp \mathbf{k}$ in 
$\mathbf{H}_0\perp \mathbf{k}$ ter da je $\mathbf{E}_0\perp \mathbf{H}_0$.
\end{definition}

Energijski tok je pri ravnem valu v izotropni snovi vedno v smeri valovnega vektorja, pravokotno
na valovne fronte. Iz definicije za gostoto svetlobnega toka (enačba \ref{eq:jReS}) 
ter Maxwellovih enačb (enačbe~\ref{eq:Maxwell1}--\ref{eq:Maxwell4}) 
sledi\index{Gostota energijskega toka}
\begin{equation}
j=\Re\left(\frac{1}{4}E_{0}H_{0}^{*}+\frac{1}{4}E_{0}^*H_{0}\right)=
\frac{1}{2}c\epsilon\epsilon_{0}\left|E_{0}\right|^{2}=\frac{1}{2}c_{0}n\epsilon_{0}
\left|E_{0}\right|^{2}.
\label{eq:j}
\end{equation}
Gostota svetlobnega toka je torej sorazmerna
s kvadratom amplitude jakosti električnega polja. Poglejmo nekaj primerov.
Gostoti toka $j=1~\si{kW/m^{2}}$
(približna gostota svetlobnega toka s Sonca na Zemljinem površju) tako v praznem prostoru ustreza 
jakost električnega polja $E_{0}=868~\si{\volt/\meter}$, gostoti $j=1~\si{W/\micro m^{2}}$ 
(intenziteta močno fokusiranega laserskega žarka) pa $E_{0}=27~\si{MV/m}$. 

Zapišimo še povprečno gostoto energije valovanja\index{Gostota energije}. 
K energiji prispevata tako magnetno kot električno polje. Oba prispevka sta enaka, zato velja
\begin{equation}
\left\langle w\right\rangle =\frac{1}{4}\epsilon\epsilon_{0}\left|E_{0}\right|^{2}+
\frac{1}{4}\mu\mu_{0}\left|H_{0}\right|^{2}=\frac{1}{2}\epsilon\epsilon_{0}\left|E_{0}\right|^{2}.
\end{equation}
Povprečna gostota energije $w$, pomnožena s hitrostjo svetlobe v
snovi, da gostoto energijskega toka $j$ oziroma intenziteto 
svetlobe\index{Gostota energijskega toka}
\boxeq{eq:jcw}{
j=cw.
}
Gornji izraz nazorno kaže, da je intenziteta svetlobe pravzaprav pretok
energije. To si lahko predstavljamo, če vzamemo valj s prečnim presekom
$S$ in dolžino $c\Delta t$. V volumnu $Sc\Delta t$ je potem shranjene $wSc\Delta t$
energije. Energija, ki preteče skozi presek $S$ v času $\Delta t$,
je ravno $cw$. 

Intenziteta ravnega vala je neodvisna od kraja in časa, iz česar sledi,
da je povsod po prostoru enaka. Če bi hoteli izračunati energijo,
ki jo nosi ravni val, bi opazili, da je ta energija neskončna. To
seveda ni mogoče, zato se je vedno treba zavedati, da je ravni val
le idealiziran, a nazoren in praktičen približek elektromagnetnega
vala.

\section{Polarizacija EM valovanja}
Jakost električnega polja elektromagnetnega valovanja je v izotropnem
sredstvu vedno pravokotna na smer valovnega vektorja\footnote{V splošnem velja $\mathbf{E}\perp\mathbf{S}$ 
(enačba~\ref{eq:Poyntingov-vektor}) in $\mathbf{D}\perp\mathbf{k}$ (naloga~\ref{naloga-TEM-ortogonalnost}). 
To velja tudi v anizotropnih sredstvih, vendar je tam $\mathbf{E} \nparallel \mathbf{D}$ in 
$\mathbf k\nparallel\mathbf S$.}. Vektor $\mathbf{E}$
tako leži v ravnini, ki je pravokotna na smer valovnega vektorja, njegovo
smer pa opiše polarizacija\index{Polarizacija}. 

Električno polje ravnega vala lahko razstavimo na dve medsebojno 
pravokotni komponenti vektorja $\mathbf{E}(\mathbf{r})$. Ti dve komponenti
nihata sinusno z enako frekvenco, lahko pa se razlikujeta v amplitudi in fazi. 
V splošnem je tak val eliptično polariziran\index{Polarizacija!eliptična} in
vrh vektorja električne poljske jakosti $\mathbf E$ v ravnini, ki je pravokotna na smer
širjenja, orisuje elipso. Kadar je elipsa izrojena v daljico, govorimo o linearno polariziranem 
valovanju\index{Polarizacija!linearna},
kadar pa je krog, govorimo o cirkularno polariziranem valu\index{Polarizacija!cirkularna}. 
Poljubno polarizacijo lahko vedno zapišemo kot vsoto dveh linearno ali dveh cirkularno polariziranih
valovanj. 

Priročen zapis polarizacije je s kompleksnim \index{Jonesov vektor}Jonesovim 
vektorjem\footnote{Ameriški fizik Robert Clark Jones, 1916--2004.}
$\mathbf{J}$. Za monokromatski ravni val, ki se širi v smeri $z$ in ima komponenti $E_x$ in $E_y$, je
Jonesov vektor
\begin{equation}
\mathbf{J}=\frac{1}{|E_{0}|}\left[\begin{array}{c}
E_{x}\\
E_{y}
\end{array}\right].
\end{equation}
Dodali smo normalizacijski faktor $|E_{0}|=\sqrt{|E_{x}|^{2}+|E_{y}|^{2}}$,
da je Jonesov vektor normiran in $\mathbf{J}\cdot\mathbf{J}^{*}=1$.
Ravni val, linearno polariziran v smeri $x$, tako zapišemo kot $\mathbf{J}=\left(1,0\right)$,
val, ki je linearno polariziran pod kotom $\ang{45}$ glede na osi
$x$ in $y$, pa je $\mathbf{J}=\left(1,1\right)/\sqrt{2}$.
Za zapis cirkularno polariziranega valovanja ni enotnega dogovora, tukaj zapišimo
desno cirkularno polarizirano valovanje kot 
$\mathbf{J}=\left(1,-i\right)/\sqrt{2}$,
levo cirkularno polarizirano pa z $\mathbf{J}=\left(1,i\right)/\sqrt{2}$.
V našem zapisu je desno polariziran val tisti val, pri katerem se električna
poljska jakost na danem mestu vrti v desno, če gledamo proti izvoru valovanja. 

Zapis z Jonesovim vektorjem je prikladen, saj omogoča preprost izračun
prehoda ravnega vala skozi optične elemente, ki spreminjajo polarizacijo,
a ohranjajo obliko ravnega vala. V splošnem se pri prehodu skozi
sredstvo spremeni kompleksna amplituda $\mathbf{E}_1 = (E_{1x}, E_{1y})$
\begin{align}
E_{2x} & =A_{11}E_{1x}+A_{12}E_{1y}\\
E_{2y} & =A_{21}E_{1x}+A_{22}E_{1y},
\end{align}
pri čemer so komponente $A_{ij}$ odvisne od lastnosti sredstva. Enačbi zapišemo 
v matrični obliki $\mathbf{E}_{2}=A\cdot\mathbf{E}_{1}$ oziroma z Jonesovimi
vektorji $\mathbf{J}_{2}=A\cdot\mathbf{J}_{1}$, kjer $\mathbf{J}_{1}$ in $\mathbf{J}_{2}$
opisujeta polarizaciji vstopnega in izstopnega vala, $A$ pa je \index{Jonesova matrika}Jonesova
matrika. Jonesova matrika za prehod skozi linearni polarizator, ki
polarizira v smeri $x$, je
\begin{equation}
A=\left[\begin{array}{cc}
1 & 0\\
0 & 0
\end{array}\right].
\end{equation}
Oglejmo si še dva zanimiva primera.
Jonesova matrika za optični element, ki eni komponenti doda fazni zamik $\pi$ (tak element 
imenujemo ploščica $\lambda/2$\index{Ploščica $\lambda/2$}), je
\begin{equation}
A=\left[\begin{array}{cc}
1 & 0\\
0 & -1
\end{array}\right].
\end{equation}
Ploščica $\lambda/2$ tako spremeni desno cirkularno polariziran val v levo polariziran in obratno,
linearno polariziran val pa prezrcali čez koordinatno os. Podobno je 
Jonesova matrika za element, ki eni komponenti doda fazni zamik $\pi/2$ (imenujemo ga 
ploščica $\lambda/4$\index{Ploščica $\lambda/4$}), enaka
\begin{equation}
A=\left[\begin{array}{cc}
1 & 0\\
0 & i
\end{array}\right].
\end{equation}
Ploščica $\lambda/4$ linearno polarizirano valovanje z Jonesovim vektorjem $(1,1)/\sqrt{2}$
spremeni v levo cirkularno polarizirano valovanje, cirkularno polarizirano
valovanje pa nazaj v linearno. 

\begin{definition}
Pokaži, da je Jonesova matrika za polarizator,
ki prepušča polarizacijo pod kotom $\vartheta$ glede na os $x$, podana z matriko
\begin{equation}
A=\left[\begin{array}{cc}
\cos^{2}\vartheta & \sin\vartheta\cos\vartheta\\
\sin\vartheta\cos\vartheta & \sin^{2}\vartheta
\end{array}\right].
\end{equation}
Namig: matriko $A'$, ki opisuje polarizator v smeri $x$, zapiši v zasukanem
koordinatnem sistemu $A=R(\vartheta) \cdot {A'}\cdot R(\vartheta)^\textrm{T}$, 
kjer je $R(\vartheta)$ rotacijska matrika.
\end{definition}

\section{Lom in odboj EM valovanja}
Na ravni meji med dvema izotropnima dielektrikoma se del svetlobe
odbije po odbojnem zakonu, ki pravi, da je odbojni kot enak vpadnemu, 
del pa lomi po lomnem zakonu\index{Lomni zakon} \boxeq{eq:lomni_zakon}{
n_{1}\sin\vartheta_{1}=n_{2}\sin\vartheta_{2}.
}
S kotoma $\vartheta_{1}$in $\vartheta_{2}$ smo označili vpadni in lomni
kot, $n_{1}$in $n_{2}$ pa sta lomna količnika prve in druge snovi.
Poglejmo, kaj se pri lomu in odboju zgodi s polarizacijo valovanja (slika~\ref{fig:Lom}).
Dogovorimo se, da valovanje, pri katerem je električna poljska jakost vzporedna z mejno 
ravnino in pravokotna na vpadno ravnino, imenujemo transverzalno električno
(TE) valovanje\index{Polarizacija!TE}. Kadar leži v mejni ravnini 
jakost magnetnega polja, govorimo o transverzalnem magnetnem valovanju 
(TM)\index{Polarizacija!TM}.\\

\begin{figure}[h]
\centering \def\svgwidth{140truemm} 
  \input{slike/01_lom.pdf_tex}
\caption{Lom elektromagnetnega valovanja na meji dveh izotropnih dielektrikov. 
Levo: transverzalno električno (TE) valovanje. Desno: transverzalno magnetno (TM) valovanje.}
\label{fig:Lom}
\end{figure}

Z $E_1$ označimo amplitudo jakosti električnega polja vpadnega valovanja, 
z $E_2$ prepuščenega in z $E_3$ odbitega.
Nato vpeljemo amplitudno prepustnost $t$ in amplitudno odbojnost $r$, 
ki pa sta odvisni od vpadne polarizacije. Zapišemo
\begin{align}
E_{2\mathrm{TE}} & =t_{\mathrm{TE}}E_{1\mathrm{TE}} & E_{3\mathrm{TE}} &=r_{\mathrm{TE}}E_{1\mathrm{TE}}\\
E_{2\mathrm{TM}} & =t_{\mathrm{TM}}E_{1\mathrm{TM}} & E_{3\mathrm{TM}}&=r_{\mathrm{TM}}E_{1\mathrm{TM}}.
\end{align}

Koeficiente $r$ in $t$ izračunamo iz robnih pogojev (enačbe 
\ref{eq:robni-pogoji}--\ref{eq:robni-pogoji5}). Enačbe, ki opisujejo odvisnost odbojnosti
in prepustnosti od vpadnega kota za različni vpadni polarizaciji, imenujemo
\index{Fresnelove enačbe}Fresnelove enačbe\footnote{Francoski fizik in inženir Augustin Jean Fresnel, 1788--1827.}.
Za TE polarizacijo jih zapišemo 
\begin{equation}
r_{\mathrm{TE}}=\frac{n_{1}\cos\vartheta_{1}-n_{2}\cos\vartheta_{2}}{n_{1}\cos\vartheta_{1}+n_{2}\cos\vartheta_{2}},
\label{eq:Fresnel1}
\end{equation}
\begin{equation}
t_{\mathrm{TE}}=1+r_{\mathrm{TE}}=\frac{2n_{1}\cos\vartheta_{1}}{n_{1}\cos\vartheta_{1}+n_{2}\cos\vartheta_{2}}
\end{equation}
in za TM polarizacijo
\begin{equation}
r_{\mathrm{TM}}=\frac{n_{1}\cos\vartheta_{2}-n_{2}\cos\vartheta_{1}}{n_{1}\cos\vartheta_{2}+n_{2}\cos\vartheta_{1}},
\end{equation}
\begin{equation}
t_{\mathrm{TM}}=(1+r_{\mathrm{TM}})\frac{\cos\vartheta_{1}}{\cos\vartheta_{2}}=\frac{2n_{1}\cos\vartheta_{1}}
{n_{1}\cos\vartheta_{2}+n_{2}\cos\vartheta_{1}}.
\label{eq:Fresnel2}
\end{equation}

V splošnem sta odbojnost $r$ in prepustnost $t$ kompleksni
količini, saj iz lomnega zakona sledi, da je $\cos\vartheta_{2}=
\sqrt{1-\left(n_{1}/n_{2}\right)^{2}\sin^{2}\vartheta_{1}}$
lahko kompleksen. Velikost števila $\left|r\right|$ tako predstavlja
odbojnost, argument $\arg\{r\}$ pa spremembo faze
pri odboju.

Amplitudna odbojnost $r$ in amplitudna prepustnost $t$ povesta, kako se spremeni 
kompleksna amplituda jakosti električnega polja pri odboju oziroma lomu.
Razmerje med intenziteto odbite in vpadne svetlobe $\mathcal{R}$ oziroma 
razmerje med intenziteto prepuščene in vpadne svetlobe
$\mathcal{T}$ izračunamo kot 
\begin{equation}
\mathcal{R}=\left|r\right|^{2} \qquad \mathrm{in} \qquad \mathcal{T}=1-\mathcal{R}.
\end{equation}
Slednja enačba sledi iz ohranitve energije. V splošnem $\mathcal{T}$
ni enak $\left|t\right|^{2},$ saj energijski tok potuje po različnih
snoveh in v različnih smereh. Velja zveza
\begin{equation}
\mathcal{T}=\frac{n_{2}\cos\vartheta_{2}}{n_{1}\cos\vartheta_{1}}\left|t\right|^{2}.
\end{equation}
Najpreprostejši primer je pravokotni vpad na mejo dveh sredstev. Zaradi simetrije 
morata biti odbojnost in prepustnost neodvisni od polarizacije in 
\begin{equation}
r_{\mathrm{TE}} = r_{\mathrm{TM}} = \frac{n_1-n_2}{n_1+n_2}
\end{equation}
in
\begin{equation}
t_{\mathrm{TE}} = t_{\mathrm{TM}} = \frac{4n_1n_2}{n_1+n_2}. 
\end{equation}
Ob pravokotnem vpadu iz zraka na steklo ($n_2 \approx 1,5$) je tako odbojnost 
\begin{equation}
\mathcal{R} = \left(\frac{1-n_2}{1+n_2}\right) \approx 0,04.
\end{equation}

Poglejmo še odvisnost odbojnosti in prepustnosti od vpadnega kota (slika~\ref{fig:Brewster}). 
Pri tem je pomembno, ali se svetloba lomi v optično gostejše ($n_1<n_2$) ali v optično
redkejše sredstvo ($n_1>n_2$). Najprej obravnavajmo primer loma v optično gostejšo snov (sliki a in c).
Vidimo, da je pri nekem kotu, imenujemo ga \index{Brewsterjev kot}Brewsterjev 
kot\footnote{Škotski fizik in znanstvenik Sir David Brewster, 1781--1868.}, odbojnost 
za TM polarizirano valovanje enaka nič. Ob vpadu svetlobe na mejo dveh sredstev pod 
Brewsterjevim kotom je tako odbito valovanje vedno TE polarizirano. Po drugi strani to pomeni,
da je pri tem vpadnem kotu prepustnost za TM komponento enaka $\mathcal{T}=1$, za TE komponento pa $\mathcal{T}<1$. 

Negativni predznak odbojnosti $r_{\mathrm{TE}}$ pomeni, da ima odbiti TE polarizirani val pri vpadu
na optično gostejše sredstvo ravno nasprotno 
fazo od vpadnega. Za TM val pa je faza za vpadne kote, ki so manjši od Brewsterjevega, nasprotna, 
za večje vpadne kote pa ima odbita svetloba enako fazo kot vpadna. 

\begin{figure}[h]
\centering
  \def\svgwidth{140truemm} 
  \input{slike/01_BrewsterNew.pdf_tex}
\caption{Amplitudna odbojnost $r$ za obe vpadni polarizaciji (a, b) in razmerje med intenziteto odbite
in vpadne svetlobe $\mathcal{R}$ v odvisnosti od vpadnega kota (c, d). Za primer na slikah (a) in (c) velja
$n_1<n_2$, za primer na slikah (b) in (d) pa $n_1>n_2$. Z zeleno je označen Brewsterjev kot $\vartheta_B$, 
z vijolično pa kot mejni totalnega odboja $\vartheta_T$.}
\label{fig:Brewster}
\end{figure}

\begin{definition}
Pokaži, da Brewsterjev kot izračunamo kot 
\begin{equation}
\vartheta_{B}=\arctan\left(\frac{n_2}{n_1}\right).
\label{eq:Brew}
\end{equation}
\end{definition}

\begin{remark}
Prozorne ploščice, ki so postavljene pod Brewsterjevim kotom glede na vpadno svetlobo, imenujemo
Brewsterjeva okna\index{Brewsterjevo okno}. Njihova značilnost je, da eno polarizacijo (TM) prepustijo v celoti, druge 
polarizacije (TE) pa se del odbije, del pa prepusti. Brewsterjeva okna so zelo uporabna 
pri izdelavi resonatorjev plinskih laserjev, saj so izgube za izbrano polarizacijo zelo majhne, 
za drugo pa razmeroma velike.  
\end{remark}

Pri vpadu na optično redkejše sredstvo (sliki~\ref{fig:Brewster}\,b in d) je pomemben še en kot, to je mejni kot
totalnega odboja $\vartheta_T = \arcsin\left(n_2/n_1\right)$. Pri vpadnih kotih, ki so večji od 
$\vartheta_T$, se svetloba v celoti odbije in govorimo o totalnem ali popolnem odboju\index{Totalni odboj}. 
Kljub temu električna poljska jakost v optično redkejšem sredstvu
ni enaka nič, saj se tam pojavi evanescentno polje\index{Evanescentno polje}.
To je polje, ki se širi v smeri mejne 
ravnine, njegova amplituda pa pojema eksponentno z oddaljenostjo od mejne plasti. Pri tem je 
vdorna globina odvisna od valovne dolžine valovanja, lomnega količnika snovi in tudi od vpadnega kota.
Čeprav se v optično redkejši snovi pojavi električno polje, je Poyntingov
vektor v smeri pravokotno na mejno ploskev v povprečju enak nič in zato ne pride do prenosa energije
v drugo snov. 

\begin{remark}
Pri prehodu skozi optične elemente se --
razen TM polariziranega valovanja pri Brewsterjevem 
kotu -- vedno nekaj
svetlobe odbije. Da zmanjšamo te izgube, optične elemente navadno
prekrijemo z antirefleksno plastjo, to je nanosom ene ali več primerno
izbranih debelin plasti dielektrikov z ustreznimi lomnimi količniki.
Zaradi destruktivne interference se količina odbite svetlobe z izbrano
valovno dolžino občutno zmanjša. Ker so laserji
koherentni izvori svetlobe s točno določeno valovno dolžino, za zmanjšanje
izgub, na primer v resonatorju laserja, uporabljamo optične
elemente (leče, kristale, akusto-optične modulatorje ... ) z ustrezno
antirefleksno plastjo.
\end{remark}

\section{Uklon svetlobe}\index{Uklon}
Kadar svetloba vpade na oviro, za oviro nastane senca. Nastala senca ni ostra, ampak
ima zaradi uklona zabrisane robove. Obravnave 
uklona svetlobe na odprtinah ali zaslonkah se lotimo z uporabo
skalarnega približka teorije elektromagnetnega polja. To pomeni, da vpliva 
polarizacije ne upoštevamo. Ta je pomemben zgolj pri zelo majhnih odprtinah, 
kjer je velikost odprtine po velikosti podobna valovni dolžini svetlobe $a \sim \lambda$. 
Vendar so tudi v tem primeru uklonske slike za različne polarizacije podobne, 
razlikujejo pa se po intenziteti uklonjene svetlobe.

\begin{remark}
Primer, kjer skalarni približek ne da pravih rezultatov, je uklon na mrežici, narejeni 
iz zelo tankih prevodnih žic. Takšna mrežica deluje kot polarizator za vpadno elektromagnetno valovanje.
Elektromagnetni val s polarizacijo, ki je vzporedna žicam, pri prečkanju v žicah inducira tok
in val se delno odbije in delno absorbira. Za valovanje, ki je polarizirano pravokotno 
na žice, je ta inducirani tok bistveno manjši, saj je smer toka omejena
vzdolž žice. Posledično je prepustnost velika le za eno polarizacijo. 
Takšni polarizatorji se večinoma uporabljajo v mikrovalovni tehniki, 
vendar se v zadnjih letih z razvojem in izboljšavo litografskih postopkov
vse pogosteje uporabljajo tudi v bližnjem infrardečem področju svetlobe.
\end{remark}

Pri velikostih odprtin $a\gg\lambda$ torej uporabimo skalarno obliko valovne enačbe 
(enačba~\ref{eq:valovna-skalarna})
\begin{equation}
\nabla^2 E - \frac{1}{c^2}\frac{\partial^2 E}{\partial t^2} = 0.
\label{eq:skalarna-valovna-enačba}
\end{equation}
Zapisana valovna enačba velja za vse komponente električne $E$ in tudi 
magnetne poljske jakosti $H$. Časovna odvisnost polja $E$ je harmonična funkcija in 
je sorazmerna z $e^{-i \omega t}$. Z uporabo Greenovega teorema lahko 
jakost polja $E_P$ v točki $P$ izrazimo s poljem na poljubni ploskvi, ki obkroža točko $P$ (slika \ref{fig:UklonFK}). 
Zvezo opisuje Kirchhoffov integral\footnote{Nemški fizik Gustav Kirchhoff, 1824--1887.} 
\index{Kirchhoffov integral}
\begin{equation}
E_P = -\frac{1}{4\pi}\oint \left(E\,\mathbf{n}\cdot \nabla \frac{e^{ikr}}{r}-
\frac{e^{ikr}}{r}\mathbf{n}\cdot \nabla E \right) dS,
\label{eq:Kirchhoffov-integral}
\end{equation}
kjer je $\mathbf{n}$ normala na ploskev, po kateri teče integral, $r$ pa je oddaljenost od točke P
do ploskve $dS$. Kirchhoffov integral velja splošno za katerokoli harmonično 
funkcijo, ki reši valovno enačbo (enačba~\ref{eq:skalarna-valovna-enačba}), ne samo $E$, in torej
ni vezan na obravnavo uklona svetlobe. 

Vzemimo točkast izvor v točki $S$. Svetloba iz njega vpada na zaslon, 
z odprtino poljubne oblike. Izračunajmo skalarno polje v točki $P$ na drugi 
strani zaslona. Vpadno svetlobo zapišemo kot
\begin{equation}
\label{eq:polje-krogelni-val}
E = A \frac{e^{ikr'}}{r'},
\end{equation}
kjer je $r'$ razdalja od izvora do točke na zaslonu, A pa zaradi ohranitve energije konstanta.
Integracijska ploskev je lahko poljubna sklenjena ploskev, ki objema točko $P$. Izberemo tako, ki
zajema odprtino na zaslonu, poleg tega pa naredimo še dva približka:\\
\begin{enumerate}
\item Jakost polja $E$ in njen gradient doprineseta k integralu le na odprtini, na preostanku ploskve
pa sta njuna prispevka zanemarljivo majhna.\\
\item Vrednost $E$ in njen gradient sta na odprtini takšna, kot da zaslona ne bi bilo.\\
\end{enumerate}
Gornja približka sta precej groba, vendar se izkaže, da se kljub temu
dobro ujemata z eksperimentalno določeno uklonsko sliko, s čimer 
upravičimo njuno uporabo.

\begin{figure}[]
\centering \def\svgwidth{85truemm} 
  \input{slike/01_uklonFK.pdf_tex}
\caption{Integracijska ploskev v Kirchhoffovem integralu zajema odprtino in objema točko $P$.}
\label{fig:UklonFK}
\end{figure}

Kirchhoffov integral za primer točkastega izvora svetlobe se potem zapiše kot integral, pri čemer
integriramo zgolj po odprtini
\begin{equation}
E_P = -\frac{ik A e^{-i\omega t}}{4\pi}\int\frac{e^{ik(r+r')}}{rr'}\left[\cos(\mathbf{n},
\mathbf{r})-\cos(\mathbf{n},\mathbf{r'})\right] dS.
\label{eq:Fresnel-Kirchoffov-integral}
\end{equation}
V literaturi ga pogosto imenujejo Fresnel-Kirchhoffov uklonski integral\index{Fresnel-Kirchhoffov integral}.
\begin{definition}
\label{naloga-Fresnel-Kirchhoff-uklon}
Uporabi Kirchhoffov integral (enačba~\ref{eq:Kirchhoffov-integral}) in pokaži, da za primer krožnega vpadnega vala
(enačba~\ref{eq:polje-krogelni-val}) polje v točki $P$ zapišemo s Fresnel-Kirchhoffovim integralom 
(enačba~\ref{eq:Fresnel-Kirchoffov-integral}). Pri tem privzemi, da je oddaljenost točke $P$ od odprtine
$r \gg \lambda$.
\end{definition}

Oglejmo si poseben primer, ko leži točkast izvor svetlobe na osi okrogle odprtine. Razdalja
$r'$ je potem konstantna in polje v točki $P$ izračunamo kot 
\begin{equation}
\label{eq:Fresnelov-uklon}
E_P =  -\frac{ik}{4\pi} \int E_S\frac{ e^{ikr-i\omega t}}{r}\left[\cos(\mathbf{n},\mathbf{r})+1\right] dS,
\end{equation}
pri čemer $E_S$ predstavlja kompleksno amplitudo vpadnega polja v odprtini
\begin{equation}
E_S = A \frac{e^{ikr'}}{r'}.
\end{equation} 

Zgornja oblika Fresnel-Kirchhoffovega integrala ni pravzaprav nič drugega kot 
matematičen zapis \index{Huygensovo načelo}Huygensovega 
načela\footnote{Nizozemski znanstvenik Christiaan Huygens, 1629--1695.}. 
Spomnimo se, da Huygensovo načelo pravi, da lahko vsako točko valovne fronte obravnavamo 
kot izvor novega krogelnega vala. Točno to je zapisano tudi v gornjem integralu. Vpadni val
$E_S$ v vsakem od elementov $dS$ odprtine vzbudi krogelno valovanje s
kompleksno amplitudo
\begin{equation}
E = A_0 \frac{e^{ikr}}{r},
\end{equation} 
polje v izbrani točki $P$ pa je vsota prispevkov posameznih krogelnih valovanj.

Za razliko od osnovnega Huygensovega načela, v Fresnel-Kirchhoffovem integralu 
(enačba~\ref{eq:Fresnel-Kirchoffov-integral})
nastopa še faktor $\left[\cos(\mathbf{n},\mathbf{r})+1\right]$, ki poskrbi, da ni valovanja 
v smeri nazaj proti izvoru. Tudi faktor $-i$ manjka v osnovnem Huygensovem načelu,
pomeni pa, da je uklonjeno valovanje fazno zakasnjeno za $\pi/2$ glede na osnovno
valovanje $E_S$.

Uporabna razširitev Fresnel-Kirchhoffovega uklona je z uporabo prepustnostne funkcije odprtine $T$.
Z njo v splošnem popišemo amplitudne in fazne spremembe, do katerih lahko pride na raznih 
odprtinah, lečah, uklonskih mrežicah ... Razširjen uklonski integral zapišemo kot
\boxeq{eq:Fresnelov-uklon2}{
E_P =  -\frac{ik}{4\pi} \int T(r) E_S(r) \frac{ e^{ikr-i\omega t}}{r}
\left[\cos(\mathbf{n},\mathbf{r})+1\right] dS,
}
pri čemer smo upoštevali tudi splošno obliko vpadnega vala $E_S(r)$.

\subsection*{Fraunhoferjev in Fresnelov približek}
\label{FFuklon}
Izračun Fresnel-Kirchhoffovega integrala je v splošnem zelo zapleten, zato se 
pogosto poslužujemo dveh približkov: Fraunhoferjevega\footnote{Nemški fizik 
Joseph von Fraunhofer, 1787--1826.} in Fresnelovega\index{Fresnelov uklon}. 
Fraunhoferjeva uklonska slika\index{Fraunhoferjev uklon} velja za daljno polje, kadar lahko 
vpadni in uklonjeni val zapišemo kot ravna valova. Bolj zapleteno Fresnelovo uklonsko sliko pa moramo
uporabiti, kadar obravnavamo primer bližnjega polja in ukrivljene valovne fronte.
\begin{figure}[h]
\centering \def\svgwidth{90truemm} 
  \input{slike/01_uklonFF2.pdf_tex}
\caption{V Fraunhoferjevem približku valovanji obravnavamo kot ravna valova 
in dobimo znano uklonsko sliko (levo). Fresnelov približek moramo uporabiti za obravnavno uklona v 
primeru bližnjega polja (desno).}
\label{fig:UklonFF}
\end{figure}

Mejo med daljnim in bližnjim poljem kvalitativno določa Fresnelovo 
število\index{Fresnelovo število}
\begin{equation}
F= \frac{a^2}{L\lambda}.
\end{equation} 
Pri tem je $a$ karakteristična dimenzija odprtine, $L$ tipična oddaljenost od odprtine ter
$\lambda$ valovna dolžina. V grobem velja, da lahko Fraunhoferjev približek uporabimo, kadar je $F<1$ in je odstopanje
faze od ravnega vala znotraj odprtine majhno. Sicer moramo uklon obravnavati v Fresnelovem 
približku ali celo v polni obliki. 

Zapišimo uklonske integrale za oba približka. Izhajajmo iz Fresnel-Kirchhoffovega integrala
za točkast izvor
(enačba~\ref{eq:Fresnel-Kirchoffov-integral}) in zapišimo razdaljo $r$ s koordinatama na zaslonu $x', y'$ ter
lego točke $P$ s koordinatami $x,y,z$ (slika \ref{fig:Uklon-koordinate}). Privzamemo,
da je oddaljenost do zaslona bistveno večja od prečnih dimenzij. 

\begin{figure}[h]
\centering \def\svgwidth{105truemm} 
  \input{slike/01_uklon_koordinate.pdf_tex}
\caption{K izračunu Fraunhoferjevega in Fresnelovega uklona. Zaslon, kjer opazujemo uklonsko sliko,
je od odprtine oddaljen za $z$.}
\label{fig:Uklon-koordinate}
\end{figure}

Zapišemo
\begin{equation}
r = \sqrt{(x-x')^2+(y-y')^2 + z^2}
\end{equation}
in razvijemo
\begin{equation}
r \approx z + \frac{(x-x')^2}{2z} +\frac{(y-y')^2}{2z}.
\end{equation}

V Fraunhoferjevem približku zadošča uporaba le linearnih členov v gornjem izrazu in za Fraunhoferjev
uklonski integral dobimo
\begin{equation}
\label{eq:FraunhoferApprox}
E_P =  \frac{1}{i\lambda z} e^{i k (z + (x^2+y^2) /2z)}\int \int E_S e^{-ik (xx'+yy')/z} dx' dy',
\end{equation}
kar ni nič drugega kot Fouriereva transformacija polja $E_S$.

V Fresnelovem približku upoštevamo tudi kvadratne člene v razvoju in dobimo
\begin{equation}
\label{eq:FresnelApprox}
E_P =  \frac{1}{i \lambda z } e^{i k z}\int \int E_S e^{-ik ((x-x')^2+(y-y')^2)/2z} dx' dy'.
\end{equation}

\begin{definition}
\label{naloga-Frauhofer-Kirchhoff-uklon}
Pokaži, da je v Fraunhoferjevi uklonski sliki uklon na okrogli odprtini s premerom $a$ podan z
\begin{equation}
I_P = I_0\frac{4 J_1(\pi a \rho/ \lambda z)}{\pi a \rho/ \lambda z},
\end{equation}
kjer je $\rho = \sqrt{x^2+y^2}$.
\end{definition}

\section{EM valovanje v anizotropnih snoveh}\label{chap:anizotropni}
Do zdaj smo obravnavali elektromagnetno valovanje
v izotropnih snoveh, v katerih je dielektričnost skalar in hitrost širjenja
valovanja neodvisna od njegove smeri. V splošnem so snovi anizotropne,
dielektričnost\index{Dielektričnost} je tenzor, hitrost potovanja svetlobe \index{Hitrost valovanja}
skozi snov pa je odvisna od smeri razširjanja in od polarizacije valovanja. 

Gostoto električnega polja~\index{Električno polje!gostota} v anizotropni snovi zapišemo kot 
\begin{equation}
\mathbf{D}=\epsilon_{0}\underline{\epsilon} \cdot\mathbf{E} = 
\epsilon_{0}
\left[\begin{array}{ccc}
\epsilon_{11} & \epsilon_{12}& \epsilon_{13}\\
\epsilon_{21} & \epsilon_{22}& \epsilon_{23}\\
\epsilon_{31} & \epsilon_{32}& \epsilon_{33}\\
\end{array}\right]\mathbf{E},
\label{eq:gostota-elektricnega-polja-tenzor}
\end{equation}
kjer je $\underline{\epsilon}$ tenzor drugega reda in ima v splošnem devet komponent.
V dielektričnih snoveh, v katerih ne pride do optične aktivnosti ali absorpcije, je tenzor
realen in simetričen $\epsilon_{ij}=\epsilon_{ji}^*$. Tak tenzor lahko vedno
diagonaliziramo, torej poiščemo koordinatni sistem, v katerem je 
diagonalen. V takem koordinatnem sistemu velja 
\begin{equation}
\mathbf{D} = \epsilon_{0}
\left[\begin{array}{ccc}
\epsilon_{1} & 0& 0\\
0 & \epsilon_{2}& 0\\
0 & 0& \epsilon_{3}\\
\end{array}\right]\mathbf{E}
\end{equation}
in 
\begin{align}
D_{1}&=\epsilon_{0}\epsilon_{1}E_{1}\nonumber \\
D_{2}&=\epsilon_{0}\epsilon_{2}E_{2}\nonumber \\
D_{3}&=\epsilon_{0}\epsilon_{3}E_{3}.\label{eq:gostota-elektricnega-polja-lastni}
\end{align}
Glavne osi sistema določajo smeri, vzdolž katerih sta jakost
in gostota električnega polja vzporedni, lastne vrednosti 
pa ustrezajo trem lomnim količnikom\index{Lomni količnik} $\epsilon_{i}=\sqrt{n_{i}}$. Snovi,
za katere so vse tri vrednosti $n_i$ različne, imenujemo optično dvoosne snovi\index{Dvoosne snovi}, 
medtem ko sta v optično enoosnih snoveh\index{Enoosne snovi} dve lastni vrednosti enaki $n_{1}=n_{2}$. 
Če so enake vse tri lastne vrednosti, je snov izotropna.

\subsection*{Elipsoid lomnega količnika}
\index{Elipsoid lomnega količnika}
Poglejmo, kako se v anizotropnih snoveh širi valovanje v odvisnosti
od njegove smeri in polarizacije. Preprost primer je valovanje, ki se širi vzdolž lastne
osi $z$, polarizirano pa je vzdolž lastne osi $x$. Pri prehodu
skozi kristal se polarizacija valovanja ohrani, lomni količnik za
tak val pa je $n_{1}$. Podobno velja za val, polariziran v smeri
$y$, za katerega je lomni količnik enak $n_{2}$. Če se valovanje širi vzdolž lastne
osi $z$, vendar njegova polarizacija ne sovpada z lastnima osema
$x$ ali $y$, nastane po prehodu skozi kristal iz vpadnega linearno polariziranega valovanja
v splošnem eliptično valovanje\index{Polarizacija!eliptična}. Lastni komponenti namreč potujeta z različnima
hitrostma, zato pride med njima do faznega zamika.

\begin{figure}[h]
\centering
\def\svgwidth{40truemm} 
\input{slike/01_indikatrisa.pdf_tex}
\caption{Elipsoid lomnega količnika oziroma optična indikatrisa. Glavne osi
elipsoida sovpadajo z glavnimi osmi tenzorja $\varepsilon$, polosi elipsoida pa so enake
lomnim količnikom $n_i$. V primeru optično enoosnega kristala je indikatrisa  
rotacijski elipsoid, v primeru izotropne snovi pa je optična indikatrisa krogla.}
\label{fig:Indikatrisa}
\end{figure}

Za poljubno smer širjenja valovanja ter poljubno polarizacijo je račun
bolj zapleten in je treba lomne količnike še izračunati. Pri tem si pomagamo 
z grafično upodobitvijo, s tako imenovanim elipsoidom lomnega količnika oziroma
\index{Optična indikatrisa|see {Elipsoid lomnega količnika}}optično indikatriso. 

Optična indikatrisa je elipsoid, podan z enačbo
\begin{equation}
\frac{x^{2}}{n_{1}^{2}}+\frac{y^{2}}{n_{2}^{2}}+\frac{z^{2}}{n_{3}^{2}}=1.
\end{equation}
Glavne osi elipsoida sovpadajo z glavnimi osmi tenzorja $\epsilon$, 
polosi elipsoida pa so $n_{1}$, $n_{2}$ in $n_{3}$. 
Lomne količnike za val, ki se širi z valovnim vektorjem $\mathbf{k}$, skozi izhodišče
narišemo ravnino, pravokotno na smer valovnega vektorja (slika~\ref{fig:Indikatrisa}).
Presečišče ravnine in elipsoida je elipsa, katere glavni osi podata
vrednosti lomnih količnikov $n_{a}$ in $n_{b}$ za obe lastni polarizaciji,
njuni smeri pa predstavljata lastni smeri gostote električnega polja. Ker
sta to lastni osi, smer jakosti električnega polja izračunamo z
enačbo (\ref{eq:gostota-elektricnega-polja-lastni}).

\subsection*{Optično enoosni kristali}
\index{Enoosne snovi}
V optično enoosnih kristalih sta dve lastni vrednosti enaki in indikatrisa je rotacijski 
elipsoid. Po dogovoru izberemo lastne vrednosti tako, da velja $n_{1}=n_{2}\neq n_{3}$. 
Za valovanje, ki se razširja v smeri $z$, sta lomna količnika za obe polarizaciji 
enaka in os $z$ imenujemo \index{Optična os}optična os. Hitrost valovanja, ki
se širi vzdolž optične osi, je torej neodvisna od njegove polarizacije.
Navadno vpeljemo tudi nove oznake: $n_{1}=n_{2}=n_{o}=n_{\perp}$, ki označuje redni (\textit{ordinary})
lomni količnik\index{Lomni količnik!redni}, $n_{3}=n_{e}=n_\parallel$ pa izredni 
(\textit{extraordinary}) lomni količnik\index{Lomni količnik!izredni}. 

\begin{table}[h]
 \centering
\begin{tabular}{|l|c|c|} \hline  
      Snov & $n_o$ & $n_e$ \\ \hline
      CaCO$_3$ (kalcit) & 1,6557 & 1,4849 \\ \hline
      BaTiO$_3$ & 2,4042 & 2,3605 \\ \hline
      LiNbO$_3$ & 2,2864 & 2,2022 \\ \hline
      KH$_2$PO$_4$ & 1,5074 & 1,4669 \\ \hline
      tekoči kristal 5CB ($25~\si{\degreeCelsius}$) & 1,5319 & 1,7060 \\ \hline
      telur ($\lambda = 10~\si{\micro\metre}$) & 4,7969 & 6,2455 \\
\hline 
\end{tabular}
  \caption{Redni in izredni lomni količnik za nekaj izbranih kristalov. Razen v primeru telurja
   veljajo vrednosti za svetlobo z valovno dolžino 633~\si{\nano\metre}.}
\label{table:none}
\end{table}

Za lažjo predstavo skiciramo ploskev valovnega vektorja, ki jo zaradi simetrije v tem 
primeru lahko upodobimo kar ravninsko (slika~\ref{fig:Elipsa}). 
Smer valovnega vektorja določa le kot 
$\vartheta$ med valovnim vektorjem $\mathbf{k}$ in optično osjo $z$. Dvema 
lastnima polarizacijama pa pripadata dva različna lomna količnika. 
Lomni količnik za žarek, ki je polariziran pravokotno na vpadno ravnino, 
je enak $n_o$ in neodvisen od $\vartheta$, zato narišemo krožnico. Takemu žarku pravimo redni žarek. 
Žarek, ki je polariziran v vpadni ravnini, je izredni žarek. Pripadajoč
lomni količnik\index{Lomni količnik} je odvisen od kota $\vartheta$ in ga izračunamo iz sledeče enačbe elipse
s polosema $n_o$ in $n_e$: 
\boxeq{eq:izreden}{
\frac{1}{n^{2}(\vartheta)}=\frac{\cos^{2}\vartheta}{n_{o}^{2}}+\frac{\sin^{2}\vartheta}{n_{e}^{2}}.
}
Opazimo, da se krožnica in elipsa dotikata ravno na osi $z$. Takrat se 
valovanje širi vzdolž optične osi in lomna količnika sta za obe polarizaciji enaka $n_o$. 

\begin{figure}[h]
\centering
\def\svgwidth{140truemm} 
\input{slike/01_elipsa.pdf_tex}
\caption{V optično enoosnih kristalih je lomni količnik odvisen
od smeri valovnega vektorja in njegove polarizacije. a) primer pozitivno anizotropne
snovi ($n_e>n_o$). b) Redni žarek je polariziran pravokotno na vpadno ravnino. Zanj velja, 
da je $\mathbf{D} \parallel \mathbf{E}$ in $\mathbf{S} \parallel \mathbf{k}$. Izredni žarek
je polariziran v vpadni ravnini. Smer širjenja žarka $\mathbf{S}$ ni vzporedna z valovnim vektorjem
$\mathbf{k}$,
prav tako valovne fronte niso pravokotne nanjo. c) Primer negativno anizotropne snovi ($n_e< n_o$).}
\label{fig:Elipsa}
\end{figure}

Navadno sta pri ravnem valu vektorja $\mathbf{E}$ in $\mathbf{D}$ vzporedna, 
prav tako $\mathbf{k}$ in $\mathbf{S}$. Žarek se širi v smeri valovnega vektorja, 
valovne fronte pa so pravokotne nanj. To velja tudi za redni žarek v anizotropnih
snoveh. Izredni žarek pa ima, kot že ime nakazuje, ``izredne'' lastnosti. Vektorja
$\mathbf{E}$ in $\mathbf{D}$ nista vzporedna, zato tudi valovni vektor $\mathbf{k}$ ni vzporeden
toku energije oziroma Poyntingovemu vektorju $\mathbf{S}$ (slika \ref{fig:Elipsa}\,b). 
Žarek, ki ga vidimo, tako potuje v smeri, ki ni enaka smeri valovnega vektorja. Smer
Poyntingovega vektorja določimo kot normalo na elipso pri kotu $\vartheta$. 

\subsection*{Dvojni lom}
Ko vpade žarek na anizotropno snov, se lomi. Privzemimo, da je lomni količnik snovi,
iz katere valovanje prehaja v anizotropen kristal, enak 1. Vemo, da je hitrost valovanja v 
snovi -- in s tem tudi kot, pod katerim se lomi -- odvisna od polarizacije valovanja. V splošnem
se v anizotropnih snoveh pojavita dva lomljena žarka z različnima polarizacijama, kar
da ime pojavu: \index{Dvolomnost}dvolomnost (slika~\ref{fig:dvolomnost}). 
Da zadostimo ohranitvi faze
pri prehodu, moramo popraviti tudi \index{Lomni zakon}lomni zakon (enačba \ref{eq:lomni_zakon}).

Za redni val s TE polarizacijo (pravokotno na vpadno ravnino) velja navadni lomni zakon, pri čemer
je lomni količnik snovi enak rednemu lomnemu količniku $n_o$
\begin{equation}
\sin\vartheta_{1}=n_{o}\sin\vartheta_{o}.
\end{equation}
Za izredni val s TM polarizacijo (električna poljska jakost leži v vpadni ravnini) pa velja
\begin{equation}
\sin\vartheta_{1}=n(\vartheta_e)\sin\vartheta_{e},
\end{equation}
pri čemer $n(\vartheta_e)$ izračunamo iz enačbe (\ref{eq:izreden}).

\begin{figure}
\centering
\def\svgwidth{140truemm} 
\input{slike/01_dvolom.pdf_tex}
\caption{Dvojni lom. a) Pri poševnem vpadu na anizotropno snov se
valovanje loči na dva različno polarizirana žarka. b) Tudi pri pravokotnem vpadu se žarka 
ločita, če je optična os pod poljubnim kotom glede na normalo mejne ravnine. Valovni vektorji
so v tem primeru kolinearni, Poyntingovi vektorji pa imajo različne smeri.}
\label{fig:dvolomnost}
\end{figure}

Kadar vpada valovanje pravokotno na izotropno snov, se žarek ne lomi. V 
dvolomnih snoveh pa lahko tudi pri pravokotnem vpadu pride do razklona žarkov (slika
\ref{fig:dvolomnost}\,b). Kadar je optična os nagnjena glede na vpadnico, sta valovna vektorja
obeh prepuščenih žarkov sicer vzporedna valovnemu vektorju vpadnega žarka ($\mathbf{k}_1 \parallel
\mathbf{k}_o \parallel \mathbf{k}_e$), razlikujejo pa se smeri Poyntingovih vektorjev
($\mathbf{S}_1 \parallel \mathbf{S}_o \nparallel \mathbf{S}_e$). Ob prehodu skozi 
plast anizotropne snovi tako nastaneta dva vzporedna, a razmaknjena žarka z medsebojno
pravokotnimi polarizacijami. Enoosne kristale, odrezane pod primernim kotom in prave dolžine,
zato lahko uporabimo kot polarizacijski delilnik žarkov.

\begin{figure}[h]
\centering
\def\svgwidth{140truemm} 
\input{slike/01_FotoDvolom.pdf_tex}
\caption{Dvojni lom v kristalu kalcita (islandski dvolomec). \index{Kalcit}
Z linearnim polarizatorjem pokažemo, da sta lomljena žarka različnih
polarizacij.}
\label{foto:dvolom}
\end{figure}
