\chapterimage{Resonator.jpg} % Chapter heading image

\chapter{Optični resonatorji}
Osnovni gradnik vsakega laserja je resonator. V njem je vzbujeno lastno valovanje,
ki skozi delno prepustno steno resonatorja izhaja iz sistema. V tem poglavju bomo 
najprej spoznali optične resonatorje in kriterije za njihovo stabilno delovanje,
nato pa izračunali lastne frekvence resonatorjev ter povezali spektralno širino črt
izsevane svetlobe z izgubami v laserskem resonatorju. 

\section{Odprti resonatorji}
Optični resonatorji\index{Resonator} so votline, v katerih je mogoče 
vzpostaviti stoječe svetlobno\index{Stoječe valovanje} valovanje. Če vzbujamo v resonatorju valovanja z 
različnimi frekvencami, se pri nekaterih diskretnih frekvencah pojavi resonančno
obnašanje. Te frekvence imenujemo lastne frekvence\index{Lastne frekvence resonatorja}
resonatorja. Zaradi dušenja imajo lastne frekvence končno frekvenčno oziroma spektralno 
širino, pri čemer je čas dušenja mnogo daljši od nihajne periode. 

Optične resonatorje uporabljamo predvsem v dva namena:
\begin{enumerate}
\item V resonatorjih lahko ob razmeroma šibkem zunanjem vzbujanju nastane velika
električna poljska jakost pri resonančni frekvenci. Za vzdrževanje
konstantne amplitude lastnega nihanja mora zunanji vir zgolj pokrivati izgube
v resonatorju. Če so te majhne, je zunanji vir lahko šibek, polje
v resonatorju pa vseeno veliko.\\
\item Resonatorji delujejo kot filtri, ki prepuščajo le polje s točno  
določeno frekvenco in prostorsko odvisnostjo. Primer take uporabe je 
Fabry-Perotov interferometer\index{Fabry-Perotov interferometer}, opisan 
v nadaljevanju. 
\end{enumerate}

Resonatorje poznamo z različnih področij, na primer akustične pri glasbilih ali 
mikrovalovne in radijske. V mikrovalovnem področju, kjer so valovne dolžine valovanja
okoli $1~\si{\centi\metre}$, 
so resonatorji zaprte kovinske votline z dimenzijami, ki so primerljive z 
valovno dolžino vzbujenega valovanja. Resonančne frekvence so v takem resonatorju 
med seboj dobro ločene in ni težko doseči enega samega nihanja v izbranem 
frekvenčnem intervalu.

V optičnem področju je drugače, saj so resonatorji navadno veliko večji
od valovne dolžine svetlobe. Vzemimo kocko s stranico $L\gg \lambda$, 
v kateri je vzbujeno zelo veliko število lastnih valovanj. Število lastnih valovanj $N$ 
na frekvenčni interval $\Delta \nu$ zapišemo z zvezno gostoto stanj\index{Gostota stanj}
\begin{equation}
N=2 \frac{1}{8} 4\pi k^{2}\Delta k \frac{1}{(\pi/L)^3}=\frac{\omega^{2}}{\pi^{2}c^{3}}V\Delta\omega=
\frac{8\pi \nu^{2}}{c^{3}}V\Delta\nu.
\label{eq:N-stevilo-stanj}
\end{equation}
Do gornje enačbe smo prišli tako, da smo prešteli stoječa valovanja z velikostjo
valovnega vektorja med $k$ in $k+\Delta k$: volumen pripadajoče krogelne lupine v pozitivnem oktantu
$\frac{1}{8} 4\pi k^{2}\Delta k$ smo delili z volumnom $(\pi/L)^3$, ki pripada enemu stanju. 
Upoštevali smo tudi, da ima vsako lastno nihanje lahko dve polarizaciji.

Poglejmo primer. Izberimo osrednjo frekvenco $\nu=3\cdot10^{14}~\si{\hertz}$, ki
ustreza valovni dolžini $1~\si{\micro\metre}$, in širino frekvenčnega
intervala $\Delta\nu=3\cdot10^{9}~\si{\hertz}$, ki je tipična za Dopplerjevo
razširjeno emisijsko črto v plinu\index{Dopplerjeva razširitev}. 
Potem je v votlini s stranico $L=1~\si{\centi\metre}$ 
in prostornino $V=1~\si{\centi\metre^3}$ število lastnih nihanj na izbran interval 
po enačbi~(\ref{eq:N-stevilo-stanj}) enako
$N=2,5\cdot10^{8}$. Če bi bile vse stene votline idealna zrcala,
bi bil čas dušenja vseh lastnih nihanj približno enako dolg in vedno bi vzbujali
zelo veliko število resonanc hkrati. Tak resonator bi bil neuporaben.

Hkratnemu vzbujanju velikega števila nihanj se izognemo z zmanjšanjem odbojnosti
stranskih sten votline, saj s tem povečamo dušenje stoječih valov v prečni smeri.
Dušenje valov, ki so pravokotni na idealno odbojni končni steni, ostane nespremenjeno.
Nazadnje lahko stranske stene povsem odstranimo, tako da stoječih valov v prečni smeri
sploh ni več. Ostane le še nekaj valov, ki so pravokotni na končni
steni. Takemu resonatorju pravimo odprt resonator\index{Resonator!odprt} 
(slika~\ref{fig:Odprt_resonator}).
\begin{figure}[h]
\centering
\def\svgwidth{120truemm} 
\input{slike/04_Odprt_resonator.pdf_tex}
\caption{Levo: lastni nihajni načini odprtega resonatorja imajo 
diskretne vrednosti valovnih vektorjev $k_{n}$. Desno: žarki, ki niso pravokotni na zrcali, 
uidejo iz odprtega resonatorja.}
\label{fig:Odprt_resonator}
\end{figure}

Oglejmo si odprte resonatorje podrobneje. V zaprti pravokotni votlini velikosti
$a\times a\times L$ z idealno prevodnimi (zrcalnimi) stenami so dovoljene vrednosti
valovnega vektorja za lastna nihanja 
\begin{equation}
\mathbf{k}_{l,m,n}=\left(\frac{l\pi}{a}\,,\frac{m\pi}{a}\,,\frac{n\pi}{L}\right),\label{eq:k-votlina}
\end{equation}
 kjer so $l$, $m$ in $n$ cela števila, $L$ dolžina, $a$ pa prečna
dimenzija resonatorja. Lastne frekvence so 
\begin{equation}
\omega_{l,m,n}=c|\mathbf{k}_{l,m,n}|.\label{eq:omega-votlina}
\end{equation}
Dolžina resonatorja $L$ je velika v primerjavi z $\lambda$ in zato je $n$
zelo veliko število. Če prečnih sten ni, mora biti $\mathbf{k}$ približno
vzporeden z osjo $z$, zato morata biti $l$ in $m$ majhna. Tedaj
lahko velikost valovnega vektorja razvijemo in frekvenco zapišemo kot
\begin{equation}
\omega_{l,m,n}=c\left(\frac{n\pi}{L}+\frac{l^{2}+m^{2}}{2n}\frac{L \pi}{a^{2}}\right).
\label{eq:delta-omega-resonator-razvoj}
\end{equation}
Drugi člen v oklepaju je navadno le majhen popravek, tako da so
lastne frekvence odprtih optičnih resonatorjev odvisne predvsem od
števila vozlov v vzdolžni smeri. To število je navadno veliko, med $10^{5}$
in $10^{7}$. Možna so tudi lastna nihanja z nekaj vozli v prečni
smeri, ki pa le malo vplivajo na lastne frekvence. Zato bomo v nadaljevanju
obravnavali predvsem lastna stanja brez vozlov v prečni smeri in
jih označili z enim samim indeksom $n$.

Razlika med frekvencama dveh zaporednih lastnih nihanj\index{Lastne frekvence resonatorja} 
$n$ in $n+1$ je
\boxeq{eq:delta-omega-resonator}{
\Delta\omega_{n}=\frac{\pi c}{L}.
}
Pri dolžini resonatorja $L=30~\si{\centi\metre}$ je v frekvenčnem
intervalu s širino $3\cdot10^{9}~\si{\hertz}$ le še $6$ nihanj, ki jih
je brez težav mogoče razločiti.

V zaprtih votlinah z zrcalnimi stenami obstajajo dobro določena lastna
stanja pri poljubni obliki votline. To ne drži za odprte resonatorje.
Da se pojavijo lastna nihanja z majhnimi izgubami, morata
biti izpolnjena dva pogoja:

\begin{enumerate} 
\item Snop žarkov mora ostati po mnogih odbojih ujet med zrcaloma resonatorja.\\
\item Polmer zrcala mora biti večji od polmera uklonsko razširjenega snopa, ki se širi od nasprotnega zrcala. 
\end{enumerate}

Resonatorje, ki zadoščajo gornjima pogojema, imenujemo stabilni 
resonatorji\index{Stabilnost resonatorja}.
Drugi pogoj lahko zapišemo tudi z enačbami, izhajajoč iz enačbe za oceno divergence (enačba~\ref{eq:kot_ocena}):
\begin{equation}
\vartheta = \frac{\lambda}{a_1} < \frac{a_2}{L} \qquad \Rightarrow \qquad
\frac{a_{1}a_{2}}{\lambda L}>1,
\label{eq:Fresnelovo_stevilo}
\end{equation}
kjer sta $a_{1}$ in $a_{2}$ polmera zrcal resonatorja. Izrazu 
$
F = a_{1}a_{2}/\lambda L
$
pravimo tudi Fresnelovo število\index{Fresnelovo število} (glej enačbo~\ref{eq:Fst})
in za stabilen resonator mora veljati $F>1$. Laserji imajo pogosto $F\gg 1$, lahko tudi $F \sim 100$.

\subsection*{Fabry-Perotov interferometer}
Poglejmo resonator, omejen z dvema vzporednima ravnima zrcaloma
z veliko odbojnostjo. Takemu sistemu pravimo \index{Fabry-Perotov 
interferometer}Fabry-Perotov 
interferometer\footnote{Francoska fizika Maurice Paul Auguste Charles Fabry, 1867--1945, in 
Jean-Baptiste Alfred Perot, 1863--1925.}. 
Da nastane med zrcaloma stoječe valovanje\index{Stoječe valovanje}, mora biti razdalja 
med zrcaloma večkratnik polovične valovne dolžine (enačba~\ref{eq:k-votlina})
\begin{equation}
L = \frac{n \lambda}{2}.
\end{equation}
Prepustnost interferometra za ravni val, ki vpada pravokotno na zrcali z
odbojnostjo ${\cal R}$, je 
\begin{equation}
T=\frac{1}{1+\frac{4{\cal R}}{(1-{\cal R})^{2}}\sin^{2}kL}
\label{eq:Fabry-Perot-prepustnost}
\end{equation}
in je prikazana na sliki (\ref{fig:Fabry-Perot}).
Ko je frekvenca vpadnega valovanja ravno enaka lastni frekvenci
resonatorja, je sistem v resonanci in  $T=1$. Širina resonance je tem manjša, čim
večja je odbojnost zrcal ${\cal R}$. Ta tudi določa čas dušenja vzbujenega
lastnega nihanja.
\begin{figure}[h!]
\centering
\def\svgwidth{95truemm} 
\input{slike/04_fabry_perot.pdf_tex}
\caption{
Prepustnost Fabry-Perotovega interferometra
v odvisnosti od valovnega vektorja $k$ pri različnih odbojnostih zrcal
$\mathcal{R}.$}
\label{fig:Fabry-Perot}
\end{figure}

\begin{definition}
Pokaži, da je prepustnost Fabry-Perotovega interferometra, katerega zrcali imata odbojnost $\cal{R}$, enaka 
\begin{equation}
T = \frac{1}{1+\frac{4\cal{R}}{(1-{\cal R})^2}\sin^2 \delta},
\end{equation}
kjer je $\delta = kL\cos{\vartheta}$, $L$ razmik med zrcaloma, $\vartheta$ vpadni kot, 
$k$ pa valovni vektor svetlobe.
\end{definition}

Prvemu pogoju za stabilnost ustrezajo v Fabry-Perotovem interferometru
le žarki, ki so natanko pravokotni na zrcali. Čim sta zrcali le
malo nevzporedni, stabilnih žarkov sploh ni več. Tako imenovani 
planparalelni interferometer\index{Resonator!planparalelni} tako deluje 
na meji stabilnosti. Z izpolnjevanjem drugega
pogoja ni težav, saj morata biti pri na primer $30~\si{\centi\metre}$ dolgem resonatorju in 
valovni dolžini $0,5~\si{\micro\metre}$ zrcali večji od $0,4~\si{\milli\metre}$, da zadostita pogoju, 
zapisanem v enačbi~(\ref{eq:Fresnelovo_stevilo}).

Bolj stabilni resonatorji so sestavljeni iz dveh konkavno ukrivljenih
zrcal. Tedaj so žarki, ki potujejo pod majhnim kotom glede na osrednjo os, ujeti
med zrcaloma in energija lastnih valovanj ostaja lokalizirana blizu
osi.

\section{Gaussovi snopi v resonatorjih}
V stabilnih resonatorjih s konkavnima zrcaloma pričakujemo, da so
lastna valovanja omejena na bližino osrednje osi in zrcali znatno 
večji od polmera lastnega nihanja. Tedaj lahko za obravnavo
električnega polja uporabimo obosno valovno\index{Obosna valovna enačba} 
enačbo (enačba~\ref{eq:obosna-valovna-enacba}). Upoštevamo še zahtevo,
da svetloba po odboju od zrcala konstruktivno interferira sama s sabo.
Od tod izhaja robni pogoj, ki pravi, da se valovna fronta stoječega valovanja na 
zrcalu ujema z obliko zrcala, električno polje na površini zrcala
pa je približno enako nič.

\begin{figure}[h]
\centering
\def\svgwidth{100truemm} 
\input{slike/04_Resonator_Gauss.pdf_tex}
\caption{Gaussov snop v odprtem resonatorju s konkavno ukrivljenima zrcaloma
s krivinskima radijema $R_1$ in $R_2$. 
Krivinski radij zrcal se ujema s krivinskim radijem čela snopa. Kadar $R_1 \neq R_2$, 
grlo snopa ne leži na sredini resonatorja.}
\label{fig:Gaussov-snop-v-resonatorju}
\end{figure}

V prečni smeri omejene rešitve obosne enačbe smo našli v obliki potujočih
Gaussovih snopov\index{Gaussov snop} (enačba~\ref{eq:gaussov-snop}). 
Podobno kot lahko zapišemo stoječe valovanje na vrvi kot
vsoto valovanj v nasprotnih smereh,
lahko stoječe snope zapišemo s superpozicijo snopov, ki se širijo v različnih smereh ob osi. 
Da zadostimo robnim pogojem na zrcalih, se mora krivinski radij valovnih front snopa na zrcalih ujemati s krivinskima radijema zrcal $R_{1}$ in $R_{2}$ 
(slika \ref{fig:Gaussov-snop-v-resonatorju}).
Pri tem sta neznanki polmer grla snopa $w_0$, ki je povezan s parametrom $z_{0}$,
in lega grla. 

Postavimo izhodišče osi $z$ v grlo, kot smo
navajeni, tako da sta zrcali pri $z_{1}<0$ in $z_{2}>0$. Z uporabo enačbe
za krivinski radij snopa $R$ (enačba~\ref{eq:R}) zapišemo 
\begin{eqnarray}
-R_{1} & = & z_{1}\left[1+\left(\frac{z_{0}}{z_{1}}\right)^{2}\right] \quad  \label{eq:krivinski1} \textrm{in}\\
R_{2} & = & z_{2}\left[1+\left(\frac{z_{0}}{z_{2}}\right)^{2}\right].
\label{eq:krivinski}
\end{eqnarray}
Pri tem smo upoštevali, da je krivinski radij za konkavno zrcalo glede na 
pozitivno smer $z$ pozitiven. Prvo enačbo smo zapisali z negativnim 
predznakom pri $R_1$, zato da je krivinski radij pozitivno število.
Veljati mora še 
\begin{equation}
z_{2}-z_{1}=L.
\label{eq:razlikaz}
\end{equation}
Iz gornjih enačb najprej izračunamo razdaljo $z_{1}$, ki določa
lego grla v resonatorju, nato pa parameter $z_{0}$, ki določa
območje bližnjega polja in preko enačbe~(\ref{eq:z0}) enolično tudi polmer grla
\begin{equation}
z_{0}^{2}=\frac{L(R_{1}-L)(R_{2}-L)(R_{1}+R_{2}-L)}{(R_{1}+R_{2}-2L)^{2}}.
\label{eq:z0_stab}
\end{equation}
Da je snop realen in omejen, mora biti $z_{0}^{2}>0$ in zato števec
gornjega izraza pozitiven. Ta pogoj lahko po kratkem računu zapišemo
v obliki stabilnostnega kriterija\index{Stabilnost resonatorja}
\boxeq{eq:stabilnost}{
0<\left(1-\frac{L}{R_{1}}\right)\left(1-\frac{L}{R_{2}}\right)<1.
}
Resonatorji, ki zadoščajo gornjemu pogoju, so stabilni. 

\begin{definition}
Izhajajoč iz pogoja, da se na zrcalih krivinski radij valovnih front snopa ujema s krivinskim
radijem zrcal (enačbi~\ref{eq:krivinski1} in \ref{eq:krivinski}), izpelji 
kriterij za stabilno delovanje resonatorja 
(enačba~\ref{eq:stabilnost}).
\end{definition}

Stabilnostni kriterij za resonatorje je najbolj nazorno predstaviti na diagramu, 
kjer na os $x$ nanašamo $L/R_{1}$, na os $y$ pa $L/R_{2}$. Na 
sliki~(\ref{fig:Podrocje-stabilnih-resonatorjev}) je stabilno območje delovanja 
resonatorjev, kot ga razberemo iz enačbe~(\ref{eq:stabilnost}), označeno senčeno.
Opazimo, da je možnih veliko različnih vrst stabilnih resonatorjev, ob tem da 
resonatorji, ki so sestavljeni iz dveh konveksnih zrcal, niso nikoli stabilni.
Če je eno zrcalo ravno (plankonkavni resonatorji), je grlo Gaussovega snopa vedno 
na ravnem zrcalu. V primeru konveksno-konkavnega resonatorja leži grlo snopa
izven resonatorja. Podrobneje si oglejmo nekaj posebnih primerov stabilnih resonatorjev. 

\begin{figure}[h]
\centering
\def\svgwidth{100truemm} 
\input{slike/04_stabilnost.pdf_tex}
\caption{Področje stabilnih resonatorjev. Resonator je stabilen, 
če sta parametra $L/R_{1}$ in $L/R_{2}$ znotraj osenčenega območja: 
(a) konfokalni \index{Resonator!konfokalni}resonator,
(b) koncentrični resonator\index{Resonator!koncentrični}, 
(c) planparalelni resonator (Fabry-Perot)\index{Resonator!planparalelni}, 
(d) konveksno-konkavni resonator 
in (e) plankonkavni resonator.}
\label{fig:Podrocje-stabilnih-resonatorjev}
\end{figure}

\subsection*{Simetrični resonatorji}
Za simetrični resonator\index{Resonator!simetrični} velja $R_{1}=R_{2}=R$. Na diagramu 
(slika~\ref{fig:Podrocje-stabilnih-resonatorjev}) 
se taki resonatorji nahajajo na simetrali lihih kvadrantov pri $x=y$. Pri simetričnih resonatorjih 
je grlo v sredini resonatorja in enačba~(\ref{eq:z0_stab}) se poenostavi v 
\begin{equation}
z_{0}=\frac{1}{2}\sqrt{(2R-L)L}.
\label{eq:zosim}
\end{equation}
Polmer grla v simetričnem resonatorju je enak
\begin{equation}
w_{0}=\sqrt{\frac{\lambda z_{0}}{\pi}}=\sqrt{\frac{\lambda}{2\pi}}\sqrt[4]{(2R-L)L}.
\label{eq:grlo_v_res}
\end{equation}
Po enačbi~(\ref{eq:w}) lahko izračunamo še polmer snopa na ogledalu
\begin{equation}
w_{1}^{2}=w_{0}^{2}\left(1+\left(\frac{L}{2z_{0}}\right)^{2}\right)=
\frac{\lambda}{\pi}\frac{R\sqrt{L}}{\sqrt{2R-L}}.
\end{equation}

Pri izbrani dolžini simetričnega resonatorja je polmer snopa na zrcalu najmanjši,
kadar je $R=L$. Tedaj sovpadata geometrijski gorišči obeh zrcal,
zato imenujemo tak resonator konfokalni\index{Resonator!konfokalni}. 
Hiter račun pokaže, da velja $z_{0}=L/2$, snop od grla do zrcala pa se razširi
za $w_1/w_0=\sqrt{2}$. 
\begin{definition}
\label{naloga:uklon_konf}
 Pokaži, da je polmer snopa na izhodnem zrcalu v simetričnem resonatorju z
 danima parametroma $R$ in $L$ najmanjši, kadar je resonator konfokalni.
\end{definition}

Pri dejanskem načrtovanju laserjev velja dodatna omejitev, saj želimo 
ojačevalno sredstvo, ki je v resonatorju, čim bolj izkoristiti. Pogosto je zato 
polmer grla snopa razmeroma velik. Iz enačbe~(\ref{eq:grlo_v_res})
sledi, da mora biti v tem primeru velik tudi krivinski radij zrcal. S tem
pridobimo na ojačenju, po drugi strani pa se nekoliko poslabša stabilnost.

Poglejmo primer. Naj bo dolžina resonatorja laserja $L=1~\si{\metre}$ in valovna
dolžina $\lambda = 633~\si{\nano\metre}$. Tedaj je v konfokalni geometriji po enačbi~(\ref{eq:grlo_v_res})
polmer grla $w_{0}=0,32~\si{\milli\metre}$. Premer razelektritvene cevi je navadno
nekaj milimetrov in približno tako debel mora biti tudi svetlobni
snop, da dobro izkoristi ojačenje zaradi stimuliranega sevanja.
Da bi pri isti dolžini laserja dobili grlo s premerom $2~\si{\milli\metre}$, bi morali
uporabiti zrcali s krivinskim radijem okoli $50~\si{\metre}$. 
Primer kaže, da že majhna ukrivljenost zrcal zagotovi dokaj ozke snope.

\begin{remark}
Izkaže se, da so konfokalni resonatorji najmanj občutljivi na majhne zasuke zrcal. 
Pri zasuku zrcala se v navadnih stabilnih resonatorjih namreč premakne os, ki gre skozi 
krivinski središči obeh zrcal. Če želimo, da je največji odmik nove osi čim
manjši, uporabimo konfokalne resonatorje. 
\end{remark}

Skrajna primera stabilnega simetričnega resonatorja sta 
koncentrični resonator\index{Resonator!koncentrični},
pri katerem sovpadata krivinski središči zrcal in $L=2R$, in planparalelni 
resonator\index{Resonator!planparalelni}, pri katerem sta zrcali ravni.
V prvem primeru gre po enačbi~(\ref{eq:grlo_v_res}) polmer grla proti nič, v drugem pa raste sorazmerno
z $R^{1/4}$. Pri ravnih zrcalih postanejo uklonske
izgube na robovih znatne. Račun z Gaussovimi snopi tedaj ni več veljaven
in treba je uporabiti pristope, ki jih bomo opisali
v razdelku~(\ref{Resonator_uklon}).

Poleg osnovnega Gaussovega snopa rešijo obosno enačbo tudi snopi višjega reda. 
Imajo enak parameter $z_{0}$ in enako ukrivljenost $R$, zato so tudi 
rešitve za polje v stabilnih
resonatorjih. Pri tem je treba vedeti, da je pri enakem $w_{0}$
dejanski polmer snopa reda $n$ za približen faktor $\sqrt{n}$ večji 
(glej nalogo~\ref{naloga:HG}). Če želimo, da iz laserja izhaja samo 
osnovni Gaussov snop (imenovan tudi TEM$_{00}$ -- {\it Transverse Electromagnetic Mode}, 
transverzalno elektromagnetno valovanje)\index{TEM$_{00}$}, ki ima najmanjšo 
divergenco pri danem $w_0$, pogosto uporabimo zaslonko, ki snope višjih redov poreže, 
ali kakšen drug element, npr. Fabry-Perotov etalon\footnote{Razlika med Fabry-Perotovim
interferometrom in etalonom je v tem, da je interferometer sestavljen iz premičnih 
zrcal, etalon pa je ploščica z nespremenljivo debelino. Pri izrazu za prepustnost etalona
je treba upoštevati še lomni količnik ploščice.}, ki poveča izgube za snope višjega reda.
 
\begin{remark}
Včasih se uporabljajo tudi nestabilni resonatorji\index{Resonator!nestabilen}, za 
katere ne obstajajo rešitve v obliki Gaussovih snopov. Taki resonatorji 
imajo velike izgube na robovih zrcal, ker žarki v njih niso ujeti. 
Uporabni so v laserjih z velikim ojačenjem. Njihova prednost je, da je cel
volumen resonatorja enakomerno pokrit s svetlobo.
\end{remark}

\section{Stabilnostni kriterij z ABCD formalizmom}
V prejšnjem razdelku smo izpeljali pogoj za stabilnost resonatorja, 
sestavljenega iz dveh zrcal s polmeroma $R_1$ in $R_2$, ki sta med 
seboj oddaljeni za $L$. Izhajali smo iz pogoja, da se ukrivljenost
valovnih front na zrcalu ujema z ukrivljenostjo zrcal. Vendar so sistemi z
zgolj dvema zrcaloma razmeroma redki. Pogosto so v resonatorju
druge optične komponente, ki jih je treba upoštevati pri zapisu
kriterija za stabilnost. Takrat si pomagamo z matrikami ABCD\index{ABCD matrike}. 

Izhajamo iz zahteve, da se v stabilnem resonatorju snop po enem celotnem obhodu
preslika sam vase. Gaussov snop na neki točki v resonatorju 
zapišemo s kompleksno ukrivljenostjo $q$ (enačba~\ref{eq:q-inv}).
V najpreprostejšem primeru resonatorja snop prepotuje dano razdaljo, se odbije od zrcala, prepotuje
resonator v nasprotni smeri, se odbije od drugega zrcala in se vrne v začetno lego. V bolj 
zapletenih primerih dodamo še prehode skozi druge optične elemente. Matriko 
za celotni prehod zapišemo kot produkt matrik ABCD za posamezne prehode $M = M_N M_{N-1} ...M_2 M_1$.
Končni produkt za celoten obhod je matrika oblike
\begin{equation}
M = \left[\begin{array}{cc}
A & B\\
C & D
\end{array}\right],
\end{equation}
kompleksni krivinski radij po obhodu pa je enak začetnemu kompleksnemu radiju
\begin{equation}
q_2 = \frac{Aq+B}{Cq+D} = q.
\end{equation}
Gornjo enačbo prepišemo v 
\begin{equation}
Cq^2+(D-A)q-B=0.
\end{equation}
Da je $w$ realen, mora biti $q$ kompleksen (enačba~\ref{eq:q-inv}) 
in diskriminanta kvadratne enačbe negativna
\begin{equation}
(D-A)^2+ 4BC<0.
\end{equation}
Upoštevamo še, da je determinanta matrike $AD-BC=1$ in pogoj za 
stabilnost\index{Stabilnost resonatorja} zapišemo s koeficienti matrike $M$
\boxeq{eq:stabilnost_ABCD}{
-1 < \left(\frac{A+D}{2}\right) < 1.
}

\begin{definition}
Pokaži, da je za resonator z dolžino $L$ in s konkavnima zrcaloma s krivinskima radijema $R_1$ in $R_2$ 
pogoj za stabilnost (enačba~\ref{eq:stabilnost_ABCD}) ekvivalenten pogoju~(\ref{eq:stabilnost}).
\end{definition}

\begin{remark}
Podoben, a malo bolj zapleten račun lahko naredimo tudi za resonatorje, v katerih se snop 
ne vrne v svojo začetno obliko po enem obhodu, ampak je teh obhodov $n$. V tem primeru zapišemo
matriko $M$ kot produkt elementov v enem obhodu, celotna matrika pa je enaka $M^n$. Iz pogoja,
da matrika ne divergira za velike $n$, izpeljemo pogoj za stabilnost, ki je natančno 
enak pogoju~(\ref{eq:stabilnost_ABCD}). 
\end{remark}

\section{Resonančne frekvence}
Doslej smo obravnavali le prostorsko obliko polja v resonatorju, ničesar pa še nismo
povedali o časovni odvisnosti lastnih nihanj. Frekvence
lastnih nihanj\index{Lastne frekvence resonatorja} izpeljemo iz pogoja, 
da se mora faza snopa pri enem obhodu\footnote{Prelet resonatorja je prehod svetlobe od enega zrcala
do drugega, obhod pa imenujemo prelet v obe smeri.}
spremeniti za mnogokratnik
$2\pi$. Fazo za osnovni snop zapišemo po enačbi~(\ref{eq:gaussov-snop})\index{Gouyeva faza}
\begin{equation}
\phi = kz+\frac{kr^{2}}{2R} -\eta(z) = kz-\arctan \left(\frac{z}{z_{0}}\right),
\label{eq:fazag}
\end{equation}
pri čemer  smo se omejili na valovanje na osi, pri $r=0$. 
Razlika faze po obhodu je 
\begin{equation}
2\frac{\omega_{n}}{c}L-2\left(\arctan \left(\frac{z_{2}}{z_{0}}\right)-
\arctan\left(\frac{z_{1}}{z_{0}}\right)\right)=2n\pi.
\label{eq:fazan}
\end{equation}
Člen v oklepaju je enak za vsa nihanja in le za delček valovne dolžine 
spremeni efektivno dolžino resonatorja. Ker dolžine resonatorja niti ne poznamo
tako natančno, lahko ta prispevek zanemarimo. Iz istega razloga lahko izpustimo
tudi spremembo faze na zrcalu. Od tod sledi znana enačba za resonančne frekvence 
\begin{equation}
\omega_{n}=\frac{n\pi c}{L}.
\label{eq:omega}
\end{equation}
Konstantni členi, ki jih pri zapisu frekvence nismo upoštevali, ne vplivajo
na razmik med dvema zaporednima lastnima frekvencama, ki je v skladu
enačbo~(\ref{eq:delta-omega-resonator})
\begin{equation}
\Delta\omega=\frac{\pi c}{L}.
\label{eq:deltaomega}
\end{equation}

Za snope višjega reda, ki imajo vozle v prečni smeri, je fazni premik
odvisen tudi od števila vozlov. Tako je na primer v cilindričnih koordinatah
(enačba~\ref{eq:etaGL})
\begin{equation}
\eta_{p,l}(z)=(2p+l+1)\arctan\left(\frac{z}{z_{0}}\right).
\end{equation}
Resonančni pogoj zapišemo kot
\begin{equation}
2\frac{\omega_{n,p,l}}{c}L-2(2p+l+1)\left(\arctan\left(\frac{z_{2}}{z_{0}}\right)-
\arctan\left(\frac{z_{1}}{z_{0}}\right)\right)=2n\pi.
\end{equation}
Resonančne frekvence so torej odvisne tudi od števila prečnih vozlov (slika~\ref{fig:crte}),
kar je dodaten razlog, da v laserjih vzbujamo le osnovno lastno nihanje.

Zanimiv in praktično pomemben je primer konfokalnega resonatorja\index{Resonator!konfokalni},
pri katerem je $z_{0}=L/2$ in $\arctan(L/2z_{0})= \arctan(1)=\pi/4$. Resonančne frekvence
so 
\begin{equation}
\omega_{n,p,l}=\frac{\pi c}{L}\left(n+\frac{1}{2}(2p+l+1)\right).
\label{eq:omega_konf}
\end{equation}

Snopi, pri katerih je $2p+l$ liho število, imajo iste resonančne frekvence kot
osnovni snopi, pri sodih $2p+l$ pa se pojavijo še resonance na sredini
med osnovnimi. Razmik med so\-sed\-nji\-mi frekvencami je tako $\Delta\nu=c/4L$
in konfokalni interferometer se obnaša kot dvakrat daljši planparalelni.
To lahko razumemo tudi iz geometrijske slike: žarek, ki vstopi v konfokalni
interferometer vzporedno z osjo, se šele po dveh preletih vrne sam
vase.

\begin{figure}[h]
\centering
\def\svgwidth{55truemm} 
\input{slike/04_Konfokalni.pdf_tex}
\caption{V konfokalnem resonatorju se žarek šele po dveh preletih
vrne sam vase.}
\label{fig:Konfokalni_zarek}
\end{figure}

Pri skoraj planparalelnem\index{Resonator!planparalelni} 
resonatorju je $z_{0}\gg L$, $\arctan(L/2z_{0})$ lahko razvijemo, 
upoštevamo enačbo~(\ref{eq:zosim}) in zapišemo
\begin{equation}
\omega_{n,p,l}=\frac{\pi c}{L}\left(n+(2p+l+1)\frac{1}{\pi}\sqrt{\frac{2L}{R}}\right).
\end{equation}
Ker je $L$ majhen v primerjavi z $R$, so resonance snopov nizkega prečnega reda 
zelo blizu resonancam osnovnih snopov. Poglejmo primer. Vzemimo 
simetrični resonator z dolžino $L=1~\si{\metre}$ in $R=50~\si{\metre}$, 
valovna dolžina pa naj bo $\lambda= 500~\si{\nano\metre}$. Z uporabo enačbe~(\ref{eq:zosim})
hitro lahko izračunamo $z_0 = 4,97~\si{\metre}$. Ker je pogoj $z_0\gg L$ izpolnjen, lahko uporabimo
gornji približek. Razlika med frekvencama dveh osnovnih snopov je $\Delta \omega_{n,p,l} = 
\pi c/L = 940~\si{\mega\hertz}$, medtem ko je razlika med frekvencama dveh prečnih snopov
$\Delta \omega_{n,p,l} = c\sqrt{2/RL} = 60~\si{\mega\hertz}$.
\begin{figure}[h]
\centering
\def\svgwidth{90truemm} 
\input{slike/04_crte.pdf_tex}
\caption{Resonančne frekvence za skoraj planparalelni ($R\gg L$) resonator}
\label{fig:crte}
\end{figure}

\section{Izgube v resonatorjih}
Energija lastnega nihanja odprtega resonatorja ni konstantna, ampak se počasi
zmanjšuje. Razlogov je več:
\begin{enumerate}
\item Odbojnost zrcal ni enaka 1. Tudi če bi znali narediti popolno odbojna zrcala, 
tak resonator ne bi bil uporaben, saj nihanja ne bi mogli sklopiti z zunanjim poljem. Če 
naj torej laser oddaja svetlobo, mora biti odbojnost vsaj enega od zrcal manjša od 1.\\
\item Na sredstvu in na optičnih elementih v resonatorju pride do absorpcije in
sipanja svetlobe. Te izgube želimo navadno kar se da zmanjšati.\\
\item Uklonske izgube so odvisne od velikosti zrcal in velikosti snopa na njih.
V dani geometriji imajo najmanjši polmer osnovni snopi, snopi višjega
reda so širši, zato imajo večje uklonske izgube. Merilo za uklonske
izgube je Fresnelovo število\index{Fresnelovo število} (enačba~\ref{eq:Fresnelovo_stevilo}) 
$N_{F}=a^{2}/(L\lambda)$, kjer je $a$ polmer ogledal. Pri enakem $N_{F}$ je
polmer snopa na izhodnem zrcalu najmanjši, če je resonator konfokalni\index{Resonator!konfokalni}, 
zato ima tak resonator tudi najmanjše uklonske izgube (glej nalogo~\ref{naloga:uklon_konf}).
Če je $N_{F}$ znatno večji od 1, kar navadno je, so uklonske izgube zanemarljive. 
\end{enumerate}

Vse izgube lahko popišemo z razpadnim časom za energijo 
\begin{equation}
\frac{dW}{dt}=-\frac{2}{\tau}W,
\label{eq:dW}
\end{equation}
in 
\begin{equation}
W = W_0 e^{-2t/\tau},
\label{eq:dW1}
\end{equation}
pri čemer $\tau$ imenujemo življenjski ali razpadni čas nihanj\index{Življenjski čas nihanj}.
Izguba energije na celoten obhod resonatorja\index{Izgube v resonatorju} je
sestavljena iz izgub na zrcalih in drugih (predvsem notranjih) izgub
\begin{equation}
\Delta W = -(1-{\cal R}_{1})W-(1-{\cal R}_{2})W -\Lambda_{0} W = -\Lambda W.
\label{eq:Lambda}
\end{equation}
Z ${\cal R}_{1}$ in ${\cal R}_{2}$ smo označili odbojnosti zrcal, od katerih je navadno ena
kolikor mogoče blizu 1. Parameter $\Lambda_{0}$ popiše absorpcijo in
sipanje znotraj resonatorja ter morebitne uklonske izgube. Tipične vrednosti 
$\Lambda_0$ so do nekaj stotink. Celotne izgube združimo v parameter $\Lambda$. 

Razpadni čas nihanja lahko neposredno povežemo z izgubami, če v enačbo~(\ref{eq:dW})
vstavimo čas enega obhoda $t=2L/c$. Zapišemo
\begin{equation}
\frac{\Delta W}{W}= -\Lambda = -\frac{2}{\tau}\, \frac{2L}{c},
\end{equation}
od koder sledi
\begin{equation}
\frac{1}{\tau}=\frac{\Lambda c}{4L}=\frac{1}{\tau_{0}}+\frac{c}{4L}\left((1-{\cal R}_{1})
+(1-{\cal R}_{2})\right),
\label{taulambda}
\end{equation}
kjer smo s 
\begin{equation}
\tau_{0}=\frac{4L}{\Lambda_0 c}
\label{eq:tau0n}
\end{equation}
označili razpadni čas zaradi notranjih izgub\index{Izgube v resonatorju!notranje}. 

Notranje izgube so navadno zelo majhne, odbojnost enega zrcala
pa je približno enaka 1, tako da je življenjski čas nihanj približno
\boxeq{eq:taucca}{
\frac{1}{\tau}=\frac{c}{4L} (1-{\cal R}).
}

Zaradi dušenja se energija lastnih nihanj eksponentno zmanjšuje, prav tako tudi
amplituda jakosti električnega polja, ki pojema s karakterističnim časom $\tau$. 
Spekter eksponentno pojemajoče funkcije lahko izračunamo
(glej nalogo~\ref{naloga-spekter}) in dobimo Lorentzovo krivuljo\index{Spekter!Lorentzov} 
s širino črte, ki ustreza ravno
\begin{equation}
\Delta\omega_{1/2}=\frac{1}{\tau}.
\label{3.26}
\end{equation}
Lastne frekvence torej niso neskončno ozke, ampak imajo končno širino $2/\tau$.
Poglejmo primer. Naj bodo notranje izgube na obhod $\Lambda_0=0,01$,
eno zrcalo naj bo idealno, drugo naj ima odbojnost ${\cal R}=0,93$. Dolžina
resonatorja naj bo $L=0,5~\si{\metre}$, valovna dolžina pa $500~\si{\nano\metre}$. Tedaj
je $1/\tau=12\cdot10^{6}$~s$^{-1}$. Zanimivo je pogledati razmerje med 
razliko resonančnih frekvenc $\Delta \omega$ 
(enačba~\ref{eq:delta-omega-resonator}) in širino resonance $2/\tau$. 
Za opisani primer velja $\Delta\omega\tau/2 \approx 80$.

\begin{remark}
Namesto razpadnega časa $\tau$ se za opis izgub pogosto uporablja
dobrota resonatorja\index{Dobrota resonatorja}, ki jo vpeljemo kot
razmerje med resonančno frekvenco in širino črte 
\begin{equation}
Q=\frac{\omega_{n}}{2\Delta\omega_{1/2}} = \frac{\omega_{n}\tau}{2}.
\end{equation}
Za tipične optične resonatorje je resonančna 
frekvenca $\omega_n \sim 10^{15}~\si{\hertz}$, širina pa reda 
 $1/\tau \sim 10^{7}~\si{\hertz}$. Faktor dobrote je tako $Q \sim 10^{8}$. Optični 
 resonatorji imajo zelo velike faktorje dobrote!
\end{remark}

Poglejmo še primer, ko so notranje izgube zanemarljive in sta odbojnosti obeh zrcal enaki.
Potem sledi iz enačbe~(\ref{taulambda})
\boxeq{eq:taucca2}{
\frac{1}{\tau}=\frac{c}{2L} (1-{\cal R}).
}

Do istega rezultata pridemo tudi z razvojem izraza za prepustnost Fabry-Perotovega 
interferometra\index{Fabry-Perotov interferometer} (enačba~\ref{eq:Fabry-Perot-prepustnost})
okoli vrha pri $\omega_{n}$
\begin{equation}
T=\frac{1}{1+\frac{4{\cal R}}{(1-{\cal R})^{2}}\sin^{2}L\frac{(\omega-\omega_{n})}{c}}\approx 
\frac{1}{1+\left(\frac{2}{(1-{\cal R})}\frac{L}{c}(\omega-\omega_{n})\right)^{2}},
\label{3.27}
\end{equation}
 kjer smo upoštevali še, da je odbojnost ${\cal R} \approx 1$. Rezultat je znana Lorentzova
 krivulja oblike
 \begin{equation}
 T = \frac{(\Delta\omega_{1/2})^2}{(\omega - \omega_n)^2+(\Delta\omega_{1/2})^2},
 \label{eq:FBi2}
 \end{equation}
od koder hitro razberemo 
\begin{equation}
\frac{1}{\tau}=\Delta\omega_{1/2} = \frac{c}{2L}(1-{\cal R}).
\end{equation}

\section{*Obravnava z uklonskim integralom}
\label{Resonator_uklon}

V nestabilnih resonatorjih stacionarna rešitev v obliki stoječega
Gaussovega snopa ne obstaja. Zato je v takih resonatorjih rešitev za električno polje precej
zahtevno poiskati. Ker pa je problem soroden uklonu, ga lahko obravnavamo z 
uklonsko teorijo\index{Uklon}.

Označimo jakost električnega polja v točki $P_{1}$ prvega zrcala z $E_{1}$.
Polje na drugem zrcalu v točki $P_2$ lahko zapišemo s Kirchhoffovim uklonskim
integralom\index{Kirchhoffov integral} (enačba~\ref{eq:Fresnelov-uklon})
\begin{eqnarray}
E_{2} & = & -\frac{i}{2\lambda}\int_{1}E_{1}(P_{1})\frac{e^{ikr}(1+\cos\vartheta)}{r}\, dS_{1} \\
 & = & \int_{1}E_{1}(P_{1})K(P_{1},P_{2})dS_{1},
\label{eq:resuklon}
\end{eqnarray}
kjer je $r$ razdalja med točkama $P_{1}$ in $P_{2}$, $\vartheta$
je kot med zveznico in normalo na zrcali v osi, druge faktorje pa smo pospravili v faktor
\begin{equation}
K(P_{1},P_{2}) = -\frac{i}{2\lambda}\frac{e^{ikr}(1+\cos\vartheta)}{r},
\label{jedro}
\end{equation}
ki predstavlja jedro integralske enačbe. Polje na prvem zrcalu mora
biti na enak način povezano s poljem na drugem. Če naj zapisano polje predstavlja lastno nihanje
resonatorja, mora biti po dveh odbojih sorazmerno začetnemu polju
\begin{equation}
E_{1}(P)=\gamma\int_{1}\int_{2}E_{1}(P_{1})K(P_{1},P_{2})K(P_{2},P_1)\, dS_{1}\, dS_{2}.
\label{3.71}
\end{equation}
Enačba~(\ref{3.71}) je homogena integralska enačba, katere lastne
rešitve so iskana lastna nihanja elektromagnetnega polja v resonatorju.
Kompleksne lastne vrednosti $\gamma$ določajo frekvenco in dušenje
nihanj. V splošnem rešitev ni mogoče poiskati analitično in treba je uporabiti 
numerične metode. Najenostavnejša je iterativna
metoda, pri kateri začnemo z nekim začetnim poljem in ponavljamo integracijo
v enačbi~(\ref{3.71}), dokler se polje ne spreminja več.

Integralsko enačbo (enačba~\ref{3.71}) je mogoče rešiti v posebnem primeru
konfokalnega resonatorja\index{Resonator!konfokalni}. Privzemimo, da je brez izgub. 
Ker je resonator simetričen, se polje na obeh zrcalih lahko razlikuje kvečjemu
za predznak.
\begin{figure}[h]
\centering
\def\svgwidth{110truemm} 
\input{slike/04_Uklon.pdf_tex}
\caption{Koordinatna sistema na zrcalih resonatorja}
\label{fig:uklon_res_shema}
\end{figure}
Vpeljemo kartezične koordinate na obeh zrcalih, kot kaže slika~(\ref{fig:uklon_res_shema}).
Pričakujemo, da bo prečna razsežnost lastnega stanja majhna v primerjavi
z dolžino resonatorja $L$, zato lahko $r$ razvijemo, pri čemer se kvadratni členi ravno odštejejo 
zaradi ukrivljenosti zrcal. Ostaneta le še mešana člena
\begin{equation}
r\approx L-\frac{xx^{\prime}+yy^{\prime}}{L}.
\label{3.72}
\end{equation}
Ker obravnavamo konfokalni resonator, je krivinski radij zrcal kar enak dolžini resonatorja.
V imenovalcu jedra integrala (enačba~\ref{jedro}) zato lahko $r$ nadomestimo
z $L$. Koti med zveznico točk na obeh zrcalih in normalo na zrcali
so majhni, zato je  $\cos\vartheta \approx 1$. 

Tako iz enačbe~(\ref{eq:resuklon})
sledi
\begin{equation}
E(x',y')=\pm\frac{ie^{ikL}}{\lambda L}\int E(x,y)\exp
\left(\frac{-ik(xx^{\prime}+yy^{\prime})}{L}\right)\, dx\, dy.
\label{3.73}
\end{equation}
Integracija poteka po celem zrcalu. Jedro integrala je produkt dveh
faktorjev, ki vsebujeta vsak le $x'$ ali $y'$ koordinati. Zato poiščemo
rešitev enačbe~(\ref{3.73}) v obliki produkta 
$
E(x',y')=E_{0}f(x')g(y').
$
S tem nastavkom morata biti funkciji $f(x')$ in $g(y')$ rešitvi enačbe
\begin{equation}
\alpha f(x')=\int f(x)\exp\left(\frac{-ikxx^{\prime}}{L}\right)\, dx,
\label{3.75}
\end{equation}
kjer je $\alpha$ še neznana konstanta. Meje integrala so od enega do 
drugega roba zrcala. Če je zrcalo dovolj veliko,
pričakujemo, da je polje na robu dovolj majhno, da lahko meje vzamemo
kar od $-\infty$ do $\infty$. Vpeljemo še brezdimenzijski koordinati
\begin{equation}
X'=x'\sqrt{k/L} \quad \mathrm{in} \quad Y'=y'\sqrt{k/L}
\label{3.76}
\end{equation}
in dobimo
\begin{equation}
\alpha f(X')=\sqrt{\frac{L}{k}}\int_{-\infty}^{\infty}f(X)e^{-iXX^{\prime}}\, dX
\label{3.77}
\end{equation}
ter podobno enačbo za $g(Y')$. Enačba~(\ref{3.77}) pravi, da mora
biti $f(X)$ podobna svoji Fourierevi transformiranki. Najpreprostejša
funkcija s to lastnostjo je Gaussova funkcija 
\begin{equation}
f(X)=\exp\left(-\frac{1}{2}X^{2}\right).
\label{3.78}
\end{equation}
Polje na zrcalu ima tako po pričakovanju obliko Gaussovega snopa
\begin{equation}
E(x,y)=E_{0}\exp\left(-\frac{k(x^{2}+y^{2})}{2L}\right).
\label{3.79}
\end{equation}

Poglejmo še konstanto $\alpha$. Imeti mora imeti vrednost 
$\alpha = \sqrt{2\pi L/k}=\sqrt{\lambda L}$.
Enaki izrazi veljajo tudi za smer $y$. Postavimo zdaj izračunano električno 
poljsko jakost (enačba~\ref{3.79}) z ustrezno vrednostjo za $\alpha$ v 
enačbo~(\ref{3.73}) in upoštevajmo, da mora biti
$E_{2}=\pm E_{1}$. Sledi
\begin{equation}
i\alpha^{2}\frac{e^{ikL}}{\lambda L}=ie^{ikL}=\pm1.
\label{3.80}
\end{equation}
Izpeljemo resonančni pogoj za
frekvenco lastnega nihanja in izračunamo razliko med dvema lastnima frekvencama, ki jo že poznamo~(enačba~\ref{eq:deltaomega})
\begin{equation}
\Delta \omega=\frac{\pi c}{L}.
\label{3.81}
\end{equation}
Integralska enačba, dobljena iz uklonske teorije, tako da
isti rezultat kot stoječe valovanje oblike Gaussovih snopov, ki so
rešitve obosne valovne enačbe. To nas seveda ne preseneča, saj je
obosna valovna enačba enako natančna kot Fresnelova uklonska teorija.

\begin{definition}
Pokaži, da so funkcije, ki so sorazmerne svoji Fourierevi transformaciji, 
Hermite-Gaussove funkcije\index{Hermite-Gaussovi snopi} (enačba~\ref{eq:Gauss-Hermitevi}), in 
izračunaj lastne frekvence stanj višjega reda.
\end{definition}

\begin{remark}
V tem poglavju smo obravnavali samo dva primera laserskih resonatorjev: Fabry-Perotov
resonator z dvema vzporednima ravnima zrcaloma (a) in resonator z dvema sferičnima zrcaloma (b).
Poleg teh dveh primerov je v uporabi še cela vrsta različnih resonatorjev. Ciklični
resonator\index{Resonator!ciklični} (c) je
sestavljen iz več zrcal in žarek ciklično potuje med njimi. V vlakenskih laserjih je resonator
optično vlakno med dvema zrcaloma ali pa sklenjeno vlakno\index{Laser!vlakenski}. 
Kot laserski resonator lahko delujejo tudi mikroploščice (d) ali mikrokroglice (e), 
v katere je s totalnim odbojem\index{Totalni odboj} ujeta svetloba. 
Namesto zrcal v mikroresonatorjih uporabimo tudi periodične dielektrične strukture, na katerih 
pride do Braggovega odboja (f).\index{Braggov odboj}
\end{remark}

\begin{figure}[h]
\centering
\def\svgwidth{140truemm} 
\input{slike/04_resonatorji.pdf_tex}
\label{fig:resonatorji}
\end{figure}

\section{*Sklopitev resonatorja z okolico}
\index{Sklopitev resonatorja!z okolico}
Na začetku poglavja smo omenili, da resonatorjev ne uporabljamo samo pri 
izdelavi laserjev, ampak lahko služijo tudi kot frekvenčni in
prostorski filtri za svetlobno valovanje. Povezavo med lastnimi nihanji
resonatorja in prepustnostjo ter odbojnostjo za valovanje, ki na resonator
vpada, bomo poiskali s formalizmom sklapljanja valovanj, 
ki je neke vrste perturbacijska analiza in je pogosto zelo uporaben.

Začnimo z resonatorjem z idealno odbojnimi stenami brez notranjih izgub. Stoječe
lastno valovanje\index{Stoječe valovanje} v resonatorju zapišemo kot produkt krajevnega in časovnega
dela
\begin{equation}
E(\mathbf{r},t)=f(t)g(\mathbf{r}).
\label{3.31}
\end{equation}
Krajevni del $g(\mathbf{r})$ naj bo normaliziran tako, da je $\int g^{2}dV=1$. Iz valovne
enačbe (enačba~\ref{eq:valovna-skalarna}) sledi, da mora časovni del zadoščati 
nihajni enačbi drugega reda
\begin{equation}
\frac{\partial^2 f}{\partial t^2}+\omega_{n}^{2}f= \ddot{f} + \omega_{n}^{2}f=0.
\label{3.32}
\end{equation}
Vpeljemo novo kompleksno spremenljivko $a$, ki je kombinacija funkcije $f$ in njenega
časovnega odvoda
\begin{equation}
a=\sqrt{\frac{\epsilon_{0}}{2}}(f+\frac{i}{\omega_{n}}\dot{f}).
\label{3.33}
\end{equation}
Izbira predfaktorja bo razvidna v nadaljevanju. 
Z odvajanjem in uporabo nihajne enačbe (enačba~\ref{3.32}) ugotovimo, da za $a$ velja 
diferencialna enačba 
\begin{equation}
\dot{a}=-i\omega_{n}a.
\label{3.34}
\end{equation}
Funkcija $a$ ima preprosto odvisnost od časa $e^{-i\omega_{n}t}$. 

Poglejmo elektromagnetno energijo v resonatorju. Električni
del energije polja, ki je ravno polovica celotne energije, zapišemo kot
\begin{equation}
W_e = \frac{1}{2}\varepsilon_0 \int E^2 dV = \frac{1}{2}\epsilon_{0}f^{2}
\int g^{2}dV = \frac{1}{4} (a+a^{*})^2.
\end{equation}
V spremenljivki $a$ in konjugirani spremenljivki $a^{*}$ lahko prepoznamo 
klasično obliko anihilacijskih in kreacijskih operatorjev v kvantno-mehanskem 
opisu harmonskega oscilatorja. 

Celotna energija lastnega nihanja resonatorja je po analogiji enaka
\begin{equation}
W=|a|^2.
\label{3.35}
\end{equation}
Vpeljana nova spremenljivka $a(t)$ je sorazmerna z $e^{-i\omega_{n}t}$, 
zato ji pravimo tudi komponenta amplitude s pozitivno frekvenco. 
Prednost spremenljivke $a$ je v preprostejših enačbah, ki so le prvega reda. 

Do zdaj smo obravnavali resonator brez izgub. \index{Izgube v resonatorju}
Poglejmo si še primer z izgubami. 
Celotne izgube resonatorja lahko opišemo z dodatnim členom v enačbi~(\ref{3.34})
\begin{equation}
\dot{a}=-i\omega_{n}a-\frac{1}{\tau}a.
\label{3.36}
\end{equation}
V taki obliki lahko zapišemo enačbo le v primeru, če so izgube majhne. Če niso, 
je treba uporabiti navadno nihajno enačbo drugega reda. Gornji približek
namreč ne vsebuje zmanjšanja nihajne frekvence pri velikem dušenju.
Prehod na dve nesklopljeni enačbi prvega reda za $a$ in $a^*$
je točen le, kadar ni izgub. Izgube sklopijo enačbi za $a$ in $a^{\ast}$, 
vendar smo v našem približku to sklopitev zanemarili.

Naj bo odbojnost enega zrcala resonatorja nekoliko manjša od 1. Izgube v resonatorju 
tako znašajo (enačba~\ref{eq:taucca}) $1/\tau=c/(4L)(1-{\cal R})$. Druga posledica
zrcala z zmanjšano odbojnostjo pa je sklopitev resonatorja z okolico. To pomeni, 
da valovanje izhaja iz resonatorja, po drugi strani pa to pomeni tudi, da je 
lastno nihanje mogoče vzbujati z valovanjem, ki na resonator vpada.

Naj $s_{+}$ opiše snop valovanja, ki vpada na resonator. Amplituda $s_{+}$
naj bo izbrana tako, da je $|s_{+}|^{2}$ enako moči vpadnega valovanja. Zaenkrat
tudi zanemarimo notranje izgube resonatorja. Potem lahko
zapišemo 
\begin{equation}
\dot{a}=-i\omega_{n}a-\frac{1}{\tau}a+\kappa s_{+},
\label{3.37}
\end{equation}
kjer je $\kappa$ sklopitveni koeficient med vpadnim valovanjem in
amplitudo lastnega nihanja. Koeficient $\kappa$ je določen
s prepustnostjo zrcala, ki pa je vsebovana tudi v $1/\tau$. Koeficient
$\kappa$ torej ni neodvisen in poiščimo zvezo med $\kappa$ in $1/\tau$.

Naj ima vpadno valovanje frekvenco $\omega$. Tedaj lahko iz enačbe~(\ref{3.37}) 
izračunamo amplitudo nihanja v stacionarnem stanju. Upoštevamo, 
da mora imeti v stacionarnem stanju nihanje enako frekvenco
kot vpadni val. Sledi 
\begin{equation}
a=\frac{\kappa s_{+}}{i(\omega_{n}-\omega)+1/\tau}.
\label{3.38}
\end{equation}
 
Označimo del valovanja, ki se od resonatorja odbije ali iz njega izvira, s $s_{-}$.
Če vpadnega vala ni, energija nihanja pojema zaradi odtekanja v $s_{-}$.
Ohranitev energije da 
\begin{equation}
-\frac{dW}{dt}=-\frac{d}{dt}|a|^{2}=\frac{2}{\tau}|a|^{2}=|s_{-}|^{2}
\label{3.39}
\end{equation}
ali 
\begin{equation}
|s_{-}|=\sqrt{\frac{2}{\tau}}|a|,
\label{3.40}
\end{equation}
kjer smo fazo $s_{-}$ priredili z izbiro referenčne ravnine, v kateri
opazujemo $s_{-}$.

Ob prisotnosti vpadnega vala $s_{+}$ lahko izhajajoči val zapišemo kot
vsoto direktnega odboja vpadnega vala $s_{+}$ in prispevka iz resonatorja
\begin{equation}
s_{-}=rs_{+}+\sqrt{\frac{2}{\tau}}a,
\label{3.41}
\end{equation}
kjer $r$ zaenkrat še ne poznamo. Ker ni notranjih izgub, mora bit v 
stacionarnem stanju vpadna moč enaka izhajajoči
\begin{equation}
|s_{+}|^{2}=|s_{-}|^{2}.
\label{3.42}
\end{equation}
Uporabimo še izraz za stacionarno vrednost $a$  (enačba~\ref{3.38}) in zapišemo
enakost 
\begin{equation}
r^{2}+\frac{2(\tau\kappa^{2}+r\kappa\sqrt{2\tau})}{1+\tau^{2}(\omega_{n}-\omega)^{2}}=1.
\label{3.43}
\end{equation}
Gornja enačba mora veljati pri vsaki frekvenci $\omega$, to je pri vsaki vrednosti imenovalca
ulomka. Zato mora biti $r^{2}=1$ in $\tau\kappa=-r\sqrt{2\tau}$.
Ker sta $\tau$ in $\kappa$ pozitivna, je $r=-1$ in 
\begin{equation}
\kappa=\sqrt{\frac{2}{\tau}}.
\label{3.44}
\end{equation}
 Odbito valovanje lahko torej zapišemo 
\begin{equation}
s_{-}=-s_{+}+\sqrt{\frac{2}{\tau}}a.
\label{3.45}
\end{equation}

Z upoštevanjem notranjih izgub (enačba~\ref{eq:tau0n}) se amplituda nihanja spremeni v 
\begin{equation}
\dot{a}=-i\omega_{n}a-\left(\frac{1}{\tau_{0}}+\frac{1}{\tau}\right)a+
\sqrt{\frac{2}{\tau}}s_{+}.
\label{3.46}
\end{equation}
 Enačbi (\ref{3.45}) in (\ref{3.46}) sta osnovna izraza za sklapljanje
resonatorjev z enim vhodom. Za primer uporabe izračunajmo odbojnost
resonatorja $s_{-}/s_{+}$ kot funkcijo frekvence vpadnega vala. V
enačbo~(\ref{3.45}) vstavimo izraz za stacionarno vrednost amplitude
nihanja 
\begin{equation}
a=\frac{\sqrt{\frac{2}{\tau}}s_{+}}{i(\omega_{n}-\omega)+(\frac{1}{\tau_{0}}+
\frac{1}{\tau})}.
\label{3.47}
\end{equation}
Sledi
\begin{equation}
\frac{s_{-}}{s_{+}}=\frac{\frac{1}{\tau}-\frac{1}{\tau_0}-i(\omega_{n}-\omega)}
{\frac{1}{\tau}
+\frac{1}{\tau_0}+i(\omega_{n}-\omega)}.
\label{3.48}
\end{equation}
Daleč od resonance je odbojnost -1. V resonanci (pri $\omega_{n}=\omega$)
odboja ni, kadar je $\tau=\tau_{0}$. Takrat je moč, ki gre iz
vpadnega valovanja v vzbujanje resonatorja, največja in je sklopitev,
ki jo meri $\tau$, popolnoma prilagojena izgubam. Taka prilagoditev
je analogna zahtevi, da mora biti impedanca bremena na koncu valovoda
ali koaksialnega kabla enaka impedanci valovoda oziroma kabla.

Če sta obe zrcali resonatorja delno prepustni, kot na primer pri 
Fabry-Perotovem interferometru, je enačba za amplitudo nihanja 
\begin{equation}
\dot{a}=-i\omega_{n}a-\left(\frac{1}{\tau_{0}}+\frac{1}{\tau_{1}}+\frac{1}{\tau_{2}}\right)
a+\kappa_{1}s_{+1}+\kappa_{2}s_{+2},
\label{3.49}
\end{equation}
 kjer sta $s_{+1}$ in $s_{+2}$ valovanji, ki vpadata z ene in druge strani.
Izgube zaradi končne prepustnosti so
\begin{equation}
\frac{1}{\tau_{1,2}}=\frac{c}{4L}(1-{\cal R}_{1,2}).\label{3.50}
\end{equation}
S podobnim razmislekom kot prej, s tem da postavimo najprej eno, nato drugo
vpadno valovanje na nič, dobimo 
\begin{equation}
\kappa_{1,2}=\sqrt{\frac{2}{\tau_{1,2}}}.
\label{3.51}
\end{equation}
Prepustnost resonatorja oziroma razmerje med močjo vpadnega valovanja
z ene strani in izhodnega z druge, je
\begin{equation}
T=\frac{|s_{-2}|^{2}}{|s_{+1}|^{2}}=\frac{2}{\tau_{2}}\frac{|a|^{2}}{|s_{+1}|^{2}}=\frac{4\tau_c^{2}/
\tau_{1}\tau_{2}}{1+\tau_c^{2}(\omega_{n}-\omega)^{2}},
\label{3.52}
\end{equation}
 kjer je $1/\tau_c=1/\tau_{0}+1/\tau_{1}+1/\tau_{2}$. 
 Če ni notranjih izgub, je prepustnost v resonanci 
\begin{equation}
T=\frac{4/\tau_{1}\tau_{2}}{(1/\tau_{1}+1/\tau_{2})^{2}}.
\label{3.53}
\end{equation}
Prepustnost je v resonanci popolna, če sta obe zrcali enaki in $\tau_{1}=\tau_{2}$.
Gornja izraza se ujemata z znanim izrazom 
za prepustnost Fabry-Perotovega interferometra v bližini resonanc,
če so izgube in prepustnost zrcal majhne (enačbi~\ref{eq:Fabry-Perot-prepustnost} 
in \ref{eq:FBi2}). \index{Fabry-Perotov interferometer}

Resonatorji imajo mnogo lastnih nihanj. Očitno veljajo gornji izrazi
za vsako lastno nihanje posebej in celoten odziv resonatorja na poljubno vpadno
valovanje zapišemo kot vsoto po vseh lastnih nihanjih. Pri tem ne smemo pozabiti,
da mora vpadno valovanje, ki se sklaplja z izbranim lastnim nihanjem,
imeti prostorsko odvisnost, ki ustreza lastnemu stanju. V primeru
stabilnih resonatorjev iz prejšnjih razdelkov mora torej biti vpadni
snop Gaussov z enakim $w_{0}$ in istega prečnega reda kot resonatorsko
stanje. Če vpadno valovanje ni tako, ga moramo najprej razviti po
Gaussovih snopih, ki ustrezajo resonatorju.
Pri zahtevnejših interferometričnih meritvah je treba za vzbujanje le ene 
resonance vpadni snop prilagoditi resonatorju, tako da je polmer na vhodnem 
zrcalu enak polmeru lastnega nihanja, krivinski radij vpadne valovne fronte 
pa enak krivinskemu radiju zrcala. 

Resonator pa ne deluje le kot frekvenčni filter, ampak tudi
kot prostorski. Naj ima vpadno valovanje isto frekvenco kot
eno od nihanj resonatorja. Prepuščeno valovanje ima tedaj obliko
Gaussovega snopa, kot jo določa resonator, ne glede na obliko vpadnega
snopa.

Gornji način obravnave resonatorjev in sklopitve z vpadnim valovanjem
je posebej prikladen za račun nestacionarnega obnašanja in za primer,
ko je resonator napolnjen z nelinearnim sredstvom.

\section{*Sklopitev dveh resonatorjev}
\index{Sklopitev resonatorja!z resonatorjem}
Podobno kot sklopitev z zunanjim valovanjem lahko obravnavamo tudi
sklopitev med dvema resonatorjema. Naj bosta dva resonatorja brez izgub
sklopljena z delno prepustnim zrcalom. Sklopitev naj bo šibka, tako da lahko zapišemo 
\begin{eqnarray}
\dot{a}_{1} & = & -i\omega_{1}a_{1}+\kappa_{12}a_{2} \\
\dot{a}_{2} & = & -i\omega_{2}a_{2}+\kappa_{21}a_{1}.
\end{eqnarray}

Zaradi ohranitve energije sklopitvena koeficienta $\kappa_{12}$ in
$\kappa_{21}$ nista neodvisna. Vsota energij obeh resonatorjev mora
biti konstantna, zato 
\begin{eqnarray}
\frac{d}{dt}(|a_{1}|^{2}+|a_{2}|^{2}) & = & a_{1}\dot{a}_{1}^{*}+a_{1}^{*}\dot{a}_{1}+
a_{2}\dot{a}_{2}^{*}+a_{2}^{*}\dot{a}_{2}\nonumber \\
 & = & a_{1}^{*}\kappa_{12}a_{2}+a_{1}\kappa_{12}^{*}a_{2}^{*}+a_{2}^{*}\kappa_{21}a_{1}+
 a_{2}\kappa_{21}^{*}a_{1}^{*}\nonumber \\
 & = & 0.
\end{eqnarray}
Veljati mora torej
\begin{equation}
\kappa_{12}+\kappa_{21}^{*}=0.
\label{3.56}
\end{equation}

Poglejmo primer dveh sklopljenih resonatorjev, pri čemer je v drugem resonatorju vzbujeno 
stoječe valovanje. Moč tistega dela, ki potuje proti prvemu resonatorju,
je polovica energije, deljena s časom preleta od enega zrcala do drugega
\begin{equation}
|s_{+}|^{2}=\frac{1}{2}|a_{2}|^{2}\frac{c}{L}.
\label{3.57}
\end{equation}
 Z upoštevanjem enačbe (\ref{3.44}) je 
\begin{equation}
\kappa_{12}a_{2}=\sqrt{\frac{2}{\tau}}\, s_{+}=\sqrt{\frac{c}{\tau L}}\, a_{2},
\label{3.58}
\end{equation}
 tako da je 
\begin{equation}
\kappa_{12}=\frac{c}{2L}\sqrt{1-{\cal R}},\mbox{\hskip1cm}\kappa_{21}=-\kappa_{12}.
\label{3.59}
\end{equation}

Zaradi sklopitve se spremenijo lastne frekvence resonatorjev. 
Poglejmo dva enaka sklopljena resonatorja
\begin{eqnarray}
\dot{a}_{1} & = & -i\omega_{0}a_{1}+\kappa_{12}a_{2}\\
\dot{a}_{2} & = & -i\omega_{0}a_{2}-\kappa_{12}a_{1}.
\end{eqnarray}
 Iščemo rešitve oblike $A_{i}e^{-i\omega t}$. Če uporabimo ta nastavek v gornjih
diferencialnih enačbah, dobimo homogen linearen sistem za $A_{1}$ in
$A_{2}$, ki je netrivialno rešljiv, če je determinanta enaka nič.
Ta pogoj da enačbo za frekvenco 
\begin{equation}
(\omega-\omega_{0})^{2}=\kappa_{12}^{2}\label{3.61}
\end{equation}
 in 
\begin{equation}
\omega_{1,2}=\omega_{0}\pm\kappa_{12}.\label{3.62}
\end{equation}
 Zaradi sklopitve sta se prej enaki frekvenci resonatorjev razcepili v dve, kot
smo lahko pričakovali.
